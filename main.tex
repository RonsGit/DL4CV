%%%%%%%%%%%%%%%%%%%%%%%%%%%%%%%%%%%%%%%%%
%
% Important note:
% Chapter heading images should have a 2:1 width:height ratio,
% e.g. 920px width and 460px height.
%
%%%%%%%%%%%%%%%%%%%%%%%%%%%%%%%%%%%%%%%%%


%----------------------------------------------------------------------------------------
%	PACKAGES AND OTHER DOCUMENT CONFIGURATIONS
%----------------------------------------------------------------------------------------
\DocumentMetadata{}
\documentclass[11pt,fleqn,openany]{book} % Default font size and left-justified equations. One can remove the 'openany' part to remove unwanted pages between sections. 
\raggedbottom

\usepackage[top=3cm,bottom=3cm,left=3.2cm,right=3.2cm,headsep=10pt,letterpaper]{geometry} % Page margins

\usepackage{xcolor} % Required for specifying colors by name
\definecolor{ocre}{RGB}{52,177,201} % Define the orange color used for highlighting throughout the book

% Font Settings
\usepackage{avant} % Use the Avantgarde font for headings
%\usepackage{times} % Use the Times font for headings
\usepackage{mathptmx} % Use Adobe Times Roman as the default text font together with math symbols from the Sym­bol, Chancery, and Com­puter Modern fonts
\usepackage{microtype} % Slightly tweak font spacing for aesthetics
\usepackage[utf8]{inputenc} % Required for including letters with accents
\usepackage[T1]{fontenc} % Use 8-bit encoding that has 256 glyphs
\usepackage{amsthm}

\usepackage{enumitem}
\setlist[enumerate]{topsep=0.5cm}

% Bibliography
\usepackage[backend=biber,style=numeric,sortcites,sorting=nty,backref,natbib,hyperref]{biblatex}
\usepackage{csquotes}
\addbibresource{bibliography.bib} % BibTeX bibliography file
\defbibheading{bibempty}{}

\input{structure} % Insert the commands.tex file which contains the majority of the structure behind the template

%----------------------------------------------------------------------------------------
%	Definitions of new commands
%----------------------------------------------------------------------------------------

\def\R{\mathbb{R}}
\newcommand{\cvx}{convex}

\begin{document}
%----------------------------------------------------------------------------------------
%	TITLE PAGE
%----------------------------------------------------------------------------------------

\begingroup
\thispagestyle{empty}
\AddToShipoutPicture*{\put(0,0){\includegraphics[scale=1.25]{esahubble}}} % Image background
\centering
\vspace*{5cm}
\par\normalfont\fontsize{35}{35}\sffamily\selectfont
\textbf{EECS498 - Deep Learning for Computer Vision}\\
{\LARGE An essential guide to modern computer vision based on UMich's EECS498 (Instructor: Justin Johnson)}\par % Book title
\vspace*{1cm}
{\LARGE Ron Korine, 2025}\par % Author name
\endgroup

%----------------------------------------------------------------------------------------
%	COPYRIGHT PAGE
%----------------------------------------------------------------------------------------

\newpage
~\vfill
\thispagestyle{empty}

%\noindent Copyright \copyright\ 2014 Andrea Hidalgo\\ % Copyright notice

\noindent \textsc{github.com/RonsGit/}\\ % URL
This document is based on content from various sources—including books and lecture notes. Each source used is duly cited to give proper credit to the original authors. If you find any content in this document that you would prefer to be removed, please feel free to contact me by clicking \textbf{\href{mailto:eecs498summary@gmail.com}{here}}, and I will address your request promptly.\\ % License information

\noindent \textit{First release, March 2025} % Printing/edition date

%----------------------------------------------------------------------------------------
%	TABLE OF CONTENTS
%----------------------------------------------------------------------------------------

\chapterimage{head1.png} % Table of contents heading image

\pagestyle{empty} % No headers

\tableofcontents % Print the table of contents itself

%\cleardoublepage % Forces the first chapter to start on an odd page so it's on the right

\pagestyle{fancy} % Print headers again

\chapter*{Preface}
\addcontentsline{toc}{chapter}{Preface}

\section{Getting Started: About the Project and How to Navigate It}

\subsection{Why This Document?}
I am a data scientist passionate about making deep learning and computer vision more accessible. In December 2023, I observed that many newcomers struggled to find a clear starting point, often getting lost in selecting topics and resources. To address this, I created this document, which consolidates key introductory concepts, curated resources, and practical insights. It is inspired by the University of Michigan’s EECS498 course, taught by \href{https://web.eecs.umich.edu/~justincj/}{Justin Johnson}. By revisiting each lecture, I aimed to provide readers with the foundational knowledge and guidance necessary to dive into deep learning and computer vision.

This summary is tailored for undergraduate and early graduate students in science and engineering disciplines, particularly those with foundational knowledge of Python, calculus, and linear algebra. These skills are essential for completing the course exercises and understanding the material effectively. If you’re unsure about your readiness, I recommend brushing up on Python programming (e.g., using \href{https://www.learnpython.org/}{Learn Python}), calculus (e.g., \href{https://www.khanacademy.org/math/calculus-1}{Khan Academy Calculus}), and linear algebra (e.g., \href{https://www.3blue1brown.com/topics/linear-algebra}{3Blue1Brown's Linear Algebra Series}).

While not strictly required, taking an introductory course in machine learning can significantly enhance your understanding. For those seeking a beginner-friendly approach, I recommend \href{https://www.coursera.org/learn/machine-learning}{Andrew Ng's Machine Learning course on Coursera}, which provides an intuitive introduction to the field. If you're looking for a more mathematical perspective, consider Stanford's \href{https://cs229.stanford.edu/}{CS229: Machine Learning}, which delves deeper into the theoretical foundations.

\subsection{Acknowledgments and Contributions}
This document stands on the shoulders of giants, extensively referencing Justin Johnson’s lectures, publicly available resources, and insightful diagrams. Sources are cited at the beginning of each section or alongside relevant content. If you find missing or incorrect credits or would like content removed, please \href{mailto:eecs498summary@gmail.com}{email me}, and I will address your concerns promptly.

\subsection{Your Feedback Matters}
This is a personal project created in my free time and reviewed by only a few individuals. Your feedback is invaluable in improving this document. If you have suggestions, corrections, or feedback, please \href{mailto:eecs498summary@gmail.com}{email me}.

\subsection{How to Navigate This Document}
\begin{enumerate}
	\item \textbf{New to the Field:} Start by watching the \href{https://www.youtube.com/watch?v=dJYGatp4SvA&list=PL5-TkQAfAZFbzxjBHtzdVCWE0Zbhomg7r&index=1&ab_channel=MichiganOnline}{EECS498 lectures on YouTube}. To reinforce your understanding, follow along with this document in parallel—referencing the corresponding sections while watching the videos. If a concept is unclear or requires deeper exploration, consult the document for a more detailed breakdown. Additionally, revisiting the chapters after completing the lecture video will help consolidate your knowledge before tackling the course assignments.
	
	\item \textbf{Practitioner or Advanced Learner:} Use the table of contents or the search function to quickly navigate to topics of interest. This document is structured for both quick reference and in-depth study, making it valuable for revisiting fundamental principles or delving into advanced discussions. If a topic in the lectures requires more comprehensive understanding, the relevant sections in this document provide further insights and clarity.
\end{enumerate}


\subsection{Staying Updated in the Field}
Deep learning evolves rapidly. To stay current, explore new papers using tools like \href{https://www.connectedpapers.com/}{Connected Papers}. Resources like \href{https://www.paperswithcode.com/}{Papers with Code} provide trending research and implementations. For high-level insights, follow YouTube channels like \href{https://www.youtube.com/@yannickilcher?si=xybWvQBUwVkFLJK3}{Yannic Kilcher} and \href{https://www.youtube.com/@andrejkarpathy?si=7DpJQABbAVeANwUc}{Andrej Karpathy}. Major conferences such as CVPR, ICCV, and ECCV are excellent for discovering state-of-the-art developments. Curated repositories, like the \href{https://www.github.com/chicleee/Image-Matching-Paper-List}{Image Matching Papers List}, are also valuable resources.

\subsection{The Importance of Practice}
Practice is crucial to mastering computer vision. As Richard Feynman famously said, “What I cannot build, I do not understand.” Actively engaging with exercises and implementing concepts is one of the most effective ways to solidify your understanding.

To maximize learning, I highly recommend completing the course assignments provided for the EECS498 course, available \href{https://web.eecs.umich.edu/~justincj/teaching/eecs498/WI2022/}{here}. These assignments are designed to complement the lectures and provide hands-on experience with key concepts in deep learning and computer vision.

Platforms like \href{https://www.colab.research.google.com/}{Google Colab} are excellent for running experiments, while \href{https://www.kaggle.com/}{Kaggle} offers competitive challenges to apply your skills. Pairing your efforts with tools like \href{https://wandb.ai/}{WandB}, \href{https://clear.ml/}{ClearML}, or \href{https://www.tensorflow.org/tensorboard}{TensorBoard} can streamline your workflow and enhance your learning journey.

By tackling exercises alongside the lectures, you deepen your theoretical understanding and gain practical insights that are invaluable for real-world applications.

% Include chapters
\chapterimage{head2.png} % Chapter heading image

% Chapter-specific content starts here
\chapter{Lecture 1: Course Introduction}

%----------------------------------------------------------------------------------------
%\tCHAPTER 1 - Lecture 1: Course Introduction
%----------------------------------------------------------------------------------------
\section{Getting Started: About the Project and How to Navigate It}

\subsection{Why This Document?}
I am a data scientist passionate about making deep learning and computer vision more accessible. In December 2023, I observed that many newcomers struggled to find a clear starting point, often getting lost in selecting topics and resources. To address this, I created this document, which consolidates key introductory concepts, curated resources, and practical insights. It is inspired by the University of Michigan’s EECS498 course, taught by \href{https://web.eecs.umich.edu/~justincj/}{Justin Johnson}. By revisiting each lecture, I aimed to provide readers with the foundational knowledge and guidance necessary to dive into deep learning and computer vision.

This summary is tailored for undergraduate and early graduate students in science and engineering disciplines, particularly those with foundational knowledge of Python, calculus, and linear algebra. These skills are essential for completing the course exercises and understanding the material effectively. If you’re unsure about your readiness, I recommend brushing up on Python programming (e.g., using \href{https://www.learnpython.org/}{Learn Python}), calculus (e.g., \href{https://www.khanacademy.org/math/calculus-1}{Khan Academy Calculus}), and linear algebra (e.g., \href{https://www.3blue1brown.com/topics/linear-algebra}{3Blue1Brown's Linear Algebra Series}).

While not strictly required, taking an introductory course in machine learning can significantly enhance your understanding. For those seeking a beginner-friendly approach, I recommend \href{https://www.coursera.org/learn/machine-learning}{Andrew Ng's Machine Learning course on Coursera}, which provides an intuitive introduction to the field. If you're looking for a more mathematical perspective, consider Stanford's \href{https://cs229.stanford.edu/}{CS229: Machine Learning}, which delves deeper into the theoretical foundations.

\subsection{Acknowledgments and Contributions}
This document stands on the shoulders of giants, extensively referencing Justin Johnson’s lectures, publicly available resources, and insightful diagrams. Sources are cited at the beginning of each section or alongside relevant content. If you find missing or incorrect credits or would like content removed, please \href{mailto:eecs498summary@gmail.com}{email me}, and I will address your concerns promptly.

\subsection{Your Feedback Matters}
This is a personal project created in my free time and reviewed by only a few individuals. Your feedback is invaluable in improving this document. If you have suggestions, corrections, or feedback, please \href{mailto:eecs498summary@gmail.com}{email me}.

\subsection{How to Navigate This Document}
\begin{enumerate}
	\item \textbf{New to the Field:} Start by watching the \href{https://www.youtube.com/watch?v=dJYGatp4SvA&list=PL5-TkQAfAZFbzxjBHtzdVCWE0Zbhomg7r&index=1&ab_channel=MichiganOnline}{EECS498 lectures on YouTube}. After each lecture, refer to the corresponding sections in this document for clarification and deeper insights. Ensure you have access to the course lecture slides, available \href{https://web.eecs.umich.edu/~justincj/slides/eecs498/WI2022/598_WI2022_lecture01.pdf}{here}, for a comprehensive learning experience.
	\item \textbf{Practitioner or Advanced Learner:} Use the table of contents or the text search feature to navigate directly to topics of interest. This document is structured to facilitate targeted learning and enrichment, whether you’re revisiting fundamental concepts or exploring advanced ideas.
\end{enumerate}

\subsection{Staying Updated in the Field}
Deep learning evolves rapidly. To stay current, explore new papers using tools like \href{https://www.connectedpapers.com/}{Connected Papers}. Resources like \href{https://www.paperswithcode.com/}{Papers with Code} provide trending research and implementations. For high-level insights, follow YouTube channels like \href{https://www.youtube.com/@yannickilcher?si=xybWvQBUwVkFLJK3}{Yannic Kilcher} and \href{https://www.youtube.com/@andrejkarpathy?si=7DpJQABbAVeANwUc}{Andrej Karpathy}. Major conferences such as CVPR, ICCV, and ECCV are excellent for discovering state-of-the-art developments. Curated repositories, like the \href{https://www.github.com/chicleee/Image-Matching-Paper-List}{Image Matching Papers List}, are also valuable resources.

\subsection{The Importance of Practice}
Practice is crucial to mastering computer vision. As Richard Feynman famously said, “What I cannot build, I do not understand.” Actively engaging with exercises and implementing concepts is one of the most effective ways to solidify your understanding.

To maximize learning, I highly recommend completing the course assignments provided for the EECS498 course, available \href{https://web.eecs.umich.edu/~justincj/teaching/eecs498/WI2022/}{here}. These assignments are designed to complement the lectures and provide hands-on experience with key concepts in deep learning and computer vision.

Platforms like \href{https://www.colab.research.google.com/}{Google Colab} are excellent for running experiments, while \href{https://www.kaggle.com/}{Kaggle} offers competitive challenges to apply your skills. Pairing your efforts with tools like \href{https://wandb.ai/}{WandB}, \href{https://clear.ml/}{ClearML}, or \href{https://www.tensorflow.org/tensorboard}{TensorBoard} can streamline your workflow and enhance your learning journey.

By tackling exercises alongside the lectures, you deepen your theoretical understanding and gain practical insights that are invaluable for real-world applications.

\newpage

\section{Lecture Notes}

This section presents my detailed lecture notes, designed to complement the material from the course. These notes build upon the lecture content, incorporating figures, examples, and concepts introduced by Justin Johnson in his lecture slides. 

You can follow along with the lecture slides available \href{https://web.eecs.umich.edu/~justincj/slides/eecs498/WI2022/598_WI2022_lecture01.pdf}{here} or watch the corresponding lecture video on \href{https://www.youtube.com/watch?v=dJYGatp4SvA&list=PL5-TkQAfAZFbzxjBHtzdVCWE0Zbhomg7r}{YouTube}. Together, these resources provide a comprehensive understanding of the topics covered.

\subsection{Core Terms in the Field}

A solid understanding of the fundamental terminology in artificial intelligence (AI) and its subfields is crucial for following this course, engaging with the lecture materials, and navigating the broader field of deep learning and computer vision. Defining these terms provides a shared foundation for deeper exploration and application, ensuring clarity as we delve into more advanced topics.

\subsubsection{Artificial Intelligence (AI)}
\begin{definition}[Artificial Intelligence (AI)]
	The overarching field focused on creating systems capable of performing tasks that typically require human intelligence. These tasks include reasoning, decision-making, language understanding, and visual perception. AI encompasses a wide range of approaches, including symbolic logic, rule-based systems, and learning-based techniques, to address complex problems across diverse domains.
\end{definition}

\subsubsection{Machine Learning (ML)}
\begin{definition}[Machine Learning (ML)]
	A subset of AI that enables systems to learn from data and improve their performance on tasks without explicit programming. Machine learning relies on algorithms and statistical models to analyze data, identify patterns, and make predictions or decisions. 
\end{definition}
Popular techniques include:
\begin{itemize}
	\item \textbf{Supervised Learning:} Models learn from labeled data, mapping inputs to desired outputs (e.g., classifying images into categories like cats and dogs).
	\item \textbf{Unsupervised Learning:} Models identify patterns and structures in unlabeled data (e.g., clustering similar images).
	\item \textbf{Reinforcement Learning:} Models learn to make sequential decisions by interacting with an environment to maximize rewards.
\end{itemize}
This course primarily focuses on supervised and unsupervised learning, which are widely applied in deep learning for computer vision.

\subsubsection{Deep Learning (DL)}
\begin{definition}[Deep Learning (DL)]
	A specialized subset of machine learning characterized by hierarchical algorithms that process data through multiple layers. Each layer extracts increasingly abstract features, enabling systems to learn complex representations. For instance, in image analysis, early layers often identify edges and textures, while deeper layers detect objects and scenes. Deep learning has driven major advancements in fields like natural language processing, speech recognition, and computer vision.
\end{definition}

\newpage

\subsubsection{Computer Vision (CV)}
\begin{definition}[Computer Vision (CV)]
	A domain within AI that focuses on enabling artificial systems to analyze, interpret, and process visual data, such as images and videos. CV intersects with, but is not a subset of, machine learning or deep learning. Instead, learning-based approaches like convolutional neural networks (CNNs) have become indispensable tools within CV, solving tasks such as image classification, object detection, and semantic segmentation. Applications are widespread, powering smartphone cameras, surveillance systems, autonomous vehicles, and robotics.
\end{definition}

\subsubsection{Connecting the Dots}
The relationships between these terms highlight their interdependence:
\begin{itemize}
	\item AI is the parent discipline encompassing all methods of creating intelligent systems.
	\item ML is a subset of AI, focusing on learning from data to improve performance.
	\item DL is a subset of ML, leveraging layered neural networks to solve complex problems.
	\item CV is a domain within AI that intersects with ML and DL, applying their techniques to visual data.
\end{itemize}

\begin{figure}[H]
	\centering
	\includegraphics[width=0.8\textwidth]{Figures/Chapter_1/Slide_13.jpg}
	\caption{In this course, we study 'Deep Learning' for Computer Vision.}
	\label{fig:chapter1_slide13}
\end{figure}

\subsubsection{Why Study Deep Learning for Computer Vision?}
Learning-based approaches have transformed computer vision by outperforming traditional algorithms in handling complex, real-world data. Unlike manual feature engineering, deep learning allows systems to automatically extract representations directly from data, making them more adaptable and effective. This shift has made \textbf{Deep Learning for Computer Vision} the dominant paradigm, enabling breakthroughs in areas like healthcare (e.g., disease detection through medical imaging), transportation (e.g., autonomous vehicles), and security (e.g., facial recognition). By leveraging the synergy of AI, ML, and DL, deep learning continues to drive innovation and solve increasingly sophisticated challenges across industries.

\subsection{Motivation for Deep Learning in Computer Vision}
Computer Vision (CV) is a transformative force in modern technology, enabling machines to perceive and interpret the world as humans do—or better. By leveraging deep learning, CV has revolutionized industries and unlocked groundbreaking capabilities, from the smartphone in your hand to the autonomous vehicles navigating our streets.

In healthcare, CV drives advancements in medical imaging, facilitating early disease detection and life-saving diagnostics. It powers safer, more efficient transportation through autonomous systems on the road. In agriculture, CV optimizes crop monitoring and pest detection, while in astronomy, it deciphers galaxy formations, expanding our understanding of the cosmos.

\begin{figure}[H]
	\centering
	\includegraphics[width=0.8\textwidth]{Figures/Chapter_1/Appen_road_image_annotation.jpeg}
	\caption{Road annotation for autonomous vehicles. Image credit: Appen \cite{appen_road_annotation}.}
	\label{fig:road_annotation}
\end{figure}

Beyond industry, CV impacts daily life—enhancing security with facial
 recognition, enriching entertainment with augmented reality, and revolutionizing commerce with smart retail solutions. Its potential to create a safer, healthier, and more connected world makes CV a compelling and impactful field, offering countless opportunities to shape the future.

\subsection{Historical Milestones}

This lecture offers a comprehensive journey through the evolution of computer vision, starting with its roots in neuroscience and progressing to its modern-day applications in artificial intelligence. The milestones covered are foundational to understanding the field, providing a historical perspective on the advancements that have shaped computer vision as we know it today. While many technical terms and concepts, such as convolutional neural networks (CNNs), vanishing gradients, recurrent neural networks (RNNs), long short-term memory (LSTM) networks, Transformers, and others, are briefly introduced, readers are encouraged not to feel deterred. Each of these topics will be explored in greater depth throughout the course and this summary, ensuring a thorough and accessible understanding of these pivotal ideas.


\subsubsection{Hubel and Wiesel (1959)}
Hubel and Wiesel’s pioneering work in the late 1950s explored the visual cortex of cats using electrodes, uncovering two critical insights. First, they identified specialized neurons that respond to specific visual stimuli, such as edges with particular orientations. Second, they revealed a hierarchical structure in visual processing, where simple features combine to form complex patterns. These discoveries laid the foundation for artificial neural networks and convolutional architectures, which are integral to modern computer vision \cite{hubel1959_receptivefields}.

\begin{figure}[H]
    \centering
    \includegraphics[width=0.8\textwidth]{Figures/Chapter_1/Slide_16.jpg}
    \caption{Hubel \& Wiesel's study, revolutionizing our understanding of visual processing \cite{hubel1959_receptivefields}}
    \label{fig:chapter1_slide16}
\end{figure}

\subsubsection{Larry Roberts (1963)}
Larry Roberts’ groundbreaking PhD thesis in 1963 is often regarded as one of the earliest foundational works in computer vision. Inspired by the findings of Hubel and Wiesel on visual processing, Roberts focused on extracting edges from images, proposing methods to detect keypoints like corners. His work went beyond edge detection, leveraging these features to analyze the 3D geometry of objects in images, thus laying the groundwork for object recognition and scene understanding \cite{roberts1963_3dsolids}.

\begin{figure}[H]
    \centering
    \includegraphics[width=0.8\textwidth]{Figures/Chapter_1/Slide_17.jpg}
    \caption{Larry Roberts' 1963 thesis introduced edge detection as a critical component of early computer vision systems \cite{roberts1963_3dsolids}}
    \label{fig:chapter1_roberts}
\end{figure}

\subsubsection{David Marr (1970s)}
David Marr revolutionized computer vision in the 1970s by introducing a theoretical framework for understanding visual processing, which remains influential to this day. His theory, detailed in his book 'VISION', proposed that human and artificial vision involve hierarchical, multi-stage processing to extract meaningful information from visual data. Marr's framework consists of three key stages:

\begin{itemize}
    \item \textbf{Primal Sketch}: Captures basic image features such as edges, textures, and regions of high contrast, forming a simplified representation of the scene.
    \item \textbf{2.5D Sketch}: Incorporates depth and surface orientation, providing a viewer-centric representation that bridges raw image data and object geometry.
    \item \textbf{3D Model}: Creates a complete, three-dimensional understanding of the scene, enabling recognition and interaction with objects.
\end{itemize}

These concepts profoundly influenced computer vision by emphasizing structured, incremental processing and inspired algorithms for edge detection, depth estimation, and object modeling. Marr’s work continues to shape the field, bridging biological vision studies and artificial intelligence \cite{marr1982_vision}.

\begin{figure}[H]
    \centering
    \includegraphics[width=0.8\textwidth]{Figures/Chapter_1/Slide_19.jpg}
    \caption{David Marr's theory of multi-stage visual processing \cite{marr1982_vision}}
    \label{fig:chapter1_marr}
\end{figure}

\subsubsection{Recognition via Parts (1970s)}

In the 1970s, researchers shifted their focus to recognizing complex objects by building upon earlier advancements in feature extraction. A key breakthrough was the introduction of \textbf{Generalized Cylinders} by Brooks and Binford in 1979 \cite{brooks1979_modelbased}, which proposed a model for representing intricate objects using simple geometric shapes. This approach enabled the decomposition of complex structures into manageable components, facilitating object recognition.

Earlier in the decade, Fischler and Elschlager’s \textbf{Pictorial Structures} (1973) introduced a complementary method for object representation. Their approach modeled objects as a collection of interconnected parts with defined spatial relationships, emphasizing the importance of how parts relate to each other in forming a complete object \cite{fischler1973_pictorialstructures}. This method refined the concept of part-based recognition by incorporating spatial constraints, making object recognition systems more robust to variations in appearance.

Both Generalized Cylinders and Pictorial Structures laid the groundwork for part-based models in computer vision, influencing modern techniques such as deformable part models and pose estimation. These foundational ideas continue to impact research in object recognition and scene understanding.

\begin{figure}[H]
    \centering
    \includegraphics[width=0.8\textwidth]{Figures/Chapter_1/Slide_20.jpg}
    \caption{Recognition via parts: Generalized Cylinders and Pictorial Structures, foundational to modern object recognition \cite{brooks1979_modelbased, fischler1973_pictorialstructures}}
    \label{fig:chapter1_parts_recognition}
\end{figure}

\subsubsection{Recognition via Edge Detection (1980s)}

The 1980s marked a pivotal era in computer vision, as advancements in digital cameras and processing hardware enabled researchers to work with more realistic and complex images. A significant focus of this decade was object detection, with edge detection techniques taking center stage.

A landmark contribution was the introduction of the \textbf{Canny Edge Detector} by John Canny in 1986 \cite{canny1986_edgedetection}. This algorithm provided a systematic and efficient method for detecting edges, employing a multi-stage process: noise reduction to enhance clarity, gradient calculation to identify regions of rapid intensity change, non-maximum suppression to thin edges, and edge tracking by hysteresis to ensure continuity. Due to its robustness and accuracy, the Canny edge detector remains a cornerstone in computer vision, widely used in both academic research and industrial applications.

Building upon edge detection, David Lowe’s work in 1987 explored \textbf{template matching}, using edge-based features. Lowe introduced the concept of "razor templates," which were derived from reference images to identify similar objects in new images \cite{lowe1987_objectrecognition}. This approach demonstrated the potential of leveraging edges for object recognition, setting the stage for more sophisticated methods.

However, despite their groundbreaking nature, edge detection and template matching faced limitations. These techniques often struggled with complex, cluttered, or occluded scenes, where edges alone provided insufficient context for robust object detection. For instance, variations in lighting, scale, and viewpoint could significantly degrade the performance of edge-based methods. These challenges highlighted the need for more advanced approaches that could group edges into meaningful structures and match objects more effectively—advancements that would emerge in subsequent decades.

The innovations of the 1980s laid the groundwork for modern object detection, influencing the development of algorithms that continue to shape computer vision systems today.

\begin{figure}[H]
    \centering
    \includegraphics[width=0.8\textwidth]{Figures/Chapter_1/Slide_21.jpg}
    \caption{Recognition via edge detection: Canny Edge Detector and template matching by Lowe, foundational to object detection \cite{canny1986_edgedetection, lowe1987_objectrecognition}}
    \label{fig:chapter1_edge_detection}
\end{figure}

\subsubsection{Recognition via Grouping (1990s)}

The 1990s saw significant progress in addressing the challenges of recognizing objects in increasingly complex images and scenes. Researchers shifted their focus towards grouping techniques to partition images into meaningful regions, enabling more effective object recognition and scene understanding.

A landmark contribution from this period was the introduction of \textbf{Normalized Cuts and Image Segmentation} by Shi and Malik in 1997 \cite{shi1997_normalizedcuts}. This method formulated image segmentation as a graph partitioning problem. In their approach, an image is represented as a graph, where pixels or groups of pixels form the nodes, and the edges represent the similarity between these nodes based on features such as color, texture, and spatial proximity.

The primary objective of normalized cuts was to partition the graph into disjoint regions such that:
- The similarity within each region (intra-region similarity) is maximized.
- The dissimilarity between different regions (inter-region dissimilarity) is minimized.

This framework provided a mathematically rigorous approach to image segmentation, allowing for the grouping of image regions that are internally cohesive while being distinct from other regions. Compared to earlier heuristic-based methods, normalized cuts offered a more unified and generalizable solution, capable of handling a wide range of segmentation tasks.

Shi and Malik’s method was particularly impactful as it addressed the need for global optimization in segmentation, rather than relying solely on local features. It paved the way for further advancements in scene analysis, object recognition, and video segmentation. The ability to group and label regions effectively has since become a foundational concept in computer vision, influencing modern techniques such as region proposal networks used in deep learning-based object detection.

\begin{figure}[H]
    \centering
    \includegraphics[width=0.8\textwidth]{Figures/Chapter_1/Slide_22.jpg}
    \caption{Recognition via grouping: Normalized Cuts by Shi and Malik, a groundbreaking approach to image segmentation \cite{shi1997_normalizedcuts}}
    \label{fig:chapter1_grouping}
\end{figure}

\subsubsection{Recognition via Matching and Benchmarking (2000s)}

The 2000s marked a transformative period in computer vision, characterized by advancements in feature matching and the establishment of benchmarks that fueled innovation.

One of the era's most influential algorithms was the \textbf{Scale-Invariant Feature Transform (SIFT)}, introduced by David Lowe in 1999 \cite{lowe1999_sift}. SIFT provided a robust framework for detecting and describing keypoints in images, enabling reliable matching across variations in scale, rotation, and lighting. The algorithm comprises three key steps:
\begin{itemize}
	\item \textbf{Keypoint Detection:} Identifies potential keypoints by detecting extrema in a Difference of Gaussian (DoG) function applied across multiple scales.
	\item \textbf{Keypoint Description:} Creates a feature vector based on the local gradient orientations around each keypoint.
	\item \textbf{Keypoint Matching:} Compares descriptors between images to establish correspondences, facilitating tasks like object recognition and image stitching.
\end{itemize}

\begin{figure}[H]
	\centering
	\includegraphics[width=0.8\textwidth]{Figures/Chapter_1/Slide_23.jpg}
	\caption{SIFT: A groundbreaking feature matching algorithm introduced by Lowe in 1999 \cite{lowe1999_sift}}
	\label{fig:chapter1_sift}
\end{figure}

Another groundbreaking development from this period was the \textbf{Viola-Jones Face Detection Algorithm}, introduced in 2001 \cite{viola2001_boosteddetection}. This method employed boosted decision trees for real-time face detection, laying the groundwork for machine learning applications in computer vision. The algorithm's efficiency and robustness made facial recognition a ubiquitous feature in consumer electronics, such as digital cameras and smartphones.

\begin{figure}[H]
	\centering
	\includegraphics[width=0.8\textwidth]{Figures/Chapter_1/Slide_24.jpg}
	\caption{Viola-Jones face detection algorithm, a milestone in real-time object detection \cite{viola2001_boosteddetection}}
	\label{fig:chapter1_viola_jones}
\end{figure}

The establishment of benchmarks during this period significantly advanced computer vision research. The \textbf{PASCAL Visual Object Challenge}, introduced in 2005, provided a competitive platform to evaluate object detection and recognition algorithms across various categories \cite{pascal2010_visualchallenge}. It encouraged collaboration and set a new standard for algorithmic performance, inspiring innovations that continue to shape the field today.

\begin{figure}[H]
	\centering
	\includegraphics[width=0.8\textwidth]{Figures/Chapter_1/Slide_25.jpg}
	\caption{PASCAL Visual Object Challenge: A benchmark for object detection \& recognition \cite{pascal2010_visualchallenge}}
	\label{fig:chapter1_pascal}
\end{figure}

\subsubsection{The ImageNet Dataset and Classification Challenge}

The introduction of the \textbf{ImageNet} dataset in 2009 marked a new era in computer vision \cite{imagenet2009_hierarchicaldatabase}. This large-scale dataset contains over 1.4 million labeled images across 1,000 object categories, providing a rich resource for training and evaluating visual recognition systems. The annual \textbf{ImageNet Large Scale Visual Recognition Challenge (ILSVRC)} became a benchmark competition, driving significant advances in image classification and object detection. Key milestones include:
\begin{itemize}
	\item \textbf{2010-2011:} Traditional feature-based methods achieved error rates of around 28-25\%.
	\item \textbf{2012:} The introduction of \textbf{AlexNet}, a deep convolutional neural network, reduced the error rate to 16\%, initiating the deep learning revolution \cite{krizhevsky2012_alexnet}.
	\item \textbf{2015:} The advent of deeper architectures, such as \textbf{ResNet}, achieved near-human performance with error rates below 5\%.
\end{itemize}

\begin{figure}[H]
	\centering
	\includegraphics[width=0.8\textwidth]{Figures/Chapter_1/Slide_27.jpg}
	\caption{Advances in the ImageNet Classification Challenge \cite{imagenet2009_hierarchicaldatabase, krizhevsky2012_alexnet}}
	\label{fig:chapter1_imagenet_challenge}
\end{figure}


\subsubsection{AlexNet: A Revolution in Computer Vision (2012)}

The success of \textbf{AlexNet} in the 2012 ImageNet Large Scale Visual Recognition Challenge (ILSVRC) was a turning point in computer vision. Developed by Alex Krizhevsky, Ilya Sutskever, and Geoffrey Hinton, AlexNet achieved a top-5 error rate of 16\%, significantly outperforming the runner-up at 26\% \cite{krizhevsky2012_alexnet}. This achievement demonstrated the practical power of deep learning, establishing convolutional neural networks (CNNs) as the dominant paradigm for computer vision.

Key innovations in AlexNet included:
\begin{itemize}
	\item \textbf{GPU Acceleration:} AlexNet utilized NVIDIA GTX 580 GPUs for parallelized training, making large-scale deep learning computationally feasible for the first time.
	\item \textbf{Rectified Linear Units (ReLU):} By addressing the vanishing gradient problem, ReLU activation functions allowed for faster convergence and deeper architectures.
	\item \textbf{Dropout Regularization:} This technique reduced overfitting by randomly deactivating neurons during training, improving model generalization.
	\item \textbf{Data Augmentation:} Methods such as random cropping and flipping artificially expanded the training dataset, mitigating overfitting and enhancing robustness.
	\item \textbf{Deep Architecture:} AlexNet’s eight-layer design enabled hierarchical feature extraction, capturing increasingly abstract patterns in visual data.
\end{itemize}

\begin{figure}[H]
	\centering
	\includegraphics[width=0.8\textwidth]{Figures/Chapter_1/Slide_28.jpg}
	\caption{AlexNet’s performance in the 2012 ImageNet Challenge, showcasing its revolutionary impact on deep learning \cite{krizhevsky2012_alexnet}.}
	\label{fig:chapter1_alexnet}
\end{figure}

AlexNet’s success not only popularized GPUs for training but also showcased the potential of deep learning to outperform traditional methods on complex visual tasks. It set the stage for a wave of transformative innovations in the years that followed.

\textbf{Building on AlexNet: Evolution of CNNs and Beyond}

The success of AlexNet catalyzed the evolution of convolutional neural networks (CNNs) and laid the foundation for subsequent advancements in neural architectures.

\textbf{ResNets and Deeper CNN Architectures (2015)}

While AlexNet demonstrated the power of deep CNNs, increasing network depth often led to the \textbf{vanishing gradient problem}, where gradients diminish as they propagate through layers, hindering effective training. The introduction of \textbf{Residual Networks (ResNets)} by He et al. in 2015 \cite{he2016_resnet} addressed this challenge with the concept of \textbf{skip connections}. These connections allowed gradients to flow directly through layers, enabling the training of networks with hundreds or even thousands of layers.

ResNets revolutionized CNNs by demonstrating that very deep networks could achieve superior performance without overfitting, achieving state-of-the-art results on tasks like image classification and object detection. Their success made deep residual learning a foundational concept in modern deep learning.

\textbf{Recurrent Neural Networks (RNNs) and LSTMs (Used in CV since the 2010s)}

\textbf{Recurrent Neural Networks (RNNs)} were initially introduced in the 1980s \cite{rumelhart1986_backpropagation}, but their application to computer vision tasks became prominent in the 2010s, particularly for sequence-based problems like video analysis, activity recognition, and image captioning. RNNs process sequential data by maintaining a hidden state that evolves over time, making them suitable for tasks requiring temporal dependencies.

However, standard RNNs struggled with long-term dependencies due to the vanishing gradient problem. The introduction of \textbf{Long Short-Term Memory (LSTM)} networks by Hochreiter and Schmidhuber in the 1990s \cite{hochreiter1997_lstm} resolved these limitations using gating mechanisms to selectively retain and forget information. LSTMs gained widespread use in CV tasks starting in the mid-2010s, enabling applications like video captioning and temporal activity recognition \cite{donahue2015_ltrcnn}. Despite their success, LSTMs face two critical challenges:
\begin{itemize}
	\item \textbf{Computational Expense:} LSTMs cannot be easily parallelized due to their sequential nature, making them computationally intensive for large-scale datasets or long sequences.
	\item \textbf{Limited Generalization to Long Contexts:} As the sequence length increases (e.g., videos with many frames), LSTMs struggle to capture global context effectively, often prioritizing recent frames over distant ones.
\end{itemize}

These limitations paved the way for attention-based models, which address these shortcomings by learning global dependencies more effectively.

\textbf{Vision Transformers (ViTs): Replacing Sequential Processing (2020)}

Inspired by the \textbf{Transformers} introduced in NLP by Vaswani et al. in 2017 \cite{vaswani2017_attention}, Vision Transformers (\textbf{ViTs}) adopted attention mechanisms to process visual data \cite{vit2020_transformers}. Unlike RNNs, which process sequences step-by-step, Transformers use self-attention to capture global dependencies in parallel, making them more scalable and efficient.

ViTs treat images as sequences of patches, learning global context and achieving state-of-the-art results in image classification, object detection, and segmentation. By leveraging attention mechanisms, ViTs overcame the limitations of both CNNs and RNNs, enabling large-scale vision tasks with greater efficiency.

\textbf{MAMBA: Multi-Agent Dynamics (2022)}

The \textbf{Multi-Agent and Multi-Body Analysis (MAMBA)} architecture, proposed in 2022 \cite{mamba2022_dynamic}, extends vision systems to dynamic, multi-agent scenarios. Designed to handle interactions between multiple entities, MAMBA integrates spatial and temporal reasoning with attention mechanisms. This architecture is particularly relevant for applications like autonomous driving and robotics, where understanding interactions between agents (e.g., vehicles and pedestrians) is critical. By building on ViTs and incorporating multi-agent dynamics, MAMBA represents a significant step forward in scene understanding and interaction modeling.

\textbf{Foundation Models: From Vision to Multimodal Intelligence}

The advancements following AlexNet set the stage for foundation models that integrate vision, language, and multimodal capabilities. These include:
\begin{itemize}
	\item \textbf{DINO (2021):} Demonstrating the power of self-supervised learning, DINO leverages Vision Transformers to learn robust visual representations without labels \cite{dino2021_selfsupervised}.
	\item \textbf{CLIP (2021):} By aligning vision and language embeddings, CLIP enables cross-modal understanding and zero-shot classification \cite{clip2021_multimodal}.
	\item \textbf{Segment Anything Model (SAM) (2023):} SAM generalizes segmentation across diverse datasets, setting a new standard for image segmentation tasks \cite{sam2023_segmentation}.
	\item \textbf{Visual Language Models (VLMs):} Models like Flamingo \cite{flamingo2022_fewshot} combine vision and language reasoning, enabling tasks such as visual question answering and multimodal dialogue.
\end{itemize}

These innovations demonstrate how AlexNet’s legacy has shaped the trajectory of computer vision, from hierarchical feature extraction to global context understanding, and from vision-specific tasks to integrated, multimodal intelligence.

\subsection{Milestones in the Evolution of Learning in Computer Vision}

The evolution of learning-based approaches in computer vision has been marked by pivotal milestones, each building upon its predecessors to push the boundaries of what artificial systems can achieve. From the early perceptron to the transformative AlexNet, these developments highlight the progression of ideas and innovations that laid the groundwork for modern deep learning.

\subsubsection{The Perceptron (1958)}
The \textbf{Perceptron}, introduced by Frank Rosenblatt in 1958, was the first neural network capable of learning from data. Designed as a single-layer classifier, it demonstrated that machines could adjust their weights iteratively based on error corrections, enabling them to classify data using linear boundaries. Despite its early promise, the perceptron had significant limitations, particularly its inability to solve non-linear problems such as XOR. These shortcomings were later critiqued in the influential book \textbf{"Perceptrons"} by Marvin Minsky and Seymour Papert in 1969, which highlighted the theoretical constraints of single-layer networks \cite{rosenblatt1958_perceptron, minsky1969_perceptrons}.

\begin{figure}[H]
	\centering
	\includegraphics[width=0.8\textwidth]{Figures/Chapter_1/Slide_29.jpg}
	\caption{Frank Rosenblatt’s Perceptron, foundational to neural network research \cite{rosenblatt1958_perceptron}.}
	\label{fig:chapter1_perceptron}
\end{figure}

\subsubsection{The AI Winter and Multilayer Perceptrons (1969)}
Minsky and Papert’s critique, while valid, inadvertently led to an "AI Winter," a period of reduced interest and funding in neural network research. However, their work also suggested that \textbf{multilayer perceptrons} could overcome the limitations of single-layer networks by introducing hidden layers. Unfortunately, at the time, efficient training methods for such architectures were unavailable, stalling progress \cite{minsky1969_perceptrons}.

\begin{figure}[H]
	\centering
	\includegraphics[width=0.8\textwidth]{Figures/Chapter_1/Slide_30.jpg}
	\caption{Minsky and Papert’s seminal book "Perceptrons," critiquing single-layer networks \cite{minsky1969_perceptrons}.}
	\label{fig:chapter1_perceptrons_book}
\end{figure}

\subsubsection{The Neocognitron (1980)}
In 1980, Kunihiko Fukushima introduced the \textbf{Neocognitron}, a hierarchical, multi-layered neural network inspired by the mammalian visual cortex. By combining convolution-like and pooling-like operations, the neocognitron could recognize complex patterns and invariances. While it conceptually resembled modern convolutional neural networks (CNNs), it lacked an efficient algorithm to train its multiple layers, limiting its practical utility. Nevertheless, the neocognitron laid the conceptual groundwork for future breakthroughs \cite{fukushima1980_neocognitron}.

\begin{figure}[H]
	\centering
	\includegraphics[width=0.8\textwidth]{Figures/Chapter_1/Slide_32.jpg}
	\caption{Kunihiko Fukushima’s Neocognitron: A precursor to modern CNNs \cite{fukushima1980_neocognitron}.}
	\label{fig:chapter1_neocognitron}
\end{figure}

\subsubsection{Backpropagation and the Revival of Neural Networks (1986)}
The development of \textbf{Backpropagation} in 1986 by Rumelhart, Hinton, and Williams addressed the key limitation of multilayer networks: the lack of an effective training algorithm. By using the chain rule of calculus to compute gradients of the loss function with respect to network weights, backpropagation enabled iterative weight updates via gradient descent. This innovation allowed for the training of deep networks, reigniting interest in neural networks and providing a framework for the architectures that followed \cite{rumelhart1986_backpropagation}.

\begin{figure}[H]
	\centering
	\includegraphics[width=0.8\textwidth]{Figures/Chapter_1/Slide_33.jpg}
	\caption{Backpropagation algorithm by Rumelhart et al., pivotal for training deep networks \cite{rumelhart1986_backpropagation}.}
	\label{fig:chapter1_backprop}
\end{figure}

\subsubsection{LeNet and the Emergence of Convolutional Networks (1998)}
In 1998, Yann LeCun and colleagues introduced \textbf{LeNet-5}, a convolutional neural network (CNN) designed for handwritten digit recognition. By incorporating convolutional layers for feature extraction and pooling layers for dimensionality reduction, LeNet demonstrated the power of hierarchical architectures for pattern recognition. Leveraging backpropagation, it was trained end-to-end, achieving remarkable performance on the MNIST dataset (solving hand-written digits classification) and establishing CNNs as a practical tool for real-world applications \cite{lecun1998_lenet}.

\begin{figure}[H]
	\centering
	\includegraphics[width=0.8\textwidth]{Figures/Chapter_1/Slide_34.jpg}
	\caption{Yann LeCun’s LeNet-5: The first practical convolutional network \cite{lecun1998_lenet}.}
	\label{fig:chapter1_lenet}
\end{figure}

\subsubsection{The 2000s: The Era of Deep Learning}
The 2000s marked the resurgence of neural networks as \textbf{Deep Learning} emerged as a dominant paradigm. Advances in GPU hardware, large-scale datasets, and improved algorithms made it possible to train deeper networks. Research in convolutional networks, recurrent networks, and self-supervised learning exploded, leading to breakthroughs across various domains.

\begin{figure}[H]
	\centering
	\includegraphics[width=0.8\textwidth]{Figures/Chapter_1/Slide_35.jpg}
	\caption{The 2000s: Advances in hardware and algorithms enabling deep learning.}
	\label{fig:chapter1_dl_2000s}
\end{figure}

\subsubsection{Deep Learning Explosion (2007-2020)}
Starting from 2007, the number of deep learning publications grew exponentially, driven by challenges like ImageNet and CVPR competitions. By 2020, deep learning had become ubiquitous, transforming computer vision and establishing itself as a cornerstone of modern AI \cite{imagenet2009_hierarchicaldatabase, krizhevsky2012_alexnet}.

\begin{figure}[H]
	\centering
	\includegraphics[width=0.8\textwidth]{Figures/Chapter_1/Slide_36.jpg}
	\caption{Exponential growth in deep learning research, from 2007 to 2020.}
	\label{fig:chapter1_dl_explosion}
\end{figure}


\subsection{2012 to Present: Deep Learning is Everywhere}

The transformative success of AlexNet in 2012 heralded the deep learning revolution, marking a paradigm shift across computer vision and artificial intelligence. Since then, deep learning has permeated diverse domains, solving increasingly complex tasks and enabling breakthroughs that were previously unattainable. Below are some of the key tasks and applications transformed by deep learning:

\subsubsection{Core Vision Tasks}
\begin{itemize}
	\item \textbf{Image Classification:} Deep learning models like AlexNet \cite{krizhevsky2012_alexnet} and ResNet \cite{he2016_resnet} have achieved state-of-the-art performance on benchmarks like ImageNet.
	\item \textbf{Image Retrieval:} Features extracted by CNNs are used to search for visually similar images in large datasets, revolutionizing search engines and digital asset management.
	\item \textbf{Object Detection:} Techniques like Faster R-CNN \cite{ren2015_fasterrcnn} accurately localize and classify objects in images, enabling applications such as autonomous driving and surveillance.
	\item \textbf{Image Segmentation:} Models such as DeepLab \cite{chen2017_deeplab} and Mask R-CNN \cite{he2017_maskrcnn} partition images into semantically meaningful regions, advancing medical imaging and autonomous systems.
\end{itemize}

\subsubsection{Video and Temporal Analysis}
\begin{itemize}
	\item \textbf{Video Classification:} Methods like Two-Stream Networks \cite{simonyan2014_twostream} analyze both spatial and temporal features, enabling tasks like activity recognition.
	\item \textbf{Activity Recognition:} Deep learning has enabled fine-grained understanding of human activities in videos, aiding applications in healthcare, sports analysis, and surveillance.
	\item \textbf{Pose Recognition:} Toshev and Szegedy \cite{toshev2014pose_estimation} proposed deep architectures for pose estimation, significantly advancing human-computer interaction and animation.
	\item \textbf{Reinforcement Learning:} In 2014, deep reinforcement learning demonstrated the ability to play Atari games at superhuman levels \cite{guo2014_atari}, showcasing the potential of neural networks in sequential decision-making.
\end{itemize}

\subsubsection{Generative and Multimodal Models}
\begin{itemize}
	\item \textbf{Image Captioning:} Vinyals et al. \cite{vinyals2015_captioning} and Karpathy and Fei-Fei \cite{karpathy2015_visualsemantic} introduced models that integrate vision and language, describing images with human-like captions.
	\item \textbf{DALL-E:} Recent advancements like DALL-E \cite{dalle2021_texttoimage} generate creative visual content, such as the iconic avocado-shaped armchair, pushing the boundaries of generative models.
	\item \textbf{Multimodal Models:} Foundation models like CLIP \cite{clip2021_multimodal} and Flamingo \cite{flamingo2022_fewshot} align visual and textual embeddings, enabling cross-modal reasoning and applications in content creation and retrieval.
\end{itemize}

\subsubsection{Specialized Domains}
\begin{itemize}
	\item \textbf{Medical Imaging:} Deep learning facilitates disease diagnosis and treatment planning, as seen in Levy et al.’s 2016 work \cite{levy2016_medicalimaging}.
	\item \textbf{Galaxy Classification:} Dieleman et al. (2014) \cite{dieleman2014_galaxycnn} used CNNs to classify galaxies, advancing astronomical research.
	\item \textbf{Wildlife Recognition:} Kaggle challenges like Whale Categorization Playground highlight the role of deep learning in biodiversity studies.
\end{itemize}

\subsubsection{State-of-the-Art Models}
Recent innovations like the Segment Anything Model (SAM) \cite{sam2023_segmentation}, DINO \cite{dino2021_selfsupervised}, and MAMBA \cite{mamba2022_dynamic} represent the cutting edge in computer vision. These models integrate self-supervised learning, multimodal reasoning, and dynamic scene analysis, setting new benchmarks for performance and versatility.

\begin{figure}[H]
	\centering
	\includegraphics[width=0.8\textwidth]{Figures/Chapter_1/Slide_45.jpg}
	\caption{The iconic avocado-shaped armchair, generated by DALL-E, exemplifies the creative potential of generative models \cite{dalle2021_texttoimage}.}
	\label{fig:dalle_avocado}
\end{figure}

\subsubsection{Computation is Cheaper: More GFLOPs per Dollar}

The exponential drop in computation costs has driven the deep learning revolution. Over the past decade, GPUs like NVIDIA's GTX 580 (a pioneering GPU in deep learning, used for training AlexNet) to RTX 3080 have vastly increased performance per dollar. Modern GPUs with tensor cores, optimized for deep learning, deliver unprecedented power for training and inference, enabling breakthroughs in computer vision, NLP, and reinforcement learning. As GFLOPs become increasingly affordable, AI innovation accelerates with fewer resource constraints.

\begin{figure}[H]
	\centering
	\includegraphics[width=0.8\textwidth]{Figures/Chapter_1/Slide_48.jpg}
	\caption{The dramatic drop in GFLOPs cost over time, enabling more accessible deep learning applications.}
	\label{fig:gflops_cost}
\end{figure}

\begin{figure}[H]
	\centering
	\includegraphics[width=0.8\textwidth]{Figures/Chapter_1/Slide_49.jpg}
	\caption{Advances in GPUs, including tensor cores, greatly enhancing GFLOPs per dollar.}
	\label{fig:gpu_tensor_cores}
\end{figure}


\subsection{Key Challenges in CV and Future Directions}

Despite remarkable progress, computer vision systems still face significant challenges that underscore their limitations and the need for continued innovation:

\begin{itemize}
	\item \textbf{Model Bias and Ethical Concerns:} Bias in training data has led to harmful outcomes, such as facial recognition systems misidentifying Black individuals as apes or employment screening tools unfairly discriminating against candidates. These issues highlight the importance of ethical considerations and fairness in model design and deployment \cite{buolamwini2018_gendershades}.
	\item \textbf{Misapplication Risks:} The potential misuse of CV systems poses serious concerns. For example, face-scanning applications might decide a person's job suitability without understanding context or fairness, raising questions about accountability and societal impact.
	\begin{figure}[H]
		\centering
		\includegraphics[width=0.8\textwidth]{Figures/Chapter_1/Slide_52.jpg}
		\caption{Ethical concerns: CV systems can amplify biases or cause harm, such as misidentifications \cite{buolamwini2018_gendershades}.}
		\label{fig:chapter1_ethics}
	\end{figure}
	\item \textbf{Adversarial Robustness:} Adversarial attacks, involving small imperceptible changes to input images, can lead to incorrect predictions. These vulnerabilities pose risks for applications like autonomous vehicles and security systems, where accuracy is critical \cite{goodfellow2014_adversarial}.
	
	\begin{figure}[H]
		\centering
		\includegraphics[width=0.8\textwidth]{Figures/Chapter_1/adversarial_img_1.jpg}
		\caption{Adversarial examples: Adding imperceptible noise to a panda image causes the model to misclassify it \cite{goodfellow2014_adversarial}.}
		\label{fig:chapter1_adversarial}
	\end{figure}
	
	\item \textbf{Complex Scene Understanding:} Current CV models struggle to grasp nuanced scenes that are intuitive to humans. For instance, a situation where President Obama pranks a man by tipping a scale, causing everyone in the room to laugh, is easily understood by humans but perplexes AI, which lacks contextual and social understanding.
\end{itemize}

\begin{figure}[H]
	\centering
	\includegraphics[width=0.8\textwidth]{Figures/Chapter_1/Slide_53.jpg}
	\caption{Complex scene understanding: AI struggles with nuanced contexts like social interactions.}
	\label{fig:chapter1_context}
\end{figure}

\textbf{Future Directions:}
\begin{itemize}
	\item \textbf{Enhancing Interpretability:} Developing models that can explain their predictions to users will increase trust and usability in critical domains like healthcare and criminal justice.
	\item \textbf{Mitigating Bias:} Building datasets that are diverse and inclusive can reduce biases and ensure fairer outcomes across demographics.
	\item \textbf{Improving Robustness:} Advancing defenses against adversarial attacks will make CV systems more reliable in high-stakes scenarios.
	\item \textbf{Integrating Contextual Reasoning:} Multi-modal approaches that combine visual, textual, and other data streams can help systems understand complex social and environmental contexts.
\end{itemize}

While these challenges highlight the current limitations, they also present opportunities for groundbreaking advancements, bringing computer vision closer to human-like understanding.


\chapterimage{head2.png} % Chapter heading image

\chapter{Lecture 2: Image Classification}

%----------------------------------------------------------------------------------------
%	CHAPTER 2 - Lecture 2: Image Classification
%----------------------------------------------------------------------------------------

\section{Introduction to Image Classification}

Image classification is one of the most fundamental tasks in computer vision and serves as the cornerstone for a wide range of applications in artificial intelligence. The objective of image classification is straightforward: given an input image, the algorithm must assign a category label from a predefined set of classes. For instance, an algorithm might label an image as one of several categories, such as "cat," "dog," or "car".

Despite its simplicity in concept, image classification presents a host of challenges when applied to real-world scenarios. Humans effortlessly recognize objects in images due to our ability to intuitively interpret visual information. However, computers face significant hurdles due to the \textit{semantic gap}, which refers to the difference between the raw pixel values of an image and the high-level semantic information we perceive.

When processing an image, a computer sees only a grid of numbers representing pixel intensities. Even minor changes, such as variations in viewpoint, lighting, or background, can drastically alter these pixel values, making it difficult to map them to a consistent semantic label. Moreover, intra-class variations, such as differences in appearance among individual objects within the same category, add another layer of complexity.

To address these challenges, image classification has evolved from early heuristic-based methods to modern data-driven approaches that leverage machine learning and deep learning. By analyzing large datasets of labeled images, these algorithms learn patterns and statistical dependencies that enable them to generalize across diverse examples.

This chapter begins by exploring the foundational concepts of image classification, including its historical background and early techniques. It then delves into common datasets used for classification, providing insights into their importance and structure. Building on this foundation, we introduce the \textit{nearest neighbor} algorithm as our first learning-based method, followed by a discussion on hyperparameter tuning, data hygiene, and cross-validation. Finally, the chapter highlights the pivotal role of image classification in powering more advanced computer vision tasks, such as object detection and image captioning, and examines the transition from raw pixel-based methods to feature-based approaches driven by deep learning.

By the end of this chapter, readers will gain a solid understanding of the principles and challenges of image classification and will be equipped with the knowledge to implement their first machine learning algorithm for visual recognition.

\section{Image Classification Challenges}

Image classification is a fundamental yet challenging task in computer vision. It requires algorithms to bridge the "semantic gap"—the disparity between human perception and raw pixel data processed by machines. This gap arises because machines interpret images as tensors (multidimensional arrays, or a generalization in n dimensions of matrices) of pixel values, devoid of inherent semantic meaning. This section explores the critical challenges in image classification, highlighting the complexities of achieving robust and accurate recognition.

\subsection{The Semantic Gap}
Humans perceive images intuitively, instantly recognizing objects and their context. Machines, however, see images as grids of numbers—pixel values in a tensor representation. For example, an image might be represented as a \(H \times W \times C\) tensor, where \(H\) and \(W\) denote the height and width of the image, and \(C\) represents color channels. These raw values lack semantic information, making it challenging for algorithms to deduce meaningful patterns.

\begin{figure}[H]
	\centering
	\includegraphics[width=0.8\textwidth]{Figures/Chapter_2/Slide_10.jpg}
	\caption{Images are represented as grids of pixel values, lacking inherent semantic meaning.}
	\label{fig:chapter2_semantic_gap}
\end{figure}

\subsection{Robustness to Camera Movement}
Images captured from different camera angles or positions can vary significantly in their pixel values, even when depicting the same scene. For example, photographing a cat from different angles produces vastly different pixel grids, despite representing the same object.

\begin{figure}[H]
	\centering
	\includegraphics[width=0.8\textwidth]{Figures/Chapter_2/Slide_11.jpg}
	\caption{Changes in camera position or angle result in varying pixel grids, complicating classification.}
	\label{fig:chapter2_camera_movement}
\end{figure}

\subsection{Intra-Class Variation}
Objects within the same category can exhibit substantial visual differences. For example, cats of different breeds or fur colors might look entirely distinct in terms of pixel patterns.

\begin{figure}[H]
	\centering
	\includegraphics[width=0.8\textwidth]{Figures/Chapter_2/Slide_12.jpg}
	\caption{Cats of different breeds show significant visual differences, a phenomenon known as intra-class variation.}
	\label{fig:chapter2_intra_class_variation}
\end{figure}

\subsection{Fine-Grained Classification}
Distinguishing between visually similar categories, such as specific breeds of cats, requires a more granular understanding of features. Fine-grained classification demands algorithms that can differentiate subtle variations within a category, such as fur patterns or ear shapes.

\begin{figure}[H]
	\centering
	\includegraphics[width=0.8\textwidth]{Figures/Chapter_2/Slide_13.jpg}
	\caption{Fine-grained classification requires distinguishing subtle differences within visually similar categories.}
	\label{fig:chapter2_fine_grained}
\end{figure}

\subsection{Background Clutter}
Objects in images often blend into complex or cluttered backgrounds, making it challenging to isolate the target object. For instance, a cat sitting amidst foliage may be difficult to distinguish due to natural camouflage or similar textures.

\begin{figure}[H]
	\centering
	\includegraphics[width=0.8\textwidth]{Figures/Chapter_2/Slide_14.jpg}
	\caption{Background clutter can obscure target objects, complicating image classification.}
	\label{fig:chapter2_background_clutter}
\end{figure}

\subsection{Illumination Changes}
Lighting conditions significantly impact the appearance of objects in images. A cat photographed in daylight might look very different when captured under dim lighting, even though its semantic identity remains unchanged.

\begin{figure}[H]
	\centering
	\includegraphics[width=0.8\textwidth]{Figures/Chapter_2/Slide_15.jpg}
	\caption{Variations in illumination conditions affect object appearance, requiring robust algorithms.}
	\label{fig:chapter2_illumination}
\end{figure}

\subsection{Deformation and Object Scale}
Objects are not rigid entities; they deform and appear at varying scales within images. For example, a cat lying stretched out versus curled up occupies different shapes and scales in an image.

\begin{figure}[H]
	\centering
	\includegraphics[width=0.8\textwidth]{Figures/Chapter_2/Slide_16.jpg}
	\caption{Objects can deform and appear at varying scales, posing challenges for classification.}
	\label{fig:chapter2_deformation_scale}
\end{figure}

\subsection{Occlusions}
Partial visibility of objects adds another layer of complexity to image classification. For instance, a cat partially hidden under a pillow, with only its tail visible, might be easily recognized by humans based on contextual reasoning. However, such occlusions often hinder algorithmic performance, as they obscure critical features necessary for classification.

\begin{figure}[H]
	\centering
	\includegraphics[width=0.8\textwidth]{Figures/Chapter_2/Slide_17.jpg}
	\caption{Occlusions, such as partial visibility of objects, obscure critical features and hinder classification.}
	\label{fig:chapter2_occlusions}
\end{figure}

\subsection{Summary of Challenges}

Image classification presents a range of challenges rooted in the complexities of visual data and the semantic gap between raw pixel values and meaningful categories. Developing effective algorithms requires bridging this gap while ensuring robustness to variations in viewpoint, illumination, occlusion, deformation, and other real-world conditions.

Traditional approaches to image classification, such as edge detection and corner detection combined with feature descriptors and matching, offered foundational insights into the problem. However, these classical methods often struggled to adapt to the diversity and unpredictability of real-world scenarios, limiting their effectiveness in practical applications.

The emergence of learning-based methods, particularly deep learning, has transformed the landscape by providing more robust and scalable solutions. These methods leverage techniques such as data augmentation and feature extraction to learn hierarchical representations directly from raw data. This ability to adapt and generalize across varying conditions has propelled significant advancements in classification performance, making these approaches the dominant paradigm in the field.

In the following sections, we will delve into these challenges and explore how learning-based methodologies address them. 

\section{Image Classification as a Building Block for Other Tasks}

Image classification serves as a foundational task in computer vision, enabling advancements in a variety of related applications. Its ability to assign meaningful labels to visual data allows more complex tasks to be framed as extensions of classification. In this section, we explore how image classification supports tasks such as object detection, image captioning, and decision-making in board games.

\subsection{Object Detection}

Object detection extends image classification by identifying not only the types of objects present in an image but also their locations. A robust image classifier can be utilized as a core component of an object detection pipeline by classifying regions within an image. 

One approach is to use a sliding window technique, where the image is divided into overlapping subregions. Each subregion is classified as either belonging to the background or containing an object. For regions identified as containing objects, the classifier further determines the type of object present. While this approach is computationally intensive and has limitations in handling scale and aspect ratio variations, it demonstrates how image classification can be repurposed to solve more advanced tasks.

\begin{figure}[H]
	\centering
	\includegraphics[width=0.7\textwidth]{Figures/Chapter_2/Slide_20.jpg}
	\caption{Using sliding windows for object detection: classifying regions as background or containing an object.}
	\label{fig:chapter2_sliding_window_bg}
\end{figure}

\begin{figure}[H]
	\centering
	\includegraphics[width=0.7\textwidth]{Figures/Chapter_2/Slide_21.jpg}
	\caption{Using sliding windows for object detection: classifying regions containing objects (e.g., person).}
	\label{fig:chapter2_sliding_window_person}
\end{figure}

\subsection{Image Captioning}

Image captioning involves generating a natural language description of the content in an image, a task that can also be framed as a sequence of classification problems. Given a fixed vocabulary of words, the algorithm determines the most fitting word at each step, effectively performing classification repeatedly until a complete sentence is formed. 

For example, starting with an input image, the first classification might yield the word "man," followed by "riding," then "horse," and eventually a "STOP" token to indicate the end of the caption. This process demonstrates how a robust image classifier can form the backbone of a more complex multimodal task that bridges vision and language.

\begin{figure}[H]
	\centering
	\includegraphics[width=0.8\textwidth]{Figures/Chapter_2/Slide_22.jpg}
	\caption{Image captioning as sequential classification: determining the first word (e.g., "man").}
	\label{fig:chapter2_caption_man}
\end{figure}

\begin{figure}[H]
	\centering
	\includegraphics[width=0.8\textwidth]{Figures/Chapter_2/Slide_23.jpg}
	\caption{Image captioning as sequential classification: determining the next word (e.g., "riding").}
	\label{fig:chapter2_caption_riding}
\end{figure}

\begin{figure}[H]
	\centering
	\includegraphics[width=0.8\textwidth]{Figures/Chapter_2/Slide_25.jpg}
	\caption{Image captioning: determining the end of the sentence with a "STOP" token.}
	\label{fig:chapter2_caption_stop}
\end{figure}

\subsection{Decision-Making in Board Games}

Board games such as Go provide another example of framing a complex task as a classification problem. Each position on the board can be viewed as an input to the algorithm, with the goal of classifying which position is most optimal for the next move. This approach enables algorithms to make strategic decisions by treating each potential move as a classification instance, demonstrating the versatility of image classification as a problem-solving tool.

\begin{figure}[H]
	\centering
	\includegraphics[width=0.8\textwidth]{Figures/Chapter_2/Slide_26.jpg}
	\caption{Board games like Go framed as classification problems: determining the optimal next move.}
	\label{fig:chapter2_board_games}
\end{figure}

\newpage
\subsection{Summary: Leveraging Image Classification}

Image classification is not just an isolated task but a fundamental building block for diverse applications in computer vision and artificial intelligence. By leveraging classification in tasks such as object detection, image captioning, and decision-making, researchers have been able to extend its utility and address increasingly complex problems. This underscores the importance of developing robust image classifiers, as they form the foundation for solving more sophisticated challenges.

\section{Constructing an Image Classifier}

Designing an image classifier is a complex process that cannot be reduced to a simple function like \texttt{def classify\_img(...)}, which takes an image tensor and directly outputs a category label. This complexity arises from the inherent challenges of translating raw pixel values into meaningful categories. Over time, the field has transitioned from feature-based methods to data-driven approaches, reflecting the increasing complexity of real-world applications.

\subsection{Feature-Based Image Classification: The Classical Approach}

Traditional strategies for building an image classifier rely on explicit feature extraction and rule-based classification:

\begin{itemize}
	\item \textbf{Edge Detection:} Algorithms like the \textit{Canny Edge Detector} \cite{canny1986_edgedetection} are used to identify object boundaries by detecting abrupt changes in pixel intensity.
	\item \textbf{Keypoint Detection:} Techniques such as the \textit{Harris Corner Detector} \cite{harris1988_combined} locate distinctive features like corners in the image.
	\item \textbf{Rule-Based Classification:} Incorporating human knowledge, explicit rules are devised to classify objects based on extracted features. For example, cats might be identified by triangular ears and whiskers.
\end{itemize}

While this approach provides a structured framework, it faces critical challenges:
\begin{itemize}
	\item \textbf{Variability:} Real-world objects exhibit significant variations, such as cats with or without whiskers.
	\item \textbf{Failure Points:} Feature detectors often fail under challenging conditions, such as poor lighting or occlusion.
	\item \textbf{Scalability:} Adding new categories requires rewriting rules and redesigning algorithms, limiting adaptability.
\end{itemize}

\begin{figure}[H]
	\centering
	\includegraphics[width=0.7\textwidth]{Figures/Chapter_2/Slide_27.jpg}
	\caption{Attempting to classify images using hard-coded features is highly challenging.}
	\label{fig:chapter2_classification_attempt}
\end{figure}

\begin{figure}[H]
	\centering
	\includegraphics[width=0.8\textwidth]{Figures/Chapter_2/Slide_28.jpg}
	\caption{Edges and corners as features for classification: an incomplete solution.}
	\label{fig:chapter2_edge_corners}
\end{figure}

\subsection{From Hand-Crafted Rules to Data-Driven Learning}

Machine learning revolutionized image classification by replacing manual feature engineering with automated learning from data. This modern approach is defined by three key steps:

\begin{enumerate}
	\item \textbf{Dataset Collection:} Collect a large dataset of images and their corresponding human-annotated labels.
	\item \textbf{Model Training:} Train a machine learning model to learn patterns and representations directly from the dataset.
	\item \textbf{Prediction and Evaluation:} Use the trained model to predict labels for unseen images and evaluate performance using metrics like accuracy or log-likelihood.
\end{enumerate}

\newpage
This pipeline, illustrated in Figure \ref{fig:chapter2_data_driven}, modularizes the process into two core functions: \texttt{train(images, labels)} and \texttt{predict(model, test\_images)}.

\begin{figure}[H]
	\centering
	\includegraphics[width=0.8\textwidth]{Figures/Chapter_2/Slide_29.jpg}
	\caption{A data-driven pipeline for training and evaluating machine learning-based image classifiers.}
	\label{fig:chapter2_data_driven}
\end{figure}

\subsection{Programming with Data: The Modern Paradigm}

Data-driven approaches redefine how we "program" computers. Instead of hard-coding rules, we train models by providing labeled datasets, allowing the algorithm to learn from examples. This shift offers significant advantages:

\begin{itemize}
	\item \textbf{Scalability:} Easily adapts to new categories by adding labeled data.
	\item \textbf{Robustness:} Handles diverse conditions, such as lighting changes and occlusions, without manual adjustments.
	\item \textbf{Automation:} Eliminates the need for explicit, domain-specific rules, making it suitable for complex, real-world tasks.
\end{itemize}

For example, a data-driven model trained on images of animals learns nuanced distinctions—like fur patterns and body shapes—directly from the data, avoiding the need for manually encoding such rules.

\subsection{Data-Driven Machine Learning: The New Frontier}

Datasets are the foundation of modern machine learning, enabling models to learn directly from examples. Unlike traditional algorithm-driven approaches, where progress relied on better feature engineering, data-driven methods depend on the quality, diversity, and scale of datasets.

\textbf{Why Data Dominates:}
\begin{itemize}
	\item \textbf{Generalization:} Models trained on diverse datasets can generalize to unseen data better than those relying on hand-crafted features.
	\item \textbf{Flexibility:} New tasks and domains require only new data, not redesigned algorithms.
	\item \textbf{Empirical Strength:} Data-driven methods align closely with real-world variability, capturing patterns that are infeasible to encode manually.
\end{itemize}

\newpage
The growing emphasis on datasets reflects a paradigm shift in machine learning research, where the quality of data often outweighs incremental algorithmic improvements. This approach empowers models to tackle the complexities of real-world visual data effectively.

In the next section, we delve into the critical role of datasets in machine learning, exploring how they shape the development and performance of image classification models.

\section{Datasets in Image Classification}

Datasets form the backbone of modern machine learning and computer vision, defining the scope and quality of what algorithms can learn. Over the years, datasets in image classification have evolved in scale, complexity, and diversity, shaping the trajectory of the field. In this section, we explore some of the most prominent datasets used in image classification, their characteristics, and their role in advancing research.

\subsection{MNIST: The Toy Dataset}

The \textbf{MNIST} dataset \cite{lecun1998_lenet} is one of the earliest and most iconic datasets in machine learning. It consists of $28 \times 28$ grayscale images of handwritten digits (0--9), making it a 10-class classification problem. With 50,000 training images and 10,000 test images, MNIST has been pivotal in demonstrating the power of early machine learning algorithms.

While MNIST is often referred to as the \textit{Drosophila of computer vision}, it is considered a toy dataset due to its simplicity and small size. Achieving high accuracy on MNIST does not necessarily translate to success on more complex datasets, limiting its utility in benchmarking modern algorithms.

\begin{figure}[H]
	\centering
	\includegraphics[width=0.7\textwidth]{Figures/Chapter_2/Slide_31.jpg}
	\caption{MNIST: A dataset of handwritten digits, often used as a toy benchmark.}
	\label{fig:chapter2_mnist}
\end{figure}

\subsection{CIFAR: Real-World Object Recognition}

The \textbf{CIFAR-10} dataset \cite{krizhevsky2009_learning} represents a significant step forward in dataset complexity. It contains 10 classes of objects (e.g., airplane, automobile, bird, cat) with $32 \times 32 \times 3$ RGB images. The dataset includes 50,000 training images (5,000 per class) and 10,000 test images (1,000 per class). 

\newpage
CIFAR-10 strikes a balance between complexity and computational feasibility, making it ideal for research and teaching purposes. In this course, CIFAR-10 is the primary dataset for homework assignments.

CIFAR-10 has a \textit{cousin}, \textbf{CIFAR-100}, which has similar statistics but with 100 categories instead of 10. These 100 categories are grouped into 20 superclasses, each containing five finer-grained classes. For example, the \textit{Aquatic Animals} superclass includes classes like beaver, dolphin, otter, seal, and whale.

\begin{figure}[H]
	\centering
	\includegraphics[width=0.8\textwidth]{Figures/Chapter_2/Slide_32.jpg}
	\caption{CIFAR-10: A dataset for object classification with 10 categories.}
	\label{fig:chapter2_cifar10}
\end{figure}

\begin{figure}[H]
	\centering
	\includegraphics[width=0.8\textwidth]{Figures/Chapter_2/Slide_33.jpg}
	\caption{CIFAR-100: An extension of CIFAR-10 with 100 categories.}
	\label{fig:chapter2_cifar100}
\end{figure}

\newpage
\subsection{ImageNet: The Gold Standard}

\textbf{ImageNet} \cite{imagenet2009_hierarchicaldatabase} is a cornerstone dataset in computer vision, widely used for benchmarking image classification algorithms. It comprises over 1.3 million training images across 1,000 categories, with approximately 1,300 images per category, alongside 50,000 validation images and 100,000 test images.

Collected from the internet, ImageNet images vary in resolution but are typically resized to $256 \times 256 \times 3$ for training and evaluation. Its scale and diversity challenge algorithms to generalize effectively, making it ideal for assessing robustness.

\begin{figure}[H]
	\centering
	\includegraphics[width=0.8\textwidth]{Figures/Chapter_2/Slide_34.jpg}
	\caption{ImageNet: A dataset of 1,000 categories pivotal to computer vision progress.}
	\label{fig:chapter2_imagenet}
\end{figure}

ImageNet's \textbf{top-5 accuracy} metric allows the algorithm to succeed if the correct label appears among its top 5 predictions, accommodating label ambiguities and noise in the data.

\begin{figure}[H]
	\centering
	\includegraphics[width=0.8\textwidth]{Figures/Chapter_2/Slide_35.jpg}
	\caption{ImageNet top-5 accuracy: A widely adopted evaluation metric.}
	\label{fig:chapter2_imagenet_top5}
\end{figure}

\newpage
\subsection{MIT Places: Scene Recognition}

While datasets like ImageNet focus on object recognition, the \textbf{MIT Places} dataset \cite{zhou2017_places} emphasizes scene classification, with categories like classrooms, fields, and buildings. This shift from object-centric to scene-centric data broadens the scope of image classification research, enabling algorithms to analyze broader contexts in visual data.

\begin{figure}[H]
	\centering
	\includegraphics[width=0.8\textwidth]{Figures/Chapter_2/Slide_36.jpg}
	\caption{MIT Places: A dataset for scene classification, focusing on diverse environmental contexts.}
	\label{fig:chapter2_places}
\end{figure}

\subsection{Comparing Dataset Sizes}

Figure \ref{fig:chapter2_dataset_sizes} compares the sizes of these datasets in terms of the total number of pixels in their training sets. The y-axis, plotted on a logarithmic scale, reveals a clear trend:

\begin{itemize}
	\item CIFAR is roughly an order of magnitude larger than MNIST.
	\item ImageNet is approximately two orders of magnitude larger than CIFAR.
	\item MIT Places is yet another order of magnitude larger than ImageNet.
\end{itemize}

This trend reflects the increasing scale of datasets over time, driven by the need for diverse and comprehensive training data. Larger datasets like ImageNet yield more convincing results but demand significant computational resources, which is why smaller datasets like CIFAR remain popular for teaching and prototyping.

\begin{figure}[H]
	\centering
	\includegraphics[width=0.8\textwidth]{Figures/Chapter_2/Slide_37.jpg}
	\caption{Comparing dataset sizes: MNIST, CIFAR, ImageNet, and MIT Places.}
	\label{fig:chapter2_dataset_sizes}
\end{figure}

\subsection{Omniglot: Few-Shot Learning}

As datasets grow larger, an emerging research direction focuses on learning from limited data. The \textbf{Omniglot} dataset \cite{lake2015_human} exemplifies this shift by providing only 20 examples per category. Omniglot contains handwritten characters from over 50 different alphabets, emphasizing the challenge of \textit{few-shot learning}, where algorithms must generalize from minimal examples.

\begin{figure}[H]
	\centering
	\includegraphics[width=0.8\textwidth]{Figures/Chapter_2/Slide_38.jpg}
	\caption{Omniglot: A dataset for few-shot learning, with minimal examples per category.}
	\label{fig:chapter2_omniglot}
\end{figure}

\subsection{Conclusion: Datasets Driving Progress}

Datasets set the limits of model capabilities, with larger and more diverse datasets enabling breakthroughs in image classification. Specialized datasets like Omniglot address challenges like few-shot learning, emphasizing the evolving needs of the field. In data-driven methodologies, the quality and diversity of datasets remain pivotal in advancing computer vision.

\section{Nearest Neighbor Classifier: A Gateway to Understanding Classification}

The \textbf{Nearest Neighbor} (NN) classifier is one of the simplest and most intuitive machine learning algorithms. While seemingly elementary, it introduces key concepts that are foundational to the field of classification. By starting with Nearest Neighbor, we gain a clear understanding of the principles of classification and the practical challenges that arise in real-world scenarios, particularly when working with high-dimensional data.

\subsection{Why Begin with Nearest Neighbor?}

The Nearest Neighbor algorithm serves as an excellent starting point for exploring machine learning for several reasons:

\begin{itemize}
	\item \textbf{Simplicity}: The algorithm's straightforward design—based on memorizing data and comparing distances—makes it easy to understand and implement.
	\item \textbf{Foundational Concepts}: It introduces the idea of \textit{similarity metrics}, the importance of \textit{distance functions}, and the impact of \textit{training data quality}.
	\item \textbf{Real-World Limitations}: Despite its theoretical appeal, Nearest Neighbor highlights practical challenges such as high inference time, sensitivity to noise, and the \textit{curse of dimensionality}, motivating the development of more sophisticated algorithms.
\end{itemize}

By dissecting the Nearest Neighbor classifier, we lay the groundwork for understanding modern approaches to robust classification.

\subsection{Setting the Stage: From Pixels to Predictions}

The fundamental task of a classifier is to assign a category label to an input image, bridging the gap between raw pixel data and semantic meaning. For example, given an image of a cat, the classifier should return the label "cat." Achieving this requires a method for comparing the input image to previously seen examples and determining the most appropriate label.

Nearest Neighbor does this by comparing the test image to all training images and selecting the label of the most similar one. This approach, while naive, provides valuable insight into the role of similarity in classification and serves as a stepping stone toward more advanced machine learning models.

The following sections detail the algorithm, its components, and the practical considerations for its use.


\subsection{Algorithm Description}

The Nearest Neighbor classifier operates using two primary methods:
\begin{itemize}
	\item \textbf{\texttt{train}:} Memorizes the training data and corresponding labels without any additional computation.
	\item \textbf{\texttt{predict}:} For a test image, computes the distance to all training images using a similarity function or distance metric. The label of the most similar training image is returned as the prediction.
\end{itemize}

\begin{figure}[H]
	\centering
	\includegraphics[width=0.8\textwidth]{Figures/Chapter_2/Slide_39.jpg}
	\caption{Nearest Neighbor classifier: memorize training data and predict based on the closest match.}
	\label{fig:chapter2_nn_description}
\end{figure}

\subsection{Distance Metrics: The Core of Nearest Neighbor}

The distance metric determines how "similar" two images are. The most common choices include:
\begin{itemize}
	\item \textbf{L1 Distance (Manhattan Distance):} Computes the sum of absolute differences between corresponding pixel values:
	\[
	\text{L1 Distance} = \sum_{i=1}^{n} |x_i - y_i|
	\]
	
	\begin{figure}[H]
		\centering
		\includegraphics[width=0.8\textwidth]{Figures/Chapter_2/Slide_40.jpg}
		\caption{L1 distance example: a simple and interpretable metric.}
		\label{fig:chapter2_l1_distance}
	\end{figure}
	
	\item \textbf{L2 Distance (Euclidean Distance):} Computes the root of the sum of squared differences:
	\[
	\text{L2 Distance} = \sqrt{\sum_{i=1}^{n} (x_i - y_i)^2}
	\]
	
	\begin{figure}[H]
		\centering
		\includegraphics[width=0.8\textwidth]{Figures/Chapter_2/Slide_63.jpg}
		\captionsetup{singlelinecheck=off}
		\caption[]{
			Comparison of L1 and L2 norm constraint regions in two dimensions.
			\begin{itemize}
				\item \textbf{Left:} The L1 norm constraint \( |w_1| + |w_2| = c \) defines a region bounded by four linear segments, forming a diamond. This shape arises because the absolute value function grows linearly and independently in each coordinate, so all points satisfying the constraint lie along lines where the sum of the horizontal and vertical distances equals \( c \).
				\item \textbf{Right:} The L2 norm constraint \( w_1^2 + w_2^2 = c^2 \) forms a circle, as it includes all points at Euclidean distance \( c \) from the origin $(0,0)$. The quadratic form symmetrically penalizes all directions, yielding a smooth, round boundary.
			\end{itemize}
		}
		\label{fig:chapter2_l1_l2_comparison}
	\end{figure}
\end{itemize}

\begin{figure}[H]
	\centering
	\includegraphics[width=0.8\textwidth]{Figures/Chapter_2/Slide_64.jpg}
	\captionsetup{singlelinecheck=off}
	\caption[]{
		Decision Boundaries for L1 vs. L2 Metrics. In distance-based methods (e.g., nearest neighbors), the choice of distance metric shapes the decision regions:
		\begin{itemize}
			\item \textbf{L1 (Manhattan)}: Diamond-shaped boundaries (as points at the same L1 distance form axis-aligned corners).
			\item \textbf{L2 (Euclidean)}: Circular (or spherical) boundaries, since points equidistant in Euclidean space lie on circles (or spheres).
		\end{itemize}
	}
	\label{fig:chapter2_l1_l2_comparison_boundaries}
\end{figure}


\begin{figure}[H]
	\centering
	\includegraphics[width=0.8\textwidth]{Figures/Chapter_2/Slide_51.jpg}
	\caption{Limitations of L1 distance: visually dissimilar objects with similar colors may be incorrectly classified.}
	\label{fig:chapter2_l1_poor_performance}
\end{figure}

Each metric produces different decision boundaries, as shown in Slide \ref{fig:chapter2_l1_l2_comparison_boundaries}. However, these pixel-based metrics often fail to capture semantic similarity. For instance, as demonstrated in Slide \ref{fig:chapter2_l1_poor_performance}, visually dissimilar objects with similar colors (e.g., a ginger cat and an orange frog) may appear "close" under L1 or L2 distance.

\subsection{Extending Nearest Neighbor: Applications Beyond Images}

While the Nearest Neighbor classifier is typically discussed in the context of image classification, its principles can be extended to other domains and data types, provided a suitable distance function is defined. 

\subsubsection{Using Nearest Neighbor for Non-Image Data}

One compelling example of Nearest Neighbor applied to non-image data involves the analysis of academic papers. In this context, similarity between documents is measured using \textbf{TF-IDF similarity} (Term Frequency--Inverse Document Frequency). This metric captures the importance of words in a document relative to a collection of documents, emphasizing unique and meaningful terms while downplaying common ones like ``the'' or ``and.''

\begin{itemize}
	\item \textbf{Term Frequency (TF):} Measures how often a term appears in a document, providing a sense of relevance within that document.
	\item \textbf{Inverse Document Frequency (IDF):} Reduces the weight of terms that appear frequently across many documents, as these are less likely to be unique or significant.
	\item \textbf{TF-IDF Score:} Combines TF and IDF to assign a weight to each term in a document, capturing its importance within a specific context.
\end{itemize}

\subsubsection{Academic Paper Recommendation Example}

\begin{figure}[H]
	\centering
	\includegraphics[width=0.8\textwidth]{Figures/Chapter_2/Slide_67.jpg}
	\caption{Nearest Neighbor using TF-IDF similarity for academic paper recommendations.}
	\label{fig:chapter2_nn_tfidf}
\end{figure}

Using TF-IDF scores, we can represent each academic paper as a feature vector. Nearest Neighbor can then be employed to find similar papers by comparing these vectors in feature space. As illustrated in Figure~\ref{fig:chapter2_nn_tfidf}, querying for a specific paper returns its closest neighbors based on semantic similarity.

\newpage

In practice, we often retrieve not just the single nearest document, but the top $k$ most similar papers. Here, $k$ denotes the number of neighbors returned (for example, $k=5$ yields the five most similar papers). This approach is commonly referred to as \textit{$k$-Nearest Neighbors}.

This example highlights the flexibility of Nearest Neighbor for non-image data. By choosing an appropriate similarity metric such as TF-IDF, we can uncover meaningful relationships between documents and build effective recommendation systems.

\subsubsection{Key Insights}

The versatility of Nearest Neighbor and K-means clustering stems from their reliance on distance metrics. This allows these algorithms to adapt to various applications, including:
\begin{itemize}
	\item Recommending academic papers based on content similarity.
	\item Grouping documents into clusters for topic modeling.
	\item Analyzing user behavior in recommendation systems.
\end{itemize}

This flexibility makes these methods powerful tools not only in computer vision but also in broader machine learning contexts.


\subsection{Hyperparameters in Nearest Neighbor}

The performance of the Nearest Neighbor classifier depends on two hyperparameters:
\begin{itemize}
	\item \textbf{k:} The number of nearest neighbors.
	\item \textbf{Distance Metric:} Determines how similarity between images is computed.
\end{itemize}

Selecting the best hyperparameters is challenging because they cannot be directly learned from training data.

Some strategies we can think of to select model hyperparameters include:
\begin{enumerate}
	\item \textbf{Using the Entire Dataset:} Selecting hyperparameters that optimize performance on the training set is misleading. For example, \(k=1\) will perform good in this case, and may lead to overfitting and poor generalization.
	\item \textbf{Train-Test Split Without Validation:} While better than the previous method, this approach lacks a clear mechanism to evaluate model performance on unseen data.
	\item \textbf{Train-Validation-Test Split:} The recommended practice is to split data into three sets:
	\begin{itemize}
		\item \textbf{Training Set:} Used for fitting the model.
		\item \textbf{Validation Set:} Helps tune hyperparameters and assess overfitting.
		\item \textbf{Test Set:} Evaluates the final model after all hyperparameters are fixed.
	\end{itemize}
	The test set is only evaluated once, at the very end, as emphasized in Slide \ref{fig:chapter2_train_val_test}. Although scary (as we tend to develop an algorithm spending a lot of time) as the resultant algorithm may prove to be ineffective on the test data, this is the correct data hygiene approach. The test set should only be used once, and near the end of the development. 
\end{enumerate}

\begin{figure}[H]
	\centering
	\includegraphics[width=0.8\textwidth]{Figures/Chapter_2/Slide_75.jpg}
	\caption{Train-validation-test split for robust evaluation.}
	\label{fig:chapter2_train_val_test}
\end{figure}

\subsection{Cross-Validation}

For smaller datasets, \textbf{k-fold cross-validation} is a widely used strategy for reliable model evaluation and hyperparameter tuning. 
\underline{Note:} this method should not be confused with \emph{k-nearest neighbors}; it refers to a technique for assessing a model’s generalization ability.

In k-fold cross-validation, the available dataset is first split into two parts:
\begin{itemize}
	\item A dedicated \textbf{test set}, held out and untouched until the final evaluation.
	\item A \textbf{training+validation set}, which is used for model selection and cross-validation.
\end{itemize}

This training+validation set is then partitioned into \(k\) equally sized folds. For each of the \(k\) iterations, one fold is treated as a temporary validation set, and the remaining \(k-1\) folds are used for training. 

\newpage
The process is repeated \(k\) times such that every fold serves as validation exactly once. The resulting performance metrics (e.g., accuracy or loss) are averaged across the \(k\) runs to produce a more stable and reliable estimate of model quality.

\begin{figure}[H]
	\centering
	\includegraphics[width=0.8\textwidth]{Figures/Chapter_2/Slide_77.jpg}
	\caption{Cross-validation accuracy for different values of \(k\). 
		Each dot represents an individual trial, and the mean accuracy across folds is shown by the line. 
		In this example, \(k = 7\) yields the highest average validation performance, so it is selected.}
	\label{fig:chapter2_cross_validation}
\end{figure}

This method provides a robust mechanism for hyperparameter selection and model comparison, particularly when limited data makes a single train-validation split unreliable. It is important to emphasize that the \textbf{test set is never used during cross-validation}. It remains completely separate and is reserved for the final, unbiased evaluation of the trained model, only after all tuning is complete.

While cross-validation is computationally feasible and valuable for smaller datasets or shallow models, it becomes impractical for large-scale datasets or deep learning models due to the repeated training involved. For such cases, a single train-validation split is typically used during training instead (often paired with mechanisms such as early stopping).

\newpage
\subsection{Implementation and Complexity}

The simplicity of Nearest Neighbor allows for a straightforward implementation:

\begin{figure}[H]
	\centering
	\includegraphics[width=0.8\textwidth]{Figures/Chapter_2/Slide_42.jpg}
	\caption{\texttt{train} method: Memorizing training data.}
	\label{fig:chapter2_train}
\end{figure}

\begin{figure}[H]
	\centering
	\includegraphics[width=0.8\textwidth]{Figures/Chapter_2/Slide_43.jpg}
	\caption{\texttt{predict} method: Computing similarity and predicting the closest label.}
	\label{fig:chapter2_predict}
\end{figure}

\textbf{Complexity:}
\begin{itemize}
	\item \textbf{Training:} \(O(1)\), as it simply stores the data.
	\item \textbf{Inference:} \(O(n)\), as every test image is compared against all \(n\) training images. This makes inference computationally expensive, particularly for large datasets.
\end{itemize}

Faster or approximate versions of Nearest Neighbor exist, taking advantage of spatial data structures like KD-trees to accelerate the search process.

\subsection{Visualization of Decision Boundaries}

Decision boundaries illustrate the regions in the input space assigned to different classes. For a 2D toy dataset, the decision boundaries of Nearest Neighbor are highly irregular and sensitive to outliers.

\begin{figure}[H]
	\centering
	\includegraphics[width=0.8\textwidth]{Figures/Chapter_2/Slide_52.jpg}
	\caption{Decision boundaries for Nearest Neighbor on a 2D dataset.}
	\label{fig:chapter2_decision_boundaries_start}
\end{figure}

Outliers can create "islands" of incorrect predictions, as shown in Slide \ref{fig:chapter2_outlier_effect}. This sensitivity makes \(k=1\) particularly problematic.

\begin{figure}[H]
	\centering
	\includegraphics[width=0.8\textwidth]{Figures/Chapter_2/Slide_58.jpg}
	\caption{Outliers disrupting decision boundaries in Nearest Neighbor classification.}
	\label{fig:chapter2_outlier_effect}
\end{figure}

\subsection{Improvements: k-Nearest Neighbors}

The \textbf{k-Nearest Neighbors} (k-NN) algorithm improves upon Nearest Neighbor by considering the \(k\) closest neighbors and using a majority vote to determine the predicted label. This smooths decision boundaries and reduces the impact of outliers.

\begin{figure}[H]
	\centering
	\includegraphics[width=0.8\textwidth]{Figures/Chapter_2/Slide_61.jpg}
	\caption{k-Nearest Neighbors ($k=3$): Smoother decision boundaries and reduced outlier influence.}
	\label{fig:chapter2_knn_smoothing}
\end{figure}

However, k-NN introduces challenges such as ties between classes when \(k>1\). These ties can be resolved using heuristics like selecting the label of the nearest neighbor or weighting votes by distance.

\subsection{Limitations and Universal Approximation}

With infinite training data, Nearest Neighbor can theoretically approximate any function. This is due to its capacity to memorize and interpolate training examples. As more points are added, the decision boundaries become increasingly fine-grained, capturing ever subtler data patterns.

\begin{figure}[H]
	\centering
	\includegraphics[width=0.8\textwidth]{Figures/Chapter_2/Slide_81.jpg}
	\caption{A step towards a dense coverage with Nearest Neighbor.}
	\label{fig:chapter2_dense_coverage}
\end{figure}

However, the practicality of this property is severely limited by the \textbf{curse of dimensionality}. For high-dimensional data, the number of required training samples grows exponentially. For example:
\begin{itemize}
	\item In \(2\) dimensions, uniform coverage requires \(4^2 = 16\) points.
	\item In \(3\) dimensions, \(4^3 = 64\) points are necessary.
	\item For even modestly sized images like \(32 \times 32\), the number of possible binary images is astronomical: \(2^{32 \times 32} \approx 10^{308}\), far exceeding the number of elementary particles in the visible universe (\(\approx 10^{97}\)).
\end{itemize}

\begin{figure}[H]
	\centering
	\includegraphics[width=0.8\textwidth]{Figures/Chapter_2/Slide_85.jpg}
	\caption{The curse of dimensionality: limitations of KNN in high-dimensional spaces.}
	\label{fig:chapter2_curse_dimensionality}
\end{figure}

This exponential growth renders dense coverage impossible for real-world datasets, as highlighted in Slide \ref{fig:chapter2_curse_dimensionality}. Furthermore, we are often dealing with real-valued RGB images of even higher resolutions, adding to the complexity.

\subsection{Using CNN Features for Nearest Neighbor Classification}

Pixel-based distance metrics, such as L1 and L2 distances, often fail to capture \textbf{semantic similarity}. As highlighted in Slide \ref{fig:chapter2_l1_poor_performance}, objects with similar pixel intensities, such as a ginger cat and an orange frog, may be incorrectly classified as similar despite their clear visual and categorical differences. This limitation underscores the need for more sophisticated representations that go beyond raw pixel comparisons.

A promising solution is to replace raw pixel distances with feature distances derived from \textbf{convolutional neural networks (CNNs)}. CNNs are adept at capturing higher-level semantic information by learning hierarchical feature representations directly from data. These features can effectively bridge the gap between low-level pixel values and meaningful object categories, enabling Nearest Neighbor classifiers to make more informed predictions,  bridging the semantic gap L1 or L2 metrics applied to raw pixel values face.

\begin{figure}[H]
	\centering
	\includegraphics[width=0.8\textwidth]{Figures/Chapter_2/Slide_87.jpg}
	\caption{Nearest Neighbor with CNN features: improved semantic similarity.}
	\label{fig:chapter2_nn_cnn}
\end{figure}

Slide \ref{fig:chapter2_nn_cnn} demonstrates the effectiveness of this approach, showcasing improved classification performance when CNN-derived features are paired with Nearest Neighbor classifiers. By leveraging these features, the algorithm becomes more robust to variations in lighting, scale, and viewpoint, which are challenging for pixel-based metrics to handle.

This method has proven particularly effective in various tasks, including \textbf{image captioning}. In their work \cite{devlin2015_imagetocaption}, Devlin et al. (2015) proposed an approach that combines Nearest Neighbor with CNN features to generate captions for images. The algorithm retrieves the most similar image from the training set (based on CNN feature similarity) and reuses its caption as the prediction. While simplistic, this method delivered coherent and contextually relevant captions. 

\begin{figure}[H]
	\centering
	\includegraphics[width=0.8\textwidth]{Figures/Chapter_2/Slide_88.jpg}
	\caption{Nearest Neighbor captioning: retrieving captions from the closest matching image.}
	\label{fig:chapter2_nn_captioning}
\end{figure}

These are examples as to how Nearest Neighbor classifiers, when augmented with learned features, can tackle complex tasks beyond basic classification. 

\subsection{Conclusion: From Nearest Neighbor to Advanced ML Frontiers}

The Nearest Neighbor classifier highlights the balance between simplicity and capability, offering theoretical guarantees such as universal function approximation. However, its reliance on raw pixel metrics limits its practical applications, particularly in high-dimensional spaces. By incorporating feature representations from CNNs, Nearest Neighbor classifiers can overcome many of these limitations, improving performance in tasks ranging from image classification to captioning. These advancements pave the way for exploring even more sophisticated machine learning algorithms and architectures in subsequent sections.








\chapterimage{head2.png} % Chapter heading image
% Chapter-specific content starts here
\chapter{Lecture 3: Linear Classifiers}

%----------------------------------------------------------------------------------------
%	CHAPTER 3 - Lecture 3: Linear Classifiers
%----------------------------------------------------------------------------------------

\section{Linear Classifiers: A Foundation for Neural Networks}

Linear classifiers are a cornerstone of machine learning and form one of the most fundamental building blocks for modern neural networks.

\begin{figure}[H]
	\centering
	\includegraphics[width=0.8\textwidth]{Figures/Chapter_3/Slide_7.jpg}
	\caption{Neural networks are constructed from stacked building blocks, much like Lego blocks. Linear classifiers are one of these foundational components.}
	\label{fig:chapter3_lego_blocks}
\end{figure}

As illustrated in Figure \ref{fig:chapter3_lego_blocks}, neural networks are constructed by stacking basic components, with linear classifiers serving as one of the foundational elements. Despite their simplicity, linear classifiers play a critical role in providing a structured, parametric framework that maps raw input data to class scores. They naturally extend to more sophisticated architectures, such as neural networks and convolutional neural networks (CNNs).

This chapter focuses on linear classifiers and their role in classification problems. To develop a comprehensive understanding of their behavior and limitations, we will examine linear classifiers from three perspectives:
\begin{itemize}
	\item \textbf{Algebraic Viewpoint:} Frames the classifier as a mathematical function, emphasizing the score computation as a weighted combination of input features and biases.
	\item \textbf{Visual Viewpoint:} Reveals how the classifier learns templates for each class and compares them with input images, highlighting its behavior as a form of template matching.
	\item \textbf{Geometric Viewpoint:} Interprets the classifier's decision-making process in high-dimensional spaces, with hyperplanes dividing the space into regions corresponding to different classes.
\end{itemize}

These viewpoints not only help us understand the mechanics of linear classifiers but also shed light on their inherent limitations, such as their inability to handle non-linearly separable data or account for multiple modes in class distributions.

Finally, we introduce the key components of linear classifiers:
\begin{itemize}
	\item A \textbf{score function} that maps input data to class scores.
	\item A \textbf{loss function} that quantifies the model's performance by comparing predictions to ground truth labels.
\end{itemize}

While this chapter will focus on understanding these perspectives and defining loss functions, we will leave the topics of optimization and regularization for the next lecture, where we will discuss how to effectively train and refine linear classifiers.

\begin{figure}[H]
	\centering
	\includegraphics[width=0.8\textwidth]{Figures/Chapter_3/Slide_12.jpg}
	\caption{Parametric linear classifier pipeline: The input image is flattened into a vector, multiplied with weights, and added to a bias vector to produce class scores.}
	\label{fig:chapter3_parametric_classifier}
\end{figure}

As seen in Slide \ref{fig:chapter3_parametric_classifier}, linear classifiers adopt a parametric approach where the input image \(\mathbf{x}\) (e.g., a \(32 \times 32 \times 3\) RGB image of a cat) is flattened into a single vector of pixel values of length \(D = 3072\). This flattening is performed consistently across all input images to maintain structural uniformity. Given \(K = 10\) classes, the classifier outputs 10 scores, one for each class. This is achieved using the function:
\[
f(\mathbf{x}, W, \mathbf{b}) = W\mathbf{x} + \mathbf{b},
\]
where \(W\) is a learnable weight matrix of shape \(K \times D\), and \(\mathbf{b}\) is a learnable bias vector of shape \(K\). 

The weight matrix \(W\) and the bias term \(\mathbf{b}\) work together to define the decision boundary in a linear classifier. To build intuition, consider the simple example of a linear equation \(y = mx + b\) in two dimensions. In this equation:
\begin{itemize}
	\item \(m\) determines the slope of the line, dictating how steeply it tilts.
	\item \(b\) shifts the line vertically, allowing it to move up or down along the \(y\)-axis. This effectively changes where the line crosses the axis, without altering its slope.
\end{itemize}

Similarly, in a linear classifier, the decision boundary is represented as \(W \mathbf{x} + \mathbf{b} = 0\), where:
\begin{itemize}
	\item The weight matrix \(W\) determines the orientation and steepness of the decision boundary in the input space by defining how the features \(\mathbf{x}\) combine to produce class scores.
	\item The bias term \(\mathbf{b}\), independent of the input features \(\mathbf{x}\), offsets the decision boundary. This shifts the hyperplane in the feature space, much like \(b\) in \(y = mx + b\) shifts the line vertically.
\end{itemize}
\newpage
\begin{enrichment}[Understanding the Role of Bias in Linear Classifiers][subsection]

The bias term \(\mathbf{b}\) in linear classifiers allows the decision boundary to shift, enabling the model to handle data distributions that are not centered at the origin. This flexibility is essential, as demonstrated in the following example:

\begin{example}
	Consider a classification task in 2D space with two data points:
	\begin{itemize}
		\item Red point (Class 1): \((1, 1)\).
		\item Blue point (Class 2): \((2, 2)\).
	\end{itemize}
	
	The decision boundary is defined as:
	\[
	W \mathbf{x} + b = 0 \implies -x - y + b = 0.
	\]
	
\paragraph{Without Bias (\(b=0\)):}
	The decision boundary simplifies to:
	\[
	w_1 x + w_2 y = 0 \quad\Longrightarrow\quad y = -\tfrac{w_1}{w_2}\,x.
	\]
	Suppose we want the model to classify:
	\[
	\text{Red point }(1,1)\;\text{on one side, and Blue point }(2,2)\;\text{on the other.}
	\]
	Concretely, for \((x,y)\) in Class 1, we want:
	\[
	w_1\cdot 1 + w_2\cdot 1 > 0,
	\]
	and for Class 2 we want:
	\[
	w_1\cdot 2 + w_2\cdot 2 < 0.
	\]
	These inequalities become:
	\[
	\begin{cases}
		w_1 + w_2 > 0, \\
		2w_1 + 2w_2 < 0.
	\end{cases}
	\]
	Dividing the second by 2 yields:
	\[
	w_1 + w_2 < 0,
	\]
	which \emph{directly contradicts} \(w_1 + w_2 > 0\). Hence, no choice of \((w_1, w_2)\) can separate the points without a bias.

	
	\paragraph{With Bias (\(b = 3\)):}
	\begin{itemize}
		\item The decision boundary becomes \(y = -x + 3\), shifting the line upward.
		\item The red point \((1, 1)\) satisfies \(-x - y + 3 > 0\) (classified as red).
		\item The blue point \((2, 2)\) satisfies \(-x - y + 3 < 0\) (classified as blue).
		\item The bias term enables correct separation of the two points.
	\end{itemize}
	
	\begin{figure}[H]
		\centering
		\includegraphics[width=0.8\textwidth]{Figures/Chapter_3/bias_importance.jpg}
		\caption{Bias shifts the decision boundary (orange line), enabling correct classification of the two points. Without bias (e.g., green line for the chosen $W$), no line passing through the origin can separate the points.}
		\label{fig:chapter3_bias_example}
	\end{figure}
\end{example}

\textbf{Key Insight:} Without the bias term, the decision boundary is constrained to pass through the origin, making it impossible to correctly separate the two points. Adding a bias term shifts the boundary, enabling proper classification.

This concept generalizes beyond toy 2D problems. In higher-dimensional spaces, the bias term provides the flexibility to shift hyperplanes, enabling the classifier to handle real-world data distributions that are not centered at the origin. Without this flexibility, the model would struggle to adapt to datasets where the mean of the input features is non-zero or misaligned with the origin.

\end{enrichment} 

\subsection{A Toy Example: Grayscale Cat Image}

To build a strong foundation for understanding how linear classifiers work, let us consider a toy example of a grayscale \(2 \times 2\) image of a cat. Each pixel has a value ranging from 0 to 255, representing its grayscale intensity. Although real cat images are much larger, this simplified scenario helps illustrate the key principles with ease.

\begin{itemize}
	\item \textbf{Image Representation:} The \(2 \times 2\) image is flattened into a column vector with 4 entries, denoted as:
	\[
	\mathbf{x} = [x_1, x_2, x_3, x_4]^T.
	\]
	\item \textbf{Weight Matrix \(\mathbf{W}\):} The weight matrix \(\mathbf{W}\) has \(K\) rows (one for each class) and 4 columns (one for each pixel). Each row of \(\mathbf{W}\) corresponds to a specific class and determines the influence of each pixel on the classification score.
	\newpage
	\item \textbf{Matrix Multiplication:} The input vector \(\mathbf{x}\) is multiplied with the weight matrix \(\mathbf{W}\) to produce a vector of scores:
	\[
	\mathbf{s} = \mathbf{W} \mathbf{x},
	\]
	where \(\mathbf{s} = [s_1, s_2, \dots, s_K]^T\) represents the scores for \(K\) classes.
	\item \textbf{Bias Term:} A bias vector \(\mathbf{b}\) of size \(K\) is added to the score vector:$\mathbf{o} = \mathbf{s} + \mathbf{b},$ resulting in the final output vector \(\mathbf{o}\), where each element represents the adjusted score for a class.
\end{itemize}

\begin{figure}[H]
	\centering
	\includegraphics[width=0.8\textwidth]{Figures/Chapter_3/Slide_14.jpg}
	\caption{A toy example of a grayscale \(2 \times 2\) cat image (Slide 14), stretched into a vector and passed through a linear classifier.}
	\label{fig:chapter3_slide14_toy_example}
\end{figure}

\noindent
This simple yet powerful operation demonstrates how linear classifiers map raw data (pixel values) to class scores using a combination of learned weights and bias terms.

\subsection{The Bias Trick}

In linear classifiers, the bias term \(\mathbf{b}\) plays a critical role in adjusting the decision boundary. An alternative way to incorporate the bias is through a technique called the \textbf{bias trick}, which eliminates the explicit bias vector by augmenting the input data and weight matrix. This approach is commonly used when the input data naturally has a vector form, such as in tabular datasets.

\begin{figure}[H]
	\centering
	\includegraphics[width=0.8\textwidth]{Figures/Chapter_3/Slide_16.jpg}
	\caption{The bias trick applied to the toy cat example: augmenting the image vector with a constant 1 and extending the weight matrix to incorporate the bias.}
	\label{fig:chapter3_bias_trick}
\end{figure}

\textbf{How the Trick Works:}
\begin{itemize}
	\item \textbf{Augmented Input Representation:} To absorb the bias term into the weight matrix, we append an additional constant value of \(1\) to the input feature vector. If the original feature vector is \(\mathbf{x} = [x_1, x_2, \dots, x_D]^T\), the augmented representation becomes:
	\[
	\mathbf{x}' = [x_1, x_2, \dots, x_D, 1]^T.
	\]
	\item \textbf{Augmented Weight Matrix:} The weight matrix \(\mathbf{W}\) is updated by adding a new column corresponding to the bias. If \(\mathbf{W}\) initially has dimensions \(K \times D\), the augmented matrix becomes \(K \times (D+1)\), where the last column holds the bias values for each class.
	\item \textbf{Unified Matrix Multiplication:} The score computation becomes:
	\[
	\mathbf{s} = \mathbf{W} \mathbf{x}',
	\]
	effectively absorbing the bias into the augmented weight matrix.
\end{itemize}

\textbf{Example with Cat Image (Slide \ref{fig:chapter3_bias_trick})}

To demonstrate the bias trick in action, consider the toy example of a \(2 \times 2\) grayscale image of a cat introduced earlier (Slide \ref{fig:chapter3_slide14_toy_example}). Initially, the image was flattened into a vector of 4 pixels, \([p_1, p_2, p_3, p_4]^T\). Using the bias trick, we augment this vector by appending a constant value of \(1\), resulting in:
\[
\mathbf{x}' = [p_1, p_2, p_3, p_4, 1]^T.
\]

Simultaneously, the weight matrix \(\mathbf{W}\), originally of shape \(K \times 4\), is augmented to \(K \times 5\) by adding a new column to account for the bias term. The computation of class scores becomes:
\[
\mathbf{s} = \mathbf{W} \mathbf{x}',
\]
where the augmented weight matrix seamlessly integrates the effect of the bias.

\newpage
\textbf{Advantages of the Bias Trick:}
\begin{itemize}
	\item \textbf{Simplified Notation:} The trick reduces the need for separate terms in the computation, allowing the bias and weights to be handled in a unified framework.
	\item \textbf{Ease of Implementation:} In frameworks where data is inherently vectorized (e.g., certain numerical libraries), this method simplifies coding and matrix operations.
	\item \textbf{Theoretical Insights:} This technique emphasizes that the bias term is simply an additional degree of freedom, equivalent to a constant input feature with fixed weight.
\end{itemize}

\textbf{Limitations in Computer Vision:}
In computer vision, the bias trick is less frequently used. For example, in convolutional neural networks (CNNs), this approach does not translate well because the input is often represented as multi-dimensional tensors (e.g., images), and the convolution operation does not naturally accommodate the bias trick. Additionally:
\begin{itemize}
	\item \textbf{Separate Initialization:} Bias and weights are often initialized differently in practice. For instance, weights may be initialized randomly, while biases might start at zero to avoid influencing initial predictions.
	\item \textbf{Flexibility in Training:} Treating bias and weights separately allows more nuanced adjustments during regularization or optimization.
\end{itemize}

\textbf{When to Use the Bias Trick:}
The bias trick is particularly useful for datasets where the input data is naturally represented as a vector (e.g., tabular data or flattened image data). It simplifies the mathematical formulation and is computationally efficient in these scenarios. However, when working with more complex data structures, such as images in their raw tensor form, separating the bias term often provides more flexibility and practical utility.

This technique highlights the elegance and adaptability of linear classifiers, demonstrating how small changes in representation can simplify computations while maintaining mathematical equivalence.

\section{Linear Classifiers: The Algebraic Viewpoint}

The algebraic viewpoint provides an elegant mathematical framework to understand linear classifiers. It emphasizes the role of the weight matrix \(\mathbf{W}\) and the bias vector \(\mathbf{b}\) in transforming input features into scores for each class. This perspective also highlights certain intrinsic properties and limitations of linear classifiers.

\subsection{Scaling Properties and Insights}

Linear classifiers exhibit a key property: their output scores scale linearly with the input. Consider a scaled input \(\mathbf{x}' = c\mathbf{x}\) (where \(c > 0\) is a constant). When passed through a classifier without a bias term, the output becomes:
\[
f(\mathbf{x}', \mathbf{W}) = \mathbf{W}(c\mathbf{x}) = c \cdot f(\mathbf{x}, \mathbf{W}).
\]

This means that scaling the input by a constant \(c\) directly scales the output scores by \(c\). Slide \ref{fig:chapter3_scaling_bias_trick} illustrates this with a practical example. A grayscale image of a cat, when uniformly brightened or darkened (scaling all pixel values by \(c\)), results in scaled class scores. Humans can still easily recognize the cat, but the classifier’s output scores are proportionally reduced. For instance:
\[
f(\mathbf{x}, \mathbf{W}) = [2.0, -1.0, 0.5], \quad f(c\mathbf{x}, \mathbf{W}) = [1.0, -0.5, 0.25] \quad \text{(for \(c = 0.5\))}.
\]

\begin{figure}[H]
	\centering
	\includegraphics[width=0.8\textwidth]{Figures/Chapter_3/Slide_18.jpg}
	\caption{Scaling effect in linear classifiers: uniform scaling of inputs leads to proportional scaling of output scores, as shown in this cat image example.}
	\label{fig:chapter3_scaling_bias_trick}
\end{figure}

This feature of linear classifiers may or may not be desirable, depending on the choice of loss function:
\begin{itemize}
	\item If the loss function focuses on relative scores, such as cross-entropy, the scaling has no effect because the final predictions depend only on the relative differences between scores.
	\item However, in other contexts, absolute score magnitudes might be important, and scaling could introduce issues.
\end{itemize}

\subsection{From Algebra to Visual Interpretability}

While the algebraic viewpoint is powerful for mathematical formulation, it can sometimes obscure the intuition behind the classifier’s behavior. A useful trick to bridge this gap involves reshaping the rows of the weight matrix \(\mathbf{W}\) into image-like blocks.

Each row of \(\mathbf{W}\) corresponds to one class, and reshaping it into the dimensions of the input image allows us to visualize what the classifier "sees" for each class. These visualizations can provide insight into:
\begin{itemize}
	\item What features the classifier considers important for each class.
	\item How the classifier might misinterpret or confuse one class with another.
	\item Biases or artifacts present in the dataset, as reflected in the learned weights.
\end{itemize}

This interpretation naturally leads into the \textbf{Visual Viewpoint}, which we will explore in detail in subsequent sections. By combining algebraic rigor with visual insights, we can better understand the strengths and limitations of linear classifiers.

\section{Linear Classifiers: The Visual Viewpoint}

The visual viewpoint provides an intuitive way to interpret the behavior of linear classifiers by visualizing the rows of the weight matrix \(\mathbf{W}\) reshaped into the input image’s dimensions. This visualization helps us understand what the classifier "learns" during training and highlights its strengths and limitations.

\subsection{Template Matching Perspective}

In a linear classifier, each row of the weight matrix \(\mathbf{W}\) corresponds to a specific output class. By reshaping these rows into the shape of the input image, we can view them as class-specific templates. The score for a given class is computed by taking the inner product (dot product) between the input image and the corresponding template. This process effectively performs template matching, where the templates are learned from data.

\begin{figure}[H]
	\centering
	\includegraphics[width=0.8\textwidth]{Figures/Chapter_3/Slide_22.jpg}
	\caption{Visualizing the rows of the weight matrix \(\mathbf{W}\) as learned templates for each class.}
	\label{fig:chapter3_template_matching}
\end{figure}

Figure \ref{fig:chapter3_template_matching} illustrates how the rows of \(\mathbf{W}\) can be visualized as templates. The inner product measures how well each template "fits" the input image, assigning a score to each class. 

This method can be compared to Nearest Neighbor classification:
\begin{itemize}
	\item Instead of storing thousands of training images, a single learned template per class is used.
	\item The similarity is measured using the (negative) inner product rather than L1 or L2 distance.
\end{itemize}

\newpage

\subsection{Interpreting Templates}

Visualizing the templates reveals the learned features for each class:
\begin{itemize}
	\item \textbf{Plane and Ship Classes:} The templates for these classes are predominantly blue, reflecting the sky and ocean backgrounds common in the training set. A strong inner product with these templates may incorrectly classify other blue-background objects (e.g., a blue shirt) as planes or ships. Conversely, planes or ships on non-blue backgrounds might be misclassified.
	\item \textbf{Horse Class:} The horse class template appears to depict a two-headed horse, as it merges training images of horses facing left and right into a single representation. 
	\item \textbf{Car Class:} The car class template is red, indicating a dataset bias toward red cars. This can lead to incorrect classifications for cars of other colors.
\end{itemize}

These observations highlight limitations of \textbf{background sensitivity}, a \textbf{single template per class}.

\subsection{Python Code Example: Visualizing Learned Templates}

Here’s an example using an SVM classifier (a type of linear classifier) to visualize learned templates:

\begin{mintedbox}{python}
	# Visualize the learned weights for each class
	w = svm.W[:-1, :]  # Strip out the bias term
	w = w.reshape(32, 32, 3, 10)  # Reshape rows into image format
	w_min, w_max = np.min(w), np.max(w)
	
	classes = ['plane', 'car', 'bird', 'cat', 'deer', 'dog', 'frog', 'horse', 'ship', 'truck']
	
	for i in range(10):
		plt.subplot(2, 5, i + 1)
		# Rescale weights to 0-255 range for visualization
		wimg = 255.0 * (w[:, :, :, i].squeeze() - w_min) / (w_max - w_min)
		plt.imshow(wimg.astype('uint8'))
		plt.axis('off')
		plt.title(classes[i])
		plt.show()
\end{mintedbox}

\begin{figure}[H]
	\centering
	\includegraphics[width=0.8\textwidth]{Figures/Chapter_3/class_templates.jpg}
	\caption{The output of the code (building upon NumPy and Matplotlib) visualizes the rows of the weight matrix reshaped into the input image format, enabling inspection of the learned templates.}
	\label{fig:chapter3_visualize_class_templates}
\end{figure}

\subsection{Template Limitations: Multiple Modes}

Linear classifiers are limited by their single-template-per-class constraint:
\begin{itemize}
	\item Categories with distinct modes (e.g., horses facing left vs. right) cannot be disentangled, as the classifier learns only one merged template.
\end{itemize}

\begin{figure}[H]
	\centering
	\includegraphics[width=0.8\textwidth]{Figures/Chapter_3/Slide_23.jpg}
	\caption{The horse class template demonstrates the limitation of learning a single template for a category with multiple modes.}
	\label{fig:chapter3_multiple_modes}
\end{figure}

\subsection{Looking Ahead}

While linear classifiers provide valuable insights, their limitations become apparent in real-world tasks. Neural networks, which will be introduced later, address these shortcomings by developing intermediate neurons in hidden layers. These neurons can specialize in features like "red car" or "blue car" and combine them into more accurate class scores, overcoming the single-template limitation of linear classifiers.

\section{Linear Classifiers: The Geometric Viewpoint}

The geometric viewpoint provides a spatial interpretation of how linear classifiers operate in high-dimensional input spaces. By treating each stretched input image as a point in a high-dimensional space, this perspective helps us understand both the capabilities and limitations of linear classifiers.

\subsection{Images as High-Dimensional Points}

Each input image corresponds to a single point in the feature space. For instance, in CIFAR-10, each \(32 \times 32 \times 3\) image represents a point in a 3072-dimensional space. The entire dataset is thus a labeled set of points, with each label corresponding to a class.

Linear classifiers define the score for each class as a linear function of the input. This corresponds to carving the high-dimensional space into regions using hyperplanes, where each region is assigned to a class.

\begin{figure}[H]
	\centering
	\includegraphics[width=0.8\textwidth]{Figures/Chapter_3/Slide_30.jpg}
	\caption{Left: Dimensionality-reduced visualization of a dataset. Right: Hyperplanes partitioning a higher-dimensional space into regions for classification.}
	\label{fig:chapter3_geometric_hyperplanes}
\end{figure}

In Figure \ref{fig:chapter3_geometric_hyperplanes}, the left side provides a simplified view after dimensionality reduction, while the right shows hyperplanes in the full space. These hyperplanes represent the boundaries where the classifier transitions between classes.

\subsection{Limitations of Linear Classifiers}

The geometric viewpoint highlights scenarios where linear classifiers fail.

\begin{figure}[H]
	\centering
	\includegraphics[width=0.8\textwidth]{Figures/Chapter_3/Slide_31.jpg}
	\caption{Examples of classification problems that linear classifiers cannot solve.}
	\label{fig:chapter3_viewpoint_failures}
\end{figure}

\textbf{Left: Non-Linearly Separable Classes}  
In this example, two classes occupy alternating quadrants. A single hyperplane cannot separate these regions, making the data not linearly separable.

\textbf{Center: Nested Classes}  
Here, one class forms a circular region inside another. The boundary between the two classes is inherently non-linear, so no hyperplane can effectively separate them.

\textbf{Right: Multi-Modal Classes}  
A single class consists of disjoint regions in the space, corresponding to multiple modes (e.g., variations in pose or orientation). Linear classifiers cannot handle such complexities because they only define a single hyperplane per class.

\subsection{Historical Context: The Perceptron and XOR Limitation}

Linear classifiers were among the first machine learning models introduced. The perceptron, developed in the late 1950s, was a milestone in artificial intelligence. However, its inability to handle the XOR function demonstrated the limitations of linear classifiers. 

\begin{figure}[H]
	\centering
	\includegraphics[width=0.8\textwidth]{Figures/Chapter_3/Slide_32.jpg}
	\caption{XOR Function: The perceptron can't separate blue \& green regions with a single line.}
	\label{fig:chapter3_xor_limitations}
\end{figure}

As shown in Figure \ref{fig:chapter3_xor_limitations}, the XOR function has two regions (blue and green) that cannot be separated by a single linear boundary. This limitation highlighted the need for more powerful tools, eventually leading to the development of neural networks. Unlike linear classifiers, neural networks can represent non-linear decision boundaries, generalize well to unseen data, and perform efficient inference.

\subsection{Challenges of High-Dimensional Geometry}

Although this viewpoint provides valuable insights, it has limitations:
\begin{itemize}
	\item \textbf{Human Intuition Fails:} Geometry behaves differently in high-dimensional spaces, often defying our intuition based on 2D/3D experiences.
	\item \textbf{Linear Limitations:} Linear classifiers rely on single hyperplanes, which are inadequate for handling non-linear or complex data distributions.
\end{itemize}

Despite these challenges, the geometric viewpoint lays the foundation for understanding why more advanced models, such as neural networks, are necessary. Neural networks overcome these issues by learning non-linear decision boundaries, a topic we will explore later.

\newpage
\section{Summary: Shortcomings of Linear Classifiers}

Linear classifiers, while foundational, exhibit several limitations that are evident through different viewpoints (Figure \ref{fig:chapter3_viewpoint_failures}).

\subsection{Algebraic Viewpoint}
Linear classifiers rely on the weighted sum of input features. Without non-linear transformations, they:
\begin{itemize}
	\item Cannot model non-linear decision boundaries.
	\item Are limited in their expressiveness when classes are not linearly separable.
\end{itemize}

\subsection{Visual Viewpoint}
Visualizing the rows of the weight matrix as templates reveals:
\begin{itemize}
	\item Templates depend heavily on backgrounds, leading to misclassifications (e.g., ships in non-ocean scenes).
	\item Multiple modes within a class (e.g., cars of different colors or orientations) cannot be represented by a single template.
\end{itemize}

\subsection{Geometric Viewpoint}
Interpreting data as points in high-dimensional space highlights:
\begin{itemize}
	\item Linear classifiers fail when class distributions are not linearly separable (e.g., XOR configuration).
	\item Disjoint or nested regions within a class cannot be handled by a single hyperplane.
\end{itemize}

\subsection{Conclusion: Linear Classifiers Aren't Enough}
These limitations necessitate more advanced models capable of non-linear decision boundaries and hierarchical feature learning, which we explore in subsequent chapters.

\subsection{Choosing the Weights for Linear Classifiers}

To effectively use linear classifiers, we must find a weight matrix \(W\) and bias vector \(\mathbf{b}\) that minimize misclassification. This involves two core tasks:
\begin{itemize}
	\item Defining a \textbf{loss function} to quantify how good a choice of \(W\) is.
	\item Optimizing \(W\) to minimize the loss function.
\end{itemize}

In the rest of this chapter, we focus on the first task—choosing an appropriate loss function—while optimization will be addressed in the next chapter.

\newpage
\section{Loss Functions}

Loss functions are fundamental to machine learning—they provide a scalar measure of how far a model's predictions deviate from the true targets. Learning proceeds by minimizing this loss across a dataset, typically using gradient-based optimization.

Given a dataset with \(N\) examples, the total loss is computed as:
\[
L = \frac{1}{N} \sum_{i=1}^{N} L_i(f(x_i, W), y_i),
\]
where:
\begin{itemize}
	\item \(x_i\) is an input example,
	\item \(y_i\) is the corresponding true label,
	\item \(f(x_i, W)\) is the model's prediction given parameters \(W\),
	\item and \(L_i\) is the loss incurred on a single example.
\end{itemize}

\subsection{Core Requirements for Loss Functions}

Regardless of the task, certain properties are essential for a loss function to be useful in optimization:

\begin{itemize}
	\item \textbf{Differentiability:} The loss should be differentiable with respect to the model parameters to enable the use of gradient-based optimization algorithms such as stochastic gradient descent (SGD).
	
	\item \textbf{Monotonicity:} The loss should increase as the model’s predictions become worse. That is, the loss should provide a signal that correlates with how "wrong" a prediction is.
	
	\item \textbf{Continuity:} Smoothness in the loss landscape helps ensure stable updates during training and prevents erratic gradient jumps.
	
	\item \textbf{Well-defined domain and range:} The loss function should handle valid model outputs and targets gracefully and return real-valued, finite outputs.
\end{itemize}

\subsection{Desirable Properties (Depending on the Task)}

Beyond the core requirements, some properties may be beneficial or even necessary depending on the specific problem, model, or dataset:

\begin{itemize}
	\item \textbf{Convexity (for simpler models):} Convex loss functions are easier to optimize because any local minimum is also a global minimum. While deep networks make the full objective non-convex, convex losses simplify training in linear models.
	
	\item \textbf{Robustness to outliers:} In tasks where noisy or mislabeled data is common, a loss function that does not over-penalize extreme errors (e.g., using absolute error instead of squared error) can improve generalization.
	
	\item \textbf{Probabilistic or geometric interpretation:} Some loss functions correspond to likelihood maximization under a specific model or enforce geometric margins. These interpretations often guide their design and applicability.
	
	\item \textbf{Alignment with evaluation metrics:} Ideally, the loss should correlate with the metric we care about at test time (e.g., accuracy, F1 score, BLEU). While exact alignment is not always feasible, closer alignment often leads to better results.
\end{itemize}

\medskip

With these principles in mind, we now turn to specific loss functions commonly used in classification and regression tasks, beginning with one of the most widely used: the cross-entropy loss.

\subsection{Cross-Entropy Loss}

The \textbf{cross-entropy loss}, often used with the \textbf{softmax function}, provides a probabilistic interpretation of the classifier's raw scores. For a single input \(x_i\) and weight matrix \(W\), the raw scores for each class are given by:
\[
s_j = f(x_i, W)_j,
\]
where \(s_j\) represents the score for class \(j\). These scores are unnormalized, and their magnitude or sign has no direct probabilistic interpretation.

\subsubsection{Softmax Function}
\label{subsec:softmax}

The \textbf{softmax} function transforms raw class scores \(\{s_j\}\) into normalized probabilities \(\{p_j\}\), ensuring that \(\sum_j p_j = 1\) and \(p_j \ge 0\). Concretely:
\[
p_j = \frac{e^{s_j}}{\sum_{k} e^{s_k}},
\]
where:
\begin{itemize}
	\item \(e^{s_j}\) is the exponentiated score for class \(j\).
	\item \(\sum_{k} e^{s_k}\) sums these exponentiated values across all classes, serving as a normalization factor.
\end{itemize}
A large score \(s_j\) results in a disproportionately large exponent \(e^{s_j}\), making \(p_j\) close to 1 while other probabilities remain small.

\paragraph{Advanced Note: Boltzmann Perspective.}
Softmax closely resembles a Boltzmann (Gibbs) distribution: each class’s weight is \(\exp(s_j)\), normalized so that \(\sum_j p_j = 1\). Although any mapping that yields a valid probability distribution could be used, \emph{softmax} is especially attractive because, in tandem with cross-entropy, the derivative of the loss with respect to each logit \(s_j\) collapses to \((p_j - y_j)\). This concise gradient form is both straightforward to implement and numerically stable, simplifying training for classification tasks.


\subsubsection{Loss Computation}

The cross-entropy loss for a single example compares the predicted probability \(p_{y_i}\) of the correct class \(y_i\) with the true label:
\[
L_i = -\log(p_{y_i}),
\]
where \(p_{y_i}\) is the softmax probability for the correct class. This loss penalizes the model heavily if the predicted probability for the correct class is small.

\paragraph{Example: CIFAR-10 Image Classification}

Consider a CIFAR-10 image (e.g., a cat) with three possible classes: cat (\(s_\text{cat} = 3.2\)), car (\(s_\text{car} = 5.1\)), and frog (\(s_\text{frog} = -1.7\)). Using the softmax function:
\begin{enumerate}
	\item Compute \(e^{s_\text{cat}}, e^{s_\text{car}}, e^{s_\text{frog}}\).
	\item Normalize by summing over all exponentiated scores.
	\item Calculate the probability for the cat class, \(p_\text{cat}\), and compute the loss:
	\[
	L_i = -\log(p_\text{cat}).
	\]
\end{enumerate}

\begin{figure}[H]
	\centering
	\includegraphics[width=0.8\textwidth]{Figures/Chapter_3/Slide_50.jpg}
	\caption{Cross-entropy loss computation for a cat image. Softmax normalizes raw scores into probabilities, and the loss is computed by comparing with the ground truth.}
	\label{fig:chapter3_ce_loss_example}
\end{figure}

\paragraph{Properties of Cross-Entropy Loss}
\begin{itemize}
	\item \textbf{Minimum Loss:} The loss is \(0\) when \(p_{y_i} = 1\), meaning the model predicts the correct class with absolute confidence.
	\item \textbf{Maximum Loss:} The loss approaches \(+\infty\) as \(p_{y_i} \to 0\), heavily penalizing incorrect predictions.
	\item \textbf{Initialization Insight:} At random initialization, the raw scores \(s_j\) are small random values. Probabilities become uniform over \(C\) classes:
	\[
	p_j = \frac{1}{C}.
	\]
	The loss becomes:
	\[
	L_i \approx -\log\left(\frac{1}{C}\right).
	\]
	This insight is useful for debugging model implementations.
\end{itemize}

\subsubsection{Why "Cross-Entropy"?}

The name arises from information theory, where cross-entropy measures the difference between two probability distributions: the true distribution (one-hot vector of ground truth labels) and the predicted distribution (softmax probabilities).

The softmax function is named for its differentiable approximation to the max function, making it suitable for gradient-based optimization.

\newpage
\begin{enrichment}[Why Cross-Entropy Uses Logarithms, Not Squared Errors][subsubsection]
	
	\noindent
	CE loss is the standard choice for classification, especially when used with the softmax output layer. But why does it use logarithms, and not something simpler like squared error?
	
	\begin{itemize}
		\item \textbf{(1) Mean Squared Error (MSE)} used as a loss for classification: 
		\[
		\text{Loss} = \sum_j (p_j - \hat{p}_j)^2,
		\]
		where \( p \) is the one-hot target and \( \hat{p} \) is the predicted probability vector.
		
		\item \textbf{(2) Replacing exponentials in softmax with squared values:}
		\[
		\text{Alternative softmax:} \quad \hat{p}_j = \frac{z_j^2}{\sum_k z_k^2},
		\]
		which preserves normalization but alters the probabilistic and geometric behavior.
	\end{itemize}
	
	We now explain why both of these alternatives are inferior to cross-entropy loss with standard softmax.
	
	\medskip
	\textbf{1. Cross-entropy arises naturally from log-likelihood and KL divergence.}
	
	Cross-entropy loss is not an arbitrary design—it is grounded in the principle of \textbf{maximum likelihood estimation (MLE)} for categorical variables. Suppose the true class is \( y \), and the model assigns it predicted probability \( \hat{p}_y \). Then, under a categorical distribution, the log-likelihood is:
	\[
	\log p(y \mid x) = \log \hat{p}_y,
	\]
	and the corresponding loss is the \textbf{negative log-likelihood}:
	\[
	\text{Loss} = -\log \hat{p}_y.
	\]
	This expression generalizes to the full cross-entropy between the true label distribution \( p \) (which is typically one-hot) and the predicted probability vector \( \hat{p} \):
	\[
	H(p, \hat{p}) = -\sum_j p_j \log \hat{p}_j.
	\]
	
	More fundamentally, this quantity appears inside the \textbf{Kullback–Leibler (KL) divergence}, which measures how far a predicted distribution \( \hat{p} \) is from the true distribution \( p \):
	\[
	\text{KL}(p \,\|\, \hat{p}) = \sum_j p_j \log \frac{p_j}{\hat{p}_j} = H(p, \hat{p}) - H(p).
	\]
	
	Since \( H(p) \), the entropy of the true distribution, is fixed (independent of model parameters), minimizing the cross-entropy is equivalent to minimizing KL divergence.
	
	\medskip
	\textbf{Why is this useful?} KL divergence is a principled and well-understood measure of distributional mismatch. By minimizing it, we are not just guessing the correct class—we are learning to approximate the entire target distribution. This ensures:
	\begin{itemize}
		\item \textbf{Probabilistic correctness:} The model assigns high probability to the true class while properly normalizing over alternatives.
		\item \textbf{Meaningful confidence:} The output reflects calibrated uncertainty, not just a one-hot choice.
		\item \textbf{Gradient quality:} The loss provides rich feedback even when the prediction is wrong, making learning faster and more stable.
	\end{itemize}
	
	Thus, cross-entropy's link to likelihood and KL divergence ensures it is not only mathematically justified but also practically effective for probabilistic classification.

	\medskip
	\textbf{2. Squared error is poorly aligned with classification objectives.}
	
	Using mean squared error (MSE) for classification tasks is both conceptually inappropriate and mathematically inefficient. MSE assumes that outputs are continuous and independent, which does not hold in categorical prediction settings.
	
	\begin{itemize}
		
		\item \textbf{Lack of asymmetry:}
		
		The MSE loss penalizes the squared difference between the predicted probability vector \( \hat{p} \) and the one-hot encoded true label \( p \). The loss is defined as:
		\[
		\text{MSE}(p, \hat{p}) = \sum_j (p_j - \hat{p}_j)^2.
		\]
		
		This loss is symmetric: overestimating or underestimating the correct class by the same amount yields the same penalty. For instance, suppose the true class is class 1. Then:
		\[
		(1 - 0.8)^2 = (1 - 1.2)^2 = 0.04.
		\]
		
		In both cases, MSE assigns the same loss—even though the first prediction is underconfident, while the second is overconfident and invalid (e.g., \(\hat{p}_1 = 1.2\) is not even a valid probability). This symmetry fails to reflect the inherently asymmetric nature of classification, where confident wrong predictions are more damaging than slightly uncertain correct ones.
		
		\item \textbf{Weak penalty for confident errors:}
		
		MSE penalizes prediction errors quadratically but lacks the steep, exponential-like penalties needed for classification. Consider predicting a low probability for the true class:
		\[
		\text{Cross-entropy:} \quad -\log(0.01) \approx 4.6
		\]
		\[
		\text{MSE:} \quad (1 - 0.01)^2 = 0.9801.
		\]
		
		Cross-entropy provides a much sharper penalty for this confident error, which encourages the model to avoid placing extremely low probabilities on the correct class. This steepness acts like a strong corrective force during learning.
		
		\item \textbf{Poor gradient behavior:}
		
		While mean squared error (MSE) and cross-entropy (CE) losses may appear similar when applied directly to predicted probabilities, their gradients behave very differently once we consider how they flow through the softmax function during backpropagation.
		
		Let’s assume the model outputs logits \( z_j \), which are passed through softmax:
		\[
		\hat{p}_j = \frac{e^{z_j}}{\sum_k e^{z_k}},
		\]
		to produce class probabilities.
		
		\medskip
		\textbf{Cross-entropy loss:}
		\[
		\mathcal{L}_{\text{CE}} = -\sum_j p_j \log \hat{p}_j.
		\]
		When computing the gradient of CE with respect to logits \( z_j \), we obtain a remarkably simple and well-behaved form:
		\[
		\frac{\partial \mathcal{L}_{\text{CE}}}{\partial z_j} = \hat{p}_j - p_j.
		\]
		This gradient is linear in the prediction error and provides a clean, direct learning signal—especially effective when the model is confidently wrong (e.g., when \( \hat{p}_y \approx 0 \), the loss and gradient are large).
		
		\medskip
		\textbf{Mean squared error:}
		\[
		\mathcal{L}_{\text{MSE}} = \sum_j (\hat{p}_j - p_j)^2.
		\]
		The gradient with respect to the predicted probabilities is:
		\[
		\frac{\partial \mathcal{L}_{\text{MSE}}}{\partial \hat{p}_j} = 2(\hat{p}_j - p_j),
		\]
		which appears similar to CE (up to a constant factor). However, when we backpropagate through the softmax, the full gradient with respect to the logits \( z_j \) becomes:
		\[
		\frac{\partial \mathcal{L}_{\text{MSE}}}{\partial z_j} = 2 \sum_k (\hat{p}_k - p_k)\hat{p}_k(\delta_{jk} - \hat{p}_j),
		\]
		which is more complex and involves \textit{interactions across all classes}. This entangled gradient signal is harder to interpret and can lead to slower, less stable learning.
		
		\medskip
		\textbf{Key distinction:} Cross-entropy provides a \emph{local} gradient per logit that depends only on the predicted probability and the true label. MSE, in contrast, introduces non-local coupling between logits due to the softmax Jacobian. As a result, cross-entropy produces sharper corrections, especially when the model is confidently incorrect, while MSE gradients may become weak or noisy in such regimes.
		
		\medskip
		\textbf{Conclusion:} Although the MSE and CE gradients appear similar at the output layer, their behavior through the softmax transformation differs significantly. Cross-entropy leads to more effective training dynamics, which is one reason it is the preferred loss for classification tasks.
		
		\item \textbf{Mismatch with softmax structure:}
		
		MSE assumes the outputs \( \hat{p}_j \) are independent scalar predictions that can be pushed toward 0 or 1 freely. But softmax outputs are constrained:
		\[
		\sum_j \hat{p}_j = 1, \quad \hat{p}_j \in (0,1).
		\]
		
		This means increasing one class's probability forces other probabilities to decrease. MSE ignores this coupling, treating each component separately. As a result, MSE fails to exploit the inter-class competition inherent to classification, and its gradients don’t reflect how increasing confidence in one class affects others.
		
		In contrast, cross-entropy is designed specifically for probability distributions. It takes into account the full predicted vector and compares it to the one-hot true label in a principled, probabilistic manner.
	\end{itemize}

	
	\medskip
	\textbf{3. Using \( z_j^2 / \sum_k z_k^2 \) instead of softmax breaks probabilistic structure.}
	
	Some have proposed using squared logits in place of exponentials to define a normalized output:
	\[
	\hat{p}_j = \frac{z_j^2}{\sum_k z_k^2}.
	\]
	While this guarantees outputs in \([0,1]\) that sum to 1, it fails in several key ways:
	
	\begin{itemize}
		\item \textbf{No log-likelihood interpretation:} This function does not arise from any known probabilistic model. There’s no equivalent of a negative log-likelihood or KL divergence for guiding learning.
		
		\item \textbf{Limited expressiveness:} The squaring operation is symmetric around 0, so it cannot distinguish between positive and negative evidence. For example, \( z_j = -5 \) and \( z_j = 5 \) produce the same result.
		
		\item \textbf{Unstable or flat gradients:} Near-zero logits yield gradients close to zero, which can stall learning. Exponentials in softmax, by contrast, ensure that even small logit differences yield sharp probability contrasts, especially early in training.
		
		\item \textbf{No exponential separation of scores:} Softmax amplifies differences exponentially, creating a margin-like separation between classes. This is essential for learning sharp decisions in high-dimensional settings; \( z^2 \) lacks this behavior.
	\end{itemize}
	
\end{enrichment}

\subsection{Multiclass SVM Loss}
\label{subsec:chpater3_hinge_loss}

The \textbf{multiclass SVM loss}, also known as the \textbf{hinge loss} (so named because its graphical shape resembles a door hinge), is a straightforward yet powerful loss function. Its goal is to ensure that the score of the correct class is higher than all other class scores by at least a predefined margin \(\Delta\). If this condition is satisfied, the loss is 0; otherwise, the loss increases linearly with the violation.
 
\subsubsection{Loss Definition}

For a single training example \((x_i, y_i)\), the multiclass SVM loss is defined as:
\[
L_i = \sum_{j \neq y_i} \max(0, s_j - s_{y_i} + \Delta),
\]
where:
\begin{itemize}
	\item \(s_j = f(x_i, W)_j\): the score for class \(j\),
	\item \(s_{y_i}\): the score for the correct class,
	\item \(\Delta\): the margin, typically set to \(1\).
\end{itemize}

The total loss across the dataset is the average of individual losses:
\[
L = \frac{1}{N} \sum_{i=1}^{N} L_i.
\]

\subsubsection{Example Computation}

Let us compute the multiclass SVM loss for a small dataset containing three images (a cat, a car, and a frog) from CIFAR-10. The model outputs the following scores for these images across three classes (cat, car, frog):

\[
\begin{aligned}
	\text{Cat Image Scores:} & \quad (s_\text{cat}, s_\text{car}, s_\text{frog}) = (3.2, 5.1, -1.7), \\
	\text{Car Image Scores:} & \quad (s_\text{cat}, s_\text{car}, s_\text{frog}) = (1.3, 4.9, 2.0), \\
	\text{Frog Image Scores:} & \quad (s_\text{cat}, s_\text{car}, s_\text{frog}) = (2.2, 2.5, -3.1).
\end{aligned}
\]

\paragraph{Loss for the Cat Image}
The true class is "cat." The loss is computed as:
\[
L_\text{cat} = \max(0, s_\text{car} - s_\text{cat} + 1) + \max(0, s_\text{frog} - s_\text{cat} + 1),
\]
\[
L_\text{cat} = \max(0, 5.1 - 3.2 + 1) + \max(0, -1.7 - 3.2 + 1),
\]
\[
L_\text{cat} = 2.9.
\]

\begin{figure}[H]
	\centering
	\includegraphics[width=0.8\textwidth]{Figures/Chapter_3/Slide_61.jpg}
	\caption{SVM loss computation for the cat image. Each term corresponds to a margin violation for an incorrect class.}
	\label{fig:chapter3_svm_loss_cat}
\end{figure}

\paragraph{Loss for the Car Image}
The true class is "car." The loss is:
\[
L_\text{car} = \max(0, s_\text{cat} - s_\text{car} + 1) + \max(0, s_\text{frog} - s_\text{car} + 1),
\]
\[
L_\text{car} = \max(0, 1.3 - 4.9 + 1) + \max(0, 2.0 - 4.9 + 1),
\]
\[
L_\text{car} = 0.
\]

\begin{figure}[H]
	\centering
	\includegraphics[width=0.8\textwidth]{Figures/Chapter_3/Slide_62.jpg}
	\caption{SVM loss computation for the car image. As the car score exceeds the rest by more than the margin, the loss is 0.}
	\label{fig:chapter3_svm_loss_car}
\end{figure}

\paragraph{Loss for the Frog Image}
The true class is "frog." The loss is:
\[
L_\text{frog} = \max(0, s_\text{cat} - s_\text{frog} + 1) + \max(0, s_\text{car} - s_\text{frog} + 1),
\]
\[
L_\text{frog} = \max(0, 2.2 - (-3.1) + 1) + \max(0, 2.5 - (-3.1) + 1),
\]
\[
L_\text{frog} = 12.9.
\]

\begin{figure}[H]
	\centering
	\includegraphics[width=0.8\textwidth]{Figures/Chapter_3/Slide_63.jpg}
	\caption{SVM loss computation for the frog image. With the correct class score being the lowest, the loss is the largest.}
	\label{fig:chapter3_svm_loss_frog}
\end{figure}

\paragraph{Total Loss}
The total loss across the dataset is the average of individual losses: $L = \frac{1}{3} (L_\text{cat} + L_\text{car} + L_\text{frog}) = \frac{1}{3} (2.9 + 0 + 12.9) = 5.27.$

\begin{figure}[H]
	\centering
	\includegraphics[width=0.8\textwidth]{Figures/Chapter_3/Slide_64.jpg}
	\caption{Total loss computed as the average of losses over the three images.}
	\label{fig:chapter3_svm_total_loss}
\end{figure}

\subsubsection{Key Questions and Insights}
\begin{itemize}
	\item \textbf{What happens if the loss sums over all classes, including the correct class?} In this case, all scores would inflate uniformly, adding an extra constant (approx. the predetermined margin) to the loss. This does not change the model’s preferences over the weight matrix \(W\).
	\item \textbf{What if we use a mean instead of a sum for the loss?} The loss values are scaled by a factor of \((1 / C-1)\), where \(C\) is the number of classes. The model’s behavior remains unaffected.
	\item \textbf{What if we square the loss terms?} Squaring would alter the loss function’s sensitivity to large deviations, changing the behavior and preferences over \(W\).
	\item \textbf{Is the weight matrix \(W\) unique when the loss is zero?} No, scaling \(W\) (e.g., multiplying it by 2) maintains zero loss because the margin condition is still satisfied. \textbf{Regularization}, which we'll later discuss thoroughly, helps select a preferred \(W\).
\end{itemize}


\subsection{Comparison of Cross-Entropy and Multiclass SVM Losses}

Both losses aim to guide the model toward correct predictions, but their behavior differs significantly:
\begin{itemize}
	\item \textbf{Score Sensitivity:} The SVM loss becomes invariant once the margin condition is satisfied, while the cross-entropy loss continues to decrease as the correct class score increases.
	\item \textbf{Probabilistic Interpretation:} The cross-entropy loss provides a natural probabilistic interpretation of predictions, whereas the SVM loss focuses on maintaining a margin.
	\item \textbf{Scaling Effects:} Scaling scores affects the cross-entropy loss but not the SVM loss, highlighting the need for regularization in SVM-based models.
\end{itemize}

\begin{figure}[H]
	\centering
	\includegraphics[width=0.8\textwidth]{Figures/Chapter_3/Slide_76.jpg}
	\caption{Impact of scaling on SVM and cross-entropy loss. The CE loss decreases, while the SVM loss remains unchanged.}
	\label{fig:chapter3_loss_comparison_scaling}
\end{figure}

\subsubsection{Debugging with Initial Loss Values}

An effective way to verify whether a model is configured correctly is to examine the loss value \emph{at the very start of training}, before any updates have been applied. For both cross-entropy and margin-based losses (e.g., SVM), there are mathematically predictable loss values when the model begins with random, unbiased weights.

\medskip
\textbf{Cross-Entropy Loss: Expected Initial Value.}

Suppose a classifier is initialized such that it produces uniform predictions over all \(C\) classes (as is often the case with random initialization and symmetric weight distributions). That is, the model assigns each class probability:
\[
\hat{p}_j = \frac{1}{C}.
\]
The cross-entropy loss for a one-hot label \(p_j = \delta_{jy}\) becomes:
\[
\mathcal{L}_{\text{CE}} = -\sum_j p_j \log \hat{p}_j = -\log \hat{p}_y = -\log \frac{1}{C} = \log C.
\]
\textbf{So at initialization, we expect the cross-entropy loss to be approximately \( \log C \)}. For example:
\[
C = 10 \quad \Rightarrow \quad \mathcal{L}_{\text{CE}} \approx \log 10 \approx 2.3.
\]

\medskip
\textbf{SVM (Hinge) Loss: Expected Initial Value.}

For the multiclass SVM or hinge loss (often used in margin-based classifiers), the typical loss formulation is:
\[
\mathcal{L}_{\text{SVM}} = \sum_{j \neq y} \max(0, f_j - f_y + 1),
\]
where \(f_j\) is the score (logit) for class \(j\), and \(y\) is the true class.

If the scores \(f_j\) are initialized to be equal or nearly equal (as in uniform random initialization), then:
\[
f_j \approx f_y \quad \Rightarrow \quad f_j - f_y + 1 \approx 1 \text{ for all } j \neq y.
\]
This means the max terms are all active, and the total loss becomes:
\[
\mathcal{L}_{\text{SVM}} \approx \sum_{j \neq y} 1 = C - 1.
\]
\textbf{So at initialization, we expect the hinge loss to be approximately \(C - 1\)}. For example:
\[
C = 10 \quad \Rightarrow \quad \mathcal{L}_{\text{SVM}} \approx 9.
\]

\medskip
\textbf{How This Helps Debugging?}

Inspecting the initial loss is a fast and effective sanity check. It helps confirm that the model, label encoding, output activations, and loss function are correctly configured—before any training begins. While this section focuses on cross-entropy and hinge losses, the principle extends to many loss functions in both classification and regression.

\begin{itemize}
	\item \textbf{If the loss is too low at initialization:} This may signal data leakage (e.g., label information leaking into the inputs), incorrect use of pretrained weights, or even a flaw in the loss implementation. For example, a cross-entropy loss close to zero implies that the model is already assigning very high probability to the correct class—unlikely if the weights are truly untrained.
	
	\item \textbf{If the loss is too high:} This might indicate degenerate model outputs (e.g., extremely large or small logits), incorrect label encoding (such as using class indices instead of one-hot vectors), or numerical instability. In CE, large logit magnitudes with wrong signs can spike the loss well above \( \log C \).
	
	\item \textbf{If the loss deviates significantly from the expected baseline (e.g., \( \log C \) for cross-entropy or \( C - 1 \) for SVM):} This may reflect label mismatches, class imbalance, or improperly scaled outputs (e.g., skipping softmax or applying wrong activation functions).
\end{itemize}

\textbf{In short:} Mismatch with expected baseline (e.g., \( \log C \) for CE or \( C - 1 \) for SVM) suggests issues like misaligned labels, class imbalance, or broken forward pass logic. Checking the initial loss should be a routine step when setting up models. It helps catch configuration bugs early—often before training even begins.

\subsubsection{Conclusion: SVM, Cross Entropy, and the Evolving Landscape of Loss Functions}

\begin{itemize}
	\item \textbf{SVM Loss:} Suitable for margin-based classification but requires regularization to handle multiple solutions with zero loss.
	\item \textbf{Cross-Entropy Loss:} Ideal for probabilistic interpretation and gradient-based optimization, often preferred in deep learning models.
\end{itemize}

Both losses have unique advantages, and the choice depends on the application and desired behavior during training.







\chapterimage{head2.png} % Chapter heading image

% Chapter-specific content starts here
\chapter{Lecture 4: Regularization \& Optimization}

%----------------------------------------------------------------------------------------
%	CHAPTER 4 - Lecture 4: Regularization & Optimization
%----------------------------------------------------------------------------------------
\section{Introduction to Regularization}
In the previous chapter, we explored loss functions as tools to evaluate the performance of machine learning models. However, machine learning is not just about minimizing the loss on the training data. In practice, this narrow focus can be counterproductive, leading to a phenomenon known as \textbf{overfitting}. 

\begin{figure}[H]
	\centering
	\includegraphics[width=0.8\textwidth]{Figures/Chapter_4/overfitting_underfitting.jpg}
	\caption{Illustration of underfitting, good fitting, and overfitting in classification and regression tasks \cite{yenigun_overfitting}. Good regularization aims to strike the balance between underfitting and overfitting.}
	\label{fig:chapter4_overfitting_underfitting}
\end{figure}

As shown in Figure \ref{fig:chapter4_overfitting_underfitting}, overfitting occurs when a model fits the training data too perfectly, capturing even noise and idiosyncrasies that do not generalize to unseen data. While this may result in excellent performance on the training set, it undermines the model’s ability to perform well on test data or real-world scenarios. Conversely, underfitting happens when the model is too simplistic to capture the underlying structure of the data, resulting in poor performance on both the training and test sets. Good regularization techniques aim to achieve a balance, ensuring the model fits the data "just right."

\textbf{Why Regularization?}  
The primary purpose of regularization is to improve the model’s \textbf{generalization}—its ability to perform well on unseen data—by discouraging it from overfitting to the training set. However, regularization serves other key roles:
\begin{itemize}
	\item \textbf{Improving Optimization:} Regularization can add curvature to the loss surface, making optimization easier and more stable.
	\item \textbf{Expressing Model Preferences:} Beyond simply minimizing training error, regularization allows us to encode preferences for simpler or more interpretable models.
\end{itemize}

In this chapter, we will dive into various regularization techniques, explore their mathematical foundations, and discuss how they help achieve better generalization. We will also revisit optimization, introducing practical methods to efficiently minimize loss functions in the presence of regularization.

\subsection{How Regularization is Used?}

As discussed in the introduction, in optimization tasks, the goal is to minimize a loss function \( L_\text{loss}(W) \), which measures the model's performance on training data. However, focusing solely on minimizing \( L_\text{loss}(W) \) can lead to overfitting, where the model memorizes the training data instead of learning patterns that generalize to unseen data.

Regularization addresses this by adding a penalty term \( R(W) \) to the optimization objective:
\[
\text{Objective} = \min_W \left[ L_\text{loss}(W) + \lambda R(W) \right],
\]
where:
\begin{itemize}
	\item \( \lambda \) is the regularization strength, a hyperparameter controlling the penalty's weight.
	\item \( R(W) \) is the regularization term, typically a function of the model weights \( W \), independent of the training data.
\end{itemize}

\subsection{Regularization: Simpler Models Are Preferred}

Regularization promotes simpler models by penalizing large weights, irrespective of their sign. Both L1 and L2 regularization measure the magnitude of weights (\(|W|\) or \(W^2\)), ensuring non-negative penalties. But why do weights increase when a model adjusts to changes in the data?

When the input data contains small perturbations or noise, a model without regularization adjusts its parameters to minimize the training loss \(L_\text{loss}(W)\). Addressing these changes typically involves increasing specific weights to emphasize features that explain the perturbation. This increase amplifies the model's sensitivity to noise, leading to larger weights. Regularization imposes a penalty for large weights, forcing the model to balance reducing \(L_\text{loss}(W)\) against increasing the regularization term \(R(W)\). This trade-off discourages addressing changes that only marginally benefit the loss, resulting in smaller weights and simpler models.
\newpage
\section{Types of Regularization: L1 and L2}

\subsection{L1 Regularization (Lasso)}

\begin{itemize}
	\item \textbf{Definition:} Adds a penalty proportional to the sum of the absolute values of the weights:
	\[
	R(W) = \sum_{i} |w_i|.
	\]
	\item \textbf{Effects:}
	\begin{itemize}
		\item Promotes sparsity by driving many weights to exact zeros, effectively performing feature selection.
		\item Encourages interpretable models by retaining only a subset of the most relevant features.
	\end{itemize}
	\item \textbf{Why L1 Produces Sparse Weights:} The sharp gradient of the absolute value penalty around zero encourages optimization to push small weights to exactly zero.
	\item \textbf{Pros:}
	\begin{itemize}
		\item Suitable for feature selection in high-dimensional datasets.
		\item Produces interpretable models with fewer active features.
	\end{itemize}
	\item \textbf{Cons:}
	\begin{itemize}
		\item Struggles with correlated features, arbitrarily selecting one over others.
		\item May exclude relevant features if sparsity is overly enforced.
	\end{itemize}
\end{itemize}

\subsection{L2 Regularization (Ridge)}

\begin{itemize}
	\item \textbf{Definition:} Adds a penalty proportional to the sum of the squared weights:
	\[
	R(W) = \sum_{i} w_i^2.
	\]
	\item \textbf{Effects:}
	\begin{itemize}
		\item Reduces all weights uniformly, discouraging large weights without enforcing sparsity.
		\item Promotes balanced use of all features, making the model less sensitive to individual feature noise.
	\end{itemize}
	\item \textbf{Why L2 Is Common in Practice:}
	\begin{itemize}
		\item Handles correlated features better by distributing weights across them.
		\item Computationally efficient with gradient-based methods due to its smooth gradients.
	\end{itemize}
	\item \textbf{Pros:}
	\begin{itemize}
		\item Effective for datasets with correlated features.
		\item Produces robust models by retaining all features.
	\end{itemize}
	\item \textbf{Cons:}
	\begin{itemize}
		\item Does not perform feature selection, retaining all features in the model.
	\end{itemize}
\end{itemize}

\subsection{Choosing Between L1 and L2 Regularization}

The choice depends on the problem:
\begin{itemize}
	\item Use \textbf{L1 Regularization} when:
	\begin{itemize}
		\item Feature selection is essential.
		\item A sparse model is required for interpretability.
	\end{itemize}
	\item Use \textbf{L2 Regularization} when:
	\begin{itemize}
		\item Features are correlated, and balance is important.
		\item Smooth optimization is desired.
	\end{itemize}
\end{itemize}
	
	\begin{enrichment}[Can We Combine L1 and L2 Regularization?][subsection]
		
	 Yes! Elastic Net is a regularization technique that combines both L1 (Lasso) and L2 (Ridge) regularization penalties. Its objective function includes a linear combination of the L1 and L2 penalties, defined as:
		
		\[
		L(W) = L_\text{loss} + \lambda_1 \sum_{i} |w_i| + \lambda_2 \sum_{i} w_i^2,
		\]
		
		where \(\lambda_1\) controls the L1 penalty, and \(\lambda_2\) controls the L2 penalty. This combination allows Elastic Net to enjoy the benefits of both regularization methods:
		\begin{itemize}
			\item The L1 penalty encourages sparsity, making Elastic Net useful for feature selection by reducing irrelevant feature weights to zero.
			\item The L2 penalty helps distribute weights among correlated features, overcoming L1's tendency to select only one feature from a group of highly correlated features.
		\end{itemize}
		
		\paragraph{When to Use Elastic Net?} Elastic Net is especially beneficial in situations where:
		\begin{itemize}
			\item The dataset contains many features, some of which are irrelevant (addressed by L1).
			\item There are groups of highly correlated features that all contribute to the prediction (handled by L2).
		\end{itemize}
		
		\paragraph{Summary:} Elastic Net strikes a balance between L1 and L2 regularization, offering a flexible approach for scenarios where sparsity and balanced weight distribution are both important.
	\end{enrichment} 
	
\subsection{Expressing Preferences Through Regularization}

Regularization helps express preferences beyond minimizing the training loss:
\begin{itemize}
	\item \textbf{L2 Regularization:} Prefers weight distributions that spread importance across features. For example:
	\[
	x = [1, 1, 1, 1], \quad w_1 = [1, 0, 0, 0], \quad w_2 = [0.25, 0.25, 0.25, 0.25].
	\]
	Both yield the same inner product (\(w_1^T x = w_2^T x = 1\)), but L2 regularization favors \(w_2\) because it minimizes \( \sum w_i^2 \), distributing importance across all features.
	\item \textbf{L1 Regularization:} Favors sparse solutions, focusing on a subset of features. In the above example, \(w_1\) would be preferred by L1 regularization.
\end{itemize}

\section{Impact of Feature Scaling on Regularization}

Regularization penalties depend on the magnitude of weights, which is influenced by the scale of the input features. Without proper scaling, features with larger values dominate the penalty term, skewing the regularization effect. This is because:

\begin{itemize}
	\item Features with larger scales (e.g., kilometers) result in smaller coefficients, contributing less to the penalty.
	\item Features with smaller scales (e.g., millimeters) result in larger coefficients, contributing more to the penalty.
\end{itemize}

For instance, if a feature \(x_j\) is multiplied by a constant \(c\), its corresponding weight \(w_j\) is divided by \(c\) to maintain the same effect on the model's predictions. This imbalance can unfairly penalize some features over others, especially in Ridge (L2) regularization, which imposes a squared penalty.

\subsection{Practical Implication} 
To ensure fair regularization, input features should be normalized (centered to have mean 0 and scaled to have variance 1). Normalization ensures that all features contribute equally to the penalty, allowing the model to prioritize based on relevance rather than scale.

\subsection{Example: Rescaling and Lasso Regression.} 
Suppose Lasso regression is applied to a dataset with 100 features. If one feature (\(F_1\)) is rescaled by multiplying it by 10, its corresponding coefficient decreases, reducing the absolute penalty. As a result, \(F_1\) is more likely to be retained in the model.

\section{Regularization as a Catalyst for Better Optimization}

Beyond preventing overfitting, regularization enhances optimization by shaping the loss surface. This added structure simplifies and stabilizes the optimization process in key ways:

\subsection{Regularization as Part of Optimization}

Regularization is seamlessly integrated into the optimization process as a penalty term in the loss function:
\[
L(W) = L_\text{loss}(W) + \lambda R(W).
\]
Here, the regularization term \(R(W)\) acts as a constraint, influencing the optimization goal. This dual role bridges the gap between regularization and optimization:
\begin{itemize}
	\item \textbf{Constraint and Balance:} Regularization balances minimizing training loss with enforcing model simplicity.
	\item \textbf{Guiding Optimization:} By penalizing specific weight configurations, regularization steers optimization toward solutions that generalize better to unseen data.
\end{itemize}

The interplay between regularization and optimization ensures that models are not only accurate but also robust and efficient.

\subsection{Augmenting the Loss Surface with Curvature}
Regularization, particularly \textbf{L2 regularization}, introduces a quadratic penalty term to the loss:
\[
L(W) = L_\text{loss}(W) + \lambda \sum_{i} w_i^2.
\]
This penalty increases curvature, especially in regions with large weight magnitudes, resulting in:
\begin{itemize}
	\item \textbf{Smoother Landscapes:} The loss surface becomes more convex, reducing flat regions and saddle points.
	\item \textbf{Stable Gradients:} Gradient-based methods like gradient descent converge more reliably with less oscillation or vanishing gradients.
\end{itemize}

\subsection{Mitigating Instability in High Dimensions}
By penalizing large weights, regularization limits excessive updates during optimization. This reduces instability, particularly in high-dimensional spaces, where large weights can lead to erratic model behavior.

\subsection{Improving Conditioning for Faster Convergence}
High-dimensional loss surfaces often exhibit anisotropy, where gradients vary sharply across dimensions. Regularization balances the curvature across directions, improving the condition number of the problem and facilitating efficient convergence.

In essence, regularization smoothens and stabilizes the optimization landscape, making it easier for algorithms to find better solutions, particularly in complex models like deep neural networks.

In the next sections, we explore optimization techniques that leverage this synergy to train effective machine learning models.

\section{Optimization: Traversing the Loss Landscape}

The optimization process can be thought of as finding the value of the weight matrix \( W^* \) that minimizes a given loss function \( L(W) \). Mathematically, this is formulated as:
\[
W^* = \arg\min_W L(W).
\]

\begin{figure}[H]
	\centering
	\includegraphics[width=0.8\textwidth]{Figures/Chapter_4/Slide_21.jpg}
	\caption{The loss landscape. Each point corresponds to a weight matrix, and the height represents its corresponding loss value.}
	\label{fig:chapter4_landscape}
\end{figure}

\newpage

\subsection{The Loss Landscape Intuition}
Imagine the optimization process as traversing a high-dimensional landscape, where:
\begin{itemize}
	\item Each point on the ground represents a potential weight matrix \( W \).
	\item The height of the landscape at any point corresponds to the value of the loss function \( L(W) \) for that weight matrix.
\end{itemize}

\begin{figure}[H]
	\centering
	\includegraphics[width=0.8\textwidth]{Figures/Chapter_4/Slide_22.jpg}
	\caption{Traversing the loss landscape toward the minimum. The person starts at a random point and follows a path downwards.}
	\label{fig:chapter4_traversal}
\end{figure}

In optimization, we aim to find the lowest point (the global minimum). However, the "traveler" does not know where the bottom is, and the problem is further complicated by the sheer size and complexity of the landscape. Writing down an explicit formula for the minimum is often impractical for most machine learning problems.

\begin{enrichment}[Why Explicit Analytical Solutions Are Often Impractical][subsection]
	\label{enrichment:why_analytical_impractical}
	While it may seem desirable to compute the minimum of the loss function \( L(W) \) directly by writing down an explicit formula, this approach is rarely practical in machine learning for several reasons:
	
	\begin{enrichment}[High Dimensionality][subsubsection]
		Modern machine learning models operate in extremely high-dimensional parameter spaces, where the weight matrix \( W \) can contain millions or billions of parameters. Computing a closed-form solution in such spaces requires solving large systems of equations, making the computational and memory demands intractable. This limitation persists even in simpler cases like linear regression when the dataset size is massive.
	\end{enrichment}
	
	\begin{enrichment}[Non-Convexity of the Loss Landscape][subsubsection]
		The loss landscapes of complex models, such as neural networks, are highly non-convex, featuring multiple local minima, saddle points, and flat regions. Analytical solutions rely on convexity assumptions that do not hold in these scenarios, making it impossible to derive a closed-form solution that guarantees a global minimum.
	\end{enrichment}
	
	\begin{enrichment}[Complexity of Regularization Terms][subsubsection]
		Regularization terms, such as \( \lambda \sum w_i^2 \) (L2 regularization) or \( \lambda \sum |w_i| \) (L1 regularization), introduce additional constraints to the optimization problem. These terms make the loss function non-quadratic or non-differentiable in certain regions, further complicating or eliminating the feasibility of finding explicit solutions.
	\end{enrichment}
	
	\begin{enrichment}[Lack of Generalizability and Flexibility][subsubsection]
		Finding an analytical solution is tailored to a specific loss function and model. If the model structure or loss function changes (e.g., switching from mean squared error to cross-entropy), a new solution must be derived from scratch, wasting time for the algorithmist. 
	\end{enrichment}
	
	\begin{enrichment}[Memory and Computational Cost][subsubsection]
		Closed-form solutions often require inverting large matrices, which is memory-intensive and computationally expensive. For instance, in linear regression, the closed-form solution involves inverting an \( n \times n \) matrix, where \( n \) is the number of features. For high-dimensional data, this operation quickly becomes impractical in terms of both time and memory requirements.
	\end{enrichment}
\end{enrichment}

As we've shown in the enrichment section, finding an explicit analytical solution to the optimization problem—determining the weight matrix \( W^* \) that minimizes the loss function \( L(W) \)—is often impractical due to the high dimensionality and complexity of modern machine learning models. While such a solution would be ideal, it is computationally infeasible or outright impossible in most real-world scenarios.

To overcome this, we begin by exploring simpler, more naive approaches before gradually building towards smarter and more practical solutions for this optimization problem. This progression will allow us to develop an intuitive understanding of the problem while introducing increasingly effective methods to address it.

\subsection{Optimization Idea \#1: Random Search}
One naive strategy for optimization is \textbf{random search}. This involves generating many random weight matrices, evaluating the loss for each, and keeping track of the best solution encountered. Although this method can improve the model's performance given sufficient time (compared to random initialization), it is highly inefficient due to the vastness of the parameter space. 

\begin{figure}[H]
	\centering
	\includegraphics[width=0.8\textwidth]{Figures/Chapter_4/Slide_23.jpg}
	\caption{Random search: A naive optimization approach.}
	\label{fig:chapter4_random_search}
\end{figure}

For example, as shown in Figure \ref{fig:chapter4_random_search}, random search achieves an accuracy of only \( 15.5\%\) on CIFAR-10, far from the \( 95\%\) state-of-the-art performance. The impracticality of densely sampling the parameter space motivates us to explore more intelligent strategies.  

\subsection{Optimization Idea \#2: Following the Slope}
A more practical approach is to \textbf{follow the slope} of the loss landscape. Imagine our traveler cannot see the bottom of the valley but can feel the ground beneath his feet. By sensing the slope at his current location, he can identify the steepest downward direction and take a step in that direction.

\begin{figure}[H]
	\centering
	\includegraphics[width=0.8\textwidth]{Figures/Chapter_4/Slide_26.jpg}
	\caption{Following the slope to descend the landscape.}
	\label{fig:chapter4_following_slope}
\end{figure}

 This strategy leverages local information about how the loss changes in the immediate vicinity of the current point. By iteratively stepping in the direction of steepest descent, the traveler progressively moves closer to the minimum. Despite relying solely on local information, this method is remarkably effective and forms the foundation of many optimization techniques used in machine learning.
 
 In the following sections, we will establish the mathematical foundations for a simple yet effective method that builds upon the idea of following the slope of the loss landscape. By leveraging local information at each step in an iterative process, we aim to develop a robust approach known as \textbf{gradient descent}. This method will serve as a cornerstone for optimization in machine learning, guiding us toward minimizing the loss function efficiently.

\subsection{Gradients: The Mathematical Basis}

The method of steepest descent relies on the concept of \textbf{gradients}, a fundamental mathematical tool for analyzing changes in functions. Recall the following:
\begin{itemize}
	\item For a scalar function \( f(x) \), the derivative \( f'(x) \) tells us how \( f(x) \) changes with a small change in \( x \). It is the slope of \( f(x) \) at any given point.
	\item In higher dimensions, the \textbf{gradient} \( \nabla f(x) \) generalizes this concept. It is a vector of partial derivatives:
	\[
	\nabla f(x) = \begin{bmatrix}
		\frac{\partial f}{\partial x_1} \\
		\frac{\partial f}{\partial x_2} \\
		\vdots \\
		\frac{\partial f}{\partial x_n}
	\end{bmatrix}.
	\]
	This vector points in the direction of the \textbf{steepest ascent}, i.e., where the function increases the fastest, and its magnitude represents the rate of this increase.
\end{itemize}

To minimize a function, we step in the opposite direction of the gradient, \( -\nabla f(x) \). This ensures the most rapid decrease in the function's value.

\subsubsection{Why Does the Gradient Point to the Steepest Ascent?}

The gradient \( \nabla L(w) \) is the direction of the steepest ascent in the loss landscape. This can be understood as follows:

\begin{itemize}
	\item The gradient \( \nabla L(w) \) is defined as the vector of partial derivatives of the loss \( L(w) \) with respect to each parameter in \( w \). It indicates how \( L(w) \) changes in response to small changes in \( w \).
	\item For any small step \( \mathbf{u} \), the change in loss can be approximated using the Taylor expansion:
	\[
	L(w + \eta \mathbf{u}) - L(w) \approx \eta (\nabla L(w) \cdot \mathbf{u}),
	\]
	where \( \eta \) is the step size and \( \nabla L(w) \cdot \mathbf{u} \) is the dot product between the gradient and the step direction.
	\item The dot product is mathematically defined as:
	\[
	\nabla L(w) \cdot \mathbf{u} = \|\nabla L(w)\| \|\mathbf{u}\| \cos(\beta),
	\]
	where \( \beta \) is the angle between \( \nabla L(w) \) and \( \mathbf{u} \).
	\item The dot product \( \nabla L(w) \cdot \mathbf{u} \) is maximized when \( \cos(\beta) = 1 \), which occurs when \( \beta = 0^\circ \) (i.e., \( \mathbf{u} \) is aligned with \( \nabla L(w) \)). This means the rate of increase in \( L(w) \) is greatest in the direction of \( \nabla L(w) \).
\end{itemize}

Thus, the gradient naturally points to the steepest ascent, where the loss increases most rapidly.
\newpage
\subsubsection{Why Does the Negative Gradient Indicate the Steepest Descent?}

The steepest descent occurs in the direction opposite to the gradient, \( -\nabla L(w) \). Here's why:

\begin{itemize}
	\item As before, the change in loss for a small step \( \mathbf{u} \) can be approximated using the Taylor expansion:
	\[
	L(w + \eta \mathbf{u}) - L(w) \approx \eta (\nabla L(w) \cdot \mathbf{u}),
	\]
	where \( \eta \) is the step size.
	\item To minimize \( L(w) \), we require:
	\[
	\nabla L(w) \cdot \mathbf{u} < 0.
	\]
	This ensures that the new loss is smaller than the old loss.
	\item The dot product \( \nabla L(w) \cdot \mathbf{u} \) depends on the angle \( \beta \) between \( \nabla L(w) \) and \( \mathbf{u} \):
	\[
	\nabla L(w) \cdot \mathbf{u} = \|\nabla L(w)\| \|\mathbf{u}\| \cos(\beta).
	\]
	To make \( \nabla L(w) \cdot \mathbf{u} \) as negative as possible, \( \cos(\beta) \) must equal \( -1 \), which occurs when \( \beta = 180^\circ \) (i.e., \( \mathbf{u} \) points exactly opposite to \( \nabla L(w) \)).
	\item Choosing \( \mathbf{u} = -\nabla L(w) \) ensures:
	\[
	\nabla L(w) \cdot \mathbf{u} = -\|\nabla L(w)\| \|\mathbf{u}\|,
	\]
	which achieves the steepest decrease in \( L(w) \).
\end{itemize}

\begin{figure}[H]
	\centering
	\includegraphics[width=0.8\textwidth]{Figures/Chapter_4/Slide_28.jpg}
	\caption{The gradient \( \nabla L(w) \) points to the steepest ascent, while \( -\nabla L(w) \) leads to the steepest descent.}
	\label{fig:chapter4_gradient_steepest_directions}
\end{figure}

This property of gradients makes them indispensable for optimization. By iteratively stepping in the direction of \( -\nabla f(x) \), we can traverse high-dimensional loss landscapes efficiently and move closer to a minimum.

In the following sections, we will explore how to efficiently compute and implement gradient-based optimization methods. These techniques form the foundation of training modern machine learning models, enabling us to navigate the vast parameter spaces effectively.

\section{From Gradient Computation to Gradient Descent}

Training machine learning models involves minimizing a loss function \( L(W) \) by finding the optimal weight matrix \( W^* \). Gradient computation plays a crucial role in this process, providing the direction to adjust \( W \) to reduce the loss. This section explores two approaches to compute gradients, their limitations, and the role of \textbf{gradient descent} in optimization.

\subsection{Gradient Computation Methods}

\subsubsection{Numerical Gradient: Approximating Gradients via Finite Differences}

The \textbf{numerical gradient} approximates the gradient by perturbing each element of the weight matrix \( W \) and observing the effect on the loss. For a given element \( w_{ij} \) in \( W \), the numerical gradient is computed using the finite difference formula:

\[
\frac{\partial L}{\partial w_{ij}} \approx \frac{L(W + \Delta_{ij}) - L(W)}{ \Delta_{ij}},
\]

where \( \Delta_{ij} \) perturbs only \( w_{ij} \) by a small value \(  \Delta_{ij} \) (e.g., \(  \Delta_{ij} = 0.00001 \)) while leaving other elements unchanged.

\paragraph{Process:}
\begin{itemize}
	\item Start with an initialized weight matrix \( W \).
	\item For some element \( w_{ij} \), compute the perturbed loss \( L(W + \Delta_{ij}) \).
	\item Use the finite difference formula to calculate \( \frac{\partial L}{\partial w_{ij}} \).
	\item Repeat for all elements of \( W \) to approximate \( \nabla L(W) \).
\end{itemize}

\begin{figure}[H]
	\centering
	\includegraphics[width=0.8\textwidth]{Figures/Chapter_4/Slide_29.jpg}
	\caption{Numerical Gradient: Computing the slope of \( L(W) \) with respect to a perturbed element of \( W \).}
	\label{fig:chapter4_numeric_gradient}
\end{figure}

\newpage 
\paragraph{Advantages:}
\begin{itemize}
	\item Easy to implement.
	\item Useful as a \textbf{debugging tool}, verifying the correctness of analytically computed gradients. For instance, PyTorch provides a built-in function (\texttt{torch.autograd.gradcheck}) to compare numerical and analytical gradients.
\end{itemize}

\paragraph{Disadvantages:}
\begin{itemize}
	\item \textbf{Computational cost:} Requires \( O(\text{\#dimensions}) \) evaluations of \( L(W) \), which becomes infeasible for large models.
	\item \textbf{Inaccuracy:} Due to the finite value of \( \Delta \), the numerical gradient is an approximation, and the perturbation \( \Delta \) cannot be infinitely small as required by the gradient's limit definition.
\end{itemize}

\subsubsection{Analytical Gradient: Exact Gradients via Calculus}

The \textbf{analytical gradient} computes \( \nabla L(W) \) using calculus, deriving an exact formula for the gradient based on the mathematical properties of the loss function. Unlike the numerical approach, this method is efficient and precise.

\begin{figure}[H]
	\centering
	\includegraphics[width=0.8\textwidth]{Figures/Chapter_4/Slide_38.jpg}
	\caption{Analytical Gradient: Exact computation of gradients via calculus.}
	\label{fig:chapter4_analytical_gradient}
\end{figure}

\paragraph{Advantages:}
\begin{itemize}
	\item \textbf{Exact results:} Provides precise gradient values, free from numerical approximation errors.
	\item \textbf{Computational efficiency:} Scales well to high-dimensional weight matrices.
\end{itemize}

\paragraph{Relation to Gradient Descent:}
Both numerical and analytical gradients are methods to compute \( \nabla L(W) \). However, these gradients are not solutions to the optimization problem—they only provide the direction and rate of change. Gradient descent leverages these computed gradients to iteratively adjust \( W \), bridging the gap between gradient computation and optimization. This necessity arises because solving for \( W^* \) in closed form is often impractical, as explained in Section~\ref{enrichment:why_analytical_impractical}.

\subsection{Gradient Descent: The Iterative Optimization Algorithm}

\subsubsection{Motivation and Concept}

Gradient descent is an iterative algorithm that updates \( W \) by moving in the direction of the steepest descent, guided by \( -\nabla L(W) \). The update rule is:

\[
W \gets W - \eta \nabla L(W),
\]

where \( \eta \) is the \textbf{learning rate}, controlling the step size.

\paragraph{Steps of Gradient Descent:}
\begin{enumerate}
	\item \textbf{Initialization:} Choose a starting point \( W_0 \), often initialized randomly.
	\item \textbf{Gradient Computation:} Calculate \( \nabla L(W) \) analytically or numerically.
	\item \textbf{Update Rule:} Adjust \( W \) using the update equation.
	\item \textbf{Stopping Criterion:} Repeat until convergence or until a maximum number of iterations is reached.
\end{enumerate}

\begin{figure}[H]
	\centering
	\includegraphics[width=0.8\textwidth]{Figures/Chapter_4/Slide_47.jpg}
	\caption{Gradient Descent: Iterative optimization using gradient updates.}
	\label{fig:chapter4_gradient_descent}
\end{figure}

\subsubsection{Hyperparameters of Gradient Descent}

\paragraph{1. Learning Rate (\( \eta \)):}
\begin{itemize}
	\item Controls the step size in the direction of \( -\nabla L(W) \).
	\item \textbf{Small \( \eta \):} Converges slowly.
	\item \textbf{Large \( \eta \):} Risks overshooting the minimum or diverging.
\end{itemize}

\paragraph{2. Weight Initialization:}
\begin{itemize}
	\item The starting point \( W_0 \) significantly affects convergence.
	\item Random initialization is common but must ensure weights are appropriately scaled to prevent vanishing or exploding gradients.
\end{itemize}

\paragraph{3. Stopping Criterion:}
\begin{itemize}
	\item Define when to terminate the algorithm, e.g., maximum iterations, small gradient magnitude, or minimal change in loss.
\end{itemize}

\section{Visualizing Gradient Descent}

\subsection{Understanding Gradient Descent Through Visualization}
Gradient descent can be difficult to conceptualize due to the high-dimensional nature of modern optimization problems. Since humans are limited to perceiving in three dimensions, two common visualization approaches are used to make the process more intuitive:

\begin{itemize}
	\item \textbf{3D Surface Plot:} This approach visualizes the loss landscape as a surface, where the \( x \)- and \( y \)-axes correspond to two parameters (e.g., \( \theta_0 \) and \( \theta_1 \)), and the \( z \)-axis represents the loss value. The objective is to find the combination of \( \theta_0 \) and \( \theta_1 \) that minimizes the loss, represented by the lowest point on the surface.
	\item \textbf{2D Contour Plot:} An alternative is a 2D contour plot of the loss function, where the lines represent level sets (i.e., combinations of parameters where \( L(\theta_0, \theta_1) \) remains constant). The gradient descent process is visualized as a path that moves across these contours toward the minimum.
\end{itemize}

\begin{figure}[H]
	\centering
	\includegraphics[width=0.8\textwidth]{Figures/Chapter_4/Slide_48.jpg}
	\caption{Visualization of Gradient Descent Using a Contour Plot. The path starts at a high-loss region (blue) and iteratively moves toward a lower-loss region (red).}
	\label{fig:chapter4_gradient_descent_contour}
\end{figure}

\subsection{Properties of Gradient Descent}
Visualization of gradient descent reveals several interesting properties of the algorithm:

\subsubsection{Curved Paths Toward the Minimum}
The path taken by gradient descent does not follow a straight line toward the bottom of the loss landscape. Instead, it arcs and curves around the surface. This behavior occurs because the gradient descent algorithm relies on local information to iteratively update the parameters, causing it to adjust its direction based on the curvature of the loss landscape.

\subsubsection{Slowing Down Near the Minimum}
Gradient descent starts with larger steps when the gradient magnitude is high and naturally slows down as the gradient magnitude decreases. This behavior is due to the relationship between the gradient and the steepness of the loss surface:

\begin{itemize}
	\item The gradient is a measure of how quickly the loss function changes with respect to the parameters. 
	\item Near the minimum of the loss surface, the loss function becomes flatter. Mathematically, this means the rate of change (i.e., the gradient) becomes smaller as we approach the minimum.
\end{itemize}

As a result:
\begin{itemize}
	\item The magnitude of the gradient decreases in flatter regions of the loss surface, leading to smaller parameter updates during each step.
	\item This natural reduction in step size ensures a more refined and precise search for the optimal solution as gradient descent approaches the minimum.
\end{itemize}

By adapting to the geometry of the loss surface, gradient descent inherently balances exploration and precision, enabling effective convergence toward the minimum.

\subsection{Batch Gradient Descent}
The version of gradient descent shown in Figure \ref{fig:chapter4_gradient_descent_contour} is known as \textbf{Batch Gradient Descent} or \textbf{Full Batch Gradient Descent}.

In this approach:
\begin{itemize}
	\item The loss function is computed as the average loss over the entire training set.
	\item The gradient at each step is computed as the sum of gradients across all training examples.
\end{itemize}

While batch gradient descent provides stable and precise updates, it becomes computationally expensive for large datasets, as each iteration requires processing the entire training set. This limitation makes it impractical for many real-world applications, where faster alternatives are needed.

\section{Stochastic Gradient Descent (SGD)}

\subsection{Introduction to Stochastic Gradient Descent}
Batch Gradient Descent, though conceptually simple, is often impractical due to its computational and memory inefficiency, especially with large datasets. A more efficient alternative is \textbf{Stochastic Gradient Descent (SGD)}, which approximates the sum over the entire dataset (used to compute the loss and gradients) by using a \textbf{minibatch} of examples.

\subsubsection{Minibatch Gradient Computation}
Instead of computing gradients over the entire dataset, \textbf{SGD} uses minibatches:
\begin{itemize}
	\item A \textbf{minibatch} is a small subset of the dataset, with common batch sizes being \(32\), \(64\), \(128\), or even larger values like \(512\) or \(1024\), depending on the available computational resources.
	\item The general heuristic is to maximize the batch size to fully utilize available GPU memory. For distributed training setups, minibatches can be spread across multiple GPUs or machines, allowing for very large effective batch sizes.
\end{itemize}

\begin{figure}[H]
	\centering
	\includegraphics[width=0.8\textwidth]{Figures/Chapter_4/Slide_50.jpg}
	\caption{Stochastic Gradient Descent: Leveraging minibatches to approximate loss and gradients.}
	\label{fig:chapter4_sgd_intro}
\end{figure}

\subsubsection{Data Sampling and Epochs}
SGD introduces randomness in data selection, which affects how it iterates through the dataset:
\begin{itemize}
	\item At the beginning of each \textbf{epoch} (a single pass through the entire dataset), the data is shuffled randomly to ensure varied sampling.
	\item During each iteration, a minibatch is selected in sequence from the shuffled data until all samples are processed, completing the epoch.
	\item This process is repeated for multiple epochs, with the dataset being reshuffled at the start of each one to avoid overfitting to a specific order of examples.
\end{itemize}

\subsubsection{Why "Stochastic"?}
\begin{figure}[H]
	\centering
	\includegraphics[width=0.8\textwidth]{Figures/Chapter_4/Slide_52.jpg}
	\caption{SGD approximates the expectation over all possible samples via minibatch sampling.}
	\label{fig:chapter4_sgd_sampling}
\end{figure}
SGD is stochastic because the loss and gradient computations are based on sampled subsets of data. From a probabilistic perspective:
\begin{itemize}
	\item The loss function can be viewed as an expectation over all possible data samples from the true underlying distribution.
	\item Averaging the sample loss over a minibatch approximates this expectation, and the same applies to the gradients.
\end{itemize}

\subsection{Advantages and Challenges of SGD}

\subsubsection{Advantages}
SGD provides significant computational advantages:
\begin{itemize}
	\item \textbf{Efficiency:} Reduces memory requirements and computational cost per iteration.
	\item \textbf{Scalability:} Enables training on datasets too large to fit entirely into memory.
\end{itemize}

\subsubsection{Challenges of SGD}
Despite its utility, SGD comes with inherent challenges:

\paragraph{High Condition Numbers}
When the loss landscape changes rapidly in one direction but slowly in another, it is said to have a \textbf{high condition number}, which can be numerically estimated as the ratio of the largest to smallest singular values of the Hessian matrix (more about it in section 3.1. of \cite{alger2019_data}). This results in:
\begin{itemize}
	\item \textbf{Oscillations:} Gradients in steep directions may overshoot the minimum, causing zig-zagging behavior.
	\item \textbf{Slow Convergence:} Reducing the step size to mitigate oscillations slows progress in shallow directions, leading to undesirable convergence times.
\end{itemize}

\begin{figure}[H]
	\centering
	\includegraphics[width=0.8\textwidth]{Figures/Chapter_4/Slide_55.jpg}
	\caption{Visualization of oscillations in SGD caused by high condition numbers.}
	\label{fig:chapter4_high_condition_number}
\end{figure}

\paragraph{Saddle Points and Local Minima}
SGD may encounter:
\begin{itemize}
	\item \textbf{Saddle Points:} Points where the gradient is zero, but the function increases in one direction and decreases in another. At the tip of the saddle, the gradient provides no useful direction, potentially stalling optimization.
	\item \textbf{Local Minima:} Points where the gradient is zero but are not the global minimum. The algorithm can become trapped, unable to escape without additional techniques.
\end{itemize}

\begin{figure}[H]
	\centering
	\includegraphics[width=0.8\textwidth]{Figures/Chapter_4/Slide_57.jpg}
	\caption{Examples of saddle points and local minima in loss landscapes.}
	\label{fig:chapter4_saddle_point}
\end{figure}

\paragraph{Noisy Gradients}
Due to the stochastic nature of SGD, gradient updates can be noisy:
\begin{itemize}
	\item \textbf{Definition of Noise:} Gradients are computed from minibatches rather than the entire dataset, making them approximate and introducing randomness.
	\item \textbf{Impact of Noise:} Noisy gradients can cause the algorithm to wander around the loss surface instead of taking a direct path to the minimum, leading to slower convergence.
\end{itemize}

\begin{figure}[H]
	\centering
	\includegraphics[width=0.8\textwidth]{Figures/Chapter_4/Slide_58.jpg}
	\caption{Noisy gradient updates in SGD resulting in slower convergence.}
	\label{fig:chapter4_noisy_gradients}
\end{figure}

\subsection{Looking Ahead: Improving SGD}
While vanilla SGD is simple and effective, its limitations motivate the development of advanced variants that address its challenges. In the following sections, we will explore these modifications, starting with simpler adjustments and progressing to state-of-the-art optimizers like \textbf{Adam}.

\section{SGD with Momentum}
\subsection{Motivation}
While \textbf{SGD} is effective, it suffers from several challenges such as oscillations in ravines, difficulties escaping local minima or saddle points, and noise in gradient computations. \textbf{SGD with Momentum} addresses these issues by incorporating a velocity term that smooths updates and accelerates convergence in the right direction.

\subsection{How SGD with Momentum Works}
The concept can be visualized as a ball rolling down a high-dimensional loss surface. Instead of directly using the gradient direction for updates, we maintain a \textbf{velocity vector} \( \mathbf{v}_t \) that combines the current gradient and past gradients through an \textbf{Exponential Moving Average (EMA)}.

\subsubsection{Update Equations}
At each step \( t \), we update the velocity and position as follows:
\begin{align*}
	\mathbf{v}_t &= \rho \mathbf{v}_{t-1} + \eta \nabla L(\mathbf{x}_t), \\
	\mathbf{x}_{t+1} &= \mathbf{x}_t - \mathbf{v}_t,
\end{align*}
where:
\begin{itemize}
	\item \( \rho \): \textbf{Momentum coefficient}, typically \( 0.9 \) or \( 0.99 \), representing the friction or decay rate. 0.9 is often the default choice, as it strikes a balance between immediate gradient and history. 
	\item \( \eta \): \textbf{Learning rate}, controlling the step size.
	\item \( \mathbf{v}_t \): Velocity at step \( t \), an EMA of past gradients.
\end{itemize}

\begin{figure}[H]
	\centering
	\includegraphics[width=0.8\textwidth]{Figures/Chapter_4/Slide_60.jpg}
	\caption{SGD with Momentum: Implementation in PyTorch.}
	\label{fig:chapter4_sgd_momentum}
\end{figure}

\subsection{Intuition Behind Momentum}
\begin{itemize}
	\item The velocity term integrates gradients over time, effectively smoothing out noisy updates.
	\item The rolling-ball analogy illustrates how momentum helps maintain speed in valleys and escape saddle points or local minima.
	\item By decaying the velocity vector with \( \rho \), we emphasize recent gradients while retaining historical trends, allowing for smoother optimization trajectories. 
\end{itemize}

Note that overall, higher momentum (e.g., 0.9 or 0.99) usually aids faster convergence and smoother updates, but can lead to overshooting or oscillations if paired with a learning rate that is too large. The larger \( \rho \), the more past gradient information is retained. Although rates like 0.99 are useful when you need to move quickly along a consistent direction, they often require careful tuning of the learning rate. Hence, 0.9 fits most tasks where consistent gradient directions exist. Larger values will be used only for deep models with large datasets, in which gradients are relatively stable.

\begin{figure}[H]
	\centering
	\includegraphics[width=0.8\textwidth]{Figures/Chapter_4/Slide_62.jpg}
	\caption{Alternative formulation of SGD with Momentum.}
	\label{fig:chapter4_momentum_alternative}
\end{figure}

\subsection{Benefits of Momentum}
Momentum addresses the three key problems of SGD:
\begin{itemize}
	\item \textbf{Local Minima and Saddle Points:} The velocity term allows the optimizer to pass through these points due to accumulated momentum.
	\item \textbf{Poor Conditioning:} Oscillations in ravines are smoothed, and updates are more stable as momentum averages out noisy gradients.
	\item \textbf{Noisy Gradients:} Momentum helps filter out random fluctuations in gradient directions, resulting in a more direct path to the minimum.
\end{itemize}

\begin{figure}[H]
	\centering
	\includegraphics[width=0.8\textwidth]{Figures/Chapter_4/Slide_63.jpg}
	\caption{Momentum accelerates convergence by smoothing oscillations and reducing noise.}
	\label{fig:chapter4_momentum_benefits}
\end{figure}

\subsection{Downsides of Momentum}
Despite its advantages, SGD with Momentum has several limitations:
\begin{itemize}
	\item \textbf{Hyperparameter Sensitivity:} The choice of \( \rho \) (momentum coefficient) and \( \eta \) (learning rate) significantly affects performance.
	\item \textbf{Memory Requirements:} Additional storage is needed to maintain velocity vectors for all parameters.
	\item \textbf{Lack of Adaptivity:} Momentum does not adapt the learning rate for individual weights, limiting its effectiveness for sparse gradients or features with varying importance.
	\item \textbf{Robustness:} Momentum can amplify noise under certain conditions, leading to erratic updates.
	\item \textbf{Slower Convergence:} Advanced optimizers like Adam often achieve faster convergence rates.
\end{itemize}

\subsection{Nesterov Momentum: A Look-Ahead Strategy}
\subsubsection{Overview}
\textbf{Nesterov Momentum} builds upon SGD with Momentum by introducing a "look-ahead" mechanism. Instead of calculating the gradient at the current position, Nesterov computes it at the projected future position, determined by the current velocity vector. This adjustment allows for more precise updates and improved convergence behavior.

\subsubsection{Mathematical Formulation}
The Nesterov update rules are given as:
\begin{align*}
	\mathbf{v}_{t+1} &= \rho \mathbf{v}_t - \eta \nabla f\big(\mathbf{x}_t + \rho \mathbf{v}_t\big), \\
	\mathbf{x}_{t+1} &= \mathbf{x}_t + \mathbf{v}_{t+1},
\end{align*}
where:
\begin{itemize}
	\item \( \rho \): Momentum coefficient, controlling the influence of past velocities.
	\item \( \eta \): Learning rate.
	\item \( \nabla f\big(\mathbf{x}_t + \rho \mathbf{v}_t\big) \): Gradient computed at the "look-ahead" position.
\end{itemize}

\begin{figure}[H]
	\centering
	\includegraphics[width=0.8\textwidth]{Figures/Chapter_4/Slide_65.jpg}
	\caption{Nesterov Momentum: Look-ahead Gradient Update.}
	\label{fig:chapter4_nesterov_momentum}
\end{figure}

\subsubsection{Motivation and Advantages}
Nesterov Momentum improves upon traditional momentum methods by providing a more precise update mechanism, leading to faster convergence and reduced oscillations. The key motivations and advantages include:

\begin{itemize}
	\item \textbf{Reduced Oscillations:}
	\begin{itemize}
		\item In traditional momentum methods, the gradient is computed at the current position, and the accumulated velocity can overshoot the minimum due to excessive momentum, especially in steep directions.
		\item Nesterov Momentum addresses this by computing the gradient at a "look-ahead" position (\( \mathbf{x}_t + \rho \mathbf{v}_t \)), effectively anticipating the overshoot and applying a correction before the step is taken.
		\item By integrating this "look-ahead gradient," Nesterov smoothens the update trajectory, particularly in ravines (areas with steep gradients in one direction and shallow gradients in another), thereby reducing zig-zagging behavior.
	\end{itemize}
	
	\item \textbf{Faster Convergence:}
	\begin{itemize}
		\item The look-ahead mechanism allows Nesterov Momentum to make more informed updates, as the gradient incorporates information about where the optimizer is heading, not just where it currently is.
		\item This results in more efficient use of gradient information, leading to quicker progress along flat regions and better handling of curved loss landscapes.
		\item Faster convergence also stems from the smaller step adjustments needed to compensate for overshooting, ensuring that the optimizer focuses on approaching the minimum directly.
	\end{itemize}
	
	\item \textbf{Improved Stability in High-Condition-Number Landscapes:}
	\begin{itemize}
		\item In poorly conditioned loss surfaces, where gradients change drastically along different directions, the look-ahead gradient reduces oscillations in the steep direction while maintaining steady progress in the shallow direction.
		\item This makes Nesterov particularly effective in minimizing the effect of uneven gradient magnitudes across dimensions, stabilizing the optimization process.
	\end{itemize}
\end{itemize}

\subsubsection{Reformulation for Practical Implementation}
Although the original Nesterov formulation depends on the gradient at a projected point, it can be rewritten for practical implementation:
\begin{align*}
	\mathbf{x}_{t+1} &= \mathbf{x}_t + \rho \mathbf{v}_t - \eta \nabla f(\mathbf{x}_t), \\
	\mathbf{v}_{t+1} &= \rho \mathbf{v}_t - \eta \nabla f(\mathbf{x}_t).
\end{align*}

This reformulation ensures compatibility with modern optimization frameworks by relying only on the gradient at the current position.

\subsubsection{Comparison with SGD and SGD+Momentum}
Momentum-based methods, including Nesterov, tend to overshoot near minima due to accumulated velocity. Nesterov's look-ahead mechanism mitigates this overshooting, producing a more efficient path to the minimum. 

\subsubsection{Limitations of Nesterov Momentum and the Need for Adaptivity}

While Nesterov Momentum addresses several shortcomings of traditional momentum methods, it still has limitations that motivate further advancements in optimization techniques:

\begin{itemize}
	\item \textbf{Uniform Learning Rate:}
	\begin{itemize}
		\item Nesterov Momentum uses a single global learning rate for all weight components, regardless of their individual gradient behavior.
		\item In scenarios where gradients vary significantly across dimensions (e.g., high-condition-number landscapes or sparse features), this uniform learning rate can lead to inefficient updates:
		\begin{itemize}
			\item Large gradients may result in overly cautious updates, slowing down convergence.
			\item Small gradients may cause under-updated weights, making progress in flat regions painfully slow.
		\end{itemize}
	\end{itemize}
	
	\item \textbf{Sensitivity to Hyperparameters:}
	\begin{itemize}
		\item Nesterov Momentum requires careful tuning of both the learning rate (\( \eta \)) and the momentum parameter (\( \rho \)).
		\item Suboptimal hyperparameter settings can lead to erratic behavior, such as oscillations, overshooting, or excessively slow convergence.
	\end{itemize}
	
	\item \textbf{No Adaptivity to Gradient Magnitudes:}
	\begin{itemize}
		\item Nesterov Momentum does not adapt the learning rate based on the magnitude of the gradients. This is particularly problematic for sparse data or infrequent features, where gradients may carry highly informative yet small signals.
		\item The lack of adaptivity can hinder optimization in modern machine learning applications, such as natural language processing or deep learning for image recognition, where gradient magnitudes can vary significantly.
	\end{itemize}
	
	\item \textbf{Stochastic Noise Amplification:}
	\begin{itemize}
		\item While Nesterov reduces oscillations, its velocity updates can amplify noise in stochastic gradients, leading to suboptimal parameter updates and slower convergence.
		\item This issue becomes particularly evident in noisy or sparse datasets, where gradient signals are less stable.
	\end{itemize}
\end{itemize}

\newpage

\paragraph{Motivation for a Better Optimizer: AdaGrad}
To overcome these limitations, we seek optimizers that:
\begin{itemize}
	\item Adjust learning rates adaptively for each parameter based on the historical behavior of its gradients.
	\item Mitigate the impact of high-condition-number landscapes by dampening updates in steep directions while accelerating progress in flat regions.
	\item Improve handling of sparse data and infrequent features by increasing learning rates for weights with smaller gradients.
\end{itemize}

AdaGrad introduces an adaptive learning rate mechanism that addresses these issues by scaling updates inversely proportional to the square root of the accumulated squared gradients. This adaptivity enables AdaGrad to make more efficient and stable progress across diverse optimization landscapes, as we will explore in the next section.

\section{AdaGrad: Adaptive Gradient Algorithm}

AdaGrad, short for \textbf{Adaptive Gradient Algorithm}, adjusts the learning rate for each parameter based on the historical squared gradients. This adaptivity allows the optimizer to handle scenarios where different parameters require significantly different learning rates.

\begin{figure}[H]
	\centering
	\includegraphics[width=0.8\textwidth]{Figures/Chapter_4/Slide_73.jpg}
	\caption{AdaGrad Implementation in PyTorch. Each parameter is updated individually, with learning rates adjusted based on the historical squared gradients.}
	\label{fig:chapter4_adagrad_impl}
\end{figure}

\subsection{How AdaGrad Works}
Rather than using a fixed global learning rate \( \eta \), AdaGrad adjusts the learning rate for each parameter \( w_i \) dynamically. We denote the parameters (weight components) as: $w=\left(w_0, w_1 \cdots w_i, \cdots, w_n\right)$, and at step t as: $w_t=\left(w_{t_0}, w_{t_1} \cdots w_{t_i}, \cdots, w_{t_n}\right)$
We denote the gradient of the loss with respect to each weight component at step t as $g_{t_i}=\nabla_w J\left(w_{t_i}\right)$. 
\newpage

\paragraph{Updating the Weight Matrix Components} 
Unlike SGD in which the update for each parameter (weight component) at step $t$ is:
\[
w_{t_{i+1}}=w_{t_i}-\eta \cdot g_{t_i}
\]
In \textbf{Adagrad} the update rule for each parameter (weight component) at step $t$ is
\[
w_{t_{i+1}}=w_{t_i}-\frac{\eta}{\sqrt{G_{t_{(i, i)}}+\epsilon}} \cdot g_{t_i}
\]
$G_t \in \mathbb{R}^{n \times n}$ is a diagonal matrix, where each diagonal element $(i, i)$ is the sum of squares of the gradients with respect to $w_i$ at the step, meaning, $G_{t_{(i, i)}}=\sum_{j=0}^t\left(g_{j_{i},}\right)^2$. Also note that $\epsilon$ serves as a smoothing term, that helps to avoid division by 0 (usually in the form of $1 e-8$ ).

As $G_t \in \mathbb{R}^{n \times n}$ has the sum of squares of all past gradients with respect to all parameters $w$ along its diagonal, We can vectorize our implementation by performing a matrix-vector product $\odot$ between $G_t$ and $g_t$:
\[
w_{t}=w_{t-1}-\frac{\eta}{\sqrt{G_{t}+\epsilon}} \cdot g_{t}
\] 

\paragraph{Why Does This Work?}
The division by \( \sqrt{G_t + \epsilon} \) achieves two things:
\begin{itemize}
	\item \textbf{Damping Large Gradients:} Parameters with consistently large gradients will accumulate larger \( G_t[i] \) values, reducing their effective learning rate. This dampens oscillations in steep regions of the loss surface.
	\item \textbf{Accelerating Small Gradients:} Parameters with small or infrequent gradients will have smaller \( G_t[i] \) values, increasing their effective learning rate. This ensures progress in flatter regions or for parameters with sparse updates.
\end{itemize}

\subsection{Advantages of AdaGrad}
\begin{itemize}
	\item \textbf{Adaptive Learning Rates:}
	\begin{itemize}
		\item No need for manual tuning of \( \eta \), as the learning rate is adjusted dynamically for each parameter.
	\end{itemize}
	\item \textbf{Effective for Sparse Gradients:}
	\begin{itemize}
		\item Particularly useful in scenarios like natural language processing or recommendation systems, where certain features or gradients are updated infrequently.
	\end{itemize}
\end{itemize}

\subsection{Disadvantages of AdaGrad}
Despite its strengths, AdaGrad has notable limitations:
\begin{itemize}
	\item \textbf{Aggressive Learning Rate Decay:}
	\begin{itemize}
		\item The cumulative sum of squared gradients \( G_t[i] \) grows over time, causing the learning rate to shrink excessively. This often leads to slow convergence or stagnation, particularly in non-convex optimization problems.
	\end{itemize}
	\item \textbf{No Momentum:}
	\begin{itemize}
		\item AdaGrad does not include a momentum term to smooth out oscillations or accelerate convergence along shallower dimensions.
	\end{itemize}
	\newpage
	\item \textbf{Inability to Forget Past Gradients:}
	\begin{itemize}
		\item All past gradients are treated equally, which can be problematic in non-convex problems with varying loss landscape dynamics. An example to emphasize how big of an issue this is: we might be going down a steep slope, then reaching a plateau, and then a steep portion again, and the fact that our $G_t$ got really big and our updates,	in turn, get really small, will make our optimization efforts ineffective.
	\end{itemize}
\end{itemize}

\section{RMSProp: Root Mean Square Propagation}

\subsection{Motivation for RMSProp}
While \textbf{AdaGrad} effectively adapts learning rates for individual parameters by accumulating squared gradients, it suffers from a major limitation: the accumulation grows indefinitely. Over time, this causes the effective learning rate to shrink excessively, slowing down optimization or halting it entirely.

\textbf{RMSProp} addresses this issue by introducing a decay factor, which transforms AdaGrad into a \textit{leaky version} of itself. By ensuring that only recent gradients significantly influence the updates, RMSProp prevents the learning rate from diminishing too aggressively, allowing optimization to maintain steady progress over time.

\subsection{How RMSProp Works}
RMSProp modifies the sum of squared gradients \( G_t \) in AdaGrad to an \textbf{exponentially weighted moving average (EWMA)} of squared gradients:
\[
G_t = \rho G_{t-1} + (1 - \rho) g_t^2,
\]
where:
\begin{itemize}
	\item \( G_t \): The EWMA of squared gradients at step \( t \),
	\item \( \rho \): The decay rate (forgetting factor, typically set to 0.9),
	\item \( g_t^2 \): The element-wise square of the gradient at step \( t \).
\end{itemize}

Using this updated \( G_t \), the parameter update rule becomes:
\[
w_{t+1} = w_t - \frac{\eta}{\sqrt{G_t + \epsilon}} \cdot g_t,
\]
where:
\begin{itemize}
	\item \( \eta \): The learning rate,
	\item \( \epsilon \): A small constant (e.g., \( 10^{-8} \)) to prevent division by zero.
\end{itemize}

\subsection{Updating the Weight Matrix Components}
We denote:
\begin{itemize}
	\item Parameters (weight components): \( w = [w_1, w_2, \ldots, w_n] \),
	\item Gradient of the loss with respect to each parameter at step \( t \): \( g_{t_i} = \nabla_w J(w_{t_i}) \).
\end{itemize}

The update for each parameter \( w_i \) is:
\[
G_t[i] = \rho G_{t-1}[i] + (1 - \rho) g_t[i]^2,
\]
\[
w_{t+1}[i] = w_t[i] - \frac{\eta}{\sqrt{G_t[i] + \epsilon}} \cdot g_t[i].
\]

This ensures parameters with consistently large gradients have reduced learning rates, while parameters with smaller gradients have relatively larger learning rates.

\begin{figure}[H]
\centering
\includegraphics[width=0.8\textwidth]{Figures/Chapter_4/Slide_74.jpg}
\caption{The transformation from AdaGrad to RMSProp using a decay rate (\( \rho \)). RMSProp ensures better progress over the course of training by forgetting older squared gradients.}
\label{fig:chapter4_rmsprop_conversion}
\end{figure}

\subsection{Advantages of RMSProp}
\begin{itemize}
\item \textbf{Prevents Learning Rate Decay:}
\begin{itemize}
	\item By introducing a forgetting factor, RMSProp avoids the excessive shrinking of learning rates observed in AdaGrad.
\end{itemize}
\item \textbf{Adaptability:}
\begin{itemize}
	\item RMSProp adjusts learning rates dynamically based on the history of squared gradients, making it suitable for non-convex problems.
\end{itemize}
\item \textbf{Stability:}
\begin{itemize}
	\item By dampening progress along steep directions, RMSProp reduces oscillations while accelerating motion in flatter regions.
\end{itemize}
\end{itemize}

\subsection{Downsides of RMSProp}
\label{sec:downsides-rmsprop}

\subsubsection{No Momentum Carry-Over}
\label{subsubsec:no-momentum-carry-over}
While RMSProp adapts its learning rate per parameter, it does not explicitly maintain a ``velocity'' term that accumulates gradients over time.
\begin{itemize}
	\item \textbf{Reduced Acceleration:} 
	In standard momentum-based methods (e.g., SGD with momentum), a portion of the previous update carries over to the next, helping the optimizer power through saddle points and shallow minima. RMSProp does not have this explicit accumulation, but despite that, it (and other adaptive optimizers like Adagrad) is not powerless against saddle points or local minima. By adjusting step sizes dimension-wise, RMSProp can still navigate tricky landscapes—sometimes more effectively than vanilla SGD. However, without an explicit momentum mechanism, it may need more careful tuning (sensitivity to hyperparameters) or additional iterations to escape challenging regions.
\end{itemize}

\subsubsection{Bias in Early Updates}
\label{subsubsec:bias-early-updates}
RMSProp maintains exponentially decaying running averages of squared gradients:
\[
G_t = \rho G_{t-1} + (1 - \rho)\,g_t^2,
\]
where \(\rho\) (\(0 < \rho < 1\)) is the \emph{decay factor}, and the model parameters \(w\) are updated as:
\[
w_{t+1} \;\leftarrow\; w_{t} \;-\; \eta \,\frac{g_t}{\sqrt{G_t + \epsilon}},
\]
with \(\eta\) being the \emph{learning rate} and \(\epsilon\) a small constant for numerical stability.

\begin{itemize}
	\item \textbf{Underestimated Variances Lead to Larger Steps:}
	Early in training, \(G_t\) can be underestimated due to insufficient historical data. This makes the denominator, \(\sqrt{G_t + \epsilon}\), smaller than it should be, which can produce updates larger than intended and potentially lead to instability or overshooting.
	\item \textbf{No Built-In Bias Correction:}
	Unlike Adam, RMSProp does not include a bias-correction mechanism to compensate for these underestimated running averages in the initial training phase.
\end{itemize}

\subsubsection{Sensitivity to Hyperparameters}
\label{subsubsec:sensitivity-hparams}
RMSProp requires two main hyperparameters:
\begin{itemize}
	\item \textbf{Decay Factor (\(\rho\)):}
	Determines how quickly the running average of the squared gradients decays. 
	\begin{itemize}
		\item A \emph{large} \(\rho\) (close to 1) makes the exponential average change more slowly, placing \emph{greater emphasis on older} gradient information.
		\item A \emph{smaller} \(\rho\) places \emph{more weight on recent} gradients, allowing faster adaptation to new changes in the loss landscape.
	\end{itemize}
	\item \textbf{Learning Rate (\(\eta\)):}
	Controls the scale of each update. Poor choices can cause exploding or vanishing updates, depending on the curvature of the loss landscape.
\end{itemize}
Because both \(\rho\) and \(\eta\) must be tuned, RMSProp can be quite sensitive to hyperparameter selection.

\subsection{Motivation for Adam, a SOTA Optimizer}
\label{subsec:motivation-adam}
\emph{Adam} (\emph{Adaptive Moment Estimation}) extends RMSProp in several key ways:
\begin{itemize}
	\item \textbf{Incorporates Momentum:}
	Adam adds an explicit exponential moving average of the gradients, giving it a ``velocity''-like term that smooths updates and helps traverse saddle points more effectively.
	\item \textbf{Bias Correction:}
	Adam corrects for the initially underestimated moving averages, preventing steps from becoming excessively large at the start of training.
	\item \textbf{Robust Defaults:}
	Adam’s standard hyperparameters (\(\beta_1 = 0.9\), \(\beta_2 = 0.999\), \(\eta = 1e-3 \text{ or } 1e-4 \) for the learning rate) are often effective across many tasks, easing the tuning burden compared to vanilla RMSProp.
\end{itemize}

By blending momentum, adaptive learning rates, and bias correction, Adam often converges more smoothly and quickly than pure RMSProp, while retaining many of RMSProp’s advantages in complex, high-dimensional optimization landscapes.

\section{Adam: Adaptive Moment Estimation}

\subsection{Motivation for Adam}
Adam combines the strengths of momentum-based methods (like SGD+Momentum) and adaptive learning rate methods (like RMSProp). By integrating these two techniques, Adam effectively handles optimization challenges such as:
\begin{itemize}
	\item Escaping saddle points and overcoming noisy gradients.
	\item Reducing sensitivity to hyperparameter tuning.
	\item Achieving faster and more stable convergence, even on complex, non-convex loss landscapes.
\end{itemize}

The name Adam stands for \textbf{Adaptive Moment Estimation}, referring to its use of \textbf{first} and \textbf{second moments} of gradients:
\begin{itemize}
	\item The \textbf{first moment} represents the mean of gradients, which estimates the rate of change of the model parameters.
	\item The \textbf{second moment} represents the variance of gradients, reflecting how spread out the gradients are around the mean value.
\end{itemize}

By utilizing these moments, Adam provides better control over optimization, leveraging the gradient's direction and its historical updates for efficient learning.

\subsection{How Adam Works}
Adam maintains two moving averages during training:
\begin{itemize}
	\item \textbf{First moment (mean)}: An exponentially weighted average of the gradients, capturing their direction and magnitude over time.
	\item \textbf{Second moment (variance)}: An exponentially weighted average of squared gradients, scaling updates based on their historical magnitudes.
\end{itemize}

The update equations are:
\[
m_t = \beta_1 m_{t-1} + (1 - \beta_1) g_t
\]
\[
v_t = \beta_2 v_{t-1} + (1 - \beta_2) g_t^2
\]
Here:
\begin{itemize}
	\item \( m_t \): First moment estimate (gradient mean).
	\item \( v_t \): Second moment estimate (gradient variance).
	\item \( g_t \): Gradient of the loss at step \( t \).
	\item \( \beta_1 \): Decay rate for the first moment (default \( 0.9 \)).
	\item \( \beta_2 \): Decay rate for the second moment (default \( 0.999 \)).
\end{itemize}

\begin{figure}[H]
	\centering
	\includegraphics[width=0.8\textwidth]{Figures/Chapter_4/Slide_78.jpg}
	\caption{Adam implementation without bias correction, as shown in PyTorch.}
	\label{fig:chapter4_adam_basic}
\end{figure}

\subsection{Bias Correction}
Adam applies \textbf{bias correction} to address the issue of initialization bias for \( m_0 = 0 \) and \( v_0 = 0 \). Without correction, the estimates for \( m_t \) and \( v_t \) would be biased toward zero, especially in the early stages of training. This is a huge issue, as the steps in the beginning of the optimization process will thus be undesirably large, and can even lead to overshooting or instability. Hence, for optimal performance bias correction is undoubtedly needed. Bias correction is computed as follows:
\[
\hat{m}_t = \frac{m_t}{1 - \beta_1^t}, \quad \hat{v}_t = \frac{v_t}{1 - \beta_2^t}.
\]
The corrected moments are used to compute the parameter updates:
\[
w_{t+1} = w_t - \frac{\eta}{\sqrt{\hat{v}_t} + \epsilon} \hat{m}_t.
\]
Here:
\begin{itemize}
	\item \( \eta \): Learning rate.
	\item \( \epsilon \): Smoothing term (default \( 10^{-8} \)) to avoid division by zero.
\end{itemize}

\begin{figure}[H]
	\centering
	\includegraphics[width=0.8\textwidth]{Figures/Chapter_4/Slide_81.jpg}
	\caption{Complete Adam implementation with bias correction as shown in PyTorch.}
	\label{fig:chapter4_adam_bias_correction}
\end{figure}

\subsection{Why Adam Works Well in Practice}
Adam's robustness lies in its ability to adaptively scale updates for each parameter while incorporating momentum. 
\begin{figure}[H]
	\centering
	\includegraphics[width=0.8\textwidth]{Figures/Chapter_4/Slide_82.jpg}
	\caption{Examples of Adam's hyperparameter usage in various deep learning papers.}
	\label{fig:chapter4_adam_hyperparams}
\end{figure}

Figure \ref{fig:chapter4_adam_hyperparams} highlights the widespread adoption of Adam with default hyperparameters (\( \beta_1 = 0.9, \beta_2 = 0.999, \eta = 10^{-3} \text{ or } \eta = 10^{-4}\)) in numerous deep learning papers. These settings work well across a variety of tasks with minimal tuning. 


\subsection{Comparison with Other Optimizers}
\begin{figure}[H]
	\centering
	\includegraphics[width=0.8\textwidth]{Figures/Chapter_4/Slide_83.jpg}
	\caption{Comparison of optimizers: SGD, SGD+Momentum, RMSProp, and Adam. Adam converges faster with fewer oscillations.}
	\label{fig:chapter4_adam_comparison}
\end{figure}

Figure \ref{fig:chapter4_adam_comparison} compares Adam with other optimizers like SGD, SGD+Momentum, and RMSProp. While all methods eventually converge, Adam typically converges faster, taking a more direct path to the minimum. It also handles noisy gradients better than momentum-based methods, resulting in fewer oscillations and faster recovery from overshooting. Regardless, it is \textbf{crucial} to remember that all the figures shown in this chapter are of 2 parameters only as we humans are limited to 3d. In very high dimensional landscapes, the behavior might greatly differ. As these are the common cases in deep learning, take this comparison with a grain of salt. It still is useful to gain some intuition regarding these optimization methods and their differences, but it's important to not have too much fate in it. 

\subsection{Advantages of Adam}
\begin{itemize}
	\item Combines momentum and adaptive learning rates for robust optimization.
	\item Handles noisy gradients effectively, reducing oscillations.
	\item Requires minimal hyperparameter tuning, making it user-friendly for practitioners.
	\item Achieves faster and more stable convergence than earlier methods.
\end{itemize}

\subsection{Limitations of Adam}
Despite its strengths, Adam has some limitations:
\begin{itemize}
	\item \textbf{Overshooting:} While less troublesome than in SGD+Momentum, Adam can still overshoot the minimum and take a while to recover.
	\item \textbf{Memory Usage:} Requires additional storage for \( m_t \) and \( v_t \), increasing memory overhead.
\end{itemize}

\paragraph{Looking Ahead}
While Adam is effective on its own, advanced variants like Nadam (Nesterov-accelerated Adam) and AdamW (Adam with weight decay) address specific issues, such as overshooting or generalization. However, for most applications, Adam remains a reliable and widely used optimizer in deep learning.

\section{AdamW: Decoupling Weight Decay from L2 Regularization}

\subsection{Motivation for AdamW}
While Adam is a widely used optimizer in deep learning, its integration with L2 regularization has been found problematic. Traditional Adam combines L2 regularization with weight updates during optimization. However, this approach can lead to unintended interactions:
\begin{itemize}
	\item \textbf{Magnitude Dependent Regularization:} L2 regularization affects the moment estimates, leading to an implicit adjustment of the learning rate for parameters with larger magnitudes.
	\item \textbf{Inconsistent Penalization:} The coupling of weight decay and optimization can distort the intended regularization effect.
\end{itemize}

To address these issues, \textbf{AdamW} decouples weight decay from L2 regularization, treating weight decay as a distinct step in the optimization process. This separation ensures that weight decay consistently penalizes parameter magnitudes without interfering with moment estimates.

\begin{figure}[H]
	\centering
	\includegraphics[width=0.8\textwidth]{Figures/Chapter_4/Slide_89.jpg}
	\caption{Integration of L2 regularization and weight decay in AdamW. Decoupling these ensures consistent penalization of parameter magnitudes.}
	\label{fig:chapter4_adamw_weight_decay}
\end{figure}

\subsection{How AdamW Works}
AdamW modifies the weight update rule by explicitly decoupling the weight decay term. The key steps are:
\begin{itemize}
	\item Compute the gradient \( g_t \) of the loss with respect to the weights.
	\item Apply bias-corrected first and second moments, as in Adam:
	\[
	\hat{m}_t = \frac{m_t}{1 - \beta_1^t}, \quad \hat{v}_t = \frac{v_t}{1 - \beta_2^t}.
	\]
	\item Update the weights using the Adam update rule, but subtract a scaled weight decay term:
	\[
	w_{t+1} = w_t - \eta \left( \frac{\hat{m}_t}{\sqrt{\hat{v}_t} + \epsilon} + \lambda w_t \right),
	\]
	where \( \lambda \) is the weight decay coefficient.
\end{itemize}

This decoupling ensures that the weight decay term acts as a pure penalization of large weights, independent of the adaptive learning rate mechanism.

\begin{figure}[H]
	\centering
	\includegraphics[width=0.8\textwidth]{Figures/Chapter_4/Slide_90.jpg}
	\caption{Pseudo-code for AdamW, illustrating the decoupling of weight decay from L2 regularization.}
	\label{fig:chapter4_adamw_pseudocode}
\end{figure}

\subsection{Note on Weight Decay in AdamW}

In the pseudo-code, the \textbf{violet term} in line 6 represents L2 regularization as it is typically implemented in Adam (not AdamW) in many deep learning frameworks. This term adds the weight decay directly to the loss function, and its gradient is incorporated into the computation of the total gradients \( g \). However, this approach introduces unintended consequences:

\begin{itemize}
	\item \textbf{Entanglement with Moving Averages:} When the regularization term is included in the loss, the moving averages \( m \) and \( v \) (used for the first and second moments of gradients) track not only the gradients of the loss function but also the contributions from the weight decay term.
	\item \textbf{Normalization Effect:} This interaction impacts the update step. Specifically, in Adam, line 12 of the pseudo-code includes \( \lambda \theta_{t-1} \) in the numerator, and this term gets normalized by \( \sqrt{\hat{v}_t} \) in the denominator. Consequently:
	\begin{itemize}
		\item Weights with large or highly variable gradients (corresponding to a larger \( \hat{v}_t \)) experience less regularization.
		\item Weights with small or slowly changing gradients are penalized more heavily, even though this may not align with the intended regularization behavior.
	\end{itemize}
\end{itemize}

This phenomenon undermines the effectiveness of L2 regularization in Adam, deviating from the intended proportionality of weight decay to the weight magnitude. It explains why models trained with Adam sometimes generalize less effectively than those trained with SGD, which handles L2 regularization as intended.

\subsection{The AdamW Improvement}

To address this issue, the authors of AdamW propose a critical modification: \textbf{decoupling the weight decay from the gradient computation}. Specifically:
\begin{itemize}
	\item The \textbf{violet term} in line 6 is removed from the gradient computation.
	\item The \textbf{green term} in line 12 applies weight decay as a direct adjustment to the parameter update after controlling for parameter-wise step sizes.
\end{itemize}
This decoupling ensures that:
\begin{enumerate}
	\item Weight decay acts only as a direct proportional penalty to the parameter values, independent of the gradient dynamics.
	\item The moving averages \( m \) and \( v \) track only the gradients of the loss function, preserving their intended role.
\end{enumerate}

\subsection{Advantages of AdamW}

Experimental results demonstrate that AdamW:
\begin{itemize}
	\item Improves training loss compared to standard Adam.
	\item Yields models that generalize significantly better, comparable to those trained with SGD+Momentum.
	\item Retains Adam's adaptability and efficiency for large-scale optimization tasks.
\end{itemize}

By resolving the shortcomings of L2 regularization in Adam, AdamW has become the recommended default optimizer for many deep learning problems.

\subsection{Why AdamW is the Default Optimizer}
AdamW combines the adaptive learning rates of Adam with the benefits of properly implemented weight decay, making it a powerful default optimizer for many deep learning tasks:
\begin{itemize}
	\item Works well out-of-the-box with minimal hyperparameter tuning.
	\item Handles large-scale, non-convex problems effectively.
	\item Avoids pitfalls of traditional L2 regularization in Adam, such as learning rate distortion.
\end{itemize}

\subsection{Limitations of AdamW}
Despite its advantages, AdamW is not without challenges:
\begin{itemize}
	\item Requires careful tuning of the weight decay coefficient \( \lambda \) for optimal performance.
	\item Sensitive to learning rate schedules, particularly for complex architectures.
\end{itemize}

\section{Second-Order Optimization}

\subsection{Overview of Second-Order Optimization}
So far, we have covered first-order optimization methods, which rely only on the gradient (the first derivative) to iteratively minimize the loss function. Second-order optimization extends this approach by incorporating information from the Hessian matrix (the second derivative), which captures the curvature of the loss landscape. 

The key idea behind second-order optimization is to approximate the objective function \( L(W) \) using a quadratic surface at the current point and then step towards minimizing this approximation. This allows the algorithm to be more adaptive, as it can:
\begin{itemize}
	\item Take larger steps in regions of low curvature.
	\item Take smaller steps in regions of high curvature.
\end{itemize}

\begin{figure}[H]
	\centering
	\includegraphics[width=0.8\textwidth]{Figures/Chapter_4/Slide_95.jpg}
	\caption{Second-order optimization forms a quadratic surface to approximate the objective function and adaptively determines step size.}
	\label{fig:chapter4_second_order_quadratic}
\end{figure}

When the Hessian is well-conditioned, second-order methods often converge in fewer iterations compared to first-order methods (e.g., standard gradient descent). They can also “jump” more directly toward a local minimum rather than zigzagging or making incremental adjustments.

In addition, because the Hessian reflects how sensitive each parameter is to changes in the loss function, second-order methods effectively adapt step sizes across different directions or dimensions. This automatic per-parameter scaling can reduce the need for manual learning-rate tuning and can help handle strongly anisotropic or ill-conditioned problems.

\subsection{Quadratic Approximation Using the Hessian}
In second-order optimization, the objective function \( L(W) \) is approximated using a Taylor series expansion around the current point \( W_t \):
\[
L(W) \approx L(W_t) + \nabla L(W_t)^T (W - W_t) + \frac{1}{2} (W - W_t)^T H(W_t) (W - W_t),
\]
where:
\begin{itemize}
	\item \( \nabla L(W_t) \) is the gradient at \( W_t \),
	\item \( H(W_t) \) is the Hessian matrix at \( W_t \), representing the second derivative of \( L(W) \) with respect to \( W \).
\end{itemize}

To minimize this quadratic approximation, the weight update rule is:
\[
W_{t+1} = W_t - H(W_t)^{-1} \nabla L(W_t).
\]

\subsection{Practical Challenges of Second-Order Methods}
While the concept of second-order optimization is theoretically appealing, it is rarely used in practice due to significant computational challenges:
\begin{itemize}
	\item \textbf{High Dimensionality:} The Hessian matrix has \( O(N^2) \) elements, where \( N \) is the number of parameters in the model. For modern deep learning models with millions or billions of parameters, storing the Hessian becomes infeasible.
	\item \textbf{Matrix Inversion:} To compute the update step, we must invert the Hessian matrix, which requires \( O(N^3) \) operations. This computational cost is prohibitive for high-dimensional problems.
	\item \textbf{Ill-Conditioned Hessians:} The Hessian matrix can be ill-conditioned, leading to numerical instability during inversion.
\end{itemize}

\begin{figure}[H]
	\centering
	\includegraphics[width=0.8\textwidth]{Figures/Chapter_4/Slide_98.jpg}
	\caption{Challenges of second-order optimization in high-dimensional spaces, including storage and computational costs.}
	\label{fig:chapter4_second_order_limitations}
\end{figure}

Despite their challenges, second-order methods can be useful for low-dimensional problems, where the parameter space is small, and computational costs are manageable.

\subsection{First-Order Methods Approximating Second-Order Behavior}
While second-order methods are computationally impractical for large-scale problems, some first-order methods attempt to approximate their behavior:
\begin{itemize}
	\item \textbf{Adagrad:} Adjusts the learning rate for each parameter based on the magnitude of past gradients, mimicking curvature information.
	\item \textbf{SGD+Momentum:} Incorporates a moving average of gradients, smoothing updates and approximating curvature.
	\item \textbf{Adam:} Combines the adaptive learning rates of Adagrad with momentum, effectively capturing some second-order properties without requiring explicit computation of the Hessian.
\end{itemize}

\subsection{Improving Second-Order Optimization: BFGS and L-BFGS}
Methods like \textbf{Broyden–Fletcher–Goldfarb–Shanno (BFGS)} and its limited-memory variant \textbf{L-BFGS} have been developed to address the challenge of expensive computations and high memory consumption. These methods approximate the Hessian matrix to reduce memory and computation costs, making second-order techniques more feasible in certain scenarios.

\begin{figure}[H]
	\centering
	\includegraphics[width=0.8\textwidth]{Figures/Chapter_4/Slide_99.jpg}
	\caption{BFGS and L-BFGS use approximations to reduce the computational and memory demands of second-order optimization.}
	\label{fig:chapter4_bfgs_lbfgs}
\end{figure}

Although getting into the bits and bytes of BFGS and L-BFGS is outside the scope of the lecture and this summary, it's still interesting to provide a high-level overview of the two algorithms and what they improve in second-order optimization methods. 

\subsubsection{BFGS: An Approximation of the Hessian Matrix}
BFGS is an iterative optimization algorithm that avoids the explicit computation of the Hessian matrix. Instead:
\begin{itemize}
	\item It uses gradient information to iteratively build an approximation of the inverse Hessian.
	\item Updates are performed using a rank-two update rule, ensuring that the approximation remains symmetric and positive definite.
	\item The update rule is efficient, allowing the algorithm to adaptively refine its estimates of the curvature.
\end{itemize}

While BFGS reduces the computational burden compared to exact second-order methods, it still requires \( O(N^2) \) storage for the approximate Hessian, making it unsuitable for high-dimensional problems.

\subsubsection{L-BFGS: Reducing Memory Requirements}
To address the memory limitations of BFGS, the \textbf{Limited-Memory BFGS (L-BFGS)} algorithm was introduced. Instead of storing the entire approximate Hessian, L-BFGS:
\begin{itemize}
	\item Maintains only a few vectors from the most recent iterations, significantly reducing memory requirements.
	\item Requires \( O(kN) \) storage, where \( k \) is the number of vectors retained and is much smaller than \( N \) (typically \( k \approx 10 \)).
	\item Iteratively updates the approximation using gradient differences and weight updates from recent steps.
\end{itemize}

This makes L-BFGS particularly useful for optimization problems with moderate dimensionality, such as natural language processing or small-scale machine learning tasks.

\subsubsection{Advantages and Limitations of BFGS and L-BFGS}
\paragraph{Advantages:}
\begin{itemize}
	\item \textbf{Adaptive Step Sizes:} BFGS and L-BFGS use curvature information to adjust step sizes, improving convergence rates compared to first-order methods.
	\item \textbf{Efficiency in Moderate Dimensions:} L-BFGS reduces memory usage, enabling the use of second-order ideas in medium-scale problems.
\end{itemize}

\paragraph{Limitations:}
\begin{itemize}
	\item \textbf{Still Computationally Expensive:} Even L-BFGS requires \( O(kN) \) storage and computations, making it impractical for very high-dimensional problems such as deep learning.
	\item \textbf{Not Robust for Non-Convex Problems:} Second-order methods, including BFGS and L-BFGS, can still struggle with saddle points and highly non-convex landscapes commonly encountered in deep learning.
\end{itemize}

\subsubsection{Applications of L-BFGS}
L-BFGS remains a valuable tool for optimization problems where:
\begin{itemize}
	\item The parameter space is not excessively high-dimensional.
	\item Precise curvature information is advantageous, such as in logistic regression or support vector machines.
	\item Fine-tuning is required near convergence to achieve high precision.
\end{itemize}

\subsection{Summary of Second-Order Optimization Approaches}
Second-order optimization methods, such as BFGS and L-BFGS, provide valuable insights into the curvature of the loss landscape, enabling adaptive step sizes and improved convergence rates. However, their computational and memory requirements make them impractical for large-scale machine learning problems. For such tasks, first-order methods like Adam remain the standard due to their scalability and effectiveness.




\chapterimage{head2.png} % Chapter heading image

% Chapter-specific content starts here
\chapter{Lecture 5: Neural Networks}

%----------------------------------------------------------------------------------------
%	CHAPTER 5 - Lecture 5: Neural Networks
%----------------------------------------------------------------------------------------

\chapterimage{head2.png} % Chapter heading image

% Chapter-specific content starts here
\chapter{Lecture 6: Backpropagation}

%----------------------------------------------------------------------------------------
%	CHAPTER 6 - Lecture 6: Backpropagation
%----------------------------------------------------------------------------------------

\section{Introduction: The Challenge of Computing Gradients}
\label{sec:introduction}
In the previous lecture, we explored how \textbf{neural networks} exploit \emph{non-linear space warping} to form complex decision boundaries—far surpassing the capabilities of linear classifiers. While activation functions like \textbf{ReLU} make this possible, they introduce a key question:

\begin{quote}
	\emph{How do we efficiently compute gradients for neural networks with millions or billions of parameters?}
\end{quote}

Traditional numerical or purely symbolic differentiation methods quickly become:
\begin{itemize}
	\item \textbf{Scalability Bottlenecks:} Deriving gradients manually does not scale to deep networks.
	\item \textbf{High Computational Complexity:} Efficient gradient computation is non-trivial at large scales.
	\item \textbf{Limited Modularity:} Any architectural modification (e.g., adding layers, changing the loss) requires recalculating derivatives from scratch.
\end{itemize}

\subsection{A Bad Idea: Manually Deriving Gradients}
\label{sec:manual-gradients}
\begin{figure}[H]
	\centering
	\includegraphics[width=0.8\textwidth]{figures/Chapter_6/slide_5.jpg}
	\caption{Manually deriving gradients for a simple linear classifier using the SVM loss. This process becomes infeasible for deep networks.}
	\label{fig:chapter6_manual_gradients}
\end{figure}

One naive approach is to \textbf{derive all gradients by hand}. As shown in Figure~\ref{fig:chapter6_manual_gradients}, it might be doable for basic linear models. Nevertheless, it also clearly demonstrates some of the problems described in the introduction (very tedious and error-prone), and mostly, it gives us a hint to how complex  this process will be for more complex models like neural networks, with many layers and millions up to billions of parameters. Hence, this approach is impractical, and we need to think of a better idea to compute gradients for the neural network optimization task.  

\subsection{A Better Idea: Utilizing Computational Graphs (Backpropagation)}
\label{sec:comp-graphs}
\begin{figure}[H]
	\centering
	\includegraphics[width=0.8\textwidth]{figures/Chapter_6/slide_6.jpg}
	\caption{Computational graphs provide a structured, automatic approach to computing gradients.}
	\label{fig:chapter6_comp_graphs}
\end{figure}

\newpage

A more robust solution is to represent computations as a \textbf{computational graph}:
\begin{itemize}
	\item Each node in the graph performs a specific operation (e.g., addition, matrix multiplication, or more complex operations like ReLU based on more basic primitives).
	\item Edges represent the flow of intermediate values (outputs of one operation become inputs to the next).
	\item \textbf{Backpropagation} can be used to automatically compute the gradient at each node by systematically applying the chain rule in reverse. Hence, each node in the graph only does simple local computations (derives the downstream gradients using a multiplication of the local gradient and the upstream gradient received from a following node). We'll see several examples of backpropagation later to understand what do we mean by that and what makes it important.
\end{itemize}

\paragraph{Why Use Computational Graphs?}
\begin{itemize}
	\item \textbf{Modularity:} Swap loss functions, activation layers, or architectures without manually re-deriving gradients.
	\item \textbf{Scalability:} Supports deep networks with millions of parameters, keeping computations local to each node.
	\item \textbf{Automation:} Greatly reduces human error and development time.
\end{itemize}

\section{Toy Example of Backpropagation: \texorpdfstring{$f(x,y,z) = (x + y)\,z$}{f(x,y,z)=(x+y)z}}
\label{sec:toy-example}
\begin{figure}[H]
	\centering
	\includegraphics[width=0.8\textwidth]{figures/Chapter_6/slide_26.jpg}
	\caption{Computational graph representation of \(f(x,y,z) = (x + y)z\), showing forward and backward passes. Intermediate values of the forward pass are presented in green on-top of the graph edges, while the corresponding backpropagation values are presented in red below the edges.}
	\label{fig:chapter6_example_fxyz}
\end{figure}

Consider a simple function:
\[
f(x,y,z) = (x + y)\,z,
\]
with \(x = -2\), \(y = 5\), and \(z = -4\). 

\newpage

\subsection{Forward Pass}
We traverse the graph from left to right:
\begin{itemize}
	\item \(\displaystyle q = x + y = -2 + 5 = 3.\)
	\item \(\displaystyle f = q \cdot z = 3 \times (-4) = -12.\)
\end{itemize}

\subsection{Backward Pass: Computing Gradients}
To find \(\frac{\partial f}{\partial x}, \frac{\partial f}{\partial y}, \frac{\partial f}{\partial z}\), we apply the chain rule:
\begin{itemize}
	\item \(\displaystyle \frac{\partial f}{\partial z} = q = 3.\)
	\item \(\displaystyle \frac{\partial f}{\partial q} = z = -4.\)
	\item \(\displaystyle \frac{\partial f}{\partial y} = \frac{\partial f}{\partial q} \times \frac{\partial q}{\partial y} = (-4) \times 1 = -4.\)
	\item \(\displaystyle \frac{\partial f}{\partial x} = \frac{\partial f}{\partial q} \times \frac{\partial q}{\partial x} = (-4) \times 1 = -4.\)
\end{itemize}

\section{Why Backpropagation?}
\subsection{Local \& Scalable Gradients Computation}
\label{subsec:local-upstream}
\begin{figure}[H]
	\centering
	\includegraphics[width=0.8\textwidth]{figures/Chapter_6/slide_35.jpg}
	\caption{During backpropagation, each node in the computational graph computes its downstream gradients using the local gradient (computed based on the local operation over the input. For instance, if we denote the input to the node as x and the node computes $\frac{1}{x}$, then the local gradient is $\frac{\partial}{\partial x}{[\frac{1}{x}]}=-\frac{1}{x^2}$) and the upstream gradient that is simply given as input from subsequent nodes.}
	\label{fig:chapter6_local_upstream}
\end{figure}

As we've seen in the above examples, within the computational graph, each node performs \textbf{only \emph{local}} gradient calculations. This is the core principal behind backpropagation, making it scalable and practical. Therefore, in order to make sure we clearly understand why, we'll zoom out and provide a high-level overview looking only at an abstract node in a computational graph independently.

\newpage

Suppose a node \(f\) outputs \(z\) from inputs \(x\) and \(y\).  
Given an \emph{upstream gradient} \(\frac{\partial L}{\partial z}\) (the partial derivative of the loss \(L\) w.r.t.\ \(z\)), the node only needs:
\begin{itemize}
	\item \(\displaystyle \frac{\partial z}{\partial x}\), 
	\item \(\displaystyle \frac{\partial z}{\partial y}\),
\end{itemize}
to compute:
\[
\frac{\partial L}{\partial x} \;=\; \frac{\partial z}{\partial x} \;\times\; \frac{\partial L}{\partial z}, 
\quad
\frac{\partial L}{\partial y} \;=\; \frac{\partial z}{\partial y} \;\times\; \frac{\partial L}{\partial z}.
\]

\begin{figure}[H]
	\centering
	\includegraphics[width=0.8\textwidth]{figures/Chapter_6/slide_31.jpg}
	\caption{Visualizing the backpropagation process for a node f that is given as inputs x, y and outputs z. As can be seen, the backpropagation gives us the upstream gradient from subsequent node in the graph, and we are only left with the local gradients computation: $\frac{\partial z}{\partial x}, \frac{\partial z}{\partial y}$ in order to compute the downstream gradients: $\frac{\partial L}{\partial x}, \frac{\partial L}{\partial y}$ allowing us to continue the graph traversal in the backpropagation process.}
	\label{fig:chapter6_indpendent_node}
\end{figure}

The \emph{local} multiplication by \(\frac{\partial L}{\partial z}\) (the upstream gradient) ensures each node can be implemented and debugged independently, enabling large-scale networks to remain tractable. This demonstrates the power and essence behind backpropagation for complex models, making it the go-to approach for gradients computation in neural networks optimization. 

\subsection{Pairing Backpropagation Gradients \& Optimizers is Easy}
While backpropagation efficiently provides the necessary gradients for each parameter, we still need \textbf{gradient-based optimizers} to use those gradients in \emph{gradient descent} updates. In practice, we pair backprop with methods like \textbf{Stochastic Gradient Descent (SGD)} or \textbf{AdamW}, which adjust parameters based on the gradients to minimize the loss. This synergy—automatic gradient computation via backprop, combined with iterative updates via gradient descent—enables neural networks to learn effectively from large and complex datasets.

\subsection{Modularity and Custom Nodes}
Because the computation graph decomposes into local operations, we can define \textbf{specialized nodes} for common functions. For example, a ``sigmoid node'' encapsulates the sigmoid function:
\[
\sigma(x) = \frac{1}{1 + e^{-x}},
\]
and uses the known derivative \(\sigma'(x) = \sigma(x)\bigl(1 - \sigma(x)\bigr)\) to backpropagate efficiently. This approach:
\begin{itemize}
	\item Simplifies the computational graph, reducing intermediate steps.
	\item Improves memory efficiency (fewer nodes, less storage for intermediate values).
	\item Allows us to treat the sigmoid as a single, optimized building block in our network, making the graph more semantically meaningful.
\end{itemize}
\label{sec:modularity-custom-nodes}
\begin{figure}[H]
	\centering
	\includegraphics[width=0.8\textwidth]{figures/Chapter_6/slide_45.jpg}
	\caption{A ``sigmoid node'' (in the blue rectangle) can replace multiple low-level operations (the intermediate nodes encapsulated within). Its known derivative simplifies backpropagation.}
	\label{fig:chapter6_sigmoid_node}
\end{figure}

\subsection{Utilizing Patterns in Gradient Flow}
\label{sec:gradient-flow-patterns}

Backpropagation is not just a mechanical process of computing derivatives; it follows structured \textbf{gradient flow patterns} that help us analyze and design computational graphs more effectively. By understanding these patterns, we can quickly construct computational graphs, debug gradient propagation issues, and optimize network structures. 

\subsection{Addition Gate: The Gradient Distributor}
The \textbf{add gate} acts as a \emph{gradient distributor} during the backward pass. When a function locally computes the output as the sum of its inputs, the local gradients for each input are simply 1. Thus, the \textbf{downstream gradient} for each input is equal to the \textbf{upstream gradient}, making it straightforward to propagate gradients backward. 

This pattern also provides an intuition about how gradient flow behaves in models that use addition operations, such as residual connections in deep networks which we'll extensively cover later.

\subsection{Copy Gate: The Gradient Adder}
The \textbf{copy gate} (or \emph{copy node}) is a trivial operation in the forward pass—it simply duplicates its input. However, it is useful when the same term appears in multiple parts of a computational graph. 

For instance, weight matrices are shared across different parts of a loss function:
\begin{itemize}
	\item In one path, they compute intermediate values such as \( h = Wx + b \).
	\item In another path, they contribute to the \textbf{regularization term} (e.g., L1/L2 regularization).
\end{itemize}
Since the weight matrix is reused, the backward pass must account for all gradient contributions. The copy gate \textbf{accumulates gradients} by summing all upstream gradients and passing the combined result as its downstream gradient.

Interestingly, the add and copy gates are \emph{dual} operations:
\begin{itemize}
	\item The forward operation of the \textbf{add gate} behaves like the backward operation of the \textbf{copy gate}.
	\item The forward operation of the \textbf{copy gate} behaves like the backward operation of the \textbf{add gate}.
\end{itemize}

\subsection{Multiplication Gate: The Gradient Swapper}
The \textbf{multiplication gate} (or \emph{mul gate}) swaps the roles of its inputs in the backward pass. For the function
\[
f(x,y) = x \cdot y,
\]
the local gradients are
\[
\frac{\partial f}{\partial x} = y, \quad \frac{\partial f}{\partial y} = x.
\]
Hence:
\begin{itemize}
	\item The downstream gradient for \(y\) is \(x\) times the upstream gradient,
	\item The downstream gradient for \(x\) is \(y\) times the upstream gradient.
\end{itemize}
\noindent
This \emph{mixing} of gradients can lead to:
\begin{itemize}
	\item \textbf{Exploding gradients}: Large products magnify updates, destabilizing training,
	\item \textbf{Vanishing gradients}: Small products reduce gradient magnitudes, slowing convergence.
\end{itemize}

\subsection{Max Gate: The Gradient Router}
The \textbf{max gate} selects the largest input value in the forward pass and routes the \textbf{entire} upstream gradient to that winning input in the backward pass. All other inputs receive zero gradient. While intuitive, this creates \textbf{gradient starvation}: only one path in the computational graph receives a nonzero gradient, potentially slowing learning.

\begin{figure}[H]
	\centering
	\includegraphics[width=0.8\textwidth]{figures/Chapter_6/slide_49.jpg}
	\caption{Visualization of gradient flow patterns: (1) Add gate—gradient distributor, (2) Copy gate—gradient adder, (3) Multiplication gate—gradient swapper, and (4) Max gate—gradient router.}
	\label{fig:chapter6_gradient_patterns}
\end{figure}


\section{Implementing Backpropagation in Code}
\label{sec:implementing-backprop}

Now that we understand gradient flow patterns, how can we \emph{implement} backpropagation in practice? One approach is to compute the \textbf{flat gradient code}. This method directly computes gradients step by step without leveraging modular APIs such as PyTorch's \texttt{autograd}. While simple, it lacks flexibility.

\begin{figure}[H]
	\centering
	\includegraphics[width=0.8\textwidth]{figures/Chapter_6/slide_52.jpg}
	\caption{A pseudo-implementation of forward and backward passes in a flat gradient backpropagation implementation.}
	\label{fig:chapter6_flat_backprop}
\end{figure}

\subsection{Flat Backpropagation: A Direct Approach}
The forward pass follows naturally from the computational graph, while the backward pass appears as a reversed version of it. The process begins with the \textbf{base case}: 
\[
\texttt{grad\_L} = 1.0
\]
(i.e., the gradient of the loss with respect to itself is 1), and then we propagate gradients in reverse order.

A step-by-step breakdown:
\begin{itemize}
	\item The loss is computed after applying a \textbf{sigmoid activation}, so we begin by computing the \emph{local gradient} of the sigmoid function:
	\[
	\frac{d}{dx} \sigma(x) = \sigma(x) \cdot (1 - \sigma(x)).
	\]
	Since \(\sigma(x)\) is the output of the sigmoid function (denoted as \(L\)), we get:
	\[
	\texttt{grad\_s3} = \texttt{grad\_L} \cdot (1 - L) \cdot L.
	\]
	
	\item The \textbf{add gate} distributes \texttt{grad\_s3} equally to its inputs, propagating gradients further back in the graph.
	
	\item The final two \textbf{mul gates} act as \emph{swapped multipliers}, computing gradients for each input based on the values of the other.
\end{itemize}

\paragraph{Why Flat Backpropagation Works Well.}
Despite its simplicity, this approach correctly computes gradients without manually deriving them. However, it has major limitations:
\begin{itemize}
	\item \textbf{Non-Modular:} Any change in the model or loss function requires rewriting the gradient code from scratch.
	\item \textbf{Hard to Scale:} Flat implementations do not easily extend to deep architectures with many layers.
\end{itemize}

\section{A More Modular Approach: Computational Graphs in Practice}
\label{sec:modular-backprop}

A more structured way to implement backpropagation is to use a \textbf{computational graph API}. Instead of manually coding backward passes, we represent the entire model as a graph structure that:
\begin{itemize}
	\item Stores \textbf{nodes} corresponding to computations.
	\item Automatically computes gradients by traversing the graph in reverse order.
	\item Allows modifications to loss functions and architectures without rewriting gradient code.
\end{itemize}

\begin{figure}[H]
	\centering
	\includegraphics[width=0.8\textwidth]{figures/Chapter_6/slide_60.jpg}
	\caption{API for a computational graph, requiring an implementation of both the forward and backward methods.}
	\label{fig:chapter6_computational_graph_api}
\end{figure}

\subsection{Topological Ordering in Computational Graphs}
A computational graph provides an efficient way to manage computations, enabling automatic differentiation and modularity. The graph structure follows a topological ordering, meaning:
\begin{itemize}
	\item Each node appears \textbf{before} all nodes dependent on it in the forward pass.
	\item In the backward pass, the nodes are traversed in the \textbf{reverse order}.
\end{itemize}
This ensures that gradients are properly propagated through the network, following the dependencies established in the forward pass.

\subsection{The API: Forward and Backward Methods}
A computational graph framework defines an API with two essential functions:
\begin{enumerate}
	\item \texttt{forward()}: Computes and stores intermediate values for later use in backpropagation.
	\item \texttt{backward()}: Applies the chain rule to compute gradients by traversing the graph in reverse.
\end{enumerate}

Many deep learning frameworks, such as PyTorch, TensorFlow, and JAX, implement automatic differentiation engines based on this principle. These frameworks eliminate the need for manually coding derivatives, making it easier to train deep models.

\subsection{Advantages of a Modular Computational Graph}
Using a computational graph for backpropagation offers several advantages over flat backpropagation:
\begin{itemize}
	\item \textbf{Modularity}: Changing the model architecture or loss function only requires modifying the forward function, and the backward function is computed automatically.
	\item \textbf{Scalability}: Works efficiently for deep networks with millions or billions of parameters.
	\item \textbf{Automatic Differentiation}: Frameworks can compute gradients dynamically, reducing the need for manual derivative calculations.
\end{itemize}

This modular approach enables modern deep learning frameworks to handle complex architectures efficiently while abstracting away the tedious details of gradient computation.

\section{Implementing Backpropagation with PyTorch Autograd}
\label{sec:pytorch-autograd}

A practical example of a modular approach to implementing backpropagation is PyTorch's \textbf{Autograd} engine. In PyTorch, functions that support automatic differentiation inherit from \\ \texttt{torch.autograd.Function}. These functions must implement two key static methods:
\begin{itemize}
	\item \textbf{Forward:} Computes the node’s output and stashes values required for the backward pass.
	\item \textbf{Backward:} Receives the upstream gradient and propagates it back using local derivatives.
\end{itemize}

\begin{figure}[H]
	\centering
	\includegraphics[width=0.8\textwidth]{figures/Chapter_6/slide_61.jpg}
	\caption{Example of PyTorch Autograd implementation for a multiplication gate (\( z = x \cdot y \)).}
	\label{fig:chapter6_autograd_multiplication}
\end{figure}

\subsection{Example: Multiplication Gate in Autograd}
Consider a simple multiplication operation \( z = x \cdot y \), implemented in PyTorch Autograd:
\begin{enumerate}
	\item The \texttt{forward} method computes \( z \) and saves \( x \) and \( y \) for the backward pass in a dedicated context variable \emph{ctx}.
	\item The \texttt{backward} method retrieves \( x \) and \( y \) as it receives the context from the forward pass, and the upstream gradient. It then applies the chain rule, and returns the downstream gradients:
	\[
	\frac{\partial L}{\partial x} = \frac{\partial L}{\partial z} \cdot y, \quad 
	\frac{\partial L}{\partial y} = \frac{\partial L}{\partial z} \cdot x.
	\]
\end{enumerate}
This pattern generalizes to all operations in a computational graph, allowing PyTorch to handle backpropagation automatically.

\subsection{Extending Autograd for Custom Functions}
Developers can extend PyTorch's Autograd by defining their own \texttt{Function} classes, implementing custom \texttt{forward} and \texttt{backward} methods. By doing so, new differentiable operations can be seamlessly integrated into neural network models.

\newpage 

An interesting example is PyTorch’s implementation of the \textbf{sigmoid activation function}, which follows this modular API:

\begin{figure}[H]
	\centering
	\includegraphics[width=0.8\textwidth]{figures/Chapter_6/slide_66.jpg}
	\caption{Example of PyTorch's \texttt{sigmoid} layer implementation with automatic differentiation.}
	\label{fig:chapter6_autograd_sigmoid}
\end{figure}

By chaining multiple such autograd functions, we construct computational graphs, enabling both \textbf{inference} (forward pass) and \textbf{training} (backward pass). This is the core essence of PyTorch.

\section{Beyond Scalars: Backpropagation for Vectors and Tensors}
\label{sec:vector-backprop}

So far, we have discussed backpropagation in the context of \textbf{scalar-valued functions}. However, real-world neural networks involve \textbf{vector-valued functions}, requiring us to extend backpropagation to \emph{gradients and Jacobians}.

\begin{figure}[H]
	\centering
	\includegraphics[width=0.8\textwidth]{figures/Chapter_6/slide_70.jpg}
	\caption{Recap of scalar derivatives, gradients, and Jacobians.}
	\label{fig:chapter6_jacobians}
\end{figure}

\subsection{Gradients vs.\ Jacobians}
\begin{itemize}
	\item \textbf{Gradient (Scalar-Valued Function):} 
	For a function \(f: \mathbb{R}^N \to \mathbb{R}\), the gradient is the vector of partial derivatives:
	\[
	\nabla f(\mathbf{x}) \;=\; \left[
	\frac{\partial f}{\partial x_1},\;
	\frac{\partial f}{\partial x_2},\;
	\dots,\;
	\frac{\partial f}{\partial x_N}
	\right].
	\]
	In a typical neural network setting, \(f\) might be a loss function \(\mathcal{L}(\theta)\) of the model parameters \(\theta\), and \(\nabla \mathcal{L}\) tells us how to update the parameters to minimize the loss.
	
	\item \textbf{Jacobian (Vector-Valued Function):}
	For a function \(F: \mathbb{R}^N \to \mathbb{R}^M\), the Jacobian is an \(M \times N\) matrix where each entry 
	\[
	J_{ij} \;=\; \frac{\partial F_i}{\partial x_j}
	\]
	represents how each input dimension \(x_j\) affects each output dimension \(F_i\). Rows of this matrix can be seen as gradients of individual output components. In neural networks, this arises very often when we compute local derivatives for a node in the graph, as if the inputs to the node are vectors x, y and the output is a vector z, and each vector has its own number of elements ($D_x, D_y, D_z$) then the local 'gradients' (derivatives) in this case are Jacobian matrices: dz/dx is of dim $[D_x \times D_z]$ and dz/dy is of dim $[D_y \times D_z]$. 
\end{itemize}

By understanding both gradients and Jacobians, we see why \textbf{backpropagation} must handle vector-valued outputs (e.g., a network’s logits or output features) rather than only scalar-valued loss functions. Modern frameworks automatically compute these matrix derivatives, enabling efficient training in multi-output scenarios. We'll now explore how the backpropagation process we've seen earlier generalizes to support such functions. 

\subsection{Extending Backpropagation to Vectors}
\label{sec:vector_backprop}

\begin{figure}[H]
	\centering
	\includegraphics[width=0.8\textwidth]{figures/Chapter_6/slide_74.jpg}
	\caption{A node \( f \) receiving two vectors
		\(\mathbf{x}\in\mathbb{R}^{D_x}\) and \(\mathbf{y}\in\mathbb{R}^{D_y}\) and producing
		\(\mathbf{z}\in\mathbb{R}^{D_z}\). We extend backpropagation to handle vector inputs and outputs.}
	\label{fig:chapter6_vector_backprop}
\end{figure}

In earlier sections, we focused on backpropagation when both inputs and outputs to a node were scalars. Real-world neural networks, however, typically process and produce \emph{vectors} or even higher-dimensional \emph{tensors}. This section provides a detailed look at how backpropagation extends to these more general scenarios.

We now move from a node that received scalars \(x, y\) and returned a scalar \(z\), to a node that handles:
\begin{itemize}
	\item \(\mathbf{x}\in\mathbb{R}^{D_x}\) and \(\mathbf{y}\in\mathbb{R}^{D_y}\) (input vectors),
	\item \(\mathbf{z}\in\mathbb{R}^{D_z}\) (output vector).
\end{itemize}

Although the output is now a vector, the overall \emph{loss function} \(L\) (e.g., a training objective) remains a scalar. Therefore:
\[
\underbrace{\frac{\partial L}{\partial \mathbf{z}}}_{\text{upstream gradient}}
\in \mathbb{R}^{D_z},
\]
tells us how sensitive \(L\) is to each component \(z_i\). The local gradients \(\tfrac{\partial \mathbf{z}}{\partial \mathbf{x}}\) and \(\tfrac{\partial \mathbf{z}}{\partial \mathbf{y}}\) become
\emph{Jacobian matrices}:
\[
\frac{\partial \mathbf{z}}{\partial \mathbf{x}} \;\in\; \mathbb{R}^{D_z \times D_x},
\quad
\frac{\partial \mathbf{z}}{\partial \mathbf{y}} \;\in\; \mathbb{R}^{D_z \times D_y}.
\]
By applying the chain rule, the \textbf{downstream gradients} \(\tfrac{\partial L}{\partial \mathbf{x}}\) and \(\tfrac{\partial L}{\partial \mathbf{y}}\) each become:
\[
\frac{\partial L}{\partial \mathbf{x}} 
= \left(\frac{\partial \mathbf{z}}{\partial \mathbf{x}}\right)^\mathsf{T}
\frac{\partial L}{\partial \mathbf{z}},
\qquad
\frac{\partial L}{\partial \mathbf{y}} 
= \left(\frac{\partial \mathbf{z}}{\partial \mathbf{y}}\right)^\mathsf{T}
\frac{\partial L}{\partial \mathbf{z}},
\]
where each result has the same dimension as the corresponding input vector (\(D_x\) or \(D_y\)).

\subsection{Example: Backpropagation for Elementwise ReLU}
\label{sec:relu_vector_backprop}

\begin{figure}[H]
	\centering
	\includegraphics[width=0.8\textwidth]{figures/Chapter_6/slide_80.jpg}
	\caption{Backprop through an elementwise ReLU node. Negative inputs
		produce zeros in the output (and zero gradients), while positive inputs pass gradients through.}
	\label{fig:chapter6_relu_backprop}
\end{figure}

Consider an elementwise \textbf{ReLU} applied to a vector \(\mathbf{x}\in \mathbb{R}^{4}\). For example:
\[
\mathbf{x} 
= \begin{bmatrix}1 \\ -2 \\ 3 \\ -1\end{bmatrix}
\quad\longrightarrow\quad
\mathbf{y} 
= \max\!\bigl(\mathbf{0},\mathbf{x}\bigr) 
= \begin{bmatrix}1 \\ 0 \\ 3 \\ 0\end{bmatrix}.
\]
If the upstream gradient (the sensitivity of the loss \(L\) w.r.t.\ each element of \(\mathbf{y}\)) is
\[
\frac{\partial L}{\partial \mathbf{y}}
= \begin{bmatrix}4 \\ -1 \\ 5 \\ 9\end{bmatrix},
\]
we must compute
\(\tfrac{\partial L}{\partial \mathbf{x}}\). Conceptually, the local Jacobian 
\(\tfrac{\partial \mathbf{y}}{\partial \mathbf{x}}\) is a \(4\times 4\) diagonal matrix:
\[
\begin{bmatrix}
	1 & 0 & 0 & 0\\
	0 & 0 & 0 & 0\\
	0 & 0 & 1 & 0\\
	0 & 0 & 0 & 0
\end{bmatrix},
\]
where the diagonal element is \(1\) if \(x_i>0\) and \(0\) otherwise. This multiplication yields:
\[
\frac{\partial L}{\partial \mathbf{x}}
= \begin{bmatrix}4 \\ 0 \\ 5 \\ 0\end{bmatrix}.
\]

\begin{figure}[H]
	\centering
	\includegraphics[width=0.8\textwidth]{figures/Chapter_6/slide_81.jpg}
	\caption{A more memory-efficient approach: do not form
		\(\tfrac{\partial \mathbf{y}}{\partial \mathbf{x}}\) explicitly. Instead, reuse the
		input mask (i.e., which elements of \(\mathbf{x}\) are positive).}
	\label{fig:chapter6_relu_implicit}
\end{figure}

\subsection{Efficient Computation via Local Gradient Slices}
\label{sec:implicit_jacobian}

High-dimensional neural network operations, such as matrix multiplications or convolutions, often have \emph{massive} Jacobians when viewed formally. Storing or iterating over all partial derivatives explicitly is impractical. Instead, we exploit \textbf{local gradient slices} to determine how each input component affects the output, then combine these slices via standard matrix multiplications.

\subsection{Backpropagation with Matrices: A Concrete Example}
\label{sec:gradient_slices}

Consider a matrix multiplication:
\[
\mathbf{Y} = \mathbf{X}\,\mathbf{W},
\]
where \(\mathbf{X}\in\mathbb{R}^{N\times D}\), \(\mathbf{W}\in \mathbb{R}^{D\times M}\), and \(\mathbf{Y}\in\mathbb{R}^{N\times M}\). We have a scalar loss \(L\), and the \emph{upstream gradient} \(\tfrac{\partial L}{\partial \mathbf{Y}}\) is also an \((N\times M)\)-shaped matrix.

\paragraph{Numerical Setup.}
Let \(N=2\), \(D=3\), and \(M=4\). Suppose:
\[
\mathbf{X} = 
\begin{bmatrix}
	2 & 1 & -3 \\
	-3 & 4 & 2
\end{bmatrix},
\quad
\mathbf{W} =
\begin{bmatrix}
	3 & 2 & 1 & -1 \\
	2 & 1 & 3 & 2  \\
	3 & 2 & 1 & -2 
\end{bmatrix}.
\]
Multiplying gives \(\mathbf{Y}\in \mathbb{R}^{2\times 4}\). Concretely,
\[
\mathbf{Y} = \mathbf{X}\,\mathbf{W}
=
\begin{bmatrix}
	-1 & -1 & 2 & 6 \\
	5  &  2 & 11 & 7
\end{bmatrix}.
\]
In the backward pass, we receive:
\[
\frac{\partial L}{\partial \mathbf{Y}}
=
\begin{bmatrix}
	2 & 3 & -3 & 9 \\
	-8 & 1 & 4 & 6
\end{bmatrix}.
\]

\begin{figure}[H]
	\centering
	\includegraphics[width=0.8\textwidth]{figures/Chapter_6/slide_98.jpg}
	\caption{%
		Computing a ``gradient slice'' for a single element \(\mathbf{X}_{i,j}\).
		Rather than storing the entire local Jacobian, we only determine how
		\(\mathbf{X}_{i,j}\) influences each output element of \(\mathbf{Y}\),
		then combine that slice with the relevant elements of
		\(\tfrac{\partial L}{\partial \mathbf{Y}}\).
	}
	\label{fig:chapter6_gradient_slice}
\end{figure}

\paragraph{Slice Logic for One Input Element.}
Focusing on a single entry, e.g.\ \(\mathbf{X}_{1,1}=2\):
\begin{itemize}
	\item Only row \(1\) of \(\mathbf{Y}\) depends on \(\mathbf{X}_{1,1}\).
	\item Each element \(y_{1,m}\) is updated by \(x_{1,1}\cdot w_{1,m}\).
	\item The second row \(y_{2,\cdot}\) is unaffected, so its local gradient is zero.
\end{itemize}
Hence, for \(\mathbf{X}_{1,1}\), the \emph{local gradient slice} 
\(\tfrac{\partial \mathbf{Y}}{\partial x_{1,1}}\) has the form:
\[
\begin{bmatrix}
	w_{1,1} & w_{1,2} & w_{1,3} & w_{1,4}\\
	0 & 0 & 0 & 0
\end{bmatrix}.
\]
Next, we take the elementwise product with \(\tfrac{\partial L}{\partial \mathbf{Y}}\) over the same region to compute
\(\tfrac{\partial L}{\partial x_{1,1}}\). Numerically:
\[
w_{1,\cdot} = [\,3,\;2,\;1,\;-1\,],
\quad
\frac{\partial L}{\partial y_{1,\cdot}}
= [\,2,\;3,\;-3,\;9\,].
\]
So
\[
\tfrac{\partial L}{\partial x_{1,1}}
= 3\times2 + 2\times3 + 1\times(-3) + (-1)\times9
= 0.
\]

\begin{figure}[H]
	\centering
	\includegraphics[width=0.8\textwidth]{figures/Chapter_6/slide_102.jpg}
	\caption{%
		Another view of the slice approach for \(\mathbf{X}_{1,1}\). 
		Only the first row of \(\mathbf{Y}\) receives a nonzero local gradient
		from this input element.
	}
	\label{fig:chapter6_gradient_first}
\end{figure}

\paragraph{Another Example: \(\mathbf{X}_{2,3}\).}
For \(\mathbf{X}_{2,3}=2\):
\begin{itemize}
	\item Only row \(2\) of \(\mathbf{Y}\) depends on it.
	\item Each element \(y_{2,m}\) is updated by \((x_{2,3}\cdot w_{3,m})\).
	\item The first row is unaffected (local gradient is zero).
\end{itemize}
If \(w_{3,\cdot} = [\,3,\;2,\;1,\;-2\,]\), then
\[
\tfrac{\partial L}{\partial x_{2,3}}
= 3\times(-8)
+ 2\times1
+ 1\times4
+ (-2)\times6
= -24 + 2 + 4 - 12
= -30.
\]

\begin{figure}[H]
	\centering
	\includegraphics[width=0.8\textwidth]{figures/Chapter_6/slide_104.jpg}
	\caption{%
		Similarly, \(\mathbf{X}_{2,3}\) affects only the second row of \(\mathbf{Y}\).
		By repeating this logic for all elements, we derive the standard
		matrix-multiplication backprop formulas.
	}
	\label{fig:chapter6_gradient_last}
\end{figure}

\subsection{Implicit Multiplication for the Entire Gradient}
\label{sec:implicit_mult}

While the slice-by-slice perspective shows how each input entry influences each output entry, in practice we \emph{combine all slices at once} with matrix multiplication. Concretely:
\[
\frac{\partial L}{\partial \mathbf{X}}
=
\left(\frac{\partial L}{\partial \mathbf{Y}}\right)
\,\mathbf{W}^\mathsf{T},
\quad
\frac{\partial L}{\partial \mathbf{W}}
=
\mathbf{X}^\mathsf{T}
\,\left(\frac{\partial L}{\partial \mathbf{Y}}\right).
\]
These expressions produce the exact same result as summing the contributions from each local slice individually—\textbf{without} constructing the full \(((N\times D)\times(N\times M))\) Jacobian.

\begin{figure}[H]
	\centering
	\includegraphics[width=0.8\textwidth]{figures/Chapter_6/slide_107.jpg}
	\caption{%
		By using vector/matrix multiplications and slicing logic,
		we avoid forming massive Jacobians in memory.
	}
	\label{fig:chapter6_matrix_implicit}
\end{figure}

\paragraph{Why Slices Are the Solution.}
\begin{itemize}
	\item \textbf{Memory Savings:} Instead of building a giant Jacobian, we focus on local slices (each shaped like \(\mathbf{Y}\), and effectively, a row of \(\mathbf{Y}\) that is not 0s) and multiply them with the matching row of \(\tfrac{\partial L}{\partial \mathbf{Y}}\) to get the corresponding element of the downstream gradient. The local slices can be discarded each time after we finish the computation of the corresponding element of the downstream gradient.  
	\item \textbf{Efficiency:} In practice, we skip per-element slicing entirely and jump to 
	\(\bigl(\tfrac{\partial L}{\partial \mathbf{Y}}\bigr)\,\mathbf{W}^\mathsf{T}\) and 
	\(\mathbf{X}^\mathsf{T}\,\bigl(\tfrac{\partial L}{\partial \mathbf{Y}}\bigr)\)
	using fast BLAS/GPU kernels.
	\item \textbf{Scalability to High Dimensional Tensors:} For high-rank tensors (e.g., images), the same principle applies. We typically flatten or reshape dimensions to perform the relevant multiplications in a similarly efficient manner.
\end{itemize}

This \emph{implicit} backprop approach avoids the exponential growth of explicit Jacobian storage, making gradient-based learning feasible even for large-scale neural networks.

\subsection{A Chain View of Backpropagation}
\label{sec:chain_view_backprop}

Another way to understand backpropagation is to see the computational graph as a chain of functions operating on intermediate variables. Suppose we have:
\[
x_1 = f_1(x_0), 
\quad 
x_2 = f_2(x_1), 
\quad 
\dots, 
\quad 
L = f_4(x_3),
\]
where \( f_4 \) outputs the scalar loss \( L \). This chain perspective is especially useful when exploring different modes of automatic differentiation.

\subsubsection{Reverse-Mode Automatic Differentiation}
\label{sec:reverse_mode_ad}

\begin{figure}[H]
	\centering
	\includegraphics[width=0.8\textwidth]{figures/Chapter_6/slide_111.jpg}
	\caption{%
		Reverse-mode automatic differentiation can exploit the associativity of matrix multiplication to replace potentially expensive matrix-matrix products with matrix-vector products, moving from right to left.
	}
	\label{fig:chapter6_reverse_mode}
\end{figure}

\newpage

When a scalar loss \(L\) appears at the end of the computational graph, \emph{reverse-mode} automatic differentiation efficiently calculates gradients with respect to a large number of parameters. By traversing the chain from \(L\) backward, matrix-vector products replace matrix-matrix products, which is much more efficient for high-dimensional problems.

\subsubsection{Forward-Mode Automatic Differentiation}
\label{sec:forward_mode_ad}

\begin{figure}[H]
	\centering
	\includegraphics[width=0.8\textwidth]{figures/Chapter_6/slide_114.jpg}
	\caption{%
		Forward-mode automatic differentiation is useful for computing the derivatives of scalar inputs with respect to multiple outputs. While not commonly used in deep learning, it is widely applied in physics simulations and sensitivity analysis.
	}
	\label{fig:chapter6_forward_mode}
\end{figure}

In contrast to reverse-mode automatic differentiation, which is optimized for computing derivatives of a scalar loss with respect to many parameters, forward-mode automatic differentiation is more suitable when:
\begin{itemize}
	\item A single scalar input (or a few inputs) affects many outputs, and we need derivatives of all outputs with respect to this input.
	\item The computational graph is narrow but deep (e.g., computing derivatives with respect to time-dependent variables in simulations).
\end{itemize}

\paragraph{When Is Forward-Mode Automatic Differentiation is Useful?}
While forward-mode differentiation is rarely used in deep learning, it plays a crucial role in other scientific and engineering domains, including:
\begin{itemize}
	\item Physics Simulations: Understanding how changes in fundamental constants (e.g., gravity, friction) affect entire system dynamics.
	\item Sensitivity Analysis: Evaluating how small variations in input parameters propagate through a model, which is essential for robust system design.
	\item Computational Finance and Engineering: Where derivative calculations are needed for risk modeling and structural analysis.
\end{itemize}

Unlike reverse-mode differentiation, which propagates gradients backward, forward-mode propagates derivatives forward through the computation graph, making it efficient for computing derivatives with respect to a few key inputs.

\subsection{Computing Higher-Order Derivatives with Backpropagation}
\label{sec:higher_order_backprop}

\begin{figure}[H]
	\centering
	\includegraphics[width=0.8\textwidth]{figures/Chapter_6/slide_120.jpg}
	\caption{%
		Using backpropagation to compute Hessian-vector products as an efficient way to obtain second-order derivatives.
	}
	\label{fig:chapter6_hessian_backprop}
\end{figure}

So far, we have focused on first-order derivatives, which capture how small changes in parameters affect the loss function. However, higher-order derivatives, such as Hessians (matrices of second-order partial derivatives), provide additional insights.

\paragraph{Why Compute Hessians?}
The Hessian matrix \( \mathbf{H} = \nabla^2 L \) captures second-order effects, helping in:
\begin{itemize}
	\item Second-Order Optimization: Methods like Newton’s method and quasi-Newton methods (e.g., L-BFGS) use Hessian information for faster convergence.
	\item Understanding Model Sensitivity: Hessians quantify how different parameters interact and influence optimization.
	\item Regularization and Pruning: Hessian-based techniques help in feature selection and gradient-based network pruning.
\end{itemize}

\paragraph{Efficient Hessian Computation: Hessian-Vector Products}
A naive approach to computing the Hessian matrix explicitly is infeasible in large-scale models (as it requires storing an \( \mathbb{R}^{N \times N} \) matrix). Instead, we use Hessian-vector multiplication:
\[
\mathbf{H} \mathbf{v} = \nabla \left( \nabla L \cdot \mathbf{v} \right),
\]
which allows us to obtain second-order information efficiently without forming the full Hessian matrix.

\subsection{Application: Gradient-Norm Regularization}
\label{sec:higher_order_regularization}

\begin{figure}[H]
	\centering
	\includegraphics[width=0.8\textwidth]{figures/Chapter_6/slide_121.jpg}
	\caption{%
		An example of using second-order derivatives: Regularizing the gradient norm to improve optimization stability.
	}
	\label{fig:chapter6_gradient_norm_reg}
\end{figure}

One practical application of higher-order derivatives in deep learning is penalizing the gradient norm:
\[
R(W) = \|\nabla_W L\|_2^2.
\]
Computing \( \tfrac{d}{d W}R(W) \) involves second derivatives of \( L \). With backpropagation for higher-order terms, we can approximate or compute this regularization effectively, potentially improving model training by smoothing out rugged loss landscapes.

\subsection{Automatic Differentiation: Summary of Key Insights}
Automatic differentiation (AD) provides a unified framework for efficiently computing derivatives of complex functions expressed as computational graphs.  
Backpropagation, as used in deep learning, is a specific instance of this framework—corresponding to \textbf{reverse-mode automatic differentiation}.

\begin{itemize}
	\item \textbf{Reverse-mode automatic differentiation (backpropagation).}  
	In deep learning, we typically minimize a scalar loss with respect to millions of parameters.  
	Reverse-mode AD propagates sensitivities backward through the computational graph, allowing all partial derivatives of a scalar output to be computed in a single backward pass. This makes it the method of choice for large-scale neural network training.
	
	\item \textbf{Forward-mode automatic differentiation.}  
	Forward-mode AD pushes derivatives forward from the inputs instead of backward from the output.  
	It is most efficient when there are few input variables but many outputs—such as in scientific computing, sensitivity analysis, and physical simulations—where the goal is to measure how a single parameter affects multiple outcomes.
	
	\item \textbf{Higher-order derivatives.}  
	The same computational principles that underlie backpropagation can be extended to compute higher-order derivatives, including Hessians and Hessian–vector products.  
	These quantities capture curvature information, enabling more refined optimization strategies.
	
	\item \textbf{Second-order methods and regularization.}  
	Hessian-based quantities support advanced techniques such as second-order optimization (e.g., Newton’s method) and curvature-aware regularization, which can improve convergence speed and stability in challenging loss landscapes.
\end{itemize}

\noindent Taken together, these insights highlight that backpropagation is not merely a training algorithm, but a powerful instance of the broader concept of automatic differentiation.  
The same underlying ideas—propagating derivatives through computational graphs—form the mathematical foundation for a wide spectrum of applications, from deep learning optimization to scientific modeling and engineering simulations.


\chapterimage{head2.png} % Chapter heading image

% Chapter-specific content starts here
\chapter{Lecture 7: Convolutional Networks}

%----------------------------------------------------------------------------------------
%	CHAPTER 7 - Lecture 7: Convolutional Networks
%----------------------------------------------------------------------------------------

\chapterimage{head2.png} % Chapter heading image

% Chapter-specific content starts here
\chapter{Lecture 8: CNN Architectures I}

%----------------------------------------------------------------------------------------
%	CHAPTER 8 - Lecture 8: CNN Architectures I
%----------------------------------------------------------------------------------------

\section{Introduction: From Building Blocks to SOTA CNNs}

Convolutional Neural Networks (CNNs) have revolutionized computer vision by providing state-of-the-art results in image classification, object detection, and many other tasks. While previous chapters introduced the core building blocks of CNNs—convolutional layers, activation functions, normalization techniques, and pooling layers—the question remains: \emph{how do we structure these components into effective architectures?}

This chapter explores the historical progression of CNN architectures, focusing on key models that have shaped modern deep learning. We ground our discussion in the \emph{ImageNet Large Scale Visual Recognition Challenge (ILSVRC)}, which has served as a driving force for innovation in deep learning-based image classification.

\section{AlexNet}

In 2012, a breakthrough in the field of computer vision—and the winner of the ImageNet classification challenge—was \textbf{AlexNet} \cite{krizhevsky2012_alexnet}. 
While modern architectures are significantly deeper, AlexNet marked the beginning of deep convolutional networks. It accepted input images of spatial dimensions \(227 \times 227\), with three color channels, leading to an input tensor shape of \(3 \times 227 \times 227\) per image, and a total input batch shape of \(N \times 3 \times 227 \times 227\).

AlexNet consisted of:
\begin{itemize}
	\item \textbf{Five convolutional layers}, interleaved with max pooling layers.
	\item \textbf{Three fully connected layers}, finalizing the classification with a softmax output.
	\item \textbf{ReLU non-linearity}, one of the first architectures to introduce it.
	\item \textbf{Local Response Normalization (LRN)}, a now obsolete normalization technique used before BatchNorm.
	\item \textbf{Multi-GPU training}, splitting the model across two NVIDIA GTX 580 GPUs, each with only 3GB of memory.
\end{itemize}

Despite its relatively simple design, AlexNet was hugely influential, accumulating over 100,000 citations since its publication. This makes it one of the most cited works in modern science, surpassing even landmark research in information theory and fundamental physics. 

\subsection{Architecture Details}

Let us analyze AlexNet layer by layer, focusing on output dimensions, memory consumption, and computational cost.

\paragraph{First Convolutional Layer (Conv1)}
The first convolutional layer has \(C_{\text{out}} = 64\) filters, kernel size \(K = 11 \times 11\), stride \(S = 4\), and padding \(P = 2\). The output spatial size is computed as:

\begin{equation}
	W' = \frac{W - K + 2P}{S} + 1 = \frac{227 - 11 + 2(2)}{4} + 1 = 56
\end{equation}

Thus, the output tensor shape is \(64 \times 56 \times 56\).

\paragraph{Memory Requirements}
Assuming 32-bit floating-point representation (4 bytes per element), the output tensor storage requirement is:

\begin{equation}
	\frac{(C_{\text{out}} \times H' \times W') \times 4}{1024} = \frac{(64 \times 56 \times 56) \times 4}{1024} = 784 \text{KB}
\end{equation}

\paragraph{Number of Learnable Parameters}
The weight tensor shape is \(C_{\text{out}} \times C_{\text{in}} \times K \times K = 64 \times 3 \times 11 \times 11\), with an additional bias term per channel:

\begin{equation}
	\text{Total Params} = (64 \times 3 \times 11 \times 11) + 64 = 23,296
\end{equation}

\paragraph{Computational Cost}
Each output element requires a convolution with a \(C_{\text{in}} \times K \times K\) receptive field, leading to the following Multiply-Accumulate operations (MACs):

\begin{equation}
	\text{\#MACs} = (C_{\text{out}} \times H' \times W') \times (C_{\text{in}} \times K \times K)
\end{equation}

\begin{equation}
	= (64 \times 56 \times 56) \times (3 \times 11 \times 11) = 72,855,552 \approx 78M \text{ MACs}
\end{equation}

Note: In practice, 1 MAC = 2 FLOPs, since each multiply-accumulate consists of both a multiplication and an addition.

\paragraph{Max Pooling Layer}
The first pooling layer follows the ReLU activation and has a \(3 \times 3\) kernel with stride \(S = 2\), reducing the spatial size:

\begin{equation}
	W' = \lfloor (W - K) / S + 1 \rfloor = \lfloor (56 - 3) / 2 + 1 \rfloor = 27
\end{equation}

Thus, the output tensor shape is \(64 \times 27 \times 27\).

\paragraph{Memory and Computational Cost}
\begin{itemize}
	\item \textbf{Memory:} \(64 \times 27 \times 27 \times 4 / 1024 = 182.25\) KB.
	\item \textbf{MACs:} Since pooling takes only a maximum over a \(3 \times 3\) window, the cost is:
	
	\begin{equation}
		(C_{\text{out}} \times H' \times W') \times (K \times K) = (64 \times 27 \times 27) \times (3 \times 3) = 0.4M \text{ MACs}.
	\end{equation}
\end{itemize}

Max pooling is computationally inexpensive compared to convolutions.

\subsection{Final Fully Connected Layers}
The final three layers form a Multi-Layer Perceptron (MLP):

\begin{itemize}
	\item \textbf{Flatten Layer:} Flattens the \(256 \times 6 \times 6\) tensor into a 9216-dimensional vector.
	\item \textbf{First FC Layer:} Maps 9216 to 4096 neurons.
	\item \textbf{Second FC Layer:} Maps 4096 to another 4096 neurons.
	\item \textbf{Final FC Layer:} Maps 4096 to 1000 output classes (ImageNet categories).
\end{itemize}

\paragraph{Computational Cost}
For the first fully connected layer:

\begin{equation}
	\text{FC Params} = (C_{\text{in}} \times C_{\text{out}}) + C_{\text{out}}
\end{equation}

\begin{equation}
	= (9216 \times 4096) + 4096 = 37,725,832
\end{equation}

\begin{equation}
	\text{\#MACs} = 9216 \times 4096 = 37,748,736.
\end{equation}

The process continues for the other FC layers, culminating in a final output of 1000 neurons for classification.

\begin{figure}[H]
	\centering
	\includegraphics[width=0.8\textwidth]{Figures/Chapter_8/slide_46.jpg}
	\caption{The AlexNet architecture, including a table summarizing memory, parameters, and FLOPs per layer.}
	\label{fig:alexnet_architecture}
\end{figure}

\subsection{Key Takeaways from AlexNet}
\begin{itemize}
	\item The memory footprint is largest in the early convolutional layers.
	\item Nearly all parameters are stored in the fully connected layers.
	\item Most computational cost (FLOPs) occurs in convolutional layers.
\end{itemize}

\begin{figure}[H]
	\centering
	\includegraphics[width=0.8\textwidth]{Figures/Chapter_8/slide_49.jpg}
	\caption{Trends in AlexNet: memory usage in early conv layers, parameter-heavy FC layers, and computational cost concentrated in convolutions.}
	\label{fig:alexnet_trends}
\end{figure}

\subsection{ZFNet: An Improvement on AlexNet}

In 2013, most competitors in the ImageNet challenge used CNNs following AlexNet’s success. The winner, \textbf{ZFNet} \cite{zeiler2014_visualizing}, was essentially a refined, larger version of AlexNet.

\begin{figure}[H]
	\centering
	\includegraphics[width=0.8\textwidth]{Figures/Chapter_8/slide_52.jpg}
	\caption{The ZFNet architecture and its improvements over AlexNet.}
	\label{fig:zfnet_architecture}
\end{figure}

\subsubsection{Key Modifications in ZFNet}
\begin{itemize}
	\item The first convolutional layer was adjusted to use a \(7 \times 7\) kernel with stride 2, instead of \(11 \times 11\) with stride 4 in AlexNet. This resulted in finer spatial resolution in early layers.
	\item Increased number of parameters and computation, leading to improved performance.
\end{itemize}

\noindent The main lesson from AlexNet and ZFNet: \textbf{larger networks tend to perform better, but architecture refinement is critical.}

\section{VGG: A Principled CNN Architecture}
\label{sec:vgg_architecture}

\paragraph{Historical Context.}
Proposed in the 2014 ImageNet challenge by Oxford’s \emph{Visual Geometry Group} (\cite{simonyan2014_vgg}), \textbf{VGG} demonstrated the power of systematically deepening CNNs. In contrast to ad-hoc predecessor designs like AlexNet, VGG introduced a uniform blueprint for increasing depth using only small kernels and structured downsampling.

\begin{figure}[H]
	\centering
	\includegraphics[width=0.8\textwidth]{figures/Chapter_8/slide_63.jpg}
	\caption{Comparison of AlexNet vs.\ VGG: model size, parameter count, and FLOPs.}
	\label{fig:slide_63_vgg_alexnet}
\end{figure}

\paragraph{Core Design Principles.}
VGG’s architecture rests on three simple rules:
\begin{enumerate}
	\item All convolutions are \(\,3\times3\), stride=1, pad=1.
	\item All pooling is \(\,2\times2\) max-pool with stride=2.
	\item After each pool, the number of channels doubles.
\end{enumerate}
These guidelines enabled much deeper networks than earlier CNNs, yet kept computations relatively manageable.

\subsection{Network Structure}
Five hierarchical \emph{stages} group VGG’s convolutional layers:
\begin{itemize}
	\item \textbf{Stages 1--3}: \([\,\mathrm{conv}\!-\!\mathrm{conv}\!-\!\mathrm{pool}\,]\).
	\item \textbf{Stages 4--5}: \([\,\mathrm{conv}\!-\!\mathrm{conv}\!-\!\mathrm{conv}\!-\![\mathrm{conv}]\!-\!\mathrm{pool}\,]\).
\end{itemize}
Popular variants are:
\begin{itemize}
	\item \textbf{VGG-16} with 16 total convolutional layers,
	\item \textbf{VGG-19} with 19 layers (extra conv in stages 4,5).
\end{itemize}

\begin{figure}[H]
	\centering
	\includegraphics[width=0.8\textwidth]{figures/Chapter_8/slide_56.jpg}
	\caption{AlexNet vs.\ VGG-16 and VGG-19, highlighting VGG’s deeper, more uniform design. (Slide~\ref{fig:slide_56_vgg_alexnet})}
	\label{fig:slide_56_vgg_alexnet}
\end{figure}

\subsection{Key Architectural Insights}

\subsubsection{Small-Kernel Convolutions (\(3\times3\))}
VGG replaces larger kernels (e.g.\ \(5\times5\), \(7\times7\)) with multiple \((3\times3)\) layers in sequence:
\begin{itemize}
	\item \textbf{Fewer Parameters:} 
	A \(5\times5\) layer (C\(\rightarrow\)C) needs \(25C^2\) params vs.\ \((2\times 3\times3)=18C^2\) for two \((3\times3)\) layers—saving \(\sim28\%\).
	\item \textbf{Fewer FLOPs:} A single \(5\times5\) convolution requires \(25C^2HW\) MACs, whereas two stacked \(3\times3\) layers require only \(18C^2HW\) MACs, reducing the computational cost significantly.
	\item \textbf{Additional Non-Linearities:}
	Each \((3\times3)\) block adds an extra \texttt{ReLU}, enhancing representational power.
	\item \textbf{Equivalent Receptive Field:}
	Stacked \((3\times3)\) kernels can mimic a \(5\times5\) or \(7\times7\) receptive field with less cost.
\end{itemize}

\subsubsection{Pooling \(\,2\times2\), Stride=2, No Padding}
Each max-pool halves the spatial resolution. This systematically shrinks \((H\times W)\) by a factor of 2 at each stage, reducing compute in subsequent conv layers while retaining key feature activations.

\subsubsection{Doubling Channels After Each Pool}
Every time \((H\times W)\) halves, VGG doubles the channel dimension:
\begin{itemize}
	\item \textbf{Keeps Compute Balanced:}
	Halving spatial size cuts the feature map area by \(\tfrac{1}{4}\). Doubling channels multiplies it by 2, netting an overall consistent computational load.
	\item \textbf{Deep Hierarchical Features:}
	As resolution shrinks, more channels capture increasingly complex patterns.
\end{itemize}

\subsection{Why This Strategy Works}
\paragraph{Balanced Computation.}
Downsampling by \(\tfrac{1}{2}\) in height/width decreases memory usage fourfold, while doubling channels boosts parameter usage. These changes roughly offset, so deeper stages keep a similar cost to earlier ones.

\paragraph{Influence on Later Architectures.}
ResNet, DenseNet, and other modern CNNs commonly adopt the \emph{“halve spatial dimension, double channels”} approach, ensuring that \emph{even as networks grow deeper,} no single layer becomes exorbitantly expensive.

\subsection{Practical Observations}
\begin{itemize}
	\item \textbf{Depth over Large Kernels:} Multiple small convs outperform fewer large-kernel layers, enabling higher nonlinearity and fewer parameters.
	\item \textbf{Uniform Design Eases Scaling:} A consistent set of kernel and pooling choices fosters more predictable performance and simpler scaling options.
	\item \textbf{Increased Memory \& FLOPs:} VGG’s deeper nature raises parameter counts and compute demands, making it significantly—less suited to edge devices and real-time applications.
\end{itemize}

\noindent
Despite higher resource usage, VGG’s straightforward, principled design pioneered deeper networks and influenced countless subsequent CNN architectures (e.g., \emph{ResNet}, \emph{EfficientNet}) that build on its core ideas and refine efficiency.



\chapterimage{head2.png} % Chapter heading image

% Chapter-specific content starts here
\chapter{Lecture 9: Training Neural Networks I}

%----------------------------------------------------------------------------------------
%	CHAPTER 9 - Lecture 9: Training Neural Networks I
%----------------------------------------------------------------------------------------

\chapterimage{head2.png} % Chapter heading image

% Chapter-specific content starts here
\chapter{Lecture 10: Training Neural Networks II}

%----------------------------------------------------------------------------------------
%	CHAPTER 10 - Lecture 10: Training Neural Networks II
%----------------------------------------------------------------------------------------

\chapterimage{head2.png} % Chapter heading image

% Chapter-specific content starts here
\chapter{Lecture 11: CNN Architectures II}

%----------------------------------------------------------------------------------------
%	CHAPTER 11 - Lecture 11: CNN Architectures II
%----------------------------------------------------------------------------------------

\chapterimage{head2.png} % Chapter heading image

% Chapter-specific content starts here
\chapter{Lecture 12: Deep Learning Software}

%----------------------------------------------------------------------------------------
%    CHAPTER 12 - Lecture 12: Deep Learning Software
%----------------------------------------------------------------------------------------

\section{Deep Learning Frameworks: Evolution and Landscape}
\label{sec:chapter12_frameworks}

Deep learning software frameworks enable researchers and engineers to efficiently prototype, train, and deploy neural networks. This chapter explores key frameworks, their underlying computational structures, and comparisons between static and dynamic computation graphs. Each framework is providing different trade-offs between usability, performance, and scalability. 

\begin{figure}[H]
    \centering
    \includegraphics[width=0.8\textwidth]{Figures/Chapter_12/slide_4.jpg}
    \caption{Overview of major deep learning frameworks and their affiliations.}
    \label{fig:chapter12_frameworks}
\end{figure}

Some notable frameworks include:

\begin{itemize}
    \item \textbf{Caffe} (UC Berkeley) – One of the earliest frameworks, optimized for speed but limited in flexibility.
    \item \textbf{Theano} (U. Montreal) – A pioneer in automatic differentiation, but now discontinued.
    \item \textbf{TensorFlow} (Google) – Popular for production deployments; originally focused on static computation graphs.
    \item \textbf{PyTorch} (Facebook) – An imperative, Pythonic framework with dynamic computation graphs, widely used in research.
    \item \textbf{MXNet} (Amazon) – Developed by multiple institutions, designed for distributed deep learning.
    \item \textbf{JAX} (Google) – A newer framework optimized for high-performance computing and auto-differentiation.
\end{itemize}

While many frameworks exist, \textcolor{red}{\textbf{PyTorch and TensorFlow}} dominate deep learning research and deployment. The following sections explore these frameworks in detail, starting with computational graphs and automatic differentiation.

\subsection{The Purpose of Deep Learning Frameworks}
\label{subsec:chapter12_purpose}  

Deep learning frameworks provide essential tools that simplify the implementation, training, and deployment of neural networks. They abstract away low-level operations, enabling users to focus on model design and experimentation rather than manual gradient computations or hardware-specific optimizations. The three primary goals of deep learning frameworks are:  

\begin{itemize}  
    \item \textbf{Rapid Prototyping:} Frameworks allow researchers to quickly experiment with new architectures, optimization techniques, and data pipelines. High-level APIs simplify model definition, while flexible debugging tools enable faster iteration.  
    \item \textbf{Automatic Differentiation:} Modern frameworks automatically compute gradients via backpropagation, eliminating the need for manual derivative calculations. This accelerates research and reduces implementation errors.  
    \item \textbf{Efficient Execution on Hardware:} Frameworks optimize computations for GPUs \& TPUs, leveraging parallel processing and efficient memory management to accelerate training and inference.  
\end{itemize}  

\subsection{Recall: Computational Graphs}  
\label{subsubsec:chapter12_computational_graphs}  

\begin{figure}[H]  
    \centering  
    \includegraphics[width=0.8\textwidth]{Figures/Chapter_12/slide_5.jpg}  
    \caption{\textbf{Computational graphs in deep learning.} These graphs define the sequence of operations for training and inference, enabling automatic differentiation and optimization.}  
    \label{fig:chapter12_computational_graphs}  
\end{figure}  
Neural networks are represented as \textbf{computational graphs}.
The graphs define the sequence of operations required to compute outputs and gradients during training. A graph consists of:  

\begin{itemize}  
    \item \textbf{Nodes:} Represent mathematical operations (e.g., Sigmoid).  
    \item \textbf{Edges:} Represent data flow between operations, forming a directed acyclic graph (DAG).  
\end{itemize}  

\noindent During training, frameworks use computational graphs to:
\begin{enumerate}
    \item \textbf{Forward Pass:} Compute the output by passing data through the graph.
    \item \textbf{Backward Pass:} Compute gradients via backpropagation, traversing the graph in reverse.
    \item \textbf{Optimization Step:} Update parameters using computed gradients.
\end{enumerate}

\noindent Understanding computational graphs is crucial, as different frameworks implement them in distinct ways. The next sections explore how PyTorch and TensorFlow utilize these graphs, comparing \textbf{dynamic} vs. \textbf{static} computation strategies.  

\section{PyTorch: Fundamental Concepts}
\label{subsec:chapter12_pytorch}

PyTorch is a deep learning framework that provides flexibility, dynamic computation graphs, and efficient execution on both CPUs and GPUs. It introduces key abstractions:

\begin{itemize}
    \item \textbf{Tensors:} Multi-dimensional arrays similar to NumPy arrays but capable of running on GPUs.
    \item \textbf{Modules:} Objects representing layers of a neural network, potentially storing learnable parameters.
    \item \textbf{Autograd:} A system that automatically computes gradients by building computational graphs dynamically.
\end{itemize}

\subsection{Tensors and Basic Computation}
\label{subsubsec:chapter12_pytorch_tensors}

To illustrate PyTorch’s fundamentals, consider a simple two-layer ReLU network trained using gradient descent on random data.

\begin{mintedbox}{python}
    import torch
    device = torch.device('cpu')  # Change to 'cuda:0' to run on GPU
    N, D_in, H, D_out = 64, 1000, 100, 10  # Batch size, input, hidden, output dimensions
    
    #Create random tensors for data and weights
    x = torch.randn(N, D_in, device=device)
    y = torch.randn(N, D_out, device=device)
    w1 = torch.randn(D_in, H, device=device)
    w2 = torch.randn(H, D_out, device=device)
    learning_rate = 1e-6
    
    for t in range(500):
        # Forward pass: compute predictions and loss
        h = x.mm(w1)  # Matrix multiply (fully connected layer)
        h_relu = h.clamp(min=0)  # Apply ReLU non-linearity
        y_pred = h_relu.mm(w2)  # Output prediction
        loss = (y_pred - y).pow(2).sum()  # Compute L2 loss
    
    # Backward pass: manually compute gradients
    grad_y_pred = 2.0 * (y_pred - y)
    grad_w2 = h_relu.t().mm(grad_y_pred)
    grad_h_relu = grad_y_pred.mm(w2.t())
    grad_h = grad_h_relu.clone()
    grad_h[h < 0] = 0  # Backpropagate ReLU
    grad_w1 = x.t().mm(grad_h)
    
    #Gradient descent step on weights
    w1 -= learning_rate * grad_w1  # Gradient update
    w2 -= learning_rate * grad_w2
\end{mintedbox}

\noindent PyTorch tensors operate efficiently on GPUs by simply setting:\\
\texttt{device = torch.device('cuda:0')}.

\subsection{Autograd: Automatic Differentiation}
\label{subsubsec:chapter12_pytorch_autograd}

PyTorch’s \textbf{autograd} system automatically builds computational graphs when performing operations on tensors with \texttt{requires\_grad=True}. These graphs allow automatic computation of gradients via backpropagation.

\begin{mintedbox}{python}
    x = torch.randn(N, D_in)
    y = torch.randn(N, D_out)
    w1 = torch.randn(D_in, H, requires_grad=True)
    w2 = torch.randn(H, D_out, requires_grad=True)
\end{mintedbox}

The forward pass remains unchanged:

\begin{mintedbox}{python}
    h = x.mm(w1)
    h_relu = h.clamp(min=0)
    y_pred = h_relu.mm(w2)
    loss = (y_pred - y).pow(2).sum()  # Compute loss
\end{mintedbox}

PyTorch automatically tracks operations and maintains intermediate values, eliminating the need for manual gradient computation. We backpropagate as follows:

\begin{mintedbox}{python}
    loss.backward()  # Computes gradients for w1 and w2
\end{mintedbox}

Gradients are accumulated in \texttt{w1.grad} and \texttt{w2.grad}, so we must clear them manually before the next update:

\begin{mintedbox}{python}
    with torch.no_grad():  # Prevents unnecessary graph construction
        w1 -= learning_rate * w1.grad
        w2 -= learning_rate * w2.grad
    
        w1.grad.zero_()
        w2.grad.zero_()
\end{mintedbox}

Forgetting to reset gradients is a common PyTorch bug, as gradients accumulate by default.

\subsection{Computational Graphs and Modular Computation}
\label{subsubsec:chapter12_pytorch_graph}

PyTorch dynamically constructs \textbf{computational graphs} during forward passes, enabling automatic differentiation and backpropagation. Each tensor operation that involves \texttt{requires\_grad=True} contributes to the computational graph.

\subsubsection{Building the Computational Graph}
\label{subsubsec:chapter12_pytorch_graph_building}

The computation graph begins when we perform operations on tensors with \texttt{requires\_grad=True}. Consider the following forward pass:

\begin{mintedbox}{python}
    h = x.mm(w1)  # Matrix multiply (fully connected layer)
    h_relu = h.clamp(min=0)  # Apply ReLU non-linearity
    y_pred = h_relu.mm(w2)  # Output prediction
    loss = (y_pred - y).pow(2).sum()  # Compute L2 loss
\end{mintedbox}

This sequence of operations results in the following computational graph:

\begin{itemize}
    \item \texttt{x.mm(w1)} creates a matrix multiplication node with inputs \texttt{x} and \texttt{w1}, producing an output tensor with \texttt{requires\_grad=True}.
\end{itemize}

\begin{figure}[H]
    \centering
    \includegraphics[width=0.8\textwidth]{Figures/Chapter_12/slide_21.jpg}
    \caption{\textbf{First computational node in the graph.} The matrix multiplication \texttt{x.mm(w1)} creates the first node in the computational graph.}
    \label{fig:chapter12_pytorch_mm_graph}
\end{figure}

\begin{itemize}
    \item \texttt{.clamp(min=0)} applies a ReLU activation, forming another node.
\end{itemize}

\begin{figure}[H]
    \centering
    \includegraphics[width=0.8\textwidth]{Figures/Chapter_12/slide_22.jpg}
    \caption{\textbf{ReLU activation node.} The ReLU function introduces a non-linearity while maintaining the computational graph structure.}
    \label{fig:chapter12_pytorch_relu_graph}
\end{figure}

\begin{itemize}
    \item \texttt{.mm(w2)} applies another matrix multiplication, producing the final prediction.
\end{itemize}

\begin{figure}[H]
    \centering
    \includegraphics[width=0.8\textwidth]{Figures/Chapter_12/slide_23.jpg}
    \caption{\textbf{Final matrix multiplication node.} The output prediction \texttt{y\_pred} is produced by matrix multiplication with \texttt{w2}.}
    \label{fig:chapter12_pytorch_mm2_graph}
\end{figure}

\subsubsection{Loss Computation and Backpropagation}
\label{subsubsec:chapter12_pytorch_loss_backprop}

After computing the loss, we backpropagate through the graph to compute gradients:

\begin{mintedbox}{python}
    loss.backward()  # Computes gradients for w1 and w2
\end{mintedbox}

During this process:
\begin{itemize}
    \item \texttt{(y\_pred - y)} creates a subtraction node with inputs \texttt{y\_pred} and \texttt{y}.
    \item \texttt{.pow(2)} squares the result, creating a new node.
    \item \texttt{.sum()} sums the squared differences, outputting a scalar loss.
\end{itemize}

\begin{figure}[H]
    \centering
    \includegraphics[width=0.8\textwidth]{Figures/Chapter_12/slide_27.jpg}
    \caption{\textbf{Loss computation node.} The final loss is computed as a scalar output in the computational graph, allowing backpropagation to all inputs requiring gradients.}
    \label{fig:chapter12_pytorch_loss_graph}
\end{figure}

Once gradients are computed, they are stored in \texttt{w1.grad} and \texttt{w2.grad}. However, PyTorch accumulates gradients by default, so they must be cleared before the next update (\texttt{grad.zero\_()}):

\begin{mintedbox}{python}
    with torch.no_grad():  # Prevents unnecessary graph construction
        w1 -= learning_rate * w1.grad
        w2 -= learning_rate * w2.grad
        w1.grad.zero_()
        w2.grad.zero_()
\end{mintedbox}

\noindent Forgetting to reset gradients is a common mistake in PyTorch. Although probably a design flaw in PyTorch, as we usually don't want to accumulate gradients, we need to be aware of that when we create models. 

\subsubsection{Extending Computational Graphs with Python Functions}
\label{subsubsec:chapter12_pytorch_modular}

PyTorch's autograd system allows users to construct computational graphs dynamically using Python functions. When a function is called inside a forward pass, PyTorch records all tensor operations occurring within it.

\begin{mintedbox}{python}
    def custom_relu(x):
        return x.clamp(min=0)  # Element-wise ReLU
	h_relu = custom_relu(h)
\end{mintedbox}

Although this function improves code readability, PyTorch still constructs the same computational graph as if we had used \texttt{.clamp(min=0)} directly.

\newpage

\subsubsection{Custom Autograd Functions}
\label{subsubsec:chapter12_pytorch_custom_autograd}

PyTorch's automatic differentiation works by building a computational graph out of primitive operations (e.g., \texttt{add}, \texttt{mul}, \texttt{exp}) and then applying the chain rule.  
In most cases this is sufficient, but sometimes we want:

\begin{itemize}
	\item To \textbf{treat a whole computation as a single semantic unit} in the graph (cleaner, fewer nodes, less bookkeeping).
	\item To \textbf{override the automatically derived backward} with a numerically more stable or more efficient formula.
\end{itemize}

For this, PyTorch lets us define custom operations by subclassing \texttt{torch.autograd.Function} and explicitly specifying \texttt{forward} and \texttt{backward}.

\paragraph{Motivating Example: Sigmoid}

A naive Python implementation of the sigmoid is:

\begin{mintedbox}{python}
    def sigmoid(x):
        return 1.0 / (1.0 + torch.exp(-x))
\end{mintedbox}

This looks harmless, but it can introduce numerical issues in deep networks:

\begin{itemize}
	\item For very \textbf{large negative} inputs $x \ll 0$, we compute \texttt{torch.exp(-x)} = \texttt{exp(large positive)}, which overflows to \texttt{inf} in \texttt{float32}.  
	The forward result is still $1/(1+\infty) \approx 0$, so we might not notice.
	\item However, during \textbf{backward}, autograd differentiates through these primitives and uses the same intermediate \texttt{inf} values.  
	Expressions such as $\frac{\infty}{(1+\infty)^2}$ or $\infty \cdot 0$ can appear, which numerically become \texttt{nan}, even though the true derivative is $0$.
\end{itemize}

Mathematically, the derivative of the sigmoid is
\[
\sigma'(x) = \sigma(x)\,(1 - \sigma(x)),
\]
and this is perfectly stable: once we know $y = \sigma(x) \in (0,1)$, the product $y(1-y)$ is always bounded in $[0, 0.25]$ and never overflows.  
So a more stable strategy is:

\begin{enumerate}
	\item Compute $y = \sigma(x)$ in the forward pass.
	\item \emph{Save} $y$.
	\item Compute the gradient in backward using $y(1-y)$ instead of recomputing exponentials.
\end{enumerate}

This is exactly what a custom autograd function allows us to do.

\begin{mintedbox}{python}
    class Sigmoid(torch.autograd.Function):
        @staticmethod
        def forward(ctx, x):
            # Forward as usual (PyTorch's built-in sigmoid is already stable;
            # here we reimplement it for illustration).
            y = 1.0 / (1.0 + torch.exp(-x))
            # Save only the stable output y for backward.
            ctx.save_for_backward(y)
            return y
\end{mintedbox}

\begin{mintedbox}{python}
    @staticmethod
        def backward(ctx, grad_y):
        # Retrieve saved output
        (y,) = ctx.saved_tensors
        # Use the stable formula seen earlier
        grad_x = grad_y * y * (1.0 - y)
        return grad_x
	
    def sigmoid(x):
        return Sigmoid.apply(x)
\end{mintedbox}

\begin{figure}[H]
	\centering
	\includegraphics[width=0.8\textwidth]{Figures/Chapter_12/slide_35.jpg}
	\caption{\textbf{Custom autograd function for sigmoid.} Left: the naive implementation expands into several primitive nodes (\texttt{exp}, \texttt{add}, \texttt{div}), each with its own backward. Right: the custom \texttt{Sigmoid} is a single node with a hand-crafted, numerically stable backward.}
	\label{fig:chapter12_pytorch_custom_autograd}
\end{figure}

\noindent Once defined, we can use the new sigmoid as any other PyTorch operation:

\begin{mintedbox}{python}
	x = torch.randn(10, requires_grad=True)
	sigmoid_out = sigmoid(x)
	sigmoid_out.sum().backward()
\end{mintedbox}

In practice, this level of control is rarely needed for basic operations: PyTorch’s built-in functions (\texttt{torch.sigmoid}, \texttt{torch.softmax}, etc.) are already implemented internally using optimized and stable autograd functions.  

Custom \texttt{Function}s become most useful when implementing new layers, composite operations, or specialized losses where we know a better backward formula than the one autograd would derive automatically.

\newpage

\subsubsection{Summary: Backpropagation and Graph Optimization}
\label{subsubsec:chapter12_pytorch_graph_summary}

\begin{itemize}
	\item \textbf{Any operation on a tensor with \texttt{requires\_grad=True} extends the computational graph.}
	\item \textbf{PyTorch dynamically records these operations} and stores just enough context (saved tensors) to evaluate gradients efficiently via the chain rule.
	\item \textbf{Forgetting to reset gradients} (e.g., omitting \texttt{optimizer.zero\_grad()}) causes gradients to accumulate across iterations, leading to incorrect updates.
	\item \textbf{Graph structure can be optimized} using custom autograd functions: they fuse multiple primitive ops into a single node, can implement numerically stable backward formulas, and provide more meaningful graph semantics than low-level primitives alone.
\end{itemize}

A solid understanding of PyTorch’s computational graphs—and how to customize them when necessary—is essential for debugging, improving numerical robustness, and optimizing the performance of deep learning models.

\subsection{High-Level Abstractions in PyTorch: \texttt{torch.nn} and Optimizers}
\label{subsec:chapter12_pytorch_nn}

PyTorch provides a high-level wrapper, \texttt{torch.nn}, which simplifies neural network construction by offering an object-oriented API for defining models. This abstraction allows for more structured and maintainable code, making deep learning models easier to build and extend.

\subsubsection{Using \texttt{torch.nn.Sequential}}
\label{subsubsec:chapter12_pytorch_sequential}

The \texttt{torch.nn.Sequential} container allows defining models as a sequence of layers. Below, we define a simple two-layer network with ReLU activation:

\begin{mintedbox}{python}
    import torch
    
    N, D_in, H, D_out = 64, 1000, 100, 10
    x = torch.randn(N, D_in)
    y = torch.randn(N, D_out)
    
    model = torch.nn.Sequential(
    torch.nn.Linear(D_in, H),
    torch.nn.ReLU(),
    torch.nn.Linear(H, D_out)
    )
    
    learning_rate = 1e-2
    for t in range(500):
        y_pred = model(x)
        loss = torch.nn.functional.mse_loss(y_pred, y)
        loss.backward()
    
        with torch.no_grad():
            for param in model.parameters():
                param -= learning_rate * param.grad
    
    model.zero_grad()
\end{mintedbox}

\begin{itemize}
    \item The \texttt{model} object is a container holding layers. Each layer manages its own parameters.
    \item Calling \texttt{model(x)} performs the forward pass.
    \item The loss is computed using \texttt{torch.nn.functional.mse\_loss()}.
    \item Calling \texttt{loss.backward()} computes gradients for all model parameters.
    \item Parameter updates are performed manually in a loop over \texttt{model.parameters()}.
    \item Calling \texttt{model.zero\_grad()} resets gradients for all parameters.
\end{itemize}

\subsubsection{Using Optimizers: Automating Gradient Descent}
\label{subsubsec:chapter12_pytorch_optimizers}

Instead of manually implementing gradient descent, PyTorch provides optimizer classes that handle parameter updates. Below, we use the Adam optimizer:

\begin{mintedbox}{python}
    import torch
    
    N, D_in, H, D_out = 64, 1000, 100, 10
    x = torch.randn(N, D_in)
    y = torch.randn(N, D_out)
    
    model = torch.nn.Sequential(
    torch.nn.Linear(D_in, H),
    torch.nn.ReLU(),
    torch.nn.Linear(H, D_out)
    )
    
    learning_rate = 1e-4
    optimizer = torch.optim.Adam(model.parameters(), lr=learning_rate)
    
    for t in range(500):
        y_pred = model(x)
        loss = torch.nn.functional.mse_loss(y_pred, y)
        loss.backward()
        
        optimizer.step()
        optimizer.zero_grad()
\end{mintedbox}

\begin{itemize}
    \item The optimizer is instantiated with \texttt{torch.optim.Adam()} and receives model parameters.
    \item Calling \texttt{optimizer.step()} updates all parameters automatically.
    \item Calling \texttt{optimizer.zero\_grad()} resets gradients before the next step.
\end{itemize}

This approach is both cleaner and less error-prone than manual updates.

\newpage

\subsubsection{Defining Custom \texttt{nn.Module} Subclasses}
\label{subsubsec:chapter12_pytorch_nn_module}

For more complex architectures, we can define custom \texttt{nn.Module} subclasses:

\begin{mintedbox}{python}
    import torch
    
    class TwoLayerNet(torch.nn.Module):
        def __init__(self, D_in, H, D_out):
            super(TwoLayerNet, self).__init__()
            self.linear1 = torch.nn.Linear(D_in, H)
            self.linear2 = torch.nn.Linear(H, D_out)
        
        def forward(self, x):
            h_relu = self.linear1(x).clamp(min=0)
            y_pred = self.linear2(h_relu)
            return y_pred
        
    N, D_in, H, D_out = 64, 1000, 100, 10
    x = torch.randn(N, D_in)
    y = torch.randn(N, D_out)
        
    model = TwoLayerNet(D_in, H, D_out)
    optimizer = torch.optim.SGD(model.parameters(), lr=1e-4)
    
    for t in range(500):
        y_pred = model(x)
        loss = torch.nn.functional.mse_loss(y_pred, y)
        loss.backward()
        
        optimizer.step()
        optimizer.zero_grad()
\end{mintedbox}

\begin{itemize}
    \item \textbf{Model Initialization:} The \texttt{\_\_init\_\_} method defines layers as class attributes.
    \item \textbf{Forward Pass:} The \texttt{forward()} method specifies how inputs are transformed.
    \item \textbf{Autograd Integration:} PyTorch automatically tracks gradients for model parameters.
    \item \textbf{Training Loop:} The optimizer updates weights based on computed gradients.
\end{itemize}

\subsubsection{Key Takeaways}
\begin{itemize}
    \item \textbf{\texttt{torch.nn.Sequential}} simplifies defining networks as a stack of layers.
    \item \textbf{Optimizers automate gradient descent}, making training loops cleaner.
    \item \textbf{Custom \texttt{nn.Module} subclasses} provide flexibility for complex architectures.
    \item \textbf{Autograd handles differentiation automatically}, eliminating the need for manual backward computations.
\end{itemize}

Using \texttt{torch.nn} and optimizers streamlines model development, making PyTorch a powerful and expressive framework for deep learning.

\subsection{Combining Custom Modules with Sequential Models}
\label{subsec:chapter12_pytorch_custom_sequential}

A common practice in PyTorch is to combine custom \texttt{nn.Module} subclasses with \texttt{torch.nn.Sequential} containers. This enables modular and scalable architectures while maintaining the expressiveness of object-oriented model design.

\subsubsection{Example: Parallel Block}
\label{subsubsec:chapter12_pytorch_parallel_block}

The following example defines a \texttt{ParallelBlock} module that applies two linear transformations to the input independently and then multiplies the results element-wise:

\begin{mintedbox}{python}
    import torch
    
    class ParallelBlock(torch.nn.Module):
        def __init__(self, D_in, D_out):
            super(ParallelBlock, self).__init__()
            self.linear1 = torch.nn.Linear(D_in, D_out)
            self.linear2 = torch.nn.Linear(D_in, D_out)
        
        def forward(self, x):
            h1 = self.linear1(x)
            h2 = self.linear2(x)
            return (h1 * h2).clamp(min=0)  # Element-wise multiplication followed by ReLU
    
    N, D_in, H, D_out = 64, 1000, 100, 10
    x = torch.randn(N, D_in)
    y = torch.randn(N, D_out)
    
    model = torch.nn.Sequential(
    ParallelBlock(D_in, H),
    ParallelBlock(H, H),
    torch.nn.Linear(H, D_out)
    )
    
    optimizer = torch.optim.Adam(model.parameters(), lr=1e-4)
    
    for t in range(500):
        y_pred = model(x)
        loss = torch.nn.functional.mse_loss(y_pred, y)
        loss.backward()
        optimizer.step()
        optimizer.zero_grad()
\end{mintedbox}

\begin{itemize}
    \item The \texttt{ParallelBlock} applies two separate linear layers to the input.
    \item The outputs are multiplied element-wise before applying ReLU.
    \item The \texttt{Sequential} container stacks multiple \texttt{ParallelBlock} instances, followed by a final linear layer.
    \item Using this approach allows rapid experimentation with modular neural network components.
\end{itemize}

\noindent Although this example is not very smart and not thing we should in practice, it demonstrates well the ability to create building blocks using torch and thus create using this abstraction some complex neural networks with ease. 

\begin{figure}[H]
    \centering
    \includegraphics[width=0.8\textwidth]{Figures/Chapter_12/slide_50.jpg}
    \caption{\textbf{ParallelBlock module design:} The implementation of the \texttt{ParallelBlock} and its corresponding computational graph visualization.}
    \label{fig:chapter12_parallel_block}
\end{figure}

\begin{figure}[H]
    \centering
    \includegraphics[width=0.8\textwidth]{Figures/Chapter_12/slide_51.jpg}
    \caption{\textbf{Stacking multiple \texttt{ParallelBlock} instances in a Sequential model.} The left side of the figure shows the computational graph produced.}
    \label{fig:chapter12_parallel_block_graph}
\end{figure}

\subsection{Efficient Data Loading with \texttt{torch.utils.data}}
\label{subsec:chapter12_pytorch_dataloader}

Training deep neural networks efficiently requires a robust data pipeline. PyTorch provides the \texttt{torch.utils.data} module, which abstracts away data loading, shuffling, batching, and parallelization—ensuring that model computation and data preparation can run concurrently.  
The two key components are:

\begin{itemize}
	\item \textbf{\texttt{Dataset}:} Represents a collection of samples. You can use built-in classes like \texttt{TensorDataset} for in-memory tensors or implement a custom \texttt{Dataset} that reads from files or databases.
	\item \textbf{\texttt{DataLoader}:} Wraps a \texttt{Dataset} to provide mini-batching, shuffling, and multi-process data loading. It also supports pinned memory for faster GPU transfer.
\end{itemize}

\subsubsection{Example: Using \texttt{DataLoader} for Mini-batching}
\label{subsubsec:chapter12_pytorch_dataloader_example}

The example below demonstrates how to use \texttt{DataLoader} with synthetic data for mini-batch training.

\begin{mintedbox}{python}
	import torch
	from torch.utils.data import TensorDataset, DataLoader
	
	# 1. Create a simple in-memory dataset
	N, D_in, H, D_out = 64, 1000, 100, 10
	x = torch.randn(N, D_in)
	y = torch.randn(N, D_out)
	
	dataset = TensorDataset(x, y)
	
	# 2. Create a DataLoader with batching and parallel loading
	loader = DataLoader(
	dataset,
	batch_size=8,
	shuffle=True,      # Shuffle each epoch for stable training
	num_workers=2,     # Parallel CPU workers for background loading
	pin_memory=True    # Speeds up host→GPU transfers
	)
	
	model = TwoLayerNet(D_in, H, D_out)
	optimizer = torch.optim.SGD(model.parameters(), lr=1e-2)
	
	# 3. Training loop using the DataLoader
    for epoch in range(20):
        for x_batch, y_batch in loader:
            y_pred = model(x_batch)
            loss = torch.nn.functional.mse_loss(y_pred, y_batch)
	
            loss.backward()
            optimizer.step()
            optimizer.zero_grad()
\end{mintedbox}

\noindent
This setup automatically handles mini-batch creation, shuffling, and memory prefetching.  
With \texttt{num\_workers > 0}, the CPU preloads data while the GPU trains on the previous batch, preventing GPU idle time—a crucial optimization for large datasets.

\paragraph{Best Practices}
\begin{itemize}
	\item Use \texttt{shuffle=True} to avoid order bias and improve gradient diversity.
	\item Adjust \texttt{num\_workers} to match your CPU cores (typical range: 2–8) for best throughput.
	\item Set \texttt{pin\_memory=True} when training on GPU to accelerate host–device transfers.
\end{itemize}

\subsubsection{Handling Multiple Datasets}
\label{subsubsec:chapter12_pytorch_dataloader_multi}

In practice, data often comes from multiple sources—different domains, modalities, or tasks.  
PyTorch offers flexible tools to combine and balance these datasets efficiently.

\paragraph{Concatenating Datasets}
When datasets share the same structure (e.g., same feature dimensions), use \texttt{ConcatDataset} to merge them into a single unified dataset.

\begin{mintedbox}{python}
	from torch.utils.data import ConcatDataset, DataLoader
	
	dataset_a = TensorDataset(torch.randn(100, 20), torch.randn(100, 1))
	dataset_b = TensorDataset(torch.randn(200, 20), torch.randn(200, 1))
	
	combined = ConcatDataset([dataset_a, dataset_b])
	
	loader = DataLoader(
	combined,
	batch_size=16,
	shuffle=True,
	num_workers=4
	)
\end{mintedbox}

\noindent
This approach interleaves samples from all datasets proportionally to their sizes. It is ideal for combining related sources (e.g., merging multiple corpora or image datasets).

\paragraph{Weighted Sampling Across Datasets}
If some datasets are much smaller or more important, you can balance sampling probabilities using \texttt{WeightedRandomSampler}. This ensures underrepresented data appears more frequently in training batches.

\begin{mintedbox}{python}
	from torch.utils.data import WeightedRandomSampler
	
	# Example: emphasize smaller dataset (dataset_a)
	weights = [1.0 / len(dataset_a)] * len(dataset_a) + \
	[1.0 / len(dataset_b)] * len(dataset_b)
	
	sampler = WeightedRandomSampler(weights, num_samples=len(weights), replacement=True)
	
	balanced_loader = DataLoader(
	combined,
	batch_size=16,
	sampler=sampler,
	num_workers=4
	)
\end{mintedbox}

\noindent
Weighted sampling is especially useful for:
\begin{itemize}
	\item \textbf{Imbalanced datasets.} For example, when rare classes need more representation during training.
	\item \textbf{Multi-source training.} Combining labeled and unlabeled data or datasets from distinct domains.
	\item \textbf{Curriculum learning.} Gradually increasing sample difficulty or diversity over time.
\end{itemize}

\paragraph{Streaming or Multi-modal Data}
For more dynamic or heterogeneous sources (e.g., loading text and image pairs), subclass \texttt{IterableDataset} to yield samples from multiple streams in real time, or define a custom \texttt{Sampler} to coordinate multi-modal alignment.

\begin{mintedbox}{python}
	from torch.utils.data import IterableDataset
	
    class MultiSourceStream(IterableDataset):
        def __iter__(self):
            for x_img, x_txt in zip(image_stream(), text_stream()):
                yield preprocess(x_img, x_txt)
\end{mintedbox}

\noindent
This design is common in large-scale vision–language or multi-task training pipelines, where data arrives asynchronously or from external APIs.

\paragraph{Summary}
\texttt{DataLoader} and its related utilities form the backbone of efficient training in PyTorch.  
They decouple data I/O from model computation, provide clean abstractions for multi-source or imbalanced data, and make large-scale experiments reproducible and scalable across CPUs and GPUs.

\subsection{Using Pretrained Models with TorchVision}
\label{subsec:chapter12_pytorch_torchvision}

PyTorch provides access to many pretrained models through the \texttt{torchvision} package, making it easy to leverage existing architectures for various vision tasks.

Using pretrained models is as simple as:

\begin{mintedbox}{python}
    import torchvision.models as models
    
    alexnet = models.alexnet(pretrained=True)
    vgg16 = models.vgg16(pretrained=True)
    resnet101 = models.resnet101(pretrained=True)
\end{mintedbox}

\begin{itemize}
    \item These models come with pretrained weights on ImageNet, making them suitable for transfer learning.
    \item Fine-tuning pretrained models often leads to faster convergence and better performance on new tasks.
    \item \texttt{torchvision.models} provides a wide variety of architectures beyond AlexNet, VGG, and ResNet.
\end{itemize}

\subsubsection{Key Takeaways}
\begin{itemize}
    \item \textbf{Custom modules and \texttt{torch.nn.Sequential} can be combined} to quickly build complex models while maintaining modularity.
    \item \textbf{Data loading utilities} such as \texttt{torch.utils.data.DataLoader} facilitate efficient mini-batching and dataset management.
    \item \textbf{TorchVision provides pretrained models}, making it easy to leverage state-of-the-art architectures for various vision tasks.
\end{itemize}

\section{Dynamic vs. Static Computational Graphs in PyTorch}
\label{subsec:chapter12_pytorch_dynamic_vs_static}

A fundamental design choice in PyTorch is its use of \textbf{dynamic computational graphs}. Unlike static graphs, which are constructed once and reused, PyTorch builds a fresh computational graph for each forward pass. Once \texttt{loss.backward()} is called, the graph is discarded, and a new one is constructed in the next iteration.

While dynamically building graphs in every iteration may seem inefficient, this approach provides a crucial advantage: \emph{the ability to use standard Python control flow during model execution}. This enables complex architectures that modify their behavior on-the-fly based on intermediate results.

\subsubsection{Example: Dynamic Graph Construction}
\label{subsubsec:chapter12_pytorch_dynamic_example}

Consider a model where the choice of weight matrix for backpropagation depends on the previous loss value. This scenario, though impractical, demonstrates PyTorch’s ability to create different computational graphs in each iteration.

\begin{figure}[H]
    \centering
    \includegraphics[width=0.85\textwidth]{Figures/Chapter_12/slide_67.jpg}
    \caption{\textbf{Example of a dynamically constructed graph:} The model structure changes at each iteration based on previous loss values.}
    \label{fig:chapter12_dynamic_graph}
\end{figure}

In dynamic graphs, every forward pass constructs a unique computation graph, allowing for models with \textbf{varying execution paths} across different iterations.

\subsection{Static Graphs and Just-in-Time (JIT) Compilation}
\label{subsubsec:chapter12_pytorch_static_graphs}

In contrast, \textbf{static computational graphs} follow a two-step process:
\begin{enumerate}
    \item \textbf{Graph Construction:} Define the computational graph once, allowing the framework to optimize it before execution.
    \item \textbf{Graph Execution:} The same pre-optimized graph is reused for all forward passes.
\end{enumerate}

\noindent While PyTorch natively operates with dynamic graphs, it also supports static graphs through \textbf{TorchScript} using \textbf{Just-in-Time (JIT) compilation}. This allows PyTorch to analyze the model’s source code, compile it into an optimized static graph, and reuse it for improved efficiency.

\subsection{Using JIT to Create Static Graphs}
\label{subsubsec:chapter12_pytorch_jit}

To convert a function into a static computational graph, PyTorch provides \texttt{torch.jit.script()}:

\begin{mintedbox}{python}
    import torch
    
    def model(x):
        return x * torch.sin(x)
    
    scripted_model = torch.jit.script(model)  # Convert to static graph
\end{mintedbox}

Alternatively, PyTorch allows automatic graph compilation using the \textbf{@torch.jit.script} annotation:

\begin{mintedbox}{python}
    import torch
    
    @torch.jit.script
    def model(x):
        return x * torch.sin(x)
\end{mintedbox}

\begin{figure}[H]
    \centering
    \includegraphics[width=0.85\textwidth]{Figures/Chapter_12/slide_73.jpg}
    \caption{\textbf{TorchScript:} Using JIT compilation to convert PyTorch models into static graphs for optimization.}
    \label{fig:chapter12_pytorch_jit}
\end{figure}

\subsection{Handling Conditionals in Static Graphs}
\label{subsubsec:chapter12_pytorch_jit_conditionals}

Static graphs struggle with conditionals because they are typically \textbf{fixed at compile time}. However, PyTorch’s JIT can represent conditionals as graph nodes, enabling runtime flexibility.

\begin{figure}[H]
    \centering
    \includegraphics[width=0.85\textwidth]{Figures/Chapter_12/slide_71.jpg}
    \caption{\textbf{Conditionals in static graphs:} JIT inserts a conditional node to handle different execution paths.}
    \label{fig:chapter12_pytorch_jit_conditionals}
\end{figure}

This allows some degree of flexibility while retaining the benefits of graph optimization.

\subsection{Optimizing Computation Graphs with JIT}
\label{subsubsec:chapter12_pytorch_graph_optimization}

One advantage of static graphs is that they enable \textbf{graph-level optimizations}. PyTorch JIT can automatically \textbf{fuse operations} such as convolution and activation layers into a single efficient operation.

\begin{figure}[H]
    \centering
    \includegraphics[width=0.85\textwidth]{Figures/Chapter_12/slide_74.jpg}
    \caption{\textbf{Operation fusion in static graphs:} Layers such as Conv + ReLU are combined into a single operation to improve efficiency.}
    \label{fig:chapter12_pytorch_fusion}
\end{figure}

This optimization is performed once, eliminating the need to optimize in every iteration.

\subsection{Benefits and Limitations of Static Graphs}
\label{subsubsec:chapter12_pytorch_static_benefits_challenges}

\textbf{Advantages of Static Graphs:}
\begin{itemize}
    \item \textbf{Graph Optimization:} The framework optimizes computation before execution, improving speed.
    \item \textbf{Operation Fusion:} Frequently used layers (e.g., Conv + ReLU) are merged into a single operation.
    \item \textbf{Serialization:} Models can be saved to disk and loaded in non-Python environments (e.g., C++).
\end{itemize}

\textbf{Challenges of Static Graphs:}
\begin{itemize}
    \item \textbf{Difficult Debugging:} Debugging static graphs can be challenging due to indirection between graph construction and execution.
    \item \textbf{Less Flexibility:} Unlike dynamic graphs, static graphs struggle with models that modify their execution path.
    \item \textbf{Rebuilding Required:} Any model change requires reconstructing the entire graph.
\end{itemize}

\subsection{When Are Dynamic Graphs Necessary?}
\label{subsubsec:chapter12_pytorch_dynamic_needed}

Certain architectures \emph{require} dynamic graphs due to their execution dependencies on input data:

\begin{itemize}
    \item \textbf{Recurrent Neural Networks (RNNs):} The number of computation steps depends on input sequence length.
    \item \textbf{Recursive Networks:} Hierarchical models, such as parse trees in NLP, require dynamic execution paths.
    \item \textbf{Modular Networks:} Some architectures dynamically select which sub-network to execute.
\end{itemize}

A well-known example is the model in \cite{johnson2017_infering}, where part of the network predicts which module should execute next.

\section{TensorFlow: Dynamic and Static Computational Graphs}
\label{subsec:chapter12_tensorflow}

TensorFlow originally adopted \textbf{static computational graphs} by default (TensorFlow 1.0), requiring users to explicitly define a computation graph before running it. However, in \textbf{TensorFlow 2.0}, the framework transitioned to \textbf{dynamic graphs} by default, making the API more similar to PyTorch. This shift caused a significant divide in the TensorFlow ecosystem, as older static-graph code intertwined with newer dynamic-graph code, creating confusion and bugs.

\subsection{Defining Computational Graphs in TensorFlow 2.0}
\label{subsec:chapter12_tensorflow_dynamic}

In PyTorch, the computational graph is built \textit{implicitly}: any operation performed on a tensor with \texttt{requires\_grad=True} is automatically tracked.  
TensorFlow 2.0 (TF2), by contrast, introduced \textbf{eager execution} as the default mode—operations execute immediately like standard Python code, producing concrete values rather than symbolic graph nodes.  
This makes TF2 intuitive and debuggable but requires an explicit mechanism for recording operations when gradients are needed. That mechanism is the \textbf{\texttt{tf.GradientTape}}.

\subsubsection{Understanding \texttt{tf.GradientTape}}
\label{subsubsec:chapter12_tensorflow_tape}

The \texttt{GradientTape} is TensorFlow's dynamic autodiff engine, analogous to PyTorch’s implicit autograd.  
It acts like a \textit{“recorder”}: while active, it logs all operations on watched tensors (typically all \texttt{tf.Variable} objects) and can later “play back” those operations to compute gradients.

\begin{itemize}
	\item Entering a \texttt{with tf.GradientTape() as tape:} block begins recording.
	\item Any operation involving watched variables is logged on the tape.
	\item Exiting the block stops recording.
	\item Calling \texttt{tape.gradient(loss, [vars])} replays the tape backward to compute exact gradients via the chain rule.
\end{itemize}

This explicit opt-in design prevents unnecessary gradient tracking (e.g., during inference) and gives developers fine-grained control over which computations are differentiable.

\begin{mintedbox}{python}
	import tensorflow as tf
	
	# Setup data and parameters
	N, Din, H, Dout = 16, 1000, 100, 10
	x = tf.random.normal((N, Din))
	y = tf.random.normal((N, Dout))
	w1 = tf.Variable(tf.random.normal((Din, H)))
	w2 = tf.Variable(tf.random.normal((H, Dout)))
	
	learning_rate = 1e-6
	
	for t in range(1000):
	    # Begin recording operations on the tape
	    with tf.GradientTape() as tape:
	        h = tf.maximum(tf.matmul(x, w1), 0)  # ReLU
	        y_pred = tf.matmul(h, w2)
	        diff = y_pred - y
	        loss = tf.reduce_mean(tf.reduce_sum(diff ** 2, axis=1))
	
	    # Compute gradients of loss w.r.t parameters
	    grad_w1, grad_w2 = tape.gradient(loss, [w1, w2])
	
	    # Parameter updates (in-place, safe for tf.Variables)
	    w1.assign_sub(learning_rate * grad_w1)
	    w2.assign_sub(learning_rate * grad_w2)
\end{mintedbox}

\noindent
This process mirrors PyTorch’s autograd but with more explicit control:
\texttt{GradientTape} defines the graph’s lifetime (inside the \texttt{with} block), rather than relying on implicit global tracking.
The resulting computation graph is ephemeral—destroyed after gradient computation unless the tape is declared as \texttt{persistent=True} (allowing multiple gradient calls).

\paragraph{Key differences from PyTorch}
\begin{itemize}
	\item PyTorch automatically tracks gradients for all tensors with \texttt{requires\_grad=True}. TensorFlow records only within the \texttt{GradientTape} context.
	\item TensorFlow’s graph is discarded after use unless marked persistent.
	\item GradientTape offers fine-grained control: you can record subsets of operations or specific variables only.
\end{itemize}

\newpage

\subsection{Static Graphs with \texttt{@tf.function}}
\label{subsec:chapter12_tensorflow_static}

While TF2 defaults to eager (imperative) execution for flexibility, static computation graphs are still essential for deployment and optimization.  
To combine both worlds, TensorFlow introduces the \textbf{\texttt{@tf.function}} decorator, which traces Python functions into optimized static graphs—comparable to \texttt{torch.jit.script()} in PyTorch.

\paragraph{Motivation}
Eager execution simplifies experimentation but adds Python overhead per operation.
Static graphs, on the other hand, allow TensorFlow to perform ahead-of-time optimizations:
operation fusion (e.g., combining \texttt{matmul + bias\_add}), kernel selection, memory reuse, and XLA compilation.
Using \texttt{@tf.function}, developers write natural Python code while TensorFlow transparently traces and compiles it.

\begin{mintedbox}{python}
	@tf.function  # Compiles to a static graph on first call
	def training_step(x, y, w1, w2, lr):
	    with tf.GradientTape() as tape:
	        h = tf.maximum(tf.matmul(x, w1), 0)
	        y_pred = tf.matmul(h, w2)
	        loss = tf.reduce_mean(tf.reduce_sum((y_pred - y) ** 2, axis=1))
	
	    grad_w1, grad_w2 = tape.gradient(loss, [w1, w2])
	    w1.assign_sub(lr * grad_w1)
	    w2.assign_sub(lr * grad_w2)
	    return loss
	
	# Regular Python loop, but graph executes under the hood
	for t in range(1000):
	    current_loss = training_step(x, y, w1, w2, learning_rate)
\end{mintedbox}

\noindent
Here, \texttt{@tf.function} traces the computation during its first execution, then caches the resulting static graph for reuse—removing Python overhead and enabling runtime optimizations.  
This achieves up to 2--10$\times$ speedups for heavy workloads while preserving eager-like syntax.

\paragraph{Summary of Modes}
\begin{itemize}
	\item \textbf{Eager mode.} Operations run immediately, ideal for debugging and experimentation.
	\item \textbf{GradientTape.} Dynamically records operations for automatic differentiation, similar to PyTorch’s autograd.
	\item \textbf{@tf.function.} Converts eager code into a reusable static graph, fusing and optimizing operations for deployment.
\end{itemize}

Together, these tools give TensorFlow 2.0 both the interactivity of PyTorch and the performance advantages of static compilation—bridging the flexibility–efficiency trade-off that defined earlier deep learning frameworks.

\newpage

\section{Keras: High-Level API for TensorFlow}
\label{subsec:chapter12_keras}

Keras provides a high-level API for building deep learning models, simplifying working with models.

\begin{mintedbox}{python}
    import tensorflow as tf
    from tensorflow.keras.models import Sequential
    from tensorflow.keras.layers import InputLayer, Dense
    
    N, Din, H, Dout = 16, 1000, 100, 10
    
    model = Sequential([
    InputLayer(input_shape=(Din,)),
    Dense(units=H, activation='relu'),
    Dense(units=Dout)
    ])
    
    loss_fn = tf.keras.losses.MeanSquaredError()
    opt = tf.keras.optimizers.SGD(learning_rate=1e-6)
    
    x = tf.random.normal((N, Din))
    y = tf.random.normal((N, Dout))
    
    for t in range(1000):
        with tf.GradientTape() as tape:
            y_pred = model(x)
            loss = loss_fn(y_pred, y)
        grads = tape.gradient(loss, model.trainable_variables)
        opt.apply_gradients(zip(grads, model.trainable_variables))
\end{mintedbox}

\noindent
Keras simplifies training by providing:
\begin{itemize}
    \item \textbf{Predefined layers}: Easily stack layers with \texttt{Sequential()}.
    \item \textbf{Common loss functions and optimizers}: Use built-in losses and optimizers like Adam.
    \item \textbf{Automatic gradient handling}: \texttt{opt.apply\_gradients()}  simplifies parameter updates.
\end{itemize}

\noindent
We can further simplify the training loop using \texttt{opt.minimize()} by defining a step function:

\begin{mintedbox}{python}
    def step():
        y_pred = model(x)
        loss = loss_fn(y_pred, y)
        return loss
    
    for t in range(1000):
        opt.minimize(step, model.trainable_variables)
\end{mintedbox}

\section{TensorBoard: Visualizing Training Metrics}
\label{subsec:chapter12_tensorboard}

\begin{figure}[H]
    \centering
    \includegraphics[width=0.85\textwidth]{Figures/Chapter_12/slide_100.jpg}
    \caption{\textbf{TensorBoard visualization:} Loss curves and weight distributions during training.}
    \label{fig:chapter12_tensorboard}
\end{figure}

\textbf{TensorBoard} is a visualization tool that helps monitor deep learning experiments. It allows users to track:
\begin{itemize}
    \item Loss curves and accuracy during training.
    \item Weight distributions and parameter updates.
    \item Computational graphs of the model.
\end{itemize}

While originally designed for TensorFlow, TensorBoard now support \textbf{PyTorch} via the \\
\texttt{torch.utils.tensorboard} API.
However, modern alternatives such as \textbf{Weights and Biases (wandb)} and \textbf{MLFlow} provide additional functionality, making them popular choices for tracking experiments.

\section{Comparison: PyTorch vs. TensorFlow}
\label{subsec:chapter12_comparison}

\begin{itemize}
    \item \textbf{PyTorch:}
    \begin{itemize}
        \item Imperative API that is easy to debug.
        \item Dynamic computation graphs enable flexibility.
        \item \texttt{torch.jit.script()} allows for static graph compilation.
        \item Harder to optimize for TPUs.
        \item Deployment on mobile is less streamlined.
    \end{itemize}
    \item \textbf{TensorFlow 1.0:}
    \begin{itemize}
        \item Static graphs by default.
        \item Faster execution but difficult debugging.
        \item API inconsistencies made it less user-friendly.
    \end{itemize}
    \item \textbf{TensorFlow 2.0:}
    \begin{itemize}
        \item Defaulted to dynamic graphs, similar to PyTorch.
        \item Standardized Keras API for ease of use.
        \item Still retains static graph capability with \texttt{tf.function}.
    \end{itemize}
\end{itemize}

\paragraph{Conclusion}
Both PyTorch and TensorFlow 2.0 now support both dynamic and static graphs, offering flexibility for different use cases. PyTorch remains the preferred choice for research due to its intuitive imperative style, while TensorFlow is still widely used in production, particularly in environments requiring static graph optimization.


\chapterimage{head2.png} % Chapter heading image

% Chapter-specific content starts here
\chapter{Lecture 13: Object Detection}

%-----------------------------------------------------------------------------------
%	CHAPTER 13 - Lecture 13: Object Detection
%-----------------------------------------------------------------------------------




\chapterimage{head2.png} % Chapter heading image

% Chapter-specific content starts here
\chapter{Lecture 14: Object Detectors}

%-----------------------------------------------------------------------------------
%    CHAPTER 14 - Lecture 14: Object Detectors
%-----------------------------------------------------------------------------------

\section{Beyond R-CNN: Advancing Object Detection}
\label{sec:chapter14_intro}

In the previous chapter we focused on \emph{what} object detection is (bounding boxes, IoU, AP/mAP, NMS) and briefly contrasted closed-set vs.\ open-set detection. We now turn to \emph{how} detectors are actually built, starting from the first successful CNN-based systems.

\textbf{R-CNN} showed that applying a deep convolutional network to region proposals could dramatically outperform traditional pipelines, firmly establishing CNNs as the backbone of modern detectors. The downside was efficiency: for each image, R-CNN runs a separate CNN forward pass on roughly \(\sim 2000\) region proposals, followed by separate SVMs and bounding box regressors. This heavy, multi-stage pipeline makes R-CNN slow to train and far too expensive for real-time or large-scale deployment.

The rest of this chapter follows the historical path toward more efficient and integrated detectors:

\begin{itemize}
	\item \textbf{Fast R-CNN} shares convolutional features across all proposals and introduces RoI Pooling / RoIAlign to speed up per-region processing.
	\item \textbf{Faster R-CNN} learns region proposals with a Region Proposal Network (RPN), removing the last major hand-crafted component.
	\item \textbf{Feature Pyramid Networks (FPNs)} exploit multi-scale feature maps to improve detection of small and large objects.
	\item \textbf{Single-stage and anchor-free detectors} such as RetinaNet and FCOS further simplify the pipeline by predicting boxes and classes densely in one pass.
	\item \textbf{YOLO}-style models show how far we can push real-time, single-shot detection in practice.
\end{itemize}

Together, these CNN-based detectors form the “classical toolkit” of object detection. While they are not widely used today (besides YOLO), as we will see, many of their core ideas—feature sharing, bounding box regression, multi-task losses, and multi-scale features—reappear inside newer architectures as well.

\newpage

\subsection{Looking Ahead: Beyond CNN-Based Object Detectors}
\label{subsec:chapter14_future_object_detection}

Even the most refined CNN-based detectors in this chapter share a common structure: convolutional backbones, dense candidate boxes (anchors or per-pixel predictions), and post-processing with NMS. Modern work pushes further toward \textbf{end-to-end architectures} that minimize hand-designed components and treat detection more like a direct set prediction problem.

A key milestone is \textbf{DETR (DEtection TRansformer)}~\cite{carion2020_detr}, which uses transformers and a set-based matching loss to predict a fixed-size set of boxes and labels, removing both region proposals and NMS from the pipeline. Follow-up works such as \textbf{Re DETR}~\cite{zhu2023_re_detr} and \textbf{DINO for detection}~\cite{zhang2022_dino} refine optimization, query design, and training recipes to improve convergence speed and accuracy, while \textbf{Mask DINO}~\cite{li2022_maskdino} extends these ideas to instance and panoptic segmentation.

At the same time, large vision backbones trained with self-supervision or vision-only pretraining, such as \textbf{DINOv2}~\cite{oquab2023_dinov2} and \textbf{DINOv3}~\cite{simeoni2025_dinov3}, provide powerful, task-agnostic image representations that can be plugged into many detection heads (Faster R-CNN, RetinaNet, DETR variants) to boost performance with minimal task-specific tuning.

In the \textbf{open-vocabulary} setting briefly discussed in Chapter~13, many state-of-the-art systems build directly on these transformer and backbone advances: \textbf{Grounding DINO}~\cite{liu2023_groundingdino}, \textbf{OWL-ViT} and \textbf{OWLv2}~\cite{minderer2022_owlvit,minderer2024_owlv2}, and \textbf{YOLO-World}~\cite{cheng2024_yoloworld} combine strong image encoders with text encoders to align region features with natural-language prompts. This allows detectors to move beyond a fixed label list and answer queries like “red umbrella” or “person holding a phone” in a zero-shot way.

We will study transformers, large vision backbones, and vision–language models in detail later in the book. For now, our goal is to master the \textbf{classic CNN-based detectors}—R-CNN, Fast R-CNN, Faster R-CNN, FPN-based two-stage models, and single-stage/anchor-free designs—since the principles they introduce are the foundation upon which these newer architectures are built.

\newpage

\section{Fast R-CNN: Accelerating Object Detection}
\label{sec:chapter14_fast_rcnn}

As running a CNN forward pass separately for each of the \(\sim2000\) region proposals per image led to massive computational overhead, despite its performance, R-CNN was too slow for practical usage.

Fast R-CNN \cite{girshick2015_fastrcnn} was proposed as a major improvement, significantly reducing inference time while maintaining strong detection accuracy. By reusing shared feature maps instead of processing each region proposal independently, it eliminated redundant computations and improved efficiency.

\subsection{Key Idea: Shared Feature Extraction}
\label{subsec:chapter14_fast_rcnn_idea}

Instead of running a CNN separately for each proposal, \textbf{Fast R-CNN} applies a deep CNN \emph{once} to the entire image.

\begin{figure}[H]
    \centering
    \includegraphics[width=0.8\textwidth]{Figures/Chapter_14/slide_18.jpg}
    \caption{\textbf{Fast R-CNN architecture:} A backbone CNN processes the full image, generating a feature map. RoI Pooling extracts regions from this shared representation, followed by classification and bounding box refinement. This significantly improves efficiency while maintaining detection accuracy.}
    \label{fig:chapter14_fast_rcnn}
\end{figure}

It does so by extracting a \textbf{shared feature representation}. Then, \textbf{Region of Interest (RoI) Pooling} is used to extract features corresponding to each region proposal from this shared representation.  A \textbf{small per-region sub-network} is then applied to each extracted region to \textbf{Classify} the region into an object category or background, and \textbf{refine the bounding box} using regression. 

\newpage

\subsection{Using Fully Convolutional Deep Backbones for Feature Extraction}
\label{subsec:chapter14_fast_rcnn_backbone}

Fast R-CNN leverages deep CNNs to extract features from the entire image in one forward pass. 

\begin{figure}[H]
    \centering
    \includegraphics[width=0.8\textwidth]{Figures/Chapter_14/slide_19.jpg}
    \caption{\textbf{AlexNet as a backbone:} Early implementations of Fast R-CNN explored the use of AlexNet for feature extraction. Only the last two FC layers were used for the per-region network.}
    \label{fig:chapter14_alexnet}
\end{figure}

\begin{figure}[H]
    \centering
    \includegraphics[width=0.8\textwidth]{Figures/Chapter_14/slide_20.jpg}
    \caption{\textbf{ResNet as a backbone:} More modern implementations utilize ResNet for feature extraction, leveraging deeper architectures for improved accuracy. In this case, only the last stage of the network was used for the per-region network, while the rest of the network was used as a backbone deriving features from the entire image.}
    \label{fig:chapter14_resnet}
\end{figure}

An interesting observation is that both approaches use a \textbf{fully convolutional backbone}. 

\noindent This is deliberate, as a fully convolutional network produces a dense, spatially organized feature map in which each element corresponds directly to a specific location in the input image. 

\newpage
This spatial correspondence is critical for RoI pooling: it allows us to accurately map the coordinates of a region proposal (generated in the original image space) onto the feature map, so that the correct features can be “cropped out” and later pooled into a fixed-size representation.

In contrast, if the backbone ended with fully connected layers, the spatial arrangement would be lost because fully connected layers mix information from all locations. Without a maintained spatial structure, there would be no straightforward way to project a region proposal onto the feature map. Consequently, each proposal would have to be processed individually from the image itself—defeating the purpose of using a shared, efficient feature extractor.

\subsection{Region of Interest (RoI) Pooling}
\label{subsec:chapter14_roi_pooling}

In Fast R-CNN, we aim to extract feature maps corresponding to each region proposal while ensuring that the process remains differentiable so we can backpropagate gradients through the backbone CNN. This challenge is addressed using \textbf{Region of Interest (RoI) Pooling}.

\paragraph{Mapping Region Proposals onto the Feature Map}
Region proposals—typically generated by methods such as selective search—are initially defined in the coordinate space of the original input image. However, because the backbone CNN downsamples the input by a factor \( k \) (e.g., \( k = 16 \)), these coordinates must be mapped onto the feature map. This transformation is given by:

\[
x' = \frac{x}{k}, \quad y' = \frac{y}{k}, \quad w' = \frac{w}{k}, \quad h' = \frac{h}{k}
\]

where \( (x, y, w, h) \) represents the original coordinates and dimensions of the region proposal on the input image, and \( (x', y', w', h') \) represents the corresponding region on the feature map.

Since this division typically results in non-integer values (e.g., \( x' = 9.25 \)), the coordinates are quantized—usually by taking the floor function:

\[
x'' = \lfloor x' \rfloor, \quad y'' = \lfloor y' \rfloor, \quad w'' = \lfloor w' \rfloor, \quad h'' = \lfloor h' \rfloor
\]

This snapping operation ensures that proposals align with the discrete grid of the feature map, making it possible to extract features corresponding to each proposal.

\paragraph{Dividing the Region into Fixed Bins}
Once the region proposal is mapped onto the feature map, the corresponding feature region is divided into a \textbf{fixed number of bins}. This binning ensures that all proposals—regardless of their original aspect ratio—are resized to a common spatial dimension. For example, if the target output size is \( 7 \times 7 \), the extracted region is divided into \( 7 \times 7 \) roughly equal spatial sub-regions.

\paragraph{Max Pooling within Each Bin}
For each bin, max pooling is applied across all the activations in that sub-region. This operation selects the maximum value within each bin, reducing variable-sized proposals to a uniform output shape while preserving strong feature responses. The output of RoI pooling for each proposal has a fixed spatial size, e.g., \( 7 \times 7 \times C \), where \( C \) is the number of channels in the feature map.

\begin{figure}[H]
    \centering
    \includegraphics[width=0.8\textwidth]{Figures/Chapter_14/slide_37.jpg}
    \caption{\textbf{RoI Pooling Process.} Each region proposal is mapped onto the feature map, divided into fixed bins, and max-pooled to a fixed output size for classification and bounding box refinement.}
    \label{fig:chapter14_roi_pooling}
\end{figure}

\paragraph{Summary: Key Steps in RoI Pooling}
\begin{enumerate}
    \item \textbf{Scaling Region Proposals}: The bounding box proposals are initially given in the coordinate space of the original image. Since the backbone CNN downsamples the input by a factor \( k \) (e.g., \( k=16 \)), the proposals must be scaled accordingly.
    \item \textbf{Extracting Feature Patches}: The scaled bounding boxes are mapped to the corresponding feature map locations, ensuring alignment with the CNN's output resolution.
    \item \textbf{Dividing into Sub-Regions}: Each extracted feature patch is divided into a fixed grid of bins (e.g., \( 7 \times 7 \)), regardless of the original proposal size.
    \item \textbf{Max Pooling per Sub-Region}: Within each bin, max pooling is applied to obtain a single representative feature value.
    \item \textbf{Fixed Output Size}: The final output for each proposal is a tensor of shape(num\_proposals, num\_channels, output\_size, output\_size), making it suitable for downstream classification and bounding box regression.
\end{enumerate}

The RoI Pooling operation can be implemented in PyTorch using a custom function that extracts fixed-size feature maps from region proposals. There is a nice implementation of \cite{patnaik2020_roi_pool} that follows the steps outlined earlier. If you want to understand how this method works in more detail, this is a good place to start. 

\paragraph{Limitations of RoI Pooling}
A key limitation of RoI pooling is the \textbf{quantization error} introduced during the coordinate snapping process. Since features are assigned to discrete grid locations using floor division, minor localization errors may occur, reducing detection accuracy. This problem becomes more prominent in tasks requiring precise bounding box localization.

\begin{figure}[H]
    \centering
    \includegraphics[width=0.8\textwidth]{Figures/Chapter_14/roi_pooling_downside.png}
    \caption{Impact of quantization in RoI Pooling. When mapping a region proposal onto the feature map (red), quantization (orange) can result in the loss of relevant object information (highlighted in \emph{dark blue}) while also introducing unwanted features from adjacent areas (\emph{green}). This misalignment reduces localization precision, as certain parts of the object may be omitted, while non-object features may be included in the pooled representation. Figure taken from \cite{erdem2020_RoIAlign}.}
    \label{fig:chapter14_roi_pooling_downside}
\end{figure}

In addition, the fact that sub-regions are not always of the same size is also weird and may prove to be sub-optimal. Due to these problems, an improved approach called \textbf{RoIAlign} emerged. RoIAlign eliminates quantization errors by using \textbf{bilinear interpolation} instead of rounding coordinates to the nearest discrete pixel. In the next section, we will explore how RoIAlign refines feature extraction to improve object detection accuracy. Although not used in Faster R-CNN, it made its way to consequent papers like Mask R-CNN that we'll cover later.

\subsection{RoIAlign}
\label{subsubsec:roi_align_intro}
In RoIAlign we avoid any quantization (rounding) of the coordinates. Instead, we sample the feature map using bilinear interpolation to obtain sub-pixel accuracy and preserve alignment. The idea is to compute a linear combination of feature values based on their Euclidean distance to the sampling point. By doing so, each sub-region in the region of interest contributes a weighted average of the feature map's values, thus preventing misalignments introduced by discrete rounding.

\subsubsection{RoIAlign: A Visual Example}
\label{subsubsec:roi_align_example}

To further understand how RoIAlign works in practice, we follow a step-by-step example inspired by Justin's lecture and \cite{patnaik2020_roi_pool}, of which the code snippets are taken (with extra documentation I added to make it a bit more clear). This example applies RoIAlign to a region proposal of a cat image projected onto the activation/feature map. For simplicity, we use an output size of $2 \times 2$, meaning the proposal is divided into four equal-sized sub-regions (bins), and we extract a single representative value per bin. In practice, output sizes of $7 \times 7, 14 \times 14$ are more reasonable and common.

\paragraph{Step 1: Projection of Region Proposal onto the Feature Map}
First, we map the region proposal onto the feature map \emph{without quantization}. The projected region is divided into $2 \times 2$ bins.

\begin{figure}[H]
    \centering
    \includegraphics[width=0.8\textwidth]{Figures/Chapter_14/slide_38.jpg}
    \caption{Projection of the region proposal onto the feature map, dividing it into $2 \times 2$ bins.}
    \label{fig:chapter14_roi_align_projection}
\end{figure}

\paragraph{Step 2: Selecting Interpolation Points in Each Bin}
In RoIAlign, each bin within a region proposal is divided into regularly spaced sampling points to avoid quantization errors. Instead of snapping to the nearest discrete grid like in RoI Pooling, RoIAlign selects \textbf{four interpolation points per bin} to estimate the feature value using bilinear interpolation.

For each bin, four sample points are computed as follows:
\begin{itemize}
    \item \( (x_1, y_1) \) – Top-left interpolation point
    \item \( (x_1, y_2) \) – Bottom-left interpolation point
    \item \( (x_2, y_1) \) – Top-right interpolation point
    \item \( (x_2, y_2) \) – Bottom-right interpolation point
\end{itemize}

As reminder, here is the part of the code in the RoIAlign method, used to compute the points to interpolate within each region of the projected proposal. 
\begin{mintedbox}{python}
    for i in range(self.output_size):
        for j in range(self.output_size):
            x_bin_strt = i * w_stride + xp0  # Bin's top-left x coordinate
            y_bin_strt = j * h_stride + yp0  # Bin's top-left y coordinate
    
            # Generate 4 points for interpolation (no rounding!)
            x1 = torch.Tensor([x_bin_strt + 0.25 * w_stride])  # Quarter into the bin
            x2 = torch.Tensor([x_bin_strt + 0.75 * w_stride])  # Three-quarters inside
            y1 = torch.Tensor([y_bin_strt + 0.25 * h_stride])  # # Quarter into the bin
            y2 = torch.Tensor([y_bin_strt + 0.75 * h_stride])  # Three-quarters inside
            
            # Bilinear interpolation will be performed at (x1, y1), (x1, y2), (x2, y1), and (x2, y2), and these values will be used to compute the final bin output for the per-region network. 
\end{mintedbox}

For each bin (sub-region), two sample points are taken along both the \(x\)-axis and \(y\)-axis, creating a total of \(2 \times 2 = 4\) sample points. The interpolation points are systematically selected as:

\[
\{x_1, x_2\} \times \{y_1, y_2\}
\]

ensuring comprehensive coverage within the bin.

\begin{figure}[H]
    \centering
    \includegraphics[width=0.8\textwidth]{Figures/Chapter_14/slide_39.jpg}
    \caption{Selection of four interpolation points in each sub-region for bilinear interpolation.}
    \label{fig:chapter14_roi_align_interpolation_points}
\end{figure}

\subparagraph{Why Choose 0.25 and 0.75 for Sampling?}
Instead of selecting points at the exact center of each bin (\(0.5\)) or at its edges (\(0.0\) and \(1.0\)), RoIAlign samples points at \(0.25\) and \(0.75\) of the bin’s width and height. This design choice serves several purposes:

\begin{itemize}
    \item \textbf{Avoiding boundary artifacts:} Sampling at \(0.0\) (bin edges) can cause rounding errors or unexpected shifts due to floating-point imprecision. Sampling at \(0.25\) and \(0.75\) keeps the points well inside the bin, ensuring they stay within the intended spatial region.
    
    \item \textbf{Capturing feature variation:} Sampling at just one location (e.g., the center at \(0.5\)) might miss important variations within the bin. By selecting two points per axis, we better approximate the feature distribution in that region.
    
    \item \textbf{Consistent coverage:} This approach systematically captures more representative “average” features, reducing the impact of noise and ensuring smooth gradient flow during backpropagation.
\end{itemize}

While RoIAlign typically uses a \(2 \times 2\) grid of sample points per bin, some implementations allow configurable sampling ratios, such as \(3 \times 3\) or higher, to improve approximation accuracy at the cost of additional computation.

By eliminating quantization artifacts and ensuring precise feature extraction, this step significantly enhances the quality of extracted region features, making RoIAlign an essential improvement over RoI Pooling.

\paragraph{Step 3: Mapping Sampled Points onto the Feature Grid}
Each of the four sampled points per bin lies within the continuous feature map, requiring us to determine its surrounding discrete grid points for bilinear interpolation. Given a sampled point \((x, y)\), it is enclosed by four neighboring integer grid points:

\begin{itemize}
    \item \( a: (x_0, y_0) \) – Top-left corner
    \item \( b: (x_0, y_1) \) – Bottom-left corner
    \item \( c: (x_1, y_0) \) – Top-right corner
    \item \( d: (x_1, y_1) \) – Bottom-right corner
\end{itemize}

\begin{figure}[H]
    \centering
    \includegraphics[width=0.8\textwidth]{Figures/Chapter_14/slide_41.jpg}
    \caption{Mapping of the selected interpolation points onto the discrete grid of the feature map. Each sampled point is enclosed by four neighboring grid points, which will be used in bilinear interpolation.}
    \label{fig:chapter14_roi_align_interpolation_grid}
\end{figure}

In our example, in the bottom-right bin, we consider a sampled point at \((x_2, y_2) = (6.5, 5.8)\) that is also the bottom-right point within the bin. The nearest integer grid points that enclose it are:

\[
a=(x_0 = 6, y_0 = 5), \quad b=(x_0 = 6, y_1 = 6), \quad c=(x_1 = 7, y_0 = 5), \quad d=(x_1 = 7, y_1 = 6).
\]

These four points are used for interpolation, ensuring that each sampled feature value is derived from its surrounding grid points rather than being snapped to the nearest one.

To determine these enclosing grid points programmatically, we perform the following computations:

\begin{mintedbox}{python}
    # Find the integer corners surrounding (x, y)
    x0 = torch.floor(x).type(torch.cuda.LongTensor)
    x1 = x0 + 1
    y0 = torch.floor(y).type(torch.cuda.LongTensor)
    y1 = y0 + 1
    
    # Clamp these coordinates to the image boundary to avoid out-of-range indexing
    x0 = torch.clamp(x0, 0, img.shape[1] - 1)
    x1 = torch.clamp(x1, 0, img.shape[1] - 1)
    y0 = torch.clamp(y0, 0, img.shape[0] - 1)
    y1 = torch.clamp(y1, 0, img.shape[0] - 1)
    
    # Extract feature values at the four surrounding grid points
    Ia = img[y0, x0]  # Top-left corner
    Ib = img[y1, x0]  # Bottom-left corner
    Ic = img[y0, x1]  # Top-right corner
    Id = img[y1, x1]  # Bottom-right corner
\end{mintedbox}

These four feature values \((I_a, I_b, I_c, I_d)\) serve as the basis for bilinear interpolation. Instead of directly snapping \((x, y)\) to the nearest feature grid location, we compute a weighted average of these values, using their relative distances as interpolation weights.

By mapping sampled points onto discrete grid locations in this manner, RoIAlign ensures that every proposal maintains precise alignment with the backbone's feature map, preserving sub-pixel accuracy and avoiding misalignment errors caused by quantization.

\paragraph{Step 4: Computing Bilinear Interpolation Weights}
Once the four nearest integer grid points for a sampled point \((x,y)\) have been identified, we compute weights that determine each corner’s contribution to the interpolated value. These weights are based on the relative distances between \((x,y)\) and the four grid points.

\subparagraph{Normalization Constant and Its Interpretation}
The normalization constant is given by
\[
\text{norm\_const} = \frac{1}{(x_1 - x_0)(y_1 - y_0)},
\]
which is the inverse of the area of the rectangle formed by the grid points \((x_0,y_0)\), \((x_1,y_0)\), \((x_0,y_1)\), and \((x_1,y_1)\). In many cases, including our example, this rectangle is a unit square (i.e., \(x_1 - x_0 = 1\) and \(y_1 - y_0 = 1\)), so the normalization constant is 1. This constant ensures that the computed weights form a convex combination that sums to 1.

\subparagraph{Weight Computation for Each Corner}
For a sampled point \((x,y) = (6.5, 5.8)\), assume the four surrounding grid points are:
\[
(x_0,y_0) = (6,5), \quad (x_1,y_0) = (7,5), \quad (x_0,y_1) = (6,6), \quad (x_1,y_1) = (7,6).
\]
We compute the distances:
\[
x_1 - x = 7 - 6.5 = 0.5, \quad x - x_0 = 6.5 - 6 = 0.5,
\]
\[
y_1 - y = 6 - 5.8 = 0.2, \quad y - y_0 = 5.8 - 5 = 0.8.
\]
The weight for each grid point is the product of the fractional distances along the x and y axes, meaning, each weight is determined by how far the sampled point is from a particular corner, considering both x and y distances. The horizontal and vertical contributions are combined as:

- \((x_1 - x) / (x_1 - x_0)\) → Fraction of the width from \((x, y)\) to the right boundary.
- \((x - x_0) / (x_1 - x_0)\) → Fraction from \((x, y)\) to the left boundary.
- \((y_1 - y) / (y_1 - y_0)\) → Fraction of the height from \((x, y)\) to the bottom boundary.
- \((y - y_0) / (y_1 - y_0)\) → Fraction from \((x, y)\) to the top boundary.

Therefore, for the top-left corner (denoted \(w_a\)), the weight is given by:
\[
w_a = (x_1 - x) \cdot (y_1 - y) = 0.5 \times 0.2 = 0.1.
\]
Similarly, for the top-right corner (denoted \(w_c\)):
\[
w_c = (x - x_0) \cdot (y_1 - y) = 0.5 \times 0.2 = 0.1.
\]
For the bottom-left corner (denoted \(w_b\)):
\[
w_b = (x_1 - x) \cdot (y - y_0) = 0.5 \times 0.8 = 0.4,
\]
and for the bottom-right corner (denoted \(w_d\)):
\[
w_d = (x - x_0) \cdot (y - y_0) = 0.5 \times 0.8 = 0.4.
\]
Thus, the weights satisfy
\[
w_a + w_b + w_c + w_d = 0.1 + 0.4 + 0.1 + 0.4 = 1.0.
\]


\begin{figure}[H]
    \centering
    \includegraphics[width=0.8\textwidth]{Figures/Chapter_14/slide_42.jpg}
    \caption{Computing interpolation weight for the top-left corner ($w_a$). Since the sampled point is far from this corner, its weight is relatively low: ($w_a=0.1$).}
    \label{fig:chapter14_roi_align_interpolation_weight_a}
\end{figure}

\begin{figure}[H]
    \centering
    \includegraphics[width=0.8\textwidth]{Figures/Chapter_14/slide_43.jpg}
    \caption{Computing interpolation weight for the top-right corner ($w_c$). Since this point is equidistant from $w_a$, the weights are equal ($w_a = w_c = 0.1$).}
    \label{fig:chapter14_roi_align_interpolation_weight_c}
\end{figure}

\begin{figure}[H]
    \centering
    \includegraphics[width=0.8\textwidth]{Figures/Chapter_14/slide_44.jpg}
    \caption{Computing interpolation weight for the bottom-left corner ($w_b$). Since the sampled point is much closer to this corner, its weight is significantly higher: ($w_b=0.4$).}
    \label{fig:chapter14_roi_align_interpolation_weight_b}
\end{figure}

\begin{figure}[H]
    \centering
    \includegraphics[width=0.8\textwidth]{Figures/Chapter_14/slide_45.jpg}
    \caption{Computing interpolation weight for the bottom-right corner ($w_d$). This weight is identical to $w_b$, because the sampled point $(x,y)$ is symmetrically placed between $b, d$.}
    \label{fig:chapter14_roi_align_interpolation_weight_d}
\end{figure}

\paragraph{Step 5: Computing the Interpolated Feature Value}
Once the interpolation weights have been determined, we compute the interpolated feature value at \((x, y)\) as a weighted sum of the four surrounding feature grid values:

\[
f_{xy} = w_a f_{x_0 y_0} + w_b f_{x_0 y_1} + w_c f_{x_1 y_0} + w_d f_{x_1 y_1}
\]

Each weight determines the contribution of the corresponding grid point to the interpolated value. Since closer grid points have higher weights, they exert more influence over the final value than those further away.

\subparagraph{\textbf{Example Computation}}
For the sampled point \((x,y) = (6.5, 5.8)\), using previously computed weights:

\[
w_a = 0.1, \quad w_b = 0.4, \quad w_c = 0.1, \quad w_d = 0.4
\]

and the corresponding feature values from the activation map:

\[
I_a = f_{6,5}, \quad I_b = f_{6,6}, \quad I_c = f_{7,5}, \quad I_d = f_{7,6}
\]

we compute the interpolated feature value as:

\[
f_{6.5,5.8} = (0.1 \times f_{6,5}) + (0.4 \times f_{6,6}) + (0.1 \times f_{7,5}) + (0.4 \times f_{7,6})
\]

\paragraph{Step 6: Aggregating Interpolated Values}
After computing the interpolated feature values for all sampled points, we aggregate them using either:
\begin{itemize}
    \item \textbf{Average pooling}: The final value is the mean of all interpolated feature values.
    \item \textbf{Max pooling}: The final value is the maximum of all interpolated values.
\end{itemize}

In Justin’s example, max pooling is used:
\[
\text{bin value} = \max(v_1, v_2, v_3, v_4)
\]

\subparagraph{\textbf{Final Output}}
After iterating over all bins, the final RoI feature map is constructed, with each bin containing an aggregated value from bilinear interpolation. The per-proposal network then uses this structured feature representation for classification and bounding-box regression.

\begin{figure}[H]
    \centering
    \includegraphics[width=0.8\textwidth]{Figures/Chapter_14/slide_46.jpg}
    \caption{Final RoIAlign result: Each bin's value is determined via bilinear interpolation and pooling.}
    \label{fig:chapter14_roi_align_final}
\end{figure}

\paragraph{Key Takeaways}
\begin{itemize}
    \item RoIAlign eliminates the quantization error of RoI Pooling by leveraging bilinear interpolation.
    \item The interpolation process ensures precise feature extraction, leading to improved localization accuracy.
    \item The final feature map maintains a fixed size per RoI, making it compatible with subsequent per-region classifiers and regressors.
\end{itemize}

Hence, RoIAlign is a core component of modern architectures used for detection and segmentation like Mask R-CNN. 

\newpage
\paragraph{RoIAlign Important Implementation Parts in PyTorch}
Following the implementation of \cite{patnaik2020_roi_pool}, here are the important code snippets that illustrate how RoIAlign works, helping to see how the process looks like from start to finish. 

\begin{mintedbox}{python}
    def _roi_align(self, features, scaled_proposal):
        """Given feature layers and scaled proposals return bilinear interpolated
        points in feature layer
        
        Args:
        features (torch.Tensor): Tensor of shape <channels x height x width>
        scaled_proposal (list of torch.Tensor): Each tensor is a bbox by which we
        will extract features from features Tensor
        """
        
        _, num_channels, h, w = features.shape
        
        # (xp0, yp0) = top-left corner of projected proposal, (xp1, yp1) = bottom-right corner.
        xp0, yp0, xp1, yp1 = scaled_proposal
        p_width = xp1 - xp0
        p_height = yp1 - yp0
        
        '''
        If we want to output a nxn tensor to the per-proposal network, then output_size=n.
        The number of sub-regions we'll produce, like in RoIPool, will be nxn as well.
        The height and width of each sub-region will be equal, as the regions are now of exactly the same size, 
        but crucially we no longer snap to integer boundaries. 
        Each sub-region's representative value will be a linear combination of the pixel values 
        that this sub-region covers (via bilinear interpolation).
        '''
        w_stride = p_width / self.output_size  # The width of each sub-region
        h_stride = p_height / self.output_size # The height of each sub-region
        
        interp_features = torch.zeros((num_channels, self.output_size, self.output_size))
    
        for i in range(self.output_size):
            for j in range(self.output_size):
                # top-left x coordinate of the i-th sub-region
                x_bin_strt = i * w_stride + xp0
                # top-left y coordinate of the j-th sub-region
                y_bin_strt = j * h_stride + yp0
                
                # generate 4 points for interpolation (no rounding!)
                x1 = torch.Tensor([x_bin_strt + 0.25*w_stride]) # quarter in the bin (x-axis)
                x2 = torch.Tensor([x_bin_strt + 0.75*w_stride]) # three-quarters in the bin (x-axis)
                y1 = torch.Tensor([y_bin_strt + 0.25*h_stride]) # quarter in the bin (y-axis)
                y2 = torch.Tensor([y_bin_strt + 0.75*h_stride]) # three-quarters in the bin (y-axis)
                
                '''
                We sample 2 points along x (0.25 and 0.75 of the bin width) 
                and 2 points along y (0.25 and 0.75 of the bin height).
                This yields 2 x 2 = 4 sample points per bin.
                
                Why at 0.25 and 0.75?
                1) Avoid boundaries: Sampling at 0 or 1 might cause rounding/boundary issues.
                2) Capture variation: Multiple sample points per bin help represent 
                the internal structure better than a single center point.
                3) Consistent coverage: 0.25 and 0.75 systematically offer an even "spread" 
                in each dimension, approximating the average effectively.
                '''
                
                for c in range(num_channels):
                # features[0, c] is the single-channel feature map for channel c
                img = features[0, c]
                v1 = bilinear_interpolate(img, x1, y1)
                v2 = bilinear_interpolate(img, x1, y2)
                v3 = bilinear_interpolate(img, x2, y1)
                v4 = bilinear_interpolate(img, x2, y2)
                
                '''
                v1, v2, v3, v4 are the bilinear-interpolated values at the four sample points.
                We average these 4 values to get a single value for bin (i, j) and channel c.
                Note: In some cases, one might take max instead of average 
                (mimicking max pooling). This is what Justin shows in the lecture. Hence, he takes max(v1, v2, v3, v4) instead. 
                '''
                interp_features[c, j, i] = (v1 + v2 + v3 + v4) / 4
        
        return interp_features
\end{mintedbox}

We now understand the RoIAlign high-level flow. Next, let us examine how bilinear interpolation works for the four regularly sampled points inside each bin, of which we'll compute the output bin value for the per-proposal network later. 

\newpage

\begin{mintedbox}{python}
    def bilinear_interpolate(img, x, y):
        ''' We are given a point (x,y) that might not be a pixel coordinate,
        and we want to interpolate its feature value from the surrounding pixels. 
        '''
        
        # find the integer corners that surround (x, y)
        x0 = torch.floor(x).type(torch.cuda.LongTensor)
        x1 = x0 + 1
        y0 = torch.floor(y).type(torch.cuda.LongTensor)
        y1 = y0 + 1
        
        # clamp these coordinates to the image boundary to avoid indexing out of range
        x0 = torch.clamp(x0, 0, img.shape[1] - 1)
        x1 = torch.clamp(x1, 0, img.shape[1] - 1)
        y0 = torch.clamp(y0, 0, img.shape[0] - 1)
        y1 = torch.clamp(y1, 0, img.shape[0] - 1)
        
        # top-left, bottom-left, top-right, bottom-right corner values
        Ia = img[y0, x0]
        Ib = img[y1, x0]
        Ic = img[y0, x1]
        Id = img[y1, x1]
        
        '''
        Next, we compute the weights for each corner. The idea:
        - (x1 - x) -> how far we are from the right edge in the x direction
        - (x - x0) -> how far we are from the left edge in the x direction
        - (y1 - y) -> how far we are from the bottom edge in the y direction
        - (y - y0) -> how far we are from the top edge in the y direction
        
        We multiply these "partial distances" and then normalize by the total "area" 
        ( (x1 - x0)*(y1 - y0) ) so that wa+wb+wc+wd = 1.
        '''
        
        norm_const = 1 / ((x1.type(torch.float32) - x0.type(torch.float32)) *
        (y1.type(torch.float32) - y0.type(torch.float32)))
        
        wa = (x1.type(torch.float32) - x) * (y1.type(torch.float32) - y) * norm_const
        wb = (x1.type(torch.float32) - x) * (y - y0.type(torch.float32)) * norm_const
        wc = (x - x0.type(torch.float32)) * (y1.type(torch.float32) - y) * norm_const
        wd = (x - x0.type(torch.float32)) * (y - y0.type(torch.float32)) * norm_const
        
        # final bilinear interpolation: weighted sum of the four corners
        return torch.t(torch.t(Ia) * wa) + torch.t(torch.t(Ib) * wb) + \
        torch.t(torch.t(Ic) * wc) + torch.t(torch.t(Id) * wd)
\end{mintedbox}

\section{Faster R-CNN: Faster Proposals Using RPNs}
\label{sec:chapter14_faster_rcnn}

\subsection{Fast R-CNN Bottleneck: Region Proposal Computation}
Although Fast R-CNN optimized the detection pipeline, the slowest component remained the region proposal generation. The external algorithm used, such as Selective Search, was still running on the CPU, making it a major bottleneck.

\begin{figure}[H]
    \centering
    \includegraphics[width=0.8\textwidth]{Figures/Chapter_14/slide_50.jpg}
    \caption{Problem: Despite Fast R-CNN’s optimizations, runtime is still dominated by region proposal computation. Selective Search runs on the CPU and remains the slowest part of the pipeline.}
    \label{fig:chapter14_runtime_bottleneck}
\end{figure}

As shown in Figure \ref{fig:chapter14_runtime_bottleneck}, even though feature extraction and classification were now efficient, generating proposals using heuristic-based methods still consumed a significant portion of the runtime. 

\subsection{Towards Faster Region Proposals: Learning Proposals with CNNs}
The natural next step in improving object detection efficiency was to replace the handcrafted, CPU-based proposal generation process with a learnable, CNN-based alternative. Faster R-CNN introduced the \textbf{Region Proposal Network (RPN)} \cite{ren2016_fasterrcnn}, an architecture that predicts object proposals directly from the feature maps produced by the backbone CNN. This approach integrates proposal generation into the deep learning pipeline, eliminating the need for slow external algorithms.

The key idea behind RPNs is:
\begin{itemize}
    \item Use convolutional feature maps to directly predict high-quality object proposals.
    \item Train the proposal generator jointly with the rest of the detection pipeline.
    \item Make the entire object detection process fully differentiable and GPU-accelerated.
\end{itemize}

By replacing Selective Search with an RPN, Faster R-CNN eliminates the last major bottleneck in Fast R-CNN and makes object detection significantly faster while maintaining high accuracy. In the next section, we will explore the details of Region Proposal Networks and their role in Faster R-CNN.

\subsection{Region Proposal Networks (RPNs)}
\label{subsec:chapter14_rpn}

\paragraph{\textbf{How RPNs Work}} 
Instead of using a separate region proposal algorithm, RPNs generate proposals directly from the shared feature map produced by a deep CNN backbone. The process follows these steps:

\begin{enumerate}
    \item \textbf{Feature Extraction:} The backbone CNN extracts a feature map from the input image while preserving spatial alignment.
    \item \textbf{Anchor Generation:} At each spatial location on the feature map, predefined \textit{anchor boxes} (of multiple sizes and aspect ratios) serve as candidate proposals.
    \item \textbf{Objectness Classification:} A small convolutional layer predicts whether each anchor contains an object.
    \item \textbf{Bounding Box Regression:} For positive anchors, another convolutional layer predicts the transformation required to refine the anchor into a better-fitting bounding box.
\end{enumerate}

Since the RPN operates directly on the shared feature map, it \textbf{adds minimal computational cost}—it is simply a small set of convolutional layers applied to the extracted backbone features. This allows the model to generate high-quality proposals without needing separate, slow region proposal methods.

\paragraph{\textbf{Anchor Boxes: Handling Scale and Aspect Ratio Variations}}  
In object detection, objects appear in diverse shapes and sizes. A single fixed-size proposal per spatial location would fail to capture this variability. To address this, RPNs generate proposals using a set of predefined \textbf{anchor boxes} at each spatial location on the feature map. Each anchor serves as a \textbf{reference box} that can be classified and refined to better fit actual objects.  

\begin{figure}[H]
    \centering
    \includegraphics[width=0.8\textwidth]{Figures/Chapter_14/slide_59.jpg}
    \caption{Anchor boxes and their classification: Positive (green) anchors contain objects, while negative (red) anchors do not.}
    \label{fig:chapter14_rpn_anchor_classification}
\end{figure}

At each spatial location, RPNs generate \textbf{$K$ anchors} with:  
\begin{itemize}
    \item \textbf{Different scales} – Capturing small, medium, and large objects.
    \item \textbf{Different aspect ratios} – Adapting to tall, square, and wide objects.
\end{itemize}

\begin{figure}[H]
    \centering
    \includegraphics[width=0.65\textwidth]{Figures/Chapter_14/slide_68.jpg}
    \caption{Examples of $K$ anchor boxes at a single location, illustrating different sizes and aspect ratios.}
    \label{fig:chapter14_rpn_anchors_sizes}
\end{figure}

The original Faster R-CNN paper used \textbf{9 anchors per location} (3 scales $\times$ 3 aspect ratios). For each anchor, the RPN predicts:

\begin{itemize}
	\item \textbf{Objectness Score} -- A binary classification indicating whether the anchor contains a foreground object or belongs to background. Conceptually, this is just \emph{logistic regression}: for each anchor we want a probability $p(\text{object} \mid \text{anchor})$. In practice, most implementations parameterize this as \emph{two logits per anchor} (foreground and background) and apply a softmax followed by a cross-entropy loss. For the binary case, this two-logit softmax formulation is mathematically equivalent to a single-logit sigmoid (standard logistic regression); it is simply more convenient to implement and extend to multi-class settings.
	\item \textbf{Bounding Box Transform} -- A transformation $(t_x, t_y, t_w, t_h)$ refining the anchor box.
\end{itemize}

These predictions are made using a small CNN applied to the feature map. The classification branch outputs a \textbf{$2K$-channel score map} (for $K$ anchors per location), i.e., for each spatial location it predicts two logits (foreground / background) for each of the $K$ anchors. If the RPN feature map has spatial size $5 \times 6$, this corresponds to a tensor of shape $2K \times 5 \times 6$ per training image. The regression branch outputs a \textbf{$4K$-channel transform map} per spatial location, yielding an output tensor of shape $4K \times 5 \times 6$ per training image.

\begin{figure}[H]
    \centering
    \includegraphics[width=0.65\textwidth]{Figures/Chapter_14/slide_60.jpg}
    \caption{RPN predicting objectness scores and bounding box transforms for each anchor.}
    \label{fig:chapter14_rpn_predictions}
\end{figure}

\paragraph{\textbf{Bounding Box Refinement: Aligning Anchors to Objects}}  
Even with multiple anchors per location, an anchor may not perfectly match an object’s true dimensions. To improve localization, the RPN predicts a refinement transformation, similar to what R-CNN and Fast R-CNN do for final detections. For details on bounding box transformations, refer to \textbf{Section~\ref{subsubsec:chapter13_bbox_regression}}.  

The refinement transformation is parameterized as follows:

\[
t_x = \frac{b_x - p_x}{p_w}, \quad
t_y = \frac{b_y - p_y}{p_h}, \quad
t_w = \ln \left( \frac{b_w}{p_w} \right), \quad
t_h = \ln \left( \frac{b_h}{p_h} \right)
\]

where $(p_x, p_y, p_w, p_h)$ are the anchor box parameters and $(b_x, b_y, b_w, b_h)$ are the refined bounding box parameters.

\begin{figure}[H]
    \centering
    \includegraphics[width=0.8\textwidth]{Figures/Chapter_14/slide_62.jpg}
    \caption{For positive anchors (green), the RPN predicts a transformation (orange) that converts the anchor to the ground-truth bounding box (gold).}
    \label{fig:chapter14_rpn_box_transform}
\end{figure}

Unlike traditional proposal generation methods, RPNs train the proposal generation process jointly with the feature extraction backbone, allowing the network to \textbf{learn proposals that are well-suited for the final detection task}. This integration improves both accuracy and computational efficiency.

\paragraph{\textbf{Training RPNs: Assigning Labels to Anchors}}
To train a Region Proposal Network (RPN), we must assign labels to the anchor boxes, distinguishing between \textbf{positive}, \textbf{negative}, and \textbf{neutral} examples. This labeling process is crucial for optimizing both classification (objectness score) and bounding box regression.

\begin{itemize}
    \item \textbf{Positive anchors}: Anchors that have an \textbf{IoU $\geq 0.7$} with at least one ground-truth box are considered positive.
    \item \textbf{Negative anchors}: Anchors with \textbf{IoU $< 0.3$} with all ground-truth boxes are labeled negative.
    \item \textbf{Neutral anchors}: Anchors with an IoU between \textbf{0.3 and 0.7} are ignored during training.
\end{itemize}

Since anchor boxes serve as a reference for object detection, \textbf{positive anchors} are used to compute both classification and regression losses.

\newpage

\textbf{Negative anchors}, on the other hand, only contribute to the classification loss, ensuring the RPN learns to distinguish objects from background effectively.

\paragraph{\textbf{Loss Function for RPN Training}}
The RPN is trained using a \textbf{multi-task loss function} that jointly optimizes \textbf{object classification} and \textbf{bounding box regression}:

\[
L(\{p_i\}, \{t_i\}) = \frac{1}{N_{\text{cls}}} \sum_{i} L_{\text{cls}}(p_i, p^*_i) + \lambda \frac{1}{N_{\text{reg}}} \sum_{i} p^*_i L_{\text{reg}}(t_i, t^*_i)
\]

where:
\begin{itemize}
    \item \( p_i \) is the predicted probability of anchor \( i \) containing an object.
    \item \( p^*_i \) is the ground-truth label (1 for objects, 0 for background).
    \item \( t_i \) is the predicted bounding box transform for anchor \( i \).
    \item \( t^*_i \) is the ground-truth bounding box transform.
    \item \( L_{\text{cls}} \) is the \textbf{binary cross-entropy loss} for object classification.
    \item \( L_{\text{reg}} \) is the \textbf{smooth \( L_1 \) loss} applied only to positive anchors.
    \item \( N_{\text{cls}} \) and \( N_{\text{reg}} \) are normalization terms.
    \item \( \lambda \) is a balancing factor, typically set to 10.
\end{itemize}

This loss function ensures that \textbf{classification and bounding box regression are optimized simultaneously}.

\subparagraph{\textbf{Assigning Ground-Truth Bounding Boxes to Anchors}}
Each \textbf{positive anchor} is assigned to the ground-truth box that has the \textbf{maximum IoU} with it. This ensures that the best-matching ground-truth object supervises the training of the anchor's bounding box regression.

\begin{itemize}
    \item If an anchor has \(\text{IoU} \geq 0.7\) with multiple ground-truth boxes, it is assigned to the object with which it has the highest IoU.
    \item Each ground-truth box must be matched to at least one anchor. If no anchor has \(\text{IoU} \geq 0.7\) with a given ground-truth box, the anchor with the highest IoU is forcibly assigned to it.
\end{itemize}

This matching process ensures that \textbf{all ground-truth objects are covered by at least one anchor}, enabling the RPN to propose accurate regions for all objects in an image.

\paragraph{\textbf{Smooth \( L_1 \) Loss for Bounding Box Regression}}
To refine anchor boxes into accurate region proposals, Faster R-CNN employs the \textbf{smooth \( L_1 \) loss}, which is defined as:

\[
L_{\text{reg}}(t_i, t^*_i) =
\begin{cases}
    0.5 (t_i - t^*_i)^2, & \text{if } |t_i - t^*_i| < 1 \\
    |t_i - t^*_i| - 0.5, & \text{otherwise}
\end{cases}
\]

This loss behaves like an \textbf{\( L_2 \) loss} (squared error) when the error is small, ensuring smooth gradients for small offsets. However, for larger errors, it switches to an \textbf{\( L_1 \) loss} (absolute error), preventing large outliers from dominating the training process.

\textbf{Why Smooth \( L_1 \) Instead of \( L_2 \) Loss?}
\begin{itemize}
    \item \textbf{Robustness to Outliers}: Unlike the \( L_2 \) loss, which heavily penalizes large errors, the smooth \( L_1 \) loss reduces the influence of extreme outliers.
    \item \textbf{Stable Training}: The transition from quadratic to linear loss ensures that large localization errors do not cause excessively high gradients, making optimization more stable.
    \item \textbf{Better Localization}: Since bounding box predictions can have large variations, the smooth \( L_1 \) loss allows more effective training, focusing on improving the fine alignment of predicted boxes.
\end{itemize}

By integrating the \textbf{smooth \( L_1 \) loss} into the RPN's training objective, Faster R-CNN achieves \textbf{more accurate and stable region proposals}, leading to improved object detection performance.

\paragraph{\textbf{Why Use Negative Anchors?}}
\textbf{Negative anchors} (IoU $<$ 0.3) play a crucial role in training the RPN. Without them, the model would lack supervision on how to classify background regions, leading to an excess of false positives. \textbf{Negative anchors}:
\begin{itemize}
    \item Ensure the RPN learns to reject background regions by reinforcing the binary classification task.
    \item Provide a balance between \textbf{object detection} and \textbf{background rejection}, making the system more robust (ensuring that the RPN does not overfit to detecting only foreground objects). 
\end{itemize}

\begin{enrichment}[Training Region Proposal Networks (RPNs)][subsubsection]
    \label{enrichment:rpn_training_pipeline}
    
    \noindent
    The \textbf{Region Proposal Network (RPN)} \cite{ren2015_fasterrcnn} is a learnable module for generating class-agnostic object proposals from convolutional feature maps. Below is a complete walkthrough of the training process.
    
    \paragraph{1. Input Feature Map}
    Given an input image \( I \in \mathbb{R}^{H \times W \times 3} \), a CNN backbone (e.g., VGG-16, ResNet-50) produces a feature map of spatial dimensions:
    \[
    F \in \mathbb{R}^{H' \times W' \times C'}, \quad \text{where } H' = H/s,\, W' = W/s.
    \]
    The stride \( s \) reflects total downsampling (often \( s = 16 \)).
    
    \paragraph{2. Sliding Window: Shared 3\(\times\)3 Conv}
    A shared \(3 \times 3\) conv is applied across all spatial locations to extract intermediate features:
    
    \begin{mintedbox}{python}
        # Shared intermediate 3x3 conv
        rpn_conv = nn.Conv2d(C_prime, 512, kernel_size=3, padding=1)
        inter_features = F.relu(rpn_conv(featmap))  # (B, 512, H', W')
    \end{mintedbox}
    
    Each spatial location corresponds to a position in the original image and will be associated with \(K\) anchor boxes.
    
    \paragraph{3. RPN Heads: Anchor-wise Classification and Regression}
    Two parallel \(1\times1\) conv layers produce:
    \begin{itemize}
        \item \textbf{Objectness scores:} \(2K\) channels (foreground vs. background for each anchor),
        \item \textbf{BBox deltas:} \(4K\) channels (\(\Delta x, \Delta y, \Delta w, \Delta h\) for each anchor).
    \end{itemize}
    
    \begin{mintedbox}{python}
        rpn_cls_logits = nn.Conv2d(512, 2 * K, kernel_size=1)(inter_features)
        rpn_bbox_deltas = nn.Conv2d(512, 4 * K, kernel_size=1)(inter_features)
    \end{mintedbox}
    
    These outputs are reshaped to \((B, H' \times W' \times K, 2)\) and \((B, H' \times W' \times K, 4)\) respectively during training for loss computation, to associate each anchor with its corresponding predictions:
    \begin{mintedbox}{python}
        rpn_cls_logits = rpn_cls_logits.permute(0, 2, 3, 1).reshape(B, -1, 2)
        rpn_bbox_deltas = rpn_bbox_deltas.permute(0, 2, 3, 1).reshape(B, -1, 4)
    \end{mintedbox}
    
    \paragraph{4. Anchor Labeling and Ground Truth Assignment}
    To train the network, we must determine which anchors are positive (object), negative (background), or ignored. For this, we compute the IoU (Intersection-over-Union) between each anchor and each ground-truth box:
    \begin{itemize}
        \item \textbf{Positive:} An anchor is labeled positive if it has an IoU \(\ge 0.7\) with any GT box, or if it is the highest-IoU anchor for a given GT.
        \item \textbf{Negative:} Labeled background if it has IoU \(\le 0.3\) with all GT boxes.
        \item \textbf{Ignored:} Anchors with intermediate IoU scores are not used in the loss.
    \end{itemize}
    
    \begin{mintedbox}{python}
        labels, matched_gt_boxes = assign_labels(all_anchors, gt_boxes)
        # labels: 1 = positive, 0 = negative, -1 = ignore
        pos_inds = torch.where(labels == 1)[0]  # Indices of positive anchors
        fg_bg_inds = torch.where(labels != -1)[0]  # Anchors involved in loss
    \end{mintedbox}
    
    \paragraph{5. Bounding-Box Regression Targets}
    For each \textbf{positive} anchor, we compute the offset required to transform the anchor into its assigned ground-truth box. These offsets form the regression \emph{targets}.
    
    Each target is parameterized as:
    \[
    \Delta x = \frac{x_{\text{gt}} - x_{\text{anchor}}}{w_{\text{anchor}}}, \quad
    \Delta y = \frac{y_{\text{gt}} - y_{\text{anchor}}}{h_{\text{anchor}}}, \quad
    \Delta w = \log \frac{w_{\text{gt}}}{w_{\text{anchor}}}, \quad
    \Delta h = \log \frac{h_{\text{gt}}}{h_{\text{anchor}}}.
    \]
    
    These values measure:
    \begin{itemize}
        \item The \emph{relative translation} (\(\Delta x, \Delta y\)) of the ground-truth box center w.r.t.\ the anchor box.
        \item The \emph{log-scale change} (\(\Delta w, \Delta h\)) needed to stretch the anchor's width/height to match the ground truth.
    \end{itemize}
    
    \begin{mintedbox}{python}
        bbox_targets = compute_regression_targets(anchors[pos_inds], matched_gt_boxes[pos_inds])
        # Shape: (N_pos, 4)
    \end{mintedbox}
    
    These targets serve as supervision: the network learns to predict these deltas for each positive anchor.
    
    \paragraph{6. Loss Computation}
    The RPN is trained using a multi-task loss:
    \[
    \mathcal{L}_{\text{RPN}} = 
    \frac{1}{N_{\text{cls}}} \sum_i \mathcal{L}_{\text{cls}}(p_i, p_i^*) 
    + \lambda \cdot 
    \frac{1}{N_{\text{reg}}} \sum_i \mathbb{1}_{\{p_i^* = 1\}} 
    \cdot \mathcal{L}_{\text{reg}}(t_i, t_i^*),
    \]
    where:
    \begin{itemize}
        \item \(p_i\): predicted objectness logits (before softmax),
        \item \(p_i^*\): binary GT label (1 for object, 0 for background),
        \item \(t_i\): predicted regression deltas (\texttt{rpn\_bbox\_deltas}),
        \item \(t_i^*\): GT regression target (\texttt{bbox\_targets}).
    \end{itemize}
    
    \begin{mintedbox}{python}
        cls_loss = F.cross_entropy(rpn_cls_logits[fg_bg_inds], labels[fg_bg_inds])
        reg_loss = smooth_l1_loss(rpn_bbox_deltas[pos_inds], bbox_targets)
        total_loss = cls_loss + lambda_ * reg_loss
    \end{mintedbox}
    
\textbf{Note:} During training, we do \emph{not} decode or apply the predicted deltas to anchors. Instead, we supervise the raw predicted deltas directly, using regression targets computed from fixed anchor–GT box pairs. This ensures stable optimization, as the anchors remain fixed while the network learns to output precise \((\Delta x, \Delta y, \Delta w, \Delta h)\) shifts. Only at inference time do we apply these predicted offsets to anchors to produce proposal boxes.

\end{enrichment}

\begin{figure}[H]
    \centering
    \includegraphics[width=0.8\textwidth]{Figures/Chapter_14/training_rpn_and_rpn_detections.jpg}
    \caption[Region Proposal Network and Example Detections]{\textbf{Left}: Region Proposal Network (RPN). \textbf{Right}: Example detections using RPN proposals on PASCAL VOC 2007 test. The method detects objects in a wide range of scales and aspect ratios. Source: \cite{ren2015_fasterrcnn}}
    \label{fig:training_rpn_and_rpn_detections}
\end{figure}

\paragraph{\textbf{Inference: Generating Region Proposals}}
At inference time, the RPN processes all anchor boxes across the image and filters out low-confidence proposals to retain the most relevant ones. The process consists of the following steps:

\begin{enumerate}
    \item \textbf{Compute objectness scores}: The classification branch predicts an \textbf{object score} for each anchor box.
    \item \textbf{Sort proposals by objectness score}: The top-scoring anchors are retained for further processing.
    \item \textbf{Apply Non-Maximum Suppression (NMS)}: Overlapping proposals with a high \textbf{IoU} are removed, keeping only the most confident detections.
    \item \textbf{Select the top $N$ proposals} (e.g., 300 proposals) as final region proposals for Fast R-CNN.
\end{enumerate}

By filtering out redundant and low-confidence proposals, this step improves both \textbf{efficiency} and \textbf{accuracy}, ensuring that only the most relevant regions are processed by the detector.

\newpage

\paragraph{\textbf{RPNs Improve Region Proposal Generation}}
Compared to previous region proposal methods like \textbf{Selective Search}, RPNs introduce several key advantages:
\begin{itemize}
	\item \textbf{Speed:} RPNs operate directly on the backbone’s shared feature map as a small conv head. Proposal generation becomes a single GPU pass instead of a slow, separate CPU algorithm.
	\item \textbf{Learned ``Objectness'':} Because the RPN is trained jointly with the detector, it learns which regions in feature space are likely to contain \emph{any} object, rather than relying on hand-crafted low-level grouping cues. This produces proposals that are more relevant to the downstream detection task (fewer obvious background regions, more boxes covering real objects).
	\item \textbf{More Precise Localization:} Each positive anchor is not only classified as “object vs.\ background,” but also refined by a learned bounding box regressor that predicts offsets $((t_x, t_y, t_w, t_h))$. This allows the network to \emph{adjust} coarse anchors to tightly hug the true object boundaries, resulting in proposals that overlap ground-truth boxes much more accurately than the fixed, heuristic boxes from Selective Search.
\end{itemize}

Thus, \textbf{Faster R-CNN} achieves \textbf{real-time object detection} by integrating RPNs and Fast R-CNN into a unified pipeline.

\subsection{Faster R-CNN Loss in Practice: Joint Training with Four Losses}
\label{subsec:chapter14_faster_rcnn_loss}

\paragraph{\textbf{Joint Training in Faster R-CNN}}
Unlike previous object detection pipelines where region proposal generation and object classification were trained separately, \textbf{Faster R-CNN jointly trains both the RPN and the object detector}. This results in a fully end-to-end learnable system with a \textbf{four-part loss function}:

\[
L = L_{\text{cls}}^{\text{RPN}} + L_{\text{reg}}^{\text{RPN}} + L_{\text{cls}}^{\text{Fast R-CNN}} + L_{\text{reg}}^{\text{Fast R-CNN}}
\]

\begin{itemize}
    \item \( L_{\text{cls}}^{\text{RPN}} \) – Classifies anchor boxes as object vs. background.
    \item \( L_{\text{reg}}^{\text{RPN}} \) – Refines anchor boxes to generate high-quality proposals.
    \item \( L_{\text{cls}}^{\text{Fast R-CNN}} \) – Classifies refined proposals into object categories.
    \item \( L_{\text{reg}}^{\text{Fast R-CNN}} \) – Further refines bounding box localization.
\end{itemize}

By training the RPN together with the detection network, the \textbf{region proposal generation and object detection become more aligned}, improving both efficiency and accuracy.

\paragraph{\textbf{How RPN Improves Inference Speed}}
Before Faster R-CNN, Fast R-CNN significantly reduced inference time compared to R-CNN by sharing computations. However, it still relied on external region proposal methods such as Selective Search, which were computationally expensive. Faster R-CNN eliminates this bottleneck by using RPN to generate region proposals directly from the feature map.

\begin{figure}[H]
    \centering
    \includegraphics[width=0.8\textwidth]{Figures/Chapter_14/slide_70.jpg}
    \caption{Comparison of inference time between R-CNN, SPP-Net, Fast R-CNN, and Faster R-CNN. RPN reduces the test-time speed from 2.3s in Fast R-CNN to 0.2s in Faster R-CNN.}
    \label{fig:chapter14_faster_rcnn_speed_comparison}
\end{figure}

\textbf{Key Takeaways:}
\begin{itemize}
    \item \textbf{Eliminating external region proposals} – Instead of using a separate CPU-based region proposal method (e.g., Selective Search), Faster R-CNN predicts region proposals using CNNs.
    \item \textbf{Fully convolutional region proposals} – The RPN operates as a small, efficient convolutional network on top of the shared feature map.
    \item \textbf{Dramatic speedup} – With RPN, the overall test-time speed improves from \textbf{2.3s in Fast R-CNN to just 0.2s in Faster R-CNN}, making real-time object detection more feasible.
\end{itemize}

By integrating \textbf{joint training}, \textbf{region proposal learning}, and \textbf{feature sharing}, Faster R-CNN achieves significant improvements over previous detectors, making it one of the most influential object detection models.

\newpage

\subsection{Feature Pyramid Networks (FPNs): Multi-Scale Feature Learning}
\label{subsec:chapter14_fpn}

Detecting objects of varying scales is a fundamental challenge in object detection. Traditional methods attempted to improve \textbf{scale invariance} by constructing an \textbf{image pyramid}, where the image is resized to multiple scales and processed separately by the detector. This approach is computationally expensive since the network must process the same image multiple times.

\begin{figure}[H]
    \centering
    \includegraphics[width=0.8\textwidth]{Figures/Chapter_14/slide_74.jpg}
    \caption{Illustration of the classic image pyramid approach, where the detector is applied to multiple resized versions of the image to improve small-object detection. However, this method is computationally expensive.}
    \label{fig:chapter14_image_pyramid}
\end{figure}

\subsubsection{Feature Pyramid Networks: A More Efficient Approach}
Rather than resizing the image, Lin et al. (2017) \cite{lin2017_fpn} proposed leveraging the inherent hierarchical structure of convolutional neural networks (CNNs). Since CNNs naturally extract features at multiple resolutions due to their deep architecture, FPNs \textbf{attach independent detectors to features from multiple levels of the backbone}. This enables the model to handle objects at different scales without requiring multiple forward passes.

\begin{figure}[H]
    \centering
    \includegraphics[width=0.8\textwidth]{Figures/Chapter_14/slide_76.jpg}
    \caption{Applying object detectors at different stages of a CNN backbone. However, early-stage features suffer from limited receptive fields and lack access to high-level semantic information, reducing detection performance.}
    \label{fig:chapter14_fpn_early_stages}
\end{figure}

\subsubsection{Enhancing Low-Level Features with High-Level Semantics}
A major drawback of using early-stage CNN features for object detection is that they lack \textbf{semantic richness}. Lower layers in CNNs retain high spatial resolution but primarily capture edges and textures, whereas deeper layers encode more complex features but at a lower resolution. This results in a trade-off: high-resolution features lack meaningful context, while low-resolution features are more informative but spatially coarse.

To address this, FPNs introduce \textbf{top-down connections} that propagate high-level information back to lower-resolution feature maps.

\begin{figure}[H]
    \centering
    \includegraphics[width=0.8\textwidth]{Figures/Chapter_14/slide_82.jpg}
    \caption{Top-down feature fusion in Feature Pyramid Networks. High-level features are progressively upsampled and combined with low-level features to enhance their semantic richness before detection.}
    \label{fig:chapter14_fpn_topdown}
\end{figure}

Specifically, the process consists of the following steps:

\begin{enumerate}
    \item Each feature map from the backbone undergoes a \textbf{$1\times1$ convolution} to change its channel dimensionality. This ensures that features from different levels are compatible when combined.
    \item The highest-level feature map (smallest spatial size, richest semantic information) is directly used as the starting point for the \textbf{top-down pathway}.
    \item The lower-resolution feature maps are then progressively \textbf{upsampled} using bilinear interpolation or transposed convolution (also known as deconvolution) to match the spatial resolution of the next finer feature map.
    \item The upsampled feature map is then \textbf{element-wise added} to the corresponding feature map from the backbone (which retains high spatial resolution but lacks deep semantic information).
    \item Finally, the fused feature maps are further processed by a \textbf{$3\times3$ convolution} to smooth out artifacts introduced by upsampling and fusion before being used for object detection.
\end{enumerate}

\paragraph{How Upsampling Works in FPNs}
Upsampling is a crucial operation in FPNs since it allows coarse but high-level features to be brought into alignment with finer-resolution feature maps. This is typically done in one of two ways:

\begin{itemize}
    \item \textbf{Bilinear Interpolation:} A non-learnable method we've covered that interpolates pixel values based on surrounding features, and can be used to produce smooth upscaled feature maps.
    \item \textbf{Transposed Convolution (Deconvolution):} A learnable operation that applies upsampling with trainable filters, allowing the network to learn an optimal way to refine features during backpropagation. We'll cover it in more detail later, when we'll discuss segmentation. 
\end{itemize}

By applying these top-down connections, FPNs create a hierarchical feature representation where \textbf{all levels of the feature pyramid benefit from deep semantic information}. This significantly improves object detection performance, especially for small objects, by ensuring that all feature levels contribute meaningful information to the final detections.

\subsubsection{Combining Results from Multiple Feature Levels}
Once object detections are generated from multiple feature levels, they must be merged to produce a final prediction. The standard approach is to apply \textbf{Non-Maximum Suppression (NMS)} across all detections:

\begin{itemize}
    \item \textbf{Sort all detected bounding boxes} by confidence score.
    \item \textbf{Iteratively suppress overlapping boxes} with lower confidence, ensuring that redundant detections do not appear in the final output.
\end{itemize}

\paragraph{Advantages of FPNs}
Feature Pyramid Networks offer several key advantages over traditional multi-scale detection approaches:

\begin{itemize}
    \item \textbf{Efficient multi-scale feature extraction} – The network processes the image only once, rather than at multiple scales.
    \item \textbf{Enhanced small-object detection} – Lower-resolution feature maps retain fine details while incorporating high-level semantics.
    \item \textbf{Lightweight and scalable} – The additional computational cost of FPNs is minimal compared to constructing an image pyramid.
\end{itemize}

By efficiently integrating information from different levels of a CNN, FPNs have become a standard component in modern object detection architectures, including Faster R-CNN.

\newpage

\paragraph{\textbf{The Two-Stage Object Detection Pipeline}}  
Faster R-CNN is a \textbf{two-stage object detector}, meaning the detection process is divided into two sequential steps:

\begin{enumerate}
    \item \textbf{Stage 1: Region Proposal Generation}
    \begin{itemize}
        \item The backbone CNN processes the entire image once to generate a feature map.
        \item The \textbf{Region Proposal Network (RPN)} applies convolutional layers to the feature map and outputs a set of \textbf{region proposals}, each with an \textbf{objectness score} and \textbf{bounding box transform}.
        \item The top $N$ proposals (e.g., 300) are selected using \textbf{Non-Maximum Suppression (NMS)} to remove redundant boxes.
    \end{itemize}
    
    \item \textbf{Stage 2: Object Detection and Classification}
    \begin{itemize}
        \item The extracted feature map is cropped using \textbf{RoIPooling}, producing fixed-size feature vectors for each proposal.
        \item Each proposal is classified into an object category or background.
        \item A final \textbf{bounding box refinement transformation} improves localization accuracy.
    \end{itemize}
\end{enumerate}

\begin{figure}[H]
    \centering
    \includegraphics[width=0.8\textwidth]{Figures/Chapter_14/slide_71.jpg}
    \caption{Visualization of Faster R-CNN as a two-stage object detector. The \textbf{first stage} (blue) generates region proposals, while the \textbf{second stage} (green) classifies objects and refines the proposals.}
    \label{fig:chapter14_faster_rcnn_pipeline}
\end{figure}

This two-stage approach provides \textbf{high accuracy} but comes at the cost of increased computational complexity. Faster R-CNN significantly improves inference speed over its predecessors, yet the sequential pipeline—first generate proposals, then run a per-proposal classifier and regressor—still limits real-time performance.

\medskip \noindent
A natural follow-up question is: \textbf{do we really need a separate second stage at all?} Notice that the RPN in Stage~1 is already a small, fully convolutional network that scans the feature map and predicts both an \emph{objectness score} and \emph{bounding box offsets} for many locations. In other words, it is almost a detector by itself—just with a very simple label space (“object vs.\ background”).

\newpage
This observation motivated a new family of \textbf{single-stage object detectors}. Instead of first proposing regions and then classifying them, these models predict object categories and bounding boxes \emph{directly} from the feature maps in one pass, removing the explicit proposal stage.

\medskip \noindent
In the following sections, we will study this paradigm through \textbf{RetinaNet} \cite{lin2018_focalloss}, which introduces the \textbf{Focal Loss} to tackle severe class imbalance in dense prediction, and \textbf{FCOS} \cite{tian2019_fcos}, a fully convolutional anchor-free detector that further simplifies the design. Later, after introducing \textbf{Transformers}, we will return to this idea with \textbf{DEtection TRansformer (DETR)} \cite{carion2020_detr}, a modern single-stage detector that formulates object detection as a set prediction problem.

\section{RetinaNet: A Breakthrough in Single-Stage Object Detection}
\label{subsec:chapter14_retinanet}

RetinaNet \cite{lin2018_focalloss} was a major breakthrough in object detection, becoming the first \textbf{single-stage detector} to surpass the performance of top two-stage methods such as Faster R-CNN. It is based on a \textbf{ResNet-101-FPN} or \textbf{ResNeXt-101-FPN} backbone, where the \textbf{Feature Pyramid Network (FPN)} serves as the neck. By leveraging FPN, RetinaNet effectively handles multi-scale object detection while maintaining high efficiency.

\subsection{Why Single-Stage Detectors Can Be Faster}
Single-stage object detectors predict object categories and bounding boxes \textbf{directly from feature maps}, eliminating the need for a region proposal step. Unlike Faster R-CNN, which processes only a few thousand region proposals per image, single-stage detectors like RetinaNet operate on a \textbf{dense grid of anchor boxes}—potentially processing over 100,000 candidate regions in a single forward pass.

\begin{itemize}
    \item \textbf{Efficiency:} Instead of applying a second-stage classifier per proposal, RetinaNet classifies objects in a single step, reducing inference time.
    \item \textbf{Parallelization:} Since all predictions are made in parallel, one-stage detectors can fully utilize modern hardware like GPUs.
\end{itemize}

However, despite these advantages, single-stage detectors historically struggled with \textbf{class imbalance}, which RetinaNet successfully addresses.

\begin{figure}[H]
    \centering
    \includegraphics[width=0.8\textwidth]{Figures/Chapter_14/slide_89.jpg}
    \caption{Inference speed comparison of RetinaNet and other detectors. Single-stage detectors like RetinaNet are significantly faster than two-stage detectors, such as FPN Faster R-CNN.}
    \label{fig:chapter14_retinanet_inference}
\end{figure}

\subsection{The Class Imbalance Problem in Dense Detection}
One of the main challenges in single-stage detection is \textbf{extreme foreground–background class imbalance}.  
Because these detectors make predictions densely over the entire feature map, they evaluate tens of thousands (sometimes over 100,000) of anchors per image, while only a tiny fraction of them actually overlap a ground-truth object.

Concretely, this means that the vast majority of anchors are \emph{easy background} examples. This imbalance causes two related problems:

\begin{enumerate}
	\item \textbf{Inefficient training:} Most negative anchors are trivial to classify as background, so their individual loss and gradients are very small. Yet they still consume most of the computation in each forward/backward pass. The network spends a lot of effort repeatedly confirming “this is background” instead of learning from the relatively few informative foreground examples and hard negatives.
	
	\item \textbf{Domination of the loss by easy negatives:} Although each easy background anchor contributes only a tiny loss, their \emph{sheer quantity} means their summed contribution can overwhelm the loss from the few positive anchors. In this regime, a degenerate solution that simply predicts “background” almost everywhere can achieve low average loss and high raw accuracy, while completely failing to detect objects (very low recall). The optimizer is therefore biased toward modeling the majority background class well, rather than learning strong features for the rare foreground class.
\end{enumerate}

This issue is much less severe in two-stage detectors like Faster R-CNN, where the RPN \textbf{filters out most background regions} before the second-stage classifier, leaving a more \textbf{balanced subset} of positive and negative proposals for training.

RetinaNet’s key contribution is to tackle this imbalance \emph{at the loss level}, introducing the \textbf{Focal Loss} to down-weight easy negatives so that training focuses on the scarce, informative examples.

\subsection{Focal Loss: Addressing Class Imbalance}

RetinaNet introduced the focal loss to tackle the severe class imbalance inherent in one-stage detectors. Instead of resorting to heuristic sampling or hard-negative mining, focal loss modifies the standard cross-entropy (CE) loss by down-weighting the loss contribution of well-classified examples, thereby shifting the model’s focus toward hard, misclassified examples.

The focal loss is defined as:

\[
FL(p_t) = - (1 - p_t)^\gamma \log(p_t)
\]

where:
\begin{itemize}
    \item \( p_t \) is the predicted probability for the ground-truth class.
    \item \( \gamma \) is the tunable focusing parameter.
\end{itemize}

For comparison, the standard cross-entropy loss is:

\[
CE(p_t) = -\log(p_t)
\]

By introducing the modulating factor \((1-p_t)^\gamma\), the focal loss reduces the loss for examples that are already well-classified (i.e., when \( p_t \) is high). For instance, with \(\gamma = 2\):
\begin{itemize}
    \item If \( p_t = 0.9 \), then \((1-0.9)^2 = 0.01\), and the loss becomes approximately \(0.01 \times -\log(0.9) \approx 0.01 \times 0.105 = 0.00105\). In contrast, the standard CE loss would be about 0.105.
    \item If \( p_t = 0.5 \), then \((1-0.5)^2 = 0.25\), and the loss is \(0.25 \times -\log(0.5) \approx 0.25 \times 0.693 = 0.173\).
    \item If \( p_t = 0.2 \), then \((1-0.2)^2 = 0.64\), and the loss is \(0.64 \times -\log(0.2) \approx 0.64 \times 1.609 = 1.029\).
\end{itemize}

These examples illustrate that as the prediction confidence \( p_t \) increases (i.e., for easy examples), the modulating factor quickly shrinks the loss, allowing the model to focus its learning capacity on the hard examples where \( p_t \) is lower.

An \(\alpha\)-balanced variant of the focal loss can further address class imbalance by assigning different weights to positive and negative examples:

\[
FL(p_t) = -\alpha_t (1 - p_t)^\gamma \log(p_t)
\]

Here, \(\alpha_t\) is chosen to down-weight the loss for the dominant class (usually the background). In practice, selecting \(\gamma = 2\) and an appropriate \(\alpha\) (e.g., 0.25) has been shown to yield robust results.

\begin{figure}[H]
    \centering
    \includegraphics[width=0.65\textwidth]{Figures/Chapter_14/focal_loss_explained.png}
    \caption{Focal loss modifies the standard cross-entropy loss by incorporating a modulating factor \((1-p_t)^\gamma\). This factor down-weights the loss for well-classified examples. For instance, when \(\gamma=2\), the loss for examples with high confidence (e.g., \(p_t \approx 0.9\)) is significantly reduced, while the loss for moderately difficult examples (e.g., \(p_t \approx 0.5\) or \(p_t \approx 0.2\)) remains similar to that of the standard cross-entropy loss. Setting \(\gamma\) too high (such as \(\gamma=5\)) can overly suppress the loss even for examples that are not trivial, potentially eliminating valuable learning signals. Thus, \(\gamma=2\) is often chosen as a good compromise, effectively reducing the loss from very easy examples while preserving enough gradient for harder examples. Source: \cite{lin2018_focalloss}.}
    \label{fig:chapter14_focal_loss}
\end{figure}

\begin{figure}[H]
    \centering
    \includegraphics[width=0.8\textwidth]{Figures/Chapter_14/focal_loss_shifts_focus.png}
    \caption{Cumulative distribution functions (CDFs) of the normalized loss for background (negative) and foreground (positive) examples under different values of \(\gamma\). As \(\gamma\) increases, the loss contribution from easy negatives is dramatically reduced, which flattens the loss distribution for background examples. Importantly, with \(\gamma=2\), the loss for foreground examples remains nearly unchanged, ensuring that the model still learns effectively from the scarce positive examples. This selective down-weighting is crucial for mitigating class imbalance. Source: \cite{lin2018_focalloss}.}
    \label{fig:chapter14_focal_loss_distribution}
\end{figure}

In summary, focal loss is a key innovation in RetinaNet that directly addresses class imbalance by dynamically down-weighting the loss from easy examples. This enables training a dense one-stage detector effectively without resorting to complex sampling heuristics, ultimately achieving state-of-the-art accuracy while maintaining fast inference speeds.

\subsection{RetinaNet Architecture and Pipeline}
\label{subsec:chapter14_retinanet_arch_pipeline}

\paragraph{Backbone and Neck (FPN)}
RetinaNet uses a standard ImageNet–pretrained backbone (e.g., ResNet-50/101 or ResNeXt-101) to produce a \emph{hierarchy} of feature maps (commonly denoted \(C_3,C_4,C_5\)). Early backbone stages are high-resolution but semantically weaker; late stages are semantically strong but very coarse. The \textbf{Feature Pyramid Network (FPN)} is a lightweight top-down pathway with lateral connections that fuses these signals to create a new set of \emph{semantically strong, multi-scale} maps \(P_3,\dots,P_7\). Concretely:
\begin{itemize}
	\item \(P_5\) is obtained from \(C_5\) by a \(1{\times}1\) lateral conv; \(P_4\) and \(P_3\) are formed by upsampling the higher level and adding a lateral projection from \(C_4\) and \(C_3\) respectively, followed by a \(3{\times}3\) conv for smoothing.
	\item \(P_6\) and \(P_7\) extend the pyramid for very large objects via stride-2 \(3{\times}3\) convs (e.g., \(P_6\) directly from \(C_5\), then \(P_7\) from \(P_6\) with a ReLU in between).
\end{itemize}
Each level has a well-defined \emph{stride} relative to the input image, typically
\(\{8,16,32,64,128\}\) pixels for \(P_3\)–\(P_7\). Thus, one spatial location at \(P_\ell\) summarizes roughly a \(\text{stride}_\ell \times \text{stride}_\ell\) patch of the input. High-resolution \(P_3\) captures small objects; low-resolution \(P_6,P_7\) capture large ones and global context.

\paragraph{Dense Anchors (per FPN level)}
Detection is made dense by tiling \emph{anchors}—predefined box prototypes—at every spatial location of every pyramid level. RetinaNet assigns each level a \emph{base side length}
\[
s_\ell \in \{32,64,128,256,512\}\quad\text{for}\quad P_3,\dots,P_7,
\]
so that level \(P_\ell\) is responsible for objects whose side lengths are \(\mathcal{O}(s_\ell)\). To cover shapes and nearby scales without exploding the search space, \(\mathbf{A=9}\) anchors are placed per location by combining
\[
\text{aspect ratios } r \in \{1/2,\,1,\,2\}\quad\text{and}\quad
\text{in-octave scales } m_k \in \{2^{0},\,2^{1/3},\,2^{2/3}\}.
\]
Given \((s_\ell,m_k,r)\), an anchor’s width and height are
\[
w_{\ell,k,r} = s_\ell\, m_k\, \sqrt{r},\qquad
h_{\ell,k,r} = s_\ell\, m_k\, / \sqrt{r},
\]
which preserves the anchor’s \emph{area} near \((s_\ell m_k)^2\) while adjusting its shape by \(r=w/h\).

\emph{Why fractional scales like \(2^{1/3}\)?} RetinaNet partitions each \emph{octave} (a doubling of size) into three equal steps in \(\log_2\) space. The multiplicative ratio between adjacent scales is \(2^{1/3}\approx 1.26\). This yields anchors that (i) are \emph{evenly spaced in scale} (no “holes” between \(32\) and \(64\), etc.), (ii) avoid redundant near-duplicates that arise with coarse integer jumps, and (iii) keep coverage smooth across object sizes. Intuitively, if an object’s true size lies between powers of two, one of the three in-octave scales will land close enough that the regressor only needs to make a small, stable adjustment.

Across \(P_3\)–\(P_7\), this construction spans effective side lengths from roughly \(32\) to \(512\) pixels (and intermediate in-octave values), producing on the order of \(10^5\) anchors per image—ample coverage for size and shape, while remaining efficient due to shared convolutions over the pyramid.

\paragraph{Two Lightweight Prediction Heads (shared across pyramid levels)}
RetinaNet attaches two small, fully convolutional ``heads'' to \emph{every} FPN level; their weights are shared across levels for parameter efficiency (the two heads do \emph{not} share weights with each other):
\begin{itemize}
	\item \textbf{Classification head:} a subnetwork of four \(3\times 3\) conv layers with 256 channels (each followed by ReLU), ending in a \(3\times 3\) conv that outputs \(A\times C\) \emph{per-class} logits per spatial location. A \emph{sigmoid} is applied independently to each of the \(C\) classes (no softmax over classes), which pairs naturally with the Focal Loss.
	\item \textbf{Box regression head:} an identically shaped subnetwork that ends in \(A\times 4\) outputs per location, parameterizing relative offsets \((t_x,t_y,t_w,t_h)\) from the anchor.
\end{itemize}
\emph{Bias initialization for stability.} To counter the extreme initial imbalance, RetinaNet initializes the final classification-layer bias to
\[
b=-\log\!\left(\frac{1-\pi}{\pi}\right),\quad \pi=0.01,
\]
so the network starts with a low prior probability for foreground, reducing spurious early gradients from the vast background set.

\paragraph{Inference (single pass)}
All FPN levels are processed in parallel, producing a total of \(\mathcal{O}(10^5)\) anchor predictions per image. After a low score threshold (e.g., 0.05), RetinaNet applies per-class NMS (e.g., IoU 0.5) and keeps the top-\(K\) detections (e.g., \(K{=}100\)).

\begin{figure}[H]
	\centering
	\includegraphics[width=0.65\textwidth]{Figures/Chapter_14/slide_87.jpg}
	\caption{RetinaNet pipeline. A backbone + FPN produces a multi-scale feature pyramid. Two lightweight heads (classification and box regression) operate densely on each pyramid level, predicting \(A\times C\) class scores and \(A\times 4\) box deltas per location in a single stage}
	\label{fig:chapter14_retinanet_pipeline}
\end{figure}

\paragraph{Why this works (and what was missing before)}
Architecturally, RetinaNet is deliberately simple: it keeps the RPN’s efficient, fully convolutional template but upgrades to multi-class classification and full box refinement over a feature pyramid. The historical blocker for single-stage accuracy was \emph{not} the architecture but the \textbf{extreme class imbalance} inherent to dense prediction. RetinaNet’s breakthrough is to pair this streamlined design with the \textbf{Focal Loss}, which down-weights the flood of easy negatives so the classifier learns from scarce positives and hard examples. The result is two-stage–level accuracy with single-stage speed.

\newpage

\section{FCOS: An Anchor-Free, Fully Convolutional Detector}
\label{sec:chapter14_fcos}

FCOS \cite{tian2019_fcos} is an \textbf{anchor-free} one-stage detector that casts detection as a dense, \textbf{per-pixel} prediction problem. Instead of matching ground-truth boxes to a large, hand-designed set of anchors (sizes, aspect ratios, and assignment rules), every spatial location on a feature map can vote for an object by predicting its class and the distances from that location to the four sides of the object’s box. This removes anchor hyperparameters and simplifies both the design and the training pipeline.

\subsection{Core Pipeline and Supervision}
\label{subsec:chapter14_fcos_pipeline}

\paragraph{Backbone and Feature Maps}
A backbone (e.g., ResNet) with FPN produces a pyramid of feature maps \(\{P_3,\dots,P_7\}\). A location \((x,y)\) on a pyramid level with stride \(s\) corresponds to an input coordinate \(\tilde{x}=x\cdot s+\delta,\ \tilde{y}=y\cdot s+\delta\) (with a fixed offset \(\delta\) such as \(s/2\)).

\paragraph{Positive/Negative Assignment}
For each feature-map location, FCOS checks whether its mapped coordinate \((\tilde{x},\tilde{y})\) lies \emph{inside} any ground-truth box \(B=(x_0,y_0,x_1,y_1)\). If not, the location is negative (background). If yes, it is positive and is assigned to (i) that class and (ii) a single box, chosen as the \emph{smallest-area} box among those covering \((\tilde{x},\tilde{y})\) to favor supervision from small, harder objects.

\paragraph{Distance-From-Point Regression Targets}
For a positive location, regression targets are the distances to the four sides of its assigned box:
\[
l^\ast=\tilde{x}-x_0,\quad t^\ast=\tilde{y}-y_0,\quad r^\ast=x_1-\tilde{x},\quad b^\ast=y_1-\tilde{y}.
\]
At inference, predicted distances \((l,t,r,b)\) are converted back to a box
\((\tilde{x}-l,\ \tilde{y}-t,\ \tilde{x}+r,\ \tilde{y}+b)\).

\begin{figure}[H]
	\centering
	\includegraphics[width=0.8\textwidth]{Figures/Chapter_14/fcos_edge_case.jpg}
	\caption{Left: FCOS regresses \((l,t,r,b)\) at each positive location to recover the box. Right: ambiguity resolution assigns a location inside multiple boxes to the smallest box}
	\label{fig:chapter14_fcos_edge_case}
\end{figure}

\subsection{Multi-Level Prediction with FPN}
\label{subsec:chapter14_fcos_fpn}

As in RetinaNet, FCOS uses FPN to divide the problem by object size rather than by anchor scale. Each level is responsible for a \emph{range} of object sizes (typical choices):
\[
\begin{aligned}
	P_3 &: (0,64]\ \text{pixels},\quad
	P_4 &: (64,128],\quad
	P_5 &: (128,256],\\
	P_6 &: (256,512],\quad
	P_7 &: (512,\infty)
\end{aligned}
\]
This assignment reduces label ambiguity across scales and lets a single set of prediction heads operate reliably at all pyramid levels.

\begin{figure}[H]
	\centering
	\includegraphics[width=0.8\textwidth]{Figures/Chapter_14/slide_96.jpg}
	\caption{FCOS with FPN: each level specializes to a size range, improving supervision and reducing scale ambiguity}
	\label{fig:chapter14_fcos_fpn}
\end{figure}

\subsection{Centerness: Definition, Role, and Intuition}
\label{subsec:chapter14_fcos_centerness}

\paragraph{Why Centerness}
Any location inside a ground-truth box is a valid positive, but locations near the \emph{edges} tend to yield lower-quality boxes: one or more distances \((l^\ast,t^\ast,r^\ast,b^\ast)\) are small on one side and large on the other, making the regression ill-conditioned. FCOS introduces a third head that predicts a \emph{centerness} score to quantify how central a location is w.r.t.\ its assigned object.

\paragraph{Target and Shape}
The centerness target is
\[
\text{centerness}^\ast
= \sqrt{
	\frac{\min(l^\ast,r^\ast)}{\max(l^\ast,r^\ast)}
	\cdot
	\frac{\min(t^\ast,b^\ast)}{\max(t^\ast,b^\ast)}
}.
\]
It is the geometric mean of horizontal and vertical “balancedness.” At the exact center, \(l^\ast=r^\ast\) and \(t^\ast=b^\ast\), so \(\text{centerness}^\ast=1\). As a point drifts toward an edge on either axis, the corresponding ratio shrinks toward \(0\), and so does the score. The square root moderates the decay so that moderately off-center locations are not over-penalized.

\paragraph{How It Is Used}
\begin{itemize}
	\item \textbf{Training:} The centerness head is trained with a binary cross-entropy loss to regress \(\text{centerness}^\ast\). In addition, FCOS weights the \emph{localization loss} of a positive location by \(\text{centerness}^\ast\), down-weighting inherently low-quality positives (near edges) during box regression.
	\item \textbf{Inference:} The final detection confidence is \(\text{score} = \text{class\_prob} \times \text{centerness}\). This suppresses spurious boxes predicted from peripheral locations without requiring extra post-processing heuristics.
\end{itemize}

\begin{figure}[H]
	\centering
	\includegraphics[width=0.6\textwidth]{Figures/Chapter_14/slide_94.jpg}
	\caption{Three parallel heads per location: classification, \((l,t,r,b)\) regression, and centerness; centerness calibrates confidence by proximity to the object center}
	\label{fig:chapter14_fcos_pipeline}
\end{figure}

\subsection{Localization with IoU Loss}
\label{subsec:chapter14_fcos_iou}

\paragraph{Computation in Distance Parameterization}
Let the predicted distances be \((l,t,r,b)\) and the targets \((l^\ast,t^\ast,r^\ast,b^\ast)\) for the same positive location. Define predicted and target areas
\[
A_p=(l+r)(t+b),\qquad A_g=(l^\ast+r^\ast)(t^\ast+b^\ast).
\]
Because both boxes are anchored at the \emph{same} location, the intersection width and height are
\[
w_I=\min(l,l^\ast)+\min(r,r^\ast),\qquad
h_I=\min(t,t^\ast)+\min(b,b^\ast),
\]
and the intersection area is \(A_I=w_I\cdot h_I\). The \textbf{IoU} is
\[
\text{IoU}=\frac{A_I}{A_p+A_g-A_I},\qquad
L_{\text{reg}}=-\log(\text{IoU})\ \ \text{or}\ \ 1-\text{IoU}.
\]

\paragraph{Why IoU, not \(L_1\)}
IoU loss is \emph{scale-invariant} and \emph{holistic}: it couples all four distances to maximize overlap. In contrast, \(L_1\)/smooth-\(L_1\) penalize each side independently and over-weight large boxes. Variants such as GIoU/DIoU/CIoU can further stabilize optimization, but vanilla IoU already yields strong localization in FCOS.

\subsection{Multi-Task Objective and Training Scheme}
\label{subsec:chapter14_fcos_loss}

Per image, let \(\mathcal{P}\) be the set of positive locations across all pyramid levels and \(N_+\!=\!|\mathcal{P}|\) (with a small \(\epsilon\) to avoid division by zero). FCOS minimizes
\[
L_{\text{total}}
=\underbrace{L_{\text{cls}}}_{\text{focal, pos+neg}}
+\lambda_{\text{reg}}
\underbrace{\frac{1}{N_+}\sum_{i\in\mathcal{P}}\text{centerness}_i^\ast\,L_{\text{reg},i}}_{\text{IoU on positives, weighted by } \text{centerness}^\ast}
+\lambda_{\text{ctr}}
\underbrace{\frac{1}{N_+}\sum_{i\in\mathcal{P}} \text{BCE}(\hat{c}_i,\text{centerness}_i^\ast)}_{\text{centerness head on positives}},
\]
where:
\begin{itemize}
	\item \(L_{\text{cls}}\) is the \textbf{Focal Loss} over all locations (positives and negatives), mitigating extreme foreground–background imbalance
	\item \(L_{\text{reg}}\) is the \textbf{IoU loss} in the distance parameterization for positives only
	\item The regression term is \emph{weighted} by \(\text{centerness}^\ast\) to de-emphasize inherently low-quality edge positives
	\item \(\lambda_{\text{reg}},\lambda_{\text{ctr}}\) balance localization and centerness terms; practical defaults often set them to \(1\)
\end{itemize}
At inference, the per-class probability is multiplied by the predicted centerness before NMS. Thus, focal loss addresses \emph{class imbalance}, IoU loss optimizes \emph{overlap quality}, and centerness calibrates both \emph{training weights} (for localization) and \emph{test-time confidences}.

\subsection{Inference}
\label{subsec:chapter14_fcos_inference}

Single forward pass over the FPN yields class scores, distances, and centerness for every location. Predictions with low class score are filtered; remaining scores are multiplied by centerness; distances are converted to boxes; per-class NMS produces final detections.

\subsection{Advantages of FCOS}
FCOS introduces several improvements over anchor-based detectors:
\begin{itemize}
    \item \textbf{Simpler Design:} Eliminates the need for anchor boxes, reducing hyper-parameter tuning.
    \item \textbf{Computational Efficiency:} Avoids anchor box computations, reducing memory and processing overhead.
    \item \textbf{Better Foreground Utilization:} Unlike anchor-based methods, which only consider a subset of anchors, FCOS treats every feature map location inside a ground-truth box as a positive sample.
    \item \textbf{Improved Detection Quality:} The centerness mechanism suppresses low-quality predictions, reducing false positives.
\end{itemize}

By leveraging fully convolutional architectures and eliminating the complexities of anchor boxes, FCOS provides a simple yet powerful alternative to traditional object detection methods.

\newpage

\begin{enrichment}[YOLO - You Only Look Once][section]
\begin{enrichment}[Background][subsection]
    YOLO (You Only Look Once) revolutionized object detection by treating it as a \textbf{single regression problem}, enabling real-time detection without requiring multiple passes over an image. 
    
    First introduced by Redmon \textit{et al.} in \cite{redmon2016_yolo}, YOLO has continuously evolved (from YOLOv1 to more advanced versions) by improving accuracy while maintaining real-time performance. Its success stems from:
    \begin{itemize}
        \item \textbf{Speed}: YOLO's one-pass approach makes it significantly faster than two-stage detectors, enabling applications in autonomous driving, surveillance, and real-time video analysis.
        \item \textbf{Global Reasoning}: By processing the entire image at once, YOLO reduces false positives from overlapping region proposals and makes more context-aware predictions.
    \end{itemize}
    Thanks to these advantages, YOLO remains one of the most widely used object detection frameworks, consistently setting new benchmarks for real-time applications.
\end{enrichment}
    
\begin{enrichment}[Step-by-Step: How YOLOv1 Processes an Input Image][subsection]
    YOLOv1 (\textit{You Only Look Once}) is a single-stage object detector that predicts bounding boxes and class probabilities in one unified forward pass. Below, we outline how YOLOv1 processes an image from start to finish.
    
    \paragraph{1. Input Image and Preprocessing}
    \begin{itemize}
        \item \textbf{Dimensions:} YOLOv1 typically expects an image resized to \(448\times448\).
        \item \textbf{Normalization:} In practice, pixel values may be scaled (e.g., to \([0,1]\) or \([-1,1]\)) to help training stability.
        \item This preprocessed image is fed into the network as a PyTorch \texttt{Tensor} of shape \\ \(\bigl[\text{batch\_size}, 3, 448, 448\bigr]\).
    \end{itemize}
    
    \paragraph{2. Feature Extraction (DarkNet + Additional Convolution Layers)}
      \texttt{YOLOv1} is composed of:
    \begin{enumerate}
        \item \texttt{DarkNet}, which produces a high-level feature map from the input image. DarkNet is a series of convolutional layers interspersed with activations (Leaky ReLU) and sometimes batch normalization.
        \item Additional convolution layers that further refine the 1024-channel output of DarkNet.
    \end{enumerate}
    Eventually, these convolutions yield a feature map of shape \(\bigl[\text{batch\_size}, 1024, S, S\bigr]\), where S is grid dimension, a hyperparameter that fits our feature extraction process (in YOLOv1, $S=7$). Hence, YOLOv1 divides the image conceptually into a \(7\times7\) grid.
    
    \paragraph{3. Flattening and Fully Connected Layers}
    After the final convolutional layer, the 7\(\times\)7\(\times\)1024 feature map is:
    \begin{itemize}
    	\item \textbf{Flattened} into a 1D vector of length \(7 \times 7 \times 1024 = 50176\).
    	\item Passed into a \texttt{Linear(50176, 4096)} layer, a Leaky ReLU, and a dropout layer.
    	\item Finally, passed into a \textbf{linear output layer} of size \(S \times S \times (5B + C)\), where:
    	\begin{itemize}
    		\item \(S=7\) is the number of grid cells per dimension.
    		\item \(B=2\) is the number of bounding boxes each cell predicts.
    		\item \(C=20\) is the number of classes (for the PASCAL VOC Dataset).
    	\end{itemize}
    \end{itemize}
    This yields an output tensor of shape:
    \[
    \bigl[\text{batch\_size},\,7,\,7,\,(5 \times 2 + 20)\bigr] = \bigl[\text{batch\_size},\,7,\,7,\,30\bigr].
    \]
    The final layer is \emph{linear}: it produces real-valued outputs that are trained, via a sum-of-squared-errors loss, to approximate normalized targets (e.g., coordinates and confidences in \([0,1]\)).
    
    \paragraph{4. Understanding the Output Format}
    Concretely, each cell’s part of the final output includes:
    \begin{enumerate}
        \item \(\mathbf{(x, y)}\): Center offsets for box~1 within the cell, in \([0,1]\).
        \item \(\mathbf{w, h}\): Width and height for box~1, also in \([0,1]\).
        \item \(\text{confidence}\): A single scalar in \([0,1]\) for how likely the predicted box is \textit{valid} (the bounding box overlaps an object).
        \item The same 5 parameters for box~2 (\(x, y, w, h, \text{confidence}\)).
        \item \(\mathbf{C}\) class probabilities for the cell, also in \([0,1]\).
    \end{enumerate}
    
    \paragraph{5. Parameterization and Normalization}
    Although the final layer is linear, YOLOv1 \emph{parametrizes} its targets so that most predicted quantities naturally lie in \([0,1]\):
    \begin{itemize}
    	\item \(\hat{x}, \hat{y}\) are trained to represent the center of the box \textit{relative to the grid cell} that predicts it, with targets in \([0,1]\). At inference time, we convert them to absolute image coordinates using the cell indices \((c_x, c_y)\) and the grid size \(S\).
    	\item \(\hat{w}, \hat{h}\) are trained to represent the box width and height \textit{relative to the full image size}, again with targets in \([0,1]\). The loss uses \(\sqrt{w}\) and \(\sqrt{h}\) to emphasize errors on small boxes.
    	\item The \textbf{confidence} output for each box is trained to regress to
    	\[
    	C = P(\text{object}) \cdot \mathrm{IoU}(\text{box}, \text{gt}) \in [0,1],
    	\]
    	where \(\mathrm{IoU}\) is the intersection-over-union with the ground-truth box.
    	\item The \textbf{class probabilities} are conditional probabilities \(P(\text{class}_c \mid \text{object})\) at the cell level, with targets given by one-hot vectors over the \(C\) classes.
    \end{itemize}
    Thus, even though the network’s outputs are unconstrained real numbers, the combination of normalized targets and an L2 loss encourages them to behave like probabilities and normalized coordinates.
    
    \paragraph{6. Converting Predictions to Actual Bounding Boxes}
    \textbf{Inside each cell}, we do:
    \[
    \hat{x}_\text{abs} = \frac{c_x + \hat{x}}{S}, \quad
    \hat{y}_\text{abs} = \frac{c_y + \hat{y}}{S},
    \]
    where \(c_x, c_y\) is the grid cell’s top-left integer index (e.g., \((2,3)\) if we are in row~2, column~3) and \(S=7\). Then,
    \[
    \hat{w}_\text{abs} = \hat{w} \times \text{image\_width}, \quad
    \hat{h}_\text{abs} = \hat{h} \times \text{image\_height}.
    \]
    The bounding box corners become:
    \[
    x_{\text{min}} = \hat{x}_\text{abs} - \tfrac{\hat{w}_\text{abs}}{2}, \quad
    y_{\text{min}} = \hat{y}_\text{abs} - \tfrac{\hat{h}_\text{abs}}{2}, 
    \quad
    x_{\text{max}} = \hat{x}_\text{abs} + \tfrac{\hat{w}_\text{abs}}{2}, 
    \quad
    y_{\text{max}} = \hat{y}_\text{abs} + \tfrac{\hat{h}_\text{abs}}{2}.
    \]
    Thus each cell contributes up to $B=2$ bounding boxes in absolute image coordinates.
    
    \paragraph{7. Loss and Training (High Level)}
    YOLO’s loss function balances three main terms:
    \begin{itemize}
    	\item \textbf{Localization Loss}: Penalizes bounding box coordinate errors \((x, y, w, h)\) for the box in each cell that is responsible for an object. The loss uses \(\sqrt{w}\) and \(\sqrt{h}\) to give relatively more weight to small boxes.
    	\item \textbf{Confidence Loss}: Penalizes errors in the objectness confidence. It pushes confidence toward 1 for responsible boxes in cells that contain objects, and toward 0 for all boxes in cells that do not contain objects.
    	\item \textbf{Classification Loss}: A sum-of-squared-errors (L2) loss on the class probabilities, applied \emph{only} to cells that contain an object.
    \end{itemize}
    To balance these contributions, the loss up-weights localization (\(\lambda_{\text{coord}} = 5\)) and down-weights the confidence loss for background cells (\(\lambda_{\text{noobj}} = 0.5\)).
    
    The full loss function is:
    \[
    L = \lambda_{\text{coord}} \sum_{i=1}^{S^2} \sum_{j=1}^{B} 1^{\text{obj}}_{ij} \bigl[(x_i - \hat{x}_i)^2 + (y_i - \hat{y}_i)^2\bigr]
    \]
    \[
    + \lambda_{\text{coord}} \sum_{i=1}^{S^2}\sum_{j=1}^{B} 1^{\text{obj}}_{ij} \bigl[(\sqrt{w_i} - \sqrt{\hat{w}_i})^2 + (\sqrt{h_i} - \sqrt{\hat{h}_i})^2\bigr]
    \]
    \[
    + \sum_{i=1}^{S^2}\sum_{j=1}^{B} 1^{\text{obj}}_{ij} \bigl(C_i - \hat{C}_i\bigr)^2
    \]
    \[
    + \lambda_{\text{noobj}} \sum_{i=1}^{S^2}\sum_{j=1}^{B} 1^{\text{noobj}}_{ij} \bigl(C_i - \hat{C}_i\bigr)^2
    \]
    \[
    + \sum_{i=1}^{S^2} 1^{\text{obj}}_{i} \sum_{c \in \text{classes}} \bigl(p_i(c) - \hat{p}_i(c)\bigr)^2.
    \]
    
    Here the confidence target for each predicted box is defined as
    \[
    C_i = P(\text{object in cell } i) \times \mathrm{IoU}(\text{predicted box}, \text{ground truth}),
    \]
    so that \(C_i = 0\) for cells without objects, and \(C_i\) equals the IoU for the “responsible” box in cells that contain an object. This ties the confidence both to object presence and to localization quality.
    
    \paragraph{8. Why It Works (and Its Trade-offs)}
    \begin{itemize}
        \item \textbf{Efficiency:} Only a single CNN forward pass is needed. This is much faster than multi-stage pipelines like \texttt{R-CNN}.
        \item \textbf{Grid-Based Reasoning:} Each cell “looks” at local features and tries to detect objects centered there, simplifying the logic behind region proposals.
        \item \textbf{No Anchors in YOLOv1:} The network directly learns bounding box shapes, which can be good for moderate object scale variety, but struggles for extremely small or large aspect ratios. Later YOLO versions added anchor priors for more robust shape handling.
    \end{itemize}
    
    \paragraph{9. Final Detections and NMS}
    Once the forward pass is done, YOLOv1 typically:
    \begin{itemize}
        \item Converts each cell’s bounding box predictions into absolute coordinates as described.
        \item Filters out boxes with low confidence.
        \item Applies \textbf{Non-Maximum Suppression} (NMS) to reduce duplicates—keeping only the highest confidence box for each object.
    \end{itemize}
    The final set of bounding boxes with class labels becomes YOLO’s detection result.
    
    \paragraph{Summary}
    \begin{enumerate}
        \item \textit{Input} (448\(\times\)448) \(\to\) \textit{DarkNet} + \textit{Conv} \(\to\) \textit{Flatten} \(\to\) \textit{Fully Connected (4096D)} \(\to\) \(\texttt{Linear}\) \(\to\) \(\texttt{Sigmoid}\).
        \item Output shape: \(\bigl[\text{batch\_size}, 7, 7, (5 \times 2 + 20)\bigr]\).
        \item Each \((7\times7)\) cell: \(\underbrace{x, y, w, h, \text{confidence}}_{\text{box 1}},\; \underbrace{x, y, w, h, \text{confidence}}_{\text{box 2}},\; \text{class probabilities}\).
        \item \(\sigma(\cdot)\) ensures values in \([0,1]\). The predicted offsets are scaled to the full image, producing final bounding boxes.
        \item Loss includes coordinate errors, objectness confidence errors, and classification errors.
        \item Post-processing merges overlapping boxes (NMS).
    \end{enumerate}
    This pipeline captures \emph{what} YOLOv1 does and \emph{why} it does it in a simple, end-to-end fashion: object localization, classification, and bounding-box regression are all learned jointly in one pass.
\end{enrichment}

\begin{figure}[H]
    \centering
    \includegraphics[width=0.8\textwidth]{Figures/Chapter_14/yolo_pipeline.png}
    \caption{YOLO pipeline: A single CNN processes the entire image, predicts bounding boxes and class probabilities, and applies NMS to refine detections. Source: \cite{redmon2016_yolo}.}
    \label{fig:chapter14_yolo_pipeline}
\end{figure}

\newpage

    \begin{enrichment}[Evolution of YOLO][subsection]
        Over time, multiple versions of YOLO have been developed to address its limitations:
        
        \begin{itemize}
            \item \textbf{YOLOv2} (2017) \cite{redmon2017_yolo9000}: Introduced anchor boxes, batch normalization, and multi-scale training, improving both accuracy and generalization.
            \item \textbf{YOLOv3} (2018) \cite{redmon2018_yolov3}: Added Darknet-53 as a backbone, feature pyramids, and objectness scores, significantly boosting detection accuracy.
            \item \textbf{YOLOv4} (2020) \cite{bochkovskiy2020_yolov4}: Focused on increasing efficiency with new activation functions (Mish), better data augmentation, and optimization techniques.
            \item \textbf{YOLOv5+} (2020s+): Introduced by Ultralytics, leveraging PyTorch and adding modern training techniques such as mosaic augmentation and hyperparameter tuning.
        \end{itemize}
        
        Each version improves upon the previous, refining accuracy, robustness, and efficiency, solidifying YOLO as one of the most influential object detection models in real-time applications.
        \end{enrichment}
    \end{enrichment}

\section{Conclusion: The Evolution of Object Detection}
\label{sec:chapter14_conclusion}

Object detection has undergone significant advancements over the years, with each iteration improving both speed and accuracy. This chapter traced the evolution of object detectors, highlighting key innovations that have shaped modern detection frameworks.

\paragraph{From R-CNN to Faster R-CNN: Learning Region Proposals}
Early object detection models, such as \textbf{R-CNN}, relied on region proposal methods like Selective Search to generate candidate object regions. While effective, R-CNN suffered from slow inference times, as it required passing each region through a CNN separately. 

\textbf{Fast R-CNN} improved this process by computing feature maps once for the entire image and then applying \textbf{RoI Pooling} or \textbf{RoIAlign} to extract features for each proposal, significantly reducing inference time. However, it still relied on external region proposals, which remained a computational bottleneck.

\textbf{Faster R-CNN} introduced \textbf{Region Proposal Networks (RPNs)}, replacing hand-crafted region proposal methods with a trainable, CNN-based approach. This enabled fully end-to-end training, where the region proposals were learned jointly with the detector. While Faster R-CNN achieved high accuracy, its two-stage nature still made it slower, and also more computationally expensive.

\paragraph{Improving Multi-Scale Detection: Feature Pyramid Networks (FPN)}
While Faster R-CNN was a breakthrough, it struggled with detecting objects at different scales, especially smaller ones. To address this, \textbf{Feature Pyramid Networks (FPNs)} were introduced, leveraging the multi-scale hierarchical features of CNNs to enhance object detection at different resolutions. By integrating top-down pathways that fused low-level spatial details with high-level semantic information, FPNs became a crucial addition to many detection architectures.

\paragraph{RetinaNet: A Breakthrough for One-Stage Detectors}
While two-stage detectors like Faster R-CNN were dominant, they were computationally expensive, motivating the need for faster alternatives. \textbf{RetinaNet} was a milestone in object detection as it was the \textbf{first single-stage detector to surpass two-stage detectors in accuracy}, all while maintaining significantly higher speed.

RetinaNet introduced \textbf{Focal Loss}, addressing the issue of class imbalance between foreground and background objects. By down-weighting easy samples and focusing on harder examples, it improved training efficiency and allowed single-stage networks to perform on par with or better than their two-stage counterparts. RetinaNet, like Faster R-CNN, leveraged FPNs for multi-scale feature extraction, making it robust for detecting objects across different sizes.

\paragraph{FCOS: Moving Toward Anchor-Free Detection}
While RetinaNet and previous detectors relied on \textbf{anchor boxes} (predefined bounding box templates), \textbf{FCOS} took a different approach. It introduced an \textbf{anchor-free detection framework}, treating object detection as a per-pixel regression problem, similar to semantic segmentation. Instead of relying on predefined priors, FCOS predicted bounding boxes directly at each spatial location. This simplified the detection pipeline by removing anchor hyperparameters while maintaining strong performance.

\paragraph{YOLO: A Widely Used Real-Time Detector}
Parallel to these developments, the \textbf{YOLO (You Only Look Once)} family of detectors emerged as a dominant force in real-time applications. YOLO takes a different approach by treating detection as a global regression problem, dividing the image into a grid and predicting bounding boxes and class probabilities in a single forward pass. Over successive versions, YOLO has been continuously refined for accuracy and efficiency, making it one of the most popular and influential object detection frameworks.

\paragraph{Looking Ahead: Transformers and SOTA Detectors}
While this chapter focused on CNN-based object detectors, modern detection frameworks have evolved further with \textbf{transformer-based architectures}. Models such as the \textbf{DEtection TRansformer (DETR)} and its variants eliminate explicit region proposal mechanisms and instead treat detection as a set prediction problem using attention. In parallel, strong self-supervised vision transformers (for example, DINOv2) provide powerful backbone representations that can be fine-tuned for detection and segmentation tasks. As we progress in this document, we will explore several examples of \textbf{state-of-the-art (SOTA)} detectors that leverage transformers to push the boundaries of both accuracy and efficiency.

\paragraph{Summary}
Modern object detection has progressed from region-based CNNs to one-stage and transformer-based architectures:
\begin{itemize}
	\item \textbf{R-CNN} introduced region-based detection but was very slow.
	\item \textbf{Fast/Faster R-CNN} amortized feature computation and learned region proposals via \textbf{RPNs}, enabling end-to-end training.
	\item \textbf{FPNs} added multi-scale feature hierarchies, improving performance on small objects.
	\item \textbf{RetinaNet} showed that one-stage detectors can match and surpass two-stage accuracy using \textbf{Focal Loss}.
	\item \textbf{FCOS} and such detectors simplified design by predicting boxes directly at each location.
	\item The \textbf{YOLO} family popularized real-time, grid-based detection.
	\item \textbf{Transformer-based detectors} (e.g., DETR) remove proposal stages entirely and rely on attention over image features.
\end{itemize}

These developments build on one another to yield today’s accurate, efficient, and scalable detection frameworks; later chapters will revisit them in the context of transformer-based vision models.

\begin{enrichment}[Detection Transformer (DeTR)][section]
	\label{enr:chapter14_detr_intro}
	
	\noindent
	The \textbf{Detection Transformer (DeTR)} \cite{carion2020_detr} is a seminal work that brought the transformer architecture into the object detection domain. Developed by Facebook AI Research (FAIR), DeTR introduced a novel framework that reformulates object detection as a \textbf{direct set prediction problem}, eliminating many traditional hand-crafted components like anchor boxes, region proposals, and non-maximum suppression (NMS).
	
	\begin{figure}[H]
		\centering
		\includegraphics[width=0.9\textwidth]{Figures/Chapter_18/DeTR_Architecture.jpg}
		\caption{Overview of the DeTR architecture. An image is passed through a CNN backbone (e.g., ResNet-50) to produce a feature map. These features are flattened and fed into a transformer encoder. A decoder attends to learned object queries and outputs a fixed number of predictions, each corresponding to a potential object. Adapted from \cite{carion2020_detr}.}
		\label{fig:chapter14_detr_architecture}
	\end{figure}
	
	\paragraph{Architecture Overview}
	\begin{itemize}
		\item The input image is first encoded by a convolutional backbone (e.g., ResNet-50), yielding a spatial feature map.
		\item The flattened feature map is treated as a sequence and passed through a transformer encoder.
		\item A transformer decoder receives a fixed number  \(N\) of learned object queries and produces a corresponding set of \(N\) object predictions.
		\item Each prediction outputs both a class label and a bounding box.
	\end{itemize}
	
	\paragraph{Why Transformers for Detection?}
	DeTR leverages the global self-attention of transformers to enable long-range dependency modeling across the image. Whereas CNN-based detectors often rely on local context and multi-scale heuristics to infer object presence, transformers can integrate information from the entire image holistically in a single forward pass.
	
	\noindent
	However, this global modeling comes with a key design shift: DeTR produces a \textbf{fixed-size set of predictions}—typically \(N = 100\)—for every image, regardless of how many objects are present. This architectural choice is critical: it allows DeTR to frame detection as a set-to-set matching problem, enabling end-to-end training using a bipartite matching loss.
	
	\noindent
	This design immediately raises a natural question: \emph{What happens when the number of actual objects is fewer than \(N\)}?
	
	\noindent
	We address this in the next subsection, where we explore how DeTR matches predictions to targets using bipartite matching, and how “no-object” padding plays a central role in the loss function and training dynamics.
	
	\newpage
	
	\begin{enrichment}[Matching Predictions and GT with No-Object Padding][subsection]
		\label{enr:chapter14_detr_matching}
		
		\noindent
		Building on the transformer encoder–decoder and self-attention mechanisms introduced in later chapters, \textbf{DEtection TRansformer (DETR)} \cite{carion2020_detr} revisits object detection as a \emph{set prediction} problem. Instead of producing a variable number of candidate boxes that must be filtered by anchors and non-maximum suppression (NMS), DETR passes image features through a transformer and predicts a fixed-size set of \(N\) object candidates per image (typically \(N = 100\)), each trained to correspond to at most one object (or a dedicated “no-object” slot).
		
		\paragraph{Challenge:}
		Most images contain fewer than \(N\) objects. This creates a mismatch between the number of predictions and the number of ground-truth annotations (\(M < N\)). How can we supervise all predictions consistently?
		
		\paragraph{Solution: No-Object Padding}
		To address this, DETR pads the ground-truth set with \textbf{“no-object”} entries—placeholder targets that carry a special background class label. The model is trained to recognize these as background predictions.
		
		\begin{itemize}
			\item Let the image contain \(M\) annotated boxes.
			\item The padded target set is expanded to size \(N\), by appending \(N - M\) dummy targets with a designated “no-object” class label.
			\item This allows a \emph{one-to-one matching} between predicted boxes and targets using the Hungarian algorithm, even when many targets are artificial.
		\end{itemize}
		
		\begin{figure}[H]
			\centering
			\includegraphics[width=0.8\textwidth]{Figures/Chapter_18/DeTR_Predictions_Vs_PaddedGT.jpg}
			\caption{
				\textbf{Prediction–Ground Truth Matching in DeTR.}
				DETR always outputs a fixed number \(N\) of predictions per image. To supervise all predictions uniformly, the ground-truth set is padded with “no-object” entries so its size matches \(N\). The Hungarian algorithm computes an optimal one-to-one matching between predictions and padded targets. Most predictions are matched to background entries, regularizing the model to produce confident “no-object” classifications for irrelevant tokens.
			}
			\label{fig:chapter14_detr_predictions_vs_paddedgt}
		\end{figure}
		
		\paragraph{Hungarian Matching:}
		Matching is solved globally using the Hungarian algorithm, which assigns each prediction to exactly one target (real or padded) to minimize the total matching cost:
		
		\[
		\mathcal{L}_{\text{match}}(i,j) = \lambda_{\text{cls}} \cdot \text{CE}(\hat{c}_i, c_j) + \lambda_{\text{L1}} \cdot \lVert \hat{b}_i - b_j \rVert_1 + \lambda_{\text{GIoU}} \cdot \bigl(1 - \text{GIoU}(\hat{b}_i, b_j)\bigr)
		\]
		
		\paragraph{Implementation Snippet:}
		\begin{mintedbox}[bgcolor=black!3, fontsize=\small, linenos]{python}
			# Assume:
			# targets = List[Dict] with keys 'boxes' and 'labels'
			# num_queries = fixed number of DETR outputs (e.g., 100)
			padded_targets = []
			
			for tgt in targets:
			boxes  = tgt["boxes"]   # [num_objects, 4]
			labels = tgt["labels"]  # [num_objects]
			
			num_objs = boxes.size(0)
			pad_size = num_queries - num_objs
			
			# Pad with dummy boxes and no-object class label (e.g., 91 for COCO)
			padded_boxes  = F.pad(boxes,  (0, 0, 0, pad_size))  # [num_queries, 4]
			padded_labels = F.pad(labels, (0, pad_size), value=no_object_class)
			
			padded_targets.append({
				"boxes": padded_boxes,
				"labels": padded_labels
			})
		\end{mintedbox}
		
		\paragraph{Why This Matters:}
		This matching-and-padding design:
		
		\begin{itemize}
			\item \textbf{Eliminates} the need for anchor boxes or NMS.
			\item \textbf{Supervises} every prediction, even those matched to background.
			\item \textbf{Enables} fully end-to-end training with standard classification and regression losses.
		\end{itemize}
		
		\noindent
		By framing detection as bipartite matching, DETR achieves a clean and interpretable training objective. In the following subsection, we’ll detail the final loss function and how it combines classification, L1 distance, and GIoU penalties over the matched pairs.
		
	\end{enrichment}
	
	\begin{enrichment}[Hungarian Matching Loss and Bounding Box Optimization][subsection]
		 \label{enr:chapter14_detr_loss}
		 
		 \noindent
		 After performing bipartite matching between predicted and ground truth boxes (see \autoref{enr:chapter14_detr_matching}), DETR computes a loss over these matched pairs to optimize both class predictions and bounding box regressions. This is known as the \textbf{Hungarian loss}, and it operates over a \emph{permutation} of predictions that minimizes the overall cost.
		 
		 \paragraph{Step 1: Optimal Bipartite Matching}
		 Let the ground truth set be \( y = \{y_1, \dots, y_N\} \), padded with “no-object” entries if the image contains fewer than \( N \) objects. Each element \( y_i = (c_i, b_i) \) contains a class label \( c_i \in \{1, \dots, K\} \cup \{\varnothing\} \) and a bounding box \( b_i \in [0,1]^4 \). Similarly, let \( \hat{y} = \{\hat{y}_1, \dots, \hat{y}_N\} \) be the \( N \) predictions, where each \( \hat{y}_j = (\hat{c}_j, \hat{b}_j) \).
		 
		 We now seek a permutation \( \hat{\sigma} \in \mathfrak{S}_N \) (the set of all permutations over \( N \) elements) that minimizes the total matching cost:
		 
		 \[
		 \hat{\sigma} = \underset{\sigma \in \mathfrak{S}_N}{\arg\min} \sum_{i=1}^N \mathcal{L}_\text{match}(y_i, \hat{y}_{\sigma(i)}).
		 \]
		 
		 This permutation defines a unique one-to-one mapping between each ground truth box and a model prediction.
		 
		 \paragraph{Step 2: Matching Cost Definition}
		 
		 The pairwise cost function accounts for classification and box quality:
		 
		 \[
		 \mathcal{L}_\text{match}(y_i, \hat{y}_{\sigma(i)}) = 
		 -\ind_{\{c_i \neq \varnothing\}} \cdot \hat{p}_{\sigma(i)}(c_i) 
		 + \ind_{\{c_i \neq \varnothing\}} \cdot \mathcal{L}_\text{box}(b_i, \hat{b}_{\sigma(i)}),
		 \]
		 
		 where:
		 \begin{itemize}
		 	\item \( \hat{p}_{\sigma(i)}(c_i) \) is the predicted probability for class \( c_i \),
		 	\item \( \mathcal{L}_\text{box} \) is a bounding box regression loss (see below),
		 	\item \( \ind \) denotes the indicator function (equal to 1 when the condition holds, 0 otherwise).
		 \end{itemize}
		 
		 The indicator ensures that background (\( c_i = \varnothing \)) entries do not contribute to the loss.
		 
		 \paragraph{Step 3: Final Loss Computation}
		 
		 Once the optimal matching \( \hat{\sigma} \) is found, the Hungarian loss is computed as:
		 
		 \[
		 \mathcal{L}_\text{Hungarian}(y, \hat{y}) = \sum_{i=1}^N \left[
		 - \log \hat{p}_{\hat{\sigma}(i)}(c_i)
		 + \ind_{\{c_i \neq \varnothing\}} \cdot \mathcal{L}_\text{box}(b_i, \hat{b}_{\hat{\sigma}(i)})
		 \right].
		 \]
		 
		 In practice, DETR downweights the classification loss for no-object classes by a factor of 10 to reduce class imbalance effects.
		 
		 \paragraph{Bounding Box Loss: Smooth L1 and GIoU Components}
		 \label{par:detr_bounding_box_loss_components}
		 
		 Once a ground truth box \( b_i \) is matched with a predicted box \( \hat{b}_{\sigma(i)} \) (via the Hungarian algorithm), DETR computes a localization loss that balances \textbf{numerical precision} and \textbf{spatial alignment}. This is achieved through a combination of \textbf{Smooth L1} (Huber) loss and \textbf{Generalized IoU (GIoU)} loss.
		 
		 \subparagraph{1. Smooth L1 Loss (Huber Variant)}
		 
		 The Smooth L1 loss—also known as the \textbf{Huber loss}—is a robust alternative to standard L1 or L2 losses. It behaves like an L2 loss near zero (ensuring smooth gradients) and like an L1 loss for larger errors (ensuring robustness to outliers). Formally:
		 
		 \[
		 \text{SmoothL1}(x) = 
		 \begin{cases}
		 	0.5 \cdot \frac{x^2}{\beta}, & \text{if } |x| < \beta \\
		 	|x| - 0.5 \cdot \beta, & \text{otherwise}
		 \end{cases}
		 \]
		 
		 \noindent
		 The hyperparameter \( \beta \) controls the transition point between the quadratic and linear regimes. For DETR, \( \beta = 1.0 \) is typically used. This makes the box regression more stable, especially during early training.
		 
		 \begin{mintedbox}[bgcolor=black!3, fontsize=\small, linenos]{python}
		 	# Smooth L1 (Huber) loss for bounding box regression
		 	import torch.nn.functional as F
		 	
		 	smooth_l1 = F.smooth_l1_loss(
		 	pred_boxes, target_boxes,
		 	reduction="none", beta=1.0
		 	)
		 \end{mintedbox}
		 
		 \noindent
		 Despite being a coordinate-wise loss, Smooth L1 doesn’t account for the box's spatial shape or overlap. This is where GIoU comes in.
		 
		 \subparagraph{2. Generalized IoU (GIoU) Loss}
		 
		 Intersection over Union (IoU) is a classic metric for bounding box overlap:
		 \[
		 \text{IoU}(A, B) = \frac{|A \cap B|}{|A \cup B|}.
		 \]
		 However, IoU suffers from a key weakness: if two boxes do not overlap, IoU is 0, providing no learning signal—regardless of how close the boxes are spatially.
		 
		 To overcome this, \cite{rezatofighi2019_giou} proposed the \textbf{Generalized IoU (GIoU)}:
		 
		 \[
		 \text{GIoU}(A, B) = \text{IoU}(A, B) - \frac{|C \setminus (A \cup B)|}{|C|},
		 \]
		 where \( C \) is the \emph{smallest enclosing box} that fully contains both \( A \) and \( B \). This makes GIoU sensitive to the spatial distance between non-overlapping boxes.
		 
		 \begin{itemize}
		 	\item \( C \) is found by taking the tightest box covering both \( A \) and \( B \), using min and max operations over the corners.
		 	\item When \( A \) and \( B \) overlap perfectly, GIoU reduces to IoU.
		 	\item When \( A \cap B = \emptyset \), GIoU is negative, providing a gradient toward reducing their separation.
		 \end{itemize}
		 
		 \begin{figure}[H]
		 	\centering
		 	\includegraphics[width=0.75\textwidth]{Figures/Chapter_18/giou_illustration.jpg}
		 	\caption{
		 		\textbf{Illustration of GIoU behavior.} Although both examples have IoU = 0, the left prediction is spatially closer to the ground truth box than the right. GIoU correctly assigns a higher similarity to the left, allowing for useful gradients even when IoU = 0. Credit: Jinsol Kim.
		 	}
		 	\label{fig:chapter14_giou_illustration}
		 \end{figure}
		 
		 \begin{mintedbox}[bgcolor=black!3, fontsize=\small, linenos]{python}
		 	from torchvision.ops import generalized_box_iou
		 	
		 	# GIoU loss: 1 - GIoU score
		 	giou = generalized_box_iou(pred_boxes, target_boxes)
		 	giou_loss = 1.0 - giou
		 \end{mintedbox}
		 
		 \subparagraph{3. Combining Smooth L1 and GIoU}
		 
		 Each loss captures a different notion of box quality:
		 
		 \begin{itemize}
		 	\item \textbf{Smooth L1 (Huber)}: Enforces numerical closeness between box coordinates (good for center, width, height alignment).
		 	\item \textbf{GIoU}: Encourages spatial alignment and overlap—especially helpful when predictions are far from the target.
		 \end{itemize}
		 
		 DETR combines the two:
		 \[
		 \mathcal{L}_\text{box}(b_i, \hat{b}_{\sigma(i)}) = 
		 \lambda_\text{L1} \cdot \text{SmoothL1}(b_i, \hat{b}_{\sigma(i)}) +
		 \lambda_\text{GIoU} \cdot \left(1 - \text{GIoU}(b_i, \hat{b}_{\sigma(i)})\right),
		 \]
		 where \( \lambda_\text{L1}, \lambda_\text{GIoU} \) are loss weights (e.g., 5.0 and 2.0 in the DETR paper).
		 
		 \paragraph{Conclusion}
		 
		 By blending coordinate-wise error with geometric overlap, DETR ensures that the model:
		 \begin{itemize}
		 	\item Learns to predict numerically accurate box coordinates,
		 	\item Gains spatial awareness even when predictions are initially far off,
		 	\item Receives informative gradients during all training phases.
		 \end{itemize}
		 
		 This elegant combination supports DETR’s end-to-end detection approach. Now that we’ve explored how predictions are matched and optimized via loss functions, we proceed to examine the \textbf{architecture and flow} of DETR, from feature extraction to transformer decoding and output prediction.
		 
	\end{enrichment}
	
	\begin{enrichment}[Architecture Overview][subsection]
		\label{enr:chapter14_detr_architecture}
		
		\noindent
		\textbf{DETR} integrates convolutional and transformer-based modules in an end-to-end object detection pipeline. The overall architecture consists of:
		
		\begin{enumerate}
			\item A convolutional backbone (e.g., ResNet-50 or ResNet-101) that extracts dense visual features.
			\item A transformer encoder-decoder that models global interactions and predicts \(N\) object candidates.
			\item A bipartite matching and loss computation mechanism to supervise predictions (see \autoref{enr:chapter14_detr_loss}).
		\end{enumerate}
		
		\paragraph{1.\ CNN Backbone}
		The input image \( X \in \mathbb{R}^{3 \times H_0 \times W_0} \) passes through a CNN backbone (e.g., ResNet-50), producing an activation map:
		\[
		f \in \mathbb{R}^{C \times H \times W}, \quad \text{where } C = 2048,\quad H = H_0 / 32,\quad W = W_0 / 32.
		\]
		These activations represent coarse spatial features extracted by the CNN. A \(1 \times 1\) convolution reduces the channel dimension from \(C\) to \(d = 256\), yielding \(d\)-dimensional patch embeddings. These are then flattened into a sequence of \(HW\) tokens, each representing a spatial location.
		
		\paragraph{2.\ Transformer Encoder}
		Each of the \(HW\) flattened patch vectors is enriched with a \textbf{2D sine/cosine positional encoding} and then passed through a standard transformer encoder (multi-head self-attention + MLP with residuals and LayerNorm). Unlike NLP models (e.g., BERT, GPT), DETR uses longer sequences (\(HW \approx 900\)) but with smaller hidden size (\(d = 256\)) to accommodate memory constraints.
		
		\begin{figure}[H]
			\centering
			\includegraphics[width=0.9\textwidth]{Figures/Chapter_18/slide_122.jpg}
			\caption{Overall DeTR architecture. A CNN backbone extracts image features that are fed into a transformer encoder. The decoder receives \(N\) learned object queries to generate predictions.}
			\label{fig:chapter14_detr_overall_arch}
		\end{figure}
		
		\paragraph{3.\ Learned Object Queries and Transformer Decoder}
		
		\begin{figure}[H]
			\centering
			\includegraphics[width=0.6\textwidth]{Figures/Chapter_18/DeTR_Transformer_Encoder_Decoder.jpg}
			\caption{Transformer architecture in DETR. The encoder aggregates image features. The decoder uses learned object queries to generate one output per prediction slot. Adapted from \cite{carion2020_detr}.}
			\label{fig:chapter14_detr_transformer_arch}
		\end{figure}
		
		The decoder takes in \(N = 100\) learnable vectors called \emph{object queries}, each intended to produce one detection result. 
		These vectors are randomly initialized and updated during training to “ask” different questions about the image content.
		
		\begin{itemize}
			\item The encoder outputs serve as \textbf{keys and values}.
			\item The learned queries serve as \textbf{queries} in the decoder's cross-attention layers.
		\end{itemize}
		
		This mirrors the original Transformer decoder from \cite{vaswani2017_attention}, adapted for detection instead of autoregressive text generation.
		
		\paragraph{4.\ Interpreting Object Queries}
		Each object query can be imagined as an attention-driven \textit{question}, probing the image for different object types or regions.
		
		\begin{figure}[H]
			\centering
			\includegraphics[width=0.8\textwidth]{Figures/Chapter_18/detr_box_predictions.jpg}
			\caption{Specialization of object queries across COCO images. Each prediction slot learns to attend to specific regions and box sizes. Color represents box scale and orientation. Source: \cite{carion2020_detr}.}
			\label{fig:chapter14_detr_box_query_specialization}
		\end{figure}
		
		For example, in the above figure \ref{fig:chapter14_detr_box_query_specialization}, the colored boxes might be asking the following questions:
		
		\begin{itemize}
			\item \textcolor{purple}{“What small object is in the bottom-left?”}
			\item \textcolor{pink}{“Is there something large in the center?”}
		\end{itemize}
		
		Through training, each query vector specializes, covering distinct spatial areas, object sizes, or semantics. This is visualized in the following figure.
		
		\paragraph{5.\ Why Attention is a Natural Fit}
		
		Transformers are inherently suited for modeling pairwise relationships—making them a natural match for object detection, where understanding spatial interactions is key.
		
		\begin{figure}[H]
			\centering
			\includegraphics[width=0.35\textwidth]{Figures/Chapter_18/example_attention_matrix.png}
			\caption{\textbf{Attention as Bounding Box Proxy.} Entries in the attention matrix may reflect spatial relationships between image regions—suggesting how attention can implicitly capture bounding box-like structures. This interpretation, proposed by \href{https://www.youtube.com/watch?v=T35ba_VXkMY}{Yannic Kilcher}.}
			\label{fig:chapter14_detr_attention_matrix}
		\end{figure}
		
		\noindent
		Hence, the encoder’s attention matrix (\(HW \times HW\)) can be viewed as modeling how each spatial location attends to others—implicitly capturing potential object extents. Though DeTR does not exploit this directly, it highlights how attention mechanisms align naturally with the structure of visual tasks, hinting at promising directions for future work in detection and region proposal learning.
		
	\end{enrichment}
	
	\begin{enrichment}[DeTR Results, Impact, and Follow-Up Work][subsection]
		\label{enr:chapter14_detr_results}
		
		\noindent
		The introduction of \textbf{DEtection TRansformer (DeTR)} \cite{carion2020_detr} marked a turning point in object detection by demonstrating that transformer-based architectures can achieve strong results \emph{without anchors or non-maximum suppression (NMS)}. DeTR generalizes remarkably well across:
		\begin{itemize}
			\item \textbf{Objects of varying sizes:} from small to large.
			\item \textbf{Different object counts:} from sparse to cluttered scenes.
			\item \textbf{Challenging layouts:} producing high-quality and coherent predictions.
		\end{itemize}
		
		\noindent
		DeTR’s learned object queries attend to semantically meaningful regions in the image. Some queries specialize in detecting small objects, others cover large or central regions, and many converge to interpretable modes that persist across datasets.
		
		\paragraph{From Detection to Segmentation}
		Thanks to its global attention mechanism and fixed set of learned queries, DeTR can be extended to perform \textbf{panoptic segmentation}. Instead of just bounding boxes, DeTR predicts a binary mask for each detected object in parallel. These masks are then merged using pixel-wise argmax, yielding instance segmentation results.
		
		\begin{figure}[H]
			\centering
			\includegraphics[width=0.85\textwidth]{Figures/Chapter_18/semantic_segmentation_detr.jpg}
			\caption{\textbf{Object-wise mask prediction in DeTR.} Binary masks are predicted independently for each object query. These are later merged using a pixel-wise argmax operation, enabling detailed instance-level segmentation. Adapted from \cite{carion2020_detr}, Figure 8.}
			\label{fig:chapter14_detr_segmentation_masks}
		\end{figure}
		
		\begin{figure}[H]
			\centering
			\includegraphics[width=0.7\textwidth]{Figures/Chapter_18/detr_semantic_segmentation.jpg}
			\caption{\textbf{Panoptic segmentation with DeTR-R101.} DETR can segment both “things” (object instances) and “stuff” (amorphous background regions) in a unified manner. The consistency and alignment of masks show that DETR learns strong spatial and semantic priors. Adapted from \cite{carion2020_detr}.}
			\label{fig:chapter14_detr_panoptic}
		\end{figure}
		
		\paragraph{Real-World Usage: HuggingFace Implementation}
		The practicality of DeTR has led to wide adoption in research and industry. For example, the HuggingFace Computer Vision Course provides a user-friendly notebook for \emph{fine-tuning DeTR on custom datasets}, demonstrating its flexibility:
		\begin{center}
			\href{https://github.com/johko/computer-vision-course/blob/main/notebooks/Unit%203%20-%20Vision%20Transformers/Fine-tuning%20Vision%20Transformers%20for%20Object%20detection.ipynb}{\texttt{Try DETR fine-tuning here}}
		\end{center}
		
		\paragraph{Follow-Up Works and Extensions}
		
		Since its release, DeTR has inspired a rich line of research focused on addressing its main limitations—particularly training speed and convergence—while extending its capabilities:
		
		\begin{itemize}
			\item \textbf{DAB-DeTR} \cite{liu2022_dab_detr} was one of the first major improvements. It introduced \emph{dynamic anchor boxes} by injecting learnable reference points into the object queries. This allowed the model to more effectively initialize and refine box predictions throughout training, leading to faster convergence and improved accuracy.
			
			\item \textbf{DN-DeTR} \cite{li2022_dn_detr} further addressed the slow training issue by adding a \emph{denoising training objective}. During training, noisy object queries are added and explicitly supervised, which stabilizes learning and accelerates convergence. This technique makes DeTR more competitive in terms of training time without sacrificing accuracy.
			
			\item \textbf{Re-DETR} \cite{zhu2023_re_detr} builds on both prior ideas and rethinks the decoder itself. It enables \emph{iterative refinement} of predictions across decoder layers, where each stage progressively improves upon previous outputs. This dramatically speeds up convergence and reduces the computational footprint—bringing DeTR closer to real-time inference scenarios.
			
			\item Finally, \textbf{NMS Strikes Back} \cite{sun2023_nms_strikes_back} challenges one of DeTR’s founding principles: the removal of non-maximum suppression. This work shows that reintroducing a lightweight form of NMS can help refine predictions and improve performance in crowded scenes—suggesting that hybrid approaches can sometimes outperform purist, end-to-end designs.
		\end{itemize}
		
		\paragraph{Broader Impact}
		DeTR reshaped object detection by:
		\begin{itemize}
			\item Eliminating the need for hand-designed anchors and post-processing.
			\item Enabling a unified architecture for detection, segmentation, and panoptic tasks.
			\item Inspiring a new wave of research around \textbf{set prediction} in vision.
		\end{itemize}
		
		\noindent
		Its clean, end-to-end formulation led to more interpretable and modular designs, with applications extending beyond vision to robotics, remote sensing, and beyond.
		
		\paragraph{Conclusion}
		DeTR is a prime example of how \textbf{Vision Transformers (ViTs)} can be used to build practical, high-performance systems in computer vision. Despite being architecturally different from traditional CNNs, ViTs can now tackle nearly every major vision task—\textit{classification}, \textit{detection}, \textit{segmentation}, and more.
		
		\medskip
		
		\noindent
		\textbf{The takeaway:} Vision Transformers are an evolution—not a revolution. They offer a different lens through which we solve the same core problems. But with strong hardware alignment (favoring matrix multiplications over convolutions), ViTs often train and run faster than CNNs at comparable FLOPs. More importantly, they provide a seamless path toward \textbf{multi-modal} understanding, as seen in models like CLIP and Vision-Language Models (VLMs), empowering unified reasoning across image, text, and video.
	\end{enrichment}
\end{enrichment}

\newpage

\begin{enrichment}[Grounding DINO: DINO with Grounded
	Pre-Training][section]

\begin{enrichment}[Motivation and Problem Setting][subsection]
	\label{enr:chapter14_groundingdino_motivation}
	
	\subsubsection{Motivation and Problem Setting}
	\label{subsubsec:chapter14_groundingdino_motivation}
	
	Classical object detectors such as Faster R-CNN or RetinaNet assume a \emph{closed-set} label space: the model is trained and evaluated on a fixed, finite list of categories (e.g., the 80 COCO classes). This assumption is incompatible with many real applications, where users wish to detect arbitrary concepts specified at test time by free-form text prompts (e.g., ``person holding a red ball'', ``traffic light with a red arrow''). In this regime, models must \emph{understand language} and \emph{localize novel categories} without box-level supervision for every possible concept.
	
	\medskip
	
	Grounding DINO~\cite{liu2023_groundingdino} addresses this problem by ``marrying'' a strong DETR-style detector (DINO-DETR~\cite{zhang2022_dino}) with \emph{grounded pre-training} on large-scale image–text data. The model is designed to:
	
	\begin{itemize}
		\item Support \textbf{closed-set detection} on standard datasets such as COCO by fine-tuning on box-annotated data.
		\item Enable \textbf{open-set detection} by conditioning on prompts containing arbitrary category names or phrases.
		\item Handle \textbf{referring expression comprehension} (REC), where the input is a single phrase (e.g., ``the man in a red shirt'') and the goal is to localize exactly the described instance.
	\end{itemize}
	
	\noindent
	The following figure (reproduced from~\cite{liu2023_groundingdino}) highlights the conceptual difference between closed-set detection and open-set phrase grounding, and illustrates an image-editing application obtained by coupling Grounding DINO with Stable Diffusion~\cite{rombach2022_ldm}.
	
	\begin{figure}[H]
		\centering
		\includegraphics[width=0.85\textwidth]{Figures/Chapter_14/GroundingDINO_concept.jpg}
		\caption{\textbf{Closed-set vs.\ open-set detection in Grounding DINO.} (a) Closed-set detectors predict boxes over a fixed label set. (b) Grounding DINO conditions on free-form text prompts and is evaluated on novel categories and REC benchmarks. (c) Example image editing application by combining Grounding DINO with Stable Diffusion~\cite{liu2023_groundingdino}. Figure adapted from~\cite{liu2023_groundingdino}.}
		\label{fig:chapter14_groundingdino_concept}
	\end{figure}
	
	\newpage
	
	\subsubsection{Grounding DINO: Multi-Level Language Fusion}
	\label{subsubsec:chapter14_groundingdino_fusion}
	
	Grounding DINO transforms the closed-set detector DINO-DETR into an open-vocabulary learner by injecting language supervision at \textbf{three tightly coupled stages} of the architecture~\cite{liu2023_groundingdino}. Rather than processing the image in isolation and classifying boxes against a fixed vocabulary, Grounding DINO treats detection as a \emph{progressive alignment} between visual features and a text prompt (e.g., ``furry animal on grass''). The same BERT-extracted text embeddings are threaded through the feature enhancer (encoder), the language-guided query initialization, and the cross-modality decoder, so that all components operate in a shared vision--language space.
	
	\begin{figure}[H]
		\centering
		\includegraphics[width=0.85\textwidth]{Figures/Chapter_14/GroundingDINO_openset.jpg}
		\caption{\textbf{From closed-set to open-set detection.} Grounding DINO conceptually divides a DINO-DETR-style detector into three phases and injects text into each~\cite{liu2023_groundingdino}. (A) A \textbf{Feature Enhancer} performs early, bi-directional fusion between image and text features, supervised by encoder-level detection and contrastive grounding losses (Contrastive Loss~A). (B) \textbf{Language-guided query selection} initializes decoder queries from encoder tokens that are most similar to the text in a shared feature space. (C) A \textbf{Cross-modality decoder} iteratively refines queries via image and text cross-attention, with decoder-level detection and grounding losses (Contrastive Loss~B).}
		\label{fig:chapter14_groundingdino_openset}
	\end{figure}
	
\end{enrichment}
	
	\begin{enrichment}[Method][subsection]
		\label{enr:chapter14_groundingdino_method}
		
		\paragraph{Core idea and progressive fusion philosophy}
		
		Grounding DINO~\cite{liu2023_groundingdino} upgrades the strong closed-set detector DINO-DETR~\cite{zhang2022_dino} into an open-vocabulary detector by replacing fixed classifier weights with \textbf{region-to-phrase matching} in a shared embedding space. Instead of predicting logits over a pre-defined label set, each region representation is compared (via dot products) to text-token embeddings produced by a BERT encoder. The same text features are woven into the network at three points so that early image--text alignment directly supports query initialization and final decoding. Architecturally, Grounding DINO combines a DETR-style set-prediction detector equipped with multi-scale deformable attention (from Deformable DETR/DINO-DETR~\cite{zhu2021_deformabledetr,zhang2022_dino}) with GLIP-style grounded pre-training on large detection + caption corpora~\cite{li2022_glip}, but with much stronger cross-modal fusion in both encoder and decoder.
		
		It is helpful to view the architecture as a \emph{three-phase refinement cascade} on top of a dual encoder. These phases are \emph{conceptual} stages within a \emph{single end-to-end forward pass}: they are \emph{not} trained separately. In every training step, the model runs through Phases~A, B, and C once, and losses from the encoder and decoder are backpropagated jointly.
		
		\newpage
		
		\begin{itemize}
			\item \textbf{Phase A: Feature Enhancer encoder.} A bi-directional image--text fusion encoder with deformable attention and dense encoder-level grounding (Contrastive Loss~A). It produces grounded image and text tokens, encoder-level box predictions, and a dense \emph{region--to--token similarity map}, whose rows are anchored to spatial image locations and whose columns correspond to text tokens. Although the image features are repeatedly updated by self- and cross-attention, these layers only change the \emph{contents} of the token vectors: they never reorder the sequence or modify each token’s associated positional/reference coordinates. The \(i\)-th output token therefore still corresponds to the same backbone cell (position \(\mathbf{p}_i\), scale \(s_i\)) as the \(i\)-th input token. This preserved spatial correspondence allows Phase~B to interpret each row of \(\tilde{X}_I \tilde{X}_T^\top\) as a spatial heatmap over the prompt and to convert high-scoring image tokens, together with their encoder-predicted boxes, into dynamic decoder anchors.
			\item \textbf{Phase B: Language-guided query selection.} A deterministic module that uses the Phase-A similarity map and encoder box predictions to seed decoder queries at text-relevant locations with \emph{dynamic} anchors.
			\item \textbf{Phase C: Cross-modality decoder.} A DINO-DETR-style decoder enriched with text cross-attention and decoder-level grounding (Contrastive Loss~B). It refines this query set into final region--phrase predictions.
		\end{itemize}
		Each phase refines the previous one: Phase~A aligns dense tokens and learns encoder-level box predictions; Phase~B converts the strongest region--text matches into sparse queries with dynamic anchors; Phase~C iteratively refines these queries using image and text, producing final boxes and open-vocabulary scores. Importantly, \textbf{bounding boxes are predicted in both Phase~A and Phase~C}: the encoder boxes (of Phase~A) provide deep supervision and good anchors, while the decoder boxes (of Phase~C) are the final outputs.
		
		\paragraph{Phase A: Feature Enhancer and encoder-level grounding}
		
		Phase~A operates on top of two unimodal encoders and gradually pulls their tokens into a shared vision--language space. Crucially, all subsequent attention and feed-forward blocks act on the \emph{features} of each token while preserving the token ordering and its associated positional information: a token that originated from backbone cell \((x_i,y_i)\) on level \(s_i\) remains the \(i\)-th image token throughout the Feature Enhancer. Attention may aggregate information from many locations and from text, but it never changes which spatial cell a given token index refers to. At the end of Phase~A we therefore still have:
		\begin{itemize}
			\item \emph{Grounded image tokens} whose indices and positional encodings anchor them to specific receptive-field regions in the image.
			\item \emph{Grounded text tokens} tied to individual words or sub-words in the prompt.
			\item Encoder-level box predictions attached to each image token.
			\item A dense similarity matrix \(M = \tilde{X}_I \tilde{X}_T^\top\) that can be read as a \emph{region--to--token} map, because rows correspond to spatially anchored image tokens and columns to text tokens.
		\end{itemize}
		Phase~B will compress this dense similarity information into a sparse set of text-guided queries, and all of this is trained jointly via Contrastive Loss~A.
		
		\medskip
		
		\noindent\textbf{Inputs and notation (dual encoder).}
		\begin{itemize}
			\item \textbf{Image backbone (Swin Transformer).}  
			A Swin Transformer~\cite{liu2021_swin}, pre-trained on large-scale classification and optionally further tuned in a DINO-DETR detector~\cite{zhang2022_dino}, produces multi-scale feature maps
			\[
			X_I^{(s)} \in \mathbb{R}^{H_s \times W_s \times d_{\text{img}}}, \quad s \in \{1,\dots,S\}.
			\]
			Each map is projected by a \(1\times 1\) convolution to a shared hidden dimension \(d = 256\) and flattened to a sequence
			\[
			X_I^{(s)} \in \mathbb{R}^{N_s \times d}, \quad N_s = H_s W_s.
			\]
			Concatenating all scales yields the image token sequence
			\[
			X_I \in \mathbb{R}^{N_I \times d}, \quad N_I = \sum_{s=1}^{S} N_s,
			\]
			where each row is tied to a specific spatial location and stride in the feature pyramid.
			
			\item \textbf{Text backbone (BERT with sub-sentence prompts).}  
			The text branch uses a BERT-base encoder~\cite{devlin2019_bert}. The prompt \(T\) is a single string of phrases separated by delimiters (e.g., ``\texttt{cat . baseball glove . fire hydrant .}''), a format that will later support sub-sentence masking. After tokenization and BERT encoding, a linear projection maps BERT’s hidden states into the same dimension \(d\):
			\[
			X_T \in \mathbb{R}^{N_T \times d}, \quad N_T \leq 256.
			\]
		\end{itemize}
		
		At this stage, \(X_I \in \mathbb{R}^{N_I \times d}\) and \(X_T \in \mathbb{R}^{N_T \times d}\) share dimension \(d\), but originate from disjoint pre-training regimes (vision vs.\ language). The Swin backbone has never seen text; BERT has never seen images. Phase~A is responsible for pulling these two streams into a shared, grounded space.
		
		\medskip
		
		\noindent\textbf{Multi-scale deformable attention (MSDeformAttn).}
		\label{enr:chapter14_groundingdino_deformable_attention}
		
		Before describing the Feature Enhancer sub-blocks, it is helpful to recall the multi-scale deformable attention module reused from Deformable DETR~\cite{zhu2021_deformabledetr} and DINO-DETR~\cite{zhang2022_dino}. It appears both in the encoder (Phase~A) and in the decoder (Phase~C).
		
		Suppose the backbone outputs \(S\) feature levels:
		\[
		X_I^{(s)} \in \mathbb{R}^{H_s \times W_s \times d}, \quad N_s = H_s W_s, \quad N_I = \sum_{s=1}^{S} N_s.
		\]
		Each token corresponds to a scale \(s\) and a coordinate \(\mathbf{p} = (x,y)\) with normalized reference point \(\mathbf{r}_i^{(s)} \in [0,1]^2\).
		
		\medskip
		
		\noindent\textbf{Dense self-attention baseline.}
		
		Standard self-attention over all image tokens would compute, for queries \(Q\), keys \(K\), and values \(V\),
		\[
		\mathrm{SA}(i) = \sum_{j=1}^{N_I} \alpha_{ij} V_j, \quad
		\alpha_{ij} = \text{softmax}_j\left(\frac{Q_i K_j^\top}{\sqrt{d_k}}\right),
		\]
		with \(O(N_I^2)\) complexity and no explicit use of the multi-scale structure beyond positional encodings.
		
		\medskip
		
		\noindent\textbf{Multi-scale deformable attention.}
		
		Deformable attention replaces dense summation with \emph{sparse, geometry-aware sampling}. For each query token \(i\), head \(h\), scale \(s\), and sampling index \(k \in \{1,\dots,K\}\), the module predicts:
		\begin{itemize}
			\item \textbf{Offsets} \(\Delta \mathbf{p}_i^{(h,s,k)} \in \mathbb{R}^2\).
			\item \textbf{Unnormalized weights} \(A_i^{(h,s,k)} \in \mathbb{R}\).
		\end{itemize}
		Sampling locations in normalized coordinates are
		\[
		\mathbf{p}_i^{(h,s,k)} = \mathbf{r}_i^{(s)} + \Delta \mathbf{p}_i^{(h,s,k)}.
		\]
		Feature values are obtained via bilinear interpolation on scale-\(s\) feature maps:
		\[
		\mathbf{v}_i^{(h,s,k)} = \text{BilinearSample}\bigl(X_I^{(s)}, \mathbf{p}_i^{(h,s,k)}\bigr) \in \mathbb{R}^{d/H}.
		\]
		After normalizing \(A_i^{(h,s,k)}\) over all \((s,k)\) for each head to obtain \(\tilde{A}_i^{(h,s,k)}\), deformable attention produces:
		\[
		\mathrm{MSDeformAttn}(i) =
		\sum_{h=1}^{H} W_h^{\mathrm{out}}
		\left(
		\sum_{s=1}^{S}
		\sum_{k=1}^{K}
		\tilde{A}_i^{(h,s,k)}\, \mathbf{v}_i^{(h,s,k)}
		\right),
		\]
		where \(W_h^{\mathrm{out}}\) are per-head output projections. Complexity is \(O(N_I H S K)\), linear in the number of tokens.
		
		In the Feature Enhancer (Phase~A), this operation refines each image token before any image--text cross-attention. A token at stride~16 near a cat’s ear, for example, can sample:
		\begin{itemize}
			\item \textbf{Finer details} from stride-8 features (fur texture, edge details).
			\item \textbf{Similar-scale neighbors} from stride-16 features (shape continuity).
			\item \textbf{Coarser context} from stride-32 features (overall body and background).
		\end{itemize}
		
		In the decoder (Phase~C), the same module is used as cross-attention from queries to image features. Each query \(q_k^{(l)}\) has a current anchor box \((c_x,c_y,w,h)\), which is converted into one or several reference points across scales. For each head and scale, the module predicts offsets and weights, samples a few positions near the anchor, and aggregates them as above. This lets each query:
		\begin{itemize}
			\item \textbf{Look locally around its current box guess} across all scales.
			\item \textbf{Refine its internal representation} with multi-scale evidence.
			\item \textbf{Prepare for text fusion} by providing a geometry-aware visual summary to the subsequent text cross-attention.
		\end{itemize}
		This MSDeformAttn block is therefore the main mechanism by which Grounding DINO inherits the efficiency and multi-scale robustness of Deformable DETR/DINO-DETR~\cite{zhu2021_deformabledetr,zhang2022_dino}.
		
		\medskip
		
		\noindent After defining MSDeformAttn, we can now describe the four sub-blocks of each Feature Enhancer layer. At layer \(\ell\), we keep image tokens \(X_I^{(\ell)} \in \mathbb{R}^{N_I \times d}\) and text tokens \(X_T^{(\ell)} \in \mathbb{R}^{N_T \times d}\); each layer applies:
		\begin{enumerate}
			\item Deformable self-attention on image tokens using MSDeformAttn.
			\item Masked self-attention on text tokens.
			\item Image-to-text cross-attention.
			\item Text-to-image cross-attention, followed by modality-specific FFNs.
		\end{enumerate}
		We describe these in turn.
		
		\medskip
		
		\noindent\textbf{(A1) Deformable self-attention on image tokens.}
		
		Using the MSDeformAttn module described above, the image stream is refined as
		\[
		\hat{X}_I^{(\ell)} = \text{MSDeformSelfAttn}\bigl(X_I^{(\ell)}\bigr), \quad \hat{X}_I^{(\ell)} \in \mathbb{R}^{N_I \times d}.
		\]
		
		\newpage
		
		Here \(X_I^{(\ell)}\) is the concatenation of all backbone scales; each token has an associated reference point across the multi-scale pyramid, and MSDeformSelfAttn aggregates a small, learned set of samples around that point across all levels. Conceptually, each patch token becomes a compact, geometry-aware summary of its local multi-scale neighborhood, rather than a raw backbone descriptor, which stabilizes the subsequent cross-modal alignment.
		
		\medskip
		
		\noindent\textbf{(A2) Text self-attention with sub-sentence mask.}
		
		Text tokens are refined by masked self-attention:
		\[
		\hat{X}_T^{(\ell)} = \text{SelfAttn}\bigl(X_T^{(\ell)}, \text{mask}_{\text{sub-sent}}\bigr), \quad \hat{X}_T^{(\ell)} \in \mathbb{R}^{N_T \times d}.
		\]
		In open-vocabulary detection, the prompt is typically a concatenation of many category phrases and referring expressions. Naive word-level attention would let ``cat'' attend to ``baseball glove'', mixing unrelated semantics. Grounding DINO avoids this by constructing a \emph{sub-sentence} mask from simple punctuation conventions:
		\begin{itemize}
			\item \textbf{Prompt formatting.} The prompt is written as a single string where phrases are separated by delimiters (e.g., ``\texttt{.}'').
			\item \textbf{Segment assignment.} After tokenization, each token is assigned a segment id according to which phrase it belongs to.
			\item \textbf{Masked attention.} The attention mask \(\text{mask}_{\text{sub-sent}}\) permits attention only within the same segment; cross-phrase entries are set to zero, yielding a block-diagonal pattern.
		\end{itemize}
		For example, for
		\[
		\texttt{"a small brown dog . red car . person wearing a blue hat ."},
		\]
		we obtain segments such as:
		\begin{itemize}
			\item \textbf{Segment 0.} Tokens of ``a small brown dog''.
			\item \textbf{Segment 1.} Tokens of ``red car''.
			\item \textbf{Segment 2.} Tokens of ``person wearing a blue hat''.
		\end{itemize}
		Tokens inside each phrase still see all of their local context (adjectives, prepositions, compound nouns), while phrases remain cleanly separated as independent detection targets. This sub-sentence representation is exactly the option found empirically best in the Grounding DINO paper~\cite{liu2023_groundingdino}, and it is reused wherever text self-attention appears.
		
		\paragraph{Sub-sentence text representation}
		\label{subsubsec:chapter14_groundingdino_method_textrepr}
		
		The paper compares three ways of encoding prompts~\cite{liu2023_groundingdino}:
		\begin{itemize}
			\item \textbf{Sentence-level.} Each phrase is encoded in a separate BERT pass and pooled; this preserves intra-phrase structure but is inefficient and discards token-level detail.
			\item \textbf{Word-level.} All phrases are concatenated and encoded jointly with full self-attention; this is efficient but allows spurious cross-phrase interactions.
			\item \textbf{Sub-sentence-level.} All phrases are encoded jointly, but self-attention is masked to stay within each phrase, as in (A2). This keeps intra-phrase context, prevents cross-phrase contamination, and amortizes BERT computation.
		\end{itemize}
		This representation is used consistently in both Phase~A and Phase~C whenever text attention appears.
		
		\begin{figure}[H]
			\centering
			\includegraphics[width=0.85\textwidth]{Figures/Chapter_14/GroundingDINO_text_representations.jpg}
			\caption{\textbf{Text representation levels in Grounding DINO.} (a) Sentence-level: separate encoding per phrase. (b) Word-level: joint encoding with full self-attention across all tokens. (c) Sub-sentence-level: joint encoding with masked self-attention restricted within each phrase. Grounding DINO adopts (c) to obtain clean, separable embeddings for each category while amortizing BERT computation across phrases~\cite{liu2023_groundingdino}.}
			\label{fig:chapter14_groundingdino_textrepr}
		\end{figure}
		
		\medskip
		
		\noindent\textbf{(A3) Image-to-text cross-attention (\(I \rightarrow T\)): text tokens collect visual evidence.}
		
		Once the unimodal streams have been strengthened, Phase~A begins cross-modal fusion. First, text tokens query the image tokens:
		\[
		\tilde{X}_T^{(\ell)} = \text{CrossAttn}_{I \rightarrow T}\bigl(Q = \hat{X}_T^{(\ell)},\, K = \hat{X}_I^{(\ell)},\, V = \hat{X}_I^{(\ell)}\bigr), \quad \tilde{X}_T^{(\ell)} \in \mathbb{R}^{N_T \times d}.
		\]
		In matrix shapes:
		\begin{align*}
			Q_T^{(\ell)} &= W_q \hat{X}_T^{(\ell)} \in \mathbb{R}^{N_T \times d_q},\\
			K_I^{(\ell)} &= W_k \hat{X}_I^{(\ell)} \in \mathbb{R}^{N_I \times d_k},\\
			V_I^{(\ell)} &= W_v \hat{X}_I^{(\ell)} \in \mathbb{R}^{N_I \times d_v},
		\end{align*}
		with learned projections \(W_q, W_k, W_v\). The attention matrix is
		\[
		A_{T \leftarrow I}^{(\ell)} = \text{softmax}\!\left(\frac{Q_T^{(\ell)} K_I^{(\ell)\top}}{\sqrt{d_q}}\right) \in \mathbb{R}^{N_T \times N_I},
		\]
		where each row gives weights from one text token to all image tokens. The cross-attended update is
		\[
		\hat{U}_T^{(\ell)} = A_{T \leftarrow I}^{(\ell)} V_I^{(\ell)} \in \mathbb{R}^{N_T \times d_v}.
		\]
		With a residual path back to dimension \(d\), the new text tokens are
		\[
		\tilde{X}_T^{(\ell)} = \hat{X}_T^{(\ell)} + \hat{U}_T^{(\ell)}.
		\]
		Each row of \(\tilde{X}_T^{(\ell)}\) is thus a \emph{mixture} of:
		\begin{itemize}
			\item \textbf{A linguistic component} coming from the original BERT embedding \(\hat{X}_T^{(\ell)}\).
			\item \textbf{A visual component} given by a weighted sum of image tokens \(V_I^{(\ell)}\).
		\end{itemize}
		For the token encoding ``cat'', the corresponding row of \(A_{T \leftarrow I}^{(\ell)}\) peaks on image locations that look like cats (fur, face, whiskers), so the visual component aggregates those regions. The residual connection keeps the text token anchored in language space while adding an image-dependent correction that reflects \emph{how this particular image instantiates ``cat''}. Shape-wise, the number of text tokens remains \(N_T\); only their contents change.
		
		\medskip
		
		\noindent\textbf{(A4) Text-to-image cross-attention (\(T \rightarrow I\)): image tokens pull semantics from text.}
		
		Next, information flows in the opposite direction: image tokens query the now image-conditioned text tokens:
		\[
		\tilde{X}_I^{(\ell)} = \text{CrossAttn}_{T \rightarrow I}\bigl(Q = \hat{X}_I^{(\ell)},\, K = \tilde{X}_T^{(\ell)},\, V = \tilde{X}_T^{(\ell)}\bigr), \quad \tilde{X}_I^{(\ell)} \in \mathbb{R}^{N_I \times d}.
		\]
		In matrix form,
		\begin{align*}
			Q_I^{(\ell)} &= W'_q \hat{X}_I^{(\ell)} \in \mathbb{R}^{N_I \times d_q},\\
			K_T^{(\ell)} &= W'_k \tilde{X}_T^{(\ell)} \in \mathbb{R}^{N_T \times d_k},\\
			V_T^{(\ell)} &= W'_v \tilde{X}_T^{(\ell)} \in \mathbb{R}^{N_T \times d_v},
		\end{align*}
		and
		\[
		A_{I \leftarrow T}^{(\ell)} = \text{softmax}\!\left(\frac{Q_I^{(\ell)} K_T^{(\ell)\top}}{\sqrt{d_q}}\right) \in \mathbb{R}^{N_I \times N_T},
		\qquad
		\hat{V}_I^{(\ell)} = A_{I \leftarrow T}^{(\ell)} V_T^{(\ell)} \in \mathbb{R}^{N_I \times d_v}.
		\]
		With a residual path,
		\[
		\tilde{X}_I^{(\ell)} = \hat{X}_I^{(\ell)} + \hat{V}_I^{(\ell)}.
		\]
		
		Each row of \(\tilde{X}_I^{(\ell)}\) therefore becomes a mixture of:
		\begin{itemize}
			\item \textbf{A visual component} inherited from the backbone and deformable self-attention.
			\item \textbf{A semantic component} given by a weighted sum of text tokens that best explain that region.
		\end{itemize}
		For a patch on the cat’s ear, the corresponding row of \(A_{I \leftarrow T}^{(\ell)}\) has high weight on the tokens of the ``cat'' phrase (and possibly modifiers such as ``small'' or ``brown'') and low weight on unrelated phrases such as ``fire hydrant''. The updated feature becomes a visually grounded but \emph{text-aligned} representation of that patch. Importantly, while the feature vector mixes information from many locations and tokens, the \emph{index} of each image token (and its reference point in the pyramid) still tells us from which patch of the input it originated; attention moves information, not coordinates. The image token grid and multi-scale structure remain intact; only the feature vectors are rotated in the joint space.
		
		\noindent\textbf{(A5) FFNs, progressive alignment, and Contrastive Loss~A.}
		
		After the two cross-attention directions, modality-specific FFNs with residual connections are applied:
		\[
		X_I^{(\ell+1)} = \text{FFN}_I\bigl(\tilde{X}_I^{(\ell)}\bigr), \quad
		X_T^{(\ell+1)} = \text{FFN}_T\bigl(\tilde{X}_T^{(\ell)}\bigr).
		\]
		Stacking \(L_{\text{enh}}\) layers yields a sequence of transformations
		\[
		(X_I^{(0)}, X_T^{(0)}) \rightarrow (X_I^{(1)}, X_T^{(1)}) \rightarrow \dots \rightarrow (X_I^{(L_{\text{enh}})}, X_T^{(L_{\text{enh}})}),
		\]
		
		\newpage
		
		where at each level:
		\begin{itemize}
			\item \textbf{Text tokens} evolve from generic BERT embeddings into mixtures of linguistic content and the image regions that instantiate each phrase in the current image.
			\item \textbf{Image tokens} evolve from purely visual patches into mixtures of visual content and the phrase embeddings that best describe them.
		\end{itemize}
		Because both branches live in \(\mathbb{R}^{d}\), we can form the similarity matrix
		\[
		M = \tilde{X}_I \tilde{X}_T^\top \in \mathbb{R}^{N_I \times N_T},
		\]
		which is precisely the dense \emph{region--to--token similarity map} mentioned above, now written explicitly as an image-token–to–text-token affinity matrix. Concretely:
		\begin{itemize}
			\item Row \(i\) of \(\tilde{X}_I\) is the embedding \(z_i^\top \in \mathbb{R}^{1 \times d}\) of the \(i\)-th \emph{image token}, which originated from a specific backbone cell (a patch at location \((x_i,y_i)\) on some feature level) and now encodes a context-enriched representation of that patch.
			\item Row \(j\) of \(\tilde{X}_T\) is the embedding \(t_j^\top \in \mathbb{R}^{1 \times d}\) of the \(j\)-th \emph{text token}, anchored to a particular word or sub-word (e.g., ``cat'', ``glove'', ``blue'') within its phrase.
		\end{itemize}
		The entry
		\[
		M_{ij} = z_i^\top t_j
		\]
		is therefore the compatibility between the patch-level token at spatial location \(i\) and the word-level token \(j\). Each \emph{row} of \(M\) is a score vector over all words for a single spatial token, and each \emph{column} is a score vector over all spatial tokens for a single word. It is thus natural to interpret \(M\) as a dense \emph{region--to--token affinity map}, which Phase~B will reuse for language-guided query selection.
		
		To \emph{drive} this alignment, Grounding DINO attaches detection heads directly to the encoder outputs and applies \textbf{Contrastive Loss~A}~\cite{liu2023_groundingdino,li2022_glip}. Each image token \(z_i\) (row of \(\tilde{X}_I\)) is treated as a candidate region:
		\begin{itemize}
			\item \textbf{Box regression (encoder-level boxes).}  
			For each image token \(z_i\), which is tied to a particular backbone cell with center \((x_i,y_i)\) and stride \(s_i\), a small MLP predicts a 4D box vector
			\[
			\hat{b}_i = (\hat{c}_x,\hat{c}_y,\hat{w},\hat{h})
			\]
			in normalized image coordinates. Conceptually, the head starts from a \emph{default} box centered at the token’s patch center \((x_i,y_i)\) with a size proportional to the feature-map stride \(s_i\), and learns offsets and scale changes around this default (mirroring the reference-point box parameterization used in Deformable DETR and DINO-DETR~\cite{zhu2021_deformabledetr,zhang2022_dino}). Hungarian matching is then applied between the set of encoder-level predictions \(\{\hat{b}_i\}\) and ground-truth boxes, with a cost that combines classification (from the contrastive scores) and geometry (L1 and GIoU) as in DETR-style detectors~\cite{carion2020_detr}; matched tokens are trained with L1 and GIoU losses. These encoder-level boxes are \emph{not} the final outputs: they (1) provide deep supervision that teaches each encoder token to propose a box anchored at its own patch, and (2) supply the \emph{dynamic anchors} that Phase~B will reuse when initializing decoder queries.
			
			\item \textbf{Contrastive classification (Contrastive Loss~A).}  
			Instead of a fixed classifier matrix, classification is performed by comparing \(z_i\) to all text tokens \(t_j\) (rows of \(\tilde{X}_T\)):
			\[
			u_{ij} = z_i^\top t_j,\qquad j = 1,\dots,N_T.
			\]
			
			\newpage
			
			For a token \(z_i\) matched (via Hungarian) to a ground-truth box annotated with a phrase, a small subset of text tokens (those belonging to that phrase) are labeled as positives; all other tokens are negatives. This yields a highly imbalanced multi-label problem: per image token, the vast majority of word tokens are negatives, just as most anchors in dense detectors are background. Grounding DINO therefore uses a \emph{focal-style multi-label contrastive loss} (inspired by GLIP~\cite{li2022_glip}), which down-weights easy negatives and focuses learning on hard negatives and the few positive token matches. In effect, this applies a CLIP-style contrastive objective at \emph{dense} spatial locations, while addressing the severe foreground--background imbalance that arises in detection.
		\end{itemize}
		Gradients from Contrastive Loss~A propagate through all Feature Enhancer layers, training the network to use its cross-attention blocks so that corresponding region and phrase features become similar and unrelated pairs become dissimilar. To summarize, by the end of Phase~A, \(\tilde{X}_I\) and \(\tilde{X}_T\) form a well-aligned pair of token sets, \(M = \tilde{X}_I \tilde{X}_T^\top\) behaves as a high-quality region--to--word affinity map, and each image token carries an encoder-level box prediction \(\hat{b}_i\) that will be exploited in Phase~B.
		
		\begin{figure}[H]
			\centering
			\includegraphics[width=0.85\textwidth]{Figures/Chapter_14/GroundingDINO_framework.jpg}
			\caption{\textbf{Grounding DINO framework.} A Swin image backbone and a BERT text backbone feed a multi-layer \textbf{Feature Enhancer} with deformable image self-attention and bi-directional image--text cross-attention. A \textbf{language-guided query selection} module then selects encoder tokens highly similar to the text prompt to initialize many decoder queries. A \textbf{cross-modality decoder} alternates query self-attention, deformable image cross-attention, and text cross-attention to produce text-grounded detections~\cite{liu2023_groundingdino}.}
			\label{fig:chapter14_groundingdino_framework}
		\end{figure}
		
		\newpage
		
		\paragraph{Phase B: Language-guided query selection}
		
		Phase~B takes the dense, grounded encoder tokens from Phase~A and converts them into a sparse set of decoder queries that are already biased toward text-relevant regions. Crucially, Phase~B is a \emph{loss-free}, deterministic transformation executed inside the same forward pass: it does not introduce new parameters or a separate training stage. Instead, it harvests the information created by Contrastive Loss~A---both the affinity matrix \(M = \tilde{X}_I \tilde{X}_T^\top\) and the encoder-level box predictions \(\hat{b}_i\)---to build a strong ``warm start'' for the decoder.
		
		After Phase~A we have \(\tilde{X}_I \in \mathbb{R}^{N_I \times d}\) and \(\tilde{X}_T \in \mathbb{R}^{N_T \times d}\). Each \emph{image token} \(\tilde{X}_I[i,:]\) is still associated with a particular spatial cell in the backbone pyramid: it originated from a specific feature map level \(s_i\) and grid location \((x_i,y_i)\) (a patch of the input image), inherited that position’s reference point for deformable attention, and now carries an encoder-level box prediction \(\hat{b}_i\) anchored around that patch. Cross-attention has mixed information between patches and phrases, but the token indices and positional encodings retain the link ``token \(i\) \(\leftrightarrow\) receptive-field region centered at \((x_i,y_i)\)''.
		
		\medskip
		
		\noindent\textbf{(B1) Scoring encoder tokens by text similarity.}
		
		Using the grounded features, Grounding DINO reuses the similarity matrix
		\begin{equation}
			S = \tilde{X}_I \tilde{X}_T^\top \in \mathbb{R}^{N_I \times N_T},
		\end{equation}
		
		where \(S_{ij}\) measures how similar image token \(i\) is to text token \(j\). To obtain a single relevance score per image token,
		\begin{equation}
			s_i = \max_j S_{ij}, \quad i = 1,\dots,N_I.
		\end{equation}
		This asks, for each spatial token: \emph{Does this look strongly like any word or phrase in the prompt?} The max pools over all words and avoids rewarding locations that weakly match many unrelated words.
		
		The indices of the top \(N_q\) tokens under this score,
		\begin{equation}
			\mathcal{I}_{N_q} = \mathrm{Top}_{N_q}\bigl(s_1,\dots,s_{N_I}\bigr),
			\quad N_q \approx 900 \text{ in the reference configurations},
		\end{equation}
		are taken as language-guided encoder locations~\cite{liu2023_groundingdino}. These are precisely the tokens that Phase~A and Contrastive Loss~A have already made strongly aligned with some phrase.
		
		\medskip
		
		\noindent\textbf{(B2) From encoder tokens to dynamic anchor boxes.}
		
		Each index \(i \in \mathcal{I}_{N_q}\) corresponds to:
		\begin{itemize}
			\item \textbf{A spatial position} \((x_i, y_i)\) and stride \(s_i\) in the feature pyramid (the patch from which the token originated).
			\item \textbf{An encoder-level box prediction} \(\hat{b}_i = (\hat{c}_x,\hat{c}_y,\hat{w},\hat{h})\) in normalized image coordinates, produced in Phase~A by the encoder head.
		\end{itemize}
		Following DAB-DETR and DINO-DETR~\cite{liu2022_dab_detr,zhang2022_dino}, Grounding DINO uses these encoder-level predictions directly as \textbf{dynamic anchor boxes} for decoder queries:
		\begin{itemize}
			\item \textbf{Anchor centers.} The initial anchor center \((c_x,c_y)\) of the query is set equal to the predicted center \((\hat{c}_x,\hat{c}_y)\).
			\item \textbf{Anchor sizes.} The initial anchor size \((w,h)\) is set equal to \((\hat{w},\hat{h})\), adapting to small objects at fine scales and large objects at coarse scales.
		\end{itemize}
		There is no separate box regression in Phase~B: Phase~B simply \emph{copies} the encoder’s prediction \(\hat{b}_i\) into the query’s anchor parameterization. 
		
		\newpage
		
		The tuple
		\[
		(c_x,c_y,w,h) := \hat{b}_i
		\]
		is then embedded (via the same sinusoidal box encoding + learned projections used in DAB-DETR/DINO-DETR) into a positional query vector. Because these anchors come from data-dependent encoder predictions instead of a fixed grid, they adapt to each image’s object sizes and locations. Conceptually, Phase~A has already told us where each phrase is likely to appear; Phase~B turns those encoder boxes into starting points for the decoder. The subsequent refinement of these anchors into final boxes happens in Phase~C, not in Phase~B.
		
		\medskip
		
				\noindent\textbf{(B3) Content embeddings and mixed query selection}
		
		As in DAB-DETR/DINO-DETR~\cite{liu2022_dab_detr,zhang2022_dino}, each decoder query is factored into:
		\begin{itemize}
			\item \textbf{A positional part} given by an (embedded) anchor box \((c_x,c_y,w,h)\).
			\item \textbf{A content part} given by a learnable embedding in \(\mathbb{R}^d\), independent of spatial location.
		\end{itemize}
		Concretely, the model maintains a \emph{bank} of content embeddings
		\[
		E_{\text{content}} \in \mathbb{R}^{N_q \times d},
		\]
		which is a parameter matrix learned over the whole training set. For a single image, these \(N_q\) rows become the content part of that image’s queries. For a mini-batch of size \(B\), the same bank is \emph{tiled} across the batch, yielding a tensor of shape \(\mathbb{R}^{B \times N_q \times d}\). In this sense, the content embeddings are “shared across images”: the \emph{same} \(N_q\) learnable query vectors are reused for every image, but their \emph{positional} part (the anchor boxes) is image-dependent.
		
		Grounding DINO adopts DINO-DETR’s \emph{mixed} query strategy, in which the same content bank \(E_{\text{content}}\) is combined with anchors coming from three different sources:
		\begin{itemize}
			\item \textbf{Purely learned queries.}  
			A subset of queries uses anchors that are \emph{also} learned parameters, not tied to any encoder token or text. Intuitively, these queries act as generic “questions” that the decoder asks about every image, such as:
			\begin{itemize}
				\item “Is there any large object roughly in the center of the image?”.
				\item “Is there a small, elongated object near the top edge?”.
				\item “Is there a blob-like region with strong contrast anywhere?”.
			\end{itemize}
			Because both their content and positional parts are image-agnostic, they can learn reusable priors about common object layouts and backgrounds. They also ensure that, even if the language signal is weak or noisy, the decoder still has some DINO-like, text-free queries probing the scene.
			
			\item \textbf{Objectness-guided queries (DINO-style).}  
			Following DINO-DETR~\cite{zhang2022_dino}, a second subset of queries uses anchors taken from encoder tokens that look \emph{object-like}, according to a generic objectness score from the encoder-level detection head (language-agnostic foreground likelihood, as in DINO). These anchors inherit:
			\begin{itemize}
				\item \textbf{Centers and sizes} from the encoder’s box predictions at those tokens.
				\item \textbf{No direct dependence on the prompt text} in their selection.
			\end{itemize}
			Conceptually, these queries play the role of “proposal-like” queries: they start on regions the encoder suspects to contain \emph{some} object, regardless of which phrase is being asked. Grounding DINO keeps a small number of such DINO-style queries mainly for stability and backward compatibility with the strong closed-set detector it builds upon.
			
			\item \textbf{Language-guided queries.}  
			Finally, a large subset of queries uses anchors derived from the language-guided indices \(\mathcal{I}_{N_q}\) in (B1)–(B2). For each selected encoder token \(i \in \mathcal{I}_{N_q}\), we take its encoder-level prediction
			\[
			\hat{b}_i = (\hat{c}_x,\hat{c}_y,\hat{w},\hat{h})
			\]
			and set the query’s anchor to \((c_x,c_y,w,h) = \hat{b}_i\). These anchors are then embedded (via sinusoidal encodings and learned projections) and combined with rows of \(E_{\text{content}}\) to form the initial query set. Because the indices were chosen by high image–text similarity, these queries start exactly on regions that Phase~A has already aligned strongly with some phrase in the prompt.
		\end{itemize}
		
		In the reference configurations, language-guided queries occupy the majority of the query budget (hundreds out of the total \(N_q = 900\) queries), while the remaining queries are split between purely learned and DINO-style objectness-guided anchors. The exact numerical split is a hyperparameter rather than a core design point; what matters is that:
		\begin{itemize}
			\item \textbf{Language-guided queries} dominate, tightly coupling many queries to the prompt.
			\item \textbf{Purely learned queries} provide text-agnostic priors and a safety net when text supervision is weak or missing.
			\item \textbf{Objectness-guided queries} retain a small pool of DINO-like, proposal-style anchors focused on visually salient regions, independent of the textual phrasing.
		\end{itemize}
		In all cases, the content embeddings are shared across images, but the anchors (and thus the positional encodings) are recomputed per image, so each image still has its own \(N_q\) queries.
		
		Formally, the language-guided part of the selection can be summarized as:
		\begin{mintedbox}{python}
			def language_guided_query_selection(X_I, X_T, num_queries):
			    # X_I: [N_I, d] grounded image features (Phase A outputs)
			    # X_T: [N_T, d] grounded text features
			
			    # 1. Compute image–text similarity
			    S = X_I @ X_T.T              # [N_I, N_T]
			
			    # 2. Collapse over text to get one relevance score per image token
			    s = S.max(dim=1).values      # [N_I]
			
			    # 3. Take top-k indices as language-guided encoder locations
			    indices = s.topk(num_queries).indices  # [num_queries]
			    return indices
		\end{mintedbox}
		
		\noindent\textbf{Intuition: Phase B as a warm start.}
		
		In DINO-DETR, encoder-based queries are chosen using generic objectness scores, so the decoder must discover both \emph{where} objects are and \emph{what} they are~\cite{zhang2022_dino}. Grounding DINO retains this idea but adds a strong language-guided path. Phase~A + Contrastive Loss~A make tokens overlapping, say, a ``cat'' box highly similar to the ``cat'' text tokens; Phase~B then:
		\begin{itemize}
			\item Keeps some purely learned and objectness-guided queries to probe object-like regions in a prompt-agnostic way.
			\item \emph{Adds many more queries} whose anchors are copied directly from the highest-scoring image tokens under the prompt, i.e., tokens that already look like some phrase in the text.
		\end{itemize}
		
		\newpage
		
		The decoder in Phase~C therefore starts from a rich mixture of queries: some asking general, text-free questions about the scene, and many already centered on plausible objects that Phase~A believes correspond to the current prompt. This dramatically shrinks the decoder’s search space and makes it much easier to converge to accurate, text-grounded detections.
		
		\paragraph{Phase C: Cross-modality decoder and Contrastive Loss~B}
		
		Phase~C takes the full query set constructed in Phase~B---dominated by, but not limited to, language-guided queries---and refines it into final region--phrase predictions by repeatedly attending to image features (for geometry and appearance) and text tokens (for semantics). As with Phase~A, this decoder is trained jointly in a single end-to-end optimization: Contrastive Loss~B is applied on top of its outputs at each training step.
		
		Phase~C uses a DINO-DETR-style decoder with an additional text cross-attention block. It takes as input:
		\begin{itemize}
			\item \textbf{Initial queries} \(Q^{(0)} \in \mathbb{R}^{N_q \times d}\) with content and anchor components (some purely learned, some objectness-guided, many language-guided).
			\item \textbf{Grounded image features} \(\tilde{X}_I \in \mathbb{R}^{N_I \times d}\) from Phase~A.
			\item \textbf{Grounded text features} \(\tilde{X}_T \in \mathbb{R}^{N_T \times d}\) from Phase~A.
		\end{itemize}
		
		At decoder layer \(l\), four sub-blocks are applied:
		\begin{enumerate}
			\item \textbf{Query self-attention.}
			\[
			\hat{Q}^{(l)} = \text{SelfAttn}\bigl(Q^{(l)}\bigr),
			\]
			enabling queries to communicate, share information, and suppress duplicates (e.g., two queries that see the same object can negotiate which one will take responsibility).
			
			\item \textbf{Image deformable cross-attention (reuse of MSDeformAttn).}
			\[
			\tilde{Q}^{(l)} = \text{MSDeformCrossAttn}\bigl(\hat{Q}^{(l)}, \tilde{X}_I\bigr),
			\]
			where each query, using its current anchor as reference, applies the same MSDeformAttn mechanism as in Phase~A, but now as cross-attention from queries to image tokens. This lets each query sample a small set of positions around its anchor across all image scales, refining its geometric and appearance representation while keeping complexity linear.
			
			\item \textbf{Text cross-attention (new relative to DINO-DETR).}
			\[
			\bar{Q}^{(l)} = \text{CrossAttn}_{\text{text}}\bigl(\tilde{Q}^{(l)}, \tilde{X}_T\bigr),
			\]
			allowing each query to aggregate information from text tokens and decide which phrases best explain its current visual evidence. Conceptually, the query “asks” the prompt: \emph{given what I see around my anchor, am I a ``red car'', a ``person in blue hat'', or background?}
			
			\item \textbf{Feed-forward network.}
			\[
			Q^{(l+1)} = \text{FFN}\bigl(\bar{Q}^{(l)}\bigr).
			\]
		\end{enumerate}
		After \(L_{\text{dec}} = 6\) layers, each query \(q_i = Q^{(L_{\text{dec}})}_i\) encodes a candidate object with refined geometry, visual features, and text alignment.
		
		\newpage
		
		\noindent\textbf{Decoder-level supervision (Contrastive Loss~B).}
		
		The final queries are supervised similarly to DINO-DETR but with open-vocabulary classification~\cite{zhang2022_dino,liu2023_groundingdino}:
		\begin{itemize}
			\item \textbf{Box regression (decoder boxes).}  
			Hungarian matching assigns each ground-truth region–phrase pair to at most one query, and matched queries predict boxes trained with L1 and GIoU losses. These decoder-level predictions, not the encoder boxes, are the final outputs used at inference time; they refine the initial anchors copied from Phase~A.
			
			\item \textbf{Region-to-phrase contrastive classification (Contrastive Loss~B).}  
			For each matched query \(q_i\) and text token \(t_j\), logits
			\[
			\hat{u}_{ij} = q_i^\top t_j
			\]
			are computed and trained with a focal-like contrastive loss~\cite{liu2023_groundingdino}. As in Phase~A, each positive query is associated with a small subset of positive text tokens (those belonging to its ground-truth phrase), while all remaining tokens are negatives. The imbalance is even more severe here: most queries are background (or redundant) and most text tokens are irrelevant for any given query. Using a focal term again down-weights the many easy negatives (queries that clearly do not match a phrase, or phrases that clearly do not match a query) and forces the model to concentrate on hard negatives and the few positive region–token pairs.
		\end{itemize}
		
		The overall training objective combines encoder- and decoder-level terms (plus standard DETR-style auxiliary losses on intermediate decoder layers). Conceptually,
		\[
		\mathcal{L}
		= \lambda_{\text{A}}\, \mathcal{L}_{\text{enc}}^{\text{box+contr}}
		+ \lambda_{\text{B}}\, \mathcal{L}_{\text{dec}}^{\text{box+contr}}
		+ \text{auxiliary terms},
		\]
		where \(\mathcal{L}_{\text{enc}}^{\text{box+contr}}\) is Contrastive Loss~A on encoder tokens and \(\mathcal{L}_{\text{dec}}^{\text{box+contr}}\) is Contrastive Loss~B on decoder queries. Both losses are active from the beginning of training; there is no staged optimization. Encoder boxes are optimized to become good anchors and a strong grounding signal, while decoder boxes are optimized to become accurate final predictions.
		
		At inference, any phrase can be used without retraining: the prompt is encoded once, encoder and decoder run as usual, query–text dot products are computed, scores are aggregated at the phrase level, and NMS is applied over boxes whose scores exceed the chosen threshold for the phrase. This enables true open-vocabulary detection.
		
		\paragraph{Connections to prior work and overall impact}
		
		The main ingredients of Grounding DINO can be traced as follows:
		\begin{itemize}
			\item \textbf{DETR-style set prediction.} Inherited from DETR~\cite{carion2020_detr}, providing a query-based, order-free framework for detection.
			\item \textbf{Multi-scale deformable attention.} Adopted from Deformable DETR and DINO-DETR~\cite{zhu2021_deformabledetr,zhang2022_dino}, enabling efficient, high-resolution, multi-scale processing in both encoder (Phase~A) and decoder (Phase~C).
			\item \textbf{Mixed query selection and denoising training.} Taken from DINO-DETR~\cite{zhang2022_dino}, stabilizing optimization and improving convergence.
			\item \textbf{Grounded contrastive losses and large-scale grounding data.} Inspired by GLIP~\cite{li2022_glip}, now applied to Transformer encoder tokens and decoder queries instead of DyHead regions.
			\item \textbf{New components specific to Grounding DINO.} Introduced in~\cite{liu2023_groundingdino}:
			\begin{itemize}
				\item \textbf{A bi-directional Feature Enhancer} that combines deformable self-attention with symmetric image--text cross-attention to produce deeply grounded encoder features.
				\item \textbf{Language-guided query selection} based on encoder-level image--text similarity and encoder boxes, seeding many queries at text-relevant regions.
				\item \textbf{Text cross-attention in each decoder layer} to keep queries in direct dialogue with the prompt throughout refinement.
				\item \textbf{Sub-sentence text representation} that cleanly separates category phrases while amortizing BERT computation.
			\end{itemize}
		\end{itemize}
		This progressive fusion---global grounding in the encoder, text-guided query seeding, and iterative query–text dialogue in the decoder---yields strong zero-shot transfer (around \(52.5\) AP on COCO zero-shot detection) while remaining compatible with standard supervised fine-tuning on downstream detection datasets~\cite{liu2023_groundingdino}.
		
	\end{enrichment}
	
	The following table summarizes how Grounding DINO compares to other open-set detectors. Grounding DINO is distinctive in: (i) using a strong Transformer detector (DINO-DETR) as its base, (ii) fusing text at three levels (Phases~A, B, C), and (iii) operating on sub-sentence prompts for fine-grained grounding.
	
	\begin{table}[H]
		\centering
		\small
		\setlength{\tabcolsep}{4pt}
		\caption{\textbf{Comparison of open-set object detectors} (adapted from Table~1 in~\cite{liu2023_groundingdino}). ``Partial label'' denotes training on only part of the labels (e.g., base categories). Models are grouped by base detector, fusion pattern, CLIP usage, and text representation level.}
		\label{tab:groundingdino_table1}
		\resizebox{\linewidth}{!}{%
			\begin{tabular}{l l c c l c l l l l}
				\toprule
				\textbf{Model} &
				\multicolumn{3}{c}{\textbf{Model Design}} &
				\multicolumn{1}{c}{\begin{tabular}[c]{@{}c@{}}\textbf{Text Prompt}\\ \textbf{Represent.\ Level}\end{tabular}} &
				\multicolumn{1}{c}{\begin{tabular}[c]{@{}c@{}}\textbf{Closed-Set}\\ \textbf{COCO}\end{tabular}} &
				\multicolumn{3}{c}{\textbf{Zero-Shot Transfer}} &
				\multicolumn{1}{c}{\begin{tabular}[c]{@{}c@{}}\textbf{Referring Detection}\\ \textbf{RefCOCO/\,+/\,g}\end{tabular}} \\
				\cmidrule(lr){2-4}\cmidrule(lr){7-9}
				& \textbf{Base Detector} & \textbf{Fusion} & \textbf{CLIP} &  &  & \textbf{COCO} & \textbf{LVIS} & \textbf{ODinW} &  \\
				\midrule
				ViLD~\cite{gu2022_vild}                    & Mask R-CNN        & --  & \cmark & Sentence     & \cmark & Partial label & Partial label & --         & -- \\
				RegionCLIP~\cite{zhong2021_regionclip}     & Faster R-CNN      & --  & \cmark & Sentence     & \cmark & Partial label & Partial label & --         & -- \\
				FindIt~\cite{kuo2022_findit}              & Faster R-CNN      & A   & --     & Sentence     & \cmark & Partial label & --            & --         & Fine-tune \\
				MDETR~\cite{kamath2021_mdetr}             & DETR              & A,C & --     & Word         & --    & Fine-tune    & Zero-shot     & --         & Fine-tune \\
				DQ-DETR~\cite{liu2022_dqdetr}             & DETR              & A,C & --     & Word         & \cmark & Zero-shot    & --            & Fine-tune  & -- \\
				GLIP~\cite{li2022_glip}                   & DyHead            & A   & --     & Word         & \cmark & Zero-shot    & Zero-shot     & Zero-shot  & -- \\
				GLIPv2~\cite{zhang2022_glipv2}            & DyHead            & A   & --     & Word         & \cmark & Zero-shot    & Zero-shot     & Zero-shot  & -- \\
				OV-DETR~\cite{zang2022_ovdetr}            & Deformable DETR   & B   & \cmark & Sentence     & \cmark & Partial label & Partial label & --         & -- \\
				OWL-ViT~\cite{minderer2022_owlvit}        & --                & --  & \cmark & Sentence     & \cmark & Partial label & Partial label & Zero-shot  & -- \\
				DetCLIP~\cite{yao2022_detclip}            & ATSS              & --  & \cmark & Sentence     & --    & Zero-shot    & Zero-shot     & --         & -- \\
				OmDet~\cite{zhao2024_omdet}               & Sparse R-CNN      & C   & \cmark & Sentence     & \cmark & Zero-shot    & --            & --         & -- \\
				\textbf{Grounding DINO}~\cite{liu2023_groundingdino} & \textbf{DINO-DETR} & \textbf{A,B,C} & \textbf{\cmark} & \textbf{Sub-sentence} & \textbf{\cmark} & \textbf{Zero-shot} & \textbf{Zero-shot} & \textbf{Zero-shot} & \textbf{Zero-shot} \\
				\bottomrule
		\end{tabular}}
	\end{table}
	
\newpage
	
	\begin{enrichment}[Architecture and Implementation Details][subsection]
		\label{enr:chapter14_groundingdino_architecture}
		
		\paragraph{Architecture and training setup}
		
		From an implementation standpoint, Grounding DINO instantiates the dual-encoder, single-decoder design~\cite{liu2023_groundingdino} with a small and a large configuration:
		\begin{itemize}
			\item \textbf{Image backbone.}  
			The image encoder is a Swin Transformer~\cite{liu2021_swin}, either Swin-T (lightweight) or Swin-L (high-capacity). Both produce a four-level feature pyramid (e.g., strides \(1/4, 1/8, 1/16, 1/32\)). For detection, Grounding DINO follows DINO-DETR’s multi-scale ``4scale'' setup~\cite{zhang2022_dino}: several pyramid levels (typically three or four) are fed into deformable attention so that queries can aggregate fine details and coarse context. After a \(1\times 1\) projection, all image tokens live in a shared hidden dimension \(d=256\), with a typical token count \(N_I > 10^4\) per image~\cite{liu2023_groundingdino}.
			
			\item \textbf{Text encoder.}  
			The text branch is a BERT-base encoder~\cite{devlin2019_bert} applied once per image to a concatenated, sub-sentence-masked prompt (Section~\ref{subsubsec:chapter14_groundingdino_method_textrepr}). After a linear projection to \(d=256\), we obtain text tokens \(X_T \in \mathbb{R}^{N_T \times d}\) with \(N_T \leq 256\). These tokens are reused throughout Phase~A (Feature Enhancer), Phase~B (query selection), and Phase~C (decoder), so their representation quality and length bound directly affect both accuracy and memory.
			
			\item \textbf{Feature Enhancer and decoder depth.}  
			The cross-modal Feature Enhancer is implemented as a 6-layer module that alternates deformable self-attention on image tokens, masked self-attention on text tokens, and bi-directional image–text cross-attention (Section~\ref{enr:chapter14_groundingdino_deformable_attention}). Its outputs feed (i) encoder-level detection heads for Contrastive Loss~A and (ii) the language-guided query selection in Phase~B. The cross-modality decoder then applies 6 layers of query self-attention, image deformable cross-attention, text cross-attention, and FFNs, mirroring DINO-DETR but with the extra text branch~\cite{zhang2022_dino,liu2023_groundingdino}.
			
			\item \textbf{Compute regime.}  
			Swin-T variants are trained on 16 V100 GPUs with global batch size 32, while Swin-L variants use 64 A100 GPUs with batch size 64~\cite{liu2023_groundingdino}. The dominant memory and compute terms scale with \(N_I\) (image tokens), \(N_T\) (text tokens), and the number of queries \(N_q\): deformable attention is linear in \(N_I\), but the dense similarity \(S = \tilde{X}_I \tilde{X}_T^\top\) is \(O(N_I N_T)\). In practice, limiting \(N_T\) (sub-sentence prompts, token cap \(\leq 256\)) and using multi-scale deformable attention instead of dense attention are key to keeping training feasible.
		\end{itemize}
		
		After feature enhancement, the language-guided query selection module (Phase~B) operates purely on indices and metadata: it uses the encoder’s similarity matrix \(S \in \mathbb{R}^{N_I \times N_T}\) and encoder-level boxes \(\hat{b}_i\) to choose the top-\(N_q\) image tokens as anchor sources and to assign them dynamic anchor boxes (positional part), while attaching a shared bank of learnable content embeddings to form the full query set (Section~\ref{enr:chapter14_groundingdino_method}). No new parameters are introduced in this phase; it is a deterministic routing mechanism inside the same forward pass.
		
		\paragraph{Losses and supervision}
		
		Training follows a DETR-like set prediction formulation~\cite{carion2020_detr,zhang2022_dino} with \emph{two} levels of supervision:
		\begin{itemize}
			\item Encoder-level heads attached to \(\tilde{X}_I\) implement Contrastive Loss~A (Phase~A), providing dense supervision and dynamic anchors.
			\item Decoder-level heads attached to \(Q^{(l)}\) (at each decoder layer, and especially the last) implement Contrastive Loss~B (Phase~C), providing the final predictions.
		\end{itemize}
		For each predicted query at the decoder (and similarly for selected encoder tokens), the model outputs a bounding box and a vector of logits over text tokens.
		
		\newpage
		
		\begin{itemize}
			\item \textbf{Box regression}.  
			Each prediction is parameterized as a normalized box \((\hat{c}_x,\hat{c}_y,\hat{w},\hat{h})\). After Hungarian matching between predictions and ground-truth region–phrase pairs, matched boxes are trained with a combination of L1 loss and GIoU loss~\cite{rezatofighi2019_giou}, exactly as in DETR-style detectors~\cite{carion2020_detr,zhang2022_dino}. At the encoder level, this teaches patch tokens to localize objects directly at their originating spatial cells and yields dynamic anchors; at the decoder level, it produces the final detection boxes used at inference.
			
			\item \textbf{Classification via contrastive focal loss}.  
			Instead of predicting over a fixed label set, each encoder token or decoder query \(z_i\) is compared to all text tokens \(t_j\) by dot product,
			\[
			u_{ij} = z_i^\top t_j,
			\]
			so that \(u_{ij}\) scores how compatible prediction \(i\) is with token \(j\). This yields a vector of logits over \emph{text tokens}, not over a closed vocabulary. A contrastive focal loss, following GLIP~\cite{li2022_glip}, is applied per token~\cite{liu2023_groundingdino}:
			\begin{itemize}
				\item \textbf{Positives} are the tokens belonging to the phrase that labels the matched ground-truth box (e.g., all tokens in ``small brown dog'').
				\item \textbf{Negatives} are all other tokens in the prompt, including tokens of other phrases and implicit background.
			\end{itemize}
			Focal weighting is crucial here: the number of negatives per prediction is very large (dozens to hundreds of tokens), while the number of positives is tiny (a few tokens per phrase). The focal term down-weights easy negatives and up-weights hard, confusing ones, preventing the loss from being dominated by background tokens and letting the model focus on subtle distinctions between similar phrases. Contrastive Loss~A and Contrastive Loss~B share this structure but operate at different locations (encoder tokens vs.\ decoder queries); the paper reuses the same focal-style formulation for both~\cite{liu2023_groundingdino}.
			
			\item \textbf{Matching}.  
			Hungarian matching uses a weighted sum of three costs: classification, box L1, and GIoU, with weights \(2.0{:}5.0{:}2.0\), respectively~\cite{zhang2022_dino,liu2023_groundingdino}. The final training loss reuses the same components but with weights \(1.0{:}5.0{:}2.0\). Intuitively, the higher weight on the box L1 term in both matching and loss reflects the importance of precise localization, while contrastive classification is still strong enough to enforce correct phrase assignment.
			
			\item \textbf{Auxiliary supervision}.  
			As in DINO-DETR~\cite{zhang2022_dino}, auxiliary prediction heads after each decoder layer provide deep supervision, stabilizing training in the multi-layer decoder. Grounding DINO extends this idea by also attaching heads to the encoder outputs, so Contrastive Loss~A shapes the Feature Enhancer from the earliest layers onward. In practice, both encoder- and decoder-level heads use the same loss components (contrastive focal classification + box L1 + GIoU), but they serve different roles: encoder heads learn good anchors and dense grounding, while decoder heads learn the final, text-grounded detections.
		\end{itemize}
		
	\end{enrichment}
	
	\newpage
	
	\begin{enrichment}[Experiments and Ablation][subsection]
		\label{enr:chapter14_groundingdino_experiments}
		
		\paragraph{Quantitative trends on COCO, LVIS, ODinW, and RefCOCO}
		
		Grounding DINO is evaluated in zero-shot, few-shot, and full fine-tuning regimes on COCO, LVIS, ODinW, and referring expression benchmarks (RefCOCO/+/g)~\cite{liu2023_groundingdino}. Rather than focusing on specific numbers from Tables~2--5, it is more useful here to highlight the main patterns and relative comparisons.
		
		\begin{itemize}
			\item \textbf{COCO detection:} With a Swin-T backbone pre-trained on large-scale detection and grounding data (e.g., Objects365, GoldG), Grounding DINO attains zero-shot COCO AP in the high-40s, outperforming both DINO-DETR and GLIP with comparable backbones by a few AP points~\cite{zhang2022_dino,li2022_glip,liu2023_groundingdino}.
			
			Moving to a larger Swin-L backbone and richer pretraining (e.g., Objects365, OpenImages, GoldG) pushes zero-shot COCO performance into the low-50s AP range without any COCO images seen during pretraining, setting a strong zero-shot baseline among fully detector-style methods~\cite{liu2023_groundingdino}. After COCO fine-tuning, the Swin-T variant reaches AP in the low-60s, slightly surpassing a Swin-L DINO baseline despite using a smaller backbone, indicating that language-guided fusion directly benefits classic supervised detection as well.
			
			\item \textbf{LVIS long-tailed detection:} On LVIS, a zero-shot Grounding DINO model with Swin-T and broad pretraining (e.g., Objects365+GoldG+Cap4M) achieves overall AP in the mid-20s, slightly ahead of GLIP-T under similar constraints but still below DetCLIP-style models that leverage even larger caption corpora~\cite{gupta2019_lvis,li2022_glip,yao2023_detclipv2,liu2023_groundingdino}. The key observation comes after fine-tuning: Grounding DINO’s LVIS AP climbs into the low-50s, overtaking DetCLIPv2 with the same backbone while relying on less pretraining data~\cite{liu2023_groundingdino}. This suggests that its region–to–phrase formulation transfers particularly well once some task-specific supervision is available.
			
			\item \textbf{ODinW (Open-World Detection in the Wild):} On the ODinW benchmark, which aggregates many small detection datasets with diverse label spaces~\cite{gupta2022_odinw}, Grounding DINO with a Swin-T backbone matches GLIP-v2 in average AP across tasks while offering improved median AP, indicating more stable performance on difficult or low-data domains~\cite{zhang2022_glipv2,liu2023_groundingdino}. With a Swin-L backbone, Grounding DINO surpasses strong alternatives such as Florence in both average and median AP, despite using fewer parameters, reinforcing that the multi-level grounding architecture scales well with backbone capacity~\cite{yuan2021_florence,liu2023_groundingdino}.
			
			\item \textbf{Referring expression comprehension (RefCOCO/+/g):} For RefCOCO/+/g, zero-shot performance is moderate and broadly comparable to GLIP-type models, which is expected because these referring-expression benchmarks require fine-grained grounding and nuanced language understanding~\cite{liu2023_groundingdino}. Once fine-tuned on REC data, however, Grounding DINO with Swin-T already reaches accuracies close to 90\% on most RefCOCO splits, and the Swin-L variant pushes these numbers slightly higher, achieving state-of-the-art or near state-of-the-art results among compared REC models~\cite{liu2023_groundingdino}. Qualitatively, the model handles complex referring phrases (e.g., ``the person on the left holding an umbrella'') significantly better than detectors that only use language as a global tag.
		\end{itemize}
		
		Overall, the empirical results show a consistent pattern: with no COCO or LVIS supervision, Grounding DINO already achieves strong zero-shot detection performance across diverse datasets; with task-specific fine-tuning, it matches or surpasses specialized closed-set detectors, confirming that its open-vocabulary design does not compromise classical supervised accuracy~\cite{liu2023_groundingdino}.
		
		\paragraph{Ablation insights and lessons}
		
		Table~7 in~\cite{liu2023_groundingdino} and related ablations systematically disable individual components under a controlled Swin-T / Objects365 pretraining setting, evaluated in zero-shot on COCO and LVIS. Exact numbers depend on the precise training recipe, but the relative deltas are stable and highlight which language-aware components matter most.
		
		\begin{itemize}
			\item \textbf{Encoder-level image--text fusion (Feature Enhancer).}  
			Removing the 6-layer bi-directional Feature Enhancer and using purely visual encoder features (while keeping the rest of the architecture intact) produces the largest degradation. In the reported setting, COCO zero-shot AP drops by roughly \(3.5\) points and LVIS zero-shot AP by about \(4\)–\(4.5\) points compared to the full Grounding DINO model with encoder fusion enabled (Table~7, model~\#0 vs.\ \#1 in~\cite{liu2023_groundingdino}). The loss is particularly pronounced on LVIS rare categories, where many classes never appear in the supervised detector training but are present in the grounding pretraining data. Lesson: early, deep, bi-directional grounding in the encoder is the primary driver of open-vocabulary strength.
			
			\item \textbf{Language-guided query selection.}  
			Replacing Grounding DINO’s language-guided query selection with DINO-style generic encoder output queries (selected solely by objectness scores, independent of text) consistently weakens zero-shot performance. In the Swin-T / Objects365 ablation, COCO zero-shot AP drops by about \(1.5\)–\(2.0\) points and LVIS zero-shot AP by roughly \(3.0\) points when text similarity is \emph{not} used to rank encoder tokens (Table~7, model~\#1 vs.\ \#2 in~\cite{liu2023_groundingdino}). When queries are instead seeded from tokens with high image–text similarity, the model recovers those points and, in particular, detects more rare LVIS categories with fewer high-confidence but semantically wrong boxes. Lesson: initializing queries at text-relevant locations, instead of generic objectness hotspots, is crucial for robust open-vocabulary.
			
			\item \textbf{Text cross-attention in the decoder.}  
			Removing the dedicated text cross-attention block from each decoder layer (while keeping encoder-level fusion and language-guided query selection) produces a further but smaller drop. The ablation reports a decrease of roughly \(0.5\)–\(1.0\) AP on COCO and about \(1.5\)–\(2.0\) AP on LVIS (Table~7, model~\#2 vs.\ \#3 in~\cite{liu2023_groundingdino}). The decoder still localizes objects reasonably well, but classification degrades, especially when multiple similar objects or fine-grained attributes are present (e.g., colors, clothing attributes). Lesson: iterative query--text interaction in the decoder refines both localization and semantics beyond what encoder fusion and text-guided seeding alone can provide.
			
			\item \textbf{Sub-sentence text prompts.}  
			Changing from the sub-sentence representation to a flat, word-level representation (joint attention across all tokens without phrase masking) leads to a small but consistent drop, on the order of \(0.5\) AP on LVIS zero-shot evaluation (Table~7, model~\#3 vs.\ \#4 in~\cite{liu2023_groundingdino}). Grouping words into short, coherent phrases (and masking attention across unrelated phrases) primarily reduces interference between categories that happen to co-occur in the same prompt. Lesson: how the text is structured and masked matters; enforcing phrase-level locality makes cross-attention more stable and less noisy.
		\end{itemize}
		
		Taken together, the ablations support a clear picture: Grounding DINO’s gains do not come from a single trick but from a stack of language-aware design choices. The encoder’s Feature Enhancer establishes an aligned vision--language space and accounts for the largest share of the zero-shot AP improvements; language-guided query selection then ensures that decoding starts at semantically meaningful locations rather than generic objectness peaks; and text cross-attention in the decoder lets queries repeatedly refine their interpretation of both the image and the prompt. Sub-sentence prompts provide an additional, low-cost layer of stability by structuring the text input in a way that matches how detection categories are typically used in practice~\cite{liu2023_groundingdino}.
		
	\end{enrichment}
	
	\begin{enrichment}[Grounding DINO 1.5][subsection]
		\label{enr:chapter14_groundingdino_onepointfive}
		
		Grounding DINO 1.5~\cite{ren2024_groundingdino15} advances the original model along two largely independent axes while preserving the same dual-encoder / cross-modality decoder and set-prediction formulation:
		\begin{enumerate}
			\item \textbf{A stronger contrastive training recipe}, in which decoder queries are contrasted against text tokens from \emph{all} images in the mini-batch, not just their own image’s prompt.
			\item \textbf{Scaling and efficiency variants}, instantiated as a high-capacity \emph{Pro} model (ViT-L backbone, Grounding-20M data) and an \emph{Edge} model with an efficient feature enhancer and an EfficientViT backbone for real-time inference.
		\end{enumerate}
		Architecturally, the detection head, Hungarian matching, and open-vocabulary scoring remain unchanged; what changes is how contrastive supervision is constructed across the batch and how the encoder’s fusion cost is traded off against throughput in the Edge variant.
		
		\paragraph{Batch-level contrastive supervision and cross-image negatives}
		
		Original Grounding DINO applies its main contrastive loss \emph{image-wise}: for an image \(I_b\) with prompt \(T_b\), only queries from \(I_b\) and tokens from \(T_b\) participate in Contrastive Loss~B. Tokens that belong to prompts of other images in the mini-batch are never seen as explicit negatives for \(I_b\).
		
		Grounding DINO 1.5 instead treats the mini-batch as a single pool of region queries and text tokens. Conceptually, one can think of forming a \emph{batch-level joint prompt}
		\[
		T_{\text{batch}} = T_1 \;.\; T_2 \;.\; \dots \;.\; T_B
		\]
		whose tokens are collected into a shared set
		\[
		X_{T,\text{batch}} = \bigl\{ t_j \bigr\}_{j=1}^{N_T^{\text{batch}}}.
		\]
		In practice, the implementation can encode each image’s prompt separately and then pool the resulting tokens; the key change is the \emph{loss}: decoder queries from \emph{all} images are contrasted against \emph{all} text tokens produced in the batch.
		
		Concretely, after the cross-modality decoder (Phase~C), each image \(I_b\) yields a set of queries \(\{q_{b,k}\}_k\), each matched (via Hungarian assignment) to a ground-truth box with an associated phrase segment or to a ``no object'' label, exactly as in Grounding DINO~\cite{liu2023_groundingdino}. For a positive query \(q_{b,k}\) matched to a phrase segment \(T^{(b,k)} \subset T_b\), Contrastive Loss~B in Grounding DINO 1.5 is constructed so that:
		\begin{itemize}
			\item \textbf{Positive tokens} are those in the matched phrase segment \(T^{(b,k)}\).
			\item \textbf{Negative tokens} include not only all other tokens in \(T_b\), but also tokens from prompts \(T_{b'}\) of \emph{other} images \(I_{b'}\) in the same mini-batch.
		\end{itemize}
		From the loss’s point of view, a query on image \(I_1\) that should align with ``dog'' must not only give low scores to unrelated words like ``car'' inside \(T_1\), but also explicitly reject tokens such as ``cat'', ``bus'', or ``red umbrella'' that correctly describe objects in \(I_2,\dots,I_B\) but are \emph{absent} from \(I_1\). This turns every batch into a richer source of \emph{hard negatives} than the original image-wise training, while leaving the model architecture unchanged.
		
		\begin{figure}[H]
			\centering
			\includegraphics[width=0.85\textwidth]{Figures/Chapter_14/GroundingDINO15_framework.jpg}
			\caption{\textbf{Grounding DINO 1.5 framework.} (a) The dual-encoder / cross-modality decoder architecture from Grounding DINO~\cite{liu2023_groundingdino} is retained. (b) During training, region queries from all images in a mini-batch participate in a batch-level contrastive loss against all text tokens in the batch, so that phrases that truly describe \emph{other} images act as hard negatives. Figure adapted from~\cite{ren2024_groundingdino15}.}
			\label{fig:chapter14_groundingdino15_framework}
		\end{figure}
		
		\noindent
		Intuitively, this batch-level contrastive supervision does two things:
		\begin{itemize}
			\item It increases the effective number and diversity of negatives seen per query at each optimization step, beyond what a single image’s prompt can provide.
			\item It explicitly teaches the model to say ``no'' to plausible phrases that are valid elsewhere in the batch but not in the current image, which empirically reduces open-vocabulary hallucinations and improves rare-category recall on LVIS~\cite{ren2024_groundingdino15}.
		\end{itemize}
		The paper reports consistent gains of roughly \(+1\)–\(2\) AP in zero-shot COCO and on LVIS rare categories when switching from the original image-wise loss to the batch-level variant, under otherwise comparable settings~\cite{ren2024_groundingdino15}.
		
		\paragraph{Scaling axis: Grounding DINO 1.5 Pro}
		
		On top of the new training recipe, Grounding DINO 1.5 Pro scales the model capacity and data:
		\begin{itemize}
			\item \textbf{Backbone.} The vision backbone is upgraded to ViT-L/14 at higher resolution (e.g., \(336\times 336\)) while keeping the same type of dual-encoder / cross-modality decoder design~\cite{ren2024_groundingdino15}.
			\item \textbf{Data.} A new Grounding-20M dataset with over 20M grounding images is introduced, substantially enlarging the grounding supervision pool compared to the original Grounding DINO training recipe~\cite{ren2024_groundingdino15,liu2023_groundingdino}.
			\item \textbf{Performance.} With batch-level contrastive training and the larger backbone and data, the Pro model reaches around \(54.3\) AP on COCO zero-shot detection and roughly \(55.7\) AP on LVIS-minival zero-shot, a sizeable improvement over the Swin-L version of Grounding DINO and over DetCLIP-style baselines on LVIS~\cite{ren2024_groundingdino15,yao2023_detclipv2}.
		\end{itemize}
		Crucially, these gains do \emph{not} come from architectural changes in the detector head: the decoder, query formulation, and set-prediction loss remain as in Grounding DINO. The improvements are attributed to (i) the stronger batch-level contrastive training, (ii) the larger ViT-L backbone, and (iii) the much broader grounding corpus.
		
		\newpage
		
		\paragraph{Efficiency axis: Grounding DINO 1.5 Edge and the efficient feature enhancer}
		
		While the Pro models target maximal zero-shot and fine-tuned performance, the Edge models target deployment on resource-limited hardware. The main architectural novelty here is an \emph{efficient feature enhancer} that reduces the cost of encoder fusion:
		\begin{itemize}
			\item \textbf{Single-scale cross-modality fusion.} Instead of running multi-scale deformable self-attention over all feature pyramid levels (e.g., \(P_3\)–\(P_6\)) interleaved with text cross-attention, the Edge enhancer restricts cross-modality fusion to a single high-level feature map (typically the stride-32 \(P_5\) level). Self-attention on this map uses standard multi-head self-attention, which is easier to optimize and deploy than custom deformable kernels.
			\item \textbf{External cross-scale injection.} Information from lower-level maps \(P_3\) and \(P_4\) is injected into \(P_5\) \emph{outside} the main cross-modality loop, via lightweight cross-scale fusion (e.g., upsampling and \(1\times 1\) convolutions or simple attention). This preserves multi-scale context without repeatedly applying heavy, multi-level deformable attention.
			\item \textbf{Efficient backbone.} The image backbone is swapped to EfficientViT-L1, which is specifically designed for fast multi-scale feature extraction on edge devices, while the BERT text encoder and decoder heads follow the original Grounding DINO design~\cite{ren2024_groundingdino15}.
		\end{itemize}
		
		\noindent
		Importantly, the \emph{detection formulation} remains identical: Edge models still output a set of boxes and phrase scores per image, trained with Hungarian matching, box regression losses, and region-to-token contrastive classification as before. The efficient feature enhancer simply computes the encoder features more cheaply, making it possible to reach, after TensorRT optimization, around \(75.2\) FPS with roughly \(36.2\) AP on LVIS-minival zero-shot on edge-class GPUs~\cite{ren2024_groundingdino15}.
		
		\begin{figure}[H]
			\centering
			\includegraphics[width=0.85\textwidth]{Figures/Chapter_14/GroundingDINO_improved_feature_enhancer.jpg}
			\caption{\textbf{Original vs.\ efficient feature enhancer.} (a) Grounding DINO uses multi-scale deformable self-attention inside the feature enhancer, repeatedly fusing all pyramid levels with text. (b) Grounding DINO 1.5 Edge confines cross-modality fusion to the high-level \(P_5\) map with vanilla self-attention and uses a separate cross-scale fusion module to inject \(P_3/P_4\) information, preserving multi-scale context at much lower cost. Figure adapted from~\cite{ren2024_groundingdino15}.}
			\label{fig:chapter14_groundingdino15_enhancer}
		\end{figure}
		
		\noindent
		In summary, Grounding DINO 1.5 can be viewed as:
		\begin{itemize}
			\item \textbf{A training upgrade} (batch-level contrastive supervision with richer cross-image negatives) that both Pro and Edge variants share.
			\item \textbf{A scaling track (Pro)} that combines this training with a ViT-L backbone and a 20M-image grounding corpus for state-of-the-art open-vocabulary performance.
			\item \textbf{An efficiency track (Edge)} that re-engineers the feature enhancer and backbone for real-time open-set detection on edge devices, without changing the detector head or output format.
		\end{itemize}
	\end{enrichment}
	
	\begin{enrichment}[Limitations and Outlook][subsection]
		\label{enr:chapter14_groundingdino_limitations}
	
	Grounding DINO and Grounding DINO 1.5 illustrate how to integrate a strong DETR-style detector with grounded pre-training for open-set detection, but several limitations remain:
	
	\begin{itemize}
		\item \textbf{Prompt-driven hallucination}. Like other open-vocabulary detectors and vision--language models, Grounding DINO can still hallucinate objects that are strongly suggested by the prompt but absent in the image (e.g., predicting a ``unicorn'' box when asked, given a vaguely horse-like shape). Grounding DINO 1.5’s batch-level contrastive training mitigates this by forcing queries to explicitly reject phrases that are correct for \emph{other} images in the batch but wrong for the current one~\cite{ren2024_groundingdino15}, yet hallucination remains an important open challenge.
		\item \textbf{Rare categories and long-tail distributions}. On LVIS, Grounding DINO shows significantly lower performance on rare categories compared to frequent ones (e.g., \(18.1\) vs.\ \(32.7\) AP in a zero-shot Swin-T model)~\cite{liu2023_groundingdino}. This reflects both the DETR family’s challenges with rare classes and the limited coverage of rare concepts in available grounding data.
		\item \textbf{Box-only outputs}. Grounding DINO predicts bounding boxes but not masks. In segmentation pipelines, it must be coupled with models such as Grounded SAM and SAM~2 (in the following chapter on segmentation), which take its boxes as prompts. This decoupling can propagate localization errors to masks.
		\item \textbf{Computational cost}. Although more efficient than some alternatives (e.g., GLIPv2 and Florence)~\cite{zhang2022_glipv2,yuan2021_florence}, Grounding DINO still requires substantial pretraining compute and multi-dataset curation. Grounding DINO 1.5 improves training efficiency via batch-level prompting and an efficient feature enhancer~\cite{ren2024_groundingdino15}, but end-to-end open-set detection remains more expensive than closed-set detectors.
		\item \textbf{Semantic granularity}. Even with sub-sentence prompts, distinguishing fine-grained attributes (e.g., ``person wearing a red hat'' vs.\ ``person wearing a blue hat'') can be challenging without high-quality attribute-level grounding data.
	\end{itemize}
	
	\noindent
	Despite these limitations, Grounding DINO establishes a compelling template for open-set detectors:
	
	\begin{itemize}
		\item Combine a strong DETR-style detector (here, DINO-DETR) with grounded language pre-training.
		\item Use deep cross-modal fusion in the encoder, text-guided query selection, and cross-modality decoding.
		\item Scale training with batch-level contrastive objectives, as in Grounding DINO 1.5.
	\end{itemize}
	
	\noindent
	Subsequent enrichments in this chapter (OWL-ViT and OWLv2) will show complementary approaches that rely more heavily on CLIP-style vision–language encoders and less on DINO-DETR-style detection heads, providing a broader view of the open-set detection design space.

\end{enrichment}	
\end{enrichment}

\newpage

\begin{enrichment}[OWL-ViT: Open-Vocabulary Detection with ViTs][section]
	\label{enr:chapter14_owlvit}
	
	\begin{enrichment}[Motivation and context][subsection]
		
		OWL-ViT (``Open-World Localization Vision Transformer'')~\cite{minderer2022_owlvit} shows that a \emph{pure} image--text contrastive model, pre-trained only on image-level captions and without any box supervision, can serve as a strong backbone for an open-vocabulary detector once lightweight detection heads are attached and fine-tuned on box-level annotations.
		This stands in contrast to DINO-DETR~\cite{zhang2022_dino} and Grounding DINO~\cite{liu2023_groundingdino}:
		\begin{itemize}
			\item DINO-DETR is a closed-set detector trained with a deformable encoder--decoder Transformer and Hungarian set prediction loss, using fixed class embeddings and no language information.
			\item Grounding DINO injects text tokens into both the encoder and decoder, performing \emph{deep early fusion} between vision and language and learning the detector \emph{jointly} on caption and grounding corpora.
			\item OWL-ViT, by contrast, starts from a contrastively pre-trained vision--language model (LiT / CLIP) and keeps its image and text encoders largely \emph{decoupled} during detection fine-tuning: images go through a ViT, queries go through a text Transformer, and they only meet at the very last layer via dot products.
		\end{itemize}
		This late-fusion design has a practical advantage over Grounding DINO: image embeddings can be precomputed and indexed offline, while text prompts can be embedded on the fly.
		In large-scale retrieval or detection-as-search scenarios, this enables querying new categories without re-running the vision backbone for the entire corpus, which is not possible with Grounding DINO’s tightly coupled encoder--decoder design.
		
	\end{enrichment}
	
	\begin{enrichment}[Method][subsection]
		\paragraph{Overview of the approach}
		\begin{figure}[H]
			\centering
			\includegraphics[width=0.85\textwidth]{Figures/Chapter_14/OwlViT_approach.jpg}
			\caption{\textbf{OWL-ViT approach}. Image-level contrastive pretraining (left) followed by transfer to open-vocabulary detection (right), where per-patch tokens are fed to classification and box heads and scored against text or image queries. Figure reproduced from Minderer et al.~\cite{minderer2022_owlvit}.}
			\label{fig:chapter14_owlvit_approach}
		\end{figure}
		
		\newpage
		
		Figure~\ref{fig:chapter14_owlvit_approach} summarizes OWL-ViT’s two-stage recipe:
		\begin{enumerate}
			\item \textbf{Stage 1: Image-level contrastive pretraining (offline).}
			Before any detection data or bounding boxes are used, OWL-ViT starts from a generic dual-encoder vision--language model trained on large-scale image--caption pairs with a CLIP/LiT-style contrastive objective~\cite{radford2021_clip,zhai2022_lit,minderer2022_owlvit}.
			A Vision Transformer~\cite{vit2020_transformers} processes each image $x$ into a sequence of visual tokens (patch embeddings, optionally preceded by a \texttt{[CLS]} token).
			These token representations are then collapsed into a \emph{single} global image embedding $z^v \in \mathbb{R}^D$, either by reading out the \texttt{[CLS]} token (CLIP-style) or by Multi-head Attention Pooling (MAP, LiT-style)~\cite{zhai2022_lit}, where a few learnable pooling queries attend over all spatial tokens.
			The text encoder $f_t$ works in an analogous way: it receives the \emph{entire caption} $c$ as a token sequence (e.g., ``a bird sitting on a tree''), produces a sequence of hidden states, and then uses a designated final token (the end-of-sequence, EOS, state) as the global caption embedding $z^t \in \mathbb{R}^D$.
			Internally, every text token is represented in the same $D$-dimensional space, but only this EOS-like summary vector participates in the contrastive loss.
			Both $z^v$ and $z^t$ are $\ell_2$-normalized, and a symmetric InfoNCE loss pulls $z^v$ toward its paired $z^t$ and pushes it away from other captions in the batch, and symmetrically for $z^t$.
			This stage therefore learns a shared $D$-dimensional embedding space for \emph{global} images and \emph{global} captions only: there is no region-level supervision, no phrase-level supervision, and no bounding box annotations at all.
			Pretraining is run to convergence on billions of image--text pairs, yielding generic encoders $f_v$ and $f_t$ that map images and text sequences to vectors in the same space; these encoders are later reused (or replaced by public CLIP checkpoints) when OWL-ViT is trained as a detector.
			
			\item \textbf{Stage 2: Transfer to open-vocabulary detection (task-specific fine-tuning).}
			After Stage~1 has converged (or when starting from a pre-existing CLIP/LiT checkpoint), the global pooling used for the contrastive head (MAP or \texttt{[CLS]}-based readout) is discarded, and the pretrained ViT trunk is repurposed as a dense feature extractor.
			The sequence of visual tokens $H^v = \{h^v_i\}_{i=1}^N$ is reshaped into an $H \times W$ grid, and two lightweight heads are attached directly to each token so that every token acts as a candidate object prediction.
			The \emph{box regression head} is a small MLP that predicts offsets and log-scales relative to a fixed prior box centered at the token’s grid location; by adding a bias so that the default box is centered on the corresponding image patch, OWL-ViT learns only local deformations around this prior, introducing a strong ``location bias'' that stabilizes localization and speeds up convergence~\cite{minderer2022_owlvit}.
			For \emph{classification}, each token representation $h^v_i$ is linearly projected into the shared $D$-dimensional image--text embedding space to yield $z_i \in \mathbb{R}^D$.
			Query strings $q_k$ (category names, short phrases, or captions) are passed through the \emph{same} text encoder $f_t$ as in Stage~1; for each query, the final EOS state is taken, projected (if needed), and $\ell_2$-normalized to obtain a query embedding $e_k \in \mathbb{R}^D$.
			Thus both $z_i$ and $e_k$ live in exactly the same $D$-dimensional space, and classification reduces to a temperature-scaled cosine similarity $s_{ik} \propto z_i^\top e_k$ between token $i$ and query $k$.
			For each training image, the per-image query set consists of all categories annotated as present or explicitly absent in that image, plus a random sample of additional category names from the global federated vocabulary (Objects365, Visual Genome, and related datasets), so that each image sees on the order of fifty negative categories~\cite{minderer2022_owlvit}.
			Detection training then fine-tunes the encoders and heads jointly on detection datasets using a DETR-style bipartite matching loss~\cite{carion2020_detr}: Hungarian matching assigns each ground-truth box to at most one token prediction, $\ell_1$ and generalized IoU losses supervise box regression for matched pairs, and a sigmoid focal loss over the per-image query set handles the large, federated, partially annotated label spaces.
			
			\newpage
			
			In practice, the new detection heads use relatively large learning rates, while the pretrained image and text encoders are updated with substantially smaller learning rates, so Stage~2 gently adapts the global image--text space from Stage~1 while endowing individual ViT tokens with localized, open-vocabulary detection capability.
		\end{enumerate}
		
		\paragraph{Pretraining: global contrastive alignment (CLIP / LiT style)}
		Let $f_v$ denote the ViT image encoder and $f_t$ the text encoder.
		Given an image $x$, the ViT processes it into a sequence of patch tokens
		\[
		H^v = \{h^v_1,\dots,h^v_N\} \in \mathbb{R}^{N \times D_v},
		\]
		where $N$ is the number of patches and $D_v$ the hidden dimension.
		To apply an image-level contrastive loss, these tokens must be aggregated into a \emph{single} global image representation $z^v$.
		OWL-ViT follows the contrastive ``dual encoder'' setups of CLIP and LiT, and therefore supports two aggregation strategies depending on the underlying pretraining recipe~\cite{radford2021_clip,zhai2022_lit}:
		
		\begin{itemize}
			\item \textbf{\texttt{[CLS]} token pooling (CLIP-style).}
			For CLIP-based checkpoints, a learnable \texttt{[CLS]} token is prepended to the patch sequence.
			After the final Transformer block, its hidden state is taken (after layer normalization and a linear projection) as the global image embedding $z^v$.
			
			\item \textbf{Multi-head Attention Pooling (MAP, LiT-style).}
			For LiT-style pretraining~\cite{zhai2022_lit}, OWL-ViT instead uses Multi-head Attention Pooling (MAP)~\cite{zhai2022_lit} to aggregate patch tokens.
			A small set of learnable pooling queries $Q_{\text{pool}} \in \mathbb{R}^{M \times D_v}$ attends over the patch tokens via multi-head attention:
			\[
			O_{\text{pool}} = \mathrm{MHA}\bigl(Q_{\text{pool}},\, K = H^v,\, V = H^v\bigr) \in \mathbb{R}^{M \times D_v}.
			\]
			The $M$ pooled outputs are then averaged (or linearly combined) to form $z^v \in \mathbb{R}^{D_v}$.
			Intuitively, MAP allows the model to ``look back'' at all spatial locations with several learnable queries, and to combine them into a global summary that can, for example, focus more strongly on salient foreground objects than on background.
		\end{itemize}
		
		In both cases, the text encoder maps the caption $c$ to a single caption embedding $z^t \in \mathbb{R}^{D}$, typically taken from the final end-of-sequence (EOS) token.
		Both $z^v$ and $z^t$ are projected into a shared space of dimension $D$ and $\ell_2$-normalized.
		For a batch of $B$ image–caption pairs $\{(x_b, c_b)\}_{b=1}^B$, OWL-ViT uses the symmetric CLIP/LiT InfoNCE loss~\cite{radford2021_clip,zhai2022_lit}:
		\[
		\mathcal{L}_{\text{pretrain}}
		= \frac{1}{2}\,\mathcal{L}_{\text{i}\rightarrow\text{t}} + \frac{1}{2}\,\mathcal{L}_{\text{t}\rightarrow\text{i}},
		\]
		where, writing $s_{uv} = \frac{1}{\tau} (z^v_u)^\top z^t_v$ for a learned temperature $\tau$,
		\[
		\mathcal{L}_{\text{i}\rightarrow\text{t}}
		= -\frac{1}{B}\sum_{b=1}^B
		\log
		\frac{\exp(s_{bb})}{\sum_{j=1}^B \exp(s_{bj})},
		\quad
		\mathcal{L}_{\text{t}\rightarrow\text{i}}
		= -\frac{1}{B}\sum_{b=1}^B
		\log
		\frac{\exp(s_{bb})}{\sum_{j=1}^B \exp(s_{jb})}.
		\]
		This aligns each image embedding with its own caption and repels it from all other captions (and symmetrically for captions).
		Crucially, this stage is \emph{purely global}: the model never sees bounding boxes or region–phrase pairs, and has no notion of object location yet.
		All localization ability is introduced only in the second, detection-specific stage.
		
		\paragraph{Detection head: encoder-only dense prediction with location bias}
		To convert the pretrained encoders into an open-vocabulary detector, OWL-ViT removes the global pooling (MAP or \texttt{[CLS]}-based) and retains the full grid of ViT output tokens as dense features.
		For an input image resized to a fixed resolution (e.g., \(768\times 768\)), the last ViT block produces a sequence
		\[
		H^v = \{h^v_1,\dots,h^v_N\},\quad h^v_i \in \mathbb{R}^{D_v},
		\]
		which can be reshaped into a 2D grid (e.g., \(24\times 24\) tokens for ViT-B/32).
		OWL-ViT then attaches two lightweight heads to each token, turning every token into a candidate prediction:
		
		\begin{itemize}
			\item \textbf{Box regression head with location bias.}
			A small MLP $\mathrm{MLP}_{\text{box}}$ takes $h^v_i$ and predicts four real-valued offsets
			\[
			(\Delta c_x,\Delta c_y,\Delta \log w,\Delta \log h)_i
			= \mathrm{MLP}_{\text{box}}(h^v_i).
			\]
			Each token $i$ is associated with a fixed prior center $(c_{x,i}, c_{y,i})$ in image coordinates (obtained by arranging the tokens on a regular grid) and a prior scale $s_i$ proportional to the patch size / feature stride.
			The final box prediction $\hat{b}_i = (\hat{c}_{x,i},\hat{c}_{y,i},\hat{w}_i,\hat{h}_i)$ is obtained as
			\[
			\hat{c}_{x,i} = c_{x,i} + s_i \,\Delta c_x,\quad
			\hat{c}_{y,i} = c_{y,i} + s_i \,\Delta c_y,\quad
			\hat{w}_i = s_i \exp(\Delta \log w),\quad
			\hat{h}_i = s_i \exp(\Delta \log h).
			\]
			In other words, before learning, the box for token $i$ is biased to be centered on its own grid patch with size proportional to that patch; the network only needs to learn \emph{local deformations} around this default anchor, similar in spirit to Region Proposal Networks~\cite{ren2015_fasterrcnn}.
			This location bias significantly accelerates convergence and improves final AP compared to predicting absolute coordinates from scratch~\cite{minderer2022_owlvit}.
			
			\item \textbf{Classification via text-derived weights and sampled negatives.}
			Instead of learning a fixed classifier over a closed label set, OWL-ViT reuses the shared image–text embedding space learned during contrastive pretraining.
			A linear projection $W_{\text{cls}}$ maps each visual token to a \emph{per-object} embedding
			\[
			z_i = W_{\text{cls}} h^v_i \in \mathbb{R}^{D},
			\]
			followed by $\ell_2$-normalization.
			A text query $q_k$ (category name, phrase, or short description) is encoded by the same text encoder,
			\[
			e_k = \frac{f_t(q_k)}{\|f_t(q_k)\|_2} \in \mathbb{R}^D,
			\]
			and acts as the classifier weight vector for ``class'' $k$.
			The logit for token $i$ and query $k$ is a temperature-scaled cosine similarity
			\[
			s_{ik} = \frac{1}{\tau}\, z_i^\top e_k,
			\]
			where the temperature $\tau$ is inherited from pretraining and optionally fine-tuned.
			There is no explicit background neuron.
			Instead, OWL-ViT uses independent sigmoid focal losses over a \emph{per-image} query set, and a token is treated as background at inference time if its maximum score over all queries falls below a confidence threshold.
			
			\newpage
			
			For each training image, the per-image query set is constructed from the federated vocabulary as follows~\cite{minderer2022_owlvit}:
			\begin{itemize}
				\item All categories annotated as \emph{present} in the image (positives).
				\item All categories annotated as \emph{absent} (known negatives), where such annotations are available.
				\item Additional ``pseudo-negative'' categories randomly sampled from the global federated label space (Objects365, Visual Genome, and possibly LVIS/COCO) until each image sees at least about 50 negative categories.
			\end{itemize}
			These sampled negatives are crucial: by repeatedly presenting category names that do \emph{not} match the image, the detector learns to drive their logits down, which is what enables robust open-vocabulary rejection rather than over-triggering on rare classes.
			
		\end{itemize}
		
		The resulting architecture is an encoder-only, dense detector: there is no Transformer decoder and no learned object queries.
		However, OWL-ViT still follows DETR’s set-prediction paradigm by using a bipartite matching loss between the $N$ token-level predictions and the (typically much smaller) set of ground-truth objects~\cite{carion2020_detr}.
		
		\paragraph{Training objective and federated label spaces}
		Detection training primarily uses Objects365 and Visual Genome with their native label vocabularies and then evaluates on LVIS and COCO~\cite{minderer2022_owlvit}.
		The overall training loop mirrors DETR’s bipartite matching formulation~\cite{carion2020_detr}, but replaces softmax with sigmoid focal classification and treats the label space in a federated manner.
		
		For a given training image, let $\{b_j, \mathcal{C}_j\}_{j=1}^M$ denote the $M$ ground-truth objects, where $b_j \in \mathbb{R}^4$ are bounding boxes and $\mathcal{C}_j \subseteq \mathcal{V}_d$ is the (possibly multi-label) set of categories annotated for object $j$ in the source dataset vocabulary $\mathcal{V}_d$.
		Let $\mathcal{Q} \subseteq \mathcal{V}$ be the \emph{per-image query set} used for this image; it is constructed as in~\cite{minderer2022_owlvit} by combining:
		\begin{itemize}
			\item All categories annotated as present ($+$) or explicitly absent ($-$) in the image (from $\mathcal{V}_d$).  
			\item Additional ``pseudo-negative'' categories sampled from the global federated vocabulary $\mathcal{V}$ until there are at least about 50 negatives per image.  
		\end{itemize}
		For each query $k \in \mathcal{Q}$, the text encoder $f_t$ produces an embedding $e_k \in \mathbb{R}^D$, and the detector produces logits $s_{ik}$ for each token $i \in \{1,\dots,N\}$ as in the previous paragraph.
		
		\medskip
		\noindent\textbf{Bipartite matching.}
		Following DETR, OWL-ViT computes a bipartite matching between the $M$ ground-truth objects and the $N$ token-level predictions using the Hungarian algorithm~\cite{carion2020_detr}.
		Let $\hat{b}_i$ be the predicted box for token $i$ and let $\sigma$ be the optimal matching
		\[
		\sigma : \{1,\dots,M\} \rightarrow \{1,\dots,N\} \cup \{\varnothing\},
		\]
		that minimizes a matching cost
		\[
		\mathcal{L}_{\text{match}}(j,i)
		= \lambda_{\text{cls}}\, \mathcal{L}_\text{cls}^{\text{match}}(j,i)
		+ \lambda_{\ell_1}\, \|\hat{b}_i - b_j\|_1
		+ \lambda_{\text{giou}}\, \mathcal{L}_{\text{giou}}(\hat{b}_i, b_j),
		\]
		where $\mathcal{L}_{\text{giou}}$ is the generalized IoU loss~\cite{rezatofighi2019_giou}.
		In practice, the weights $(\lambda_{\text{cls}},\lambda_{\ell_1},\lambda_{\text{giou}})$ are chosen following DETR-style practice; see Minderer et al.~\cite{minderer2022_owlvit} for the exact values.
		
		\newpage
		
		\noindent\textbf{Focal classification loss.}
		For classification, OWL-ViT uses sigmoid focal cross-entropy~\cite{lin2018_focalloss} instead of softmax, to support multi-label annotations and per-image query sets.
		For a logit $s_{ik}$ and binary target $y_{ik} \in \{0,1\}$, define
		\[
		p_{ik} = \sigma(s_{ik}),\qquad
		\mathrm{FL}(p_{ik}, y_{ik})
		= -\,\alpha\, y_{ik}\,(1-p_{ik})^\gamma \log p_{ik}
		- (1-\alpha)\,(1-y_{ik})\,p_{ik}^\gamma \log(1-p_{ik}),
		\]
		with $(\alpha,\gamma)$ following the standard RetinaNet-style settings used in the original implementation~\cite{minderer2022_owlvit}.
		Given the matching $\sigma$, the classification targets are defined as follows.
		For a matched pair $(j, i = \sigma(j))$ and query $k \in \mathcal{Q}$,
		\[
		y_{ik} =
		\begin{cases}
			1, & \text{if } k \in \mathcal{C}_j, \\
			0, & \text{if } k \in \mathcal{Q} \setminus \mathcal{C}_j,
		\end{cases}
		\]
		so the token is trained to be positive for all labels in $\mathcal{C}_j$ and negative for the remaining queries.
		For unmatched tokens $i$ (i.e., $i \notin \mathrm{Im}(\sigma)$), all targets are zero, $y_{ik} = 0$ for all $k \in \mathcal{Q}$, so they act as ``no-object'' background.
		The classification loss for one image is then
		\[
		\mathcal{L}_{\text{cls}}
		= \frac{1}{|\mathcal{Q}|\,N}
		\sum_{i=1}^{N}
		\sum_{k \in \mathcal{Q}}
		\mathrm{FL}\bigl(p_{ik}, y_{ik}\bigr),
		\]
		normalized over all tokens and queries for that image.
		
		\medskip
		\noindent\textbf{Box regression loss.}
		Only matched tokens receive box regression supervision.
		Writing $j(i)$ for the unique ground-truth object assigned to token $i$ by the matching (when it exists), the box loss for one image is
		\[
		\mathcal{L}_{\text{box}}
		= \frac{1}{M}
		\sum_{j=1}^{M}
		\Bigl(
		\|\hat{b}_{\sigma(j)} - b_j\|_1
		+ \mathcal{L}_{\text{giou}}(\hat{b}_{\sigma(j)}, b_j)
		\Bigr),
		\]
		with the convention that terms with $\sigma(j) = \varnothing$ are skipped.
		
		\medskip
		\noindent\textbf{Total detection loss and federated masking.}
		The total detection loss for one image is
		\[
		\mathcal{L}_{\text{det}}
		= \lambda_{\text{cls}}\, \mathcal{L}_{\text{cls}}
		+ \lambda_{\ell_1}\, \mathcal{L}_{\ell_1}
		+ \lambda_{\text{giou}}\, \mathcal{L}_{\text{giou}},
		\]
		where $\mathcal{L}_{\ell_1}$ and $\mathcal{L}_{\text{giou}}$ are the $\ell_1$ and gIoU components of $\mathcal{L}_{\text{box}}$.
		Because federated datasets such as Objects365, Visual Genome, and LVIS annotate overlapping but non-identical label vocabularies, OWL-ViT computes classification losses \emph{only} over the per-image query set $\mathcal{Q}$ for the current dataset and masks gradients for categories outside this vocabulary.
		This loss masking prevents the model from interpreting unannotated objects as negatives for labels defined only in other datasets (e.g., a ``car'' present but unannotated in Visual Genome must not be treated as negative evidence for the LVIS ``car'' class), while still benefiting from additional pseudo-negative queries sampled from the global vocabulary.
		
		\paragraph{Image- and text-conditioned detection}
		A key advantage of OWL-ViT’s symmetric design is that both image and text can act as queries without architectural changes.
		For \emph{text-conditioned detection}, class names or phrases are encoded with $f_t$ and used directly as classifier weights.
		For \emph{image-conditioned detection}, a reference image (or a crop) is passed through the same vision encoder $f_v$, and MAP produces a global embedding that is used as a query vector.
		When multiple reference images are available, their embeddings are mean-pooled to form a single query.
		
		\newpage
		
		The following figure shows qualitative one-shot image-conditioned detections: OWL-ViT correctly selects instances of the reference species even when text prompts fail for fine-grained categories.
		
		\begin{figure}[H]
			\centering
			\includegraphics[width=0.85\textwidth]{Figures/Chapter_14/OwlViT_one_shot_image_reference.jpg}
			\caption{\textbf{One-shot image-conditioned detection.} Center images serve as reference queries; OWL-ViT detects matching instances in the cluttered target images (left and right). Figure reproduced from Minderer et al.~\cite{minderer2022_owlvit}.}
			\label{fig:chapter14_owlvit_image_conditioned}
		\end{figure}
	\end{enrichment}
	
	\begin{enrichment}[Architecture variants and ablations][subsection]
		The authors evaluate three backbone families: pure ViT backbones (B/32, B/16, L/16, H/14), hybrid CNN+ViT backbones (denoted \emph{hybridCNN}, where a ResNet trunk produces a convolutional feature map that is flattened into tokens and fed to a ViT head), and pure ResNet-only models.
		The following figure summarizes two consistent trends:
		\begin{itemize}
			\item For small model sizes and tight FLOPs budgets, hybridCNN backbones (ResNet trunk + ViT head) are more compute-efficient than pure ViTs, achieving competitive AP at lower cost.
			\item As the FLOPs budget grows, pure ViTs scale better: they reach higher AP on LVIS overall and, more importantly, systematically achieve higher zero-shot AP on LVIS \emph{rare} categories than both hybridCNN and pure ResNet backbones, indicating a stronger bias toward semantic generalization (reasoning about unseen categories) rather than just localizing a fixed set of known classes.
		\end{itemize}
		
		\begin{figure}[H]
			\centering
			\includegraphics[width=0.85\textwidth]{Figures/Chapter_14/OwlViT_architecture_ablation.jpg}
			\caption{\textbf{Effect of backbone architecture on detection performance.} Comparison of pure ViT, hybridCNN (ResNet trunk + ViT head), and pure ResNet backbones.
				Pure ViTs scale better than hybrid and ResNet backbones, especially for zero-shot rare categories.
				Figure reproduced from Minderer et al.~\cite{minderer2022_owlvit}.}
			\label{fig:chapter14_owlvit_architecture_ablation}
		\end{figure}
		
		\newpage
		
		\paragraph{Scaling and transfer from image-level to object-level performance}
		A central empirical question is whether better image-level contrastive pretraining actually translates into better open-vocabulary detection.
		The following figure plots zero-shot ImageNet accuracy (after pretraining) vs.\ zero-shot LVIS AP on rare categories (after detection fine-tuning) across many pretraining configurations (backbone type, model size, image resolution, number of pretraining examples).
		
		Two patterns emerge:
		\begin{itemize}
			\item High image-level performance is \emph{necessary but not sufficient} for high object-level performance: the correlation between pretraining and LVIS rare AP is strong but imperfect (Pearson $r \approx 0.73$).
			\item The right-hand plots show that, for a fixed architecture, longer contrastive pretraining (more image--text examples) improves both ImageNet accuracy and LVIS AP, while additional detection fine-tuning provides a smaller but consistent boost.
		\end{itemize}
		These results suggest a kind of \emph{lock-in effect}: the semantic capacity of the detector is largely determined during image-level pretraining.
		Fine-tuning on detection data can teach localization, but it cannot easily recover semantic knowledge that the contrastive model never acquired.
		
		\begin{figure}[H]
			\centering
			\includegraphics[width=0.85\textwidth]{Figures/Chapter_14/OwlViT_transfer.jpg}
			\caption{\textbf{Transfer from image-level to object-level performance.} Left: each dot corresponds to a different image–text pretraining configuration and its best LVIS rare AP after detection fine-tuning; high image-level accuracy is necessary but not sufficient for high object-level AP. Right: scaling the number of pretraining examples and model size improves both ImageNet accuracy and LVIS detection AP. Figure reproduced from Minderer et al.~\cite{minderer2022_owlvit}.}
			\label{fig:chapter14_owlvit_transfer}
		\end{figure}
		
		\paragraph{Quantitative results on LVIS: open-vocabulary and zero-shot detection}
		The below table summarizes representative LVIS v1.0 validation results from Minderer et al.~\cite{minderer2022_owlvit}.
		Following the paper, \(\text{AP}_{\text{LVIS}}\) is AP over all categories and \(\text{AP}_{\text{LVIS}}^{\text{rare}}\) measures rare categories; for the zero-shot setting, labels for rare categories are removed from all detection training data.
		
		\begin{table}[H]
			\centering
			\small
			\caption{\textbf{Open-vocabulary LVIS results}. All numbers are from Minderer et al.~\cite{minderer2022_owlvit}. ``Base'' rows use LVIS base annotations during training; lower block uses unrestricted open-vocabulary training on Objects365 and Visual Genome.}
			\label{tab:chapter14_owlvit_lvis}
			\begin{tabular}{l l l c c}
				\toprule
				Method & Backbone & Training data & $\text{AP}_{\text{LVIS}}$ & $\text{AP}_{\text{LVIS}}^{\text{rare}}$ \\
				\midrule
				ViLD-ens~\cite{gu2022_vild} & ResNet-50 & LVIS base & 25.5 & 16.6 \\
				ViLD-ens~\cite{gu2022_vild} & EffNet-B7 & LVIS base & 29.3 & 26.3 \\
				RegionCLIP~\cite{zhong2021_regionclip} & R50x4-C4 & LVIS base & 32.3 & 22.0 \\
				OWL-ViT~\cite{minderer2022_owlvit} & ViT-H/14 (LiT) & LVIS base & \textbf{35.3} & 23.3 \\
				OWL-ViT~\cite{minderer2022_owlvit} & ViT-L/14 (CLIP) & LVIS base & 34.7 & \textbf{25.6} \\
				\midrule
				GLIP~\cite{li2022_glip} & Swin-L & O365 + GoldG + captions & 26.9 & 17.1 \\
				OWL-ViT~\cite{minderer2022_owlvit} & ViT-B/16 (LiT) & O365 + VG & 26.7 & 23.6 \\
				OWL-ViT~\cite{minderer2022_owlvit} & ViT-L/16 (LiT) & O365 + VG & 30.9 & 28.8 \\
				OWL-ViT~\cite{minderer2022_owlvit} & ViT-H/14 (LiT) & O365 + VG & \textbf{33.6} & \textbf{30.6} \\
				\bottomrule
			\end{tabular}
		\end{table}
		
		OWL-ViT thus improves over ViLD and GLIP on both overall AP and zero-shot rare categories, especially when scaled to large ViT backbones and trained on Objects365+VG.
		For example, the ViT-H/14 LiT model achieves \(33.6\) \(\text{AP}_{\text{LVIS}}\) and \(30.6\) \(\text{AP}_{\text{LVIS}}^{\text{rare}}\), substantially higher than GLIP’s \(26.9 / 17.1\).
		
		\paragraph{One-shot and few-shot image-conditioned detection on COCO}
		For COCO image-conditioned detection, OWL-ViT compares against SiamMask~\cite{hu2022_siammask}, CoAE (One-Shot Object Detection with Co-Attention and Co-Excitation)~\cite{hsieh2019_oneshot}, and AIT~\cite{chen2021_ait}.
		The following table reports AP$_{50}$ for seen and unseen category splits.
		
		\begin{table}[H]
			\centering
			\small
			\caption{\textbf{One- and few-shot image-conditioned detection on COCO} (AP$_{50}$). Results from Minderer et al.~\cite{minderer2022_owlvit}. OWL-ViT uses an R50+H/32 hybrid backbone; $k$ denotes the number of reference images per category.}
			\label{tab:chapter14_owlvit_coco_image_conditioned}
			\begin{tabular}{l c c c c c}
				\toprule
				& Split 1 & Split 2 & Split 3 & Split 4 & Mean \\
				\midrule
				\multicolumn{6}{c}{\textit{Seen categories}} \\
				\midrule
				SiamMask~\cite{hu2022_siammask} & 38.9 & 37.1 & 37.8 & 36.6 & 37.6 \\
				CoAE~\cite{hsieh2019_oneshot}   & 42.2 & 40.2 & 39.9 & 41.3 & 40.9 \\
				AIT~\cite{chen2021_ait}        & 50.1 & 47.2 & 45.8 & 46.9 & 47.5 \\
				OWL-ViT ($k=1$)~\cite{minderer2022_owlvit} & 49.9 & 49.1 & 49.2 & 48.2 & 49.1 \\
				OWL-ViT ($k=10$)~\cite{minderer2022_owlvit} & \textbf{54.1} & \textbf{55.3} & \textbf{56.2} & \textbf{54.9} & \textbf{55.1} \\
				\midrule
				\multicolumn{6}{c}{\textit{Unseen categories}} \\
				\midrule
				SiamMask~\cite{hu2022_siammask} & 15.3 & 17.6 & 17.4 & 17.0 & 16.8 \\
				CoAE~\cite{hsieh2019_oneshot}   & 23.4 & 23.6 & 20.5 & 20.4 & 22.0 \\
				AIT~\cite{chen2021_ait}        & 26.0 & 26.4 & 22.3 & 22.6 & 24.3 \\
				OWL-ViT ($k=1$)~\cite{minderer2022_owlvit} & 43.6 & 41.3 & 40.2 & 41.9 & 41.8 \\
				OWL-ViT ($k=10$)~\cite{minderer2022_owlvit} & \textbf{49.3} & \textbf{51.1} & \textbf{42.4} & \textbf{44.5} & \textbf{46.8} \\
				\bottomrule
			\end{tabular}
		\end{table}
		
		On unseen categories, OWL-ViT with a single reference image nearly doubles AIT’s mean AP$_{50}$ (41.8 vs.\ 24.3), and using ten reference images further boosts performance to 46.8 AP$_{50}$.
		
		\newpage
		
		This illustrates how the symmetric vision encoder can be exploited for powerful image-conditioned detection without modifying the architecture.
		
		\paragraph{Training and data ablations}
		The paper includes a detailed ablation study on LVIS and COCO, varying training data, optimizer settings, prompts, and augmentation~\cite{minderer2022_owlvit}.
		Some key findings (differences relative to a ViT-B/32 baseline trained on O365+VG):
		\begin{itemize}
			\item \textbf{Training data matters most.}
			Using only Visual Genome captions without Objects365 grounding data reduces $\text{AP}_{\text{LVIS}}$ and $\text{AP}_{\text{LVIS}}^{\text{rare}}$ by roughly 14 points, while using only OpenImages reduces them by about 7 points.
			Jointly using O365 and VG is important for both breadth and grounding.
			\item \textbf{Differential learning rates for image vs.\ text encoders.}
			Forcing the same learning rate for both encoders significantly hurts rare categories (around \(-8\) points in $\text{AP}_{\text{LVIS}}^{\text{rare}}$).
			In practice, the vision encoder is fine-tuned with a smaller learning rate than the text encoder, similar to domain adaptation methods such as ReCLIP~\cite{hu2023_reclip}.
			\item \textbf{Prompt ensembling.}
			Using multiple textual templates (e.g., ``a photo of a \{class\}'', ``a \{class\} in the scene'') and averaging their embeddings improves rare-category AP by around 5 points compared to a single, fixed template.
			\item \textbf{Random negative categories.}
			Adding random negative labels per image yields a modest but consistent improvement in zero-shot AP, especially on rare categories, showing that hard negatives sharpen the classifier.
			\item \textbf{Mosaic augmentation and localization heuristics.}
			Mosaic-style augmentations and simple heuristics (merging overlapping instances, adding a location bias, filtering cropped boxes) each contribute one or two AP points; removing mosaics harms performance more than simply training for more epochs.
		\end{itemize}
		
		\paragraph{Comparison with Grounding DINO and OWLv2}
		From the perspective of this chapter, OWL-ViT and Grounding DINO represent two ends of the open-vocabulary detection design space.
		
		\begin{itemize}
			\item \textbf{Fusion strategy.}
			OWL-ViT uses \emph{late fusion}: image and text encoders are independent, and the only interaction is in the dot product between image features and query embeddings at the detection head.
			Grounding DINO~\cite{liu2023_groundingdino} uses \emph{early and deep fusion}, injecting text tokens into both the encoder and decoder via cross-attention.
			This makes Grounding DINO stronger at phrase grounding and region–phrase alignment, but also makes it harder to reuse as a stand-alone image or text encoder.
			\item \textbf{Backbone training.}
			OWL-ViT relies heavily on large-scale contrastive pretraining and fine-tunes the pretrained ViT and text encoder with relatively small learning rates.
			Grounding DINO jointly trains (or fine-tunes) its vision backbone and text branch on grounding corpora, often starting from ImageNet- or CLIP-style initialization.
			\item \textbf{Detection head.}
			Both models reuse the DETR-style box losses ($\ell_1$ + GIoU) together with one-to-one Hungarian matching between predictions and ground-truth objects, but Grounding DINO adds a full Transformer decoder with learned object queries, whereas OWL-ViT is encoder-only and uses dense per-patch tokens as predictions.
		\end{itemize}
		
		Subsequent work OWLv2~\cite{minderer2024_owlvitv2} scales this recipe further with larger backbones, more data, and improved training, pushing LVIS zero-shot performance close to or beyond Grounding DINO while preserving the simplicity and retrieval-friendly, decoupled design.
		
		\newpage
		
		For example, Minderer et al.\ report that OWLv2 improves LVIS rare-category AP by more than ten points over the best OWL-ViT v1 configuration, closing most of the gap to fully task-specific detectors.
	\end{enrichment}
	
	\begin{enrichment}[Limitations and outlook][subsection]
		Despite its strong performance and clean design, OWL-ViT has several limitations:
		\begin{itemize}
			\item \textbf{Dense, relatively slow inference.}
			Because every ViT patch token predicts a box and scores against all query categories, large OWL-ViT models (e.g., ViT-H/14) can be significantly slower than one-stage CNN detectors, especially when many queries are used.
			In practice, the authors report a few frames per second on high-end GPUs for large models~\cite{minderer2022_owlvit}.
			\item \textbf{Bounding boxes only.}
			OWL-ViT predicts boxes but not masks.
			For segmentation, it must be combined with downstream modules such as SAM/SAM2 or Mask DINO (Chapter~15).
			\item \textbf{Dependence on pretraining quality.}
			The lock-in effect discussed above means that poor contrastive pretraining cannot be fully compensated during detection fine-tuning.
			Choosing strong image--text pretraining (e.g., LiT~\cite{zhai2022_lit}, improved CLIP variants, or domain-adapted models such as ReCLIP~\cite{hu2023_reclip}) is crucial.
			\item \textbf{Limited relational reasoning.}
			Purely per-patch scoring against independent queries makes it harder to model relational phrases (e.g., ``person holding a red umbrella'') compared to architectures like Grounding DINO that fuse language deeply into the encoder--decoder.
		\end{itemize}
		
		Nonetheless, OWL-ViT has had substantial impact.
		Its pattern of ``frozen vision--language backbone + lightweight detection head with text-derived classifier weights'' has been adopted by many later systems, including OWLv2~\cite{minderer2024_owlvitv2}, frozen-VLM detectors, and YOLO-style open-vocabulary models.
		In later sections, OWLv2 will reappear as a stronger, scaled-up variant that pushes the same design further in both accuracy and robustness.		
	\end{enrichment}
	
\end{enrichment}

\newpage

\begin{enrichment}[OWLv2: Scaling Open-Vocabulary Detection][section]
	\label{enr:chapter14_owlv2}
	
	\begin{enrichment}[Motivation and context][subsection]
		OWL-ViT v2 (often abbreviated OWLv2)~\cite{minderer2024_owlvitv2} asks a simple question: if OWL-ViT already turns a contrastively pretrained vision--language model into an open-vocabulary detector, can detection performance be scaled further \emph{without} collecting more human box annotations.
		The answer comes from self-training on web data.
		Instead of hand-labeling new detection datasets, OWLv2 uses OWL-ViT itself as a \emph{pseudo-annotator} on the WebLI image--text corpus, generating billions of noisy boxes and category labels that are then used to train stronger ``student'' detectors.
		
		This strategy contrasts with models such as GLIP, DetCLIP, and Grounding DINO~\cite{li2022_glip,zhong2021_regionclip,liu2023_groundingdino}, which improve open-vocabulary detection primarily by designing more powerful encoder--decoder architectures and pretraining on curated grounding datasets with explicit region--phrase supervision.
		OWLv2 instead keeps the simple OWL-ViT detection head and late-fusion dual encoders, and concentrates effort on \emph{data scaling} and \emph{training efficiency}.
		The resulting OWL-ST recipe (``OWL Self-Training'') scales to roughly two billion Web images and yields a ViT-G/14 detector with about \(47.2\) AP on rare LVIS categories in the zero-shot setting~\cite{minderer2024_owlvitv2}, substantially improving over OWL-ViT v1 while preserving the retrieval-friendly, encoder-only design.
	\end{enrichment}
	
	\begin{enrichment}[OWLv2: Self-training pipeline (OWL-ST)][subsection]
		
		\paragraph{Overview of the three-stage recipe}
		OWLv2 (often referred to as OWL-ST in the paper) is best understood as a \emph{strictly sequential} self-training pipeline built around OWL-ViT~\cite{minderer2022_owlvit}.
		A strong, but relatively expensive, OWL-ViT detector first plays the role of a frozen \emph{annotator} that pseudo-labels a massive multilingual web corpus (WebLI~\cite{jia2021_scalingvision}) with boxes and phrases.
		A \emph{student} detector with the same basic architecture but larger backbones (CLIP or SigLIP ViTs) is then trained on these noisy pseudo-boxes using an efficiency-optimized training loop.
		Finally, for benchmarks such as LVIS, the self-trained student can be optionally fine-tuned on human annotations and \emph{weight-ensembled} with its pre-fine-tuning version to balance in-distribution accuracy and open-world robustness.
		Figure~\ref{fig:chapter14_owlv2_overview} sketches these stages.
		
		\begin{figure}[H]
			\centering
			\includegraphics[width=0.85\textwidth]{Figures/Chapter_14/OwlV2_overview.jpg}
			\caption{\textbf{OWLv2 overview}. A frozen OWL-ViT CLIP-L/14 ``annotator'' produces pseudo-boxes and labels for WebLI images (left); a student detector is then self-trained on these pseudo-annotations with architectural and training-efficiency tweaks (middle); finally, the student can optionally be fine-tuned or weight-ensembled on standard detection corpora such as LVIS. Figure reproduced from Minderer et al.~\cite{minderer2024_owlvitv2}.}
			\label{fig:chapter14_owlv2_overview}
		\end{figure}
		
		\begin{enumerate}
			\item \textbf{Stage~0: Annotator pretraining (OWL-ViT).}
			The annotator is a standard OWL-ViT detector~\cite{minderer2022_owlvit} built on a CLIP ViT-L/14 backbone.
			It is obtained by first contrastively pretraining CLIP on web-scale image--caption pairs and then training OWL-ViT for detection on human-annotated datasets such as Objects365, OpenImages~V4, LVIS, and Visual Genome.
			This yields a strong open-vocabulary detector that can respond to arbitrary text queries with dense boxes and scores.
			In principle, any sufficiently strong OWL-ViT variant could serve as annotator, but OWLv2 consistently uses the public CLIP-L/14 OWL-ViT checkpoint in all experiments~\cite{minderer2024_owlvitv2}.
			
			\item \textbf{Stage~1: Pseudo-annotation of WebLI with the annotator.}
			The frozen OWL-ViT annotator is run on images from WebLI~\cite{jia2021_scalingvision}, a massive multilingual web image--text collection.
			For each WebLI image $x$, OWLv2 must decide \emph{which phrases} to ask the annotator to look for.
			It constructs a \emph{query list} by merging a fixed, human-curated dictionary of standard object categories with image-specific N-grams extracted from the caption:
			\[
			\mathcal{Q}(x)
			=
			\underbrace{\mathcal{V}_{\text{curated}}}_{\text{fixed across images}}
			\;\cup\;
			\underbrace{\mathcal{V}_{\text{N-gram}}(x)}_{\text{caption-derived, image-specific}}.
			\]
			OWL-ViT is applied once per image with this \emph{merged} query list, producing a dense set of candidate boxes and phrase scores.
			All boxes whose scores exceed a moderate inclusion threshold (e.g., \(0.1\)) are retained as pseudo-annotations, provided the image has at least one prediction above a higher confidence level (e.g., \(0.3\))~\cite{minderer2024_owlvitv2}.
			These settings deliberately favor \emph{quantity} over purity: instead of keeping only the top-1 box per image, OWLv2 harvests many moderately confident boxes, trading some label noise for a huge effective training set.
			
			\item \textbf{Stage~2: Self-training the student detector (OWL-ST).}
			A \emph{student} detector with the same overall design as OWL-ViT but larger backbones (e.g., CLIP ViT-G/14 or SigLIP ViT-G/14) is then trained on the pseudo-labeled WebLI data.
			In OWLv2’s main scaling experiments, backbones are initialized from CLIP or SigLIP checkpoints rather than from the annotator’s detection weights~\cite{minderer2024_owlvitv2}, so the student effectively learns detection ``from scratch'' under the supervision of the frozen annotator.
			Architecturally, the student remains very simple: a ViT image encoder, a text encoder, and lightweight per-patch heads.
			There is no Transformer decoder and no learnable query embedding set; all predictions are anchored directly to ViT patch tokens.
			
			\medskip
			
			\noindent
			\textbf{Detection objective: dense encoder-only open-vocabulary detection}
			
			The OWLv2 detector adopts the encoder-only, one-box-per-token recipe of OWL-ViT~\cite{minderer2022_owlvit,minderer2024_owlvitv2}: detection heads are attached directly to ViT patch tokens; there is no Transformer decoder. Unlike classic one-stage detectors (e.g., RetinaNet or FCOS), supervision is still formulated as a DETR-style set-prediction problem with bipartite matching rather than heuristic IoU-threshold assignment.
			
			For a training image $x$ with pseudo-annotations $\{(b_j, y_j)\}_{j=1}^M$ (pseudo boxes $b_j$ and phrases $y_j$ from the OWL-ViT annotator), OWLv2 forms an image-specific query set
			\[
			\mathcal{Q}(x)
			=
			\mathcal{V}_{\text{curated}}
			\;\cup\;
			\mathcal{V}_{\text{image}}(x)
			\;\cup\;
			\mathcal{V}_{\text{neg}}(x),
			\]
			
			\newpage
			
			where $\mathcal{V}_{\text{curated}}$ is a fixed vocabulary of human-curated object names, $\mathcal{V}_{\text{image}}(x)$ collects the image-specific phrases that actually appear as pseudo-labels for $x$ (e.g., N-grams from its WebLI caption or curated names, depending on the pseudo-annotation run), and $\mathcal{V}_{\text{neg}}(x)$ are “pseudo-negative’’ phrases sampled from other images that are guaranteed never to be positives for $x$ but act as hard negatives on the text side~\cite{minderer2024_owlvitv2}.
			
			All queries $k \in \mathcal{Q}(x)$ are encoded into text embeddings, and the image is passed through the ViT encoder to yield patch embeddings $\{h_i\}_{i=1}^N$. On top of each token $i$ OWLv2 adds three lightweight heads:
			\begin{itemize}
				\item An \emph{objectness head} predicting a scalar score $o_i$ that estimates how likely the token corresponds to an object at all.
				\item A \emph{box head} predicting a single candidate box $\hat{b}_i$ (center coordinates and size) for that token.
				\item A \emph{classification head} producing logits $s_{ik}$ for all queries $k \in \mathcal{Q}(x)$, typically implemented as scaled dot products (equivalently, cosine similarities after $\ell_2$-normalization) between visual and text embeddings, as in OWL-ViT.
			\end{itemize}
			To reduce computation, OWLv2 computes classification and box losses only for the top-$K$ tokens by objectness during training (about $10\%$ of tokens); the objectness head is trained so that tokens which later obtain high classification scores also receive high objectness scores~\cite{minderer2024_owlvitv2}.
			
			\medskip
			\noindent
			\emph{Training: bipartite matching, not IoU-threshold mining}
			
			Let $\mathcal{I}(x) \subset \{1,\dots,N\}$ be the top-$K$ tokens selected by objectness. As in OWL-ViT~\cite{minderer2022_owlvit}, OWLv2 uses a DETR-style Hungarian matching loss to define positives among these tokens. Concretely, for each image $x$ the model solves a one-to-one assignment between the $M$ pseudo boxes $\{b_j\}$ and the selected tokens $\{i \in \mathcal{I}(x)\}$:
			\[
			\pi^*
			=
			\arg\min_{\pi}
			\sum_{j=1}^M
			\Bigl[
			\mathcal{C}_\text{cls}\bigl(s_{\pi(j),\,y_j}\bigr)
			+
			\lambda_\text{box}\,\mathcal{C}_\text{box}\bigl(\hat{b}_{\pi(j)}, b_j\bigr)
			\Bigr],
			\]
			where $\pi$ ranges over one-to-one assignments from boxes to tokens, $\mathcal{C}_\text{cls}$ is the sigmoid / focal classification cost for phrase $y_j$, and $\mathcal{C}_\text{box}$ combines an $\ell_1$ distance and a (G)IoU-based cost between $\hat{b}_i$ and $b_j$~\cite{minderer2022_owlvit}. Matching is therefore \emph{not} a pure IoU-threshold rule: the assignment jointly prefers tokens that have both high phrase score and good geometric overlap with the pseudo box.
			
			The resulting permutation $\pi^*$ induces binary labels $t_{ik} \in \{0,1\}$ over all token–query pairs:
			\[
			t_{ik} = 1
			\quad\text{iff}\quad
			\exists j
			\;\text{s.t.}\;
			i = \pi^*(j)
			\;\text{and}\;
			k \text{ is the query corresponding to } y_j,
			\]
			and $t_{ik} = 0$ otherwise (including all tokens that are unmatched and all negative or pseudo-negative queries). Because each pseudo box $b_j$ is matched to \emph{one} token at most, the model is explicitly discouraged from producing many redundant positives around the same object: overlapping tokens compete in the matching, and only the best one is treated as a positive, while the others become background and are down-weighted by the focal loss.
			
			\newpage
			
			The overall detection loss decomposes into a dense classification term and a regression term applied only to matched tokens:
			\[
			\mathcal{L}_\text{det}
			=
			\underbrace{\sum_{i \in \mathcal{I}(x)} \sum_{k \in \mathcal{Q}(x)}
				\mathcal{L}_\text{cls}\bigl(s_{ik}, t_{ik}\bigr)}_{\text{sigmoid / focal classification over queries}}
			\;+\;
			\lambda_\text{box}
			\underbrace{\sum_{j=1}^M
				\mathcal{L}_\text{box}\bigl(\hat{b}_{\pi^*(j)}, b_j\bigr)}_{\ell_1 + \mathrm{GIoU on matched tokens only}},
			\]
			with a separate loss on the objectness scores $o_i$ that encourages high objectness precisely for those tokens that end up with strong classification scores~\cite{minderer2024_owlvitv2}.
			
			\medskip
			\noindent
			\emph{Inference and overlapping boxes}
			
			At inference time, no matching is solved: the model runs the encoder once, predicts a box $\hat{b}_i$ and query logits $\{s_{ik}\}_{k}$ for (essentially) all tokens, and keeps token–query pairs whose detection score $\sigma(s_{ik})$ exceeds a threshold for the user-specified queries. Thanks to the one-to-one training objective inherited from OWL-ViT, the model tends to produce at most one high-scoring token per object and query, so heavy non-maximum suppression is not strictly required in the DETR sense. In practice, implementations can still apply light per-query top-$K$ filtering and/or NMS to prune occasional near-duplicate boxes, especially when many queries are evaluated or when pseudo-labels are noisy.
			
			\medskip
			\noindent
			\textbf{How can the annotator produce boxes for phrases it never saw during detection training?}
			A subtle but important point is that the OWL-ViT annotator does not rely only on its supervised detection data (Objects365, OpenImages, LVIS, Visual Genome) to recognize concepts.
			Its visual and textual backbones are initialized from CLIP-style contrastive pretraining on hundreds of millions of image--text pairs, which already align a very broad vocabulary of phrases with corresponding visual patterns~\cite{minderer2022_owlvit}.
			Detection training then mainly teaches OWL-ViT \emph{where} to put boxes, while its knowledge of \emph{what} phrases such as ``platypus'', ``dog wearing sunglasses'', or ``rusty bicycle'' look like is inherited from this large-scale image--text pretraining.
			
			\begin{itemize}
				\item \textbf{Teacher zero-shot capability.}
				Given a caption-derived N-gram such as ``dog wearing sunglasses'', OWL-ViT embeds the phrase with its text encoder and compares it to per-patch image features, exactly as in CLIP-style zero-shot classification.
				Even if no detection dataset ever contained that phrase as a box label, the shared embedding space already makes the corresponding patches stand out, so OWL-ViT can often draw a pseudo-box \emph{zero-shot}~\cite{minderer2022_owlvit,minderer2024_owlvitv2}.
				In other words, detection is treated as \emph{localized retrieval} inside the image rather than as pure supervised classification over a fixed label set.
				
				\item \textbf{Captions as prompts.}
				The N-gram does not ask the annotator to guess blindly; it acts as an explicit prompt that says ``look for \emph{this} thing in \emph{this} image''.
				For example, if the caption contains ``two drones flying over a city'', OWL-ViT is directly encouraged to search for regions that match the text ``drone'', even if its own detection training never included a dedicated ``drone'' category.
				When the phrase genuinely describes something in the image, CLIP-style alignment usually yields at least a roughly correct box.
				
				\item \textbf{What if the annotator is wrong or misses the object.}
				For any single image, the annotator can certainly fail: it may hallucinate a box for a non-visual phrase (e.g., ``click here''), or miss a small, occluded instance entirely.
				However, WebLI contains the same concept in many different images and captions.
				
				\newpage
				
				Across thousands of occurrences of ``golden retriever'' or ``drone'', the teacher’s correct localizations are \emph{consistent} (similar dogs or drones in similar regions), whereas its mistakes are visually diverse and inconsistent.
				
				When the student is trained on billions of such pseudo-labels, gradient descent naturally fits the consistent patterns and fails to fit the idiosyncratic errors, effectively denoising the teacher’s supervision over the whole corpus~\cite{minderer2024_owlvitv2}.
			\end{itemize}
			
			\medskip
			\noindent
			\textbf{Why can OWLv2 outperform its own annotator instead of copying its mistakes?}
			Intuitively, if the student only ever sees the teacher’s outputs, one might worry that it cannot do better than the teacher.
			OWLv2 overcomes this in three complementary ways.
			
			\begin{itemize}
				\item \textbf{Many more (noisy) examples than the annotator ever saw.}
				The annotator was trained on tens of millions of human-labeled boxes.
				The student, in contrast, is trained on pseudo-boxes for billions of WebLI images (when counting all mosaics and confidence thresholds), which is one to two orders of magnitude more supervision~\cite{minderer2024_owlvitv2}.
				Even if each pseudo-box is imperfect, the sheer number of partially correct instances for each concept lets the student learn richer and more robust visual features than the annotator ever could.
				
				\item \textbf{Larger backbones and more compute-efficient training.}
				OWLv2 students use larger vision backbones (e.g., SigLIP ViT-G/14) than the CLIP ViT-L/14 annotator, and the Stage~2 training loop (token dropping, objectness-based instance selection, mosaic augmentation) is engineered to push far more data through these large models within a fixed compute budget.
				Empirically, Minderer et al.\ show that these students achieve substantially higher LVIS rare-category AP and ODinW mean AP than the original OWL-ViT teacher, even though the teacher provided all pseudo-labels~\cite{minderer2024_owlvitv2}.
				
				\item \textbf{Signal versus noise at web scale.}
				For a concept like ``hydrant'', the teacher might localize it correctly in many images and miss or mislabel it in others.
				The correct localizations all share recognizable visual structure, whereas the errors are scattered over unrelated backgrounds.
				Over billions of examples, the student can only consistently reduce its loss by latching onto the stable pattern (true hydrants) rather than the inconsistent noise.
				Thus, instead of copying the teacher’s individual mistakes, OWLv2 \emph{averages them out} and retains only what is statistically supported across the corpus.
			\end{itemize}
			
			\medskip
			\noindent
			\textbf{What happens to objects that are not in the query set?}
			A complementary concern is how an open-vocabulary detector can handle objects that are present in an image but never appear in that image’s curated+N-gram query list.
			Here it is crucial that OWLv2 trains \emph{conditionally on the query set} $\mathcal{Q}(x)$ of each image~\cite{minderer2024_owlvitv2}.
			
			\begin{itemize}
				\item \textbf{Conditional supervision per image.}
				For a given training image $x$, the student is only asked: ``\emph{Given this particular list of phrases $\mathcal{Q}(x)$, which tokens correspond to which phrases?}''.
				If ``fire hydrant'' is not in $\mathcal{Q}(x)$, then hydrants in that image are simply \emph{ignored by the loss} for that phrase: they are neither positives nor explicit negatives for ``fire hydrant''.
				The model is not told that hydrants are background; it is merely not supervised about them in this particular image.
				
				\item \textbf{Coverage across WebLI.}
				Across billions of WebLI images, most semantically meaningful concepts (``hydrant'', ``escalator'', ...) do appear as queries in many other images, either via curated labels or via N-grams extracted from captions.
				Those other images \emph{do} contribute gradients for these phrases, so the student still receives substantial supervision for each common concept, just not from every image in which it happens to appear.
				
				\newpage
				
				\item \textbf{Role of CLIP/SigLIP initialization for genuinely rare phrases.}
				For truly rare or unseen phrasings, the CLIP or SigLIP initialization already provides a coarse alignment between text and image embeddings.
				OWLv2’s self-training mainly improves localization, calibration, and robustness for phrases that the annotator can already tentatively ground.
				As in CLIP zero-shot classification, entirely new test-time prompts can still be handled if they lie in the semantic neighborhood of phrases seen during pretraining or self-training.
			\end{itemize}
			
			Thus, the absence of a phrase from $\mathcal{Q}(x)$ for a particular image does not forbid the model from ever detecting that concept; it simply means that this image does not contribute any signal for that phrase.
			Over the full WebLI corpus, consistent concepts accumulate many positive examples, while idiosyncratic or spurious N-grams (e.g., ``click here'') fail to form a coherent visual pattern and are effectively ignored by the student during training.
			
			\medskip
			\noindent
			\textbf{Scaling the objective to billions of pseudo-boxes.}
			The real difficulty in Stage~2 is not the loss itself but making it computationally feasible to run this dense, open-vocabulary objective on billions of pseudo-labeled images with tens of thousands of queries per image.
			Minderer et al.\ therefore introduce three complementary efficiency mechanisms~\cite{minderer2024_owlvitv2}:
			
			\begin{itemize}
				\item \textbf{Token dropping (static visual pruning).}
				After a few ViT blocks, OWLv2 computes a simple saliency proxy for each token (per-channel feature variance) and discards the least informative tokens for the remainder of the network.
				Uniform background patches (sky, walls, large textureless regions) have low variance and are pruned; tokens that carry edges, textures, and object structure are kept.
				This halves (or more) the sequence length for all subsequent layers and detection heads, cutting FLOPs and memory while preserving object-centric regions.
				
				\item \textbf{Objectness head and dynamic instance selection.}
				Even after token dropping, naively comparing every remaining token to every query in $\mathcal{Q}(x)$ is prohibitively expensive when $|\mathcal{Q}(x)|$ can be $10^4$–$2\times 10^4$.
				OWLv2 therefore adds a lightweight \emph{objectness} head that predicts a scalar score for each token~\cite{minderer2024_owlvitv2}.
				
				During training, only the top-$K$ tokens (typically a small fraction of the retained tokens) ranked by objectness are passed through the full open-vocabulary classification head and incur the expensive dot-product loss against all queries in $\mathcal{Q}(x)$; the box head itself remains dense and is applied to all tokens.
				Tokens with low objectness are treated as background and bypass the classification head.
				
				Objectness is learned jointly with detection: tokens that are repeatedly associated with pseudo-boxes are encouraged to have high objectness, creating a self-reinforcing mechanism that focuses compute on object-like regions.
				This strategy focuses computation where objects are likely to appear and largely decouples the training cost from the size of the text vocabulary~\cite{minderer2024_owlvitv2}.
				
				\item \textbf{Mosaic augmentation at web scale.}
				Finally, OWLv2 increases the number of distinct scenes seen per optimizer step using large mosaics: instead of a single WebLI image, the input is a grid (e.g., up to $6\times 6$) of different images tiled into one canvas~\cite{ghiasi2021_simplecopypaste,minderer2024_owlvitv2}.
				All pseudo-boxes are geometrically transformed into mosaic coordinates, and the detector is trained as if this were a single large image.
				In the default configuration, each mosaic contains on average about $13.2$ raw images.
				Scaling plots therefore report the total number of \emph{raw images} seen as
				\[
				\text{\# raw images seen}
				\approx
				13.2 \times \text{(\# of mosaics)}.
				\]
				
				\newpage
				
				Within a fixed budget of optimizer steps, mosaics allow the student to experience roughly an order of magnitude more distinct scenes than a standard one-image-per-step loop, which is critical for exploiting WebLI’s diversity.
			\end{itemize}
			
			\medskip
			\noindent
			Conceptually, Stage~2 turns OWL-ViT’s pseudo-labels into a dense but noisy supervision signal that a much larger, more compute-efficient student can exploit.
			Whereas the annotator itself was only trained on tens of millions of human-labeled boxes, the student is exposed to billions of pseudo-boxes covering far more phrases and visual situations than the teacher ever saw~\cite{minderer2024_owlvitv2}.
			Over this huge dataset, consistent visual–linguistic patterns (e.g., what ``dog wearing sunglasses'' typically looks like) reinforce one another, while spurious N-grams and mislocalized boxes fail to generalize.
			Combined with the CLIP/SigLIP initialization and the compute-aware training tricks above, this explains how the OWLv2 student can eventually surpass its OWL-ViT teacher by a large margin on both LVIS rare categories and open-world benchmarks such as ODinW.
			
			\item \textbf{Stage~3: Optional fine-tuning and weight ensembling.}
			For standard benchmarks such as LVIS, a self-trained OWLv2 student can optionally be fine-tuned on the target dataset using its native annotations.
			As observed in robust fine-tuning work~\cite{wortsman2022_robustfinetuning}, this creates a tension: pure self-training yields excellent zero-shot and ODinW performance but underperforms on LVIS base categories, while full fine-tuning improves LVIS AP but partially erodes open-world generalization.
			OWLv2 addresses this by \emph{weight-space ensembling}: the final model is a convex combination
			\[
			\theta_{\text{ens}} = \lambda\, \theta_{\text{ST}} + (1-\lambda)\, \theta_{\text{FT}},
			\]
			where $\theta_{\text{ST}}$ and $\theta_{\text{FT}}$ denote the self-trained and fine-tuned checkpoints and $\lambda \in [0,1]$ controls the trade-off between robustness and in-domain accuracy~\cite{minderer2024_owlvitv2,wortsman2022_robustfinetuning}.
			By sweeping $\lambda$, Minderer et al.\ obtain a Pareto frontier of models that can be tuned to favor LVIS, ODinW, or a balanced mix, all without changing the architecture or retraining from scratch.
		\end{enumerate}
		
		These stages are executed strictly in order: pseudo-label generation is performed offline with a frozen OWL-ViT annotator; the student is then trained end-to-end on pseudo-annotations; any dataset-specific fine-tuning and weight ensembling happen only after self-training has converged.
		There is no joint training of annotator and student, and the annotator is never updated using pseudo-labels.
		
	\end{enrichment}
	
	\begin{enrichment}[OWLv2: Pseudo-label spaces and Web-scale annotation][subsection]
		\label{enr:chapter14_owlv2_pseudo_labels}
		
		\paragraph{Curated vs.\ N-gram label spaces}
		A central design question in OWLv2 is \emph{which phrases} to use when querying the OWL-ViT annotator.
		Unlike standard detectors with a fixed class list, OWL-ViT can score arbitrary text; OWLv2 exploits this by constructing, for every WebLI image, a query list that mixes human-curated object names with free-form phrases extracted from the caption~\cite{minderer2024_owlvitv2}.
		Minderer et al.\ systematically study three label spaces:
		\begin{itemize}
			\item \textbf{Curated vocabulary.}
			A fixed, human-designed list obtained by merging the category names of standard detection datasets (LVIS, Objects365, OpenImages~V4, Visual Genome), followed by simple normalization such as lowercasing and deduplication of synonyms and plural forms (Appendix~A.1 of~\cite{minderer2024_owlvitv2}).
			This yields a few thousand canonical object labels that are shared across all images and closely aligned with evaluation benchmarks such as LVIS and ODinW.
			
			\item \textbf{Machine-generated N-gram vocabulary.}
			For each WebLI image $x$, OWLv2 extracts word N-grams up to length $10$ from the associated caption and related text fields, after removing stop words and very generic phrases such as ``click here'' or file-type indicators, and capping the number of queries per image (Appendix~A.2 of~\cite{minderer2024_owlvitv2}).
			The resulting $\mathcal{V}_{\text{N-gram}}(x)$ is image-specific and captures idiosyncratic phrases that never appear in curated taxonomies, but it also introduces label noise whenever the caption is only weakly related to the visual content.
			
			\item \textbf{Union of curated and N-grams.}
			The two label spaces are combined so that the annotator sees both benchmark-aligned category names and image-specific phrases:
			\[
			\mathcal{Q}(x) = \mathcal{V}_{\text{curated}} \cup \mathcal{V}_{\text{N-gram}}(x).
			\]
			Every OWL-ViT forward pass uses this merged query list; there is no splitting of ground truth by source, and from the student detector's perspective there is just one pool of pseudo-boxes with associated phrases.
		\end{itemize}
		
		The following figure summarizes quantitatively how these three label spaces affect downstream detection, and in particular how the extra coverage from N-grams trades off against their higher noise level.
		
		\begin{figure}[H]
			\centering
			\includegraphics[width=0.85\textwidth]{Figures/Chapter_14/OwlV2_pseudo_labels.jpg}
			\caption{\textbf{Effect of pseudo-label space on OWLv2 performance.}
				Student detectors are trained on pseudo-annotations generated from curated labels only (blue circles), N-grams only (orange squares), or the union of both (green diamonds), and evaluated on LVIS and ODinW.
				Left: LVIS frequent classes, which largely overlap with the curated taxonomy.
				Middle: LVIS rare classes, which emphasize long-tail concepts.
				Right: ODinW ``in-the-wild'' datasets.
				Figure reproduced from Minderer et al.~\cite{minderer2024_owlvitv2}.}
			\label{fig:chapter14_owlv2_pseudo_labels}
		\end{figure}
		
		The three panels make the trade-off between \emph{clean but narrow} and \emph{wide but noisy} supervision visible:
		\begin{itemize}
			\item \textbf{Curated-only (clean but narrow).}
			On LVIS frequent classes (left), the curated-only student achieves the highest or nearly highest AP for a given number of examples, reflecting that curated labels provide relatively clean pseudo-boxes on categories the teacher knows well.
			On LVIS rare classes and ODinW (middle and right), the same blue curve lags behind, because many long-tail concepts never appear in the curated list at all, so the student simply never receives labels for them.
			
			\newpage
			
			\item \textbf{N-grams-only (wide but noisy).}
			The N-gram-only student substantially improves AP on LVIS rare and ODinW compared to curated-only, showing that caption-derived phrases do expose the long tail and enable better open-world generalization.
			At the same time, on LVIS frequent classes its orange curve sits consistently below the blue curve: if N-grams were as clean as curated labels, these curves would coincide.
			This systematic gap on familiar categories is how the additional label noise introduced by N-grams manifests in the plots.
			
			\item \textbf{Union of curated and N-grams (best trade-off).}
			The union model recovers most of the curated model's strength on LVIS frequent classes while matching or exceeding the N-gram model on LVIS rare and ODinW.
			Its green curve is close to blue on the left panel but clearly above both blue and orange in the middle and right panels, indicating that combining an anchored, benchmark-aligned vocabulary with a noisy but broad N-gram explorer yields the best overall balance between precision on known classes and recall on open-world concepts.
		\end{itemize}
		
		\paragraph{Why the union matters: anchor and explorer}
		Using the union of curated and N-gram vocabularies is not redundant; it compensates for complementary failure modes.
		
		\begin{itemize}
			\item \textbf{Curated vocabulary as an anchor.}
			Web captions are frequently incomplete or metaphorical: an image can clearly contain a dog while the caption says only ``my best friend enjoying the weekend''.
			If OWLv2 relied only on N-grams, the annotator would be asked to look for ``best friend'' and ``weekend'', but never for the canonical label ``dog''.
			The curated dictionary acts as a safety net: regardless of how the caption is phrased, every image is always queried for common objects such as ``person'', ``dog'', and ``car'', which stabilizes supervision on benchmark-aligned categories and prevents obvious objects from being systematically missed.
			
			\item \textbf{N-grams as an explorer.}
			Conversely, curated lists are static and cannot cover the combinatorial richness of web text.
			They describe ``dog'' and ``sunglasses'' but not necessarily ``dog wearing sunglasses'' or rare fine-grained entities such as ``Monarch on a Zinnia''.
			N-grams promote these caption phrases to first-class labels, allowing the annotator to create pseudo-boxes for concepts that never appear in any standard taxonomy.
			Across billions of images, the consistent visual patterns behind phrases such as ``drone'', ``bento box'', or ``steampunk toaster'' reinforce each other, whereas non-visual or idiosyncratic phrases fail to form a coherent pattern and are effectively suppressed by scale.
		\end{itemize}
		
		In this sense, the curated vocabulary acts as an anchor that keeps supervision aligned with canonical benchmarks and protects recall on standard categories, while the N-grams act as an explorer that pushes supervision into the long tail of web concepts; the union label space lets the student benefit from both.
		
		\newpage
		
		\paragraph{Effect of pseudo-label confidence thresholds on downstream detection}
		The label space determines what can, in principle, be labeled; confidence thresholds determine how much of that potential supervision survives filtering.
		OWLv2 therefore ablates the confidence threshold $\tau$ used to keep OWL-ViT pseudo-boxes, training otherwise identical students with $\tau \in \{0.1, 0.3, 0.5, 0.7\}$ and evaluating them on LVIS and ODinW~\cite{minderer2024_owlvitv2}.
		
		\begin{figure}[H]
			\centering
			\includegraphics[width=0.85\textwidth]{Figures/Chapter_14/OwlV2_pseudo_annotation_impact.jpg}
			\caption{\textbf{Effect of pseudo-label confidence thresholds on OWLv2 performance.}
				Each curve corresponds to a different confidence threshold $\tau$ used when filtering OWL-ViT pseudo-annotations (legend on the right).
				The $x$-axis counts the total number of pseudo-labeled examples seen during training, including repetitions.
				Figure reproduced from Minderer et al.~\cite{minderer2024_owlvitv2}.}
			\label{fig:chapter14_owlv2_pseudo_annotation_impact}
		\end{figure}
		
		Reading Figure~\ref{fig:chapter14_owlv2_pseudo_annotation_impact} from left to right, each curve first improves as the student sees more pseudo-labeled examples and then gradually saturates once the available pseudo-labels have been revisited many times.
		The position and height of this saturation encode how OWLv2 trades off label quality against scale:
		\begin{itemize}
			\item \textbf{High thresholds shrink the dataset and hurt generalization.}
			The red curves corresponding to $\tau = 0.7$ consistently saturate earliest and at the lowest AP, especially for LVIS rare classes and ODinW.
			Minderer et al.\ report that increasing $\tau$ from $0.1$ to $0.7$ reduces the number of usable WebLI images from roughly $5$ billion to a few hundred million.
			After this smaller pool has been seen a few times, the student runs out of new, informative examples, and the red curves flatten while others continue to improve.
			
			\item \textbf{Moderate thresholds keep hard examples without collapsing under noise.}
			The blue and gray curves corresponding to $\tau = 0.1$ and $\tau = 0.3$ remain the highest in the ``Unseen classes'' and ``In the Wild'' panels, while matching the best performance on frequent LVIS classes.
			
			Lower thresholds admit many more medium-confidence detections, which include a mix of genuinely hard positives and some false positives.
			If this additional supervision were dominated by noise, these curves would deteriorate or oscillate; instead, their steady upward trend indicates that, at WebLI scale, the student successfully averages out inconsistent labels while benefiting from the extra diversity.
			
			\item \textbf{Noise manifests primarily as an efficiency penalty.}
			When we compare low-threshold curves (e.g., $\tau = 0.1, 0.3$) against stricter ones at the \emph{same} position on the $x$-axis early in training, the low-threshold models sometimes lag slightly behind.
			This is the cost of noise: the student must process more pseudo-labeled examples to statistically separate signal from spurious boxes.
			Crucially, these curves do not saturate prematurely.
			As training continues and more examples are seen, the low-threshold models overtake the high-threshold ones and reach higher final AP.
			
			\newpage
			
			Thus, in OWLv2 the additional noise from low thresholds is not a hard ceiling on performance, but an efficiency penalty that WebLI's scale can amortize.
			
		\end{itemize}
		
		Together, the label-space and threshold ablations support OWLv2's overall philosophy.
		A broad union label space ensures that most semantically meaningful concepts can, at least sometimes, be named and localized, while low-to-moderate confidence thresholds maximize the number and diversity of training examples.
		Because the student can process WebLI at this scale (using dense one-stage training, token dropping, objectness-based instance selection, and mosaic augmentation), it is able to distill a noisy but extremely rich pseudo-label stream into detectors that outperform their OWL-ViT teachers on both benchmark categories and truly in-the-wild objects.
		
	\end{enrichment}
	
	\begin{enrichment}[Architecture and training efficiency][subsection]
		\label{enr:chapter14_owlv2_architecture}
		
		\paragraph{Student detector: OWL-ViT with efficiency tweaks}
		The student detector in OWLv2 retains the basic OWL-ViT structure~\cite{minderer2022_owlvit}: a ViT image encoder \(f_v\), a text encoder \(f_t\), and lightweight box and classification heads attached to per-patch visual tokens.
		The detection objective is the same dense one-stage loss as in OWL-ViT: open-vocabulary sigmoid (often focal) classification over a per-image query set, combined with \(\ell_1\) and generalized IoU losses on bounding boxes, with positives defined by an IoU threshold (e.g., $\ge 0.5$) between predicted and (pseudo) ground-truth boxes~\cite{minderer2022_owlvit,minderer2024_owlvitv2}.
		There is no Transformer decoder and no Hungarian matching; supervision is fully dense over the patch grid.
		Queries include both positive category names and randomly sampled ``pseudo-negative'' labels from other images, just as in OWL-ViT.
		
		What changes in OWLv2 is \emph{how} this detector is trained at web scale.
		The main additions are:
		\begin{itemize}
			\item Token dropping for cheap ViT forward passes and lower memory use.
			\item An objectness head for focusing classification on likely object patches.
			\item Mosaic augmentation to expose the student to far more distinct images per training step.
		\end{itemize}
		
		\paragraph{Token dropping}
		OWLv2 adopts a form of dynamic token sparsification inspired by methods such as DynamicViT and Token Merging~\cite{rao2022_dynamicvit,bolya2023_tokenmerging}.
		After a subset of early Transformer blocks, the model computes a simple saliency score for each patch token, for example based on its feature variance across channels, and drops a fixed fraction (e.g., \(50\%\)) of the least informative tokens from subsequent layers~\cite{minderer2024_owlvitv2}.
		This reduces the number of tokens processed by later, more expensive layers without modifying the underlying ViT backbone or its final feature stride.
		During self-training, token dropping provides a substantial reduction in FLOPs and memory; at inference time, the full set of tokens can be used.
		
		\paragraph{Objectness head and instance selection}
		Running open-vocabulary classification against a very large label space (hundreds of thousands of queries) for every token is prohibitively expensive.
		To decouple training cost from vocabulary size, OWLv2 adds an \emph{objectness head} that predicts a scalar objectness score for each token.
		During self-training, only the top fraction of tokens by objectness (roughly \(10\%\) in the experiments) are passed through the full classification head and incur the expensive open-vocabulary loss~\cite{minderer2024_owlvitv2}.
		
		Importantly, objectness is itself learned.
		Its supervision signal comes from the eventual classification scores: tokens that repeatedly receive high classification probabilities for some query are encouraged to have high objectness, so the objectness head learns to anticipate where interesting objects are likely to appear.
		This mechanism is reminiscent of efficient DETR variants that use dense objectness priors to restrict decoding to promising locations~\cite{yao2021_efficientdetr}, but adapted to the encoder-only, patch-based setting of OWL-ViT.
		
		\paragraph{Mosaic augmentation and the ``13.2$\times$'' factor}
		To maximize the number of distinct images seen under limited training steps, OWLv2 uses large mosaic grids that tile multiple WebLI images into a single training example.
		Similar to copy-paste and mosaic augmentations used in CNN detectors, grids of varying sizes (e.g., $1\times 1$ to $5\times 5$) are sampled, and pseudo-boxes are geometrically transformed into the corresponding mosaic coordinates~\cite{minderer2024_owlvitv2}.
		
		In the configuration used for the main scaling experiments, a single mosaic contains on average about $13.2$ distinct raw images.
		Mosaics thus allow the student to process roughly an order of magnitude more images than a standard single-image training loop within the same compute budget.
		Consequently, the ``total examples seen'' reported on the $x$-axis of the scaling plots should be interpreted as the \emph{effective number of raw images} processed (approximately $13.2\times$ the number of optimizer steps).
		
	\end{enrichment}
	
	\begin{enrichment}[Scaling behavior, results, and trade-offs][subsection]
		
		\paragraph{Scaling laws and ``student surpasses teacher''}
		One of the main contributions of OWLv2 is an empirical study of scaling laws for open-vocabulary detection under self-training.
		The following figure illustrates performance (e.g., LVIS rare AP) against the total number of WebLI examples seen during self-training, for several model sizes and architectural variants.
		
		\begin{figure}[H]
			\centering
			\includegraphics[width=0.85\textwidth]{Figures/Chapter_14/OwlV2_scaling.jpg}
			\caption{\textbf{Scaling behavior of OWLv2 under self-training.} Zero-shot LVIS performance improves steadily as the number of self-training examples and model size increase.
				Students trained on pseudo-annotations eventually surpass the OWL-ViT annotator, and the Pareto frontier over compute budgets shifts upward with more data and larger backbones.
				Figure reproduced from Minderer et al.~\cite{minderer2024_owlvitv2}.}
			\label{fig:chapter14_owlv2_scaling}
		\end{figure}
		
		Several consistent patterns emerge~\cite{minderer2024_owlvitv2}:
		\begin{itemize}
			\item \textbf{Self-training is beneficial even at moderate compute.} For reasonable training budgets, students already outperform the frozen OWL-ViT annotator that generated their pseudo-labels, demonstrating a clear ``student surpasses teacher'' effect.
			\item \textbf{Detection exhibits familiar log-linear scaling.} As in large-scale classification and language modeling, performance grows roughly log-linearly with compute and data once models are in the high-data regime.
			
			\newpage
			
			\item \textbf{Model size vs.\ training duration trade-off.} For in-distribution benchmarks such as LVIS, larger backbones (e.g., ViT-L/14) dominate smaller ones once sufficient data is seen, but for heavily out-of-distribution settings (ODinW), it can be better to train a smaller backbone for longer rather than a larger one for fewer updates.
			\item \textbf{Largest model.} A SigLIP ViT-G/14 student trained with OWL-ST reaches mid-$40$ AP on LVIS rare categories (around $46$--$47$ AP depending on the exact training and ensembling setup), which at the time of publication represents one of the strongest reported LVIS rare results among open-vocabulary detectors~\cite{minderer2024_owlvitv2}.
		\end{itemize}
		
		\paragraph{Fine-tuning vs.\ open-world generalization}
		Like many contrastively trained vision--language models, OWLv2 exhibits a trade-off between performance on a specific target dataset and robustness to distribution shift~\cite{radford2021_clip,wortsman2022_robustfinetuning}.
		The below figure illustrates this trade-off using LVIS (target dataset) and ODinW13 (out-of-distribution benchmark).
		
		\begin{figure}[H]
			\centering
			\includegraphics[width=0.85\textwidth]{Figures/Chapter_14/OwlV2_tradeoff.jpg}
			\caption{\textbf{Trade-off between fine-tuned and open-world performance.} Self-training on WebLI improves both LVIS and ODinW13 performance (red dots).
				Fine-tuning on LVIS further improves LVIS AP but reduces ODinW13 AP (light blue squares).
				Weight-space ensembling between the self-trained and fine-tuned checkpoints (purple diamonds) yields a strictly better Pareto frontier, partially restoring open-world robustness at almost no extra cost.
				Figure reproduced from Minderer et al.~\cite{minderer2024_owlvitv2}.}
			\label{fig:chapter14_owlv2_tradeoff}
		\end{figure}
		
		Without fine-tuning, OWLv2 models already deliver strong zero-shot performance across many datasets (LVIS, ODinW13, Objects365, OpenImages) thanks to the diversity of WebLI pseudo-annotations.
		Fine-tuning on LVIS further boosts performance on LVIS categories but tends to degrade open-world generalization.
		Weight-space ensembling between self-trained and fine-tuned checkpoints offers a simple way to shift this trade-off, recovering much of the ODinW performance while maintaining high LVIS AP~\cite{minderer2024_owlvitv2}.
		
	\end{enrichment}
	
	\newpage
	
	\begin{enrichment}[Comparison to Grounding DINO and limitations][subsection]
		
		\paragraph{OWL-ViT / OWLv2 vs.\ Grounding DINO}
		From the perspective of Chapter~14, OWLv2 and Grounding DINO~\cite{liu2023_groundingdino} represent two complementary strategies for scaling open-vocabulary detection.
		\begin{itemize}
			\item \textbf{Architecture and fusion.} Grounding DINO starts from a DINO-DETR-style encoder--decoder with multi-scale deformable attention and injects text tokens deep into both encoder and decoder via cross-attention, enabling strong phrase grounding and fine-grained region--text alignment.
			By contrast, OWLv2 retains OWL-ViT’s dual-encoder, late-fusion design: image and text are encoded separately and only interact in the final dot-product similarity between per-patch features and query embeddings.
			This makes OWLv2 much closer to CLIP-style retrieval models and simplifies reuse of the encoders for other tasks.
			\item \textbf{Training data.} Grounding DINO relies on curated grounding datasets (e.g., Objects365, GoldG, Cap4M) with box-level text supervision~\cite{liu2023_groundingdino}.
			OWLv2’s main gains come from scaling to roughly two billion \emph{pseudo-annotated} WebLI images, produced automatically from captions with minimal filtering~\cite{minderer2024_owlvitv2}.
			
			\item \textbf{Inference behavior.} Grounding DINO’s tightly coupled encoder--decoder must be re-run whenever the text prompt changes, which can be expensive when exploring many complex prompts.
			OWLv2 inherits OWL-ViT’s decoupled, encoder-only inference: image features can be precomputed and indexed, while new text queries are embedded on the fly.
			This is advantageous for large-scale retrieval and detection-as-search applications.
			\item \textbf{Performance.} At publication time, OWLv2’s SigLIP ViT-G/14 student achieved state-of-the-art zero-shot rare-category AP on LVIS, substantially outperforming OWL-ViT v1 and strong baselines such as F-VLM and DetCLIP~\cite{kuo2023_fvlm,yao2022_detclip,minderer2024_owlvitv2}.
			Grounding DINO remains competitive and often superior for phrase-level grounding and tasks that require tight coupling between language and detection, especially when trained with strong region--phrase supervision.
		\end{itemize}
		
		\paragraph{Limitations and outlook}
		Minderer et al.\ highlight several limitations of OWLv2~\cite{minderer2024_owlvitv2}.
		\begin{itemize}
			\item \textbf{Compute and data cost.} Self-training at the scale of billions of images and large ViT backbones demands substantial compute and infrastructure.
			Scaling further is in principle effective but quickly becomes impractical without more efficient architectures or training recipes.
			\item \textbf{Trade-off between specialization and robustness.} Fine-tuning on a target detection dataset improves performance on its label space but reduces robustness to distribution shift and sensitivity to prompt wording, similar to CLIP fine-tuning~\cite{wortsman2022_robustfinetuning}.
			Weight ensembling mitigates but does not completely remove this trade-off.
			\item \textbf{Noise and bias in pseudo-labels.} Although OWLv2 shows that simple pseudo-annotations can be surprisingly effective at scale, they still inherit biases from the annotator, the label space, and the WebLI corpus.
			Improving pseudo-label quality or incorporating uncertainty estimates could further enhance performance.
		\end{itemize}
		
		Despite these limitations, OWLv2 demonstrates that a relatively simple OWL-ViT-style detector, combined with web-scale self-training, can close much of the gap to more architecturally complex open-vocabulary detectors.
		It also provides an important precedent for future work that treats detection as a scalable web-learning problem, much like modern image and language models.
		
	\end{enrichment}
	
\end{enrichment}


\chapterimage{head2.png} % Chapter heading image

% Chapter-specific content starts here
\chapter{Lecture 15: Image Segmentation}

%-----------------------------------------------------------------------------------
%	CHAPTER 15 - Lecture 15: Image Segmentation
%-----------------------------------------------------------------------------------

\section{From Object Detection to Segmentation}

\noindent In the previous chapter, we explored \textbf{object detection}, where the goal was to localize and classify objects within an image using bounding boxes. Object detection models such as Faster R-CNN \cite{ren2016_fasterrcnn} and YOLO \cite{redmon2016_yolo} predict discrete object regions but do not assign labels to every pixel. 
However, many real-world applications require a finer-grained understanding beyond bounding boxes. This leads us to the problem of \textbf{image segmentation}, where the task is to assign a category label to every pixel in the image.

\begin{figure}[H]
	\centering
	\includegraphics[width=0.8\textwidth]{Figures/Chapter_15/slide_36.jpg}
	\caption{Comparison of different computer vision tasks: classification, object detection, and semantic and instance segmentation. We begin with Semantic Segmentation.}
	\label{fig:chapter15_cv_tasks}
\end{figure}

\newpage

\noindent As shown in Figure~\ref{fig:chapter15_cv_tasks}, segmentation can be divided into two primary tasks:

\begin{itemize}
	\item \textbf{Semantic segmentation:} Assigns a category label to each pixel but does not differentiate between instances of the same class.
	\item \textbf{Instance segmentation:} Extends semantic segmentation by distinguishing individual object instances.
\end{itemize}

\noindent We begin by studying \textbf{semantic segmentation} because it serves as the foundation for understanding pixel-wise classification. Unlike instance segmentation, which requires distinguishing between different objects of the same category, semantic segmentation focuses solely on identifying the type of object at each pixel. 
By first mastering the fundamental principles of pixel-wise classification, we can later build upon them to incorporate instance-level distinctions.

\begin{enrichment}[Why is Object Detection Not Enough?][section]
	\noindent
	Consider an autonomous vehicle navigating through a crowded urban environment. Object detection is a crucial first step: it draws bounding boxes around pedestrians, vehicles, and traffic signs, and already provides \emph{coarse} spatial awareness (for example, that a pedestrian is somewhere near the curb rather than in the middle of the road). However, this level of understanding is still not sufficient for safe, fine-grained decision-making:
	
	\begin{itemize}
		\item \textbf{Bounding boxes are coarse approximations.} A bounding box is a rectangle that roughly encloses an object, not its true shape. In many cases this is enough to know that a pedestrian is “near the road”, but in safety-critical edge cases—such as a foot just crossing the curb versus standing safely on the sidewalk—the box does not reveal the precise contact boundary between \textit{pedestrian} and \textit{road}.
		
		\item \textbf{Occlusions and overlaps create ambiguity.} When objects overlap (e.g., a cyclist partially hidden behind a parked car), their bounding boxes may intersect or fragment. From boxes alone, it is hard to infer which pixels belong to which object, who is in front or behind, and exactly where the free space lies between them.
		
		\item \textbf{No labels for “stuff” and free space.} Object detection focuses on discrete, countable “things” (cars, pedestrians, traffic lights), but leaves the background unlabeled. It does not differentiate drivable road surface from sidewalks, bike lanes, grass, or curbs at the pixel level, even though this information is crucial for path planning and rule-following (e.g., staying within the lane markings).
	\end{itemize}
	
	\medskip \noindent
	\textbf{Semantic segmentation} addresses these limitations by assigning a class label to \emph{every pixel} in the image. Instead of just knowing that “there is a pedestrian in this box,” the model produces a dense map indicating exactly which pixels are \textit{road}, \textit{sidewalk}, \textit{pedestrian}, \textit{car}, or \textit{building}. This pixel-wise understanding provides the geometric and contextual detail needed for precise obstacle avoidance, free-space estimation, and safe navigation in complex scenes.
\end{enrichment}
	
	\begin{figure}[H]
		\centering
		\includegraphics[width=0.8\textwidth]{Figures/Chapter_15/slide_71.jpg}
		\caption{Segmentation differentiates between \textit{things} (discrete objects like cars, people) and \textit{stuff} (amorphous regions like sky, road).}
		\label{fig:chapter15_things_stuff}
	\end{figure}
	
	\noindent In Figure~\ref{fig:chapter15_things_stuff}, we see a breakdown of image elements into \textit{things} (object categories that can be separated into instances, such as \textit{cars, pedestrians, trees}) and \textit{stuff} (regions that lack clear boundaries, such as \textit{sky, road, grass}).
	This pixel-level distinction enables applications such as lane detection, drivable area estimation, and pedestrian tracking, all of which contribute to safer and more efficient navigation.

\noindent The next sections will cover the fundamental methods used in segmentation, beginning with \textbf{semantic segmentation}, before proceeding to \textbf{instance segmentation}.

\newpage

\section{Advancements in Semantic Segmentation}

\noindent In this section, we explore the evolution of semantic segmentation techniques, focusing on solutions that are convolutional neural networks (CNNs) based, reaching to more contemporary architectures. While CNNs have been foundational in image processing tasks, recent advancements indicate that transformer-based models have achieved superior accuracy in segmentation tasks, including semantic segmentation. These will only be discussed in future parts of this document. 

\subsection{Early Approaches: Sliding Window Method}

\noindent A straightforward yet inefficient approach to semantic segmentation involves the \textbf{sliding window} technique. In this method, for each pixel in the image, a patch centered around the pixel is extracted and classified using a CNN to predict the category label of the center pixel.

\begin{figure}[H]
	\centering
	\includegraphics[width=0.8\textwidth]{Figures/Chapter_15/slide_39.jpg}
	\caption{Sliding window approach for semantic segmentation, illustrating the inefficiency due to redundant computations over overlapping patches.}
	\label{fig:chapter15_sliding_window}
\end{figure}

\noindent As depicted in Figure~\ref{fig:chapter15_sliding_window}, this approach is computationally expensive because it fails to reuse shared features between overlapping patches, leading to redundant calculations.

\subsection{Fully Convolutional Networks (FCNs)}

\noindent To address the inefficiencies of the sliding window method, \textbf{Fully Convolutional Networks (FCNs)} were introduced to the task \cite{long2015_fcn}. FCNs utilize a fully convolutional backbone to extract features from the entire image, maintaining the spatial dimensions throughout the layers by employing same padding and 1x1 convolutions. The network outputs a feature map with dimensions corresponding to the input image, where each channel represents a class. The final classification for each pixel is obtained by applying a softmax function followed by an argmax operation across the channels.

\begin{figure}[H]
	\centering
	\includegraphics[width=0.8\textwidth]{Figures/Chapter_15/slide_40.jpg}
	\caption{Architecture of a Fully Convolutional Network maintaining input spatial dimensions, producing a feature map with $C \times H \times W$, where $C$ is the number of classes.}
	\label{fig:chapter15_fcn_architecture}
\end{figure}

\noindent Training is conducted using a per-pixel cross-entropy loss, comparing the predicted class probabilities to the ground truth labels for each pixel.

\subsection{Challenges in FCNs for Semantic Segmentation}

\noindent Despite their advancements, FCNs encounter specific challenges:

\begin{itemize}
	\item \textbf{Limited Receptive Field:} The effective receptive field size grows linearly with the number of convolutional layers. For instance, with $L$ layers of 3x3 convolutions, the receptive field is $1 + 2L$, which may be insufficient for capturing global context.
	\item \textbf{Computational Cost:} Performing convolutions on high-resolution images is computationally intensive. Architectures like ResNet address this by aggressively downsampling the input, but this can lead to a loss of spatial detail.
\end{itemize}

\subsection{Encoder-Decoder Architectures}

\noindent To overcome these challenges, encoder-decoder architectures have been proposed, such as the model by Noh et al. \cite{noh2015_deconvnet}. These networks consist of two main components:

\begin{itemize}
	\item \textbf{Encoder:} A series of convolutional and pooling layers that progressively downsample the input image, capturing high-level semantic features while expanding the receptive field.
	\item \textbf{Decoder:} A sequence of upsampling operations, including unpooling and deconvolutions, that restore the spatial dimensions to match the original input size, enabling precise localization for segmentation.
\end{itemize}

\noindent The encoder captures rich, abstract feature representations by reducing spatial resolution while increasing feature depth, whereas the decoder reconstructs fine-grained spatial details necessary for accurate per-pixel predictions.

\newpage

\noindent While this encoder-decoder design is applied here for semantic segmentation, it is a widely used architectural pattern in deep learning and extends to many other tasks. For example:

\begin{itemize}
	\item \textbf{Machine Translation:} Transformer-based sequence-to-sequence models such as T5 \cite{raffel2020_t5} and BART \cite{lewis2020_bart} employ an encoder to process input text and a decoder to generate translated output.
	\item \textbf{Medical Image Analysis:} U-Net \cite{ronneberger2015_unet} applies an encoder-decoder structure for biomedical image segmentation, achieving precise boundary delineation in tasks like tumor segmentation.
	\item \textbf{Anomaly Detection:} Autoencoders use an encoder to learn compressed feature representations and a decoder to reconstruct inputs, enabling anomaly detection by identifying discrepancies between the input and reconstruction.
	\item \textbf{Super-Resolution and Image Generation:} Models like SRGAN \cite{ledig2017_srgan} employ an encoder to extract image features and a decoder to generate high-resolution outputs.
\end{itemize}

\noindent As we continue, we will encounter various adaptations of this fundamental encoder-decoder structure, each tailored to the specific requirements of different tasks.

\begin{figure}[H]
	\centering
	\includegraphics[width=0.8\textwidth]{Figures/Chapter_15/slide_45.jpg}
	\caption{Encoder-decoder architecture for semantic segmentation, featuring downsampling in the encoder and upsampling in the decoder to achieve pixel-wise classification.}
	\label{fig:chapter15_encoder_decoder}
\end{figure}

\section{Upsampling and Unpooling}

\noindent To enhance spatial resolution in feature maps, we employ \textbf{upsampling} techniques. Until now in this course, we have not introduced any method for systematically enlarging the spatial dimensions of tensors in a meaningful way. While we previously used bilinear interpolation to project proposals onto feature maps after downsampling (\ref{subsubsec:roi_align_intro}), we have yet to explore how such techniques can be adapted for general upsampling—something we will examine in later sections.

\noindent Although we can increase tensor size using \textbf{zero-padding} along the borders, this does not introduce any new spatial information or recover lost details, making it ineffective for true upsampling. Instead, we require dedicated upsampling methods that intelligently restore missing details while preserving spatial coherence. Throughout this section, we will explore various approaches that allow us to increase resolution effectively, ensuring that the upsampled feature maps retain meaningful information.

\newpage

\subsubsection{Do Interpolated Pixels Need to be ``Valid'' Image Values?}
\label{subsec:chapter15_valid_pixels}
\noindent
All of the upsampling and unpooling methods we have discussed (nearest neighbor, bilinear, bicubic, transposed convolution, etc.) operate in \emph{continuous} space: they produce real-valued outputs by combining neighboring pixels or features with real-valued weights. This naturally raises two related questions:
\begin{enumerate}
	\item What happens if the resulting pixel/feature values are non-integer or fall outside the usual image range?
	\item When (if ever) do we need to enforce that the upsampled result is a \emph{valid} image (e.g., integer RGB values in \([0, 255]\))?
\end{enumerate}

\paragraph{Inside a neural network: real-valued feature maps are perfectly fine}
\noindent
Within a convolutional network, tensors represent \emph{features}, not necessarily display-ready images. In this setting:
\begin{itemize}
	\item Feature maps are typically stored as 32-bit floating-point values, and can take on any real value (positive or negative, large or small).
	\item Upsampling operations (nearest neighbor, bilinear, bicubic, transposed convolution) simply produce new floating-point values. There is no requirement that these be integers or lie within a specific range; subsequent layers and nonlinearities will transform them further.
	\item Any normalization or scaling applied to the input (e.g., mapping RGB values from \([0,255]\) to \([0,1]\) or standardizing to zero mean and unit variance) is usually \emph{inverted only at the very end}, if we want to visualize or save an image.
\end{itemize}
From this perspective, ``non-integer'' or slightly out-of-range values are not a problem at all during intermediate processing: the network is trained end-to-end to work with these continuous-valued feature maps.

\paragraph{At the output: producing a valid image for visualization or storage}
\noindent

The situation changes when the goal is to produce a \emph{valid image} as the final output (e.g., in super-resolution, image-to-image translation, or generative models). In that case, we typically want:

\begin{itemize}
	\item Pixel values in a fixed range (for example \([0,1]\) or \([0,255]\)).
	\item Integer-valued pixels if we are saving to standard formats (e.g., 8-bit \texttt{uint8} RGB).
\end{itemize}

\medskip 

Common strategies in this case are:

\begin{itemize}
	\item \textbf{Constrain the range with an activation:} Use a final activation such as \(\sigma(\cdot)\) (sigmoid) to map outputs to \([0,1]\), or \(\tanh(\cdot)\) to map to \([-1,1]\). During training, the loss is computed against normalized target images in the same range.
	\item \textbf{Post-processing after the network:} Allow the network to output unconstrained real values, then:
	\begin{enumerate}
		\item De-normalize (invert any input normalization, e.g. multiply by standard deviation and add mean).
		\item \textbf{Clamp} values to the valid range, e.g.\ \(\text{pixel} \leftarrow \min(\max(\text{pixel}, 0), 1)\) or \([0,255]\).
		\item \textbf{Quantize} to integers if needed, e.g.\ \(\text{pixel}_{\text{uint8}} = \text{round}(255 \cdot \text{pixel}_{[0,1]})\).
	\end{enumerate}
	
	\newpage
	
	\item \textbf{Handling overshoot in higher-order interpolation:} Methods like bicubic interpolation can produce values slightly outside the original range (due to oscillatory cubic kernels). In classical image processing and in deep learning code, the standard remedy is simple clamping before display or saving.
\end{itemize}

\medskip

In other words, when we care about producing a valid, displayable image, validity is enforced \emph{at the very end} by range restriction and (optionally) quantization—not by changing the upsampling method itself.

\paragraph{Summary: feature maps vs.\ final images}
\noindent
To summarize:
\begin{itemize}
	\item For \textbf{internal feature maps}, non-integer and even slightly out-of-range values are entirely acceptable; the network treats them as continuous signals and learns to use them.
	\item For \textbf{final image outputs}, we typically normalize during training and then de-normalize, clamp to a valid range, and quantize at inference time to obtain a proper image representation (e.g., 8-bit RGB).
\end{itemize}

\medskip

Thus, all of the upsampling and unpooling methods discussed in this chapter can be used without modification inside a network; concerns about ``valid pixels'' are addressed at the output layer or in a simple post-processing step when we need a real image rather than a learned feature map.

\newpage

\noindent A crucial variant of upsampling is \textbf{unpooling}, which aims to reverse the effects of pooling operations. While pooling reduces resolution by discarding spatial details, unpooling attempts to restore them, facilitating fine-grained reconstruction of object boundaries. However, unpooling alone is often insufficient for producing smooth and accurate feature maps, as it merely places values in predefined locations without estimating missing information. This can result in reconstruction gaps, blocky artifacts, or unrealistic textures. As we will see, more advanced upsampling techniques address these shortcomings by incorporating interpolation and learnable transformations.

\noindent In the decoder architecture proposed by Noh et al., unpooling plays a fundamental role in progressively recovering lost spatial information. It bridges the gap between the high-level semantic representations learned by the encoder and the dense, pixel-wise predictions required for precise classification.

\noindent In the following sections, we explore various upsampling strategies, beginning with fundamental unpooling techniques and gradually progressing toward more advanced methods.

\subsection{Bed of Nails Unpooling} 

\noindent One of the simplest forms of unpooling is known as \textbf{Bed of Nails} unpooling. To illustrate the concept, consider the following example: We're given an input tensor of size $C \times 2 \times 2$, and our objective is to produce an output tensor of size $C \times 4 \times 4$.

\noindent The method follows these steps:

\begin{itemize}
	\item An output tensor of the desired size is initialized with all zeros.
	\item The output tensor is partitioned into non-overlapping regions, each corresponding to a single value from the input tensor. The size of these regions is determined by the \textbf{upsampling factor} $s$, which is the ratio between the spatial dimensions of the output and the input. For example, if the input is $H \times W$ and the output is $sH \times sW$, then each region in the output has size $s \times s$. 
	\item Each value from the input tensor is placed in the upper-left corner of its corresponding region in the output.
	\item All remaining positions are left as zeros.
\end{itemize}

\noindent The term "Bed of Nails" originates from the characteristic sparse structure of this unpooling method, where non-zero values are positioned in a regular grid pattern, resembling nails protruding from a flat surface.

\paragraph{Limitations of Bed of Nails Unpooling}

\noindent While conceptually simple, Bed of Nails unpooling suffers from a critical flaw: it introduces severe \textbf{aliasing}, which significantly degrades the quality of the reconstructed feature maps. By sparsely placing input values into an enlarged output tensor and filling the remaining positions with zeros, this method results in a highly discontinuous representation with abrupt intensity changes. These gaps introduce artificial high-frequency components, making it difficult to recover fine spatial details and leading to distorted reconstructions.

The primary drawbacks of Bed of Nails unpooling are:

\begin{itemize}
	\item \textbf{Sparse Representation:} The method leaves large gaps of zeros between meaningful values, creating an unnatural, high-frequency pattern that distorts spatial information.
	\item \textbf{Abrupt Intensity Shifts:} The sharp transitions between non-zero values and surrounding zeros introduce edge artifacts, leading to aliasing effects such as jagged edges and moiré patterns.
	\item \textbf{Loss of Fine Detail:} The lack of interpolation prevents smooth reconstructions, making it difficult to recover object boundaries and subtle spatial features.
\end{itemize}

\newpage

Because of these limitations, Bed of Nails unpooling is rarely used in practice. Its inability to provide a smooth, information-preserving reconstruction makes it unsuitable for tasks requiring high-quality feature map upsampling.

\begin{figure}[H]
	\centering
	\includegraphics[width=0.8\textwidth]{Figures/Chapter_15/bed_of_nails.jpg}
	\caption{Comparison of a well-sampled image (left) versus one affected by aliasing (right). The right image exhibits moiré patterns due to insufficient sampling, a phenomenon similar to the high-frequency distortions introduced by Bed of Nails unpooling. Source: \cite{wiki_Aliasing}.}
	\label{fig:chapter15_bed_of_nails_artifacts}
\end{figure}

\subsection{Nearest-Neighbor Unpooling}

\noindent A more practical alternative to Bed of Nails unpooling is \textbf{Nearest-Neighbor unpooling}. Instead of placing a single value in the upper-left corner and filling the rest with zeros, this method \textbf{copies the value across the entire corresponding region}, ensuring a more continuous feature map.

\begin{figure}[H]
	\centering
	\includegraphics[width=0.8\textwidth]{Figures/Chapter_15/slide_47.jpg}
	\caption{Comparison of Bed of Nails unpooling (left) and Nearest-Neighbor unpooling (right).}
	\label{fig:chapter15_unpooling}
\end{figure}

\newpage

\noindent The key advantages of Nearest-Neighbor unpooling include:

\begin{itemize}
	\item \textbf{Smoother Transitions:} By replicating values across the upsampled regions, Nearest-Neighbor unpooling maintains spatial continuity. In contrast, Bed of Nails unpooling introduces sharp jumps between non-zero values and large zero-filled areas, which disrupts smooth feature propagation.
	\item \textbf{Reduced Aliasing:} The discontinuities introduced by zero-padding in Bed of Nails unpooling create artificial high-frequency patterns, leading to jagged edges and moiré artifacts. Nearest-Neighbor unpooling minimizes these distortions by ensuring a more uniform intensity distribution.
	\item \textbf{Better Feature Preservation:} Copying values instead of inserting zeros retains more useful information about the original feature map. Since features remain continuous rather than fragmented by empty gaps, spatial relationships between objects are better preserved.
\end{itemize}

\noindent These properties make Nearest-Neighbor unpooling a more effective choice than Bed of Nails, particularly for reducing aliasing effects. By ensuring smoother transitions and preventing artificial high-frequency noise, it produces cleaner and more reliable feature maps, making it more suitable for deep learning applications.

\noindent However, Nearest-Neighbor unpooling still has limitations. Since it simply copies values, it can produce blocky (unsmooth) artifacts and lacks the ability to generate new information between upsampled pixels. This makes it unsuitable for capturing fine details, especially when dealing with natural images or complex textures.

\noindent To achieve better reconstructions, more advanced upsampling methods are used. These include:
\begin{itemize}
	\item \textbf{Bilinear Interpolation:} A smoother alternative that interpolates pixel values using a weighted average of neighboring points. We've already covered it extensively. 
	\item \textbf{Bicubic Interpolation:} Extends bilinear interpolation by considering more neighbors and applying cubic functions for higher-quality results.
	\item \textbf{Max Unpooling:} A structured approach that retains important features by reversing pooling operations using stored indices.
	\item \textbf{Transposed Convolution:} A learnable upsampling technique that enables neural networks to reconstruct detailed feature maps through trainable filters.
\end{itemize}

\noindent In the following parts, we will explore each of these methods, highlighting their advantages and trade-offs in deep learning applications.

\subsection{Bilinear Interpolation for Upsampling}

\noindent While nearest-neighbor unpooling provides a simple way to upsample feature maps, it often introduces blocky artifacts due to the direct replication of values. A more refined approach is \textbf{bilinear interpolation}, which estimates each output pixel as a weighted sum of its surrounding neighbors, resulting in a smoother reconstruction.

\noindent Consider an input feature map of shape $C \times H \times W$ and an output of shape $C \times H' \times W'$, where the spatial dimensions are enlarged ($H' > H$, $W' > W$). Unlike unpooling, which places values at predefined locations without interpolation, bilinear interpolation calculates each pixel's intensity by considering its four nearest neighbors in the original input feature map. 

\subsubsection{Bilinear Interpolation: Generalized Case} \noindent Given an input feature map $\mathbf{I}$ of size $C \times H \times W$, we 

define an upsampled feature map $\mathbf{I'}$ of size $C \times H' \times W'$.

\newpage

To compute the value of a pixel at a location $(x', y')$ in the upsampled output, we follow these steps: 

\begin{itemize} 
	\item \textbf{Mapping to the Input Grid:} The coordinate $(x', y')$ in the output feature map is mapped back to the corresponding position $(x, y)$ in the input space using the scaling factors: \[ x = \frac{x' (W - 1)}{W' - 1}, \quad y = \frac{y' (H - 1)}{H' - 1} \] where $W'$ and $H'$ are the new width and height, and $W, H$ are the original dimensions. 
	\item \textbf{Identifying Neighboring Pixels:} The four closest integer grid points that enclose $(x, y)$ are determined as: \[ a = (x_0, y_0), \quad b = (x_0, y_1), \quad c = (x_1, y_0), \quad d = (x_1, y_1) \] where: \[ x_0 = \lfloor x \rfloor, \quad x_1 = \lceil x \rceil, \quad y_0 = \lfloor y \rfloor, \quad y_1 = \lfloor y \rfloor . \] These four points form a bounding box around $(x, y)$. 
	\item \textbf{Computing the Interpolation Weights:} Each neighboring pixel contributes to the final interpolated value based on its distance to $(x, y)$. The interpolation weights are computed as: \[ w_a = (x_1 - x) (y_1 - y), \quad w_b = (x_1 - x) (y - y_0) \] \[ w_c = (x - x_0) (y_1 - y), \quad w_d = (x - x_0) (y - y_0). \] \item 
	
	\textbf{Normalization:} To ensure that the weights sum to one, we apply a normalization factor: \[ \text{norm\_const} = \frac{1}{(x_1 - x_0)(y_1 - y_0)}. \] \item \textbf{Computing the Interpolated Value:} The final interpolated intensity at $(x', y')$ is then computed as: \[ I'(x', y') = w_a I_a + w_b I_b + w_c I_c + w_d I_d. \] \end{itemize}
	
	\begin{figure}[H] \centering \includegraphics[width=0.75\textwidth]{Figures/Chapter_15/slide_48.jpg} \caption{Bilinear interpolation applied to a $C \times 2 \times 2$ input tensor, producing a $C \times 4 \times 4$ output. Each upsampled value is computed as a weighted sum of its four nearest neighbors in the original feature map.} \label{fig:chapter15_bilinear_interpolation}
	\end{figure}
	
\subsubsection{Advantages and Limitations of Bilinear Interpolation}
\noindent
Bilinear interpolation offers clear improvements over nearest-neighbor unpooling when upsampling feature maps or images. Instead of simply copying the nearest value, each output pixel is computed as a weighted average of its four closest input pixels, with weights determined by geometric distance. This produces smoother transitions, reduces blocky artifacts, and better preserves local spatial relationships than nearest-neighbor methods.

However, bilinear interpolation also has important limitations. Because it relies on only four neighbors and uses simple linear weighting, it tends to blur high-frequency details: fine textures, sharp edges, and small-scale patterns can become softened. In effect, bilinear interpolation trades off blockiness for smoothness, but at the cost of some sharpness and detail.

\subsection{Bicubic Interpolation for Upsampling}
\noindent
Bicubic interpolation is a more advanced alternative to nearest-neighbor or bilinear upsampling. Instead of using just four neighbors, bicubic interpolation considers a \(4 \times 4\) neighborhood (16 pixels) around each output position and applies a cubic weighting function along each axis. This broader context and smoother weighting scheme allow the method to better respect local structure and produce sharper, more detailed upsampled results.

\subsubsection{Why Bicubic Interpolation?}
\noindent
The wider support of bicubic interpolation directly addresses the limitations of bilinear interpolation. By aggregating information from sixteen neighboring pixels and using cubic (rather than linear) weights, bicubic interpolation can better preserve edges, reduce blurring, and maintain fine textures. For this reason, it is commonly used as a high-quality default for image resizing and is often preferred in deep learning pipelines when visually faithful, detail-preserving upsampling is important.

\subsubsection{Mathematical Reasoning}

\noindent Bicubic interpolation extends bilinear interpolation by introducing a \textbf{cubic weighting function} that smoothly distributes the contribution of each neighboring pixel. While bilinear interpolation assigns weights based purely on distance (linearly decreasing to zero), the cubic approach tailors these weights using a function that decays gradually, allowing pixels farther from the target position to still have a small but meaningful influence.

\noindent The commonly used weighting function is piecewise-defined:

\[
W(t) =
\begin{cases}
	(a + 2)|t|^3 - (a + 3)|t|^2 + 1, & 0 \leq |t| < 1, \\
	a|t|^3 - 5a|t|^2 + 8a|t| - 4a, & 1 \leq |t| < 2, \\
	0, & |t| \geq 2,
\end{cases}
\]

\noindent where \(a\) typically takes values around \(-0.5\) to balance smoothness and sharpness. The function ensures nearby pixels carry the most weight, while more distant neighbors still contribute smoothly rather than being abruptly excluded.

\noindent A concise visual and conceptual explanation can be found in this 
\href{https://www.youtube.com/watch?v=poY_nGzEEWM&ab_channel=Computerphile}{\textbf{Computerphile video}}.

\newpage

\subsubsection{Bicubic Interpolation: Generalized Case}

\noindent Assume we have an input feature map \(\mathbf{I}\) of size \(C \times H \times W\), and we wish to produce an upsampled map \(\mathbf{I'}\) of size \(C \times H' \times W'\). The bicubic interpolation proceeds as follows:

\begin{enumerate}
	\item \textbf{Coordinate Mapping:}  
	Map the output pixel location \((x',y')\) back to the corresponding floating-point coordinate \((x,y)\) in the input grid:
	\[
	x = \frac{x'(W - 1)}{W' - 1}, 
	\quad
	y = \frac{y'(H - 1)}{H' - 1}.
	\]
	
	\item \textbf{Neighbor Identification:}  
	Determine the \(\pm1\) and \(\pm2\) offsets around \(\lfloor x \rfloor\) and \(\lfloor y \rfloor\). This yields a \(4 \times 4\) set of pixels \(\{I_{i,j}\}\) centered near \((x,y)\). 
	
	\item \textbf{Applying the Cubic Weights:}  
	Use the cubic function \(W(t)\) in both the \(x\) and \(y\) directions:
	\[
	I'(x', y') = \sum_{i=-1}^{2}\sum_{j=-1}^{2} 
	W(x - x_i)\,W(y - y_j)\,I_{i,j}.
	\]
\end{enumerate}

\begin{figure}[H]
	\centering
	\includegraphics[width=0.8\textwidth]{Figures/Chapter_15/slide_49.jpg}
	\caption{Bicubic interpolation demonstrated on a \(C \times 2 \times 2\) feature map, generating a \(C \times 4 \times 4\) output. Each interpolated value is computed by applying a cubic weighting to the nearest 16 pixels.}
	\label{fig:chapter15_bicubic_interpolation}
\end{figure}

\subsubsection{Advantages and Limitations}

\noindent \textbf{Sharper Details and Continuity.} By sampling a larger neighborhood with a smoothly decaying weight function, bicubic interpolation preserves finer structures, reduces artifacts, and transitions more smoothly across pixel boundaries than bilinear interpolation.

\noindent \textbf{Better Texture Preservation.} Rather than over-smoothing, bicubic interpolation better maintains texture information by assigning fractional influences to pixels farther than one unit away.

\noindent \textbf{Non-Learnable.} Despite these benefits, bicubic interpolation remains a fixed formula that cannot adapt to complex or domain-specific feature distributions in deep learning.

\newpage

In contrast, max unpooling or learnable upsampling layers (we'll learn about those in the following parts) can dynamically capture where and how to upscale feature maps.

\noindent Hence, while bicubic interpolation offers a clear advantage over simpler methods for image resizing tasks, its fixed nature can be sub-optimal in end-to-end neural networks that require trainable, context-dependent upsampling.

\subsection{Max Unpooling}
\label{subsec:chapter15_max_unpooling}

\noindent
\textbf{Max unpooling} is an upsampling operation designed to ``invert'' max pooling as faithfully as possible. Instead of \emph{estimating} new values via interpolation (as in bilinear or bicubic upsampling), max unpooling is a \emph{routing} mechanism: it uses the \textbf{indices of the maxima} recorded during max pooling to place activations back into their original spatial locations, producing a sparse but geometrically aligned feature map.

\noindent
Intuitively, max unpooling acts like a memory of where the network believed the most important responses were before downsampling. During encoding, max pooling keeps only the largest activation in each window and remembers \emph{where} it came from. During decoding, max unpooling re-expands the feature maps and reinstates those activations exactly at the stored positions, filling all other locations with zeros. This preserves the encoder’s notion of ``where things are'' while deferring dense reconstruction to subsequent convolutions.

\subsubsection{Max Unpooling in the DeconvNet of Noh et al.\ (ICCV 2015)}

\noindent
In the DeconvNet architecture proposed by Noh et al.\ \cite{noh2015_deconvnet}, max unpooling layers are placed symmetrically to the max pooling layers of the encoder. Each pooling layer performs:
\begin{itemize}
	\item \textbf{Max pooling with switches:} For each pooling window (e.g., \(2 \times 2\) with stride 2), the encoder selects the maximum activation and stores its \textbf{index} (row and column position) inside the window.
\end{itemize}

\noindent
The corresponding max unpooling layer in the decoder then executes three conceptually simple steps:
\begin{enumerate}
	\item \textbf{Re-expand the spatial grid:} The decoder allocates an upsampled feature map with the same spatial resolution as the pre-pooled feature map.
	\item \textbf{Place activations using indices:} Each pooled activation is written back into the upsampled grid at the exact location indicated by its recorded index; all other positions in that pooling window are set to zero.
	\item \textbf{Refine sparsity via convolutions:} This sparse, index-aligned map is passed through convolutional layers that propagate information from strong activations into nearby zero regions, gradually reconstructing dense feature maps and, ultimately, a segmentation mask.
\end{enumerate}

\noindent
This encoder–decoder symmetry has two important effects:
\begin{itemize}
	\item It \textbf{preserves spatial correspondence} between encoder and decoder: high-level features in the decoder are anchored to the same image regions where they were originally detected.
	\item It provides a \textbf{structured scaffold} for reconstruction: strong activations sit at semantically meaningful positions (edges, parts, object interiors), and subsequent convolutions learn to fill in the details around them.
\end{itemize}

\begin{figure}[H]
	\centering
	\includegraphics[width=0.8\textwidth]{Figures/Chapter_15/slide_50.jpg}
	\caption{Illustration of \textbf{max unpooling} using recorded pooling indices to restore spatial activations. Each max-pooled activation is returned to its original location, while all other positions in the window are set to zero}
	\label{fig:chapter15_max_unpooling}
\end{figure}

\subsubsection{Why Max Unpooling is More Effective Than Bed of Nails Unpooling}

\noindent
Both max unpooling and Bed of Nails unpooling produce sparse feature maps that are later densified by convolutions, but they differ crucially in \emph{where} activations are placed.

\paragraph{Spatial alignment versus arbitrary placement}
\noindent
\begin{itemize}
	\item \textbf{Bed of Nails unpooling} copies each activation from the low-resolution feature map into a fixed, predetermined location in the corresponding upsampled block (for example, always the top-left corner of a \(2 \times 2\) region), setting all other positions to zero. This ignores where the activation originally occurred inside the pooling window. As a result, features are systematically \emph{shifted} in space, breaking alignment between encoder and decoder.
	\item \textbf{Max unpooling}, by contrast, uses the stored pooling indices to place each activation back into its \emph{true} pre-pooled location. The sparse pattern therefore matches the geometry induced by the encoder, preserving object shapes, boundaries, and part locations as seen by the max-pooling layers.
\end{itemize}

\paragraph{Why zeros in max unpooling are less problematic}
\noindent
Both methods introduce many zeros, but their semantic meaning differs:
\begin{itemize}
	\item In \textbf{Bed of Nails unpooling}, zeros are inserted according to a fixed pattern that does not reflect the encoder’s decisions. They appear between activations even in regions where several pixels were originally moderately strong but not maximal. The decoder then receives an artificial ``checkerboard'' structure: a regular grid of isolated nonzeros surrounded by zeros, which can induce aliasing and unnatural high-frequency patterns unless later convolutions work hard to undo these artifacts.
	\item In \textbf{Max unpooling}, zeros appear precisely at positions that were \emph{not} selected by max pooling. In other words, they encode the fact that, in that local window, no feature exceeded the chosen maximum at those positions. 
	
	\newpage
	
	This matches the encoder’s notion of saliency: strong responses are re-instated where they originally occurred, while weaker or background responses are suppressed. Subsequent convolutions can therefore treat zeros as ``low-confidence'' or ``background'' rather than as artificial gaps; they naturally diffuse information outward from the high-activation sites, producing smooth, context-aware reconstructions.
\end{itemize}

\paragraph{Structured reconstruction}
\noindent
Because max unpooling respects the encoder’s spatial structure, the resulting sparse maps form a \textbf{data-driven blueprint} for reconstruction:
\begin{itemize}
	\item Edges and object parts are reintroduced at approximately correct locations, giving decoder convolutions a meaningful starting point.
	\item There is no need to learn to correct systematic misalignment (as with Bed of Nails); learning can instead focus on refining shapes, filling in missing detail, and resolving ambiguities.
\end{itemize}

\noindent
In summary, max unpooling remains a non-learnable upsampling operation, but by leveraging pooling indices it preserves the encoder’s spatial decisions. This makes it substantially more effective than Bed of Nails unpooling in fully convolutional decoders such as DeconvNet \cite{noh2015_deconvnet}, where accurate alignment between downsampling and upsampling stages is crucial for high-quality semantic segmentation.

\subsubsection{Bridging to Transposed Convolution}

\noindent \textbf{Max unpooling} restores spatial activations efficiently, but it lacks the ability to \textbf{generate new details} or refine spatial features dynamically. Since it is a purely index-driven process, it cannot adaptively reconstruct missing information beyond what was retained during \textbf{max pooling}.

\noindent To overcome these limitations, we now explore \textbf{transposed convolution}, a \textbf{learnable upsampling method} that optimizes filter weights to produce high-resolution feature maps. This allows for fine-grained spatial reconstructions and greater adaptability compared to fixed unpooling strategies.

\subsection{Transposed Convolution}
\label{chapter15_subsec:transposed_convolution}

\noindent \textbf{Transposed convolution}, also referred to as \textbf{deconvolution} or \textbf{fractionally strided convolution}, is an upsampling technique that enables the network to learn how to generate high-resolution feature maps from lower-resolution inputs. 

Unlike interpolation-based upsampling or max unpooling, which are fixed operations, transposed convolution is \textbf{learnable}, meaning the network optimizes the filter weights to improve the reconstruction process.

\noindent Although called \textit{deconvolution}, it is not an actual inversion of convolution. Instead, it follows a similar mathematical operation as standard convolution but differs in how the filter is applied to the input tensor.

\subsubsection{Understanding the Similarity to Standard Convolution}

\noindent In a standard \textbf{convolutional layer}, an input feature map is processed using a learned \textbf{filter (kernel)}, which slides over the input using a defined \textbf{stride}. At each step, the filter is multiplied element-wise with the corresponding input region, and the results are summed to produce a single output activation.

\noindent In \textbf{transposed convolution}, the process is similar but applied in reverse:
\begin{itemize}
	\item The filter is not applied directly to the input feature map but instead used to \textbf{spread} its contribution to the larger output feature map.
	\item Each input element is multiplied by every element of the filter, and the weighted filter values are then \textbf{copied} into the output tensor.
	\item If multiple filter applications overlap at the same location in the output, their values are \textbf{summed}.
\end{itemize}

\noindent This effectively reconstructs a higher-resolution representation while learning spatial dependencies in an upsampling operation.

\subsubsection{Step-by-Step Process of Transposed Convolution}

\noindent To illustrate how transposed convolution operates, consider a $2 \times 2$ input feature map processed with a $3 \times 3$ filter and a stride of 2, producing a $4 \times 4$ output. The process consists of the following steps:

\begin{enumerate}
	\item \textbf{Processing the First Element:}  
	\begin{itemize}
		\item The first input value is multiplied element-wise with each value in the $3 \times 3$ filter.
		\item The weighted filter response is then \textbf{placed} into its corresponding region in the output tensor, which was initially set to zeros.
	\end{itemize}
	
	\item \textbf{Processing the Second Element:}  
	\begin{itemize}
		\item The second input element undergoes the same multiplication with the filter, producing another set of weighted values.
		\item These values are positioned in the output grid according to the stride of 2.
		\item When regions of the output overlap due to filter applications, the corresponding values are \textbf{summed} instead of overwritten.
	\end{itemize}
	
	\item \textbf{Iterating Over the Remaining Elements:}  
	\begin{itemize}
		\item The process is repeated for all input elements, progressively constructing the upsampled feature map.
		\item The final reconstructed output is a $4 \times 4$ feature map, demonstrating how transposed convolution expands spatial resolution while preserving learned feature relationships.
	\end{itemize}
\end{enumerate}

\begin{figure}[H]
	\centering
	\includegraphics[width=0.8\textwidth]{Figures/Chapter_15/slide_59.jpg}
	\caption{Illustration of the first step in transposed convolution: applying the filter to the first input element.}
	\label{fig:chapter15_transposed_conv_first_step}
\end{figure}

\begin{figure}[H]
	\centering
	\includegraphics[width=0.8\textwidth]{Figures/Chapter_15/slide_61.jpg}
	\caption{The second input element is processed: its weighted filter values are placed in the output grid, with overlapping values summed.}
	\label{fig:chapter15_transposed_conv_second_step}
	
\end{figure}

\begin{figure}[H]
	\centering
	\includegraphics[width=0.8\textwidth]{Figures/Chapter_15/slide_62.jpg}
	\caption{Final constructed output after processing all input elements.}
	\label{fig:chapter15_transposed_conv_final_output}
\end{figure}

\subsubsection{1D Transposed Convolution}
\label{subsubsec:chapter15_transposed_conv_1d}

\noindent
A particularly clear way to build intuition for transposed convolution is to start from a simple \textbf{1D} example and view it as a ``scale, place, and sum'' operation. Consider a transposed convolution that maps a \textbf{2-element} input to a \textbf{5-element} output using a \textbf{3-element} kernel with stride \(S = 2\) and no padding.

\begin{itemize}
	\item \textbf{Input:} \(\mathbf{u} = [a,\, b]^\top\)
	\item \textbf{Kernel (filter):} \(\mathbf{k} = [x,\, y,\, z]^\top\)
	\item \textbf{Output:} \(\mathbf{v} \in \mathbb{R}^5\)
\end{itemize}

\noindent
The forward computation can be understood in three steps:

\paragraph{1. Scale and place each input element}
\noindent
For each input element, we multiply the entire kernel and \emph{place} the resulting block into the output at a location determined by the stride \(S\).

\begin{itemize}
	\item For the first input \(a\), we form
	\[
	a \cdot [x, y, z] = [ax,\, ay,\, az],
	\]
	and place it starting at the first output position:
	\[
	[ax,\, ay,\, az,\, 0,\, 0].
	\]
	
	\item For the second input \(b\), we again form
	\[
	b \cdot [x, y, z] = [bx,\, by,\, bz],
	\]
	but now place it \emph{shifted} by the stride \(S = 2\). This means its first element aligns with the third output position:
	\[
	[0,\, 0,\, bx,\, by,\, bz].
	\]
\end{itemize}

\paragraph{2. Sum overlapping contributions}
\noindent
The final output \(\mathbf{v}\) is the elementwise sum of these placed blocks:
\[
\mathbf{v}
=
\underbrace{[ax,\, ay,\, az,\, 0,\, 0]}_{\text{from } a}
+
\underbrace{[0,\, 0,\, bx,\, by,\, bz]}_{\text{from } b}
=
[ax,\, ay,\, az + bx,\, by,\, bz]^\top.
\]
The third position receives contributions from both \(a\) and \(b\), illustrating how transposed convolution blends neighboring inputs via overlapping kernel footprints.

\paragraph{3. Why 5 output elements? Role of stride}
\noindent
The output length is determined by the standard 1D transposed convolution formula (no padding):
\[
N_{\text{out}} = S \cdot (N_{\text{in}} - 1) + K,
\]
where
\(N_{\text{in}} = 2\) (input length),
\(K = 3\) (kernel size),
\(S = 2\) (stride). Thus,
\[
N_{\text{out}} = 2 \cdot (2 - 1) + 3 = 5.
\]
Intuitively, stride \(S = 2\) means that the two kernel ``footprints'' are placed two positions apart in the output, and each footprint spans \(K = 3\) elements, causing them to overlap in the middle.

\begin{figure}[H]
	\centering
	\includegraphics[width=0.8\textwidth]{Figures/Chapter_15/slide_64.jpg}
	\caption{Illustration of 1D transposed convolution with stride \(S=2\): a 2-element input and a 3-element filter produce a 5-element output via scale, place, and sum}
	\label{fig:chapter15_transposed_conv_1D}
\end{figure}

\noindent
In higher dimensions (e.g., 2D feature maps), exactly the same mechanism applies: each activation spreads its influence over a local neighborhood, shifted according to the stride, and overlapping contributions are summed to produce a larger, learned upsampled feature map.

\paragraph{Why use stride \(S>1\) in transposed convolutions?}
\noindent
In practice, choosing a stride \(S>1\) in a transposed convolution is precisely how we perform \emph{learnable upsampling} in a single layer. For a transposed convolution with stride \(S\), kernel size \(K\), padding \(P\), and 1D input length \(I\),
\[
O = (I - 1)\cdot S + K - 2P
\]
controls the output size. For example, \(S=2\) approximately doubles the spatial resolution, and \(S=4\) approximately quadruples it (up to boundary effects). This is why decoder architectures for semantic segmentation (e.g., U-Net, FCN-style models) or generators in GANs and super-resolution networks routinely use stride-2 (or larger) transposed convolutions: they efficiently map low-resolution feature maps back to higher resolutions while learning \emph{how} information should be distributed into the new pixels. Implementation-wise, a stride-\(S\) transposed convolution is equivalent to inserting \(S-1\) zeros between input positions and then applying a stride-1 convolution with the same kernel, but deep learning libraries realize this without explicitly constructing the enlarged, sparse intermediate tensor.

\subsection{Convolution and Transposed Convolution as Matrix Multiplication}
\label{subsec:chapter15_conv_matrix_form}

\noindent
Convolutions are linear operations and can always be written as matrix--vector products. This viewpoint is useful conceptually (it shows that convolution is just a special sparse linear map) and practically (it explains why the forward pass of a \emph{transposed convolution} corresponds to multiplying by the transpose of the convolution matrix, and why the backward pass of a standard convolution looks like a transposed convolution).

\newpage

\subsubsection{Standard Convolution via Matrix Multiplication}

\noindent
Consider a 1D convolution with stride \(S=1\) and no padding. Let

\begin{itemize}
	\item \textbf{Input:} \(\mathbf{x} = [x_1, x_2, x_3, x_4]^\top \in \mathbb{R}^{4}\).
	\item \textbf{Kernel (filter):} \(\mathbf{w} = [w_1, w_2, w_3]^\top \in \mathbb{R}^{3}\).
\end{itemize}

With valid convolution, the output has length
\[
O = I - K + 1 = 4 - 3 + 1 = 2,
\]
and its entries are
\[
y_1 = w_1 x_1 + w_2 x_2 + w_3 x_3, \qquad
y_2 = w_1 x_2 + w_2 x_3 + w_3 x_4.
\]

\noindent
We can write this as a matrix--vector product
\[
\mathbf{y} = C \mathbf{x},
\]
where \(C \in \mathbb{R}^{2 \times 4}\) is a Toeplitz matrix constructed from the kernel:
\[
C =
\begin{bmatrix}
	w_1 & w_2 & w_3 & 0 \\
	0   & w_1 & w_2 & w_3
\end{bmatrix}.
\]
Then
\[
C \mathbf{x} =
\begin{bmatrix}
	w_1 & w_2 & w_3 & 0 \\
	0   & w_1 & w_2 & w_3
\end{bmatrix}
\begin{bmatrix}
	x_1 \\ x_2 \\ x_3 \\ x_4
\end{bmatrix}
=
\begin{bmatrix}
	w_1 x_1 + w_2 x_2 + w_3 x_3 \\
	w_1 x_2 + w_2 x_3 + w_3 x_4
\end{bmatrix},
\]
which matches the convolution exactly.

\noindent
Each row of \(C\) encodes one position of the sliding kernel:

\begin{itemize}
	\item Row 1 aligns \([w_1, w_2, w_3]\) with \([x_1, x_2, x_3]\).
	\item Row 2 shifts this pattern one step to the right, aligning with \([x_2, x_3, x_4]\).
\end{itemize}

Positions that would fall outside the input are filled with zeros. In higher dimensions (2D images, 3D volumes) and with multiple channels, the same idea produces larger, block-structured Toeplitz matrices.

\begin{figure}[H]
	\centering
	\includegraphics[width=0.7\textwidth]{Figures/Chapter_15/slide_66.jpg}
	\caption{1D convolution represented as matrix multiplication \(\mathbf{y} = C \mathbf{x}\), where the Toeplitz matrix \(C\) is constructed from the kernel.}
	\label{fig:chapter15_conv_matrix_mul}
\end{figure}

\paragraph{Stride \(S>1\) in standard convolution}

\noindent
For stride \(S>1\), the convolution still has the form \(\mathbf{y} = C_S \mathbf{x}\) for a suitable sparse matrix \(C_S\). Intuitively, the kernel still slides along the input, but we only keep every \(S\)-th output. In matrix form, this corresponds either to:

\begin{itemize}
	\item Taking a subset of rows from the stride-1 Toeplitz matrix.
	\item Directly constructing a sparser matrix \(C_S\) whose rows correspond to windows starting at positions
	\[
	1,\; 1 + S,\; 1 + 2S,\; \dots
	\]
	in the input.
\end{itemize}

\subsubsection{Transposed Convolution as the Matrix Transpose}

\noindent
The \textbf{transposed convolution} associated with a given (discrete) convolution is most cleanly defined via the transpose of its convolution matrix. If a standard 1D convolution with stride \(S\) can be written as
\[
\mathbf{y} = C_S \mathbf{x},
\]
then its associated transposed convolution is the linear map
\[
\mathbf{x}' = C_S^\top \mathbf{y}.
\]
When \(C_S\) corresponds to a downsampling convolution (e.g., \(S > 1\)), this adjoint map typically \emph{increases} spatial extent, which is why transposed convolutions are used for upsampling.

\medskip

\noindent
For the stride-\(1\) example above, the convolution matrix is
\[
C =
\begin{bmatrix}
	w_1 & w_2 & w_3 & 0 \\
	0   & w_1 & w_2 & w_3
\end{bmatrix}
\in \mathbb{R}^{2 \times 4},
\]
so its transpose is
\[
C^\top =
\begin{bmatrix}
	w_1 & 0   \\
	w_2 & w_1 \\
	w_3 & w_2 \\
	0   & w_3
\end{bmatrix}
\in \mathbb{R}^{4 \times 2}.
\]
Given \(\mathbf{y} = [y_1, y_2]^\top\), the transposed convolution computes
\[
\mathbf{x}' = C^\top \mathbf{y}
=
\begin{bmatrix}
	w_1 y_1 \\
	w_2 y_1 + w_1 y_2 \\
	w_3 y_1 + w_2 y_2 \\
	w_3 y_2
\end{bmatrix}.
\]
Each element of \(\mathbf{y}\) is ``spread'' over three positions in \(\mathbf{x}'\), weighted by the kernel, and overlapping contributions are summed. For \(S=1\), both \(C\) and \(C^\top\) are Toeplitz matrices, so the adjoint is itself a \emph{normal} convolution (with a flipped kernel).

\subsubsection{Relating to the \([a, b]^\top\) and \([x, y, z]^\top\) Example (Stride \(S=2\))}

\noindent
We now connect the intuitive scale--place--sum example to the matrix view for stride \(S=2\). Consider a standard 1D convolution with:

\begin{itemize}
	\item \textbf{Input:} \(\mathbf{v} = [v_1, v_2, v_3, v_4, v_5]^\top\).
	\item \textbf{Kernel:} \(\mathbf{k} = [x, y, z]^\top\).
	\item \textbf{Stride:} \(S = 2\), no padding.
\end{itemize}

\noindent
The output \(\mathbf{u} = [u_1, u_2]^\top\) is
\[
u_1 = x v_1 + y v_2 + z v_3, \qquad
u_2 = x v_3 + y v_4 + z v_5,
\]
so the convolution matrix \(W \in \mathbb{R}^{2 \times 5}\) is
\[
W =
\begin{bmatrix}
	x & y & z & 0 & 0 \\
	0 & 0 & x & y & z
\end{bmatrix}.
\]
Each row corresponds to placing the kernel at positions \((1,2,3)\) and \((3,4,5)\) in the input, reflecting the stride \(S=2\).

\medskip

\noindent
The associated \emph{transposed convolution} uses \(W^\top\):
\[
W^\top =
\begin{bmatrix}
	x & 0 \\
	y & 0 \\
	z & x \\
	0 & y \\
	0 & z
\end{bmatrix}
\in \mathbb{R}^{5 \times 2}.
\]
Given a 2-element input \(\mathbf{u} = [a, b]^\top\), the transposed convolution computes
\[
\mathbf{v}' = W^\top \mathbf{u}
=
a
\begin{bmatrix}
	x \\ y \\ z \\ 0 \\ 0
\end{bmatrix}
+
b
\begin{bmatrix}
	0 \\ 0 \\ x \\ y \\ z
\end{bmatrix}
=
\begin{bmatrix}
	ax \\ ay \\ az + bx \\ by \\ bz
\end{bmatrix}.
\]
Thus the mapping
\[
[a, b]^\top
\;\xrightarrow{\;\text{kernel }[x,y,z]^\top,\; S=2\;}
[ax,\, ay,\, az+bx,\, by,\, bz]^\top
\]
is exactly the same 1D transposed convolution we described earlier, now written as a single matrix--vector product \(\mathbf{v}' = W^\top \mathbf{u}\).

\begin{figure}[H]
	\centering
	\includegraphics[width=0.8\textwidth]{Figures/Chapter_15/slide_68.jpg}
	\caption{Transposed convolution as the transpose of the convolution matrix: the forward map uses \(C_S^\top\) to spread each input activation over multiple output positions}
	\label{fig:chapter15_trans_conv_matrix_mul}
\end{figure}

\subsubsection{Strides, Upsampling, and the ``Normal Convolution'' Caveat}

\noindent
The matrix viewpoint is completely general: for any stride \(S\), both convolution and its adjoint remain \textbf{linear} maps and can always be written as
\[
\mathbf{y} = C_S \mathbf{x}, \qquad
\mathbf{x}' = C_S^\top \mathbf{y},
\]
for some (possibly large and sparse) matrix \(C_S\). This makes it clear that:

\begin{itemize}
	\item The operations are differentiable everywhere, with Jacobians given by \(C_S\) and \(C_S^\top\).
	\item Gradients with respect to inputs and kernels are just matrix--vector products involving these matrices or their transposes.
\end{itemize}

\noindent
However, there is an important subtlety when \(S>1\):

\begin{itemize}
	\item For \(\boldsymbol{S=1}\), the convolution matrix \(C\) is Toeplitz, and its transpose \(C^\top\) is also Toeplitz. In this case, both the forward convolution \(\mathbf{y} = C \mathbf{x}\) and the adjoint \(\mathbf{x}' = C^\top \mathbf{y}\) are \emph{normal convolutions} on the same grid, with different (flipped) kernels.
	\item For \(\boldsymbol{S>1}\), the forward convolution matrix \(C_S\) is still Toeplitz (up to zero rows corresponding to skipped positions), but its transpose \(C_S^\top\) is \emph{no longer Toeplitz}. As the stride example above shows, \(W^\top\) does not have constant diagonals, so there is no single kernel and stride configuration that realizes \(\mathbf{x}' = C_S^\top \mathbf{y}\) as a \emph{single standard convolution on the original input grid}.
\end{itemize}

\noindent
This is precisely the sense in which, for \(S>1\), a transposed convolution \emph{cannot be expressed as a normal convolution} acting directly on \(\mathbf{y}\): its matrix is not a convolution (Toeplitz) matrix on that grid. Instead, the usual implementation factorizes the operation into two steps:

\begin{enumerate}
	\item \textbf{Zero-insertion (upsampling).} Conceptually insert \(S-1\) zeros between consecutive elements of \(\mathbf{y}\), creating an enlarged, sparse feature map.
	\item \textbf{Stride-1 convolution.} Apply a \emph{normal} stride-1 convolution (with an appropriate kernel) to this upsampled signal.
\end{enumerate}

\noindent
On the \emph{upsampled} grid, the second step is again a standard convolution with a Toeplitz matrix. But on the original grid, the full operator is no longer a single convolution; it is the composition of upsampling (a fixed linear map) and a stride-1 convolution. Deep learning libraries implement transposed convolutions in exactly this way for efficiency, rather than explicitly forming \(C_S^\top\).

\medskip

\noindent
In summary:

\begin{itemize}
	\item Mathematically, for any stride \(S\), convolution and transposed convolution are linear maps with an adjoint relationship \(\mathbf{y} = C_S \mathbf{x}\), \(\mathbf{x}' = C_S^\top \mathbf{y}\).
	\item For \(S=1\), both maps are themselves normal convolutions on the same grid.
	\item For \(S>1\), the adjoint \(C_S^\top\) is \emph{not} a normal convolution on the original grid, but can be implemented as ``upsample (insert zeros) + stride-1 convolution'' on a finer grid.
\end{itemize}

\noindent
This clarifies why transposed convolutions with stride \(S>1\) are treated as a distinct primitive in modern libraries, even though they are still fully linear and differentiable and remain the exact adjoints of their corresponding strided convolutions.

\subsubsection{Advantages of Transposed Convolution}

\noindent
Relative to fixed upsampling operations such as bilinear interpolation or max unpooling, transposed convolution offers several advantages:
\begin{itemize}
	\item \textbf{Learnable weights:} The kernel parameters are trained end-to-end, allowing the network to learn how best to interpolate and refine details for the specific task.
	\item \textbf{Trainable spatial structure:} Because it is a convolution, the operation naturally captures local spatial patterns and can reconstruct sharp edges and meaningful structures rather than merely smoothing.
	\item \textbf{Flexible stride and padding:} As with standard convolutions, stride, kernel size, and padding provide fine-grained control over the output resolution, making it easy to design multi-scale encoder–decoder architectures.
\end{itemize}

\subsubsection{Challenges and Considerations}

\noindent While transposed convolution is highly effective, it introduces some challenges:

\begin{itemize}
	\item \textbf{Checkerboard Artifacts:} Overlapping filter applications can create unevenly distributed activations, leading to artifacts in the output.
	\item \textbf{Sensitivity to Stride and Padding:} Incorrect configurations can lead to distorted feature maps or excessive upsampling.
	
\end{itemize}

\subsection{Conclusion: Choosing the Right Upsampling Method}
\label{subsec:chapter15_upsampling_conclusion}

\noindent
In this chapter we examined several \textbf{upsampling and unpooling} strategies, ranging from simple, non-learnable schemes to fully learnable transposed convolutions. Each method makes a different trade-off between computational cost, spatial faithfulness, smoothness, and the ability to recover or hallucinate fine details. In practice, the “right” choice depends on the task (e.g., semantic segmentation vs.\ super-resolution), the downsampling operations used in the encoder (max pooling vs.\ strided convolutions), and the amount of computation and complexity you are willing to invest in the decoder.

\begin{table}[H]
	\centering
	\renewcommand{\arraystretch}{1.3}
	\begin{tabular}{|p{3.7cm}|p{4.8cm}|p{4.8cm}|}
		\hline
		\textbf{Upsampling Method} & \textbf{Advantages} & \textbf{Limitations} \\
		\hline
		\textbf{Nearest-Neighbor Unpooling / Upsampling} 
		& Extremely simple and fast; no learnable parameters; preserves exact values of input pixels or features 
		& Produces blocky, jagged artifacts; no notion of continuity; cannot reconstruct fine details or smooth transitions. \\
		\hline
		\textbf{Bed of Nails Unpooling} 
		& Simple non-learnable unpooling; preserves original values in fixed locations; keeps sparsity structure 
		& Places activations in arbitrary fixed positions (e.g., always top-left); breaks spatial alignment with the encoder; creates unnatural gaps and aliasing; generally inferior to max unpooling. \\
		\hline
		\textbf{Bilinear Interpolation} 
		& Fast, differentiable, and easy to implement; produces smooth transitions and avoids blocky artifacts 
		& Averages over local neighborhoods, which blurs edges and textures; cannot recover high-frequency details lost during downsampling. \\
		\hline
		\textbf{Bicubic Interpolation} 
		& Uses a larger neighborhood and cubic weights; typically sharper outputs and better detail preservation than bilinear 
		& More computationally expensive; still non-learnable and can introduce mild blurring or ringing near sharp boundaries. \\
		\hline
		\textbf{Max Unpooling} 
		& Restores activations to their exact locations recorded by max pooling; preserves spatial layout of salient features and encoder--decoder alignment 
		& Produces sparse feature maps (zeros in non-max positions) that require subsequent convolutions for refinement; only applicable when pooling indices are available. \\
		\hline
		\textbf{Transposed Convolution} 
		& Fully learnable upsampling; can reconstruct or hallucinate high-frequency structure; flexible control of output size through kernel, stride, and padding 
		& Higher computational cost; can introduce checkerboard artifacts if kernel size, stride, and padding are poorly chosen; more sensitive to implementation details. \\
		\hline
	\end{tabular}
	\caption{Summary of common upsampling and unpooling methods, highlighting their main advantages and limitations.}
	\label{tab:chapter15_upsampling_comparison}
\end{table}

\subsubsection{Guidelines for Choosing an Upsampling Method}

\noindent
The upsampling strategy should be chosen in concert with the encoder design and the target task. The following guidelines capture common patterns used in practice:

\begin{itemize}
	\item \textbf{Match the encoder’s downsampling when using max pooling.} \\
	When the encoder uses \textbf{max pooling}, \textbf{max unpooling} is a natural counterpart: it reuses the recorded pooling indices to place activations back into their original spatial locations. This preserves spatial correspondence between encoder and decoder feature maps. 
	
	\newpage
	
	Because the unpooled output is sparse, it should almost always be followed by one or more convolutional layers to “densify” and refine the feature map. In contrast, \textbf{Bed of Nails unpooling} does not respect the original pooling geometry and typically leads to misaligned features and artifacts, so it is best viewed as a simple didactic baseline rather than a practical choice.
	
	\item \textbf{Use interpolation when you want smooth, non-learnable upsampling.} \\
	For tasks where smoothness and simplicity are more important than exact detail reconstruction (or when a lightweight baseline is sufficient), \textbf{bilinear interpolation} is a robust default. It avoids blocky artifacts and is inexpensive. \textbf{Bicubic interpolation} is preferred when additional sharpness is desired and the extra cost is acceptable. In both cases, the upsampled features are often followed by a standard convolution layer to reintroduce some learnable flexibility.
	
	\item \textbf{Combine simple upsampling with convolution to avoid artifacts.} \\
	A widely used pattern in modern architectures is: \emph{resize (nearest-neighbor or bilinear) \(\rightarrow\) convolution}. The interpolation step handles the geometric upsampling, while the subsequent convolution learns to refine and reweight the features. This decoupled design avoids checkerboard artifacts associated with poorly configured transposed convolutions, yet retains learnable capacity through the convolutional layer.
	
	\item \textbf{Use transposed convolution when learnable upsampling is essential.} \\
	\textbf{Transposed convolutions} are often preferred in \textbf{semantic segmentation decoders, autoencoders, super-resolution networks, and GAN generators}, where the decoder must learn how to reconstruct or hallucinate fine details from compact representations. By choosing appropriate kernel sizes and strides (e.g., even kernel sizes and strides that match the encoder’s downsampling pattern), transposed convolutions can provide powerful, learnable upsampling. Careful design or additional smoothing (e.g., a small convolution after the transposed convolution) is recommended to mitigate checkerboard artifacts.
	
	\item \textbf{For encoders without explicit pooling, favor learned, structured upsampling.} \\
	In fully convolutional architectures that rely primarily on \textbf{strided convolutions} for downsampling, there are no pooling indices to reuse. In such cases, \textbf{transposed convolutions} or \textbf{interpolation + convolution} blocks provide a natural way to invert the spatial contraction, since they can be configured to mirror the encoder’s stride pattern and learn how to reconstruct structured high-resolution outputs.
\end{itemize}

\noindent
In summary, nearest-neighbor and Bed of Nails unpooling serve as simple baselines, interpolation methods provide smooth but non-learnable upsampling, and max unpooling plus transposed convolutions exploit encoder information or learnable filters to recover structure. Most practical decoders combine these ideas—using indices when available, interpolation when stability and simplicity matter, and learnable convolutions when detailed reconstruction is crucial.

\section{Instance Segmentation}

\noindent Instance segmentation is a critical task in computer vision that aims to simultaneously detect and delineate each object instance within an image. Unlike semantic segmentation, which assigns a class label to each pixel without distinguishing between different object instances of the same category, instance segmentation uniquely identifies each occurrence of an object. This is particularly important for applications where individual object identification is required, such as autonomous driving, medical imaging, and robotics.

\newpage

\noindent In computer vision research, image regions are categorized into two types: \emph{things} and \emph{stuff}. This distinction is fundamental to \textbf{instance segmentation}, where individual object instances are identified at the pixel level.

\begin{itemize}
	\item \textbf{Things:} Object categories that can be distinctly separated into individual instances, such as \emph{cars, people, and animals}.
	\item \textbf{Stuff:} Object categories that lack clear instance boundaries, such as \emph{sky, grass, water, and road surfaces}.
\end{itemize}

\noindent Instance segmentation focuses exclusively on \textbf{things}, as segmenting instances of \textbf{stuff} is not meaningful. The primary goal of instance segmentation is to \emph{detect all objects} in an image and assign a unique segmentation mask to each detected object, ensuring correct differentiation of overlapping instances.

\noindent This task is particularly challenging due to the need for accurate pixel-wise delineation while simultaneously handling object occlusions, varying scales, and complex background clutter. Advanced deep learning architectures, such as \textbf{Mask R-CNN}, have significantly improved the performance of instance segmentation by leveraging region-based feature extraction and mask prediction techniques. 

\noindent The development of instance segmentation models continues to evolve, driven by the increasing demand for high-precision vision systems across various domains.

\subsection{Mask R-CNN: A Two-Stage Framework for Instance Segmentation}
\label{subsec:chapter15_mask_rcnn}

\noindent \textbf{Mask R-CNN} extends \textbf{Faster R-CNN}, a widely used two-stage object detection framework, by incorporating a dedicated branch for per-instance segmentation masks. While Faster R-CNN predicts bounding boxes and class labels, Mask R-CNN further refines this process by generating high-resolution segmentation masks for each detected object.

\subsubsection{Faster R-CNN Backbone}

\noindent Faster R-CNN builds on a convolutional backbone (e.g., ResNet with or without FPN) that extracts a shared feature map for the entire image. On top of these features, a \textbf{Region Proposal Network (RPN)} predicts a set of candidate object bounding boxes (region proposals) together with objectness scores. For each proposal, features are cropped from the shared feature map (via RoI pooling or RoI Align) and passed through two parallel heads: a \textbf{classification head} that predicts the object category via softmax, and a \textbf{bounding box regression head} that refines the proposal coordinates via regression. This two-stage design yields class-labeled, refined bounding boxes and serves as the foundation for Mask R-CNN.

\subsubsection{Key Additions in Mask R-CNN}

\noindent Mask R-CNN preserves the overall Faster R-CNN structure while introducing two key modifications that enable instance-level segmentation:

\begin{itemize}
	\item \textbf{A mask prediction head.} A lightweight \textbf{fully convolutional network (FCN)} branch predicts a \textbf{binary segmentation mask} for each detected object instance. Instead of producing a single segmentation map for the whole image, Mask R-CNN outputs \textbf{one mask per region of interest (RoI)}. The mask head consists of several convolutional layers followed by a \textbf{deconvolution (transposed convolution)} layer that upsamples RoI features (e.g., from \(14 \times 14\) to \(28 \times 28\)) before a final \(1 \times 1\) convolution produces per-pixel mask logits. The weights of this head are learned jointly with the detection heads.
	
	\newpage
	
	\item \textbf{RoI Align for precise feature extraction.} Faster R-CNN originally used RoI Pooling, which quantizes RoI coordinates to discrete bins and introduces misalignment between the RoI and the underlying feature map. Mask R-CNN replaces this with \textbf{RoI Align}, which avoids any rounding and uses \textbf{bilinear interpolation} to sample feature values at exact (possibly fractional) locations. This improves alignment, especially for small objects, and is crucial for accurate mask boundaries.
\end{itemize}

\noindent As a result, the second stage of Mask R-CNN produces three parallel outputs for each region proposal:

\begin{itemize}
	\item \textbf{Class label}, predicted via a softmax classification head.
	\item \textbf{Bounding box refinement}, predicted by a regression head that outputs coordinate offsets.
	\item \textbf{Segmentation mask}, predicted by the FCN-based mask branch.
\end{itemize}

\subsubsection{Segmentation Mask Prediction: Fixed-Size Output}

\noindent A central challenge in instance segmentation is handling objects of widely varying sizes while keeping computation manageable. Mask R-CNN addresses this by predicting a \textbf{fixed-size mask} for each RoI and then resizing it to the object’s bounding box in the original image.

\noindent Concretely, for each positive RoI:

\begin{enumerate}
	\item The \textbf{RPN} generates region proposals on top of the backbone feature map.
	\item The \textbf{classification} and \textbf{bounding box regression} heads operate on RoI-aligned features to predict the object category and refine the bounding box coordinates.
	\item In parallel, the \textbf{mask head} takes the same RoI-aligned features and outputs a tensor of shape \(C \times 28 \times 28\), where \(C\) is the number of object classes. Each channel corresponds to a \textbf{class-specific} mask prediction at a fixed spatial resolution.
	\item During inference, the mask corresponding to the \textbf{predicted class} for that RoI is selected, yielding a single \(28 \times 28\) mask for that instance.
	\item This selected \(28 \times 28\) mask is then resized to the spatial extent of the refined bounding box using \textbf{bilinear interpolation} and placed at the appropriate location in the original image coordinate system.
\end{enumerate}

\noindent In other words, the transposed convolution inside the mask head learns to produce a relatively high-resolution, fixed-size mask in feature space, while a final bilinear interpolation step adapts this fixed-size mask to the object’s actual size in the input image.

\subsubsection{Training Mask R-CNN and Loss Functions}

\noindent Mask R-CNN is trained end-to-end as a \textbf{multi-task} model, jointly optimizing detection (classification and bounding boxes) and segmentation. The training objective is the sum of three losses:

\begin{itemize}
	\item \textbf{Classification loss} \(L_{cls}\). A standard softmax cross-entropy loss applied to the classification head to encourage correct object category predictions for each RoI.
	\item \textbf{Bounding box regression loss} \(L_{box}\). A smooth L1 loss applied to the predicted bounding box offsets for \textbf{positive} RoIs (those that sufficiently overlap a ground-truth object), improving localization accuracy.
	\item \textbf{Mask loss} \(L_{mask}\). A per-pixel binary cross-entropy loss applied to the mask prediction branch. For each positive RoI, this loss is computed \textbf{only on the channel corresponding to the ground-truth class}, ignoring all other class channels. This class-specific loss encourages accurate foreground–background separation and precise object boundaries.
\end{itemize}

\noindent The total loss is given by
\[
L = L_{cls} + L_{box} + L_{mask},
\]
where:
\begin{itemize}
	\item \(L_{cls}\) is the classification loss.
	\item \(L_{box}\) is the bounding box regression loss.
	\item \(L_{mask}\) is the mask prediction loss.
\end{itemize}

\noindent In practice, the backbone network (e.g., ResNet with or without FPN) is first pretrained on a large-scale image classification dataset such as ImageNet and then fine-tuned on an instance segmentation dataset such as COCO. During fine-tuning, gradients from all three heads (classification, box regression, and mask prediction) are backpropagated through the shared backbone and RPN. This joint optimization improves both detection (bounding box mAP) and segmentation (mask mAP), and the RoI Align plus mask head design enables accurate, high-resolution instance masks while reusing the mature Faster R-CNN detection pipeline.

\subsubsection{Bilinear Interpolation vs. Bicubic Interpolation}

\noindent The upsampling step in Mask R-CNN requires resizing segmentation masks to fit detected object regions. The authors chose \textbf{bilinear interpolation} over \textbf{bicubic interpolation} for the following reasons:

\begin{itemize}
	\item \textbf{Efficiency:} Bilinear interpolation is computationally less expensive than bicubic interpolation, making it suitable for processing multiple objects per image.
	\item \textbf{Minimal Accuracy Gains from Bicubic:} Bicubic interpolation considers 16 neighboring pixels, while bilinear uses only 4. Given that Mask R-CNN’s masks are already low resolution ($28 \times 28$), bicubic interpolation does not provide significant accuracy improvements.
	\item \textbf{Edge Preservation:} Bicubic interpolation introduces additional smoothing, which can blur object boundaries. Bilinear interpolation maintains sharper mask edges, improving segmentation performance.
\end{itemize}

\subsubsection{Class-Aware Mask Selection}

\noindent Unlike traditional multi-class segmentation models, which predict a single mask covering all categories, Mask R-CNN follows a \textbf{per-instance, per-class} approach:

\begin{itemize}
	\item The segmentation head predicts \textbf{C binary masks per object}, where $C$ is the number of possible classes.
	\item The classification head determines the object’s category.
	\item The corresponding mask for the predicted category is selected and applied to the object.
\end{itemize}

\noindent This method \textbf{decouples classification from segmentation}, preventing class competition within the mask and improving segmentation accuracy.

\subsubsection{Gradient Flow in Mask R-CNN}

\noindent Mask R-CNN’s forward pass for mask prediction closely mirrors the backward pass of standard convolutional networks. Gradient computations are structured as follows:

\begin{itemize}
	\item The \textbf{classification and bounding box losses} propagate through the detection pipeline, refining object proposals.
	\item The \textbf{segmentation loss} propagates gradients through the mask prediction branch, optimizing instance masks.
	\item \textbf{RoI Align} ensures spatial alignment, preventing gradient misalignment and improving mask accuracy.
\end{itemize}

\noindent Expressing these processes as matrix--vector operations clarifies how gradients flow through the network, aiding optimization and efficient deep learning framework implementation.

\subsubsection{Summary}

\noindent Mask R-CNN extends Faster R-CNN by introducing a \textbf{per-region mask prediction branch} and \textbf{RoI Align} for accurate feature extraction. The segmentation head predicts a \textbf{fixed-size} $28 \times 28$ binary mask per object, which is then resized using \textbf{bilinear interpolation}. This approach allows for accurate instance segmentation while maintaining computational efficiency, making Mask R-CNN a dominant framework in object segmentation applications.

\subsection{Extending the Object Detection Paradigm}

\noindent Mask R-CNN introduced a paradigm in which object detection models can be extended to perform new vision tasks by adding task-specific prediction heads. This flexible approach has led to the development of new capabilities beyond instance segmentation, such as:

\begin{itemize}
	\item \textbf{Keypoint Estimation:} Mask R-CNN was further extended for human pose estimation by adding a keypoint detection head. This variation, sometimes called \emph{Mask R-CNN: Keypoints}, predicts key locations such as joints in the human body, facilitating pose estimation.
	
	\begin{figure}[H]
		\centering
		\includegraphics[width=0.8\textwidth]{Figures/Chapter_15/slide_90.jpg}
		\caption{Mask R-CNN extended for keypoint estimation, predicting key locations such as joints for human pose estimation.}
		\label{fig:chapter15_keypoints}
	\end{figure}
	
	\item \textbf{Dense Captioning:} Inspired by the Mask R-CNN paradigm, \emph{DenseCap} \cite{johnson2015_densecap} extends object detection by incorporating a captioning head. This approach, illustrated below, uses an LSTM-based captioning module to describe detected regions with natural language. We'll cover this topic in depth later on. 
	
	\begin{figure}[H]
		\centering
		\includegraphics[width=0.8\textwidth]{Figures/Chapter_15/slide_93.jpg}
		\caption{Dense Captioning (DenseCap) extends object detection by adding a captioning head, enabling textual descriptions of detected objects.}
		\label{fig:chapter15_densecap}
	\end{figure}
	
	\begin{figure}[H]
		\centering
		\includegraphics[width=0.8\textwidth]{Figures/Chapter_15/slide_94.jpg}
		\caption{Example output of DenseCap: Generated captions describe detected regions with natural language.}
		\label{fig:chapter15_densecap_example}
	\end{figure}
	
	\item \textbf{3D Shape Prediction:} \emph{Mesh R-CNN} \cite{gkioxari2020_meshrcnn} builds upon Mask R-CNN to predict 3D object shapes from 2D images by adding a mesh prediction head. This enables the reconstruction of 3D object geometry directly from image-based inputs, representing a significant step toward vision-based 3D reasoning.
	
	\begin{figure}[H]
		\centering
		\includegraphics[width=0.8\textwidth]{Figures/Chapter_15/slide_97.jpg}
		\caption{Mesh R-CNN extends Mask R-CNN with a mesh prediction head, enabling 3D shape reconstruction from 2D images.}
		\label{fig:chapter15_mesh_rcnn}
	\end{figure}
\end{itemize}

\noindent These extensions highlight the versatility of the Mask R-CNN framework and demonstrate how object detection networks can serve as a foundation for diverse computer vision tasks. By incorporating additional task-specific heads, researchers continue to expand the boundaries of what can be achieved using a common underlying object detection architecture. We'll touch these ideas later on as well. 

\newpage
\begin{enrichment}[U-Net: A Fully Conv Architecture for Segmentation][section]
	\label{enr:chapter15_unet}
	
	\begin{enrichment}[Overview][subsection]
		
		\noindent \textbf{U-Net} \cite{ronneberger2015_unet} is a fully convolutional neural network designed for semantic segmentation, particularly in biomedical imaging. Unlike traditional classification networks, U-Net assigns a class label to each pixel, performing dense prediction. The architecture follows a \textbf{symmetrical encoder-decoder} structure, resembling a "U" shape. The encoder (contracting path) captures contextual information, while the decoder (expansive path) refines localization details.
		
	\end{enrichment}
	
	\begin{enrichment}[U-Net Architecture][subsection]
		\noindent U-Net consists of two key components:
		
		\begin{itemize}
			\item \textbf{Contracting Path (Encoder):} 
			\begin{itemize}
				\item Repeated \textbf{$3 \times 3$ convolutions} followed by \textbf{ReLU activations}.
				\item \textbf{$2 \times 2$ max-pooling} for downsampling, reducing spatial resolution while increasing feature depth.
				\item Captures high-level semantic information necessary for object recognition.
			\end{itemize}
			
			\item \textbf{Expansive Path (Decoder):} 
			\begin{itemize}
				\item \textbf{Transposed convolutions} for upsampling, restoring spatial resolution.
				\item \textbf{Skip connections} integrate feature maps from the encoder to retain spatial details lost during downsampling.
				\item A \textbf{$1 \times 1$ convolution} maps feature channels to the segmentation classes.
			\end{itemize}
		\end{itemize}
		
		\begin{figure}[H]
			\centering
			\includegraphics[width=0.9\textwidth]{Figures/Chapter_15/unet_architecture.png}
			\caption{U-Net architecture: The encoder (left) captures context, while the decoder (right) restores details using transposed convolutions and skip connections. Source: \cite{ronneberger2015_unet}.}
			\label{fig:chapter15_unet_architecture}
		\end{figure}
		
	\end{enrichment}
	
	\begin{enrichment}[Skip Connections and Concatenation][subsection]
		
		\noindent \textbf{Skip connections} are a key innovation in U-Net that directly link corresponding encoder and decoder layers through concatenation. This mechanism enables:
		
		\begin{itemize}
			\item \textbf{Preserving Spatial Information:} 
			\begin{itemize}
				\item Encoder feature maps are concatenated with decoder feature maps at corresponding levels.
				\item This ensures that fine-grained details lost due to downsampling are reinstated.
			\end{itemize}
			
			\item \textbf{Combining Semantic and Spatial Features:} 
			\begin{itemize}
				\item The encoder extracts abstract, high-level semantic features.
				\item The decoder restores fine details, and concatenation helps merge these representations.
			\end{itemize}
			
			\item \textbf{Enhancing Gradient Flow During Training:} 
			\begin{itemize}
				\item Skip connections allow gradients to propagate more easily through deep networks, preventing vanishing gradients.
				\item This improves convergence and stabilizes the training process.
			\end{itemize}
		\end{itemize}
		
		\noindent The concatenation operation is crucial, as it ensures that both low-level spatial features and high-level semantic features contribute to final pixel-wise classification.
		
	\end{enrichment}
	
	\begin{enrichment}[Training U-Net][subsection]
		
		\noindent U-Net is trained end-to-end in a supervised manner, typically using:
		
		\begin{itemize}
			\item \textbf{Loss Function:} 
			\begin{itemize}
				\item The standard loss function for U-Net is \textbf{Binary Cross-Entropy (BCE)} for binary segmentation tasks.
				\item For multi-class segmentation, \textbf{Categorical Cross-Entropy} is used.
				\item When dealing with imbalanced datasets, \textbf{Dice Loss} or a combination of BCE and Dice Loss is applied.
			\end{itemize}
			
			\item \textbf{Optimization:} 
			\begin{itemize}
				\item U-Net is typically trained using 
				\textbf{Adam} or 
				\textbf{Stochastic Gradient Descent (SGD)} with momentum.
			\end{itemize}
			
			\item \textbf{Data Augmentation:} 
			\begin{itemize}
				\item Given the limited availability of annotated medical data, U-Net heavily relies on augmentation techniques such as:
				\begin{itemize}
					\item Random rotations, flips, and intensity shifts.
					\item Elastic deformations to improve robustness.
				\end{itemize}
			\end{itemize}
		\end{itemize}
		
		\noindent The combination of skip connections, effective loss functions, and augmentation techniques ensures that U-Net achieves high accuracy even with limited training data.
		
	\end{enrichment}
	
	\begin{enrichment}[Comparison with Mask R-CNN][subsection]
		
		\noindent While both U-Net and Mask R-CNN perform segmentation, they differ in:
		
		\begin{itemize}
			\item \textbf{Task Type:} U-Net performs \textbf{semantic segmentation}; Mask R-CNN performs \textbf{instance segmentation}.
			\item \textbf{Architecture:} U-Net follows an \textbf{encoder-decoder} design, while Mask R-CNN uses a \textbf{two-stage detection-segmentation} approach.
			\item \textbf{Application Domains:} U-Net is dominant in \textbf{medical imaging and satellite imagery}, whereas Mask R-CNN excels in \textbf{object detection and video analytics}.
		\end{itemize}
		
	\end{enrichment}
	
	\begin{enrichment}[Impact and Evolution of U-Net][subsection]
		
		\noindent Since its introduction, U-Net has significantly influenced segmentation research, inspiring numerous adaptations and improvements:
		
		\begin{itemize}
			\item \textbf{U-Net++} \cite{zhou2018_unetpp}: Incorporates dense connections between encoder-decoder layers to improve gradient flow and feature reuse.
			\item \textbf{3D U-Net} \cite{cciccek2016_3dunet}: Extends the architecture to volumetric data, benefiting applications like MRI and CT scan analysis.
			\item \textbf{Residual U-Net} \cite{zhang2018_resunet}: Integrates residual blocks to enhance gradient flow and stabilize training for deeper architectures.
			\item \textbf{Hybrid U-Net Variants:} Many modern adaptations replace the convolutional backbone with newer architectures, such as vision transformers, to enhance feature extraction.
		\end{itemize}
		
		\noindent Although \textbf{Attention U-Net} \cite{oktay2018_attentionunet} introduces an attention mechanism to selectively focus on relevant features, we have not yet covered attention mechanisms in this course. However, the core U-Net structure remains effective even without attention mechanisms and is widely used in practice. With continuous enhancements, U-Net’s impact on segmentation research persists across various domains.
	\end{enrichment}
\end{enrichment}

\newpage

\begin{enrichment}[Striding Towards SOTA Image Segmentation][section]
	\label{enr:sec_chapter15_towards_sota_segmentation}
	\noindent\textbf{Foundational segmentation systems}
	By late 2025, modern segmentation has consolidated around two complementary families of models:
	
	\begin{itemize}
		\item \textbf{Promptable foundation models} (e.g., SAM, SAM~2, SAM~3) treat segmentation as answering \emph{queries} about an image or video. Given sparse prompts---originally points, boxes, and masks, and now increasingly text and visual exemplars---they return high-quality masks, largely independent of any fixed label taxonomy.
		\item \textbf{Universal task-trained transformers} (e.g., Mask2Former, Mask~DINO) treat segmentation as a \emph{closed-set prediction} problem. They are trained on a fixed label space and directly output semantic, instance, or panoptic predictions for all categories in that taxonomy.
	\end{itemize}
	
	\noindent
	Our focus in this section is on the first family. \emph{Segment Anything (SAM)}~\cite{kirillov2023_sam} reframed interactive segmentation as large-scale, \emph{promptable} inference: given geometric hints (points, boxes, or a coarse mask), the model predicts the corresponding object mask, independent of category names. Its capabilities are driven both by a transformer-based encoder--decoder and by the SA-1B data engine, which couples model proposals with large-scale human correction to produce over one billion high-quality masks. Extending this idea from still images to videos, \emph{SAM~2}~\cite{ravi2024_sam2} adds a lightweight \emph{streaming memory} that stores compact state across frames, enabling real-time propagation and interactive correction of masks over long videos; its data engine similarly scales from static images to large video corpora.
	
	\medskip
	\noindent Most recently, \emph{SAM~3}~\cite{carion2025_sam3} unifies this geometric precision with \emph{concept-level} understanding. Instead of relying on external detectors for text prompts (as in Grounding~DINO~$\to$~SAM-style pipelines), SAM~3 natively supports \emph{concept prompts}: short noun phrases (e.g., ``yellow school bus''), image exemplars, or combinations of both. The corresponding task, termed \emph{Promptable Concept Segmentation} (PCS), takes such prompts and returns segmentation masks and identities for all matching instances in images and videos. Architecturally, SAM~3 shares a vision backbone between an image-level detector and a memory-based video tracker, and introduces a \emph{presence head} that decouples recognition (``is this concept present here?'') from localization, improving open-vocabulary detection and tracking. In the remainder of this subsection we will treat the SAM family (SAM, SAM~2, SAM~3) as canonical examples of promptable segmentation; later sections return to SAM~2 and SAM~3 in more architectural detail.
	
	\medskip
	\noindent In parallel, a second line of work focuses on \emph{task-specific}, closed-world performance. \emph{Universal transformers} such as Mask2Former~\cite{cheng2022_mask2former} and Mask~DINO~\cite{li2022_maskdino} (covered later in this chapter) are trained to jointly solve semantic, instance, and panoptic segmentation on a fixed label set (e.g., COCO, Cityscapes), typically achieving state-of-the-art mIoU/PQ when the deployment taxonomy matches the training one. Their outputs are directly aligned with benchmark metrics and do not require user prompts at inference time.
	
	\paragraph{Text-grounded segmentation: composite vs native}
	A third, closely related direction is \emph{text-grounded} segmentation. Before SAM~3, open-vocabulary segmentation typically relied on \emph{composite pipelines}. Systems such as Grounding~DINO~\cite{liu2023_groundingdino} or OWLv2~\cite{minderer2024_owlvitv2} first performed \emph{grounding}---mapping text prompts to boxes and labels---and then SAM or SAM~2 converted those boxes or points into precise masks. This pattern, often referred to as \emph{Grounded~SAM}~\cite{ren2024_groundedsam}, explicitly splits the problem into two stages: (1) a vision--language detector for text-to-box grounding, and (2) a promptable segmenter for box-to-mask refinement.
	
	Conceptually, this brings us full circle to the two-stage design of classical detectors such as Mask R-CNN~\cite{he2017_maskrcnn}. There, a Region Proposal Network (RPN) first generates category-agnostic boxes, and a second-stage head turns each box into class scores and a binary mask. Grounded~SAM follows the same high-level pattern---``boxes first, masks second''---but with a crucial difference in \emph{scale and modularity}. Instead of a single backbone with lightweight heads, it \emph{chains two large foundation models}: a vision--language detector (Grounding~DINO/OWLv2) and a high-capacity segmenter (SAM/SAM~2). This is attractive from an engineering perspective, because each component can be trained, deployed, and upgraded independently, but it also means that a single input triggers two expensive forward passes and two sets of model weights.
	
	SAM~3 alters this landscape by internalizing much of the grounding functionality. Through \emph{concept prompts} and the PCS objective, it allows users to query directly for ``all instances of \emph{red baseball cap}'' or ``all objects that look like this exemplar patch'' and obtain masks and tracks without a separate grounding detector. Architecturally, SAM~3 still has a logical detector-plus-mask-head structure, but both pieces share a joint vision--language backbone and are trained end-to-end on phrase-level supervision. As a result, text, exemplars, boxes, and masks are all expressed in a single representation, rather than stitched together across separate models. Composite pipelines remain valuable---for example, when reusing an existing detector stack, when detector outputs must be logged and audited as first-class artifacts, or when a legacy detection system already dominates the deployment budget---but SAM~3 offers a simpler, native alternative for language- and exemplar-driven segmentation.
	
	\paragraph{Deployment landscape: late 2025}
	In practice, practitioners now choose among three main paradigms, depending on their constraints and goals.
	
	\begin{itemize}
		\item \textbf{Closed-world baselines (Mask2Former/Mask~DINO).} For applications with a stable label set (e.g., urban-scene semantics, COCO-style panoptic segmentation, product taxonomies), Mask2Former and Mask~DINO remain the default production choices. They directly optimize mIoU, PQ, and AP under fixed evaluation protocols and require no prompts at inference time. In such workflows, SAM-family models are primarily used as \emph{annotation accelerators}: they speed up dataset creation (especially on video) and help human annotators correct systematic failure modes.
		\item \textbf{Composite grounded pipelines (Grounding~DINO $\to$ SAM).} For open-vocabulary scenarios where \emph{modularity} is paramount, the classic Grounding~DINO~$\to$~SAM/SAM~2 pattern dominates. The detector owns responsibility for text-to-box grounding, while SAM refines each box into a high-quality mask. This effectively recreates a two-stage Mask-R-CNN-style architecture, but with two heavy backbones instead of one, offering fine-grained control over intermediate box outputs and making it easy to swap in new detectors without retraining the segmenter.
		\item \textbf{Native concept models (SAM~3).} SAM~3 represents the unified frontier: it accepts multimodal prompts (points, boxes, masks, short text, visual exemplars) and outputs concept-conditioned instance masks and trajectories in a single forward pass. This simplifies deployment in settings where a single, unified model for concept-level segmentation and tracking is preferable to a modular detector+segmenter stack, and where tight coupling between grounding and segmentation is beneficial.
	\end{itemize}
	
	\newpage
	
	As helpful complements, universal transformers such as OneFormer~\cite{jain2023_oneformer} and X-Decoder/SEEM-style models~\cite{zou2022_xdecoder} broaden the closed-world trend by training a single model for multiple segmentation tasks (semantic, instance, panoptic, referring expression), while HQ-SAM~\cite{ke2023_hqsam} and related variants refine SAM’s boundary quality when fine detail (e.g., hair, thin structures) is critical.
	
	\paragraph{When to prefer specific-task training}
	The decision between a generic promptable model and a task-trained specialist is driven more by deployment constraints than by raw accuracy in isolation. Task-specific supervised training with Mask2Former/Mask~DINO (plus domain-curated data and augmentations) is usually preferred if your system has:
	
	\begin{itemize}
		\item \textbf{A stable, audited label space.} Classes are fixed, owned by QA/compliance, and changes require formal review.
		\item \textbf{Strict quantitative targets.} You must meet or exceed specific thresholds on mIoU, PQ, or AP under a benchmark-style protocol.
		\item \textbf{Non-trivial domain shift or sensing quirks.} Examples include medical imaging, remote sensing, night/rain conditions, or unusual optics (fisheye, industrial microscopes).
		\item \textbf{High-stakes boundary quality.} Small localization errors are unacceptable, as in defect inspection, surgical margin estimation, or metrology.
	\end{itemize}
	
	In these regimes, the common pattern is to use SAM-family models upstream to \emph{create and refine labels} quickly---especially on video, where SAM~2 and SAM~3’s memory-based tracking can amortize annotator effort---and then to distill or fine-tune a universal model on this curated dataset for reliable, closed-world deployment. Open-vocabulary grounding (via Grounding~DINO or SAM~3’s concept prompts) can then be added selectively for exploration, discovery, or monitoring wherever text-driven queries are genuinely needed.
	
	\paragraph{Example: defect inspection workflow}
	Consider an automated optical inspection (AOI) pipeline in a factory with a fixed set of surface-defect classes (scratch, dent, burr, contamination).
	
	\begin{itemize}
		\item \textbf{Phase 1: Discovery and data collection with SAM~3.} Engineers use SAM~3 interactively on short video bursts from the production line. Instead of clicking every defect manually, they prompt with phrases such as ``scratch'' and ``dent'' or provide a few reference patches for each defect type. SAM~3 segments and tracks all matching instances across frames, using its memory to handle motion and occlusions. Annotators only correct failure cases or ambiguous regions.
		\item \textbf{Phase 2: Training a universal model.} The resulting masks and labels form a high-quality, domain-specific dataset at relatively low labeling cost. A Mask2Former or Mask~DINO model is then fine-tuned on this dataset, learning the plant’s exact optics, materials, and defect appearances. At deployment, this universal model runs efficiently on the fixed taxonomy and directly optimizes PQ and edge tolerances under the factory’s evaluation protocol.
		\item \textbf{Phase 3: Fallback and extension.} SAM~3 (and, when needed, a Grounding~DINO~$\to$~SAM~2 pipeline) remains available as an interactive backup. It is used to investigate new defect types, analyze corner cases that the closed-set model mis-handles, and rapidly extend the dataset whenever the defect taxonomy is updated.
	\end{itemize}
	
	\newpage
	
	\begin{enrichment}[SAM: Segment Anything Model][subsection]
		
		\label{enr:subsec_chapter15_sam}
		
		\paragraph{Background}
		Classical segmentation approaches such as U-Net~\cite{ronneberger2015_unet} and Mask R-CNN~\cite{he2017_maskrcnn} are trained for fixed, \emph{closed-set} tasks: they assume a pre-defined label space, require costly pixel-accurate masks for each class, and directly predict both \emph{what} to segment (which categories) and \emph{how} to delineate them (pixel masks). This tight coupling between model, dataset, and taxonomy makes adaptation to new domains (e.g., medical, satellite, or artistic images) expensive, and offers little flexibility at inference time to specify \emph{which} particular object in a scene should be segmented. Segment Anything (SAM)~\cite{kirillov2023_sam} breaks this pattern by reframing segmentation as a \emph{promptable} task: a user or another system supplies lightweight prompts (points, boxes, or coarse masks), and the model returns high-quality object masks in real time. SAM is trained both as a segmentation model and as a large-scale annotation engine, powering the SA-1B dataset (1.1B masks over 11M images) that in turn supports open-set behavior. Architecturally, SAM relies on Vision Transformers and MAE-style self-supervised pretraining introduced later in this book (Chapters~17--18 for ViTs and Chapter~21 for self-supervised pretraining); only the essentials are summarized here, and it is useful to revisit this section after those chapters.
		
		\begin{figure}[H]
			\centering
			\includegraphics[width=0.85\textwidth]{Figures/Chapter_15/SAM_idea.jpg}
			\caption{\textbf{Task, model, and data engine}. A promptable segmentation task, a model (SAM) supporting interactive and zero-shot use, and a data engine that scales mask collection to SA-1B; credit: Kirillov \emph{et~al.}~\cite{kirillov2023_sam}.}
			\label{fig:chapter15_sam_idea}
		\end{figure}
		
		\paragraph{Core idea, task, and motivation}
		SAM treats segmentation as answering a generic, prompt-conditioned query rather than predicting a fixed set of semantic classes. Given an image \(I\) and a prompt \(P\), the model outputs multiple candidate masks and associated quality scores,
		\begin{equation}
			f_\theta:\ \langle I,\,P\rangle \longmapsto \big(\{m^{(k)}\}_{k=1}^{K},\,\{\hat{s}_k\}_{k=1}^{K}\big),
			\label{eq:chapter15_sam_prompt}
		\end{equation}
		where \(P\) may be a foreground/background point, an axis-aligned box, or a coarse mask; \(\{m^{(k)}\}\) are binary mask hypotheses; and \(\{\hat{s}_k\}\) are predicted IoUs used for ranking or automatic selection. In this formulation, prompts externalize the \emph{intent} (which object in the scene?), while SAM specializes in \emph{delineation} (where exactly is its boundary?).
		
		This decoupling directly addresses the limitations of classical detectors and segmenters, which must jointly decide \emph{what} and \emph{where} from a fixed label set. In closed-set models, anything outside the training taxonomy is effectively ``unknown'', and adding a new category requires collecting dense masks and retraining. In SAM, intent is supplied externally: detectors, text-grounding models, or simple heuristics propose regions (boxes or points), and SAM upgrades them to precise masks. By \emph{not} baking a semantic label space into the segmentation module, SAM becomes a reusable, label-agnostic \emph{mask engine} that composes with many upstream systems.
		
		Three design pillars underpin this formulation:
		\begin{enumerate}
			\item \textbf{Encode once, decode many.} A large ViT encoder, pre-trained as a masked autoencoder, computes a dense image embedding once per image and caches it. Subsequent prompts reuse this embedding, so only a lightweight decoder is invoked per query, enabling millisecond-level interactive updates.
			\item \textbf{Promptable, open-set task.} Prompts supply ``which thing?'' without class labels, allowing SAM to focus on a largely class-agnostic notion of \emph{segmentable objects}: regions with closed boundaries, coherent parts, and consistent appearance. This makes the task naturally open-set and suitable for zero-shot transfer across many, though not all, domains.
			\item \textbf{Ambiguity awareness.} A single prompt is often ambiguous (e.g., a click on a torso could mean shirt, person, or crowd). SAM therefore predicts several plausible masks \(\{m^{(k)}\}\) and scores them with \(\{\hat{s}_k\}\), so a user or system can select or refine the hypothesis that best matches intent instead of averaging incompatible solutions.
		\end{enumerate}
		
		\paragraph{Architecture and SA-1B data engine}
		These ideas are realized through a ViT-based architecture coupled with a self-bootstrapping data engine:
		\begin{itemize}
			\item \textbf{Image and prompt encoders; mask decoder.} A large ViT image encoder (e.g., ViT-H) produces a coarse but rich embedding \(E \in \mathbb{R}^{H/64 \times W/64 \times C}\) once per image and caches it. A prompt encoder converts points (2D coordinates with a foreground/background flag), boxes (corner coordinates), or downsampled masks into a small set of prompt tokens. A transformer-based mask decoder then fuses prompt tokens with \(E\) to produce \(K\) mask logits and their predicted IoU scores in tens of milliseconds on a modern GPU. During training, a \emph{min-over-masks} objective matches only the best predicted mask in the set to the ground truth, encouraging the hypotheses to cover typical whole/part/subpart interpretations instead of collapsing to a single averaged mask.
			
			\item \textbf{SA-1B via a three-stage data engine.} To support broad, open-set behavior, SAM is trained on SA-1B, a web-scale corpus of \(\sim\)1.1B masks over 11M licensed images. This dataset is constructed by an iterative data engine:
			\begin{enumerate}
				\item \emph{Assisted manual phase.} Human annotators use early SAM variants as interactive tools to draw high-quality masks, seeding the dataset.
				\item \emph{Semi-automatic phase.} As SAM improves, it proposes masks given simple prompts (e.g., boxes), and annotators mainly verify or lightly correct them, dramatically increasing throughput.
				\item \emph{Automatic phase.} A strong SAM model runs in a ``segment everything'' mode: a grid of prompts across each image yields candidate masks that are filtered and deduplicated automatically, adding hundreds of millions of masks with minimal human effort.
			\end{enumerate}
			The result is a diverse collection of masks for objects, stuff, and parts, providing the coverage needed to learn a broad, class-agnostic notion of objectness.
		\end{itemize}
		
		\paragraph{Zero-shot prompting, interaction, and ambiguity}
		Because prompts supply intent, SAM can often generalize to new domains without fine-tuning~\cite{kirillov2023_sam}. Pretraining on SA-1B induces a class-agnostic sense of objectness (closed contours, part--whole structure, texture and contrast cues). At inference time, prompts are encoded as tokens that condition the decoder, which attends jointly to these tokens and the cached image embedding \(E\).
		
		A typical interactive loop is:
		\begin{enumerate}
			\item \textbf{Initial prompt.} Start with a positive click near the interior of the target object or a loose box around it.
			\item \textbf{Select a hypothesis.} Inspect the small set of returned masks; pick the one that best matches intent, often simply the highest-\(\hat{s}_k\) mask. A low maximum IoU signals that more guidance is needed.
			\item \textbf{Refine with sparse feedback.} If the mask \emph{misses} a region, add a positive click in the missing area; if it \emph{leaks} into background or neighboring objects, add a negative click there. Re-running the decoder with updated prompts refines the mask while reusing the same image embedding.
			\item \textbf{Accept or reuse.} Once satisfactory, the mask is accepted as the final output or reused as a dense prompt to further tighten boundaries.
		\end{enumerate}
		In practice, prompts often induce a natural hierarchy of hypotheses: a whole object, a coherent part (e.g., clothing), and a finer subpart (e.g., a logo). The min-over-masks training encourages SAM to populate this hierarchy rather than settle on a single compromise mask.
		
		\paragraph{Applications, limitations, and fine-tuning}
		SAM's promptable, ambiguity-aware design supports a wide range of workflows:
		\begin{itemize}
			\item \textbf{Biomedical pathology.} On high-resolution tiles (e.g., \(2048{\times}2048\) at \(20{\times}\)), a positive click inside a lesion yields whole/part/subpart masks (e.g., lesion core vs.\ lesion+halo). A few positive/negative clicks typically suffice to obtain high-quality lesion contours despite scanner and stain shifts.
			\item \textbf{Remote sensing.} A coarse box around a city block can be refined into masks that follow roof footprints rather than roads or vegetation; in ``segment everything'' mode, a grid of prompts plus IoU-based filtering and non-maximum suppression yields instance masks that can be polygonized for GIS layers.
			\item \textbf{Creative photo/video editing.} A click on hair produces masks at different granularity (entire person, hair-only). After selecting and lightly refining the hair-only mask, one can generate high-quality alpha mattes for recoloring or compositing.
			\item \textbf{Robotics and 3D perception.} Detectors provide coarse boxes; SAM upgrades them to precise instance masks, which are then used to compute silhouettes and principal axes for grasp planning, or to associate 2D regions with depth measurements in a 3D pipeline.
			\item \textbf{Document layout and UI parsing.} Positive clicks on text blocks or UI elements produce tight component masks that can be vectorized into regions for OCR, reading-order inference, or accessibility tools, avoiding brittle, hand-crafted heuristics.
		\end{itemize}
		
		Despite its strong zero-shot performance on many natural-image-like domains, SAM is not a magic solution for all settings. Its notion of objectness is learned from SA-1B, which is still biased toward web imagery.
		
		\newpage
		
		In highly specialized or ``weird'' domains (e.g., certain medical modalities, industrial inspection, or non-optical sensors), zero-shot performance can be suboptimal, and practitioners routinely fine-tune SAM or adapt its lightweight components (e.g., via LoRA-style adapters or decoder fine-tuning) on a modest number of in-domain masks. In such cases, SAM should be viewed as a segmentation \emph{foundation model}: it provides a strong, promptable starting point that substantially reduces annotation and training costs, but high-stakes applications may still require domain-specific adaptation and careful evaluation.
		
		\noindent\emph{Summary.} A unified prompt-conditioned interface \(\langle I,P\rangle \!\to\!\) masks, an encode-once/decode-many architecture for low-latency interaction, and a web-scale mask corpus together yield a segmentation foundation model that generalizes widely without task-specific retraining, serves as a fast interactive tool across domains, and can be further fine-tuned where necessary to meet stringent domain-specific requirements.
		
		\begin{figure}[H]
			\centering
			\includegraphics[width=0.75\textwidth]{Figures/Chapter_15/SAM_point_mask_examples.jpg}
			\caption{\textbf{Ambiguity-aware outputs.} Each \emph{column} shows three valid masks produced by SAM from a \emph{single point prompt} (green dot). \emph{Rows:} top = \textit{whole} object, middle = \textit{part}, bottom = \textit{subpart}. The examples illustrate hierarchical ambiguity under the same cue (e.g., person$\rightarrow$backpack$\rightarrow$pocket; bird$\rightarrow$torso$\rightarrow$head). SAM proposes multiple hypotheses ranked by a predicted IoU, enabling the user or downstream code to select or refine the intended extent. Credit: Kirillov \emph{et\,al.}~\cite{kirillov2023_sam}.}
			\label{fig:chapter15_sam_point_ambiguity}
		\end{figure}
		
		\begin{figure}[H]
			\centering
			\includegraphics[width=0.85\textwidth]{Figures/Chapter_15/SAM_point_removal.jpg}
			\caption{\textbf{Add/remove refinement.} Starting from a full bear mask, a negative click removes the torso to retain only the head, illustrating part-focused refinement. Example created by interacting with the official demo at \href{https://segment-anything.com/}{segment-anything.com}.}
			\label{fig:chapter15_sam_point_removal}
		\end{figure}
		
		\noindent This interactive perspective sets up the detailed method next: SAM’s image encoder (a ViT pretrained via MAE~\cite{he2022_mae}), prompt encoder (including point/box encodings and dense mask prompts), two-way mask decoder with cross-attention, and training losses (focal + dice with min-over-masks). 
		
		\subsubsection{Method}
		\label{enr:subsubsec_chapter15_sam_method}
		
		\paragraph{Model overview and data flow}
		SAM follows an \emph{encode once, prompt many} design~\cite{kirillov2023_sam}. An input image (typically resized to $1024{\times}1024$) is passed once through a heavy \textbf{image encoder} to produce a cached dense embedding. At interaction time, a \textbf{prompt encoder} turns user intent (points, boxes, or a coarse mask) into compact tokens. A \textbf{lightweight mask decoder} then fuses prompt tokens with the cached image embedding via two-way attention and produces up to three candidate masks \emph{plus} a predicted IoU score to rank them. In interactive use, the newly accepted mask is fed back as a dense prompt for the next refinement step, forming a fast loop: encode image $\rightarrow$ decode mask(s) $\rightarrow$ add corrective prompt(s) $\rightarrow$ decode again, until satisfactory alignment.
		
		\begin{figure}[H]
			\centering
			\includegraphics[width=0.85\textwidth]{Figures/Chapter_15/SAM_overview.jpg}
			\caption{\textbf{SAM overview}. A heavyweight image encoder outputs a cached image embedding; a prompt encoder converts points/boxes/masks to tokens; a two-way transformer mask decoder fuses them to predict multiple candidate masks with IoU scores at interactive speed; credit: Kirillov \emph{et\,al.}~\cite{kirillov2023_sam}.}
			\label{fig:chapter15_sam_overview}
		\end{figure}
		
		\paragraph{Image encoder}
		The image encoder is a large Vision Transformer (ViT, e.g., ViT-H) initialized from MAE pretraining~\cite{he2022_mae}. MAE masks a high fraction of image patches and learns to reconstruct them from the visible ones, yielding strong, general-purpose visual features. Given a $1024{\times}1024$ input, the encoder produces a dense embedding on a lower-resolution grid (e.g., $64{\times}64$ tokens) that SAM projects to a channel dimension $C{=}256$ for efficient decoding~\cite{kirillov2023_sam}. This pass is amortized: it runs once per image and is reused for all subsequent prompts.
		
		\paragraph{Prompt encoder}
		SAM supports \emph{sparse} and \emph{dense} prompts in its official release; text enters only indirectly via external systems:
		
		\begin{itemize}
			\item \textbf{Sparse prompts}. Points are represented by their $(x,y)$ coordinates plus a learned \emph{type} embedding indicating foreground, background, or padding; boxes are represented by their two corners (top-left, bottom-right), each with positional encodings summed with a corner-type embedding~\cite{kirillov2023_sam}. These yield $d{=}256$-dimensional tokens compatible with the image embedding.
			\item \textbf{Dense prompts}. A coarse mask (e.g., a previous prediction) is downsampled and linearly projected to $d{=}256$, then \emph{added} to the image embedding so that subsequent decoding is conditioned on the prior mask~\cite{kirillov2023_sam}.
			\item \textbf{On text prompts}. The SAM paper defines prompts broadly and notes that, in principle, text embeddings (for example from CLIP~\cite{radford2021_clip}) could be injected as additional tokens into the prompt encoder. However, the publicly released SAM and SAM~2 models are trained and shipped with \emph{visual} prompts only (points, boxes, masks) and have no built-in text encoder or phrase-level segmentation supervision~\cite{kirillov2023_sam,ravi2024_sam2}. In practice, “text-prompted SAM” systems route text through a separate vision--language model (e.g., CLIP, Grounding~DINO, OWLv2) to produce boxes or points, which are then fed to SAM/SAM~2 as standard sparse prompts. Native, end-to-end concept-level text prompting is introduced only later in SAM~3 (covered in a subsequent part).
		\end{itemize}
		
		\newpage
		
		\paragraph{Positional encodings for 2D prompts}
		\label{enr:par_chapter15_sam_posenc}
		
		\noindent\textbf{Goal and constraint.} A prompt (point or box corner) is a \emph{continuous} image coordinate \((x,y)\). Its embedding should satisfy two geometric desiderata: (i) \emph{locality}: vectors for nearby points are similar and similarity decays with the Euclidean distance \(\|p-q\|_2\); (ii) \emph{isotropy}: the decay is direction-agnostic (no axis bias), and the mapping extrapolates to arbitrary resolutions and subpixel locations.
		
		\noindent\textbf{Why standard PEs fall short.} Absolute learned PEs in ViTs tie positions to a fixed grid index, hurting extrapolation to new resolutions. Separable 1D sinusoidal PEs~\cite{vaswani2017_attention} are continuous but \emph{anisotropic} in 2D: concatenating \(\mathrm{PE}_x(x)\) and \(\mathrm{PE}_y(y)\) yields similarities that drop faster along axes than along diagonals at the same \(\|p-q\|_2\), biasing attention and making mask boundaries “slip’’ along \(x\)/\(y\).
		
		\begin{figure}[H]
			\centering
			\includegraphics[width=0.70\textwidth]{Figures/Chapter_15/SAM_regular_PE_vs_Fourier.jpg}
			\caption{\textbf{Positional similarity: separable 1D PE vs.\ 2D random Fourier features.}
				Each heatmap shows the dot-product between the embedding at the center (origin) and all other grid locations.
				With \emph{separable 1D sinusoidal} PE (concatenating $x$-only and $y$-only sin/cos terms), iso-similarity
				contours are axis-aligned, producing anisotropy. We mark two points, $P_1$ (axis-aligned) and $P_2$
				(diagonal), chosen so that $\|P_1\|\!\approx\!\|P_2\|$; nevertheless
				$\langle \mathrm{PE}_{\text{1D-sep}}(0),\,\mathrm{PE}_{\text{1D-sep}}(P_1)\rangle
				\gg
				\langle \mathrm{PE}_{\text{1D-sep}}(0),\,\mathrm{PE}_{\text{1D-sep}}(P_2)\rangle$,
				i.e., $d(0,P_1)\!\approx\!d(0,P_2)$ but the embedding similarity differs markedly—an undesirable bias.
				In contrast, \emph{2D random Fourier features} (RFF) draw frequencies over the joint $(x,y)$ space,
				yielding near-isotropic similarity that decays primarily with Euclidean distance, so the center’s
				similarity to $P_1$ and $P_2$ is comparable. Inspired by~\cite{li2021_learnable_fourier}.}
			\label{fig:chapter15_sam_regular_vs_fourier}
		\end{figure}
		
		\medskip
		\noindent\textbf{SAM’s choice: random Fourier features (RFF).} SAM treats prompt coordinates as continuous and uses a joint 2D Fourier mapping~\cite{tancik2020_fourier}:
		\[
		\gamma(x,y) \;=\;
		\begin{bmatrix}
			\cos\!\big(2\pi\,B\,[\hat{x},\hat{y}]^{\!\top}\big)\\[2pt]
			\sin\!\big(2\pi\,B\,[\hat{x},\hat{y}]^{\!\top}\big)
		\end{bmatrix} \in \mathbb{R}^{2D},\qquad
		(\hat{x},\hat{y}) \in [-1,1]^2,
		\]
		where \(B\in\mathbb{R}^{D\times 2}\) has i.i.d.\ entries \(B_{ij}\sim\mathcal{N}(0,\sigma^2)\) and \((\hat{x},\hat{y})\) are the normalized coordinates (e.g., \(\hat{x}=2(x/W)-1\), \(\hat{y}=2(y/H)-1\)). Each row of \(B\) defines a sinusoid over a \emph{tilted} direction (a linear combination of \(x\) and \(y\)), so stacking rows yields a bank of multi-frequency, multi-orientation waves that respect 2D geometry. The final prompt token adds a small learned \emph{type} embedding (e.g., foreground/background for points, corner identity for boxes): \(t=\gamma(x,y)+e_{\text{type}}\).
		
		\begin{figure}[H]
			\centering
			\includegraphics[width=0.70\textwidth]{Figures/Chapter_15/SAM_2D_fourier_feature_examples.png}
			\caption{\textbf{Fourier feature basis on a plane.} Rows of \(B\) induce oriented sinusoids at different spatial frequencies; their stack forms a rich 2D positional code. Credit: explanatory \href{https://www.youtube.com/watch?v=iKyIJ_EtSkw}{video}.}
			\label{fig:chapter15_sam_2d_ff_examples}
		\end{figure}
		
		\medskip
		\noindent\textbf{Why RFF helps—two lenses.}
		\begin{itemize}
			\item \emph{Spectral-bias lens (representation).} Coordinate-fed networks learn low frequencies first (“blurry” fits). Prepending \(\gamma(\cdot)\) injects high-frequency basis functions, letting shallow decoders express sharp edges with few updates~\cite{tancik2020_fourier}. Empirically, replacing raw \((x,y)\) or separable 1D PE with RFF improves fine boundary fidelity with fewer corrective clicks.
			\item \emph{Kernel/NTK lens (geometry).} Wide networks trained by gradient descent behave like kernel machines with the Neural Tangent Kernel (NTK)~\cite{jacot2018_ntk}. With \(B\sim\mathcal{N}(0,\sigma^2 I)\), the expected inner product of two encodings depends only on the offset \(\Delta=p-q\):
			\[
			\mathbb{E}_{B}\!\left[\gamma(p)\!\cdot\!\gamma(q)\right]
			\;=\; \exp\!\big(\!-\;2\pi^2\sigma^2\,\|\Delta\|_2^2\big),
			\]
			i.e., a Gaussian RBF (up to constants). Thus, \(\sigma\) controls an \emph{isotropic} notion of locality: small \(\sigma\) \(\Rightarrow\) wide kernel (smooth, risk of underfitting); large \(\sigma\) \(\Rightarrow\) narrow kernel (sharp, risk of aliasing). This aligns vector similarity with Euclidean distance—exactly what prompt geometry needs.
		\end{itemize}
		
		\begin{figure}[H]
			\centering
			\includegraphics[width=0.60\textwidth]{Figures/Chapter_15/SAM_kernel_regression.png}
			\caption{\textbf{Kernel regression analogy.} A fit is a sum of local bumps; kernel width trades smoothness for detail. The NTK plays the same role for wide networks. Credit: explanatory \href{https://www.youtube.com/watch?v=iKyIJ_EtSkw}{video}.}
			\label{fig:chapter15_sam_kernel_regression}
		\end{figure}
		
		\begin{figure}[H]
			\centering
			\includegraphics[width=0.70\textwidth]{Figures/Chapter_15/SAM_width_of_kernel_is_critical.png}
			\caption{\textbf{Kernel width is critical.} Too wide \(\Rightarrow\) blurred structure (underfit). Too narrow \(\Rightarrow\) noisy/aliased (overfit). RFF exposes a single knob—\(\sigma\)—to dial the effective width via the scale of \(B\). Credit: explanatory \href{https://www.youtube.com/watch?v=iKyIJ_EtSkw}{video}.}
			\label{fig:chapter15_sam_kernel_width}
		\end{figure}
		
		\medskip
		\noindent\textbf{From derivation to practice.} The RFF mapping arises from Bochner’s theorem: any shift-invariant positive-definite kernel has a nonnegative Fourier transform \(\hat{k}(\omega)\) with
		\(k(\Delta)=\mathbb{E}_{\omega\sim\hat{k}}[\cos(2\pi\,\omega^\top\Delta)]\).
		Sampling \(\omega\) from a Gaussian \(\mathcal{N}(0,\sigma^2 I)\) gives a Gaussian RBF kernel; Monte Carlo features \(\gamma(\cdot)\) approximate it~\cite{tancik2020_fourier}. Normalizing coordinates to \([-1,1]^2\) avoids phase wrapping and makes the code resolution-agnostic.
		
		\begin{figure}[H]
			\centering
			\includegraphics[width=0.70\textwidth]{Figures/Chapter_15/SAM_neural_tangent_kernel.png}
			\caption{\textbf{NTK perspective.} RFF turns the network’s effective kernel into a stationary, radial form whose bandwidth is governed by \(\sigma\). Tuning \(\sigma\) navigates the bias–variance trade-off. Credit: explanatory \href{https://www.youtube.com/watch?v=iKyIJ_EtSkw}{video}.}
			\label{fig:chapter15_sam_ntk}
		\end{figure}
		
		\medskip
		\noindent\textbf{How to tune \(\sigma\) (and what SAM does).} Choose \(\sigma\) by a small grid/linear search on validation data: fix a random \(B\) per \(\sigma\), evaluate a proxy (e.g., mIoU of point-to-mask or reconstruction PSNR in a coordinate MLP), and pick the best trade-off (sharp boundaries without aliasing). In SAM, \(B\) is sampled \emph{once} and then frozen; \(\sigma\) is treated as a hyperparameter, keeping the prompt path parameter-free and fast at inference.
		
		\begin{figure}[H]
			\centering
			\includegraphics[width=0.7\textwidth]{Figures/Chapter_15/SAM_linear_search_optimal_sigma_blurred.png}
			\caption{\textbf{Too small \(\sigma\) (wide kernel).} High-frequency details are missed and outputs look over-smoothed/blurred. Credit: explanatory \href{https://www.youtube.com/watch?v=iKyIJ_EtSkw}{video}.}
			\label{fig:chapter15_sam_sigma_small}
		\end{figure}
		
		\begin{figure}[H]
			\centering
			\includegraphics[width=0.7\textwidth]{Figures/Chapter_15/SAM_linear_search_optimal_sigma_less_blurred.png}
			\caption{\textbf{Near-optimal \(\sigma\).} Fine detail is preserved without a lot of aliasing; quality peaks near this region. Credit: explanatory \href{https://www.youtube.com/watch?v=iKyIJ_EtSkw}{video}.}
			\label{fig:chapter15_sam_sigma_opt}
		\end{figure}
		
		\begin{figure}[H]
			\centering
			\includegraphics[width=0.75\textwidth]{Figures/Chapter_15/SAM_fourier_features.jpg}
			\caption{\textbf{Fourier features mitigate spectral bias.} A coordinate MLP remains blurry at equal iterations, whereas the same MLP with RFF recovers high-frequency detail much earlier. Credit: explanatory \href{https://www.youtube.com/watch?v=iKyIJ_EtSkw}{video}; see also~\cite{tancik2020_fourier}.}
			\label{fig:chapter15_sam_ff_blur_vs_sharp}
		\end{figure}
	
		\noindent \textbf{RFF Effect on SAM.} Compared with separable 1D PE, RFF delivers:
		
		\begin{itemize}
			\item \emph{Isotropic locality.} Similarity decays with \(\|p-q\|_2\), so point and box-corner tokens condition the decoder uniformly in all directions, reducing axis bias at boundaries.
			\item \emph{High-frequency readiness.} The decoder’s small MLPs receive multi-frequency inputs, enabling crisp, click-efficient refinements around thin parts and textured edges.
		\end{itemize}
		
		\paragraph{Mask decoder (two-way attention and dynamic heads)}
		\noindent\textit{High-level intuition.}
		Once the heavyweight ViT encoder has produced a rich, cached feature \emph{map} of the image, the mask decoder turns this static representation into an interactive tool. User prompts (points, boxes, and optionally a prior mask) are converted into a small set of \emph{prompt tokens} that act as sparse ``pins'' indicating what the user cares about. The mask decoder, implemented as a lightweight two-layer transformer, runs a short \emph{two-way attention} procedure: prompt tokens query the image embedding for visual evidence, and image features in turn query the prompts to understand which parts of the scene are relevant. This bidirectional exchange yields a set of prompt-aware features and updated tokens from which the decoder predicts a few candidate masks together with a quality score for each, enabling fast, interactive refinement.
		
		\begin{figure}[H]
			\centering
			\includegraphics[width=0.75\textwidth]{Figures/Chapter_15/SAM_mask_decoder_details.jpg} 
			\caption{\textbf{SAM's lightweight mask decoder.} \textbf{(a)} Inputs: prompt tokens plus four learned output tokens (three mask tokens and one IoU token); if available, the previously accepted mask is injected as a \emph{dense prompt} by adding its embedding to the image features. \textbf{(b)} Two stacked two-way attention blocks: token self-attention fuses prompt cues; token$\rightarrow$image attention retrieves spatial evidence; image$\rightarrow$token attention makes image features prompt-aware (positional encodings are added to image features and the original prompt is re-added to token queries/keys for stability). \textbf{(c)} Upscaled features feed dynamic heads: mask tokens, via an MLP and dot products, yield multiple mask hypotheses; the IoU token scores them for ranking/selection. Adapted from~\cite{kirillov2023_sam}.}
			\label{fig:chapter15_sam_decoder_details}
		\end{figure}
		
		\noindent\textit{Two-way attention as a conversation.}
		The two-way attention block can be viewed as a two-step ``conversation'' between prompts and image:
		
		\begin{itemize}
			\item \textbf{Prompts $\rightarrow$ image (token$\rightarrow$image attention).} Prompt and output tokens ask the image embedding: ``Where in this feature map is the evidence that supports my click or box?'' Each token pulls in edges, textures, and contextual cues from relevant spatial locations.
			\item \textbf{Image $\rightarrow$ prompts (image$\rightarrow$token attention).} Image features then ask back: ``Given these prompts, which of them are relevant for this local patch?'' This makes the image embedding \emph{prompt-aware}, amplifying features consistent with the prompts and suppressing distractors (e.g., shadows or adjacent objects).
		\end{itemize}
		
		\newpage
		
		\medskip \noindent Because both directions are present, the prompts become evidence-aware and the image becomes intent-aware; neither side dominates, which is crucial for producing masks that both follow the user’s clicks and respect the global image structure.
		
		\medskip \noindent\textit{Why this structure?}
		Two-way attention avoids failure modes of one-sided designs: a prompt-only decoder might hallucinate shapes that match the clicks but ignore global context, while an image-only decoder might segment the most salient object and disregard the specific prompt. Learned \emph{mask tokens} act as dynamic heads that specialize into different plausible extents (e.g., whole object, part, subpart) without introducing heavy per-pixel branches. A separate IoU token learns to predict the quality (approximate IoU) of each candidate mask, turning the set of hypotheses into a ranked list. In practice, this design yields high-quality, multi-mask predictions in tens of milliseconds, supporting real-time interaction~\cite{kirillov2023_sam}.
		
		\noindent\textit{Step-by-step (one decode).}
		\begin{enumerate}
			\item \textbf{Assemble inputs.}
			Encode user prompts into tokens:
			\begin{itemize}
				\item Points are embedded from their image coordinates and a foreground/background flag.
				\item Boxes are embedded from corner coordinates.
				\item An optional coarse-mask token encodes a prior mask in sparse form.
			\end{itemize}
			If a previous mask was accepted, it is downsampled, projected, and \emph{added} as a dense prompt to the image embedding. This biases features near the existing boundary, making interactive refinement more efficient in subsequent passes.
			
			\item \textbf{Add output tokens.}
			Append four learned output tokens to the prompt tokens:
			\begin{itemize}
				\item Three \emph{mask tokens}, each responsible for producing one candidate mask.
				\item One \emph{IoU token}, responsible for predicting the quality score of those masks.
			\end{itemize}
			These tokens start as content-agnostic vectors and will be shaped by the two-way attention blocks into object-specific descriptors.
			
			\item \textbf{Two-way block \#1 (gather evidence).}
			The first two-way attention block runs three sub-steps:
			\begin{enumerate}
				\item \emph{Token self-attention.} Prompt and output tokens attend to each other to fuse their cues. For example, multiple positive clicks on the same object reinforce one instance, while negative clicks help suppress distractors.
				\item \emph{Token$\rightarrow$image attention.} Tokens query the cached image embedding to retrieve spatial evidence, pulling in local structure (edges, textures) and part/whole context around the prompts.
				\item \emph{Image$\rightarrow$token attention.} Image features attend back to the current tokens, becoming \emph{prompt-aware} by emphasizing regions that are compatible with the prompts. Positional encodings are added on the image side, and the original prompt embeddings (with position encodings) are re-added to token queries and keys to maintain stability and spatial anchoring~\cite{kirillov2023_sam}.
			\end{enumerate}
			After this block, tokens carry evidence-rich context and the image embedding is already shaped by user intent.
			
			\newpage
			
			\item \textbf{Two-way block \#2 (synthesize and refine).}
			A second, identical two-way block repeats the three sub-steps on the updated tokens and image features. The first block primarily \emph{gathers} evidence; the second \emph{synthesizes} it, refining object boundaries and resolving ambiguities such as part-versus-whole choices.
			
			\item \textbf{Predict masks and scores.}
			Finally, the prompt-aware image embedding is upsampled with lightweight transposed convolutions to a decoder resolution (e.g., \(256 \times 256\)).
			\begin{itemize}
				\item Each \emph{mask token} passes through a small MLP to produce a mask embedding. A dot product between this embedding and the upscaled feature map at each spatial location yields one logit map per mask token, corresponding to different hypotheses (e.g., whole object, part, subpart).
				\item The \emph{IoU token} is fed through its own MLP to predict a scalar quality score for each mask, trained to approximate its IoU with the ground-truth mask.
			\end{itemize}
			The resulting masks are produced at decoder resolution and then resized to the original image resolution (or to the box region) for visualization and downstream use. The highest-scoring mask can be selected automatically, while alternative hypotheses are available for interactive correction.
		\end{enumerate}
		
		\paragraph{Training objective and loss}
		\label{enr:par_chapter15_sam_training}
		
		\noindent\textbf{High-level goal (how to supervise a \emph{promptable} model).}
		Classical segmentation trains a network to label \emph{all} pixels at once. SAM instead learns a \emph{conditional} mapping
		\(\langle I,\text{prompt}\rangle \!\mapsto\! \text{mask(s)}\),
		so supervision must (i) treat points/boxes as \emph{inputs}, not targets; (ii) compare only \emph{predicted masks} to ground-truth; and (iii) support \emph{multiple hypotheses} because a single prompt can mean whole/part/subpart. The losses below implement this recipe efficiently at SA-1B scale~\cite{kirillov2023_sam}.
		
		\medskip
		\noindent\textbf{Targets and supervision signal.}
		Each training example consists of an image \(I\) and a \emph{binary, pixel-accurate instance mask}
		\(M \in \{0,1\}^{H\times W}\) for a single segment (foreground \(=1\), background \(=0\)).
		SAM is trained to predict a mask \(\hat{M}\) \emph{conditioned on a prompt} \(P\); it does \emph{not} predict boxes or points themselves.
		During training, prompts are \emph{simulated from \(M\)} (see below).
		Supervision always compares \(\hat{M}\) against \(M\) (mask–vs–mask); there is no box loss.
		
		\noindent\textbf{Prompt simulation (teaching interactivity without human clicks).}
		To expose the decoder to realistic inputs, we synthesize prompts \(P\) from \(M\):
		\begin{itemize}
			\item \textbf{Positive / negative points.} Sample positives uniformly inside \(M\); sample negatives outside \(M\)
			(optionally biased near the boundary to mimic corrective clicks).
			\item \textbf{Boxes.} Use the tight bounding rectangle of \(M\), then apply random scale/aspect jitter;
			optionally draw from cropped regions to vary context.
			\item \textbf{Dense prior (previous mask).} Downsample \(M\) (or a perturbed version via erode/dilate) to form a
			coarse “dense prompt’’ used for refinement training.
			\item \textbf{Multi-round chains.} In a subset of batches, decode once, place corrective points on disagreement regions,
			and decode again—simulating click–refine loops.
		\end{itemize}
		Prompts are \emph{inputs}; supervision remains purely mask–vs–mask.
		
		\newpage
		
		\noindent\textbf{Multi-hypothesis supervision (min-over-masks).}
		Given one prompt, the decoder emits up to three plausible masks
		\(\{\hat{M}_j\}_{j=1}^3\subset[0,1]^{H\times W}\)
		to capture whole/part/subpart ambiguity. With only one ground truth \(M\), we compute a segmentation loss for each \(\hat{M}_j\) and backpropagate through the \emph{best} one:
		\[
		\mathcal{L}_{\text{seg}}
		=\min_{j\in\{1,2,3\}}
		\Big[\,
		\lambda_{\text{focal}}\,\mathcal{L}_{\text{focal}}(\hat{M}_j,M)
		+\lambda_{\text{dice}}\,\mathcal{L}_{\text{dice}}(\hat{M}_j,M)
		\,\Big].
		\]
		Intuition: under an ambiguous prompt, we want \emph{at least one} candidate to match the user’s intent. The “min’’ lets the three heads specialize (e.g., one tends to whole, one to part) instead of collapsing all to the same mask. A strong focal{:}dice ratio (reported \(20{:}1\)) emphasizes boundary decisions under severe fg/bg imbalance~\cite{kirillov2023_sam}.
		
		\medskip
		\noindent\textbf{Loss components (what they measure and why).}
		\begin{itemize}
			\item \emph{Focal loss} combats extreme class imbalance by down-weighting easy pixels and amplifying hard ones near edges. With logits \(z\) and post-sigmoid probability \(p=\sigma(z)\), for a target \(y\!\in\!\{0,1\}\) the binary focal loss is
			\[
			\mathcal{L}_{\text{focal}}(p,y)
			=-\alpha_t(1-p_t)^{\gamma}\log(p_t),
			\quad
			p_t=\begin{cases}
				p,&y=1\\
				1-p,&y=0
			\end{cases}
			\]
			with typical \(\gamma\!>\!0\) and \(\alpha_t\) rebalancing fg/bg. In SAM, this term dominates to focus learning where it matters most (thin structures, uncertain boundaries).
			\item \emph{Dice loss} directly optimizes region overlap (shape agreement). For probabilities \(\hat{M}\in[0,1]^{H\times W}\),
			\[
			\mathcal{L}_{\text{dice}}(\hat{M},M)
			=1-\frac{2\,\langle \hat{M},M\rangle + \varepsilon}{\|\hat{M}\|_1+\|M\|_1+\varepsilon},
			\]
			where \(\langle\cdot,\cdot\rangle\) sums pixelwise products and \(\varepsilon\) stabilizes small masks. Dice penalizes false positives/negatives at the \emph{shape} level, complementing focal’s pixel focus.
		\end{itemize}
		
		\begin{figure}[H]
			\centering
			\includegraphics[width=0.70\textwidth]{Figures/Chapter_15/SAM_dice_loss.jpg}
			\caption{\textbf{Dice loss intuition.} Dice complements focal by measuring region-level overlap: it decreases as symmetric set difference shrinks, and rises when FP/FN inflate the union. A high focal{:}dice weight in SAM targets boundary imbalance while preserving shape fidelity.}
			\label{fig:chapter15_sam_dice}
		\end{figure}
		
		\medskip
		\noindent\textbf{Quality calibration (IoU head).}
		Beyond masks, SAM predicts for each hypothesis a scalar \(\hat{s}_j\) that should approximate the true IoU,
		\[
		\mathrm{IoU}(\hat{M}_j,M)=\frac{|\hat{M}_j\cap M|}{|\hat{M}_j\cup M|}.
		\]
		A simple MSE trains this calibration:
		\[
		\mathcal{L}_{\text{iou}}=\frac{1}{3}\sum_{j=1}^{3}\big(\hat{s}_j-\mathrm{IoU}(\hat{M}_j,M)\big)^2.
		\]
		At inference, \(\hat{s}_j\) ranks candidates and flags low-confidence cases (“add a click?”), matching SAM’s interactive use.
		
		\medskip
		\noindent\textbf{Total objective and gradients.}
		The overall loss is
		\[
		\mathcal{L}
		= \mathcal{L}_{\text{seg}} + \lambda_{\text{iou}}\mathcal{L}_{\text{iou}},
		\quad \text{with }\lambda_{\text{iou}}=1 \text{ in~\cite{kirillov2023_sam}}.
		\]
		Let \(\ell_j=\lambda_{\text{focal}}\mathcal{L}_{\text{focal}}(\hat{M}_j,M)+\lambda_{\text{dice}}\mathcal{L}_{\text{dice}}(\hat{M}_j,M)\).
		If \(j^\star=\arg\min_j \ell_j\), then
		\(\nabla \mathcal{L}_{\text{seg}}=\nabla \ell_{j^\star}\) (other branches receive no seg-gradients),
		encouraging diversity across heads while the IoU head learns to score \emph{all} candidates.
		
		\medskip
		\noindent\textbf{Why this works (design intuition).}
		\begin{itemize}
			\item \emph{Prompt-conditioned supervision} teaches the decoder to “follow the cue’’ rather than memorize taxonomies—key for zero-shot transfer.
			\item \emph{Min-over-masks} aligns training with usage: present alternatives, let one match intent, keep others diverse for ambiguity.
			\item \emph{Focal\,+\,Dice} balances boundary hardness and global overlap—crucial when fg pixels are scarce and shapes vary widely.
			\item \emph{IoU calibration} closes the loop for interactivity: the model not only proposes masks but also knows which is best and when to ask for help.
		\end{itemize}
		
		\paragraph{Pseudo-code for interactive inference}
		
		\noindent\textbf{Single image, multi-round interaction.}
		\begin{enumerate}
			\item \textbf{Encode once:} \(E \leftarrow \textsc{ImageEncoder}(I)\) \hfill (cache the heavy image embedding).
			\item \textbf{Repeat until accepted:}
			\begin{enumerate}
				\item[(a)] \textbf{Encode prompt:} \(P \leftarrow \textsc{PromptEncoder}(\text{points},\,\text{boxes},\,M_{\text{prev}})\),
				where \(M_{\text{prev}}\) is the previously accepted mask used as a dense prompt (optional).
				\item[(b)] \textbf{Decode:} \((\hat{M}_1,\hat{M}_2,\hat{M}_3,\hat{s}_1,\hat{s}_2,\hat{s}_3) \leftarrow \textsc{MaskDecoder}(E,P)\).
				\item[(c)] \textbf{Select \& display:} \(j^\star = \arg\max_{j\in\{1,2,3\}} \hat{s}_j\); render \(\hat{M}_{j^\star}\) at image resolution.
				\item[(d)] \textbf{Refine or stop:} If boundaries deviate, add a positive point to include a missed region or a negative point to exclude leakage; set \(M_{\text{prev}} \leftarrow \hat{M}_{j^\star}\) and repeat. Otherwise, accept the mask.
			\end{enumerate}
		\end{enumerate}
		
		\newpage
		
		\subsubsection{Data engine and SA-1B}
		\label{enr:subsubsec_chapter15_sam_data}
		The authors construct SA-1B via a three-stage engine~\cite{kirillov2023_sam}: assisted manual collection (browser-based tool powered by early SAM), semi-automatic (detector-seeded prompts with human verification), and fully automatic generation using grid prompts and multi-scale crops followed by ranking, stability checks, de-duplication, hole-filling, and small-component pruning.
		\begin{figure}[H]
			\centering
			\includegraphics[width=0.85\textwidth]{Figures/Chapter_15/SAM_dataset_examples.jpg}
			\caption{\textbf{SA-1B examples}. 11M licensed and privacy-protecting images and \(\sim\)1.1B masks; images grouped by masks-per-image to illustrate density and diversity; credit: Kirillov \emph{et~al.}~\cite{kirillov2023_sam}.}
			\label{fig:chapter15_sa1b_examples}
		\end{figure}
		
		\paragraph{Dataset properties and diversity}
		SA-1B is geographically and visually diverse, with broader coverage of object locations and shapes than prior datasets.
		\begin{figure}[H]
			\centering
			\includegraphics[width=0.70\textwidth]{Figures/Chapter_15/SAM_normalized_mask_centers.jpg}
			\caption{\textbf{Normalized mask centers}. Heatmaps of mask centers across datasets indicate that SA-1B reduces strong center bias and covers corners/edges more uniformly; credit: Kirillov \emph{et~al.}~\cite{kirillov2023_sam}.}
			\label{fig:chapter15_sam_centers}
		\end{figure}
		
		\begin{figure}[H]
			\centering
			\includegraphics[width=0.85\textwidth]{Figures/Chapter_15/SAM_dataset_mask_properties.jpg}
			\caption{\textbf{Mask properties}. SA-1B contains many images with high mask density, a broad distribution of mask sizes, and comparable or greater concavity diversity than prior datasets; credit: Kirillov \emph{et~al.}~\cite{kirillov2023_sam}.}
			\label{fig:chapter15_sam_mask_props}
		\end{figure}
		
		\begin{figure}[H]
			\centering
			\includegraphics[width=0.85\textwidth]{Figures/Chapter_15/SAM_dataset_image_distribution.jpg}
			\caption{\textbf{Geographic distribution}. Estimated distribution by country shows global coverage; the top three countries come from different regions; credit: Kirillov \emph{et~al.}~\cite{kirillov2023_sam}.}
			\label{fig:chapter15_sam_geo}
		\end{figure}
		
		\subsubsection{Experiments and ablations}
		\label{enr:subsubsec_chapter15_sam_experiments}
		
		\paragraph{Zero-shot samples across domains}
		\begin{figure}[H]
			\centering
			\includegraphics[width=0.85\textwidth]{Figures/Chapter_15/SAM_zero_shot.jpg}
			\caption{\textbf{Zero-shot qualitative results}. Samples from 23 diverse datasets (autonomous driving, medical, aerial, egocentric, etc.) segmented by SAM without fine-tuning; credit: Kirillov \emph{et~al.}~\cite{kirillov2023_sam}.}
			\label{fig:chapter15_sam_zeroshot}
		\end{figure}
		
		\paragraph{Interactive point-to-mask evaluation}
		SAM is evaluated \emph{zero-shot} on 23 unseen datasets with a simulated interactive protocol (place a point on the largest error, iterate) and both one-click and multi-click metrics~\cite{kirillov2023_sam}. On the one-click setting, SAM exceeds prior interactive baselines on \textbf{16/23} datasets and the gap reaches \textbf{+47 mIoU} on some sets; when an oracle picks the best of its three hypotheses (\emph{SAM–oracle}), it outperforms all baselines on all 23 datasets~\cite{kirillov2023_sam}. Human quality ratings fall in the 7--9 range (Likert-style) and SAM’s oracle masks are rated close to ground truth, indicating high fidelity~\cite{kirillov2023_sam}. Multi-click curves show steady gains with diminishing returns after about \textasciitilde8 clicks; the training curriculum mirrors this with sequences up to 11 interactions to teach refinement~\cite{kirillov2023_sam}.
		
		\begin{figure}[H]
			\centering
			\includegraphics[width=0.85\textwidth]{Figures/Chapter_15/SAM_evaluation.jpg}
			\caption{\textbf{Zero-shot point-to-mask across 23 datasets.}
				\emph{(a) One-click mIoU.} SAM (automatic selection via its IoU head) surpasses RITM on most datasets; the \emph{SAM–oracle} bar (best-of-3 selection) is an upper bound, illustrating the benefit of ambiguity-aware decoding. \emph{(b) Human study.} Mean mask-quality ratings place \emph{SAM–oracle} near ground-truth and above prior interactive systems. \emph{(c,d) Multi-click curves.} mIoU improves with additional corrective clicks (simulation places the next click at the largest error); gains taper after \textasciitilde8 clicks, matching the training curriculum (up to 11 prompts). Panels adapted from Kirillov \emph{et\,al.}~\cite{kirillov2023_sam}.}
			\label{fig:chapter15_sam_eval}
		\end{figure}
		
		\paragraph{Ablations (highlights)}
		\begin{itemize}
			\item \textbf{Multi-mask hypotheses + min-over training.} Predicting multiple masks per prompt and supervising with a \emph{min-over-masks} loss lets the model represent whole/part/subpart alternatives without averaging incompatible solutions; it is a core ingredient in SAM’s ambiguity handling and one-click strength~\cite{kirillov2023_sam}.
			\item \textbf{Two-way attention in the decoder.} Letting tokens \emph{query} image features and image features \emph{query back} the tokens (prompt-aware feature refinement) improves mask quality versus token$\rightarrow$image only; the authors report this bidirectional variant as particularly helpful for ambiguous, sparse prompts~\cite{kirillov2023_sam}.
			\item \textbf{Prompt encodings with Fourier features.} Using random Fourier feature (RFF) positional encodings for sparse prompts yields near-isotropic geometry and better alignment than separable 1D encodings or raw coordinates, reducing axis-bias in point/box conditioning~\cite{kirillov2023_sam,tancik2020_fourier}.  
			\item \textbf{IoU prediction head for ranking.} A small head trained to predict the mask IoU enables reliable automatic selection among the three hypotheses; in aggregate plots, SAM’s auto-selected mask tracks the oracle closely, validating the calibration~\cite{kirillov2023_sam}.
			\item \textbf{Interactive curriculum.} The evaluation and training are aligned: simulated clicks are placed on the largest current error, improvement slows after \textasciitilde8 clicks, and SAM is trained with sequences up to 11 interactions to learn the refine–correct loop~\cite{kirillov2023_sam}. 
		\end{itemize}
		
		\subsubsection{Limitations and future directions}
		\label{enr:subsubsec_chapter15_sam_limits}
		\begin{itemize}
			\item \textbf{Heavy encoder cost.} The ViT-H encoder is computationally expensive; although amortized for interactivity, deployment on resource-limited devices is challenging. Subsequent works (e.g., SAM~2) explore efficiency and streaming settings.
			\item \textbf{Open-vocabulary text-to-mask.} Fully integrated text grounding is limited in SAM; later systems combine grounding detectors (e.g., Grounding DINO) with SAM for text-to-region prompts, leading to Grounded-SAM variants.
			\item \textbf{Fine structures and thin parts.} Performance can degrade for extremely thin or low-contrast structures; higher-resolution backbones and tailored decoders are active directions.
			\item \textbf{Temporal/video.} SAM operates on single images; extensions to video streaming and memory-aware decoding are developed in SAM~2, covered next.
		\end{itemize}
		
	\end{enrichment}
	
	\newpage
	
	\begin{enrichment}[SAM 2: Segment Anything in Images and Videos][subsection]
	\label{enr:subsec_chapter15_sam2}
	
		\noindent\textit{Context.} We proceed to cover SAM~2~\cite{ravi2024_sam2}, the video-capable successor to SAM~\cite{kirillov2023_sam}. SAM~2 extends promptable segmentation from single images to \emph{videos} by equipping a SAM-like encoder–decoder with a streaming memory that maintains object state over time. As before, we assume familiarity with encoder–decoder transformers and MAE-style pretraining; full treatments of Vision Transformers and self-supervised pretraining appear later in the book (as for SAM).
		
		\begin{figure}[H]
			\centering
			\includegraphics[width=0.85\textwidth]{Figures/Chapter_15/SAM2_overview.jpg}
			\caption{\textbf{SAM~2 overview}. SAM~2 extends promptable segmentation to images \emph{and} videos by adding a streaming memory that stores compact tokens distilled from prompts and predictions in earlier frames. A model-in-the-loop SA-V data engine scales training via human-in-the-loop collection and automatic propagation; credit: Ravi \emph{et~al.}~\cite{ravi2024_sam2}.}
			\label{fig:chapter15_sam2_overview}
		\end{figure}
		
		\paragraph{Core idea: streaming memory for video}
		SAM~2’s central contribution is a \emph{streaming memory bank} maintained per tracked instance. Instead of storing full frames, the system keeps compact tokens summarizing past accepted masks and prompts. This yields two complementary behaviors:
		\begin{itemize}
			\item \textbf{Propagation.} For each new frame \(t\), the decoder reads instance-specific memory tokens, fuses them with the current-frame features, and predicts the mask for that frame. Prior masks act as a strong prior for the object’s location and appearance, so the model effectively refines an existing estimate rather than re-segmenting from scratch.
			\item \textbf{Recovery.} When propagation drifts (e.g., due to occlusion or fast motion), a user provides a corrective prompt on some later frame \(t^\star\). The model produces a corrected mask, encodes it into new tokens, and writes them into memory. Subsequent frames read this updated state and continue with the corrected identity, without re-annotating intermediate frames.
		\end{itemize}
		Thus SAM~2 preserves SAM’s promptable interface while adding a lightweight mechanism for temporal consistency and interactive correction in videos.
		
		\newpage
		
		\subsubsection{Motivation}
		\label{enr:subsubsec_chapter15_sam2_motivation}
		
		\noindent SAM showed that an ``encode once, decode many'' architecture with promptable, ambiguity-aware decoding yields strong zero-shot segmentation on \emph{images}~\cite{kirillov2023_sam}. Extending this paradigm to \emph{videos} introduces additional requirements:
		\begin{itemize}
			\item \textbf{Identity persistence.} The same object must be followed through motion, deformation, occlusion, disappearance, and reappearance.
			\item \textbf{Sparse correction.} Users should be able to repair drifts with a few clicks rather than repeatedly re-prompting from scratch.
			\item \textbf{Interactive speed.} Per-frame latency must remain low for real-time annotation and editing, even on long sequences.
		\end{itemize}
		
		A naïve SAM\,+\,tracker pipeline struggles on all three fronts: the tracker does not share SAM’s notion of objectness, corrections on a later frame do not automatically propagate forward, and failures often require full reinitialization. SAM~2 addresses these limitations by coupling the promptable decoder to a streaming memory that aggregates compact embeddings of past masks and prompts. Every new prompt or correction is written into this memory, and each subsequent frame reads the updated state, so improvements on frame \(t^\star\) immediately benefit frames \(t{>}t^\star\) without revisiting earlier predictions.
		
		\begin{figure}[H]
			\centering
			\includegraphics[width=0.85\textwidth]{Figures/Chapter_15/SAM2_interactive_segmentation.jpg}
			\caption{\textbf{Interactive video segmentation with SAM~2}. An initial prompt on frame~1 yields a \emph{masklet} (a contiguous run of predictions for one instance) that propagates forward. If tracking drifts, a single corrective click in a later frame writes a corrected state into memory, allowing SAM~2 to recover the object and continue propagation with identity consistency; credit: Ravi \emph{et~al.}~\cite{ravi2024_sam2}.}
			\label{fig:chapter15_sam2_interactive}
		\end{figure}
		
		\subsubsection{Method}
		\label{enr:subsubsec_chapter15_sam2_method}
		
		\paragraph{Problem setup}
		Given a video \(\{I_t\}_{t=1}^{T}\) and prompts \(\mathcal{P}\) provided on one or more frames (points, boxes, or masks), SAM~2 produces a temporally consistent mask \(\hat{M}_t\) per frame for the same instance specified by the prompts. A temporally contiguous run of such predictions for a single instance is called a \emph{masklet}, i.e., a sequence of masks belonging to one object track. The system is designed to support:
		\begin{itemize}
			\item Image-only use (\(T{=}1\)), matching SAM.
			\item One-shot video prompting (prompts on an initial frame only, then fully automatic propagation).
			\item Sparse interactive corrections on arbitrary later frames, with each correction immediately influencing future predictions.
		\end{itemize}
		
		\paragraph{What is new compared to SAM}
		SAM~2 preserves SAM’s basic decomposition (image encoder, prompt encoder, mask decoder) but augments it with temporal reasoning and a video-scale data engine:
		\begin{itemize}
			\item \textbf{Streaming memory.} For each tracked instance, SAM~2 maintains a dedicated memory bank of compact (\(\approx 64\)-dimensional) tokens distilled from past \emph{accepted} masks and prompts. These tokens summarize both local appearance and coarse spatial position, and are much cheaper to store than full feature maps or frames. At each new frame, a bounded subset of tokens is retrieved (e.g., based on recency and/or similarity), providing identity cues at roughly constant per-frame cost.
			\item \textbf{Memory-conditioned decoding.} The lightweight decoder now conditions on three sources: current-frame image features, optional prompts on the current frame, and retrieved memory tokens. This injects temporal context directly into the promptable decoder without heavy re-encoding or explicit optical flow.
			\item \textbf{Masklet supervision and video-scale data engine.} Training uses SA-V, a large-scale video dataset with frame-wise masks grouped into masklets, disappearance/reappearance events, and model-in-the-loop propagation. Supervision is applied while the memory pathway is active, so the network learns to write informative tokens and to read them effectively for video segmentation.
		\end{itemize}
		
		To highlight the evolution from SAM to SAM~2, the following table summarizes key differences.
		
		\begin{table}[H]
			\centering
			\caption{\textbf{SAM vs.\ SAM~2 at a glance}. SAM~2 generalizes SAM from images to videos by introducing streaming memory, a new Hiera-based backbone, and the SA-V dataset.}
			\label{tab:chapter15_sam_vs_sam2}
			\begin{tabular}{p{0.23\textwidth}p{0.32\textwidth}p{0.32\textwidth}}
				\toprule
				\textbf{Aspect} & \textbf{SAM~\cite{kirillov2023_sam}} & \textbf{SAM~2~\cite{ravi2024_sam2}} \\
				\midrule
				Domain & Images. & Images + videos (promptable VOS). \\
				Key new module & -- & Streaming memory (per-instance). \\
				Image encoder & ViT-H (MAE). & Hiera (hierarchical MAE) + FPN. \\
				Training data & SA-1B (11M images). & SA-1B + SA-V (\(\sim 50\)K videos, \(\sim 642\)K masklets). \\
				Supervision mode & Image masks only. & Clip-level masks with memory active. \\
				Per-frame throughput & \(\sim\)tens of FPS for images. & Real-time video (optimized predictor \(\sim 130\) FPS per object). \\
				\bottomrule
			\end{tabular}
		\end{table}
		
		\paragraph{Why streaming memory? Design goals}
		The streaming memory in SAM~2 is tailored to promptable video segmentation with three main goals:
		\begin{itemize}
			\item \textbf{Interactive recovery.} A corrective click and its resulting mask are encoded into new memory tokens that replace outdated information about the object. Later frames then read from this updated state, so propagation resumes from the corrected configuration rather than from a stale 
			track.
			
			\newpage
			
			\item \textbf{Efficient propagation.} The decoder reuses prior memory tokens as a prior over the object’s location and appearance, reducing the amount of per-frame reasoning required to maintain a coherent track. Reading a bounded set of compact tokens keeps computation per frame approximately constant.
			\item \textbf{Identity stability.} Memory is instance-specific and selective: only the chosen hypothesis (the accepted mask) is written for that instance. This reduces contamination from competing masks and helps maintain a stable identity through occlusions, appearance changes, and background clutter.
		\end{itemize}
		
		\paragraph{High-level data flow}
		At a high level, SAM~2 processes each frame \(t\) through a lightweight streaming pipeline:
		\begin{enumerate}
			\item \textbf{Image encoding} \(\rightarrow\) The current frame \(I_t\) is passed once through the Hiera+FPN backbone to produce dense multi-scale features; a main stride-\(s\) feature map \(F_t\) is cached for use by memory and the decoder.
			\item \textbf{Memory read} \(\rightarrow\) For each tracked instance, a bounded subset of memory tokens is selected from its bank (e.g., based on recency and/or similarity). These tokens summarize past masks and prompts and provide identity-specific context.
			\item \textbf{Prompt encoding (optional)} \(\rightarrow\) Any new clicks, boxes, or masks on frame \(t\) are encoded as prompt tokens \(P_t\), as in SAM.
			\item \textbf{Decoding} \(\rightarrow\) The mask decoder fuses the current-frame features \(F_t\), retrieved memory tokens, and prompt tokens to predict up to three candidate masks \(\{\hat{M}_{t,j}\}_{j=1}^3\) with associated IoU scores. One mask is selected as the \emph{active} hypothesis for that instance.
			\item \textbf{Memory write} \(\rightarrow\) The selected mask and any prompts are transformed by a memory encoder into new tokens, which are appended to the instance’s memory bank (evicting the oldest entries if needed).
		\end{enumerate}
		Together, these steps implement a constant-cost loop: \emph{encode frame} \(\to\) \emph{read memory} \(\to\) (optional) \emph{encode prompt} \(\to\) \emph{decode masks} \(\to\) \emph{write memory}, sustaining interactive throughput over long videos.
		
		\newpage
		
		\paragraph{Streaming memory mechanics}
		We now outline the memory internals at a slightly more formal level. Let \(F_t \in \mathbb{R}^{C \times H \times W}\) denote the main stride-\(s\) feature map for frame \(t\). For each tracked instance, SAM~2 maintains a bounded memory bank
		\[
		\mathcal{B} = \Big\{(K^{(j)}, V^{(j)}, \pi^{(j)})\Big\}_{j \in \mathcal{J}}, \qquad |\mathcal{J}| \leq N_{\text{recent}} + N_{\text{prompt}},
		\]
		where \((K^{(j)},V^{(j)}) \in \mathbb{R}^{HW \times d_k} \times \mathbb{R}^{HW \times d_v}\) are spatial key/value tokens distilled from frame \(j\), and \(\pi^{(j)} \in \mathbb{R}^{d_o}\) is a compact \emph{object pointer} that carries instance identity. Recent non-prompted frames are stored in a FIFO queue, while frames where the user interacted (prompts) are stored in a smaller, longer-lived queue so that corrections remain influential.
		
		\begin{itemize}
			\item \textbf{What is stored.} Let \(\hat{M}_t \in \{0,1\}^{H \times W}\) be the chosen mask for an instance at frame \(t\). A memory encoder \(g_{\text{mem}}\) gates the backbone features by the mask and projects them to key/value channels:
			\[
			\tilde{F}_t = F_t \odot \text{Down}(\hat{M}_t) \in \mathbb{R}^{C \times H \times W}, \qquad
			(K^{(t)}, V^{(t)}) = g_{\text{mem}}(\tilde{F}_t),
			\]
			where \(\odot\) is channel-wise multiplication and \(\text{Down}\) resamples \(\hat{M}_t\) to match the stride \(s\). In practice, \(g_{\text{mem}}\) is a small conv/MLP stack that compresses channels \(C \to d_k,d_v\) and flattens spatially to \(HW\) tokens. Each token thus summarizes appearance and position for a visible part of the object. The pointer \(\pi^{(t)}\) is derived from the decoder’s mask token for that instance (or a learned split of it); if an occlusion head predicts invisibility, a learned ``occluded'' embedding is added to \(\pi^{(t)}\) to mark that the object is temporarily not visible.
			
			\item \textbf{How memory is read.} Before decoding frame \(t\), keys and values from all entries in \(\mathcal{B}\) are concatenated:
			\[
			K_{\mathcal{B}} = [K^{(j)}]_{j \in \mathcal{J}} \in \mathbb{R}^{(HW|\mathcal{J}|) \times d_k}, \quad
			V_{\mathcal{B}} = [V^{(j)}]_{j \in \mathcal{J}} \in \mathbb{R}^{(HW|\mathcal{J}|) \times d_v}.
			\]
			Queries are obtained by projecting the flattened current-frame features, \(Q_t = \phi(F_t) \in \mathbb{R}^{HW \times d_k}\). A memory-attention stack computes
			\[
			\text{MemAttn}(F_t,\mathcal{B}) = \text{softmax}\!\Big(\tfrac{Q_t K_{\mathcal{B}}^\top}{\sqrt{d_k}} + \Psi_{\text{pos}}\Big)\, V_{\mathcal{B}},
			\]
			where \(\Psi_{\text{pos}}\) encodes 2D spatial (and short-range temporal) relations, typically via rotary or relative positional encodings. The object pointers \(\{\pi^{(j)}\}\) are broadcast and concatenated to each \((K^{(j)},V^{(j)})\) to bias attention toward tokens of the target instance. The result is a memory-conditioned feature map \(F'_t\) with the same shape as \(F_t\), which is then fed into the mask decoder. Keeping \(|\mathcal{J}|\) small (e.g., a few recent frames plus a few prompted ones) ensures predictable \(\mathcal{O}(HW \cdot |\mathcal{J}|)\) cost and stable attention.
			
			\item \textbf{How memory is written.} After decoding frame \(t\), the system selects one hypothesis per instance (typically the mask with highest predicted IoU) and then writes \((K^{(t)},V^{(t)},\pi^{(t)})\) into \(\mathcal{B}\), evicting the oldest unprompted entry if the bank is full. Because these tokens are derived from the same \(F_t\) and \(\hat{M}_t\) used by the decoder, every user correction immediately produces a new, informative memory entry that future frames can read. For multi-object tracking, each object maintains its own memory bank, while the image encoder and backbone features are shared across instances.
		\end{itemize}
		
		\noindent In summary, SAM~2 augments SAM’s promptable segmentation with a carefully designed streaming memory that enables efficient propagation, interactive recovery, and stable identities across time, all while preserving the same user-facing interface (points, boxes, masks) that made SAM broadly usable on images.
		
		\paragraph{Prompt encoder}
		We preserve SAM’s prompt vocabulary but make shapes explicit. For sparse prompts, a 2D point $p\!=\!(x,y)$ in pixel coordinates is normalized to $[-1,1]^2$, then mapped by random Fourier features $\gamma(p)\!=\![\sin(2\pi Bp),\cos(2\pi Bp)]\in\mathbb{R}^{2m}$ with $B\in\mathbb{R}^{m\times 2}$ sampled once~\cite{tancik2020_fourier}. The final point token is
		\[
		e_{\text{pt}}(p,\tau)=W_{\text{pt}}\,[\gamma(p)\,\|\,e_{\text{type}}(\tau)]\in\mathbb{R}^{d_p},
		\]
		where $\tau\in\{\text{fg},\text{bg},\text{pad}\}$ and $e_{\text{type}}$ is a learned embedding. Boxes are encoded by four corner points with distinct corner-type embeddings and optionally by $(x_c,y_c,w,h)$ as a second token. Dense prompts (masks) use a small conv projector $h$ to produce $D\!=\!C$-channel features at stride $s$, then \emph{add} to $F_t$:
		\[
		F_t^{\text{prompt}} = F_t + h(\text{Down}(M^{\text{in}})),
		\]
		so the mask acts as a soft spatial prior aligned to the backbone’s feature space~\cite{kirillov2023_sam}.
		
		\paragraph{Mask decoder with memory conditioning}
		Let $F'_t$ be the memory-conditioned map and $P_t$ the (optional) prompt tokens. The decoder is a compact transformer with \emph{two-way} token$\leftrightarrow$image attention as in SAM, extended with memory conditioning through $F'_t$:
		\begin{itemize}
			\item \emph{Token SA:} output tokens (three mask tokens $m_k$ and one IoU token $u$) and prompt tokens self-attend.
			\item \emph{Token$\rightarrow$image CA:} token queries attend to $F'_t$ (flattened) to gather spatial evidence (token-to-image).
			\item \emph{Image$\rightarrow$token CA:} image queries (from $F'_t$) attend to token keys to inject prompt/object context (image-to-token).
		\end{itemize}
		High-resolution skips from early encoder stages are fused late to restore detail. As in SAM, we emit up to $K\!=\!3$ mask logits $\{\hat{Y}_{t}^{(k)}\}\in\mathbb{R}^{H_\text{img}\times W_\text{img}}$ (after upsampling by light deconvs) and per-mask IoU scores $\{\hat{s}^{(k)}_t\}$. The \emph{mask token} state serves as the object pointer $\pi^{(t)}$ for memory writing. An auxiliary occlusion head (MLP on a dedicated token or pooled decoder state) predicts visibility $\hat{y}^{\text{occ}}_t\in[0,1]$ so invisible frames do not incur mask loss.
		
		\paragraph{Training objective and supervision}
		Following SAM, we supervise only the best hypothesis per frame (\emph{min-over-masks}). Let $k^\star=\arg\min_k \mathcal{L}_{\text{seg}}(\hat{Y}_{t}^{(k)},Y_t)$ with $\mathcal{L}_{\text{seg}}=\lambda_{\text{foc}}\mathcal{L}_{\text{focal}}+\mathcal{L}_{\text{Dice}}$. The total loss is
		\[
		\mathcal{L} = \underbrace{\mathcal{L}_{\text{seg}}(\hat{Y}_t^{(k^\star)},Y_t)}_{\text{only }k^\star}
		+ \lambda_{\text{IoU}}\!\sum_{k=1}^{K}\!\bigl\lVert \hat{s}^{(k)}_t - \text{IoU}(\hat{Y}_t^{(k)},Y_t)\bigr\rVert_1
		+ \lambda_{\text{occ}}\;\text{CE}\!\left(\hat{y}^{\text{occ}}_t,\,y^{\text{occ}}_t\right),
		\]
		skipping $\mathcal{L}_{\text{seg}}$ if $y^{\text{occ}}_t{=}1$. Prompts are simulated as in SAM: positive/negative clicks sampled inside/outside $Y_t$, jittered boxes from mask bounds, and dense prompts from prior predictions~\cite{kirillov2023_sam}. Crucially, training uses short clips $(t_1{<}\cdots{<}t_L)$ where early frames \emph{write} memory $(\hat{M}_{t_\ell}\!\to\!\mathcal{B})$ and later frames \emph{read} it ($F'_{t_{\ell+1}}\!\leftarrow\!\mathcal{B}$), mirroring deployment. Random temporal reversal (with probability $0.5$) regularizes for bi-directional propagation. We also apply \emph{teacher forcing} by occasionally writing ground-truth masks to memory to stabilize early training, and \emph{memory drop} (randomly masking entries in $\mathcal{B}$) to reduce over-reliance on any single view. SA-V clips with disappearance/reappearance provide explicit supervision for gap-robust propagation~\cite{ravi2024_sam2}.
		
		\paragraph{Pseudo-code for streaming interactive inference}
		\label{enr:par_chapter15_sam2_pseudocode}
		\begin{enumerate}
			\item \textbf{Initialize} $\mathsf{Mem} \leftarrow \emptyset$.
			\item \textbf{For} $t{=}1,\dots,T$:
			\begin{enumerate}
				\item[(a)] $F_t \leftarrow \texttt{ImageEncoder}(I_t)$.
				\item[(b)] $P_t \leftarrow \texttt{PromptEncoder}(\texttt{points/boxes/mask at }t)$ (optional).
				\item[(c)] $R_t \leftarrow \texttt{Select}(\mathsf{Mem})$.
				\item[(d)] $(\{\hat{M}_{t,j}\}, \{\hat{s}_{t,j}\}) \leftarrow \texttt{MaskDecoder}(F_t, R_t, P_t)$.
				\item[(e)] $j^\star \leftarrow \arg\max_j \hat{s}_{t,j}$, output $\hat{M}_t \leftarrow \hat{M}_{t,j^\star}$.
				\item[(f)] $\mathsf{Mem} \leftarrow \texttt{Update}\!\big(\mathsf{Mem}, \texttt{MemoryEncoder}(F_t, \hat{M}_t, P_t)\big)$.
			\end{enumerate}
		\end{enumerate}
		
		\subsubsection{Architecture \& Implementation Details}
		\label{enr:subsubsec_chapter15_sam2_arch}
		
		\begin{figure}[H]
			\centering
			\includegraphics[width=0.85\textwidth]{Figures/Chapter_15/SAM2_architecture.jpg}
			\caption{\textbf{Architecture.} Each frame is encoded once; memory tokens from prior frames are retrieved and fused with current features (and optional prompts) via a lightweight decoder to predict the mask. Predictions are transformed by a memory encoder for use in future frames. Credit: SAM~2~\cite{ravi2024_sam2}.}
			\label{fig:chapter15_sam2_arch}
		\end{figure}
		
		\noindent \textbf{Backbone} Large ViT-style encoders (e.g., Hiera variants) with MAE initialization produce a dense feature map per frame, reused within the frame.\footnote{Architectural variants and checkpoints are cataloged in the official repository~\cite{sam2_repo}.}
		
		\noindent \textbf{Prompt pathway} Identical to SAM for sparse and dense prompts, with random Fourier features for 2D coordinate encoding (Section~\ref{enr:par_chapter15_sam_posenc}).
		
		\noindent \textbf{Decoder} A compact transformer augments SAM's two-way attention with a memory cross-attention branch, outputting up to three masks and their IoU scores per frame.
		
		\noindent \textbf{Streaming memory} Memory tokens are kept in a rolling buffer with constant-time selection (e.g., windowed or top-$k$ retrieval) to preserve predictable per-frame cost. A memory encoder transforms the chosen prediction and frame features into new tokens.
		
		\subsubsection{Experiments and Ablations}
		\label{enr:subsubsec_chapter15_sam2_experiments}
		
		\paragraph{SA-V dataset and data engine}
		SAM~2 uses a model-in-the-loop engine extended to videos, producing SA-V with tens of millions of masks and hundreds of thousands of masklets. Qualitative examples appear in Figure~\ref{fig:chapter15_sam2_examples} and dataset statistics in Table~\ref{tab:chapter15_sam2_dataset_compare}. The data engine phases demonstrate decreasing clicks and time per frame as SAM~2 is folded into the loop (see the below data-engine table).
		
		\begin{figure}[H]
			\centering
			\includegraphics[width=0.70\textwidth]{Figures/Chapter_15/SAM2_example_videos.jpg}
			\caption{\textbf{SA-V qualitative examples.} Masklets overlaid on sample videos; each color denotes a distinct masklet. Frames are sampled at 1-second intervals. Credit: SAM~2~\cite{ravi2024_sam2}.}
			\label{fig:chapter15_sam2_examples}
		\end{figure}
		
		\begin{table}[H]
			\centering
			\scriptsize
			\caption{\textbf{Data engine phases.} Average annotation time per frame, percent of edited frames per masklet, clicks per clicked frame, and mask alignment to Phase~1 by size. Credit: SAM~2~\cite{ravi2024_sam2}.}
			\label{tab:chapter15_sam2_dataengine} 
			\begin{tabular}{lccccccc}
				\toprule
				\textbf{Model in loop} & \textbf{Time/frame} & \textbf{Edited frames} & \textbf{Clicks/} & \multicolumn{4}{c}{\textbf{Phase 1 Mask Alignment (IoU>0.75)}} \\
				\cmidrule(lr){5-8}
				& \textbf{(s)} & \textbf{(\%)} & \textbf{clicked frame} & \textbf{All} & \textbf{Small} & \textbf{Medium} & \textbf{Large} \\
				\midrule
				Phase 1 SAM only         & 37.8 & 100.00 & 4.80 & --   & --   & --   & --   \\
				Phase 2 SAM + SAM~2 Mask & 7.4  & 23.25  & 3.61 & 86.4 & 71.3 & 80.4 & 97.9 \\
				Phase 3 SAM~2            & \textbf{4.5} & \textbf{19.04} & \textbf{2.68} & \textbf{89.1} & \textbf{72.8} & \textbf{81.8} & \textbf{100.0} \\
				\bottomrule
			\end{tabular}
		\end{table}
		
		\begin{table}[H]
			\centering
			\scriptsize
			\caption{\textbf{Dataset comparison.} SA-V versus common VOS datasets. Disappearance rate indicates the fraction of frames where the object is absent. Credit: SAM~2~\cite{ravi2024_sam2}; benchmarks include DAVIS~\cite{ponttuset2017_davis}, YouTube-VOS~\cite{xu2018_youtubevos}, UVO~\cite{wang2021_uvo}, VOST~\cite{tokmakov2022_vost}, BURST~\cite{athar2023_burst}, and MOSE~\cite{ding2023_mose}.}
			\label{tab:chapter15_sam2_dataset_compare}
			\begin{tabular}{lrrrrrr}
				\toprule
				\textbf{Dataset} & \textbf{\#Videos} & \textbf{Duration (hr)} & \textbf{\#Masklets} & \textbf{\#Masks} & \textbf{\#Frames} & \textbf{Disapp.\ (\%)} \\
				\midrule
				DAVIS~2017 & 0.2K & 0.1 & 0.4K & 27.1K & 10.7K & 16.1 \\
				YouTube-VOS & 4.5K & 5.6 & 8.6K & 197.3K & 123.3K & 13.0 \\
				UVO-dense & 1.0K & 0.9 & 10.2K & 667.1K & 68.3K & 9.2 \\
				VOST & 0.7K & 4.2 & 1.5K & 175.0K & 75.5K & 41.7 \\
				BURST & 2.9K & 28.9 & 16.1K & 600.2K & 195.7K & 37.7 \\
				MOSE & 2.1K & 7.4 & 5.2K & 431.7K & 638.8K & 41.5 \\
				Internal & 62.9K & 281.8 & 69.6K & 5.4M & 6.0M & 36.4 \\
				SA-V Manual & 50.9K & 196.0 & 190.9K & 10.0M & 4.2M & 42.5 \\
				SA-V Manual+Auto & 50.9K & 196.0 & 642.6K & 35.5M & 4.2M & 27.7 \\
				\bottomrule
			\end{tabular}
		\end{table}
		
		\paragraph{Zero-shot semi-supervised VOS}
		SAM~2 outperforms decoupled SAM\,$+$\,tracker baselines across $17$ video datasets under various prompt types (see the below table). The gain is largest for low-click regimes, reflecting the value of memory for propagation.
		
		\begin{table}[H]
			\centering
			\scriptsize
			\caption{\textbf{Semi-supervised VOS: zero-shot accuracy across 17 video datasets.} Average accuracy for different first-frame prompts. In the ``ground-truth mask'' case, masks are passed directly to XMem++/Cutie without SAM. Credit: SAM~2~\cite{ravi2024_sam2}; baselines from XMem++~\cite{bekuzarov2023_xmempp} and Cutie~\cite{cheng2024_cutie}.}
			\label{tab:chapter15_sam2_vos}
			\begin{tabular}{lccccc}
				\toprule
				\textbf{Method} & \textbf{1-click} & \textbf{3-click} & \textbf{5-click} & \textbf{Box} & \textbf{GT mask} \\
				\midrule
				SAM + XMem++ & 56.9 & 68.4 & 70.6 & 67.6 & 72.7 \\
				SAM + Cutie  & 56.7 & 70.1 & 72.2 & 69.4 & 74.1 \\
				\textbf{SAM~2} & \textbf{64.7} & \textbf{75.3} & \textbf{77.6} & \textbf{74.4} & \textbf{79.3} \\
				\bottomrule
			\end{tabular}
		\end{table}
		
		\paragraph{Segment Anything across 37 datasets}
		Table~\ref{tab:chapter15_sam2_sa37} summarizes average 1- and 5-click mIoU on SA-23 (image) and $14$ additional zero-shot video datasets, along with throughput.
		
		\begin{table}[H]
			\centering
			\scriptsize
			\caption{\textbf{Segment Anything task across 37 datasets.} Average 1- and 5-click mIoU for SAM and SAM~2 on SA-23 and additional video datasets; FPS from the optimized video predictor. Credit: SAM~2~\cite{ravi2024_sam2}.}
			\label{tab:chapter15_sam2_sa37}
			\begin{tabular}{lccccc}
				\toprule
				\textbf{Model} & \textbf{Data} & \textbf{SA-23 All} & \textbf{SA-23 Image} & \textbf{SA-23 Video} & \textbf{14 New Video / FPS} \\
				\midrule
				SAM & SA-1B & 58.1 (81.3) & 60.8 (82.1) & 54.5 (80.3) & 59.1 (83.4) / 21.7 \\
				SAM~2 & SA-1B & 58.9 (81.7) & 60.8 (82.1) & 56.4 (81.2) & 56.6 (83.7) / \textbf{130.1} \\
				\textbf{SAM~2} & \textbf{Our mix} & \textbf{61.9 (83.5)} & \textbf{63.3 (83.8)} & \textbf{60.1 (83.2)} & \textbf{69.6 (85.8)} / \textbf{130.1} \\
				\bottomrule
			\end{tabular}
		\end{table}
		
		\newpage
		
		\paragraph{Ablations}
		We summarize the most decision-shaping empirical findings the authors report across the main paper and appendices, focusing on the pieces that guided design choices (metrics follow VOS convention: region/boundary mean $\text{J\&F}$; for images we use mIoU on the SA benchmarks).
		
		\textbf{Data mixture vs.\ architecture.} Training only on images (SA-1B) already yields higher \emph{image} mIoU for SAM~2 than SAM at substantially higher speed (e.g., with the Hiera-B+ image encoder: 58.9/81.7 1-/5-click mIoU vs.\ SAM ViT-H 58.1/81.3, while running $\sim6\times$ faster) and improves further when mixing videos (SA-V + internal + open VOS) to 61.4/83.7 and large gains on frames from video datasets (e.g., “14 new Video” average rises from 56.6/83.7 to 69.6/86.0). These tables isolate the \emph{data} contribution versus pure architecture, establishing that joint image+video training is key for transfer to video frames while retaining strong image performance.
		
		\textbf{Speed/accuracy operating points.} For semi-supervised VOS, the authors report real-time throughput with two encoder scales: Hiera-B+ at 43.8 FPS and Hiera-L at 30.2 FPS on a single A100 (batch size 1). The larger encoder improves accuracy across DAVIS/YTVOS/LVOS/SA-V, quantifying the classic capacity–speed trade-off and showing the decoder/memory remain light enough for interactive use.
		
		\textbf{Streaming memory design choices.}
		Appendix~C details the memory pathway used during ablations:
		\begin{itemize}
			\item \emph{Compact projections.} Memory features are projected to 64-D, and the 256-D mask token (object pointer) is split into four 64-D tokens for cross-attention.
			\item \emph{Position encoding.} Memory attention employs 2D RoPE for spatial (and short-range temporal) structure but excludes the pointer (no fixed spatial locus).
			\item \emph{Encoder reuse.} The memory encoder \emph{reuses} the image encoder’s embeddings instead of a second backbone.
		\end{itemize}
		These choices keep retrieval cheap and stable while improving long-horizon consistency. Although per-choice deltas are not tabulated as separate lines, these are the components retained in the final model after iterative experimentation.
		
		\textbf{Interactive robustness after failure cases.}
		In the online/interactive protocol, SAM~2’s ability to \emph{prompt at any frame} plus its streaming memory lets a single corrective click re-acquire objects after occlusion, unlike decoupled “SAM + tracker” pipelines that require re-annotation of full objects when drift occurs. Figure-level analyses (e.g., Fig.~2) explicitly compare the number and placement of clicks needed to recover, supporting the claim that memory is the dominant factor for robustness under occlusions/long motions in the interactive setting.
		
		\textbf{Dataset scale and coverage.}
		The SA-V data engine (50.9K videos, $\sim$642.6K masklets) is shown to be much larger and more diverse (disappearance rates, geography, parts vs.\ wholes) than prior VOS datasets—motivating why memory-based propagation is learnable at scale and why performance saturates for prior methods on SA-V while SAM~2 keeps improving.
		
		\smallskip\noindent\emph{Takeaway.}
		The evidence pattern is consistent:
		\begin{itemize}
			\item Joint image+video training establishes the base.
			\item The streaming memory pathway (with compact 64-D memories + object pointers + RoPE) translates that base into temporal robustness for occlusions/long motions.
			\item The decoder remains compact enough to preserve real-time throughput even at 1024-px inputs.
		\end{itemize}
		
		\newpage
		
		\subsubsection{Limitations and Future Directions}
		\label{enr:subsubsec_chapter15_sam2_limits}
		
		We restate the authors’ limitations in spirit and connect each to concrete directions the community has begun to pursue. For deeper treatments, see SAMuRAI~\cite{yang2024_samurai} (motion + memory selection for tracking), Grounded-SAM~\cite{ren2024_groundedsam} and its video-centric follow-up Grounded-SAM~2~\cite{groundedsam2_repo} (language-grounded detection/segmentation/tracking), and long-horizon memory variants such as SAM2Long~\cite{ding2024_sam2long}.
		
		\begin{itemize}
			\item \textbf{Memory selection at long horizons.} The model ``may fail to segment objects across shot changes and can lose track of or confuse objects in crowded scenes, after long occlusions or in extended videos''. This reflects a bounded, recency-biased FIFO memory that can evict rare but diagnostic past views. \emph{Next steps:} learned retention/retrieval policies and compact identity-aware state (e.g., evolving object vectors); explicit shot-change handling. See also \emph{SAM2Long}~\cite{ding2024_sam2long}, which explores training-free tree memories to keep multiple hypotheses over long videos.
			
			\item \textbf{Extreme appearance changes and fast motion.} Severe deformations, lighting shifts, or thin/fast structures can induce drift before correction. \emph{Next steps:} stronger temporal priors (optical flow cues; longer-range video transformers) and motion-aware selection. \emph{SAMuRAI}~\cite{yang2024_samurai} adds motion modeling and a motion-aware memory selection mechanism on top of SAM~2 for zero-shot tracking, improving robustness without fine-tuning.
			
			\item \textbf{Dense multi-object interactions.} Although SAM~2 can track multiple objects, independent per-object decoding can suffer identity swaps under heavy overlap or look-alike instances. \emph{Next steps:} joint, conflict-aware reasoning (e.g., shared object-level context/graph layers) and stronger identity cues. Language-grounded pipelines such as \emph{Grounded-SAM} and \emph{Grounded-SAM~2}~\cite{ren2024_groundedsam,groundedsam2_repo} help disambiguate identities with text-conditioned detection before segmentation/tracking.
			
			\item \textbf{Prompt dependence and ambiguity.} Ambiguous clicks can bias hypotheses; predicted IoU is a useful uncertainty signal but not a remedy. \emph{Next steps:} UI policies that surface low-IoU regions and actively \emph{guide} users to high-value clicks; integration with open-vocabulary grounding to replace ambiguous geometric prompts with unambiguous text prompts (cf.\ \emph{Grounded-SAM}~\cite{ren2024_groundedsam}).
			
			\item \textbf{Domain coverage.} Despite SA-V’s scale, niche modalities (thermal, medical, satellite) remain underrepresented. \emph{Next steps:} continued data-engine iteration with targeted mining/verification and domain-specific adapters; language-grounded retrieval (as in \emph{Grounded-SAM} families) can further lower annotation cost when scaling to new domains.
		\end{itemize}
		
		\bigskip
		\noindent\textit{Implementation note.} The official repository provides image/video predictors, checkpoints, notebooks, and an optimized video predictor with compiled kernels suitable for high-throughput VOS. The docs and Colab demonstrate interactive prompting, memory behavior, and speed/accuracy trade-offs out-of-the-box.
	
	\end{enrichment}
	
	\newpage
	
	\begin{enrichment}[Mask DINO: Unified DETR-Style Detection and Segmentation][subsection]
		\label{enr:chapter15_maskdino}
		
		\subsubsection{Motivation and Context}
		\label{subsubsec:chapter15_maskdino_context}
		
		Mask DINO~\cite{li2022_maskdino} is driven by a natural but ambitious question: Can one build a \emph{single} DETR-style Transformer that, under one architecture and one training recipe, is competitive with state-of-the-art detectors on COCO while also achieving state-of-the-art performance on the major segmentation tasks (instance, panoptic, semantic) on benchmarks such as COCO and ADE20K? Before Mask DINO, the strongest models in these regimes were largely specialized: DINO-DETR~\cite{zhang2022_dino} (built on DAB-DETR~\cite{liu2022_dab_detr} and DN-DETR~\cite{li2022_dn_detr}) on the detection side, and Mask2Former~\cite{cheng2022_mask2former} on the segmentation side. These models already share many ingredients (CNN or ViT backbones, multi-scale feature pyramids, Transformer encoders/decoders), but are each carefully tuned for their own objective and loss structure.
		
		A natural baseline is to \emph{add} a segmentation head on top of DETR or DINO-DETR and either fine-tune a detection-pretrained model or train the whole system from scratch in a multi-task fashion. In practice, both variants tend to be unsatisfactory.
		
		\paragraph{Fine-tuning a detector with an added mask head}
		Suppose we start from a DINO-DETR model pretrained for detection. Its decoder queries
		\[
		Q = \{q_i\}_{i=1}^{N_q}, \qquad q_i \in \mathbb{R}^d,
		\]
		have been optimized to support classification and box regression, i.e., to predict $(c_i, \mathbf{b}_i)$ where $\mathbf{b}_i \in \mathbb{R}^4$ is a coarse bounding box. Box losses encourage features that are good at capturing object extent and location, but do not explicitly enforce fine-grained, boundary-sensitive information. If we now attach a new mask head and fine-tune with an additional dense loss on masks $\mathbf{m}_i \in [0,1]^{H \times W}$, two problems arise:
		\begin{itemize}
			\item At the beginning of fine-tuning, the new mask head sees queries $q_i$ that are already specialized for boxes, not for detailed shapes. Early mask predictions are therefore poor, and the gradients from the mask loss attempt to substantially reshape $q_i$, in conflict with the existing detection objective.
			\item Even after long fine-tuning, the model typically converges to a compromise where queries remain mostly box-oriented and the mask head learns to produce only approximate object silhouettes. The masks can improve over time, but they tend to lag behind strong segmentation baselines because the upstream query semantics were never designed with fine pixel-level accuracy as a primary goal.
		\end{itemize}
		In short, simply “bolting on” a mask head after detection pretraining gives the mask branch too little influence over how queries are formed and used throughout the network.
		
		\paragraph{Training a unified detector+segmenter from scratch}
		Alternatively, one can train DETR or DINO-DETR from scratch with both detection and segmentation losses active from the beginning. Here, the queries $q_i$ are simultaneously pulled by a sparse, low-dimensional box loss (e.g., L1 and GIoU on $\mathbf{b}_i$) and a dense, high-dimensional mask loss (e.g., BCE or focal loss on $\mathbf{m}_i$). Without architectural mechanisms that explicitly couple how queries access spatial information, this often leads to:
		\begin{itemize}
			\item \emph{Noisy early supervision:} Hungarian matching is typically dominated by box and class terms, so early in training the queries are encouraged to specialize for box-level localization first. Masks are then supervised on queries whose spatial alignment is still unstable, making mask gradients noisy and hard to use effectively.
			\item \emph{Gradient conflict:} Box regression pushes queries toward features that summarize overall object geometry, while mask prediction pushes toward features that resolve local boundaries and textures. In a naïve multi-head setup these signals are not coordinated, so the model can settle at a compromise where neither detection nor segmentation reaches the level of specialized methods.
		\end{itemize}
		Running two separate models (a detector and a segmenter) avoids some of these issues but is computationally expensive and still fails to exploit potential synergies: detection predictions are not explicitly used to guide masks, and masks do not feed back to improve box localization or query selection.
		
		\paragraph{Mask DINO: aligning detection and segmentation at the query level}
		Mask DINO addresses these issues by taking DINO-DETR as its detection backbone and adding a \emph{tightly integrated} segmentation branch instead of an independent head. Concretely, it brings in the Mask2Former idea of predicting masks via dot-products between query embeddings and a high-resolution pixel embedding map, while keeping DINO-DETR's detection-oriented machinery (dynamic anchor boxes, mixed query selection, contrastive denoising, multi-scale deformable attention) largely intact. The core design principle is to \emph{align} detection and segmentation at the level of queries and features, so that both tasks are driven by the same semantics and trained jointly from the earliest layers onward.
		
		Formally, let
		\[
		Q = \{q_i\}_{i=1}^{N_q}, \qquad q_i \in \mathbb{R}^d,
		\]
		denote the set of decoder content queries produced and refined by DINO-DETR (with $N_q$ queries and hidden dimension $d$). Let
		\[
		E \in \mathbb{R}^{H_4 \times W_4 \times d}
		\]
		denote a stride-4 pixel embedding map constructed from backbone and encoder features (the precise construction of $E$ will be described later). Mask DINO makes the key decision to \emph{reuse} these refined content queries as mask queries: the same token $q_i$ that predicts an object's category $c_i$ and bounding box $\mathbf{b}_i$ is also responsible for its mask $\mathbf{m}_i$.
		
		A lightweight mask head turns $q_i$ into a mask embedding, and mask logits at stride~4 are obtained by per-location inner products
		\[
		\ell_i(x,y) = \langle q_i, E(x,y) \rangle,
		\]
		followed by upsampling and a sigmoid to yield a full-resolution mask $\mathbf{m}_i$. In parallel, the same $q_i$ is fed to a classification head to produce $c_i$ and to a box head to produce $\mathbf{b}_i$.
		
		Operationally, this yields a conceptually simple unified architecture in which a single set of queries feeds three heads in parallel---a class head, a box head, and a mask head---so that $(c_i,\mathbf{b}_i,\mathbf{m}_i)$ are all anchored to the same underlying query semantics and benefit from shared supervision. Detection-oriented components such as dynamic anchors, query denoising, and multi-scale deformable attention improve the quality and localization of queries, which in turn sharpens masks; conversely, dense mask losses help refine query semantics, which can improve classification and box regression. The remainder of this section explains how Mask DINO inherits and adapts components from DAB-DETR and DINO-DETR on the detection side, and from Mask2Former on the segmentation side, to realize this unified design and to substantially strengthen naïve ``add-a-head'' baselines.
		
		\subsubsection{From DAB-DETR and DINO-DETR to Mask DINO}
		
		\paragraph{Dynamic anchor boxes in DAB-DETR}
		Vanilla DETR represents each decoder query as a learned content embedding plus a fixed sinusoidal positional encoding. In this design, a single vector must \emph{implicitly} discover both what object it represents and where that object is likely to be, purely through end-to-end training. Queries therefore start ``geometry-agnostic'': early in training they attend rather uniformly over the feature map, and the model only gradually learns to concentrate attention near object extents. This makes localization hard to optimize, slows convergence, and makes performance sensitive to initialization and training schedule.
		
		DAB-DETR~\cite{liu2022_dab_detr} makes this positional component \emph{explicit and refinable} by turning it into a \emph{4D anchor box} that is updated at every decoder layer. Concretely, each query $i$ at decoder layer $\ell$ is represented as a pair
		\[
		q_i^{(\ell)} \in \mathbb{R}^d, \qquad
		a_i^{(\ell)} = \bigl(c_x^{(\ell)}, c_y^{(\ell)}, w^{(\ell)}, h^{(\ell)}\bigr) \in [0,1]^4,
		\]
		where $q_i^{(\ell)}$ is a content embedding encoding ``what'' the query looks for (appearance and semantics), and $a_i^{(\ell)}$ is a normalized box encoding ``where'' this query currently believes its object lies (box center and size, relative to the image).
		
		At each decoder layer, a shallow MLP predicts an \emph{offset} to the previous anchor,
		\[
		\Delta a_i^{(\ell)} = f_{\text{box}}\!\bigl(q_i^{(\ell-1)}\bigr), \qquad
		a_i^{(\ell)} = a_i^{(\ell-1)} + \Delta a_i^{(\ell)},
		\]
		so boxes are refined \emph{coarse-to-fine} across layers rather than being regressed in a single step from a fixed prior. This iterative refinement has two important consequences:
		\begin{itemize}
			\item Each layer only needs to predict a \emph{small correction} to an existing box hypothesis, which is an easier optimization problem and leads to smoother gradients than one-shot regression from scratch.
			\item Early layers can focus on rapidly moving anchors from generic priors toward roughly correct regions, while later layers spend their capacity on tightening boxes around object boundaries.
		\end{itemize}
		In practice this significantly accelerates training and improves detection quality compared to vanilla and Conditional DETR.
		
		The updated anchor $a_i^{(\ell)}$ is then converted into a positional embedding (via a sinusoidal mapping) and added to the content query $q_i^{(\ell)}$ before cross-attention, so that each query carries both semantic and geometric information. In the DAB-DETR implementation built on Deformable DETR, these anchors play a second, crucial role: their centers are used as \emph{reference points} for multi-scale deformable attention. Rather than attending over \emph{all} spatial locations in each feature map, each query samples only a \emph{small} set of offsets around its current anchor, across multiple FPN levels. For a given query, attention thus operates on a fixed budget of $M$ sampling points per head and per scale, whose positions are predicted relative to the anchor center. Larger, less precise anchors induce broader sampling patterns; as anchors are refined, the sampling region narrows around the object.
		
		This ``anchor-as-reference'' mechanism is both statistically and computationally attractive:
		\begin{itemize}
			\item Statistically, it encodes the idea that evidence for an object should be found near its current box hypothesis, and that different scales should be consulted depending on the box size.
			\item Computationally, it reduces the complexity of cross-attention from $O(N_q \cdot H W)$ dense dot-products (queries against all spatial positions) to $O(N_q \cdot N_{\text{levels}} \cdot M)$ sampled positions, which can be orders of magnitude smaller for typical feature map sizes.
		\end{itemize}
		As a result, queries no longer waste attention on irrelevant regions: they use their anchors as geometric ``compasses'' that determine \emph{where} to probe the multi-scale feature pyramid.
		
		\begin{figure}[H]
			\centering
			\includegraphics[width=0.8\textwidth]{Figures/Chapter_15/DAB_DeTR_comparisons.jpg}
			\caption{\textbf{From DETR to DAB-DETR.} DAB-DETR replaces DETR's purely learned positional queries with 4D anchor boxes that are iteratively refined and used to guide cross-attention. This explicit geometric prior leads to faster convergence and stronger detection compared to vanilla and Conditional DETR.}
			\label{fig:chapter15_dab_detr_comparisons}
		\end{figure}
		
		\noindent
		Architecturally, DAB-DETR preserves the DETR backbone--encoder--decoder layout. 
				
		\begin{figure}[H]
			\centering
			\includegraphics[width=0.6\textwidth]{Figures/Chapter_15/DAB_DeTR_architecture.jpg}
			\caption{\textbf{DAB-DETR architecture.} Each decoder query consists of a content embedding and an associated 4D anchor box. At every decoder layer, the anchor is refined by a box head and the updated box parameters define reference points and sampling patterns that guide multi-scale deformable cross-attention.}
			\label{fig:chapter15_dab_detr_architecture}
		\end{figure}
		
		Nevertheless, it augments the decoder in two ways (as can be seen in the figure~\ref{fig:chapter15_dab_detr_architecture}). 
		
		\noindent \medskip
		First, decoder queries are now pairs of content embeddings and dynamic anchors, with per-layer box heads that refine anchors via residual updates. 
		
		\noindent \medskip
		Second, multi-scale deformable cross-attention is parameterized by the current anchors: their centers define reference points from which a small set of sample locations is predicted and used to read from the multi-scale feature maps. The rest of the pipeline (backbone feature extraction, encoder processing of multi-scale features) remains unchanged. 
		
		\noindent \medskip
		For Mask DINO, this design is fundamental: the same dynamically refined anchors that make DAB-DETR and DINO-DETR queries geometrically well-localized for detection will later serve as strong spatial priors when those queries are reused for mask prediction.
		
		\paragraph{DINO-DETR: improved denoising and query mechanics}
		While DAB-DETR equips each query with a progressively refined anchor box and anchor-guided deformable attention, it still largely relies on randomly initialized query embeddings and a single, somewhat brittle training signal from the Hungarian-matched detection loss. DINO-DETR~\cite{zhang2022_dino} builds directly on DAB-DETR and DN-DETR~\cite{li2022_dn_detr} to address these shortcomings. It preserves dynamic anchor boxes and multi-scale deformable cross-attention, but strengthens the \emph{training dynamics} and \emph{query initialization} via three key ideas:
		\begin{itemize}
			\item \textbf{Contrastive denoising (CDN)} to provide a stable, auxiliary reconstruction task that accelerates convergence and improves robustness.
			\item \textbf{Mixed query selection} that uses encoder outputs as data-dependent priors for decoder queries, avoiding purely ``cold-start'' learnable queries.
			\item A \textbf{``look-forward-twice''} box update mechanism that smooths the gradient path through the box regression heads and leads to more accurate localization.
		\end{itemize}
		
		\medskip
		Together, these make decoded queries not only well-localized (thanks to DAB-style anchors) but also semantically stronger and more robust---properties that Mask DINO will later exploit when reusing these queries for mask prediction.
		
		\begin{figure}[H]
			\centering
			\includegraphics[width=0.85\textwidth]{Figures/Chapter_15/DINO_DeTR_architecture.jpg}
			\caption{\textbf{DINO architecture.} DINO-DETR inherits the DAB-DETR backbone--encoder--decoder structure and multi-scale deformable cross-attention, while adding denoising, mixed query selection, and a refined box update strategy. These changes improve training stability, convergence speed, and detection accuracy, and form the detection backbone on which Mask DINO builds.}
			\label{fig:chapter15_dino_architecture}
		\end{figure}
		
		\paragraph{Contrastive denoising (CDN): a stable auxiliary objective}
		In DETR-style models, the main supervision comes from Hungarian-matched predictions: each decoder layer produces a fixed set of queries, and a bipartite matching assigns some of them to ground-truth objects while the rest become ``no object''. Early in training, when predictions are essentially random, this matching is unstable and gradients are noisy; queries receive weak, highly variable signals and learning is slow.
		
		DN-DETR~\cite{li2022_dn_detr} and DINO-DETR~\cite{zhang2022_dino} address this by adding a \emph{denoising branch} alongside the usual ``free'' (detection) queries. The training queries are split into two groups:
		\begin{itemize}
			\item \textbf{Detection queries}, which behave as in standard DETR: they start from learned embeddings and anchors, are matched to ground truth via the Hungarian algorithm, and are trained to \emph{discover} objects from scratch.
			\item \textbf{Denoising queries}, which are constructed directly from ground-truth box--label pairs and are trained to \emph{reconstruct} the clean targets from corrupted versions.
		\end{itemize}
		
		Concretely, for each ground-truth box--label pair $(\mathbf{b}, c)$, the model samples one or more \emph{noised copies}
		\[
		\tilde{\mathbf{b}} = \mathbf{b} + \delta_{\text{box}}, \qquad
		\tilde{c} = c + \delta_{\text{cls}},
		\]
		where $\delta_{\text{box}}$ jitters the box center and size (within a controlled range) and $\delta_{\text{cls}}$ may randomly flip the class to a nearby or ``wrong'' category. Each corrupted pair $(\tilde{\mathbf{b}}, \tilde{c})$ is then encoded as a denoising query (content embedding plus anchor initialized from $\tilde{\mathbf{b}}$) and passed through the same decoder as the detection queries. For these denoising queries, the supervision is \emph{direct}: the loss compares their outputs to the \emph{clean} ground-truth $(\mathbf{b}, c)$ from which they were generated, without any Hungarian matching.
		
		This auxiliary task is intentionally \emph{easier} than full detection: the model is told which approximate location and (possibly noisy) label to start from, and only needs to ``pull'' them back to the correct object. However, it does \emph{not} leak information at inference time or make the overall problem trivial:
		\begin{itemize}
			\item At test time, there are no denoising queries and no ground-truth boxes; the decoder runs only on detection queries, which still must localize and classify objects from scratch using the standard detection loss.
			\item During training, detection and denoising queries share the \emph{same} decoder weights and heads. Gradients from the denoising branch therefore shape the shared representation: they teach the decoder how to refine noisy, roughly located hypotheses into accurate boxes and labels, which directly benefits the harder detection queries once they start receiving meaningful matches.
			\item Because $\tilde{\mathbf{b}}$ can be perturbed by different magnitudes and $\tilde{c}$ can be corrupted, the model learns robustness to a range of geometric and semantic errors rather than memorizing exact ground-truth positions.
		\end{itemize}
		
		DINO-DETR further strengthens this idea with a \emph{contrastive} formulation: denoising queries are organized so that each one is encouraged not only to match its own ground-truth target, but also to be clearly better (lower loss) for that target than for other ground truths in the same batch. This contrastive pressure sharpens the learned representation and reduces confusion between nearby objects.
		
		Overall, the contrastive denoising (CDN) module supplies strong, stable gradients from the very beginning of training and makes queries robust to perturbations in both geometry and semantics. In the context of Mask DINO, these properties are crucial: when denoising is extended from boxes to masks, queries must learn to recover fine-grained shapes starting from noisy box-based hints, and CDN provides exactly the kind of robust refinement behavior that this requires.
		
		\begin{figure}[H]
			\centering
			\includegraphics[width=0.85\textwidth]{Figures/Chapter_15/DINO_DeTR_CDN.jpg}
			\caption{\textbf{Contrastive denoising in DINO-DETR.} During training, a subset of queries is reserved for denoising: they are initialized from noised ground-truth boxes and labels and are trained to reconstruct the original objects without Hungarian matching. These denoising queries share the decoder with the standard detection queries, providing a stable auxiliary objective that accelerates convergence and makes the learned queries robust to geometric and semantic perturbations.}
			\label{fig:chapter15_dino_cdn}
		\end{figure}
		
		\paragraph{Mixed query selection: encoder priors for an image-aware warm start}
		DAB-DETR equips each decoder query with a dynamic anchor, but the \emph{initial} queries are still \emph{image-agnostic}: at the first decoder layer they are generated from a fixed set of learned content embeddings and anchors that are identical for every image. The decoder must therefore discover, purely through end-to-end training, which queries should correspond to which objects in a new image. Early in training this leads to many queries drifting in empty background regions, unstable Hungarian matches, and slow convergence.
		
		DINO-DETR~\cite{zhang2022_dino} replaces this ``cold start'' by using the encoder as an \emph{image-specific proposal generator}. After the encoder has processed the multi-scale features, we obtain a set of encoder tokens
		\[
		\{e_j\}_{j=1}^{N_e}, \qquad e_j \in \mathbb{R}^d,
		\]
		indexed over all spatial locations and FPN levels. Lightweight heads are applied to these tokens to predict, for each $e_j$,
		\begin{itemize}
			\item A classification score vector $\hat{p}_j$ (probability over categories).
			\item A box prediction $\hat{a}_j = (\hat{c}_{x,j}, \hat{c}_{y,j}, \hat{w}_j, \hat{h}_j)$.
		\end{itemize}
		These act as \emph{dense, coarse proposals}: they tell us which encoder locations already look object-like and what rough boxes they suggest.
		
		\emph{Mixed query selection} then constructs the decoder's initial queries from a mixture of these encoder proposals and a smaller pool of purely learned queries:
		\begin{enumerate}
			\item Score each encoder token $e_j$ with a scalar objectness measure (for example, the maximum foreground class probability derived from $\hat{p}_j$).
			\item Select the top $K$ tokens according to this score; these are the encoder's ``hot spots'' that are most likely to contain objects.
			
			\newpage
			
			\item For each selected token $e_j$:
			\begin{itemize}
				\item Use its feature $e_j$ to initialize a decoder \emph{content query} $q^{(0)}_i$.
				\item Use its predicted box $\hat{a}_j$ to initialize the associated anchor $a^{(0)}_i$.
			\end{itemize}
			\item Fill the remaining decoder slots with a small number of image-agnostic learned queries (both content and anchors) to preserve flexibility and allow discovery of objects missed by the encoder proposals.
		\end{enumerate}
		
		The result is that most decoder queries at layer~0 no longer start as generic, image-independent ``slots'' scattered over the feature maps. Instead, they are \emph{image-aware}: their content embeddings and anchors are initialized from locations and appearances that the encoder already believes correspond to objects. The decoder's role becomes primarily \emph{refinement}: sharpen these coarse proposals, resolve overlaps and duplicates, and correct mistakes, rather than searching blindly over the entire image.
		
		This has two concrete benefits:
		\begin{itemize}
			\item \textbf{Faster and more stable training.} From the very first epochs, many queries are already near true objects, so Hungarian matching becomes less random and gradients are less noisy. Empirically, this accelerates convergence and improves final box AP compared to starting all queries from learned, image-agnostic embeddings.
			\item \textbf{Better queries for downstream tasks.} Because the decoder refines proposals that already roughly localize objects, the resulting query embeddings tend to be semantically meaningful and spatially well grounded. For Mask DINO, which reuses these same queries for mask prediction, this is crucial: many queries that will later ``paint'' masks already correspond to plausible object candidates rather than arbitrary background locations.
		\end{itemize}
		
		\begin{figure}[H]
			\centering
			\includegraphics[width=0.85\textwidth]{Figures/Chapter_15/DINO_DeTR_query_init.jpg}
			\caption{\textbf{Query initialization in DINO-DETR.} After the encoder, lightweight classification and box heads score each encoder token. Mixed query selection chooses the top-ranked tokens and uses their features and predicted boxes to initialize most decoder content queries and anchors. A few learned queries are mixed in for diversity. This data-dependent, spatially grounded ``warm start'' makes decoder refinement easier and faster than starting from purely learned, image-agnostic queries.}
			\label{fig:chapter15_dino_query_init}
		\end{figure}
		
		\newpage
		
		\paragraph{``Look-forward-twice'' box updates: shorter gradient paths for better localization}
		DAB-DETR already refines anchors layer by layer, but supervision for boxes is still relatively indirect: the strongest box losses are applied on the final decoder outputs, and their gradients must backpropagate through the entire stack of decoder layers and intermediate anchor updates. As depth grows, this long gradient path can weaken the learning signal reaching early layers, especially for the box regression branch, making it harder to steadily improve coarse localization.
		
		DINO-DETR introduces a simple but effective modification, often described as ``look-forward-twice''. At each decoder layer $\ell$, the box head does not emit a single box prediction, but instead produces two closely related outputs for each query:
		\begin{itemize}
			\item an \emph{intermediate} box $\mathbf{b}_i^{\text{inter},(\ell)}$ that is used to update the anchor for the next layer, and
			\item a \emph{refined} box $\mathbf{b}_i^{\text{ref},(\ell)}$ that is supervised directly by the box regression loss.
		\end{itemize}
		The intermediate box preserves the residual refinement view introduced in DAB-DETR: it is added to the current anchor $a_i^{(\ell)}$ to form the next-layer anchor $a_i^{(\ell+1)} = a_i^{(\ell)} + \mathbf{b}_i^{\text{inter},(\ell)}$. The refined box, in contrast, is not fed forward but is explicitly compared to ground-truth boxes via the usual L1 and IoU-based losses at \emph{that} decoder layer.
		
		This dual prediction effectively creates a short, direct gradient path from the box loss at layer~$\ell$ back to the parameters of that layer's box head and to its contributing query features. Instead of relying solely on losses applied at the very end of the decoder, every layer receives its own box-level supervision through $\mathbf{b}_i^{\text{ref},(\ell)}$, while $\mathbf{b}_i^{\text{inter},(\ell)}$ continues to drive the iterative anchor refinement. In practice, this improves gradient flow through the box regression branch, stabilizes training, and yields more accurate localization, particularly for small or thin objects where precise box edges matter.
		
		For Mask DINO, this refinement is important because the same queries and anchors that benefit from DINO's ``look-forward-twice'' design are later reused as spatial priors for mask prediction. Better-behaved, well-localized boxes mean that the queries start their mask prediction from anchors already close to the true object extent, so the subsequent mask head and pixel embedding map can focus on sharpening boundaries rather than compensating for large geometric errors.
		
		\begin{figure}[H]
			\centering
			\includegraphics[width=0.85\textwidth]{Figures/Chapter_15/DINO_DeTR_Box_Update.jpg}
			\caption{\textbf{Box update in DINO-DETR.} Each decoder layer predicts both an intermediate box (used to update the anchor for the next layer) and a refined box (supervised directly by the box regression loss at that layer). This ``look-forward-twice'' design shortens gradient paths for box supervision and leads to more accurate, stable localization---a property that Mask DINO later exploits when reusing these queries and anchors for mask prediction.}
			\label{fig:chapter15_dino_box_update}
		\end{figure}
		
		\newpage
		
		\subsubsection{From Mask2Former to Mask DINO}
		\label{subsubsec:chapter15_mask2former_to_maskdino}
		
		The detection-oriented lineage culminating in DAB-DETR and DINO-DETR is built around \emph{box-savvy queries}: each decoder query carries a dynamic 4D anchor box, and multi-scale deformable cross-attention lets it pull just enough context from the feature pyramid to localize objects with high-quality bounding boxes. However, segmentation requires more than tight boxes; it needs pixel-level shapes and boundaries. Mask2Former~\cite{cheng2022_mask2former} attacks this problem from the opposite direction: it reinterprets queries as \emph{semantic projectors} that ``paint'' dense masks over a high-resolution pixel embedding map via simple dot-products. Mask DINO is explicitly inspired by this idea and can be viewed as a fusion of DINO-DETR's geometric machinery with Mask2Former's unified, mask-centric segmentation pipeline.
		
		\subsubsection{From Mask2Former to Mask DINO}
		\label{subsubsec:chapter15_mask2former_to_maskdino}
		
		The detection-oriented lineage culminating in DAB-DETR and DINO is built around \emph{box-savvy queries}: each decoder query carries a dynamic 4D anchor box, and multi-scale deformable cross-attention lets it pull just enough context from a feature pyramid to localize objects with high-quality bounding boxes. However, segmentation requires more than tight boxes; it needs pixel-level shapes and boundaries. Mask2Former~\cite{cheng2022_mask2former} attacks this problem from the opposite direction: it reinterprets queries as \emph{semantic projectors} that ``paint'' dense masks over a high-resolution pixel embedding map via simple dot-products. Mask DINO is explicitly inspired by this idea and can be viewed as a fusion of DINO-DETR's geometric machinery with Mask2Former's unified, mask-centric segmentation pipeline.
		
		\paragraph{Mask2Former: queries as semantic projectors for unified segmentation}
		
		Mask2Former treats segmentation as set prediction over mask--class pairs, using a fixed set of abstract, learnable queries
		\[
		Q = \{q_i\}_{i=1}^{N_q}, \qquad q_i \in \mathbb{R}^d.
		\]
		Each query $q_i$ is a slot that is trained to represent one ``thing'' instance or one ``stuff'' region. The architecture has three tightly coupled components:
		
		\begin{itemize}
			\item \textbf{Backbone and pixel decoder: multi-scale pyramid and stride-4 pixel map.} A CNN/ViT backbone first produces feature maps at several strides
			\[
			\{F_s\}_{s \in \{4,8,16,32\}}, \qquad F_s \in \mathbb{R}^{C_s \times H/s \times W/s},
			\]
			where $s$ is the downsampling factor relative to the input resolution $H \times W$. The pixel decoder (an FPN-style module) then:
			\begin{enumerate}
				\item Projects each backbone map to a common channel dimension $d$ via $1\times 1$ convolutions, giving $\tilde F_s = W_s * F_s$.
				\item Fuses them in a top-down manner:
				\[
				G_{32} = \tilde F_{32}, \qquad
				G_s = \tilde F_s + \operatorname{Upsample}(G_{2s}) \quad (s=16,8,4),
				\]
				where $\operatorname{Upsample}$ is typically bilinear upsampling to the spatial resolution of $F_s$. This builds a coherent multi-scale pyramid $\{G_s\}$ that combines high-level semantics (from coarse maps) with fine spatial detail (from shallow maps).
				\item Optionally refines each $G_s$ with \emph{multi-scale deformable attention} (MS-DeformAttn) to obtain $G_s'$: at every spatial location, MS-DeformAttn uses that location as a query and samples a small number of adaptive points across all scales, producing a content-adaptive blend of pyramid features with $O(M)$ samples instead of $O(HW)$ dense attention.
			\end{enumerate}
			The finest refined map, which we denote $G_4' \in \mathbb{R}^{H/4 \times W/4 \times d}$, is singled out as the \emph{pixel embedding map}. In Mask2Former this is often written as $F_4$; in the Mask DINO context we will later rename the same object to $E$ to emphasize its role as a generic pixel embedding map. It lives at stride~4: high enough resolution to capture detailed boundaries, yet downsampled enough for efficient dense computation.
			
			\item \textbf{Transformer decoder with masked attention: queries carve out regions.} A Transformer decoder takes the query set $Q$ and lets the queries iteratively refine themselves by attending to the multi-scale feature maps (typically the $\{G_s'\}$). The distinctive ingredient is \emph{masked cross-attention}: at decoder layer $\ell$, each query $q_i^{(\ell)}$ is associated with a current estimated mask on the stride-4 grid. This mask is used to restrict the spatial positions that the query is allowed to attend to in the next cross-attention step. Intuitively, the query starts with a broad receptive field and, layer by layer, focuses its attention onto the pixels that it believes belong to its own region. This mechanism helps different queries specialize to different objects or stuff regions and stabilizes training by reducing interference between queries.
			
			\item \textbf{Dot-product mask head: projecting queries onto the pixel map.} After several decoder layers, each refined query $\hat q_i$ is mapped through a small mask-embedding head to a vector $\hat m_i \in \mathbb{R}^d$. The stride-4 pixel embedding map $G_4'$ is treated as a dense grid of $d$-dimensional vectors. Mask logits are obtained by per-location dot-products:
			\[
			\ell_i(x,y) = \langle \hat m_i, G_4'(x,y) \rangle, \qquad (x,y) \in \{1,\dots,H/4\} \times \{1,\dots,W/4\},
			\]
			and then upsampled (typically bilinearly by a factor of~4) and passed through a sigmoid to produce a soft mask
			\[
			m_i = \sigma\bigl(\operatorname{Upsample}(\ell_i)\bigr) \in [0,1]^{H \times W}.
			\]
			In parallel, a classification head applied to $\hat q_i$ predicts a semantic label $c_i$. Geometrically, $\hat m_i$ defines a direction in the $d$-dimensional embedding space; pixels whose embeddings $G_4'(x,y)$ align with that direction receive high logit values and are included in the mask. The same set of $(c_i, m_i)$ pairs can be supervised as instance masks, panoptic segments, or semantic regions by changing only the loss and aggregation logic.
		\end{itemize}
		
		\medskip \noindent
	 	Putting these components together, the Mask2Former pipeline can be viewed as a coherent flow from raw pixels to mask--class pairs. An input image
	 	\[
	 	I \in \mathbb{R}^{H \times W \times 3}
	 	\]
	 	is mapped by the backbone to a multi-scale feature pyramid $\{F_s\}_{s \in \{4,8,16,32\}}$. The pixel decoder then reshapes this pyramid into a set of task-ready maps $\{G_s'\}$ and, in particular, into a stride-4 pixel embedding map $G_4' \in \mathbb{R}^{H/4 \times W/4 \times d}$ that serves as a dense, high-resolution canvas for segmentation.
	 	
	 	\newpage
	 	
	 	A fixed set of queries $Q$ enters the Transformer decoder, where masked cross-attention lets each query iteratively carve out its own region by repeatedly attending to $\{G_s'\}$ under the guidance of its current mask. After several layers, the refined queries are converted into mask embeddings, and a simple dot-product between each mask embedding and the stride-4 pixel map $G_4'$ produces low-resolution mask logits, which are finally upsampled and thresholded to yield full-resolution masks $m_i(x,y)$, with a parallel head predicting class labels $c_i$.
	 	
	 	\begin{figure}[H]
	 		\centering
	 		\includegraphics[width=0.85\textwidth]{Figures/Chapter_15/Mask2Former_architecture.jpg}
	 		\caption{\textbf{Mask2Former architecture.} An input image \(I \in \mathbb{R}^{H \times W \times 3}\) is mapped by a backbone to a multi-scale feature pyramid \(\{F_s\}\). A pixel decoder refines these into \(\{G_s'\}\) and, in particular, into a stride-4 pixel embedding map \(G_4' \in \mathbb{R}^{H/4 \times W/4 \times d}\) (often denoted \(F_4\) in the original paper, and later reused as \(E\) in Mask DINO). A fixed set of learnable queries \(Q \in \mathbb{R}^{N_q \times d}\) interacts with \(\{G_s'\}\) via masked attention in a Transformer decoder, producing refined query embeddings \(\hat Q\). A mask-embedding head maps each \(\hat q_i\) to a vector \(\hat m_i\), whose dot-products with \(G_4'\) yield mask logits \(\ell_i\), upsampled and sigmoided into dense masks \(m_i \in [0,1]^{H \times W}\), while a parallel classification head predicts labels \(c_i\).}
	 		\label{fig:chapter15_mask2former_architecture}
	 	\end{figure}
	 	
	 	From the perspective of DAB-DETR and DINO-DETR, Mask2Former innovates by \emph{densifying} the prediction pipeline. Rather than equipping queries with explicit 4D anchors and training them primarily to regress boxes in \(\mathbb{R}^4\), it reallocates capacity toward:
	 	\begin{itemize}
	 		\item A strong pixel decoder that builds a high-quality multi-scale pyramid and a stride-4 pixel embedding map suited to pixel-level decisions.
	 		\item MS-DeformAttn-based multi-scale fusion inside the pixel decoder, which allows each spatial location to aggregate a small number of informative samples across all scales instead of performing dense attention over all positions, and
	 		\item A dot-product mask head that treats each query (after a small mask-embedding MLP) as a semantic direction in the embedding space and uses that direction to assign labels to pixels on the stride-4 grid.
	 	\end{itemize}
	 	In this formulation, queries are no longer geometric anchors tied to 4D boxes; they are semantic projectors that decide which pixels on a high-resolution embedding canvas belong to which region.
	 	
	 	\paragraph{The architectural gap between DINO-DETR/DAB-DETR and Mask2Former}
	 	
	 	Despite sharing many underlying components (CNN/ViT backbones, multi-scale pyramids, Transformer encoders/decoders), the two families crystallize around different objectives and therefore make different design choices:
	 	
	 	\begin{itemize}
	 		\item \textbf{DINO-DETR and DAB-DETR: box-first, sparse geometry.} Decoder queries carry dynamic 4D anchor boxes and interact with features via multi-scale deformable attention guided by reference points. Training emphasizes accurate box regression and classification, supported by contrastive denoising, mixed query selection, and per-layer box refinement. There is no pixel decoder producing a dedicated stride-4 embedding map for masks, no dot-product mask head, and no mechanism to turn each query into a dense mask \(m_i(x,y)\) over the image plane.
	 		
	 		\item \textbf{Mask2Former: mask-first, dense semantics.} Decoder queries are purely semantic (DETR-style content plus positional encodings) and do not maintain explicit 4D anchors. Masked attention and the pixel decoder are optimized to refine masks on the stride-4 grid, not to iteratively refine bounding boxes. As a result, Mask2Former achieves state-of-the-art segmentation quality, but its box localization lags behind anchor-based detectors such as DINO-DETR and DAB-DETR: it lacks dynamic anchors, per-layer box updates, and a denoising scheme tailored to box regression.
	 	\end{itemize}
	 	
	 	These contrasting design commitments create a clear architectural gap. DINO-DETR and DAB-DETR provide robust, geometry-aware queries and excellent box predictions but no dense mask head, whereas Mask2Former provides an elegant, unified mask-prediction pipeline but weaker geometric priors for boxes. Mask DINO is motivated precisely by this complementarity and will fuse DINO-DETR's box-savvy queries with a Mask2Former-style pixel decoder and dot-product mask head so that a single set of queries can jointly support detection and segmentation in the form of \((c_i, \mathbf{b}_i, m_i)\) tuples.
		
		\paragraph{Mask DINO: DINO-DETR queries as Mask2Former-style projectors}
		
		Mask DINO closes the gap between box-first DINO-DETR and mask-first Mask2Former by \emph{reusing} DINO-DETR's detection-strength queries as Mask2Former-style semantic projectors on a stride-4 pixel embedding map. Conceptually, the architecture keeps the entire DINO-DETR detector intact, and attaches a Mask2Former-inspired segmentation branch that interprets the same decoder content queries as directions in a dense embedding space.
		
		At a high level, the end-to-end flow for an input image
		\[
		I \in \mathbb{R}^{H_0 \times W_0 \times 3}
		\]
		is as follows:
		
		\begin{itemize}
			\item \textbf{DINO-DETR backbone, encoder, and decoder (detection part, ``blue'').} As in DINO-DETR, a CNN/ViT backbone produces multi-scale features, which are consumed by a multi-scale deformable Transformer encoder. Unified query selection uses encoder heads to pick high-quality object priors and initialize decoder queries with dynamic anchor boxes. A stacked decoder with multi-scale deformable cross-attention, contrastive denoising, mixed query selection, and ``look-forward-twice'' box refinement converts these into a refined set of content queries
			\[
			Q^{\text{dec}} = \{q_i\}_{i=1}^{N_q}, \qquad q_i \in \mathbb{R}^d,
			\]
			along with their associated anchors. Per-query class and box heads (inherited from DINO-DETR) predict logits $c_i$ and 4D boxes $\mathbf{b}_i$, preserving state-of-the-art detection performance.
			
			\item \textbf{Mask2Former-style pixel decoder and pixel embedding map (segmentation canvas, ``red'').} In parallel, a pixel decoder (as in Mask2Former) fuses backbone (and optionally encoder) features into a refined multi-scale pyramid $\{G'_s\}_{s \in \{32,16,8,4\}}$. Its stride-4 output
			\[
			E \equiv G'_4 \in \mathbb{R}^{H_4 \times W_4 \times d}, \qquad H_4 = \frac{H_0}{4}, \; W_4 = \frac{W_0}{4},
			\]
			serves as the \emph{pixel embedding map}: each location $(x,y)$ holds a $d$-dimensional embedding $E(x,y)$ summarizing local appearance and context at stride~4. Subsequent subsections will refine this description and show how $E$ is constructed from backbone and encoder features.
			
			\item \textbf{Dot-product mask head driven by DINO queries (segmentation part, ``red'').} The same refined decoder content queries $q_i$ that feed the detection heads are reinterpreted as mask projectors. A light mask-embedding head maps $q_i$ to a mask vector $\hat m_i \in \mathbb{R}^d$, and mask logits at stride~4 are obtained by per-location dot-products
			\[
			\ell_i(x,y) = \langle \hat m_i, E(x,y) \rangle, \qquad (x,y) \in \{1,\dots,H_4\} \times \{1,\dots,W_4\},
			\]
			followed by upsampling (typically bilinear $\times 4$) and a sigmoid to produce a full-resolution mask
			\[
			m_i = \sigma\bigl(\operatorname{Upsample}(\ell_i)\bigr) \in [0,1]^{H_0 \times W_0}.
			\]
			Thus, each single query $q_i$ now produces a triplet $(c_i, \mathbf{b}_i, m_i)$: a class, a bounding box, and a dense mask.
		\end{itemize}
		
		\begin{figure}[H]
			\centering
			\includegraphics[width=0.85\textwidth]{Figures/Chapter_15/MaskDino_architecture.jpg}
			\caption{\textbf{Mask DINO architecture.} The blue part is DINO-DETR: backbone, multi-scale deformable encoder, and decoder with dynamic anchor boxes, mixed query selection, contrastive denoising, and per-layer box/class heads. The red part is the Mask2Former-style extension: a pixel decoder builds a stride-4 pixel embedding map \(E \in \mathbb{R}^{H_4 \times W_4 \times d}\), encoder and decoder gain mask heads, and the same decoder content queries are used as mask projectors via dot-products with \(E\). Hybrid box--mask matching and unified denoising train all three outputs \((c_i, \mathbf{b}_i, m_i)\) jointly.}
			\label{fig:chapter15_maskdino_architecture}
		\end{figure}
		
		\newpage
		
		Mask DINO further extends DINO-DETR's encoder heads, denoising scheme, and Hungarian matching cost so that boxes and masks are \emph{selected}, \emph{denoised}, and \emph{matched} jointly rather than in separate pipelines: encoder heads output both box and mask scores for unified query selection; denoising operates on noised box--label and mask targets; and the matching cost combines classification, box, and mask terms. In effect, DINO-DETR's geometry-aware query engine provides precise localization priors, while Mask2Former's pixel decoder and dot-product head turn those same queries into powerful mask projectors on a stride-4 embedding canvas.
		
		This unified design preserves the geometric rigor of DINO-DETR (dynamic anchors, multi-scale deformable attention, denoising, mixed query selection) while importing Mask2Former's core insight that queries can be turned into semantic projectors via dot-products with a stride-4 pixel embedding map. The remainder of this section unpacks each component in detail, starting from the backbone and multi-scale features and moving through the pixel decoder, encoder and decoder design, and the segmentation branch and training losses.
		
		\subsubsection{Backbone and Multi-Scale Features}
		
		Mask DINO uses a CNN or ViT backbone (e.g., ResNet-50, Swin-L) to produce multi-scale feature maps
		\[
		\{F_{32}, F_{16}, F_{8}, F_{4}\},
		\]
		where the subscript denotes the stride relative to the input resolution. Each map has shape
		\[
		F_s \in \mathbb{R}^{C_s \times H_s \times W_s}, \qquad
		H_s = \frac{H_0}{s}, \; W_s = \frac{W_0}{s}.
		\]
		
		A pixel decoder (inherited from Mask2Former) then transforms these backbone features into a refined pyramid
		\[
		\{\tilde F_s\}_{s \in \{32,16,8,4\}}, \quad \{G_s\}_{s \in \{32,16,8,4\}}, \quad \{G'_s\}_{s \in \{32,16,8,4\}},
		\]
		where:
		\begin{itemize}
			\item $\tilde F_s$ are lateral projections to a common dimensionality $d$.
			\item $G_s$ are FPN-style top--down fused features.
			\item $G'_s$ are refined features produced by multi-scale deformable attention.
		\end{itemize}
		
		Formally, a $1\times 1$ convolution
		\[
		\tilde F_s = W_s^{\text{lat}} * F_s, \qquad
		W_s^{\text{lat}} \in \mathbb{R}^{d \times C_s \times 1 \times 1},
		\]
		projects each $F_s$ into a $d$-dimensional feature space without changing spatial resolution. A top--down FPN fusion then constructs
		\[
		G_{32} = \tilde F_{32}, \quad
		G_{16} = \tilde F_{16} + \operatorname{Upsample}(G_{32}), \quad
		G_{8} = \tilde F_{8} + \operatorname{Upsample}(G_{16}), \quad
		G_{4} = \tilde F_{4} + \operatorname{Upsample}(G_{8}),
		\]
		where $\operatorname{Upsample}$ is typically stride-2 bilinear interpolation to match the spatial size of the next finer scale.
		
		Finally, each $G_s$ is refined by multi-scale deformable attention (MS-DeformAttn) to obtain $G'_s$. The coarser levels $G'_8, G'_{16}, G'_{32}$ are then fed into the Transformer encoder as multi-scale inputs, while the pixel embedding map will later be constructed separately from the stride-4 backbone feature $C_b$ and the stride-8 encoder feature $C_e$. 
		
		\subsubsection{Multi-Scale Deformable Attention in the Pixel Decoder}
		
		Mask DINO (via Mask2Former) uses MS-DeformAttn~\cite{zhu2021_deformabledetr} both in the encoder and in the pixel decoder. This subsection focuses on the pixel decoder refinement
		\[
		G_s \;\longrightarrow\; G'_s,
		\]
		and explains how queries, keys, values, sampling locations, and attention weights are defined.
		
		\paragraph{Standard global attention (reminder)}
		In standard dot-product attention, we have
		\[
		Q \in \mathbb{R}^{N_q \times d}, \quad
		K, V \in \mathbb{R}^{N_k \times d},
		\]
		and
		\begin{equation}
			\operatorname{Attn}(Q,K,V)
			= \operatorname{softmax}\!\left(\frac{QK^\top}{\sqrt{d}}\right)V.
			\label{eq:chapter15_maskdino_standard_attn}
		\end{equation}
		For a single query $q \in \mathbb{R}^d$, this reads
		\[
		\alpha_j = \frac{\exp(\langle q,k_j\rangle / \sqrt{d})}{\sum_{j'} \exp(\langle q,k_{j'}\rangle / \sqrt{d})},
		\qquad
		\operatorname{Attn}(q,K,V) = \sum_j \alpha_j v_j.
		\]
		Each query attends \emph{densely} to all $N_k$ keys/values, which is too expensive when $K,V$ come from large feature maps.
		
		\paragraph{High-level idea of MS-DeformAttn}
		MS-DeformAttn replaces dense global attention by:
		\begin{itemize}
			\item \emph{Sparse} attention: each query attends to $M$ sampling locations per head instead of all spatial positions.
			\item \emph{Multi-scale} attention: those sampling locations span multiple pyramid levels (e.g., strides $4,8,16,32$).
			\item \emph{Content-adaptive} attention: sampling locations and their weights are predicted from the query itself.
		\end{itemize}
		
		Instead of computing all dot products $\langle q,k_j\rangle$, the module:
		\begin{enumerate}
			\item Predicts where to look (sampling offsets) from the query $q_p$.
			\item Samples values $V_{s'}$ at those locations across scales.
			\item Predicts how strongly to weight each sample from $q_p$ via a small linear layer.
		\end{enumerate}
		
		There is no explicit dense set of keys; the query implicitly defines both the sampling locations and the pattern in which sampled values are mixed.
		
		\paragraph{Inputs to MS-DeformAttn in the pixel decoder}
		Fix a scale $s \in \{4,8,16,32\}$. The inputs are:
		\begin{itemize}
			\item \textbf{Queries.} The query features are the current fused map
			\[
			Q_s = G_s \in \mathbb{R}^{d \times H_s \times W_s},
			\]
			flattened into $Q_s^{\text{flat}} \in \mathbb{R}^{(H_s W_s) \times d}$. Each spatial position $p=(i,j)$ corresponds to a query vector
			\[
			q_p = G_s(i,j) \in \mathbb{R}^d.
			\]
			\item \textbf{Keys and values.} Keys and values are all laterally projected backbone features
			\[
			\mathcal{F} = \{\tilde F_{4}, \tilde F_{8}, \tilde F_{16}, \tilde F_{32}\},
			\]
			with $K_{s'} = V_{s'} = \tilde F_{s'}$ for each scale $s'$. These are flattened per scale but are conceptually kept as separate feature maps.
			\item \textbf{Reference points.} Each spatial position $p=(i,j)$ at scale $s$ has a normalized reference coordinate
			\[
			r_p = \left(\frac{j}{W_s}, \frac{i}{H_s}\right) \in [0,1]^2,
			\]
			indicating where in the image this query lives.
		\end{itemize}
		
		\paragraph{Step-by-step computation for a single query}
		Fix a scale $s$, a spatial position $p$, a head index $h \in \{1,\dots,H\}$, and a sampling index $m \in \{1,\dots,M\}$.
		
		\textbf{(1) Sampling offsets.}  
		From the query $q_p$ we predict a 2D offset
		\[
		\Delta_{h,m}(p) = W^{\Delta}_{h,m} q_p \in \mathbb{R}^2,
		\]
		where $W^{\Delta}_{h,m} \in \mathbb{R}^{2 \times d}$ is a learned matrix. This offset specifies how far, and in what direction, to move from the reference point $r_p$.
		
		\textbf{(2) Sampling locations.}  
		The normalized sampling location is
		\begin{equation}
			\phi_{h,m}(p) = r_p + \Delta_{h,m}(p) \in \mathbb{R}^2.
			\label{eq:chapter15_maskdino_sampling}
		\end{equation}
		$\phi_{h,m}(p)$ is continuous and may fall between pixels. Each sampling point is also associated with a scale $s'_{h,m} \in \{4,8,16,32\}$ (in practice, each head has a fixed or learned pattern for how its $M$ samples are distributed across scales).
		
		\textbf{(3) Sampling values from multi-scale features.}  
		At the chosen scale $s'_{h,m}$, the location $\phi_{h,m}(p)$ is interpreted in that feature map's coordinate frame, and a value vector is obtained via bilinear interpolation:
		\[
		v_{h,m}(p) = V_{s'_{h,m}}\bigl(\phi_{h,m}(p)\bigr) \in \mathbb{R}^d.
		\]
		
		\textbf{(4) Unnormalized attention scores.}  
		For each sample we compute a scalar logit
		\[
		w_{h,m}(p) = u_{h,m}^\top q_p,
		\]
		where $u_{h,m} \in \mathbb{R}^d$ is a learned vector (or a row of a small linear layer). This plays a role analogous to $\langle q,k_j\rangle$ in standard attention: it scores how much the query wants to use the $m$-th sample for head $h$.
		
		\textbf{(5) Attention weights.}  
		For fixed $h$ and $p$, these logits are normalized across $m$:
		\[
		A_{h,m}(p)
		= \frac{\exp(w_{h,m}(p))}{\sum_{m'=1}^M \exp\bigl(w_{h,m'}(p)\bigr)}.
		\]
		The weights $\{A_{h,m}(p)\}_{m=1}^M$ form a convex combination over the sampled values $\{v_{h,m}(p)\}_{m=1}^M$ and are the sparse analogue of the softmax over \emph{all} keys in \eqref{eq:chapter15_maskdino_standard_attn}. The form is chosen to provide:
		\begin{itemize}
			\item Smooth, differentiable importance scores.
			\item Normalization to control magnitude.
			\item Content-adaptive weighting of a small learned set of sampling points.
		\end{itemize}
		
		\textbf{(6) Head output and multi-head output.}  
		Each head combines its sampled values as
		\[
		\operatorname{head}_h(q_p)
		= \sum_{m=1}^M A_{h,m}(p)\, W_h v_{h,m}(p),
		\]
		where $W_h \in \mathbb{R}^{d_h \times d}$ is a learned projection and $\sum_h d_h = d$. The multi-head output is then
		\[
		\operatorname{MSDeformAttn}(q_p)
		= \operatorname{Concat}_{h=1}^H \operatorname{head}_h(q_p) \in \mathbb{R}^d.
		\]
		
		\paragraph{Residual refinement of $G_s$}
		At each scale $s$, MS-DeformAttn produces a refined feature for each position $p$ via a residual update:
		\begin{equation}
			G'_s(p)
			= G_s(p) + \operatorname{MSDeformAttn}\!\bigl(
			q_p, \{\tilde F_{s'}\}_{s' \in \{4,8,16,32\}}
			\bigr), \quad q_p = G_s(p).
			\label{eq:chapter15_maskdino_msdeform_update}
		\end{equation}
		Thus \emph{every} scale $s \in \{4,8,16,32\}$ is refined into $G'_s$.
		
		\paragraph{Why weights depend only on the query}
		In standard attention, weights are $\alpha_j \propto \exp(\langle q,k_j\rangle)$. Here, we do not explicitly maintain a dense set of keys $k_j$:
		\begin{itemize}
			\item The \emph{choice of where to sample} is already conditioned on the query via $\Delta_{h,m}(p)$ and $\phi_{h,m}(p)$.
			\item The \emph{content} at those positions is captured in $v_{h,m}(p)$.
			\item We only need a relative ranking among the $M$ samples for each head and position.
		\end{itemize}
		The simple bilinear form
		\[
		w_{h,m}(p) = u_{h,m}^\top q_p
		\]
		is thus sufficient and efficient: the query chooses a pattern, offsets choose where to look, values encode what is found, and attention weights decide how to mix the sampled patterns.
		
		\paragraph{Cross-attention vs.\ self-attention}
		
		In the pixel decoder:
		\begin{itemize}
			\item Queries are the fused features $G_s$ (or $G'_s$) at a given scale.
			\item Keys and values are the multi-scale backbone features $\{\tilde F_{s'}\}_{s' \in \{4,8,16,32\}}$.
		\end{itemize}
		Here MS-DeformAttn acts as \emph{multi-scale cross-attention}: fused features at scale $s$ attend sparsely to the backbone pyramid.
		
		In the Transformer encoder, queries, keys, and values all come from the same multi-scale set of encoder features, so MS-DeformAttn plays the role of \emph{multi-scale self-attention}, analogous to the encoder in Deformable DETR~\cite{zhu2021_deformabledetr}.
		
		In the Transformer decoder, queries are the object queries and keys/values are the encoder’s multi-scale outputs, so MS-DeformAttn again acts as \emph{multi-scale cross-attention} from queries to the encoder memory.
		
		\paragraph{Why all scales $G'_s$ must be refined}
		Although only the stride-4 refined map $G'_4$ is directly used to build the pixel embedding map, Mask DINO refines \emph{every} scale $s\in\{4,8,16,32\}$:
		\begin{itemize}
			\item The FPN fusion is top--down: $G_4$ depends on $G_8$, which depends on $G_{16}$, and so on. If coarse maps are unrefined, noise and misalignment propagate down.
			\item MS-DeformAttn at scale $4$ samples from all scales $\tilde F_{s'}$. If those maps are not semantically clean, the sampling locations and values seen by $G'_4$ will be inconsistent.
			\item Coarse scales carry global semantic and shape information; refining them first ensures that when $G'_4$ samples them, it receives coherent context rather than raw backbone features.
		\end{itemize}
		In short, a clean stride-4 representation \emph{cannot} be built by sampling from unrefined coarse maps; therefore all $G_s$ must be refined into $G'_s$ before $G'_4$ can serve as a high-quality pixel embedding map.
		
		\subsubsection{Transformer Encoder and Decoder in Mask DINO}
		
		Up to this point we have taken an input image
		\[
		I \in \mathbb{R}^{H_0 \times W_0 \times 3}
		\]
		through a backbone and pixel decoder to obtain a refined multi-scale pyramid
		\[
		\{G_s'\}_{s \in \{4,8,16,32\}}, \qquad
		G_s' \in \mathbb{R}^{d \times H_s \times W_s}, \quad
		H_s = \tfrac{H_0}{s},\; W_s = \tfrac{W_0}{s}.
		\]
		For a typical COCO-style input of $H_0 \approx 800$, $W_0 \approx 1333$, the finest map $G_4'$ has resolution $H_4 \approx 200$, $W_4 \approx 333$, while $G_{32}'$ is much coarser but carries strong semantic context. You can think of $\{G_s'\}$ as a multi-scale canvas where each location already encodes a blend of local detail and global context thanks to MS-DeformAttn in the pixel decoder.
		
		Mask DINO now feeds these features into a DINO-style encoder--decoder to produce a set of refined, \emph{box-savvy} queries that will later drive both bounding boxes and masks. The goal of this stage is: given the refined pyramid $\{G_s'\}$, identify a small set of promising object candidates and iteratively refine them into high-quality query embeddings $\hat q_i$ that can support joint $(c_i,\mathbf{b}_i,m_i)$ prediction.
		
		\paragraph{Encoder: from refined pyramid to token proposals}
		
		With the pixel decoder in place, we start the Transformer stage from the refined multi-scale pyramid
		\[
		\{G_s'\}_{s \in \{4,8,16,32\}}, \qquad
		G_s' \in \mathbb{R}^{d \times H_s \times W_s}, \quad
		H_s = \tfrac{H_0}{s},\; W_s = \tfrac{W_0}{s}.
		\]
		Mask DINO follows DINO and runs the encoder on the three coarser pyramid levels, typically $s \in \{8,16,32\}$, while the stride-4 map is reserved for constructing the pixel embedding map. Coarse levels provide global semantics and rough shapes, whereas the finest encoder level ($s = 8$) carries more detailed edges and boundaries; the encoder’s job is to let every location see just enough of both.
		
		Concretely, the encoder follows DINO-DETR but replaces dense self-attention with MS-DeformAttn:
		\begin{itemize}
			\item Each $G_s'$ is flattened into a sequence of tokens
			\[
			T_s \in \mathbb{R}^{(H_s W_s) \times d},
			\]
			and all scales are processed jointly through $L_{\text{enc}}$ layers of MS-DeformAttn-based self-attention and feed-forward networks, yielding encoder outputs $E_{e,s} \in \mathbb{R}^{(H_s W_s) \times d}$ at each scale.
			\item In each MS-DeformAttn layer, the \emph{queries} are the current tokens at positions $p$ on each $G_s'$, while the \emph{keys} and \emph{values} come from the entire multi-scale set $\{G_{s'}'\}_{s' \in \{4,8,16,32\}}$ treated as a single multi-scale memory.
			For a token $q_p$ at position $p$ on level $s$, the module predicts a small set of offsets and scales, samples a few values from nearby and cross-scale positions (via bilinear interpolation), and mixes them with learned attention weights, exactly as in the MS-DeformAttn formulation described earlier.
			\item The effect is that each encoder token becomes a \emph{context-aware} descriptor: a location on $G_4'$ near a dog’s ear can pull in coarse information from $G_{32}'$ about the overall dog silhouette and complementary detail from $G_8'$ or $G_{16}'$, without paying the cost of dense global attention.
		\end{itemize}
		
		On top of these encoder outputs, Mask DINO attaches three lightweight heads that turn tokens into dense proposals:
		\begin{itemize}
			\item A classification head that predicts how likely each token is to correspond to a foreground object instance or a stuff region.
			\item A box head that predicts a coarse 4D bounding box for each token, typically as normalized coordinates or offsets relative to its spatial location.
			\item A mask head that predicts a coarse mask for each token by projecting its embedding onto the stride-4 pixel embedding map $E$ (constructed in the pixel-decoder branch) via a dot-product, producing early mask logits on the $H_4 \times W_4$ grid.
		\end{itemize}
		
		For every encoder token, this yields an early triplet $(\hat c,\hat{\mathbf{b}},\hat m)$.
		Intuitively, the encoder behaves like a dense proposal generator.
		Given the refined pyramid, it produces a heatmap of ``interesting'' locations and associated coarse boxes and masks.
		For example, in an image containing a dog and a person, tokens around the dog’s torso will tend to have high ``dog'' scores, a roughly dog-shaped mask, and a bounding box enclosing the dog; tokens near the person will produce analogous proposals for the person.
		Mask DINO will not keep all of these dense proposals; instead, it uses them to initialize a much smaller set of decoder queries.
		
		\paragraph{Unified query selection: seeding decoder queries and anchors}
		
		Rather than starting decoder queries from purely learned embeddings, Mask DINO adopts DINO’s \emph{unified query selection} and extends it to take mask quality into account.
		The idea is to select a small number of the most promising encoder tokens and turn them into decoder queries with good initial content and good initial anchor boxes.
		
		For each encoder token, Mask DINO attaches three prediction heads that share their architecture with the decoder heads. In practice:
		
		\begin{itemize}
			\item The classification head outputs a foreground class score. This score alone is used to rank encoder tokens and select the top-ranked features to become decoder content queries.
			\item The box head predicts a coarse 4D box for each token. These boxes are supervised with the standard detection loss and will become initial anchors for the decoder.
			\item The mask head predicts a coarse mask for each token by dot-producting the token with the high-resolution pixel embedding map $E$, and is supervised with a mask loss. These coarse masks are also used to derive tighter boxes that better follow the object support.
		\end{itemize}
		
		During unified query selection, Mask DINO simply takes the top $N_q$ encoder tokens by classification confidence, uses their encoder features as the initial \emph{content} embeddings $q_i^{(0)}$, and uses the associated predicted boxes (optionally refined using the coarse masks) as the initial \emph{anchor boxes} $a_i^{(0)}$. This matches the ``unified and enhanced query selection'' design in the paper: the three heads provide detection and segmentation priors on encoder tokens, but the classification score drives ranking, while the predicted boxes and masks supply supervised initial anchors for the decoder.
		
		Mask DINO then:
		\begin{itemize}
			\item Selects the top $N_q$ encoder tokens according to this unified score (optionally mixing in a small number of learned queries for robustness).
			\item Uses the selected encoder features as the initial \emph{content} embeddings of decoder queries $q_i^{(0)}$.
			
			\newpage
			
			\item Uses the selected boxes (often refined using the corresponding coarse masks to better fit object extent) as the initial \emph{anchor boxes} $a_i^{(0)}$ for those queries.
		\end{itemize}
		
		This step converts a dense field of thousands of encoder tokens into a compact set of $N_q$ object candidates that are already biased toward good shapes.
		Because masks are usually easier to learn than precise boxes early in training (pixel-level supervision provides many more gradients than sparse box corners), the encoder mask head often provides the most reliable early signal.
		Mask DINO exploits this by effectively performing \emph{mask-enhanced anchor box initialization}: boxes derived from regions with good masks provide stronger geometric priors for the decoder than boxes coming from box regression alone.
		
		\paragraph{Decoder: from proposals to refined box-aware queries}
		
		The decoder then refines these $N_q$ selected queries using the DINO-DETR design, but now with the added role of supporting mask prediction.
		Each decoder query carries two coupled components: a content embedding $q_i^{(\ell)} \in \mathbb{R}^d$ and an anchor box $a_i^{(\ell)} \in \mathbb{R}^4$ at layer $\ell$.
		
		At a high level, each decoder layer performs three standard Transformer operations, plus per-layer prediction heads:
		\begin{itemize}
			\item Self-attention over the current set of queries so that different queries can exchange information and resolve competition when they cover overlapping regions.
			\item Multi-scale deformable cross-attention from queries to the encoder outputs $\{E_{e,s}\}$, using the current anchor box $a_i^{(\ell)}$ of each query to define its reference points for MS-DeformAttn.
			For query $i$, the center of $a_i^{(\ell)}$ is normalized and used as the reference; the module then predicts a small set of offsets and scales, samples values from nearby positions across all scales, and aggregates them to update $q_i^{(\ell)}$.
			This allows each query to pull just the relevant context from the refined pyramid, focusing on its hypothesized object region.
			\item A position-wise feed-forward network (FFN) that further refines the updated query embeddings.
		\end{itemize}
		
		As in DINO-DETR, each decoder layer also has prediction heads:
		\begin{itemize}
			\item A classification head that predicts class scores from the current query embedding $q_i^{(\ell)}$.
			\item A box head that predicts both intermediate and refined boxes from $q_i^{(\ell)}$, enabling the ``look-forward-twice'' mechanism where intermediate boxes become the anchors $a_i^{(\ell+1)}$ for the next layer, and refined boxes are directly supervised.
		\end{itemize}
		Mask DINO adds:
		\begin{itemize}
			\item A decoder mask head that predicts a mask for each query at selected layers by projecting $q_i^{(\ell)}$ onto the stride-4 pixel embedding map $E$ via a dot-product, providing additional shape-focused supervision while boxes are still being refined.
		\end{itemize}
		
		After $L_{\text{dec}}$ decoder layers, we obtain a final set of refined content queries
		\[
		\hat Q = \{\hat q_i\}_{i=1}^{N_q}, \qquad \hat q_i \in \mathbb{R}^d,
		\]
		together with their associated refined anchor boxes and the per-layer predictions from intermediate heads.
		Each $\hat q_i$ can be viewed as a compact description of one predicted object or stuff region: it has been geometrically grounded via dynamic anchors and multi-scale deformable attention, and semantically shaped by both box and mask supervision.
		These refined queries are then passed to the final classification, box, and mask heads described in the next part to produce the model’s $(c_i,\mathbf{b}_i,m_i)$ outputs.
		
		\subsubsection{Segmentation Branch and Pixel Embedding Map}
		
		The segmentation branch turns refined queries $\hat q_i$ into dense masks by projecting them onto a stride-4 pixel embedding map, in the spirit of Mask2Former.
		Conceptually, this is where DINO's box-savvy queries become Mask2Former-style semantic projectors.
		
		\paragraph{Constructing the pixel embedding map}
		
		Mask DINO constructs a stride-4 pixel embedding map by fusing backbone and encoder features.
		Let:
		\begin{itemize}
			\item $C_b \in \mathbb{R}^{C_b \times H_4 \times W_4}$ be the stride-4 feature map from the backbone.
			\item $C_e \in \mathbb{R}^{d \times H_8 \times W_8}$ be an encoder feature at stride $8$, obtained by reshaping the corresponding encoder tokens $E_{e,8}$.
			\item $T$ be a $1\times 1$ convolution mapping $C_b$ to the Transformer hidden dimension $d$.
			\item $F$ be a $2\times$ upsampling operation that brings $C_e$ from stride $8$ to stride $4$ via bilinear interpolation.
			\item $M$ be a lightweight segmentation head (for example, a few $3\times 3$ convolutions with normalization and nonlinearity) operating on stride-4 features.
		\end{itemize}
		The fused pixel embedding map is then
		\begin{equation}
			E(x,y) = M\bigl(T(C_b)(x,y) + F(C_e)(x,y)\bigr) \in \mathbb{R}^d,
			\label{eq:chapter15_maskdino_pixel_map}
		\end{equation}
		for $(x,y)$ on the stride-4 grid $(H_4 \times W_4)$.
		Intuitively, $T(C_b)$ contributes high-resolution local detail (edges, textures) while $F(C_e)$ injects encoder-level context (which object is likely present at that location).
		The head $M$ mixes these signals into a per-pixel embedding $E(x,y)$ that is well-suited for deciding ``which object or region this pixel belongs to.''
		
		\paragraph{Query–pixel dot products as mask logits}
		
		Each refined decoder content query $\hat q_i \in \mathbb{R}^d$ is then turned into a mask by a simple dot-product with the pixel embedding map.
		Formally, the implementation applies a small linear layer to produce a mask embedding $\hat m_i \in \mathbb{R}^d$ and uses
		\[
		\ell_i(x,y) = \langle \hat m_i, E(x,y) \rangle, \qquad
		(x,y) \in \{1,\dots,H_4\} \times \{1,\dots,W_4\}.
		\]
		These $\ell_i$ are stride-4 mask logits, which are then upsampled to the input resolution and passed through a sigmoid:
		\[
		m_i = \sigma\bigl(\operatorname{Upsample}(\ell_i)\bigr) \in [0,1]^{H_0 \times W_0}.
		\]
		
		Geometrically, $\hat m_i$ defines a direction in the $d$-dimensional embedding space.
		Pixels whose embeddings $E(x,y)$ align strongly with $\hat m_i$ receive large positive logits and are included in the mask.
		For example, if $\hat q_i$ has learned to represent ``this particular dog instance'', then $E(x,y)$ at the dog's pixels will be close to $\hat m_i$, and $\ell_i(x,y)$ will be high there, while background pixels will have low logits.
		This is exactly the Mask2Former mechanism, now driven by DINO-style queries.
		
		In parallel, the same $\hat q_i$ is fed to the class and box heads inherited from DINO-DETR to predict
		\[
		c_i \in \mathbb{R}^{C+1}, \qquad \mathbf{b}_i \in \mathbb{R}^4.
		\]
		The crucial point is that \emph{the same query representation} $\hat q_i$ is responsible for all three outputs $(c_i,\mathbf{b}_i,m_i)$.
		Box and mask supervision therefore shape a shared representation rather than training two disjoint branches.
		In practice this improves both detection and segmentation: better masks help the model learn tighter boxes, and better boxes help the model focus its mask predictions.
		
		
		\subsubsection{Unified Denoising and Hybrid Matching}
		
		The final piece is how Mask DINO trains this unified architecture so that queries become simultaneously good classifiers, localizers, and mask projectors.
		
		\paragraph{Extending denoising from boxes to masks}
		
		DINO-DETR uses contrastive denoising: extra decoder queries are reserved for reconstructing ground-truth boxes and labels from \emph{noised} versions of those targets.
		This stabilizes training and accelerates convergence for detection.
		Mask DINO extends this idea to segmentation:
		\begin{itemize}
			\item Ground-truth boxes and masks are perturbed, for example by jittering box coordinates or slightly eroding/dilating masks, to generate noisy training targets.
			\item A subset of decoder queries is dedicated to reconstructing the original boxes and masks from these noised targets, in addition to the standard detection denoising.
		\end{itemize}
		
		Because masks carry finer-grained information than boxes, this \emph{mask-aware denoising} provides strong gradients that encourage queries to capture detailed object shapes early in training.
		For a dog example, even if the initial box roughly covers the dog and some background, the denoising loss forces the query to predict a mask that tightly follows the dog's silhouette, which in turn helps refine the box in later decoder layers.
		Empirically, this leads to faster convergence and better mask quality.
		
		\paragraph{Hybrid matching cost for joint detection and segmentation}
		
		As in other DETR-style models, Mask DINO uses the Hungarian algorithm to match predicted queries to ground-truth instances.
		The key difference is that the matching cost combines classification, box, and mask terms:
		\[
		L_{\text{match}} =
		\lambda_{\text{cls}} L_{\text{cls}}
		+ \lambda_{\text{box}} L_{\text{box}}
		+ \lambda_{\text{mask}} L_{\text{mask}}.
		\]
		Here $L_{\text{cls}}$ is a classification loss (for example cross-entropy or focal loss), $L_{\text{box}}$ is a combination of $\ell_1$ and GIoU losses on boxes, and $L_{\text{mask}}$ measures mask quality (for example a combination of binary cross-entropy and Dice loss on the predicted masks).
		
		This hybrid cost ensures that a query is considered a good match only if it is simultaneously:
		\begin{itemize}
			\item Confident about the correct class.
			\item Accurate in terms of its bounding box.
			\item Accurate in terms of its mask.
		\end{itemize}
		The same combined losses are then used to train the matched queries.
		Queries that are good at boxes but poor at masks, or vice versa, are penalized until they become good at both.
		
		For panoptic segmentation, Mask DINO follows the paper in removing box terms for ``stuff'' categories in the matching cost.
		Boxes are still predicted and used to guide deformable attention as geometric priors, but they are not penalized for amorphous, image-wide regions where a tight bounding box is ill-defined.
		This keeps the box machinery useful for attention while letting masks carry the main supervision for stuff classes.
		
		\medskip
		
		Putting these pieces together, the end-to-end flow can be summarized as follows.
		The image $I$ is mapped by the backbone and pixel decoder to refined multi-scale features $\{G_s'\}$ and a stride-4 pixel embedding map $E$.
		The encoder turns $\{G_s'\}$ into dense proposal tokens and associated coarse $(c,\mathbf{b},m)$ predictions.
		Unified query selection chooses the best proposals and converts them into a compact set of dynamic anchor queries.
		The decoder refines these queries using deformable attention and per-layer heads, producing final query embeddings $\hat q_i$ that are both geometry-aware and mask-aware.
		
		\newpage
		
		Finally, each $\hat q_i$ feeds class, box, and mask heads to produce $(c_i,\mathbf{b}_i,m_i)$, trained jointly via mask-extended denoising and a hybrid matching cost.
		This closes the loop from $I \in \mathbb{R}^{H_0 \times W_0 \times 3}$ to a set of predictions that simultaneously provide class labels, bounding boxes, and dense segmentation masks.
		
		\subsubsection{Empirical Performance, Ablation Insights, Limitations, and Outlook}
		\label{subsubsec:chapter15_maskdino_performance}
		
		Mask DINO’s unified design is not only conceptually clean but also empirically strong. On COCO with a ResNet-50 backbone, Mask DINO consistently improves over both DINO-DETR for detection and Mask2Former for segmentation, achieving higher box AP and higher mask AP than either specialized model under comparable training schedules and query budgets \cite{li2022_maskdino,zhang2022_dino,ke2023_hqsam}. With a stronger Swin-L backbone and additional detection pretraining on Objects365, Mask DINO reaches state-of-the-art results among models under one billion parameters, including approximately \(54.5\) AP for COCO instance segmentation, \(59.4\) PQ for COCO panoptic segmentation, and \(60.8\) mIoU on ADE20K semantic segmentation \cite{li2022_maskdino}. These results confirm that a single Transformer, trained end-to-end on a joint detection–segmentation objective, can match or surpass carefully engineered task-specific architectures across instance, panoptic, and semantic segmentation benchmarks.
		
		From a methodological perspective, the gains are not accidental; they reflect the way the tasks reinforce one another through shared queries and losses. Detection-side innovations from DINO—contrastive denoising, dynamic anchor queries, and multi-scale deformable cross-attention provide well-localized, geometry-aware queries \cite{zhang2022_dino}. Mask2Former’s mask head, in turn, injects dense pixel-level supervision via the query–pixel dot-product mechanism \cite{ke2023_hqsam}. In practice, this synergy manifests as faster convergence and higher final AP: masks help queries learn sharper object boundaries and category-discriminative features, while strong boxes and anchors help the mask head focus on the correct regions rather than drifting to nearby distractors.
		
		\paragraph{Ablation insights}
		
		The Mask DINO paper backs up this qualitative picture with a series of ablations \cite{li2022_maskdino}. While exact numbers differ across backbones and schedules, several trends are consistent:
		
		\begin{itemize}
			\item \textbf{Unified training vs.\ separate tasks.} Training detection and segmentation jointly with shared queries outperforms training either task alone or combining independently trained detectors and segmenters. Removing the joint training and using only detection or only segmentation losses leads to a noticeable drop in both box AP and mask AP. This indicates that both tasks genuinely benefit from sharing the same query set and feature pipeline instead of competing for capacity.
			\item \textbf{Mask-enhanced query initialization.} Mask DINO’s \emph{mask-enhanced anchor box initialization}—using early coarse masks from the encoder to tighten anchor boxes before they enter the decoder—provides a consistent improvement over using box regression alone. Ablations where anchors are initialized only from box heads (without mask feedback) show degraded performance, especially in crowded scenes where box-only proposals tend to be too loose or misaligned around object boundaries.
			\item \textbf{Hybrid matching and mask-extended denoising.} The hybrid Hungarian cost that combines classification, box, and mask terms yields better assignments than using detection-only or mask-only costs. Similarly, extending DINO’s denoising objective from boxes and labels to include masks stabilizes training and accelerates convergence. 
			
			\newpage
			
			Ablations that remove the mask term from the matching cost or restrict denoising to boxes only consistently reduce both detection and segmentation metrics, confirming that the model learns better queries when it is required to be simultaneously good at classifying, localizing, and segmenting.
			\item \textbf{Encoder proposals and query selection.} Removing the encoder-side proposal heads (class, box, mask) and reverting to purely learned queries degrades final performance and slows down training. The ablations in \cite{li2022_maskdino} show that using encoder proposals to initialize decoder queries not only improves AP but also makes training more robust to hyperparameters such as the number of queries \(N_q\) and the training schedule length.
		\end{itemize}
		
		Intuitively, these ablations support the view that Mask DINO’s improvements come from a coherent design rather than from a single trick. The encoder acts as a dense proposal generator; the unified query selection ensures that only high-quality, mask-consistent candidates reach the decoder; and the hybrid matching plus denoising tie everything together, forcing each query to become a compact, multi-purpose representation that works well for class, box, and mask simultaneously.
		
		\paragraph{Limitations and practical caveats}
		
		Despite its strong performance, Mask DINO remains squarely in the classical \emph{closed-set} regime. The classifier head predicts over a fixed label set (for example, COCO’s \(80\) thing and \(53\) stuff categories), and the model cannot directly recognize categories that were not present in its supervised training data. In other words, Mask DINO answers “Which of these predefined classes is present here?” rather than “What object, described in arbitrary language, is present here?”. For many real applications—long-tail categories, domain-specific entities, evolving taxonomies—this closed-set assumption becomes a hard limitation.
		
		There are also architectural and computational considerations. The unified encoder–decoder is still a relatively heavy DETR-style model. Inference cost grows with the number of queries \(N_q\), and there is a trade-off between accuracy and throughput when scaling \(N_q\) or backbone capacity. Using fewer queries improves speed but tends to hurt AP, especially on crowded scenes; using many more queries can improve recall but increases memory and latency. Training stability and efficiency benefit strongly from the denoising and hybrid matching machinery, but also inherit some of the usual DETR sensitivities to learning-rate schedules, warm-up strategies, and data scale. In practice, Mask DINO works very well on standard datasets such as COCO, ADE20K, and Cityscapes, but still requires fine-tuning or continued training to adapt to substantially shifted domains (for example, medical imaging or aerial imagery).
		
		\paragraph{Summary and outlook: from unified closed set to open vocabulary}
		
		In summary, Mask DINO demonstrates that a single DETR-like model can serve as a strong, unified engine for detection, instance segmentation, panoptic segmentation, and semantic segmentation. Its central ideas—encoder-driven proposal generation, mask-enhanced query initialization, hybrid box–mask matching, and mask-extended denoising—show how box prediction and mask prediction can be made to \emph{help} each other rather than compete for representational capacity.
		
		However, Mask DINO and its ancestors DINO and Mask2Former are still closed-set recognizers. The natural next step is to combine this unified query-based formulation with \emph{open-vocabulary} recognition: replacing fixed classification heads with language-aware heads that align image regions to text embeddings. This direction is pursued by models such as Grounding DINO, which aligns region features with text queries through contrastive training \cite{liu2023_groundingdino}, and by Grounded SAM, which feeds such grounded boxes as prompts into powerful segmentation models like SAM \cite{ren2024_groundedsam,kirillov2023_sam}.
		
		\newpage
		
		In the following parts we will build on Mask DINO’s unified query view to understand how these grounded and promptable architectures extend the same ideas to text-conditioned and eventually video-conditioned segmentation.
		
		\paragraph{Summary}
		
		Mask DINO can be viewed as a culmination of the “closed-set unified recognition” line of work. It shows that one set of Transformer queries can simultaneously support classification, bounding box regression, and dense mask prediction; that box and mask supervision can be made to help rather than compete; and that a single model can reach or exceed the performance of separate detection and segmentation systems on standard benchmarks \cite{li2022_maskdino, zhang2022_dino, ke2023_hqsam}. Conceptually, it provides a clean template: backbone and pixel decoder build a multi-scale canvas, the Transformer core distills it into a small set of object-centric slots, and lightweight heads turn each slot into a \((c,\mathbf{b},m)\) tuple.
		
		The next frontier, however, is to move beyond fixed taxonomies and toward \emph{open-vocabulary} and \emph{promptable} segmentation. Instead of predicting a class index from a fixed label set, recent models align image regions with text embeddings, so that a user can query the model with arbitrary phrases such as “red backpack”, “road damage”, or “company logo”. Grounding DINO extends DETR-style detection in this direction by replacing the closed-set classifier with a text-conditioned grounding head \cite{liu2023_groundingdino}. Grounded SAM then composes such open-vocabulary detections with the powerful mask decoder of SAM, using language-guided boxes as prompts for high-quality masks \cite{ren2024_groundedsam, kirillov2023_sam}. More recently, SAM~2 generalizes promptable segmentation to videos via a streaming memory mechanism \cite{ravi2024_sam2}, and combinations of Grounding DINO with SAM~2 inherit both open-vocabulary grounding and long-term temporal consistency.
		
		In this sense, Mask DINO forms an important stepping stone. It establishes that unified, query-based Transformers are an effective backbone for detection and segmentation in the closed-set regime. Open-vocabulary models such as Grounding DINO, Grounded SAM, and SAM~2 can then be seen as extending the same query-based template with language-conditioning and promptable interfaces, allowing users to move from “predict masks for 80 classes” to “segment whatever I can describe in natural language”, and eventually to do so consistently over time in videos. Subsequent parts will build on this connection, showing how these newer models inherit many of Mask DINO’s architectural ideas while relaxing its closed-set limitation.
		
		\newpage
		
	\end{enrichment}
	
	\begin{enrichment}[Grounded SAM: From Text Prompts to Any-Object Masks][subsection]
		\label{enr:subsec_chapter15_groundedsam}
		
		Grounded SAM~\cite{ren2024_groundedsam} is a practical blueprint for \emph{open-world segmentation}: instead of training yet another monolithic foundation model, it composes existing expert models for open-vocabulary detection, promptable segmentation, tagging, captioning, generative editing, and 3D human analysis.
		The central idea is simple but powerful.
		Given an image and a text description, an open-vocabulary detector (Grounding~DINO~\cite{liu2023_groundingdino}) localizes regions relevant to the text, and a promptable segmentation model (SAM~\cite{kirillov2023_sam} or SAM~2~\cite{ravi2024_sam2}) converts those regions into precise masks.
		Around this core, additional experts such as BLIP~\cite{li2022_blip}, the Recognize Anything Model (RAM)~\cite{zhang2023_ram}, Stable Diffusion~\cite{rombach2022_ldm}, and OSX~\cite{lin2023_osx} are attached to build automatic dense image annotation, controllable image editing, and text-driven 3D human motion analysis.
		
		\begin{figure}[H]
			\centering
			\includegraphics[width=0.8\textwidth]{Figures/Chapter_15/GroundedSAM_usages.jpg}
			\caption{\textbf{Architecture and application versatility of Grounded SAM}. 
				Top: Grounded SAM combines Grounding~DINO for open-vocabulary detection and SAM for promptable segmentation, yielding an open-vocabulary detection-and-segmentation pipeline.
				Middle: Coupled with BLIP and RAM, it becomes an automatic dense image annotation system that generates captions or tags and grounds them to image regions.
				Bottom: Grounded SAM enables downstream applications such as controllable image editing with Stable Diffusion and promptable human motion analysis when integrated with OSX.
				Figure reproduced from Ren et al.~\cite{ren2024_groundedsam}.}
			\label{fig:chapter15_groundedsam_usages}
		\end{figure}
		
		\subsubsection{Motivation and context}
		\label{subsubsec:chapter15_groundedsam_motivation}
		
		\paragraph{From closed-set segmentation to open-world understanding}
		Classical segmentation backbones such as Mask R-CNN or Mask2Former learn on fixed label sets (for example, the 80 COCO categories) and cannot directly handle unseen concepts.
		
		\newpage
		
		Mask~DINO~\cite{li2022_maskdino} improved this situation by unifying detection and segmentation in a single Transformer decoder, but it still assumes a closed vocabulary tied to the training datasets.
		Extending such architectures to genuinely open-vocabulary segmentation requires either retraining on large-scale vision–language corpora or adding a separate recognition head.
		
		In parallel, foundation segmentation models such as SAM~\cite{kirillov2023_sam} and SAM~2~\cite{ravi2024_sam2} took a different path.
		They abandon fixed labels altogether, and instead learn a powerful \emph{promptable segmentation} model trained on billions of masks (SA-1B).
		Given an image and spatial prompts (points, bounding boxes, or coarse masks), SAM returns high-quality instance masks even for rare or never-before-seen categories.
		However, SAM and SAM~2 do not know \emph{which} object to segment from text; they require the user to specify prompts in image coordinates.
		
		Open-vocabulary detectors such as Grounding~DINO~\cite{liu2023_groundingdino} fill the complementary gap.
		Grounding~DINO extends DETR-style detection to arbitrary phrases by aligning region features with text embeddings, returning boxes and phrase matches for arbitrary natural language queries.
		Yet Grounding~DINO only produces bounding boxes; it does not output segmentation masks and thus cannot be used directly for fine-grained per-pixel tasks.
		
		Grounded SAM is motivated precisely by these complementary strengths and weaknesses.
		It asks: \emph{Instead of training another huge model, can one assemble existing open-world detectors and promptable segmenters into a single pipeline that supports open-vocabulary detection, segmentation, and more complex tasks?}
		
		\paragraph{Model assembly as an alternative to unified training}
		The introduction of Ren et al.~\cite{ren2024_groundedsam} contrasts three families of approaches for open-world vision tasks.
		
		\begin{itemize}
			\item \textbf{Unified models.}
			Large, multi-task models such as UNINEXT or OFA are trained end-to-end on a mixture of datasets to support multiple tasks in one network.
			They are conceptually elegant, but require huge training compute, large curated datasets, and are difficult to maintain or extend once deployed.
			Moreover, performance on specialized tasks like open-set segmentation is often limited by compromises inherent in joint training.
			\item \textbf{LLM-as-controller pipelines.}
			Recent systems such as HuggingGPT and Visual ChatGPT treat vision models as callable tools under the control of a large language model.
			These systems are flexible and user-friendly, but remain bottlenecked by the availability and quality of specialized vision tools for core tasks like segmentation.
			\item \textbf{Ensemble foundation models (Grounded SAM).}
			Grounded SAM chooses a middle ground: instead of training a new foundation model or delegating everything to an LLM, it \emph{assembles} existing open-world models into a modular pipeline.
			Open-vocabulary detection is delegated to Grounding~DINO, pixel-accurate segmentation to SAM or SAM~2, tagging to RAM, captioning to BLIP, generation to Stable Diffusion, and 3D pose/mesh recovery to OSX.
			This decomposition retains the strengths of each expert while enabling new compound tasks like automatic open-vocabulary dense annotation and prompt-based human motion analysis.
		\end{itemize}
		
		The next parts describe the core detection–segmentation pipeline, then detail how Grounded SAM composes additional experts around this core.
		
		\subsubsection{Core pipeline: from text prompts to segmentation masks}
		\label{subsubsec:chapter15_groundedsam_method}
		
		Grounded SAM’s central component is an open-vocabulary detection–and–segmentation pipeline that takes an image and natural language prompt as input and outputs bounding boxes, category phrases, and corresponding masks.
		
		\newpage
		
		Conceptually, it decomposes open-set segmentation into two steps:
		
		\begin{enumerate}
			\item Use Grounding~DINO to perform \emph{open-set detection} and obtain bounding boxes associated with text phrases and confidence scores.
			\item Use SAM (or SAM~2 / HQ-SAM variants) to perform \emph{promptable segmentation} conditioned on those boxes.
		\end{enumerate}
		
		While the paper is largely qualitative and does not introduce new losses or training objectives, it is useful to formalize this pipeline.
		
		\paragraph{Notation and problem setup}
		Let $I \in \mathbb{R}^{H \times W \times 3}$ denote an RGB image and $T$ a text prompt that may contain one or multiple phrases describing desired targets, for example
		\[
		T = \text{``Butterfly, bag, shoes, hair, white T-shirt.''}.
		\]
		Grounding~DINO tokenizes $T$ into phrases $\{p_k\}_{k=1}^K$ and produces text embeddings $\mathbf{t}_k$.
		Given $(I,T)$, the detector predicts a set of $N$ candidate regions
		\[
		\mathcal{R} = \{(b_i, s_i, q_i)\}_{i=1}^{N},
		\]
		where $b_i$ is a bounding box in image coordinates, $s_i$ is a scalar confidence score, and $q_i \in \{1,\dots,K\}$ is the index of the matched text phrase for that box.
		Boxes and phrase matches are obtained by a DETR-style decoder with cross-attention between visual and textual tokens, as detailed in the Grounding~DINO enrichment.
		
		SAM receives the image $I$ and a set of spatial prompts $\{b_i\}$ and returns a collection of binary masks $\{M_i\}$ and quality scores $\{\hat{s}_i\}$, where each $M_i \in \{0,1\}^{H \times W}$ is a segmentation mask corresponding approximately to $b_i$.
		Grounded SAM associates each mask $M_i$ with the phrase $p_{q_i}$ and the combined score $s_i$ (possibly fused with $\hat{s}_i$).
		
		Formally, the pipeline implements a mapping
		\[
		(I, T) \longmapsto \{(p_{q_i}, b_i, M_i, s_i)\}_{i=1}^{N'},
		\]
		where $N'$ is the number of detections remaining after thresholding and non-maximum suppression (NMS).
		
		\paragraph{Step 1: open-vocabulary detection with Grounding DINO}
		Grounding~DINO~\cite{liu2023_groundingdino} encodes the image with a Vision Transformer and the text prompt with a BERT-style text encoder.
		Through a multi-stage feature enhancer and cross-modality decoder, it produces region features aligned with text tokens and predicts:
		
		\begin{itemize}
			\item Bounding boxes $\{b_i\}_{i=1}^{N}$ in $(x_{\min}, y_{\min}, x_{\max}, y_{\max})$ coordinates.
			\item Per-box matching scores over phrases, often realized as dot products between region features and text embeddings, followed by a sigmoid.
		\end{itemize}
		
		Grounded SAM uses the official Grounding~DINO implementations for both Base and Large backbones.
		Given detection outputs, it applies configurable thresholds:
		
		\begin{itemize}
			\item A \textbf{box threshold} $\tau_{\text{box}}$ serves as a primary confidence filter, keeping only regions whose maximal alignment score with the prompt exceeds a preset value.
			\item A \textbf{text threshold} $\tau_{\text{text}}$ further filters the phrase associations for each surviving box, ensuring that the predicted phrase is strongly aligned with the corresponding region.
		\end{itemize}
		
		Boxes failing either threshold are discarded, and the remaining boxes undergo standard NMS.
		The result is a set of high-confidence, text-labeled boxes $\{(b_i, p_{q_i}, s_i)\}_{i=1}^{N'}$.
		
		\paragraph{Step 2: promptable segmentation with SAM}
		SAM~\cite{kirillov2023_sam} consists of a ViT image encoder, a prompt encoder, and a mask decoder.
		For Grounded SAM, the relevant prompt type is the bounding box prompt.
		For each selected box $b_i$:
		
		\begin{enumerate}
			\item The image $I$ is resized and padded to $1024\times1024$ resolution (SAM’s default) and passed once through the SAM image encoder to obtain an image embedding $\mathbf{E} \in \mathbb{R}^{H' \times W' \times C}$.
			\item The box $b_i$ is transformed to the resized coordinate frame and encoded by SAM’s prompt encoder into a low-dimensional embedding $\mathbf{p}_i$.
			\item The mask decoder attends to $\mathbf{E}$ conditioned on $\mathbf{p}_i$, producing several candidate masks $\tilde{M}_i^{(k)}$ and corresponding mask quality scores $\hat{s}_i^{(k)}$.
		\end{enumerate}
		
		In the official pipeline, only the highest-quality mask per box is retained:
		\[
		M_i = \tilde{M}_i^{(k^\star)}, \quad k^\star = \arg\max_k \hat{s}_i^{(k)}.
		\]
		The combined confidence for the instance can be taken as $s_i$ (from Grounding~DINO), $\hat{s}_i^{(k^\star)}$ (from SAM), or a product of both; the public code uses a simple combination that preserves the detector’s ranking.
		
		\paragraph{Step 3: merging detections and masks}
		Once masks are predicted for all boxes, Grounded SAM performs light-weight post-processing:
		
		\begin{itemize}
			\item Masks with extremely small area or low combined confidence are suppressed.
			\item Overlapping masks can be resolved by favoring higher-scoring instances, optionally in a class-wise manner (that is, by phrase).
			\item The final output is a set of instance masks $M_i$ with their associated phrases $p_{q_i}$ and boxes $b_i$, effectively yielding open-vocabulary instance segmentation.
		\end{itemize}
		
		The following figure illustrates the resulting behavior: arbitrary phrases, including long-tail species names, can be grounded to both boxes and masks.
		
		\begin{figure}[H]
			\centering
			\includegraphics[width=0.8\textwidth]{Figures/Chapter_15/GroundedSAM_examples.jpg}
			\caption{\textbf{Examples of Grounded SAM on diverse text prompts}. 
				Given natural language phrases such as ``Butterfly, bag, shoes, hair, white T-shirt'', ``Iron Man'', or fine-grained species names like ``Zale horrida'' and ``Gazania linearis'', Grounded SAM detects corresponding regions (middle column) and produces precise segmentation masks (right column).
				Several demonstration images are sampled from the V3Det dataset~\cite{wu2023_v3det}; image and figure credit: Ren et al.~\cite{ren2024_groundedsam}.}
			\label{fig:chapter15_groundedsam_examples}
		\end{figure}
		
		\paragraph{Pseudo-code for the core pipeline}
		The open-vocabulary detection–and–segmentation pipeline implemented in the official repositories can be summarized as follows.
		
			\begin{mintedbox}{python}
				def grounded_sam(image,       text_prompt,
				    box_threshold=0.25,
				    text_threshold=0.25,
				    nms_iou_threshold=0.5):
				    """
				    Core Grounded SAM pipeline:
				    open-vocabulary detection (Grounding DINO)
				    + promptable segmentation (SAM).
				    """
				
				    # 1. Open-vocabulary detection with Grounding DINO.
				    #    Returns boxes (xyxy in image coords), phrase ids,
				    #    and detection scores.
				    boxes, phrase_ids, det_scores = grounding_dino(
				    image=image,
				    text=text_prompt
				    )
				
				    # 2. Thresholding and non-maximum suppression.
				    keep = []
				    for i, (b, pid, score) in enumerate(zip(boxes,
				    phrase_ids,
				    det_scores)):
				        if score < box_threshold:
				            continue
				        if phrase_score(pid) < text_threshold:
				            continue
				        keep.append(i)
				    boxes = boxes[keep]
				    phrase_ids = phrase_ids[keep]
				    det_scores = det_scores[keep]
				    boxes, phrase_ids, det_scores = nms(
				    boxes, phrase_ids, det_scores,
				    iou_thresh=nms_iou_threshold
				    )
				
				    # 3. Single SAM image embedding.
				    sam_image = preprocess_for_sam(image)  # resize+pad to 1024x1024
				    image_features = sam_image_encoder(sam_image)
				
				    # 4. For each box, run SAM mask decoder.
				    masks, mask_scores, phrases = [], [], []
				    for b, pid, score in zip(boxes, phrase_ids, det_scores):
				        box_prompt = encode_box_prompt(b, image_shape=image.shape)
				        candidate_masks, candidate_scores = sam_mask_decoder(
				        image_features, box_prompt
				        )
				        # Keep highest-scoring mask for this box.
				        best_idx = candidate_scores.argmax()
				        masks.append(candidate_masks[best_idx])
				        mask_scores.append(candidate_scores[best_idx])
				        phrases.append(id_to_phrase(pid))
				
				    return boxes, masks, phrases, det_scores, mask_scores
			\end{mintedbox}
		
		\subsubsection{Assembling open-world models around Grounded SAM}
		\label{subsubsec:chapter15_groundedsam_assembly}
		
		Beyond open-vocabulary instance segmentation, Grounded SAM serves as a hub connecting multiple foundation models.
		The paper and code highlight three particularly impactful compositions, all illustrated in Figure~\ref{fig:chapter15_groundedsam_usages}.
		
		\paragraph{Automatic dense image annotation with BLIP and RAM}
		For building large-scale detection/segmentation datasets, manual annotation is expensive.
		Grounded SAM integrates:
		
		\begin{itemize}
			\item \textbf{RAM}~\cite{zhang2023_ram}, a strong image tagging model trained on large-scale image–text pairs, capable of predicting thousands of semantic tags per image.
			\item \textbf{BLIP}~\cite{li2022_blip}, a vision–language model for captioning and image–text understanding.
		\end{itemize}
		
		Two complementary pipelines are described.
		
		\begin{itemize}
			\item \textbf{Tag-driven annotation (RAM + Grounded SAM).}
			RAM predicts a list of tags $\{t_k\}$ for an image.
			These tags are assembled into one or more text prompts $T$ (in practice, long tag lists are often split into several batches to respect the text encoder’s context length and avoid attention dilution) and fed to Grounded SAM, which produces boxes and masks for each tag, yielding dense, open-vocabulary instance annotations without any manual supervision.
			\item \textbf{Caption-driven annotation (BLIP + Grounded SAM).}
			BLIP generates a caption describing the scene.
			An LLM or simple noun-phrase extractor can convert this caption into a list of object phrases, again fed into Grounded SAM as the text prompt.
			This variant is especially useful in settings where human-like descriptions are more natural than category lists.
		\end{itemize}
		
		From a systems perspective, Grounded SAM supplies spatial grounding (where objects are), while RAM and BLIP supply semantic coverage (what they are).
		This composition is explicitly positioned as a practical recipe for building diverse segmentation datasets without manual labeling.
		
		\paragraph{Controllable image editing with Stable Diffusion}
		Grounded SAM also acts as a front-end for region-aware image editing with latent diffusion models such as Stable Diffusion~\cite{rombach2022_ldm}.
		The high-level pipeline is:
		
		\begin{enumerate}
			\item Use Grounded SAM with a text prompt like ``bench'' to obtain a mask $M$ for the target object.
			\item Convert $M$ into an inpainting mask (typically a binary map where $1$ marks pixels to be resynthesized) and pass $(I, M)$, together with a new text prompt (for example, ``a bench with floral upholstery''), to Stable Diffusion’s inpainting model.
			\item The diffusion model resynthesizes only the masked region while preserving the unmasked pixels of the original scene.
		\end{enumerate}
		
		Figure~\ref{fig:chapter15_groundedsam_usages} (third row) demonstrates editing a dog’s bench into different textures while maintaining consistent background and dog appearance.
		The key insight is that, for controllable editing, precise region masks are more valuable than global CLIP-style text–image alignment; Grounded SAM provides exactly such masks given natural language descriptions.
		
		\paragraph{Promptable human motion analysis with OSX}
		Finally, Grounded SAM integrates with OSX~\cite{lin2023_osx}, a one-stage 3D whole-body mesh recovery model that predicts SMPL-X parameters from images.
		Typically, OSX operates on person crops localized by an off-the-shelf detector.
		Grounded SAM replaces this detector with text-driven grounding:
		
		\begin{enumerate}
			\item A user query such as ``the person in white clothes'' is passed, together with the input frame, into Grounded SAM, obtaining a mask and bounding box for the target person.
			\item The RGB crop defined by this bounding box is fed into OSX, which estimates 3D body, hand, and face meshes (background context inside the crop is retained, since it helps resolve pose and camera ambiguity).
			\item Downstream analytics or visualization tools operate on the reconstructed 3D motion.
		\end{enumerate}
		
		This composition realizes \emph{promptable motion analysis}: text specifies which subject to track, Grounded SAM localizes and segments that subject across frames (often using SAM~2 for video), and OSX reconstructs the 3D motion.
		
		\newpage
		
		\subsubsection{Architecture and implementation details}
		\label{subsubsec:chapter15_groundedsam_architecture}
		
		Although Grounded SAM introduces no new neural architectures, implementation choices significantly affect usability and performance.
		
		\paragraph{Backbones and model variants}
		The paper and code support several detector–segmenter combinations.
		
		\begin{itemize}
			\item \textbf{Grounding DINO variants.}
			Experiments primarily use Grounding~DINO-Base and Grounding~DINO-Large backbones, which differ in depth and width of the visual backbone (often Swin transformers) and the text encoder configuration.
			\item \textbf{Segmentation backbones.}
			For segmentation, SAM-H (ViT-H backbone) is the default, but the public demos also support SAM-B and SAM-L, HQ-SAM (a high-quality mask-refinement variant of SAM), and efficient SAM versions for faster inference.
			\item \textbf{Grounded HQ-SAM.}
			``Grounded HQ-SAM''~\cite{ke2023_hqsam} denotes the configuration combining Grounding~DINO-Base with HQ-SAM-H, used in some experiments to probe the effect of segmentation quality.
		\end{itemize}
		
		\paragraph{Preprocessing and coordinate handling}
		Coordinate consistency between Grounding~DINO and SAM is crucial.
		
		\begin{itemize}
			\item The input image is read at its original resolution, and Grounding~DINO’s preprocessing stack (typically resizing the shorter side and normalizing) is applied before detection.
			\item Detected bounding boxes are output either in the resized image frame or in normalized $[0,1]$ coordinates; they must first be projected back to absolute coordinates on the \emph{original} image.
			\item Before segmentation, the original image is resized and padded to SAM’s square input size (for example, $1024\times1024$), and the boxes are then rescaled and shifted into this coordinate system, preserving the spatial correspondence between boxes and content.
		\end{itemize}
		
		The official repositories hide these details behind helper functions, but for reproducibility in research code it is important to respect this two-stage rescaling.
		
		\paragraph{Thresholds and hyperparameters}
		Default hyperparameters in the demo scripts include:
		
		\begin{itemize}
			\item Box threshold $\tau_{\text{box}}$ in the range $[0.25, 0.5]$, controlling detector confidence.
			\item Text threshold $\tau_{\text{text}}$ typically around $0.2$--$0.25$, filtering weak phrase--region alignments.
			\item NMS IoU threshold around $0.5$.
			\item Maximum number of detections per image, often $N' \leq 100$, to keep segmentation overhead manageable.
		\end{itemize}
		
		In practice, users adjust these according to their application: lower thresholds for recall-oriented data annotation, higher thresholds for interactive editing.
		
		\paragraph{SAM vs.\ SAM~2}
		The original paper~\cite{ren2024_groundedsam} and the initial repositories focus on image-level SAM, but subsequent extensions (Grounded-SAM-2) integrate SAM~2~\cite{ravi2024_sam2}.
		In this configuration:
		
		\begin{itemize}
			\item Grounding~DINO is typically run only on key frames to obtain initial boxes.
			\item SAM~2’s streaming memory tracks and updates masks across subsequent frames, using the initial boxes and masks as prompts.
		\end{itemize}
		
		This design preserves the text grounding benefits of Grounding~DINO while exploiting SAM~2’s efficient video mask propagation.
		
		\subsubsection{Experiments and analysis}
		\label{subsubsec:chapter15_groundedsam_experiments}
		
		\paragraph{SegInW zero-shot benchmark}
		The primary quantitative evaluation in the paper is on SegInW (Segmentation in the Wild), a challenging zero-shot benchmark that unifies 25 segmentation datasets covering diverse domains and label spaces.
		SegInW reports mean average precision (mAP) across datasets without any training on SegInW itself.
		
		Grounded SAM is evaluated as a plug-in on top of Grounding~DINO and SAM variants, and compared to strong open-world segmentation baselines including X-Decoder, ODISE, OpenSeeD, SAN-CLIP, and UNINEXT.
		The summary results reported in Ren et al.~\cite{ren2024_groundedsam} are:
		
		\begin{itemize}
			\item A configuration with Grounding~DINO-Base and SAM-H (denoted ``Grounded-SAM (B+H)'') achieves \textbf{48.7} mean AP, substantially outperforming unified segmentation models such as OpenSeeD-L (36.7 mean AP) and UNINEXT-H (42.1 mean AP), and also improving over SAN-CLIP-ViT-L (41.4 mean AP).
			\item Replacing SAM-H with HQ-SAM-H (``Grounded-HQ-SAM (B+H)'') further improves mean AP to \textbf{49.6}, highlighting that segmentation quality can be a limiting factor once detection and text grounding are strong.
		\end{itemize}
		
		These numbers underscore a central message of Grounded SAM: composing specialized open-world detectors and segmenters can outperform complex unified models, even without joint training.
		
		\paragraph{Qualitative analysis on long-tail categories}
		Beyond SegInW, the paper presents numerous qualitative examples, some of which are shown in Figure~\ref{fig:chapter15_groundedsam_examples}.
		A notable aspect is robustness to \emph{long-tail and fine-grained categories}, for example:
		
		\begin{itemize}
			\item Botanical and entomological species such as ``Gazania linearis'' and ``Zale horrida''.
			\item Compositional or attribute-rich phrases such as ``white T-shirt'', ``yellow flower'', or ``Iron Man'' (referring to a toy figure).
		\end{itemize}
		
		Because Grounding~DINO is trained on large-scale grounding datasets and benefits from CLIP-like text--image alignment, it can localize such phrases even when segmentation datasets do not contain corresponding labels.
		SAM then refines localization to pixel-level masks, yielding high-quality visualizations.
		
		\paragraph{Effect of segmentation backbones}
		Although the paper does not present extensive ablation tables, comparisons across segmentation backbones (SAM vs.\ HQ-SAM) effectively act as an ablation on the segmentation component.
		
		\begin{itemize}
			\item Grounded-HQ-SAM consistently improves or matches Grounded-SAM on SegInW, especially on datasets where boundary precision is critical.
			\item This suggests that, once text grounding and detection are sufficiently strong, downstream performance is largely bottlenecked by mask quality, and improvements in SAM-like models translate directly into better open-world segmentation.
		\end{itemize}
		
		This observation is important for downstream practitioners: upgrading the segmentation backbone can yield measurable gains without modifying the detection or training pipeline.
		
		\newpage
		
		\subsubsection{Limitations and the case for a unified model}
		\label{subsubsec:chapter15_groundedsam_limitations}
		
		Grounded SAM is intentionally a systems paper, focusing on model assembly rather than new architectures.
		This makes it an excellent practical recipe, but it also exposes structural bottlenecks that become increasingly problematic as one pushes toward real-time, large-scale, open-set segmentation.
		These bottlenecks point directly toward the need for a unified foundation model such as SAM~3, which will be introduced next.
		
		\paragraph{Redundant encoders and runtime cost}
		A central limitation of Grounded SAM is computational redundancy.
		Because the detector (Grounding~DINO) and the segmenter (SAM or SAM~2) are distinct models with separate weights, the image must be processed multiple times.
		
		\begin{itemize}
			\item \textbf{Double encoding.}
			The pipeline typically runs a large visual backbone for Grounding~DINO (for example, a Swin-L or ViT-based encoder) to produce open-vocabulary box proposals, and then runs a second heavy backbone in SAM (for example, ViT-H) to produce mask features.
			Even though both stages extract high-level visual features, there is no shared computation between them, roughly doubling latency and memory usage compared to a single-encoder design.
			\item \textbf{Scaling to large images and video.}
			For high-resolution images or video streams, this double encoding becomes especially expensive: every frame must be encoded by the detector and then again by the segmenter.
			Even with SAM~2’s efficient memory mechanism, the need to invoke a separate open-vocabulary detector on key frames remains a significant runtime and deployment cost.
		\end{itemize}
		
		For applications that need interactive performance, deployment on edge devices, or long video sequences, this two-stage architecture is therefore a poor match.
		A natural next step is to design a single shared encoder whose features serve both open-vocabulary localization and pixel-level segmentation, as pursued by SAM~3.
		
		\paragraph{Boxes as a lossy interface between text and masks}
		Grounded SAM passes information between the text-conditioned detector and the segmenter only through bounding boxes.
		This design is simple and modular, but it introduces an information bottleneck that limits both accuracy and the expressiveness of open-set segmentation.
		
		\begin{itemize}
			\item \textbf{Heuristic coupling and thresholds.}
			The interface between models is governed by hand-tuned thresholds such as the box score threshold $\tau_{\text{box}}$ and text score threshold $\tau_{\text{text}}$, plus non-maximum suppression.
			These hyperparameters control which detections are forwarded to SAM, but they are not learned jointly with the segmentation objective, and there is no gradient flow from masks back to text embeddings or detector features.
			\item \textbf{Imperfect proxies for shape.}
			Bounding boxes are a coarse, axis-aligned approximation to object extent.
			For thin, elongated, occluded, or overlapping objects, boxes may cover large background regions or multiple instances.
			SAM is then asked to infer the intended mask from a noisy, sometimes ambiguous box prompt that was not optimized for SAM’s training distribution, leading to suboptimal boundaries or missed instances.
		\end{itemize}
		
		From the perspective of open-set segmentation, this means that text grounding and mask prediction live in separate spaces: text directly influences box proposals, but not the mask decoder.
		A unified model such as SAM~3 is motivated to let text tokens interact directly with segmentation tokens, avoiding boxes as the sole communication channel.
		
		\paragraph{Limited learning of open-set segmentation behavior}
		Because Grounded SAM assembles pre-trained components without end-to-end optimization, its ability to \emph{learn} better open-set segmentation behavior is constrained.
		
		\begin{itemize}
			\item \textbf{Frozen experts and no joint training.}
			Grounding~DINO, SAM, RAM, and BLIP are typically used in frozen form.
			Errors from later stages (such as poor masks) cannot drive improvements in the detector or text encoder, and dataset-specific supervision cannot be used to jointly refine all modules in a coherent way.
			\item \textbf{Open-vocabulary only in detection.}
			Open-vocabulary capability resides almost entirely in Grounding~DINO’s text–box matching.
			SAM’s segmentation head remains class-agnostic and is never explicitly aligned with text embeddings.
			As a result, the system behaves as an open-vocabulary detector followed by a generic segmenter, rather than a genuinely text-aware open-set segmentation model.
		\end{itemize}
		
		A unified architecture can instead learn a single multi-modal representation in which detection queries, segmentation tokens, and text embeddings are trained together, closing this gap.
		
		\paragraph{Toward SAM~3: fusing detection, text, and segmentation}
		In summary, Grounded SAM demonstrates that carefully assembling existing open-world detectors, segmenters, taggers, captioners, generative models, and 3D estimators can yield powerful open-vocabulary workflows without retraining gigantic unified models.
		At the same time, its double-encoder design, reliance on bounding boxes as a lossy interface, and lack of joint optimization limit both runtime efficiency and the quality of open-set segmentation, especially at scale.
		
		These observations motivate the next step in this sequence of enrichments: SAM~3, covered in the next subsection.
		Rather than gluing together a detector and a segmenter, SAM~3 is designed as a \emph{single} foundation architecture that unifies text grounding, region localization, and pixel-precise segmentation within one end-to-end trainable model, addressing many of the structural limitations highlighted above.
		
	\end{enrichment}
	
	\newpage
	
	\begin{enrichment}[SAM 3: Segment Anything with Concepts][subsection]
		\label{enr:subsec_chapter15_sam3}
		
		\subsubsection{Motivation: from visual prompts to concepts}
		\label{subsubsec:chapter15_sam3_motivation}
		
		\paragraph{From promptable visual segmentation to concept-level segmentation}
		
		The original Segment Anything Model (SAM)~\cite{kirillov2023_sam} introduced \emph{promptable visual segmentation} (PVS): given an image and a \emph{visual prompt} (points, boxes, or masks), the model returns a high-quality mask for the object indicated by that prompt. SAM~2~\cite{ravi2024_sam2} extended this idea to video, adding a memory-based tracker that propagates masks across frames while still relying on geometric prompts to specify \emph{which} object to follow. In both cases, prompts are spatial and instance-specific: the user must \emph{explicitly enumerate} the targets by providing a separate local prompt for each object of interest, and the model only segments objects that have been individually prompted, remaining agnostic to their semantic categories.
		
		Mask DINO~\cite{li2022_maskdino} and other DETR-style models unified detection and segmentation under a fixed label vocabulary, but remained fundamentally \emph{closed-set}: they predict over a pre-defined list of categories. Grounded SAM~\cite{ren2024_groundedsam} partially broke this limitation by composing an open-vocabulary detector (Grounding DINO~\cite{liu2023_groundingdino}) with SAM or SAM~2. Text prompts such as ``a red car'' are first converted to bounding boxes, which are then handed to SAM for mask prediction. This composite design enables open-vocabulary segmentation but keeps recognition (text-to-box) and segmentation (box-to-mask) as separate systems, with no single model deciding both \emph{whether} a concept is present and \emph{how} it should be segmented. It also inherits several practical drawbacks: two large models are executed sequentially (detector then segmentor), increasing latency and memory footprint; there is no joint optimization of box prediction and mask quality, so errors in the detector (e.g., missed or mis-localized boxes, confidence miscalibration) propagate directly into the segmentation stage; and the detector is trained to produce boxes that are optimal for detection metrics rather than for downstream mask refinement. 
		
		SAM~3~\cite{carion2025_sam3} is designed to close this gap by changing the underlying objective. It introduces \emph{Promptable Concept Segmentation} (PCS), where the input is a \emph{concept prompt} and the default output is \emph{all} instances of that concept, segmented and (for video) tracked with persistent identities. Unlike PVS---which assumes a user has already selected a specific instance to segment and requires explicit spatial enumeration of each target---PCS must first decide whether the concept is present anywhere in the media and then discover and localize every matching instance from a single semantic command. Crucially, this is done within a single model that jointly learns recognition and segmentation: the same Perception Encoder backbone and DETR-style decoder are optimized end-to-end for concept presence, localization, and mask quality, avoiding the error compounding and objective mismatch of separated detector+segmentor pipelines. In practice, this does not preclude single-object workflows: users can either fall back to PVS-style visual prompts (points, boxes) or run PCS once and then select a single predicted instance to refine and track as needed. Concept prompts are defined as either:
		\begin{itemize}
			\item \textbf{Short noun phrases.} Simple text such as ``striped cat'', ``round cell'', or ``hard hat'', restricted to a noun plus optional modifiers.
			\item \textbf{Image exemplars.} One or more bounding boxes on example instances, each marked as positive or negative.
			\item \textbf{Text--image combinations.} A noun phrase plus positive and negative exemplars to refine or disambiguate the concept.
		\end{itemize}
		
		In PCS, the model must decide both \emph{whether} the concept is present at all and, if so, \emph{which} pixels and object instances match the prompt across the entire image or video.
		
		\begin{figure}[H]
			\centering
			\includegraphics[width=0.85\textwidth]{Figures/Chapter_15/SAM3_examples.jpg}
			\caption{\textbf{Promptable visual segmentation vs.\ promptable concept segmentation in SAM~3.} The left side illustrates promptable visual segmentation (PVS), where SAM and SAM~2 segment a single object per prompt using clicks, boxes, or masks. The right side shows SAM~3's promptable concept segmentation (PCS), which segments all instances of a concept specified by a short noun phrase, image exemplars, or their combination. Figure reproduced from \cite{carion2025_sam3}.}
			\label{fig:chapter15_sam3_examples}
		\end{figure}
		
		The comparison in Figure~\ref{fig:chapter15_sam3_examples} highlights two regimes.
		\begin{itemize}
			\item \textbf{Promptable Visual Segmentation (PVS).} A user clicks on a particular object (e.g., a cat or a whale), and the model returns a mask for that specific instance, which SAM~2 can then track across a video; additional instances require additional, explicitly enumerated prompts.
			\item \textbf{Promptable Concept Segmentation (PCS).} A user issues a concept prompt like ``a striped cat'' or ``a round cell''. SAM~3 segments \emph{all} instances matching that concept in an image or video, and assigns consistent instance IDs over time. In a downstream single-object setting, one of these instances can then be selected and treated as the target of interest.
		\end{itemize}
		
		\paragraph{Ambiguity and the need for new data and metrics}
		
		Open-vocabulary PCS is inherently ambiguous. Simple noun phrases can be:
		\begin{itemize}
			\item \textbf{Polysemous.} Phrases like ``mouse'' may refer to an animal or a computer device depending on context.
			\item \textbf{Subjective or vague.} Adjectives such as ``cozy'' or ``large'' depend on human judgment.
			\item \textbf{Boundary-ambiguous.} Concepts like ``mirror'' may or may not include the frame; ``toilet roll holder'' may or may not include the roll itself.
			\item \textbf{Occluded or blurred.} Partial visibility complicates deciding whether an instance should be included and where its boundaries lie.
		\end{itemize}
		Standard closed-vocabulary benchmarks deliberately avoid much of this messiness. Closed-vocabulary datasets (e.g., LVIS) mitigate these issues by carefully curating class definitions and mask guidelines. In contrast, SAM~3 targets \emph{any} simple noun phrase that can be grounded visually, which requires:
		
		\begin{itemize}
			\item \textbf{A large, diverse dataset.} With millions of unique noun phrases and high-quality instance masks across images and videos.
			\item \textbf{Evaluation protocols.} That admit multiple valid interpretations of the same phrase.
			\item \textbf{Model components.} Specifically designed to decouple recognition (``is this phrase present?'') from localization (``where are its instances?'') and to handle ambiguous cases.
		\end{itemize}
		
		\newpage
		
		To this end, SAM~3 introduces a large-scale \emph{Segment Anything with Concepts} (SA-Co) dataset and a \emph{classification-gated F1} metric (cgF1) tailored for PCS, discussed below in the Experiments part. The following figure previews qualitative improvements over a strong open-vocabulary baseline.
		
		\begin{figure}[H]
			\centering
			\includegraphics[width=0.85\textwidth]{Figures/Chapter_15/SAM3_improving_segmentation.jpg}
			\caption{\textbf{Qualitative comparison of open-vocabulary segmentation.} Examples from the SA-Co benchmark where SAM~3 improves over OWLv2~\cite{minderer2024_owlvitv2}. For prompts such as ``a white flower'', ``young plant'', or ``colander'', SAM~3 more accurately identifies the intended objects, handles fine detail, and avoids common confusions (e.g., masking the pan instead of the colander). Figure reproduced from \cite{carion2025_sam3}.}
			\label{fig:chapter15_sam3_improving_segmentation}
		\end{figure}
		
		\subsubsection{Method: promptable concept segmentation}
		\label{subsubsec:chapter15_sam3_method}
		
		\paragraph{Task inputs and outputs}
		
		Formally, SAM~3 solves the PCS task defined in Section~2 of the paper~\cite{carion2025_sam3}: given a concept prompt and a piece of visual media, it must decide whether the concept is present and, if so, detect, segment, and (for videos) track \emph{all} matching instances.
		
		\textbf{Inputs.}
		\begin{itemize}
			\item \textbf{Media.} A single RGB image or a short video clip.
			\item \textbf{Concept prompt.} A global description of the target concept, applied to the \emph{entire} image or video. It can take any of the following forms:
			\begin{itemize}
				\item \textbf{Text-only.} A simple noun phrase (NP), such as ``striped cat'', ``round cell'', or ``hard hat''.
				\item \textbf{Exemplar-only.} One or more image exemplars, given as bounding boxes labeled positive or negative, defining the concept purely in visual terms.
				\item \textbf{Text + exemplars.} A noun phrase combined with positive/negative exemplars to refine or disambiguate the concept.
			\end{itemize}
			Unlike PVS prompts, which are tied to a specific instance (a particular click or box), PCS prompts define the \emph{concept} and ask the model to find all instances that match it.
		\end{itemize}
		
		\textbf{Outputs.}
		\begin{itemize}
			\item \textbf{Instance-level outputs.} A set of object hypotheses for the concept, each with a bounding box, an instance mask, and a confidence score (used for cgF1 evaluation and downstream selection).
			\item \textbf{Semantic output.} A binary segmentation map indicating, for each pixel, whether it belongs to \emph{any} instance of the prompted concept (obtained by aggregating instance masks).
			\item \textbf{Video tracks.} For videos, a collection of spatio-temporal \emph{masklets}: sequences of masks with persistent identities across frames, representing how each instance of the concept moves and evolves over time.
		\end{itemize}
		
		\newpage
		
		\subsubsection{Architecture and implementation details}
		\label{subsubsec:chapter15_sam3_architecture}
		
		\paragraph{High-level data flow from prompts to outputs}
		
		The previous paragraphs defined PCS at the \emph{interface} level: given a media input (image or short video) and a concept prompt, SAM~3 returns instance-level hypotheses, semantic masks, and (for video) temporally consistent masklets. Internally, these outputs are produced by a tightly coupled but modular architecture:
		\begin{itemize}
			\item \textbf{Perception Encoder (PE).} A shared vision backbone that extracts a multi-scale feature pyramid for each frame.
			\item \textbf{Prompt-conditioned DETR-style detector.} Consumes PE features and prompt tokens (text and exemplars) to predict concept-specific queries, boxes, and masks.
			\item \textbf{Global presence head.} Decides whether the concept exists anywhere in the input and gates the local query scores.
			\item \textbf{SAM~2-style tracker.} Propagates masklets over time using PE features and a memory bank, periodically re-anchored by the detector.
			\item \textbf{Training pipeline and data engine.} Jointly shape the PE, detector, and tracker, and provide large-scale SA-Co supervision tailored to PCS.
		\end{itemize}
		
		At inference time for an image:
		\begin{itemize}
			\item \textbf{PE.} Produces a multi-scale feature pyramid from the input image.
			\item \textbf{Prompt encoders.} Turn the concept prompt (text and optional exemplars) into a sequence of tokens.
			\item \textbf{Detector.} A fusion encoder and DETR-style decoder output query-wise scores, boxes, and masks.
			\item \textbf{Presence head.} Gates these scores to yield calibrated instance-level and semantic predictions.
		\end{itemize}
		For videos, the same detector runs per frame while the tracker maintains and updates masklets using a memory bank and periodic re-prompting from the detector.
		
		\begin{figure}[H]
			\centering
			\includegraphics[width=0.85\textwidth]{Figures/Chapter_15/SAM3_architecture.jpg}
			\caption{\textbf{Overview of the SAM~3 architecture.} A shared Perception Encoder (PE) backbone feeds both a DETR-style concept detector (yellow, ``new in SAM~3'') and a SAM~2-style tracker (blue). The detector consumes vision features and prompt tokens (text and exemplars) to find concept instances, while the tracker propagates masklets over time using a memory bank. Their outputs are merged to produce concept masks and IDs for each frame. The ``Image Encoder'' block in the diagram corresponds to the frozen, spatially aligned branch of the shared Perception Encoder (PE) backbone. Figure reproduced from \cite{carion2025_sam3}.}
			\label{fig:chapter15_sam3_architecture}
		\end{figure}
		
		The next paragraphs first detail the \textbf{Perception Encoder} backbone that powers the system, then describe how its features are consumed by the \textbf{Detector}, \textbf{Presence head}, and \textbf{Tracker}, and finally summarize the \textbf{Training} stages and \textbf{Data pipeline} that make this unified PCS design effective.
		
		\subsubsection*{The Perception Encoder (PE)}
		\label{subsubsec:chapter15_sam3_pe}
		
		\paragraph{Motivation: beyond CLIP for dense prediction}
		
		Standard contrastive vision--language models such as CLIP~\cite{radford2021_clip} excel at zero-shot \emph{image-level} classification and retrieval, but they are poorly suited to dense prediction tasks. The CLIP objective compresses an entire image into a single global vector, and the network is explicitly trained to discard spatial detail that is not needed to decide \emph{which} caption matches the image. This is ideal for global recognition, but problematic for tasks like segmentation and tracking that require precise boundaries and temporally stable spatial structure.
		
		Conversely, dense models such as SAM~2~\cite{ravi2024_sam2} are optimized for geometry, not semantics; they produce excellent masks given geometric prompts, but they do not natively understand open-vocabulary text. Grounding-style models partially bridge this gap by pairing a CLIP-like encoder with a detector, but they still treat \emph{global semantics} and \emph{dense geometry} as largely separate modules.
		
		The Perception Encoder (PE)~\cite{bolya2025_perceptionencoder} is designed as a ``better CLIP'' that is directly usable for dense and video tasks. It provides a \emph{single} large vision trunk whose representations are:
		\begin{itemize}
			\item \textbf{Semantically expressive.} Capable of interpreting open-vocabulary text prompts via contrastive vision--language pretraining.
			\item \textbf{Spatially precise.} Preserving pixel-level geometry and object boundaries required for mask decoding.
			\item \textbf{Temporally stable.} Robust under frame-to-frame variations, which is essential for tracking.
		\end{itemize}
		This combination is particularly attractive for SAM~3: instead of gluing together a CLIP-like semantic encoder and a separate dense backbone, SAM~3 can share one alignment-tuned visual encoder across concept detection, segmentation, and tracking.
		
		\paragraph{Two-stage design: PE Core and alignment-tuned variants}
		
		Conceptually, PE provides a single high-capacity visual trunk that is trained once and then adapted to two usage regimes:
		\begin{itemize}
			\item \emph{PE Core:} a contrastively trained vision backbone used as the starting point.
			\item \emph{Alignment-tuned variants:} \emph{PE Language} for multimodal language modeling and \emph{PE Spatial} for dense prediction and tracking.
		\end{itemize}
		
		PE follows a \emph{two-stage} training process:
		\begin{enumerate}
			\item \textbf{Stage~1: Contrastive vision--language pretraining (PE Core).} Starting from an OpenCLIP ViT-L/14 baseline, PE Core is trained on large-scale image and video data with a CLIP-style contrastive objective. For each image $x$ and paired text $y$, a vision encoder $f_{\theta}$ and a text encoder $g_{\phi}$ produce global embeddings, and a temperature-scaled InfoNCE loss encourages matched pairs to have high cosine similarity and mismatched pairs to be far apart. Video clips are treated as additional views with temporal augmentations and shared captions.
			\item \textbf{Stage~2: Alignment tuning from intermediate layers.} After contrastive pretraining, PE Core’s \emph{intermediate} layers are probed. Lightweight heads are then trained on top of frozen (or lightly tuned) intermediate features to specialize the encoder for:
			\begin{itemize}
				\item \textbf{Language alignment (PE Language).} Features are projected into a multimodal LM token space so that downstream LMs can ``read'' them as visual tokens (e.g., for OCR, captioning, VQA).
				\item \textbf{Spatial alignment (PE Spatial).} Features are aligned to dense prediction teachers (including SAM~2.1) so that final-layer tokens remain geometrically precise while retaining strong semantics.
			\end{itemize}
		\end{enumerate}
		
		The following figure summarizes this framework: the left block shows contrastive pretraining; the middle illustrates frozen-feature extraction from intermediate layers; and the right side shows the two specialized heads, \emph{PE Language} and \emph{PE Spatial}.
		
		\begin{figure}[H]
			\centering
			\includegraphics[width=0.85\linewidth]{Figures/Chapter_15/Perception_Encoder_highlevel.jpg}
			\caption{\textbf{Perception Encoder (PE) framework.} PE first undergoes large-scale contrastive pretraining on images and videos to produce a unified, high-capacity vision backbone (\emph{PE Core}). Alignment tuning repurposes intermediate representations via frozen-feature extraction to produce specialized variants. \emph{PE Language} focuses on semantic alignment for OCR, captioning, and Q\&A; \emph{PE Spatial} prioritizes geometric precision for detection, segmentation, depth, and tracking. Figure taken from ~\cite{bolya2025_perceptionencoder}.}
			\label{fig:chapter15_pe_overview}
		\end{figure}
		
		Promptable Concept Segmentation (PCS) places three simultaneous demands on the visual backbone:
		\begin{itemize}
			\item \textbf{Semantically expressive.} The model must interpret short noun phrases and relate them to visual content.
			\item \textbf{Spatially precise.} The model must preserve fine-grained geometry to decode masks and boundaries.
			\item \textbf{Temporally stable.} The model must support tracking of masklets across video sequences without drift.
		\end{itemize}
		PE’s two-stage design and dual alignment pathways are specifically constructed to meet these requirements, enabling SAM~3 to unify open-vocabulary reasoning, dense segmentation, and robust tracking.
		
		\paragraph{Stage~1: PE Core --- contrastive pretraining and robust features}
		
		PE Core uses a CLIP-style contrastive objective but with a heavily refined training recipe aimed at robustness and transfer, rather than just ImageNet accuracy. At a high level, PE Core encodes an input image $x$ into a grid of patch tokens
		\[
		z = f_{\theta}(x) \in \mathbb{R}^{H'W' \times D},
		\]
		and an attention-pooling block produces a global embedding $h(x) \in \mathbb{R}^D$ from these tokens. Paired text $y$ is encoded by a separate text tower into $t(y) \in \mathbb{R}^D$, and a symmetric contrastive loss is applied over $(h(x), t(y))$ across the batch. For video, temporal crops, frame sampling, and shared captions are used to create additional positive pairs.
		
		The evolution of the PE Core recipe is shown in the following figure.
		
		\begin{figure}[H]
			\centering
			\includegraphics[width=0.75\linewidth]{Figures/Chapter_15/Perception_Encoder_pretraining.jpg}
			\caption{\textbf{Evolution of the PE pretraining recipe.} Ablations of each training improvement over an OpenCLIP baseline. Inner bars show robustness (average over six benchmarks); outer bars show ImageNet top-1 accuracy. Several steps---notably progressive resolution training, LAMB optimizer, tuned augmentation, and masked regularization---significantly boost robustness without increasing compute. Figure taken from ~\cite{bolya2025_perceptionencoder}.}
			\label{fig:chapter15_pe_pretraining}
		\end{figure}
		
		The main modifications, and how they are implemented, are:
		\begin{itemize}
			\item \textbf{Progressive resolution.} Training begins at low resolution (e.g., $128\times 128$) and progressively increases to high resolution (e.g., $448\times 448$). Early in training, smaller images reduce FLOPs and stabilize optimization; later, higher resolutions restore fine detail, improving downstream dense prediction without requiring a full high-resolution schedule from scratch.
			\item \textbf{Large-batch optimization with LAMB.} Contrastive learning benefits from very large batch sizes (tens of thousands of examples) to provide many hard negatives. However, standard optimizers such as AdamW struggle at this scale. PE therefore uses LAMB (Layer-wise Adaptive Moments optimizer for Batch training), which computes for each layer $l$ a \emph{trust ratio}
			\[
			r_l = \frac{\|w_l\|_2}{\|\hat{g}_l\|_2 + \epsilon},
			\]
			where $w_l$ are the weights and $\hat{g}_l$ is the Adam-style preconditioned gradient. The actual update is then scaled by $r_l$, so layers with small weights and large gradients receive smaller effective steps and vice versa. This layer-wise normalization lets the optimizer safely scale to very large batch sizes without divergence, improving the quality of the contrastive negatives.
			\item \textbf{2D RoPE positional embeddings.} Instead of learned absolute positional embeddings tied to a specific resolution, PE uses 2D Rotary Positional Embeddings (RoPE) applied to query/key vectors in attention. This encodes relative positions in a resolution-agnostic way, improving robustness to crops, rescaling, and aspect-ratio changes.
			
	       \item \textbf{Attention pooling.} PE replaces the standard ViT/CLIP strategy of using a
	       single learned CLS token for global aggregation with a dedicated
	       \emph{attention-pooling} module (AttnPool)~\cite{zhai2023_siglip,bolya2025_perceptionencoder}.
	       This module summarizes the entire spatial feature map \emph{after} the backbone
	       has finished processing it, rather than forcing global aggregation to happen
	       \emph{inside} the backbone itself.
	       
	       \newpage
	       
	       \noindent\textbf{How global pooling normally works in ViT/CLIP.}
	       In a standard ViT, we prepend a learnable CLS token $c_0$ to the input
	       sequence. During each of the $L$ transformer layers, this CLS token participates
	       in full self-attention:
	       \[
	       c_{\ell} = \text{SelfAttn}_{\ell}(c_{\ell-1}, Z_{\ell-1}),
	       \quad \ell = 1,\dots,L,
	       \]
	       so by the end of the network, $c_L$ has absorbed information from
	       \emph{every patch}. However, this mechanism has two drawbacks:
	       \begin{itemize}
	       	\item \textbf{Global mixing leaks into the backbone.} Since CLS attends to
	       	patches at every layer, patches also attend back to CLS. This
	       	gradually spreads global context into every patch token.
	       	Spatial tokens lose their locality and become ``washed out,'' which is
	       	harmful for precise mask boundaries.
	       	\item \textbf{A single token must represent everything.}
	       	The network is forced to compress semantics into one vector
	       	$c_L$ through $L$ layers of coupled attention, which makes optimizing
	       	both global semantics and local structure simultaneously difficult.
	       \end{itemize}
	       
	       \medskip
	       \noindent\textbf{How PE's attention pooling works and why it is different.}
	       Instead of using the CLS token as an in-network aggregator, PE treats the
	       backbone purely as a \emph{spatial feature extractor}. The ViT backbone outputs a
	       grid of spatial tokens:
	       \[
	       Z = \{z_1,\dots,z_N\} \in \mathbb{R}^{N\times D},
	       \qquad N = H'W'.
	       \]
	       These tokens remain sharply localized because no CLS token is attending to them
	       during backbone computation.
	       
	       Only \emph{after} the final backbone block, PE attaches a lightweight
	       Transformer called the \textbf{AttnPool module}. It introduces a small set of
	       ``query'' tokens (often just a single learnable vector $q$) that attends
	       \emph{once} over the frozen spatial features:
	       \[
	       Q = q,\quad K = Z,\quad V = Z,
	       \]
	       \[
	       \tilde{c} = \text{Softmax}\!\left(\frac{QK^\top}{\sqrt{D}}\right) V.
	       \]
	       Thus, $\tilde{c}$ is a \emph{content-weighted average} of all patches.
	       Key differences from standard CLS:
	       \begin{itemize}
	       	\item \textbf{Backbone stays purely spatial.} No global token is mixed inside
	       	the ViT layers, so patch tokens maintain strong locality and detailed
	       	geometry—crucial for SAM3's mask head and tracker.
	       	\item \textbf{Global aggregation happens only at the end.}
	       	The global representation $h(x)$ is computed by only one or a few
	       	AttnPool layers, not by polluting the entire backbone with global
	       	context.
	       	\item \textbf{Pooling is learned, not fixed.}
	       	Unlike average pooling or CLS propagation, the attention weights let
	       	the model dynamically emphasize concept-relevant regions—for example
	       	focusing on a dog’s head when the query is “striped dog”.
	       \end{itemize}
	       Finally, the pooled vector is defined as:
	       \[
	       h(x) = \tilde{c}.
	       \]
	       
	       \newpage
	       
	       \noindent\textbf{Why this is beneficial for SAM~3.}
	       SAM~3 needs a backbone that can serve two roles simultaneously:
	       \begin{enumerate}
	       	\item \textbf{Provide high-resolution, local, spatially faithful features} for
	       	mask decoding and tracking.
	       	\item \textbf{Provide a global semantic embedding} compatible with CLIP-style
	       	contrastive pretraining and concept presence prediction.
	       \end{enumerate}
	       
	       A CLS-through-the-backbone design forces global semantics into every patch,
	       making spatial features less precise. AttnPool solves this tension:
	       \begin{itemize}
	       	\item Spatial tokens stay sharp --- ideal for segmentation.
	       	\item Global semantic vector $h(x)$ is produced cleanly at the end --- ideal
	       	for text–image matching and presence classification.
	       \end{itemize}
	       
	       This late, dedicated pooling step is one of the reasons the Perception Encoder
	       succeeds as a unified backbone for both global vision–language alignment and
	       dense prediction, a requirement at the heart of SAM~3.
			
			\item \textbf{Masked regularization.} Standard CLIP training only supervises the final global embedding $h(x)$, so the encoder can in principle rely on a few discriminative regions and ignore the rest of the image. To encourage PE Core to build coherent, spatially structured features at \emph{every} location, the authors add a MaskFeat-style regularizer~\cite{wei2022_maskfeat,bolya2025_perceptionencoder}. For a small fraction of the batch (roughly $1/16$), they create a heavily masked view $x_{\text{mask}}$ by dropping a large subset of patches. Both the original image $x$ and the masked image $x_{\text{mask}}$ are passed through the same encoder, producing token grids
			\[
			Z_{\text{full}} = \{z^{\text{full}}_i\}_{i=1}^N, \qquad
			Z_{\text{mask}} = \{z^{\text{mask}}_i\}_{i=1}^N.
			\]
			For all \emph{visible} positions $i$ (patches that were not masked out in $x_{\text{mask}}$), a token-level alignment loss is added:
			\[
			\mathcal{L}_{\text{mask}} = \frac{1}{|\mathcal{V}|} \sum_{i \in \mathcal{V}} \bigl(1 - \cos\bigl(z^{\text{full}}_i,\; z^{\text{mask}}_i\bigr)\bigr),
			\]
			where $\mathcal{V}$ denotes the set of visible patches and $\cos(\cdot,\cdot)$ is cosine similarity. Importantly, gradients from the CLIP loss are \emph{not} backpropagated through the masked branch, so the regularizer shapes token-level features without perturbing the main contrastive objective~\cite{bolya2025_perceptionencoder}. Intuitively, this forces each visible patch in $x_{\text{mask}}$ to predict the feature it would have had in the full image, encouraging the network to model object extent, context, and continuity rather than relying on a few texture glimpses. For SAM~3, this is exactly the regime encountered in practice: objects are frequently occluded, partially out of frame, or blurred over time. Masked regularization trains PE to produce stable, informative tokens even under such partial observations, which directly benefits PE Spatial’s dense features and, downstream, the SAM~3 detector and tracker.
			
		\end{itemize}
		
		\newpage
		
		\paragraph{Intermediate-layer hypothesis: where do dense features live?}
		
		A key empirical observation in the PE paper is that the features best suited for dense prediction do \emph{not} reside in the final transformer layer. During contrastive pretraining, the top layers are strongly optimized for global alignment with text: they pool over space, discard fine spatial details, and become highly specialized for image-level semantics.
		
		Intermediate layers, by contrast, have:
		\begin{itemize}
			\item Sufficiently high-level semantics to recognize objects and concepts.
			\item Still-preserved spatial structure, since they have not yet fully collapsed spatial information into a single global representation.
		\end{itemize}
		
		The following figure compares frozen features from several large vision models, probing each layer on semantic tasks (captioning) and spatial tasks (self-supervised detection/segmentation).
		
		\begin{figure}[H]
			\centering
			\includegraphics[width=0.85\linewidth]{Figures/Chapter_15/Perception_Encoder_layers.jpg}
			\caption{\textbf{Intermediate layer analysis across models.} Frozen features from different layers are evaluated on a captioning task (left), spatial self-supervision (middle), and PE’s own contrastive pretraining recipe (right). CLIP-like models excel at semantic tasks in deeper layers but underperform on spatial tasks; DINO-like models excel at spatial structure but are weaker on language-aligned tasks. Intermediate layers of \emph{PE Core} perform well on both, motivating alignment tuning that leverages these intermediate representations rather than only the final projected features. Figure taken from ~\cite{bolya2025_perceptionencoder}.}
			\label{fig:chapter15_pe_layers}
		\end{figure}
		
		The authors find that a \emph{single} late-intermediate layer (for example, layer $L$ near the top of the stack) strikes the best trade-off: earlier layers are too low-level, and the final layer is too spatially collapsed. One might imagine aggregating many layers (as in FPNs), but this increases memory and complexity; instead, PE selects a single strong intermediate layer and, when needed, builds a lightweight multi-scale pyramid from it. In SAM~3, this design is reflected by:
		\begin{itemize}
			\item Extracting tokens from a chosen intermediate layer of the frozen PE backbone.
			\item Passing the tokens through a shallow FPN-style adapter to obtain the multi-scale features required by the detector and tracker.
		\end{itemize}
		This avoids re-training the entire massive backbone while still providing dense, semantically rich feature maps.
		
		\paragraph{Stage~2: alignment tuning and layer selection}
		
		Once PE Core is trained, it serves as a generic vision encoder whose layers exhibit different strengths. A key empirical finding in the PE paper~\cite{bolya2025_perceptionencoder} is that the \emph{final} layer of a contrastive model is often not the best compromise for downstream tasks: it is highly optimized for global text–image matching, but tends to over-compress spatial detail. In contrast, \emph{intermediate} layers can provide a better balance between semantics and geometry.
		
		To make this precise, the authors perform a layer-wise probing experiment: for each encoder block \(l\), they freeze PE Core, attach a small task head on top of the features \(z^{(l)}(x)\), and measure performance on semantic tasks (captioning, OCR, VQA) and spatial tasks (self-supervised detection/segmentation). This reveals two “sweet spot” depths:
		\begin{itemize}
			\item A \emph{late–intermediate} layer \(L_{\text{lang}}\) (e.g., block~47 in a 50-block ViT) that performs best on language-aligned tasks. 
			\item An \emph{earlier} layer \(L_{\text{spatial}}\) (e.g., block~41) that preserves more spatial structure and performs best on dense prediction tasks.
		\end{itemize}
		Stage~2, alignment tuning, then constructs two specialized variants—\emph{PE Language} and \emph{PE Spatial}—by branching from these layers and fine-tuning lightweight heads (and, in the spatial case, the top of the backbone) under regime-specific objectives.
		
		\subparagraph*{PE Language: visual tokens for multimodal LMs.}
		
		The \emph{PE Language} variant adapts PE Core so that its features can be consumed as tokens by a multimodal large language model (MLLM), while preserving the benefits of contrastive pretraining.
		
		\begin{itemize}
			\item \textbf{Layer selection and architecture.}
			Instead of using the very last layer of PE Core, the authors branch from the late–intermediate “language-optimal” layer \(L_{\text{lang}}\) (e.g., block~47), discarding a few top blocks that are overly specialized for retrieval. A small vision projector (a 2-layer MLP) maps the tokens \(z^{(L_{\text{lang}})}(x)\) into the MLLM’s embedding space, forming a sequence of visual tokens. These are concatenated with text tokens and fed into an unfrozen Llama-style MLLM.
			\item \textbf{Training data and objective.}
			PE Language is trained on mixed multimodal data: captioning corpora, OCR-centric datasets, and vision–language QA. The loss is the standard autoregressive next-token prediction on the text side. Gradients backpropagate through the language model, the vision projector, and the top of the vision encoder, aligning the visual tokens with the MLLM’s internal language space.
			
			\newpage
			
			\item \textbf{Effect of alignment.}
			After alignment tuning, a new layer-wise probe shows that, for the PE Language variant, the \emph{final} layer now becomes the best-performing layer on language tasks. In other words, alignment tuning “lifts” the strong intermediate representation up to the end of the network while preserving the CLIP-trained trunk.
		\end{itemize}
		
		The resulting PE Language encoder is semantically aligned with text and well suited for tasks where dense geometry is less critical but precise language understanding (e.g., OCR, captioning, VQA) is paramount.
		
		\subparagraph*{PE Spatial: dense, geometry-aware features for SAM~3.}
		
		The \emph{PE Spatial} variant is tailored for dense prediction and is the one used by SAM~3. Its goal is to endow the final-layer tokens with the sharp boundaries and spatial coherence required for segmentation and tracking, without sacrificing the semantics inherited from PE Core.
		
		\begin{itemize}
			\item \textbf{Layer selection.}
			PE Spatial branches from the earlier “spatial-optimal” intermediate layer \(L_{\text{spatial}}\) of PE Core (e.g., block~41), where the representation still closely tracks object layout and fine geometry. This layer acts as a \emph{semantic teacher}: its frozen features encode which regions correspond to which concepts, but they are not yet trained to produce explicit masks.
			
			\item \textbf{Geometric teacher: SAM~2.1 masks.}
			SAM~2.1 is a higher-accuracy variant of SAM~2, trained with stronger augmentation and improved decoders to maximize boundary fidelity. It has no language or concept modeling, but produces exceptionally clean, high-frequency mask logits that capture thin structures and precise edges that CLIP-style contrastive models typically miss.
			
			For each training image \(x\), the authors run SAM~2.1 offline with a dense grid of point prompts (typically hundreds of points spread over the image). For each point, SAM~2.1 outputs one or more mask logits; stacking these over all points yields a 3D tensor
			\[
			m_{\text{SAM2.1}}(x) \in \mathbb{R}^{K \times H \times W},
			\]
			where \(K\) indexes the point-prompts / mask slots, and each slice \(m_{\text{SAM2.1}}^{(k)}(x) \in \mathbb{R}^{H \times W}\) is a logit mask that is sharply aligned to object boundaries. These masks form a dense, purely \emph{geometric} teacher: they encode where objects begin and end, including thin structures and occlusion boundaries, but carry no text or concept labels.
			
			\item \textbf{Student head and dual distillation objective.}
			Starting from PE Core, they fine-tune the upper part of the encoder and attach a shallow dense head to produce a matching tensor
			\[
			m_{\text{spatial}}(x) \in \mathbb{R}^{K \times H \times W}
			\]
			from the PE Spatial tokens. The dense head is trained to imitate SAM~2.1’s grid of masks \emph{for the same set of prompts} (same grid, same \(K\)), so that for each mask slot \(k\) and each pixel \((u,v)\), the student is asked to reproduce the teacher’s logit:
			\[
			m_{\text{spatial}}^{(k)}(x)[u,v] \approx m_{\text{SAM2.1}}^{(k)}(x)[u,v].
			\]
			At the same time, a global pooling head on top of PE Spatial is regularized to stay semantically close to PE Core at depth \(L_{\text{spatial}}\). Concretely, for each image \(x\) they minimize
			\[
			\mathcal{L}_{\text{sem}} = \lambda_{\text{sem}} \,\bigl\| h_{\text{spatial}}(x) - h_{\text{core}}^{(L_{\text{spatial}})}(x) \bigr\|_2^2,
			\qquad
			\mathcal{L}_{\text{geom}} = \lambda_{\text{geom}} \,\mathcal{L}_{\text{KD}}\bigl(m_{\text{spatial}}(x), m_{\text{SAM2.1}}(x)\bigr),
			\]
			
			\newpage
			
			where:
			\begin{itemize}
				\item \(h_{\text{core}}^{(L_{\text{spatial}})}(x)\) is a global summary (e.g., attention-pooled) from the frozen intermediate teacher layer \(L_{\text{spatial}}\) of PE Core.
				\item \(h_{\text{spatial}}(x)\) is the corresponding summary from the PE Spatial student.
				\item \(m_{\text{spatial}}(x)\) are dense mask logits predicted from PE Spatial tokens via the shallow decoder.
				\item \(\mathcal{L}_{\text{KD}}\) is a per-pixel distillation loss (e.g., cross-entropy or KL divergence) that compares teacher and student logits for every mask slot \(k\) and every pixel \((u,v)\).
			\end{itemize}
			
			Intuitively, \(\mathcal{L}_{\text{geom}}\) forces the student to reproduce SAM~2.1’s very sharp decision boundaries \emph{everywhere} in the image: along object edges, across thin structures, and in occluded regions. To succeed, the PE Spatial tokens must organize themselves so that nearby pixels on the same object map to similar features and pixels across boundaries map to clearly separated features. Compared to the original CLIP-only supervision (which only constrains the \emph{global} embedding), this token-level distillation injects detailed geometric structure into the backbone.
			
		\end{itemize}
		
		Through this dual distillation, PE Spatial inherits semantics from PE Core (via \(\mathcal{L}_{\text{sem}}\)) and boundary precision from SAM~2.1 (via \(\mathcal{L}_{\text{geom}}\)). The resulting final-layer tokens are both globally meaningful (for concept prompts and presence prediction) and geometrically sharp (for decoding instance masks and tracking), making PE Spatial an ideal shared backbone for SAM~3’s PCS detector and tracker.
		
		\paragraph{Feature visualizations: geometry--semantics tradeoff}
		
		The geometry--semantics tradeoff is visible when projecting final-layer features into a 3D color space (PCA followed by LCh mapping). The figure below compares PE variants on example images.
		
		\begin{figure}[H]
			\centering
			\includegraphics[width=0.7\linewidth]{Figures/Chapter_15/Perception_Encoder_layer_visualization.jpg}
			\caption{\textbf{Final-layer PE feature visualizations (PCA to LCh).} \emph{PE Core} (second column) captures semantics but exhibits noisy, spatially incoherent patterns. Distillation to an intermediate PE layer (third column) restores coarse spatial coherence but boundaries remain fuzzy. Distillation only to SAM~2.1 logits (fourth column) produces sharp boundaries but inconsistent semantics across similar objects. The final \emph{PE Spatial} model (fifth column) combines both: sharp edges with semantically consistent regions, ideal for dense concept segmentation and tracking. Figure taken from ~\cite{bolya2025_perceptionencoder}.}
			\label{fig:chapter15_pe_layer_visualization}
		\end{figure}
		
		\paragraph{Frozen-feature dense prediction performance}
		
		Quantitatively, PE Spatial is evaluated as a frozen backbone on a suite of dense prediction tasks: zero-shot tracking (DAVIS), semantic segmentation (ADE20K), and monocular depth estimation (NYUv2). A representative comparison (adapted from the PE paper) is:
		
		\begin{table}[H]
			\centering
			\footnotesize
			\begin{tabular}{lcccccc}
				\toprule
				& \multicolumn{2}{c}{Tracking (DAVIS) $\uparrow$} & \multicolumn{2}{c}{Segm.\ (ADE20K) $\uparrow$} & \multicolumn{2}{c}{Depth (NYUv2) $\downarrow$} \\
				\cmidrule(lr){2-3} \cmidrule(lr){4-5} \cmidrule(lr){6-7}
				Encoder & Best & Last & Best & Last & Best & Last \\
				\midrule
				DINOv2-L & $58.7$ & $58.2$ & $47.3$ & $47.3$ & $0.297$ & $0.308$. \\
				DINOv2-g & $58.5$ & $58.5$ & $48.7$ & $48.4$ & $0.279$ & $0.290$. \\
				PE Core$_{\mathrm{G}}$ & $56.8$ & $42.8$ & $41.5$ & $38.6$ & $\mathbf{0.249}$ & $0.309$. \\
				\textbf{PE Spatial$_{\mathrm{G}}$} & \textbf{61.5} & \textbf{61.5} & \textbf{49.3} & \textbf{48.9} & $0.262$ & \textbf{0.275}. \\
				\bottomrule
			\end{tabular}
			\caption{\textbf{Frozen-feature dense prediction benchmarks.} PE Spatial outperforms prior large-scale backbones on tracking (DAVIS), semantic segmentation (ADE20K), and depth estimation (NYUv2) when used as a frozen encoder, validating its suitability as a shared backbone for SAM~3’s detection and tracking components.}
			\label{tab:chapter15_pe_dense}
		\end{table}
		
		Here, ``Best'' denotes the best checkpoint encountered during training (highest validation performance), while ``Last'' denotes the final checkpoint at the end of training. 
		
		The gap between PE Core and PE Spatial illustrates the importance of the spatial alignment stage: without it, final-layer features are too semantic and spatially coarse for high-quality segmentation and tracking.
		
		\paragraph{Integration into SAM~3: running example}
		
		In SAM~3’s architecture (Figure~\ref{fig:chapter15_sam3_architecture}), the Perception Encoder provides a multi-scale feature pyramid shared by:
		\begin{itemize}
			\item \textbf{DETR-style concept detector.} Consumes PE Spatial features plus text and exemplar tokens to predict concept-conditioned queries, boxes, and masks.
			\item \textbf{SAM~2-style tracker.} Uses the same PE Spatial features to propagate masklets over time via a memory bank and transformer-based propagation.
		\end{itemize}
		
		To make the flow concrete, consider a single image $I$ of a crowded street with several \emph{red buses} and many other objects, and a prompt $P =$ ``red bus''. SAM~3 processes this input as follows.
		\begin{enumerate}
			\item \textbf{Feature extraction (PE Spatial).} The image $I$ is passed through PE Spatial, whose ViT backbone produces a single high-resolution feature map at a fixed stride (typically 14 or 16). Because a single-scale feature map is insufficient for detecting objects of very different sizes, SAM~3 converts this map into a \emph{multi-scale} pyramid
			\[
			\{F^{(s)}\}_{s=1}^{S},
			\]
			where each $F^{(s)}$ is a feature map at a different spatial resolution. 
			A lightweight FPN-style adapter upsamples the backbone output to create a finer map (stride~4) for small-object and boundary detail, and downsamples it to create coarser maps (stride~32 or~64) for large objects and global context. This yields $S$ pyramid levels that jointly capture fine geometry and broad semantic cues. 
			The detector and mask head consume all $\{F^{(s)}\}$ rather than a single PE output because multi-scale context is essential for accurate localization and instance segmentation across a wide range of object sizes.
			\item \textbf{Prompt encoding.} The text ``red bus'' is tokenized and encoded into a sequence of text embeddings. If exemplars are provided (e.g., a positive box on one bus), ROI pooling over $F^{(s)}$ plus positional and label embeddings yields exemplar tokens.
			\item \textbf{Fusion and decoding.} A fusion encoder conditions the visual tokens in $\{F^{(s)}\}$ on the prompt tokens via cross-attention, biasing features toward regions that might be ``red buses''. A DETR-style decoder then produces object queries, each predicting a box, a match score to the concept, and a corresponding mask.
			\item \textbf{Presence gating.} A global presence token, operating on PE Spatial’s global context, predicts whether \emph{any} ``red bus'' is present in the image. If the presence score is low, all local detections are suppressed; if high, local scores are passed through. This separation between global presence and local localization is directly reflected in the cgF1 metric discussed in the experiments section.
		\end{enumerate}
		For video, the same PE Spatial backbone is run frame by frame, and its features are shared by the detector and the SAM~2-style tracker, which propagates and periodically re-anchors masklets, as described in the subsequent subsection.
		
		\subsubsection*{Concept detector and tracker}
		
		With the PE Spatial backbone serving as the foundation, SAM~3 constructs its two primary modules: a prompt-conditioned detector and a video tracker.
		
		\paragraph{DETR-style detector conditioned on prompts}
		
		The SAM~3 detector adapts the standard DETR paradigm~\cite{carion2020_detr} to be conditional on open-vocabulary prompts. For each input frame (or image) $I$, the computation proceeds in three stages.
		
		\begin{itemize}
			\item \textbf{Multi-scale visual features.}
			The PE Spatial backbone is a ViT that, for an input of size $H \times W$, produces a single grid of patch tokens at stride $s_{\text{PE}}$ (e.g., $s_{\text{PE}} = 16$). Concretely, the final token grid can be reshaped into a feature map
			\[
			F_{\text{PE}} \in \mathbb{R}^{H' \times W' \times D}, \qquad H' = \tfrac{H}{s_{\text{PE}}}, \; W' = \tfrac{W}{s_{\text{PE}}},
			\]
			where $D$ is the channel dimension.
			
			Since DETR-style detectors and MaskFormer-style heads benefit from a multi-scale pyramid, SAM~3 adds a lightweight neck (similar to SimpleFPN) on top of $F_{\text{PE}}$. This neck produces a set of $S$ feature maps
			\[
			\{F^{(s)}\}_{s=1}^S, \qquad F^{(s)} \in \mathbb{R}^{H_s \times W_s \times C},
			\]
			at different strides (e.g., $4, 8, 16, 32$), using a combination of $1\times 1$ convolutions, upsampling, and downsampling. High-resolution levels ($s=1,2$) are critical for fine mask boundaries, while coarse levels ($s=3,4$) capture large objects and global context. PE Spatial itself is kept frozen during SAM~3 training; all gradients are confined to the neck, fusion encoder, decoder, presence head, and tracker.
			
			\item \textbf{Prompt encoders.}
			The prompt is encoded into a sequence of \emph{prompt tokens} that jointly represent the noun phrase and any image exemplars.
			\begin{itemize}
				\item \textbf{Text.} The noun phrase is tokenized and passed through a text tower, producing a sequence of text embeddings $\{t_j\}$.
				
				\newpage
				
				\item \textbf{Image exemplars.} Each exemplar consists of a bounding box $b$ and a binary label $\ell \in \{\text{pos}, \text{neg}\}$. For each exemplar, SAM~3 constructs:
				\begin{itemize}
					\item A \emph{position embedding} encoding the box coordinates $b$.
					\item A \emph{label embedding} for the sign of the exemplar (include vs.\ exclude).
					\item An \emph{ROI-pooled feature} obtained by pooling PE Spatial features from $F_{\text{PE}}$ (or from an appropriate $F^{(s)}$) over the box region.
				\end{itemize}
				These components are concatenated and passed through a small Transformer to yield a single \emph{exemplar token} per box, capturing its spatial location, inclusion/exclusion label, and visual appearance.
			\end{itemize}
			The text tokens and exemplar tokens are concatenated into a unified prompt sequence.
			
			\item \textbf{Fusion encoder and DETR decoder.}
			A fusion encoder takes the multi-scale visual tokens derived from $\{F^{(s)}\}$ and conditions them on the prompt tokens via cross-attention. In each layer, visual tokens attend to the prompt sequence, so the resulting feature maps are explicitly biased toward regions that may match the concept. This design is asymmetric: prompts influence visual features, but visual features do not update the prompt representation, keeping the prompt embedding stable across images.
			
			A DETR-like decoder then uses a set of $Q$ learned object queries to attend to these prompt-conditioned feature maps and produce object-level predictions. Unlike standard DETR, which predicts over a fixed class vocabulary, each query predicts a \emph{binary} match score relative to the input concept.
		\end{itemize}
		
		\paragraph{Decoding, losses, and mask prediction}
		
		Each decoder layer refines a set of $Q$ object hypotheses. For query $q_i$ at layer $\ell$, the head predicts:
		\begin{itemize}
			\item \textbf{Classification logit.} A scalar $s_i^{(\ell)}$ indicating whether $q_i$ matches the prompted concept (conditional on the concept being present at all; the global factorization is handled by the presence head).
			\item \textbf{Bounding box refinement.} A box $(\hat{b}_i^{(\ell)})$ obtained by adding a learned delta to the previous layer’s box prediction, following iterative refinement as in Deformable DETR~\cite{zhu2021_deformabledetr}.
		\end{itemize}
		
		Several architectural and loss-design choices are used to make this detector more robust:
		\begin{itemize}
			\item \textbf{Box-region positional bias.}
			SAM~3 augments attention with box-region positional bias~\cite{lin2023_boxposbias}. For an object query associated with a reference box, attention scores to spatial tokens are modulated by a learned function of the relative position between the token and the box (e.g., whether the token lies inside, near the edges, or far outside). This encourages the model to focus its attention on the region likely to contain the object, without resorting to multi-scale deformable attention.
			
			\item \textbf{Dual supervision from DAC-DETR and Align loss.}
			Training uses Hungarian matching between predicted queries and ground-truth objects, as in DETR, but with two important refinements:
			\begin{itemize}
				\item \emph{DAC-DETR}~\cite{hu2023_dacdetr} (Divide-And-Conquer) splits queries into groups (such as anchor and auxiliary branches) and performs matching in a way that stabilizes training and encourages diverse query utilization. This reduces degenerate behaviors where only a small subset of queries carry most of the signal.
				\item The \emph{Align loss}~\cite{cai2024_align} encourages the classification score to correlate with localization quality: predictions with high IoU to ground-truth boxes are penalized if their classification scores are low, and conversely, high scores are discouraged for low-IoU boxes.
				
				\newpage
				
				This ties confidence calibration tightly to geometric accuracy, which is crucial in open-vocabulary PCS where false positives are especially harmful.
			\end{itemize}
			The total detection loss combines bipartite-matching losses (box regression and classification) with these auxiliary terms.
			
			\item \textbf{MaskFormer-style instance masks.}
			Instance masks are produced by a MaskFormer-style head~\cite{cheng2021_maskformer}. A high-resolution pixel embedding map is computed from the top levels of the feature pyramid (e.g., by upsampling $F^{(1)}$ and fusing with other scales). Each query $q_i$ is projected into a vector of mask coefficients, and the mask logits for query $i$ are given by a dot product between these coefficients and the pixel embeddings at each location. This yields a dense mask $\hat{m}_i \in \mathbb{R}^{H\times W}$ aligned with the input image.
			
			\item \textbf{Handling mask ambiguity.}
			Certain concepts are inherently ambiguous at the mask level (e.g., ``wheel'' vs.\ ``tire'', or whether to include accessories). To model such ambiguity, the mask head predicts multiple candidate masks per query (e.g., $K$ variants), each with an associated confidence. During training, the best-matching candidate (in IoU) is used for supervision; at inference time, SAM~3 selects the most confident candidate, or, when needed, aggregates them. This lets the detector represent multiple plausible segmentations for the same object hypothesis.
		\end{itemize}
		
		Alongside instance masks, a separate semantic head aggregates query predictions into a per-pixel binary membership map for the prompted concept, yielding the \emph{semantic PCS output}.
		
		\paragraph{Presence head: decoupling recognition and localization}
		
		A central design idea in SAM~3 is to decouple \emph{recognition} (``is the concept present at all?'') from \emph{localization} (``where are its instances?''). The decoder’s object queries are inherently local, making them well suited for localization but less ideal for deciding global presence, which may depend on subtle contextual cues.
		
		SAM~3 introduces a learned global \emph{presence token} whose sole responsibility is to predict whether the noun phrase NP is present \emph{anywhere} in the input. Formally, the model factorizes the classification probability as
		\begin{equation}
			p(q_i \text{ matches NP})
			= p\bigl(q_i \text{ matches NP} \mid \text{NP is present in input}\bigr)\cdot p(\text{NP is present in input}),
			\label{eq:chapter15_sam3_presence_factorization}
		\end{equation}
		where:
		\begin{itemize}
			\item \textbf{$p(\text{NP is present in input})$.} A single scalar predicted by the presence token using global context (e.g., pooled features over the entire image or clip). This score is shared by all object queries and acts as a global gate: if it is low, all local detections are suppressed.
			\item \textbf{$p(q_i \text{ matches NP} \mid \text{NP is present in input})$.} Predicted by each proposal query $q_i$ from its local evidence and geometry, conditioned on the concept being present.
		\end{itemize}
		
		SAM~3 employs a \textbf{decoupled supervision strategy}. The presence head is always supervised with binary cross-entropy based on image-level labels (present vs.\ absent). The local object queries $q_i$ receive box/mask supervision and matching gradients \emph{only} on images where the concept is present. On negative images, they learn simply that all queries should remain ``background'' under the presence gate, rather than being forced to hallucinate boxes for a missing concept. At evaluation time, the global presence score contributes to the IL\_MCC term in cgF1, while query-level localization quality drives pmF1.
		
		\newpage
		
		\paragraph{Image exemplars and interactive refinement}
		
		SAM~3 extends SAM and SAM~2 by allowing exemplars to define or refine the concept, not just select a single instance. Given a positive bounding box on one object (e.g., a dog), the detector interprets the exemplar as ``find all objects that look like this dog''. Negative exemplars exclude specific visual modes (e.g., a different fish species) from the concept. As illustrated in Figure~\ref{fig:chapter15_sam3_optional_refinement}, adding a negative exemplar on an undesired fish species removes that sub-concept from the predicted masks while preserving the intended striped fish.
		
		During inference, exemplars are encoded as described above and concatenated with text tokens into a single prompt token sequence. By adding exemplars iteratively, users can refine both recognition (which visual mode corresponds to the phrase) and localization (which pixels belong to the intended instances).
		
		\begin{figure}[H]
			\centering
			\includegraphics[width=0.85\textwidth]{Figures/Chapter_15/SAM3_optional_refinement.jpg}
			\caption{\textbf{Interactive refinement with text and exemplars in PCS.} The initial concept prompt ``a fish'' plus a positive exemplar (green box) leads SAM~3 to segment all fish in the scene. Adding a negative exemplar (red dashed box) on an undesired species refines the concept so that only the intended striped fish are kept. Figure reproduced from \cite{carion2025_sam3}.}
			\label{fig:chapter15_sam3_optional_refinement}
		\end{figure}
		
		\paragraph{Video PCS: detector--tracker factorization}
		
		For videos, SAM~3 combines its concept detector with a SAM~2-style tracker. Given a video and prompt $P$, the detector finds concept instances on each frame while the tracker propagates existing masklets forward in time. Let $I_t$ be the frame at time $t$, $M_{t-1}$ the set of masklets from frame $t-1$, and $O_t$ the set of newly detected objects at frame $t$. SAM~3 defines
		\begin{equation}
			\hat{M}_t = \text{propagate}(M_{t-1}), \quad
			O_t = \text{detect}(I_t, P), \quad
			M_t = \text{match\_and\_update}\bigl(\hat{M}_t, O_t\bigr).
			\label{eq:chapter15_sam3_tracking_update}
		\end{equation}
		Here:
		\begin{itemize}
			\item \textbf{Propagation.} The tracker predicts the new locations $\hat{M}_t$ of previously tracked objects using a single-frame propagation step similar to SAM~2: track tokens attend to the current PE Spatial features and a memory bank of past features to update each masklet.
			\item \textbf{Detection.} The detector runs on $I_t$ with prompt $P$ to find new instances $O_t$ that match the concept, including objects that enter the scene or were previously missed.
			\item \textbf{Matching and updating.} A simple IoU-based matching function associates propagated masklets $\hat{M}_t$ with current detections $O_t$, forming the updated set of masklets $M_t$. New detections that are unmatched spawn new masklets.
		\end{itemize}
		
		To improve temporal robustness, SAM~3 introduces two strategies:
		\begin{itemize}
			\item \textbf{Masklet detection score.} For each masklet, a temporal score accumulates how consistently it has been re-matched to detector outputs over a sliding window. Masklets whose detection score falls below a threshold are suppressed, reducing drift and spurious tracks.
			\item \textbf{Periodic re-prompting.} At regular intervals, the tracker is re-anchored using high-confidence detector masks: the tracker’s internal state for a masklet is refreshed from the detector’s current prediction. This prevents the memory bank from drifting away from the true object when occlusions or appearance changes occur.
		\end{itemize}
		
		As in the image detector, the mask decoder can output multiple candidate masks per tracked object along with confidences; SAM~3 then selects the most confident mask on each frame, which helps resolve per-frame ambiguities in cluttered or low-contrast regions.
		
		\paragraph{Instance refinement with visual prompts}
		
		After initial concept detection and tracking, SAM~3 supports finer instance-level refinement with PVS-style visual prompts (points, boxes) in the SAM~2 fashion. A user can:
		\begin{itemize}
			\item \textbf{Refine a masklet.} Provide positive and negative clicks on a specific object; SAM~3 encodes the clicks and runs the mask decoder to adjust the mask on that frame.
			\item \textbf{Propagate refinements.} In video, the refined mask is propagated across the entire sequence to update the masklet consistently.
		\end{itemize}
		In many practical workflows, the user first runs PCS to discover all instances of a concept, then selects one masklet and refines it with PVS-style prompts, effectively turning PCS into single-object segmentation or tracking for that chosen instance. This design unifies concept-level prompting (PCS) with instance-level visual refinement (PVS), providing both global coverage and local precision.
		
		\begin{figure}[H]
			\centering
			\includegraphics[width=0.75\textwidth]{Figures/Chapter_15/SAM3_video_example.jpg}
			\caption{\textbf{Promptable concept segmentation in complex scenes.} Examples from the SA-Co benchmark showing SAM~3 segmenting and tracking multiple instances defined by open-vocabulary prompts (top: video sequence, bottom: crowded images). Instance IDs remain consistent over time, even under occlusion and clutter. Negative prompts help exclude look-alike distractors (e.g., non-target fruits or objects). Figure reproduced from \cite{carion2025_sam3}.}
			\label{fig:chapter15_sam3_video_example}
		\end{figure}
		
		\subsubsection*{Training and data}
		
		\paragraph{Training stages}
		
		SAM~3 is trained in four stages~\cite{carion2025_sam3}.
		\begin{itemize}
			\item \textbf{Perception Encoder pre-training.} The PE backbone is trained on large-scale vision tasks (contrastive image and video pretraining with alignment tuning) to learn strong general visual representations before being used for PCS.
			\item \textbf{Detector pre-training.} The DETR-style detector is trained with text and exemplar prompts on SA-Co and related data, supervised with both box and mask objectives and the presence head, so that it can perform open-vocabulary detection and segmentation conditioned on concept prompts.
			\item \textbf{Detector fine-tuning.} The detector is further fine-tuned on curated subsets and external datasets (e.g., LVIS, COCO, ADE-847) for PCS and related tasks, balancing open-vocabulary behavior with strong performance on standard benchmarks.
			\item \textbf{Tracker training.} With the backbone frozen, the tracker is trained on video PCS data to propagate masklets and maintain identities, using SAM~2-style propagation losses and consistency objectives.
		\end{itemize}
		
		\paragraph{Data engine and SA-Co dataset}
		
		Achieving strong PCS performance requires a large, diverse dataset over many domains. SAM~3 introduces a model- and human-in-the-loop data engine (see the below figure) that iteratively improves both the dataset and the model.
		
		\begin{figure}[H]
			\centering
			\includegraphics[width=0.85\textwidth]{Figures/Chapter_15/SAM3_data_pipeline.jpg}
			\caption{\textbf{The SAM~3 data engine.} Media are mined from a large pool and paired with noun phrases proposed by an ontology and language models. SAM~3 (and earlier models) generate candidate masks, which are verified for quality and exhaustivity by human and AI verifiers. Incomplete or low-quality masks are sent for manual correction, and the resulting high-quality annotations are fed back to retrain SAM~3. The figure depicts the mature (Phase~2+) pipeline, where AI verifiers operate alongside human verifiers rather than the initial human-only stage. Figure reproduced from \cite{carion2025_sam3}.}
			\label{fig:chapter15_sam3_data_engine}
		\end{figure}
		
		The data engine operates in phases.
		\begin{itemize}
			\item \textbf{Phase~1: Human verification.} Initial image--NP pairs are generated using SAM~2 plus an open-vocabulary detector, and all verification is done by humans, producing the first SA-Co/HQ subset with millions of pairs.
			\item \textbf{Phase~2: Human + AI verification.} Human labels from Phase~1 are used to fine-tune Llama-based AI verifiers for mask quality and exhaustivity, roughly doubling annotation throughput. Hard negative NPs are mined adversarially to challenge SAM~3.
			\item \textbf{Phase~3: Scaling and domain expansion.} AI models mine increasingly difficult cases and broaden coverage to many visual domains, while the ontology is used to expand long-tail concept coverage.
			\item \textbf{Phase~4: Video annotation.} The pipeline is extended to videos, using SAM~3 to propose masklets that are then verified and corrected, focusing human effort on crowded scenes and likely tracking failures.
		\end{itemize}
		
		\newpage
		
		From this process, the authors build several datasets for training and evaluation~\cite{carion2025_sam3}.
		\begin{itemize}
			\item \textbf{SA-Co/HQ.} High-quality image PCS data with about $5.2$M images and $4$M unique noun phrases.
			\item \textbf{SA-Co/SYN.} A synthetic dataset with about $38$M phrases and $1.4$B masks, generated using a mature data engine without human involvement.
			\item \textbf{SA-Co/EXT.} Fifteen external datasets with existing instance masks, enriched with hard negatives using the ontology.
			\item \textbf{SA-Co/VIDEO.} About $52.5$K videos and $24.8$K unique noun phrases, forming approximately $134$K video--NP pairs.
		\end{itemize}
		The SA-Co benchmark for evaluation contains $207$K unique phrases and over $3$M media--phrase pairs, spanning multiple splits (Gold, Silver, Bronze, Bio, and VEval) with varying levels of redundancy and domain focus. With this architecture, training protocol, and data engine in place, the authors next quantify how well SAM~3 performs on PCS across images and videos and how each design choice contributes to the final performance.
		
		\subsubsection{Experiments and ablations}
		\label{subsubsec:chapter15_sam3_experiments}
		
		\paragraph{Evaluation metrics: why open-vocabulary PCS needs new metrics}
		
		Conventional detection metrics such as AP (Average Precision) were developed for closed-set detection, where the model predicts over a fixed label set and is not judged on its ability to \emph{refuse} arbitrary new phrases. In Promptable Concept Segmentation (PCS), every query phrase can be novel, and evaluation must answer two separate questions:
		\begin{enumerate}
			\item \emph{Presence:} does the concept appear at all in this image or video?
			\item \emph{Localization:} given that it does, are all instances segmented accurately?
		\end{enumerate}
		Standard AP collapses these into a single ranking-based score. A model that often hallucinates confident detections for concepts that are not present can still achieve a seemingly good AP if it ranks detections well on positive images. This is misaligned with the real PCS goal:
		\emph{``Do you know when the concept is here, and when it is, can you segment it well?''}
		
		SAM~3 therefore evaluates PCS using a calibration-sensitive metric called \emph{classification-gated F1} (cgF1)~\cite{carion2025_sam3}, which explicitly factorizes the task into a localization component and a presence component.
		
		\medskip
		\noindent\textbf{Localization: positive micro-F1 (pmF1).}
		
		A standard F1 score is defined from precision and recall,
		\[
		\mathrm{F1} = \frac{2 \,\mathrm{Precision} \cdot \mathrm{Recall}}{\mathrm{Precision} + \mathrm{Recall}},
		\]
		where precision and recall are computed from counts of true positives (TP), false positives (FP), and false negatives (FN). There are two common aggregation schemes:
		\begin{itemize}
			\item \emph{Per-example (macro) F1:} compute F1 separately for each image, then average.
			\item \emph{Micro-F1:} first sum TP, FP, and FN \emph{across all examples}, then compute a single F1 from the totals.
		\end{itemize}
		
		pmF1 is a \emph{micro-F1 over instances, restricted to positive media--phrase pairs}. Concretely:
		\begin{itemize}
			\item We consider only media--phrase pairs for which the concept is known to be present (at least one ground-truth instance exists).
			\item For each such pair, predicted \emph{instances} (boxes + masks) are matched to ground-truth instances using an IoU-based one-to-one matching.
			\item Over \emph{all} positive pairs, we accumulate instance-level counts
			\[
			\mathrm{TP}_\text{pos},\quad \mathrm{FP}_\text{pos},\quad \mathrm{FN}_\text{pos},
			\]
			and define
			\[
			\mathrm{Precision}_\text{pos} =
			\frac{\mathrm{TP}_\text{pos}}{\mathrm{TP}_\text{pos} + \mathrm{FP}_\text{pos}}, 
			\qquad
			\mathrm{Recall}_\text{pos} =
			\frac{\mathrm{TP}_\text{pos}}{\mathrm{TP}_\text{pos} + \mathrm{FN}_\text{pos}},
			\]
			\[
			\mathrm{pmF1} =
			\frac{2 \,\mathrm{Precision}_\text{pos} \cdot \mathrm{Recall}_\text{pos}}
			{\mathrm{Precision}_\text{pos} + \mathrm{Recall}_\text{pos}}.
			\]
		\end{itemize}
		
		The crucial point is what pmF1 \emph{ignores}: it does not see any media--phrase pairs where the concept is absent. It answers only:
		\begin{center}
			\emph{``When the concept truly appears, how accurately do I detect and segment its instances?''}
		\end{center}
		Presence hallucinations on negative images are handled separately.
		
		\medskip
		\noindent\textbf{Presence classification: image-level MCC (IL\_MCC).}
		
		To measure whether the model correctly decides \emph{if} a concept is present, SAM~3 uses an image-level Matthews Correlation Coefficient (MCC) over all media--phrase pairs. For each pair, the ground truth provides a binary label
		\[
		y \in \{0,1\} \quad (\text{absent/present}),
		\]
		and the model predicts $\hat{y} \in \{0,1\}$ based on its global presence head and query scores. This yields four pair-level counts:
		\[
		\mathrm{TP},\ \mathrm{TN},\ \mathrm{FP},\ \mathrm{FN},
		\]
		and IL\_MCC is given by
		\[
		\mathrm{IL\_MCC} =
		\frac{\mathrm{TP} \cdot \mathrm{TN} - \mathrm{FP} \cdot \mathrm{FN}}
		{\sqrt{(\mathrm{TP}+\mathrm{FP})(\mathrm{TP}+\mathrm{FN})(\mathrm{TN}+\mathrm{FP})(\mathrm{TN}+\mathrm{FN})}}.
		\]
		
		MCC can be viewed as a \emph{correlation coefficient for binary classification}: it is $1$ for perfect predictions, $0$ for random guessing or a constant classifier, and $-1$ for perfectly wrong predictions. It is chosen here for two reasons:
		\begin{itemize}
			\item \textbf{Robust to class imbalance.} SA-Co contains far more negative than positive pairs. A trivial classifier that always predicts ``absent'' can achieve high raw accuracy, but its MCC stays near $0$. MCC therefore prevents models from exploiting imbalance by always refusing.
			\item \textbf{Symmetric treatment of FP and FN.} IL\_MCC decreases both when the model hallucinates concepts (many FP on negatives) and when it misses obvious ones (many FN on positives). Both failure modes matter for PCS deployment.
		\end{itemize}
		
		Intuitively, IL\_MCC answers:
		\begin{center}
			\emph{``Across all images and phrases, how strongly are my presence predictions correlated with reality, after accounting for imbalance and both kinds of mistakes?''}
		\end{center}
		
		\medskip
		\noindent\textbf{Combined metric: classification-gated F1 (cgF1).}
		
		cgF1 combines these two orthogonal requirements into a single scalar:
		\begin{equation}
			\mathrm{cgF1} = 100 \times \mathrm{pmF1} \times \mathrm{IL\_MCC}.
			\label{eq:chapter15_sam3_cgF1}
		\end{equation}
		This multiplicative design acts as a harsh gate. It forces a model to be \emph{simultaneously} effective at:
		\begin{itemize}
			\item \textbf{Localization} (high pmF1): accurately segmenting instances when the concept is actually present.
			\item \textbf{Calibration} (high IL\_MCC): reliably predicting ``absent'' when the concept is missing.
		\end{itemize}
		
		Consider a ``hallucination-prone'' model that segments every cat perfectly (pmF1 $\approx 1.0$) but also incorrectly claims a cat exists in every empty room (IL\_MCC $\approx 0$). Its final cgF1 will collapse to near zero. This ensures that for open-vocabulary deployment, the model learns that silence is golden: it must confidently refuse to segment irrelevant inputs. Conversely, a conservative model that almost never hallucinates (high IL\_MCC) but misses many true instances (low pmF1) will also obtain a low cgF1.
		
		cgF1 is thus directly aligned with the core PCS requirement of jointly reliable \emph{recognition} (``is the concept here?'') and \emph{segmentation} (``if so, where and how well?''). Finally, SA-Co/Gold provides three independent human annotation variants per phrase. To account for semantic and boundary ambiguity, oracle scores compare model predictions against all variants and take the best match, so that models are not penalized for choosing one reasonable interpretation among several.
		
		\paragraph{Image PCS with text prompts: large gains over prior work}
		
		SAM~3 is evaluated on instance segmentation, box detection, and semantic segmentation for a wide variety of natural language prompts~\cite{carion2025_sam3}. Baselines include OWLv2, Grounding DINO, LLMDet, Gemini~2.5, APE, and DINO-X. Three high-level takeaways emerge:
		
		\begin{itemize}
			\item \textbf{Open-vocabulary PCS on SA-Co.} On the SA-Co/Gold split, SAM~3 attains a cgF1 of \textbf{53.6}, more than \emph{doubling} the performance of OWLv2$^\star$ (cgF1 $\sim$26). This corresponds to roughly \textbf{74\%} of measured human performance. Improvements are even larger on SA-Co Silver, Bronze, and Bio.
			\item \textbf{Closed-vocabulary performance.} Zero-shot LVIS mask AP is \textbf{48.5}, which is notable because SAM~3 is not optimized for LVIS and yet surpasses prior CLIP-based detectors and approaches the supervised performance of specialist models from 2022--2023.
			\item \textbf{Open-vocabulary semantic segmentation.} On ADE-847, PascalConcept-59, and Cityscapes, SAM~3 outperforms APE---a strong specialist for open-vocabulary semantic segmentation---demonstrating that the PCS machinery generalizes from instance to pixel-wise semantics.
		\end{itemize}
		
		Qualitative comparisons in Figure~\ref{fig:chapter15_sam3_improving_segmentation} illustrate that SAM~3 handles long-tail concepts (``cheesecloth'', ``toilet roll holder'') and cluttered scenes that confuse previous systems such as OWLv2 and Grounding DINO.
		
		\paragraph{Few-shot adaptation and exemplar prompting}
		
		Few-shot transfer is evaluated on ODinW13 and RF-100VL using their native labels as prompts. Fine-tuned without mask losses, SAM~3 achieves state-of-the-art \textbf{10-shot detection}, outperforming Gemini's in-context learning and specialist detectors such as Grounding DINO.
		
		A particularly compelling aspect is the impact of exemplar prompts. With only a single positive exemplar:
		\begin{itemize}
			\item SAM~3 substantially outperforms T-Rex2 on COCO, LVIS, and ODinW.
			\item Joint text+exemplar prompting consistently produces the strongest results.
		\end{itemize}
		This suggests that SAM~3's prompt-conditioning architecture effectively fuses appearance cues (exemplars) with semantic cues (text), enabling fine-grained discrimination between visually similar subcategories.
		
		\paragraph{Efficiency of PCS vs.\ PVS prompting}
		
		One of the motivating hypotheses for SAM~3 is that PCS is fundamentally more \emph{interaction-efficient} than classical PVS. In PVS (as in SAM~2), each object typically requires an explicit prompt (point, box, or mask). PCS, by contrast, uses a single semantic prompt to discover \emph{all} instances of a concept simultaneously.
		
		This hypothesis is validated on SA-Co/Gold, where cgF1 is plotted against the number of interactive box prompts:
		
		\begin{figure}[H]
			\centering
			\includegraphics[width=0.7\textwidth]{Figures/Chapter_15/SAM3_cgF1_numprompts.jpg}
			\caption{\textbf{Efficiency of concept vs.\ visual prompting.} cgF1 on SA-Co/Gold as a function of the number of interactive box prompts. Promptable Concept Segmentation (PCS) with SAM~3 reaches high cgF1 with a single prompt, while an idealized PVS baseline (segmenting instances one by one) requires several prompts to catch up. Figure reproduced from \cite{carion2025_sam3}.}
			\label{fig:chapter15_sam3_cgF1_numprompts}
		\end{figure}
		
		The trends are striking:
		\begin{itemize}
			\item \textbf{PCS achieves $\sim$0.72 cgF1 with just one prompt.}  
			For a single prompt, PCS already outperforms a PVS baseline (roughly $\sim$0.68 cgF1 at one box prompt), and this PCS level typically requires \emph{four to five} PVS prompts in SAM~2-style annotation.
			\item \textbf{PVS scales linearly with the number of objects.}  
			For scenes with many instances (e.g., ``all screws on the table''), PVS becomes prohibitively costly, whereas PCS remains constant.
			\item \textbf{Hybrid prompting delivers the peak performance ($\sim$0.80 cgF1).}  
			Use one PCS prompt to retrieve most instances, then refine with 1--2 visual prompts where needed.
		\end{itemize}
		
		\paragraph{Domain adaptation and data engine ablations}
		
		The SAM~3 data engine produces both human-verified and synthetic annotations. To study domain adaptation, the authors evaluate on a Food-domain subset using three data sources: high-quality human annotations (HQ), synthetic annotations from mature SAM~3 teachers (SYN), and naive pseudo-labels (PL).
		
		\begin{figure}[H]
			\centering
			\includegraphics[width=0.65\textwidth]{Figures/Chapter_15/SAM3_cgF1_new_domain.jpg}
			\caption{\textbf{Scaling behavior on a new domain.} cgF1 on a Food domain vs.\ the amount of domain-specific training data. Synthetic data generated by SAM~3 plus AI verifiers (SYN) scales similarly to high-quality human-annotated data (HQ), while naive pseudo-labeled data (PL) saturates at a lower performance level. Figure reproduced from \cite{carion2025_sam3}.}
			\label{fig:chapter15_sam3_cgF1_new_domain}
		\end{figure}
		
		Three practical insights emerge:
		\begin{itemize}
			\item \textbf{Quality matters as much as quantity.} SYN data approaches HQ performance, both reaching cgF1~$\sim$56 with sufficient training volume.
			\item \textbf{Verification is essential.} PL data, without verification, plateaus early around cgF1~$\sim$45, underscoring that naive pseudo-labeling is insufficient for open-vocabulary tasks.
			\item \textbf{Teacher models can self-scale.} When paired with an AI verifier, a strong SAM~3 model can bootstrap high-quality synthetic data for new domains, reducing human annotation cost.
		\end{itemize}
		
		\paragraph{Ablations: identifying key components}
		
		A series of ablations isolates which design decisions most affect PCS performance:
		
		\begin{itemize}
			\item \textbf{Presence head (improves cgF1 by $\sim$3.5).}  
			Removing the global presence token causes the model to hallucinate concepts more often, lowering IL\_MCC and reducing cgF1. This confirms that separating global recognition from local localization is critical.
			\item \textbf{Hard-negative prompts (improves cgF1 by $\sim$2.1).}  
			Including adversarially mined negative noun phrases (e.g., “nail’’ when querying “screw’’) is essential for discriminating fine-grained concepts.
			\item \textbf{Ambiguity modeling.}  
			Allowing the mask decoder to output multiple candidate masks improves robustness on SA-Co/Gold, where human annotators legitimately disagree about which pixels belong to a concept.
			\item \textbf{Backbone capacity (PE Spatial).}  
			Upgrading to the Perception Encoder boosts both PCS and PVS performance. Compared to SAM~2’s original encoder, PE Spatial yields significantly stronger tracking, better fine detail, and improved open-vocabulary grounding.
		\end{itemize}
		
		Collectively, these experiments validate the core design of SAM~3:  
		\emph{a well-calibrated presence classifier, strong spatial-semantic features from PE, concept-level prompting, and ambiguity-aware mask decoding together produce a substantial leap in open-vocabulary segmentation and tracking.}
		
		\subsubsection{Limitations and future directions}
		\label{subsubsec:chapter15_sam3_limitations}
		
		Despite its strong performance, SAM~3 has several limitations that also suggest promising directions for future work.
		
		\paragraph{Language complexity and reasoning}
		
		SAM~3 is intentionally restricted to simple noun phrases. It is not designed to handle long referring expressions or prompts requiring complex reasoning (e.g., ``the person holding the red umbrella but not standing on the stairs''). While the authors show that SAM~3 can be combined with a Multimodal Large Language Model (MLLM) to parse such queries into simpler noun phrases and concept prompts, this is handled outside the core SAM~3 architecture. A natural next step is tighter integration between PCS models and MLLMs so that reasoning and segmentation are trained jointly.
		
		\paragraph{Ambiguity and annotation effort}
		
		Even with three annotations per phrase and an ambiguity-aware evaluation protocol, some prompts remain intrinsically ambiguous or ungroundable. The data engine partially mitigates this by allowing annotators to reject such phrases, but this requires substantial human effort and careful guideline design. Future work could explore uncertainty-aware prompting, where the model can explicitly flag phrases it cannot reliably ground.
		
		\paragraph{Domain and modality coverage}
		
		SA-Co covers many visual domains, but performance still varies across them, and specialized domains (e.g., medical imaging, scientific microscopy) may require dedicated data collection and domain-specific ontology expansions. Extending PCS to additional modalities (e.g., 3D scenes, multi-view setups) or to richer temporal reasoning (beyond short videos) remains an open research area.
		
		\paragraph{Computational cost and deployment}
		
		Although SAM~3 is optimized for efficiency---running in about $30$~ms per image with $100+$ detected objects on an H200 GPU and near real-time for a few concurrent video objects---deployment at scale still requires substantial compute and memory. Lightweight variants or distillation schemes, possibly leveraging concept-specific student models, could make PCS more accessible in resource-constrained settings.
		
		\paragraph{Compositionality and structured prompts}
		
		Finally, SAM~3 treats each noun phrase largely independently, without explicit modeling of compositional structure across multiple prompts (e.g., intersecting or subtracting concepts). Interactive exemplars partially address this, but richer structured prompting interfaces and corresponding model architectures could better exploit the compositional nature of language and concepts.
		
	\end{enrichment}

\end{enrichment}


\chapterimage{head2.png} % Chapter heading image

% Chapter-specific content starts here
\chapter{Lecture 16: Recurrent Networks}

%-----------------------------------------------------------------------------------
%    CHAPTER 16 - Lecture 16: Recurrent Networks
%-----------------------------------------------------------------------------------
\section{Introduction to Recurrent Neural Networks (RNNs)}

\noindent Many real-world problems involve sequential data, where information is not independent but instead follows a temporal or ordered structure. Traditional neural networks, such as fully connected (FC) networks and convolutional neural networks (CNNs), assume that inputs are independent of each other, making them ineffective for tasks where past information influences future outcomes. Recurrent Neural Networks (RNNs) are specifically designed to handle such problems by incorporating memory through recurrent connections, enabling them to process sequences of variable length.

\subsection{Why Study Sequential Models?}

\noindent Sequential modeling is crucial for various applications where past observations influence future predictions. Without specialized architectures, we cannot effectively solve tasks such as:

\begin{itemize}
    \item \textbf{Image Captioning (One-to-Many)}: Generating a sequence of words to describe an image requires understanding both spatial and sequential dependencies \cite{vinyals2015_showtell}.
    \item \textbf{Video Classification (Many-to-One)}: Classifying an action or event in a video requires processing frames as a sequence, capturing motion and context \cite{karpathy2014_largevideo}.
    \item \textbf{Machine Translation (Many-to-Many)}: Translating sentences from one language to another requires modeling sequential dependencies across different languages \cite{sutskever2014_seq2seq}.
    \item \textbf{Time-Series Forecasting}: Financial market predictions, weather forecasting, and power grid monitoring depend on capturing trends and long-term dependencies.
    \item \textbf{Sequence Labeling}: Named entity recognition, part-of-speech tagging, and handwriting recognition require assigning labels to elements of a sequence while maintaining context.
    \item \textbf{Autoregressive Generation}: Music composition, text generation, and speech synthesis involve generating outputs where each step depends on previous ones.
\end{itemize}

\begin{figure}[H]
    \centering
    \includegraphics[width=0.8\textwidth]{Figures/Chapter_16/slide_9.jpg}
    \caption{Illustration of different sequence modeling problems and their RNN structures: One-to-One, One-to-Many, Many-to-One, and Many-to-Many.}
    \label{fig:chapter16_rnn_types}
\end{figure}

\subsection{RNNs as a General-Purpose Sequence Model}

\noindent Unlike traditional models that require a fixed input size, RNNs provide a unified architecture for handling sequences of \textbf{arbitrary length}. This flexibility allows RNNs to process short and long sequences using the same model, making them suitable for tasks ranging from speech processing to video analysis.

\noindent Although RNNs are designed for sequential data, they can also be applied to \textbf{non-sequential tasks} by processing an input sequentially. For instance, instead of analyzing an image in a single forward pass, an RNN can take a series of glimpses and make a decision based on accumulated information.

\subsection{RNNs for Visual Attention and Image Generation}

\noindent \textbf{Recurrent Neural Networks} are traditionally used for sequence modeling, but they can also be leveraged to process images in a sequential manner. Two notable applications include:

\begin{itemize}
    \item \textbf{Visual Attention Mechanisms}: Instead of processing an entire image at once, an RNN can take a series of \emph{glimpses}, deciding where to focus next based on previous observations.
    \item \textbf{Autoregressive Image Generation}: Instead of generating an image in one step, an RNN can incrementally refine an output, painting it sequentially over time.
\end{itemize}

\subsubsection{Visual Attention: Sequential Image Processing}

\noindent A compelling use case of RNNs in non-sequential tasks is \textbf{visual attention}, where an RNN dynamically determines where to focus within an image. This approach is exemplified by \cite{ba2015_attention}, which uses an RNN to sequentially analyze different parts of an image before making a classification decision.

\begin{itemize}
    \item At each timestep, the network decides which region of the image to examine based on all previously acquired information.
    \item This process continues over multiple timesteps, accumulating evidence before making a final classification decision.
    \item A practical example is using RNNs for \textbf{MNIST digit classification}, where instead of viewing the full image at once, the network sequentially attends to different regions before determining the digit.
\end{itemize}

\subsubsection{Autoregressive Image Generation with RNNs}

\noindent Another fascinating application of RNNs is in \textbf{image generation}, as demonstrated by \cite{gregor2015_draw}. Instead of generating an entire image in one step, the model incrementally constructs it over multiple timesteps:

\begin{itemize}
    \item The model "draws" small portions of the image sequentially, refining details at each step.
    \item At each timestep, the RNN decides \textbf{where to modify the canvas} and \textbf{what details to add}.
    \item This mimics the human drawing process, where an artist sequentially sketches and refines different parts of an image.
\end{itemize}

\noindent The DRAW model \cite{gregor2015_draw} exemplifies this approach, using recurrent layers to iteratively generate and improve an image.

\noindent These examples illustrate that RNNs are not limited to temporal sequences—they can also be used in spatially structured tasks by treating an image as a sequence of observations or drawing steps.

\subsection{Limitations of Traditional Neural Networks for Sequential Data}

\noindent The inability of FC networks and CNNs to capture temporal dependencies leads to major limitations when dealing with sequential tasks. The following table highlights the key differences:

\begin{table}[H]
    \centering
    \resizebox{\textwidth}{!}{%
        \begin{tabular}{|c|c|c|c|}
            \hline
            \textbf{Characteristic} & \textbf{FC Networks} & \textbf{CNNs} & \textbf{RNNs} \\
            \hline
            Handles Sequential Data & \textbf{No} & \textbf{No} & \textbf{Yes} \\
            Shares Parameters Across Time & \textbf{No} & \textbf{No} & \textbf{Yes} \\
            Captures Long-Term Dependencies & \textbf{No} & \textbf{No} & \textbf{Partially} (with LSTMs/GRUs) \\
            Suitable for Variable-Length Input & \textbf{No} & \textbf{Partially} (1D CNNs) & \textbf{Yes} \\
            \hline
        \end{tabular}
    }
    \caption{Comparison of RNNs with Fully Connected and Convolutional Networks.}
    \label{tab:rnn_vs_fc_cnn}
\end{table}

\subsection{Overview of Recurrent Neural Networks (RNNs) and Their Evolution}
\label{subsec:chapter16_rnn_overview}

\noindent
Many tasks in modern machine learning involve sequential or time-dependent data, where the observation at time $t$ depends on the history of inputs $x_1,\dots,x_{t-1}$. Classical feedforward networks (fully connected or convolutional) typically assume that inputs are independent and identically distributed (i.i.d.), so they struggle to model such temporal dependencies. \textbf{Recurrent Neural Networks (RNNs)} address this limitation by introducing a \emph{hidden state} that is passed from one timestep to the next, allowing the model to accumulate information over sequences of (in principle) arbitrary length.

\begin{enrichment}[How to read this overview][subsubsection]
	This subsection is intentionally a \textbf{high-level roadmap} of sequence modeling architectures, from basic recurrence to modern attention-based models. Our goal here is to explain \emph{why} each step in this evolution was introduced and how it addresses the limitations of the previous step. We only sketch the core ideas and equations; rigorous derivations (including Backpropagation Through Time, gating equations, and attention mechanisms), implementation details, and additional examples will follow in dedicated subsections later in this chapter and in the subsequent chapter on Transformers.
	\label{enr:chapter16_rnn_overview_roadmap}
\end{enrichment}

\newpage

\subsubsection{RNN progression: from vanilla units to gated architectures}

\paragraph{Vanilla RNNs: the basic recurrent idea}
The simplest recurrent architecture, often called an \emph{Elman RNN}, maintains a hidden state $\mathbf{h}_t$ that is updated at each timestep $t$ via

\begin{equation}
	\mathbf{h}_t
	=
	\tanh\Bigl(
	\mathbf{W}_{hh}\,\mathbf{h}_{t-1}
	+
	\mathbf{W}_{xh}\,\mathbf{x}_t
	+
	\mathbf{b}
	\Bigr),
	\label{eq:chapter16_vanilla_rnn_recurrence}
\end{equation}

where $\mathbf{x}_t$ is the input at time $t$, $\mathbf{h}_t$ is the hidden state, and the same parameters $\mathbf{W}_{hh}$, $\mathbf{W}_{xh}$, and $\mathbf{b}$ are reused for all timesteps. This weight sharing is what gives RNNs their ability to generalize across sequence length.

However, as we will see in detail when we derive \emph{Backpropagation Through Time (BPTT)}, repeatedly multiplying by $\mathbf{W}_{hh}$ causes gradients to either shrink to zero or explode in magnitude over long sequences. This is the classical \textbf{vanishing/exploding gradient problem} \cite{bengio1994_learning, pascanu2013_difficulty}. In practice:
\begin{itemize}
	\item Gradients often \emph{vanish}, making it hard for vanilla RNNs to learn dependencies beyond roughly 10--50 timesteps.
	\item Gradients can also \emph{explode} when $\|\mathbf{W}_{hh}\|$ is too large or activations allow unbounded growth, which is typically mitigated with gradient clipping.
\end{itemize}
Later in this chapter we will revisit Vanilla RNN and formally analyze why these issues arise and how techniques such as truncated BPTT partially alleviate them.

\paragraph{LSTMs: gating and additive memory for long-term dependencies}
To handle much longer temporal dependencies (hundreds of steps), \textbf{Long Short-Term Memory (LSTM)} networks \cite{hochreiter1997_lstm} modify the recurrence in two crucial ways:

\begin{enumerate}
	\item They maintain a separate \emph{cell state} that is updated \emph{additively}, creating a path where information and gradients can flow over many timesteps with minimal attenuation.
	\item They introduce \emph{gates} (input, forget, and output) that learn when to write new information to the cell state, when to erase old information, and when to expose the cell state to the hidden state.
\end{enumerate}

Intuitively, the LSTM turns the hidden dynamics into a differentiable memory system that can learn to “remember” and “forget” over long horizons. This largely solves the vanishing gradient problem for many practical sequence lengths and made LSTMs the dominant architecture for years in speech recognition, language modeling, and other temporal tasks. The trade-off is increased complexity: each LSTM cell contains several interacting affine transformations and gates, increasing parameter count and compute cost relative to vanilla RNNs.

\paragraph{GRUs: simplifying the LSTM while keeping most benefits}
\textbf{Gated Recurrent Units (GRUs)} \cite{cho2014_gru} were proposed as a streamlined alternative to LSTMs. GRUs merge the LSTM’s input and forget gates into a single \emph{update} gate and remove the explicit cell state, directly updating the hidden state instead. This yields:

\begin{itemize}
	\item Fewer parameters and simpler computation compared to LSTMs.
	\item Empirically similar performance to LSTMs on many language and sequence modeling benchmarks, especially for moderate sequence lengths.
\end{itemize}

\newpage

From an evolutionary perspective, GRUs are motivated by a design question: \emph{how much of the LSTM’s complexity is truly necessary to combat vanishing gradients?} GRUs show that a simpler gating mechanism can capture much of the benefit, which is attractive in resource-limited or latency-sensitive settings.

\paragraph{Bidirectional RNNs: using both past and future}
Vanilla RNNs, LSTMs, and GRUs as defined above are \emph{causal}: at time $t$, the model only has access to the past and current inputs $(x_1,\dots,x_t)$. For many applications, however, the entire sequence is available at once. \textbf{Bidirectional RNNs} address this by running one RNN forward in time and another backward, then combining their hidden states (e.g., by concatenation) at each timestep.

This evolution is motivated by \emph{disambiguation through context}: for the token “bank” in the sentence “He went to the bank to fish”, a backward RNN that sees “to fish” can help decide that “bank” refers to the side of a river rather than a financial institution. Bidirectional RNNs therefore excel in tasks like text classification, named entity recognition, and offline speech transcription, but they are not suitable for real-time streaming applications where future inputs are not yet observed.

\subsubsection{Motivation toward Transformers and attention-based models}

\paragraph{The sequential bottleneck and fixed-size state}
Despite the success of LSTMs, GRUs, and bidirectional variants, all RNN-based models share two structural limitations:

\begin{enumerate}
	\item \textbf{Sequential computation across time:} To compute $\mathbf{h}_t$, we must first compute $\mathbf{h}_{t-1}$. This dependency chain prevents parallelization across timesteps, making training and inference less efficient on modern accelerators for very long sequences.
	\item \textbf{Fixed-size hidden state:} The hidden state $\mathbf{h}_t$ is a vector of fixed dimension that must compress \emph{all} past information. For extremely long contexts (thousands of tokens), this global bottleneck can limit the model’s capacity to selectively remember detailed information.
\end{enumerate}

These limitations motivated architectures that could (i) process all positions in a sequence in parallel, and (ii) dynamically allocate capacity by letting each position \emph{attend} to the most relevant parts of the sequence.

\paragraph{Transformers: replacing recurrence with self-attention}
The \textbf{Transformer} architecture \cite{vaswani2017_attention} removes recurrence altogether and instead uses \emph{self-attention} layers: each token computes weighted combinations of all other tokens in the sequence. At a high level:

\begin{itemize}
	\item All timesteps can be processed in parallel within a layer, dramatically improving training efficiency on GPUs and TPUs.
	\item Long-range dependencies are handled naturally, since attention weights can connect arbitrarily distant positions without repeatedly multiplying by a transition matrix.
\end{itemize}

\newpage

However, this shift introduces new trade-offs:
\begin{itemize}
	\item The memory and compute cost of self-attention scales quadratically as $O(T^2)$ with sequence length $T$, which becomes challenging for very long inputs.
	\item For \emph{autoregressive} generation (e.g., language modeling), outputs are still typically produced token by token, and each new token requires computing attention over the growing context. This can be slow for extremely long outputs, although techniques such as \emph{speculative decoding} \cite{yao2022_improving} and \emph{non-autoregressive} models \cite{gu2018_nonautoregressive} aim to alleviate this by partially parallelizing generation or reducing the number of decoding steps.
\end{itemize}
Later, when we discuss attention mechanisms in depth, we will connect these design decisions back to the limitations of RNNs described above.

\subsubsection{Roadmap for the rest of the chapter}

\noindent
The remainder of this chapter builds on this evolutionary story and revisits each model family in more depth:

\begin{enumerate}
	\item \textbf{Vanilla RNNs and BPTT:} We begin by formalizing vanilla RNNs, deriving \emph{Backpropagation Through Time}, and precisely characterizing why and when vanishing and exploding gradients occur.
	\item \textbf{LSTMs and GRUs:} We then introduce LSTMs and GRUs from first principles, writing out their gating equations and explaining how additive memory paths and learned gates mitigate vanishing gradients, along with their remaining limitations (sequential computation, fixed-size state).
	\item \textbf{Beyond RNNs:} Finally, we use the insights from gated RNNs to motivate attention-based architectures and Transformers, which replace recurrent hidden states with self-attention, enabling highly parallel training and more flexible modeling of long-range dependencies. Detailed coverage of Transformer variants and attention mechanisms appears in the following chapter.
\end{enumerate}

By first presenting this high-level overview and then returning to each model class in detail, we aim to make the connections between architectures explicit: each new design (gating, bidirectionality, attention) can be understood as an attempt to systematically overcome the optimization and representation bottlenecks of its predecessors.

\newpage

\section{Recurrent Neural Networks (RNNs) - How They Work} 
\label{sec:chapter16_rnn_how_it_works}

\noindent Recurrent Neural Networks (RNNs) process sequential data by maintaining an \textbf{internal state} that evolves over time. Unlike feedforward neural networks that process inputs independently, RNNs retain memory through recurrent connections, enabling them to model dependencies across time steps. 

\noindent At each timestep \( t \), a new input \( x_t \) is provided to the RNN. The network updates its hidden state \( h_t \) based on both the current input and the previous hidden state \( h_{t-1} \), producing an output \( y_t \):
\[
h_t = f_W(h_{t-1}, x_t),
\]
where \( f_W \) is the recurrence function, typically a non-linear function such as \(\tanh\). A key property of RNNs is that the \textbf{same function and parameters} are used at every time step. The weights \( W \) are shared across all time steps, allowing the model to process sequences of arbitrary length.

\noindent Expanding this, a simple or "vanilla" RNN is formally defined as:
\[
h_t = \tanh(W_{hh}h_{t-1} + W_{xh}x_t + b),
\]
\[
y_t = W_{hy}h_t.
\]
This architecture, sometimes called a \textbf{vanilla RNN} or \textbf{Elman RNN} after Prof. Jeffrey Elman, efficiently processes sequences by applying the same weight matrices repeatedly. \textbf{Note: we'll often omit the bias from the notation for simplicity, but don't forget it when you implement RNNs.}

\subsection{RNN Computational Graph}
\label{sec:chapter16_rnn_computational_graph}

\noindent Since RNNs process sequences iteratively, we can represent their computation graph by unrolling the network over time. The computational graph depends on how inputs and outputs are structured, leading to different sequence processing scenarios.

\subsubsection{Many-to-Many}

\noindent In a \textbf{many-to-many} setup, an RNN processes a sequence of inputs and generates a sequence of outputs. Each hidden state depends on the previous state and the current input:
\[
h_t = f_W(h_{t-1}, x_t), \quad y_t = W_{hy}h_t.
\]

\noindent The initial hidden state \( h_0 \) is typically initialized as a zero vector or sampled from a normal distribution. However, in some architectures, \( h_0 \) is treated as a learnable parameter, allowing it to be optimized during training. This can be beneficial when early time steps contain little useful information.

\noindent The network processes each input sequentially:
\begin{itemize}
    \item \( x_1 \) is combined with \( h_0 \) using \( f_W \), producing \( h_1 \) and output \( y_1 \).
    \item \( h_1 \) is used with \( x_2 \) to compute \( h_2 \), which generates \( y_2 \).
    \item This process repeats until reaching the final time step \( T \).
\end{itemize}

\noindent Since the same weight matrix is reused at every time step, the computational graph is \textbf{unrolled} for as long as the sequence continues. During backpropagation, gradients must be summed across all timesteps (As we use the same node in multiple parts of the computation graph).

\noindent Training an RNN involves applying a loss function at each timestep:
\[
L = \sum_{t=1}^{T} L_t,
\]
where \( L_t \) is the loss at time \( t \). The summed loss is then used for backpropagation.

\begin{figure}[H]
    \centering
    \includegraphics[width=0.8\textwidth]{Figures/Chapter_16/slide_24.jpg}
    \caption{RNN Computational Graph for Many-to-Many Processing.}
    \label{fig:chapter16_rnn_many_to_many}
\end{figure}

\subsubsection{Many-to-One}

\noindent Some tasks require processing a sequence of inputs but generating only a single output at the final time step. This \textbf{many-to-one} setting is common in applications such as video classification, where the entire sequence is used to predict one label.

\noindent Instead of computing outputs at each timestep, the RNN produces a final output at step \( T \), based on the last hidden state \( h_T \):
\[
y = W_{hy}h_T.
\]
\noindent The loss function is then computed using only the final output \( y_T \), such as cross-entropy (CE) loss for classification.

\begin{figure}[H]
    \centering
    \includegraphics[width=0.8\textwidth]{Figures/Chapter_16/slide_25.jpg}
    \caption{RNN Computational Graph for Many-to-One Processing.}
    \label{fig:chapter16_rnn_many_to_one}
\end{figure}

\noindent This structure is particularly useful when the full context of the sequence is needed to make an informed decision, such as recognizing an action from a video or predicting sentiment from a passage of text.

\subsubsection{One-to-Many}

\noindent In contrast, \textbf{one-to-many} architectures take a single input and generate a sequence of outputs. This is commonly used in generative tasks such as image captioning, where the network produces a sequence of words based on an input image.

\noindent The RNN is initialized with an input \( x \) and generates outputs iteratively:
\[
h_1 = f_W(h_0, x), \quad y_1 = W_{hy}h_1.
\]
\noindent The output \( y_1 \) is then fed as input at the next timestep:
\[
h_2 = f_W(h_1, y_1), \quad y_2 = W_{hy}h_2.
\]

\noindent The sequence continues until a special \textbf{END token} is produced, signaling termination.

\begin{figure}[H]
    \centering
    \includegraphics[width=0.8\textwidth]{Figures/Chapter_16/slide_26.jpg}
    \caption{RNN Computational Graph for One-to-Many Processing.}
    \label{fig:chapter16_rnn_one_to_many}
\end{figure}

\noindent The network must learn to balance sequential coherence while ensuring that the generated sequence remains contextually relevant.

\subsection{Seq2Seq: Sequence-to-Sequence Learning}
\label{subsec:chapter16_seq2seq}

\noindent
Many real-world problems involve mapping an input sequence to an output sequence where the lengths can differ arbitrarily ($T \neq M$). A canonical example is \textbf{machine translation}, where an input sentence in one language (e.g.\ English) is converted into a sentence in another language (e.g.\ French); the two sentences may have different lengths and word orders.

\newpage

\noindent
To handle this setting, \textbf{Sequence-to-Sequence (Seq2Seq)} models \cite{sutskever2014_seq2seq} use a composite architecture consisting of two Recurrent Neural Networks with separate parameters:
\begin{itemize}
	\item \textbf{Encoder (many-to-one).} Uses weights $W_1$ to read the input sequence and compress it into a single vector.
	\item \textbf{Decoder (one-to-many).} Uses weights $W_2$ to expand this single vector into an output sequence.
\end{itemize}
In Justin Johnson’s slides (see the below figure), this is summarized as
\[
\text{Seq2Seq} = (\text{many-to-one}) + (\text{one-to-many}).
\]

\subsubsection{The encoder–decoder architecture}

\noindent
Let the input sequence be $\mathbf{x} = (x_1,\dots,x_T)$ and the output sequence be $\mathbf{y} = (y_1,\dots,y_M)$.

\begin{enumerate}
	\item \textbf{Encoder (many-to-one, weights $W_1$).}
	The encoder RNN processes the input sequence step by step:
	\begin{equation}
		h^{\text{enc}}_t = f_W\!\bigl(h^{\text{enc}}_{t-1}, x_t; W_1\bigr),
		\qquad t = 1,\dots,T,
		\label{eq:chapter16_seq2seq_encoder}
	\end{equation}
	where $f_W$ denotes the recurrent update (RNN, LSTM, GRU, etc.) and $h^{\text{enc}}_0$ is typically initialized to the zero vector.
	The final encoder state
	\begin{equation}
		h^{\text{enc}}_T
	\end{equation}
	serves as a fixed-size \emph{context vector} summarizing the entire input sequence. In the figure, these encoder states are drawn as $h_0, h_1, \dots, h_T$.
	
	\item \textbf{Information transfer (many-to-one $\to$ one-to-many).}
	The decoder is initialized from the encoder’s final state:
	\begin{equation}
		h^{\text{dec}}_0 = h^{\text{enc}}_T,
		\label{eq:chapter16_seq2seq_handover}
	\end{equation}
	which corresponds to the arrow from $h_T$ into the first decoder cell in the below figure. This vector $h^{\text{enc}}_T$ is the single “input” to the decoder side.
	
	\item \textbf{Decoder (one-to-many, weights $W_2$).}
	Starting from $h^{\text{dec}}_0$ and a special \texttt{<START>} token $y_0$, the decoder generates the output sequence autoregressively using its own parameters $W_2$:
	\begin{equation}
		h^{\text{dec}}_t = f_W\!\bigl(h^{\text{dec}}_{t-1}, y_{t-1}; W_2\bigr),
		\qquad t = 1,\dots,M,
		\label{eq:chapter16_seq2seq_decoder_state}
	\end{equation}
	\begin{equation}
		p(y_t \mid y_{<t}, \mathbf{x}) = \mathrm{softmax}\bigl(W_{\text{out}} h^{\text{dec}}_t\bigr),
		\label{eq:chapter16_seq2seq_decoder_output}
	\end{equation}
	where $W_{\text{out}}$ maps hidden states to vocabulary logits.
	During training, we typically use \emph{teacher forcing}, feeding the ground-truth token $y_{t-1}$ into \eqref{eq:chapter16_seq2seq_decoder_state}; at inference time, $y_{t-1}$ is the token predicted at the previous step (e.g.\ $\arg\max$ of \eqref{eq:chapter16_seq2seq_decoder_output}).
	In the figure, the decoder hidden states $h^{\text{dec}}_1, h^{\text{dec}}_2,\dots$ are drawn as $h_1, h_2,\dots$, each producing outputs $y_1, y_2,\dots$.
\end{enumerate}

\begin{figure}[H]
	\centering
	\includegraphics[width=0.9\textwidth]{Figures/Chapter_16/slide_28.jpg}
	\caption{Computational graph of a Sequence-to-Sequence (Seq2Seq) model. The encoder (left, weights $W_1$) applies the same recurrent update $f_W$ to compress the input sequence $x_1,\dots,x_T$ into a single vector $h_T^{\text{enc}}$. This vector initializes the decoder (right, weights $W_2$), which repeatedly applies $f_W$ to produce hidden states $h^{\text{dec}}_t$ and outputs $y_t$ one token at a time.}
	\label{fig:chapter16_seq2seq_computational_graph}
\end{figure}

\noindent
Decoding continues until the model emits a special \texttt{<END>} token, indicating that the output sequence is complete. In this way, a single encoded vector $h^{\text{enc}}_T$ is “unrolled” into an output of arbitrary length $M$.

\subsubsection{Significance and the information bottleneck}

\noindent
Seq2Seq models extend RNNs from fixed-size input/output settings to a general framework for transforming one sequence into another, enabling applications such as:
\begin{itemize}
	\item \textbf{Machine translation:} Converting text between languages (e.g.\ English $\to$ French).
	\item \textbf{Speech recognition:} Mapping acoustic feature sequences to text.
	\item \textbf{Text summarization:} Compressing long documents into shorter summaries.
	\item \textbf{Conversational AI:} Generating responses in dialog systems.
\end{itemize}

\noindent
At the same time, the basic encoder–decoder design introduces a fundamental \textbf{information bottleneck}:
\begin{itemize}
	\item All information about the input sequence $\mathbf{x}$ must be packed into the single vector $h^{\text{enc}}_T$ in \eqref{eq:chapter16_seq2seq_handover}.
	\item For long inputs, early tokens $(x_1, x_2, \dots)$ may have only a weak influence on $h^{\text{enc}}_T$ due to vanishing gradients and limited capacity, leading to degraded translation or generation quality.
\end{itemize}

\noindent
This limitation motivates several extensions that we will develop later in the chapter and in subsequent chapters:
\begin{itemize}
	\item \textbf{Gated recurrent units (LSTMs, GRUs)} improve how information and gradients propagate through time, making the context vector $h^{\text{enc}}_T$ more robust for longer sequences.
	\item \textbf{Attention mechanisms} allow the decoder to look back at all encoder states $(h^{\text{enc}}_1,\dots,h^{\text{enc}}_T)$ instead of relying solely on $h^{\text{enc}}_T$, thereby softening the bottleneck.
\end{itemize}

\newpage

In the next parts, we will connect this generic Seq2Seq template to concrete tasks such as \textbf{language modeling}, derive \textbf{Backpropagation Through Time (BPTT)} for training these models, and then revisit the roles of LSTMs, GRUs, and attention in improving sequence-to-sequence learning.

\section{Example Usage of Seq2Seq: Language Modeling}
\label{sec:chapter16_seq2seq_language_model}

\noindent
A concrete example of how recurrent networks operate in practice is a \textbf{character-level language model}. The goal is to process a stream of input characters and, at each timestep, predict the \emph{next} character in the sequence. By learning the conditional distribution
\[
p(x_t \mid x_1,\dots,x_{t-1}),
\]
the model captures the statistical structure of the training text and can later be used to generate new text.

\subsection{Formulating the problem}

\noindent
Consider the toy training sequence ``hello'' with vocabulary
\[
\mathcal{V} = \{\text{h}, \text{e}, \text{l}, \text{o}\}.
\]
We view this as a supervised learning problem where, at each timestep, the input is the current character and the target is the next character:
\begin{center}
	\begin{tabular}{ccl}
		t & Input & Target \\
		\hline
		1 & ``h'' & ``e'' \\
		2 & ``e'' & ``l'' \\
		3 & ``l'' & ``l'' \\
		4 & ``l'' & ``o''
	\end{tabular}
\end{center}

\noindent
Each character is represented as a \textbf{one-hot vector} in $\mathbb{R}^{|\mathcal{V}|}$, for example:
\[
\mathbf{x}_{\text{h}} = [1\;0\;0\;0]^\top,\quad
\mathbf{x}_{\text{e}} = [0\;1\;0\;0]^\top,\quad
\mathbf{x}_{\text{l}} = [0\;0\;1\;0]^\top,\quad
\mathbf{x}_{\text{o}} = [0\;0\;0\;1]^\top.
\]

\subsection{Forward pass through time}

\noindent
A simple RNN with hidden state dimension $H$ maintains a hidden vector $\mathbf{h}_t \in \mathbb{R}^H$ and updates it according to
\[
\mathbf{h}_t = \tanh\!\bigl(\mathbf{W}_{hh}\mathbf{h}_{t-1} + \mathbf{W}_{xh}\mathbf{x}_t + \mathbf{b}_h\bigr),
\]
where $\mathbf{h}_0$ is an initial state (often the zero vector), $\mathbf{W}_{xh}$ maps inputs to the hidden layer, $\mathbf{W}_{hh}$ maps the previous hidden state to the new one, and $\mathbf{b}_h$ is a bias term shared across time.

\noindent
From the hidden state, the network produces unnormalized scores (logits) over the next character:
\[
\mathbf{y}_t = \mathbf{W}_{hy}\mathbf{h}_t + \mathbf{b}_y,
\]
with $\mathbf{W}_{hy}$ and $\mathbf{b}_y$ shared at all timesteps. Applying a softmax gives a probability distribution over the vocabulary:
\[
\mathbf{p}_t = \mathrm{softmax}(\mathbf{y}_t), \qquad
(\mathbf{p}_t)_k = \frac{\exp((\mathbf{y}_t)_k)}{\sum_{j}\exp((\mathbf{y}_t)_j)}.
\]

\noindent
For example, for the first character ``h'', we feed $\mathbf{x}_1 = \mathbf{x}_{\text{h}}$ into the RNN to obtain a hidden state and logits:
\[
\mathbf{h}_1 = [0.3,\,-0.1,\,0.9]^\top,\qquad
\mathbf{y}_1 = [1.0,\,2.2,\,-3.0,\,4.1]^\top,
\]
so that $\mathbf{p}_1 = \mathrm{softmax}(\mathbf{y}_1)$ assigns high probability to the correct next character ``e''. The same computation is repeated for ``e'', ``l'', and ``l'', with the hidden state carrying information about the context seen so far (``h'', ``he'', ``hel'', ``hell'').

\begin{figure}[H]
	\centering
	\includegraphics[width=0.8\textwidth]{Figures/Chapter_16/slide_35.jpg}
	\caption{Character-level RNN language model on the sequence ``hello''. At each timestep, the current character is represented as a one-hot vector at the input layer, transformed into a hidden representation, and mapped to scores over the vocabulary at the output layer. The hidden state is reused across timesteps, allowing the model to condition on the full prefix.}
	\label{fig:chapter16_rnn_language_model_step}
\end{figure}

\noindent
Figure~\ref{fig:chapter16_rnn_language_model_step} illustrates this process: given characters up to time $t-1$ (for example, ``he''), the model predicts character $t$ (``l''); then the new hidden state is forwarded to the next timestep.

\subsection{Training: losses and gradient flow through time}

\noindent
To train the model, we compare its predictions with the ground-truth next characters and update the shared weights
\[
\Theta = \{\mathbf{W}_{xh}, \mathbf{W}_{hh}, \mathbf{W}_{hy}, \mathbf{b}_h, \mathbf{b}_y\}.
\]

\subsubsection*{Per-timestep loss}

\noindent
At each timestep $t$, the target next character $x_{t+1}$ is represented as a one-hot vector $\mathbf{t}_{t+1} \in \mathbb{R}^{|\mathcal{V}|}$. We compute the \textbf{cross-entropy loss} between $\mathbf{p}_t$ and $\mathbf{t}_{t+1}$:
\[
L_t = - \sum_{k=1}^{|\mathcal{V}|} (\mathbf{t}_{t+1})_k \log (\mathbf{p}_t)_k
= - \log (\mathbf{p}_t)_{k^\star},
\]
where $k^\star$ is the index of the true next character at time $t+1$. For the sequence ``hello'' we obtain losses $L_1,\dots,L_4$ corresponding to the four training pairs listed above.

\subsubsection*{Sequence loss and gradient}

\noindent
The total loss for the sequence is the sum of per-timestep losses:
\[
\mathcal{L} = \sum_{t=1}^{T} L_t.
\]
Because the same parameters $\Theta$ are reused at every timestep, the gradient of the sequence loss with respect to any parameter (for example, $\mathbf{W}_{hh}$) is the sum of its contributions from each timestep:
\[
\frac{\partial \mathcal{L}}{\partial \mathbf{W}_{hh}} = \sum_{t=1}^{T} \frac{\partial L_t}{\partial \mathbf{W}_{hh}}.
\]

\noindent
Each term $\partial L_t / \partial \mathbf{W}_{hh}$ is itself a chain of derivatives that passes backward through time. For instance, the loss $L_4$ (predicting ``o'' given the prefix ``hell'') depends on $\mathbf{h}_4$, which depends on $\mathbf{h}_3$, which depends on $\mathbf{h}_2$, and so on back to $\mathbf{h}_0$. Computing the gradient therefore requires propagating error signals through the entire sequence of hidden states:
\[
\mathbf{h}_4 \rightarrow \mathbf{h}_3 \rightarrow \mathbf{h}_2 \rightarrow \mathbf{h}_1 \rightarrow \mathbf{h}_0.
\]

\noindent
Once the gradient $\nabla_{\Theta} \mathcal{L}$ has been computed, we update the parameters using gradient descent or a variant such as Adam:
\[
\Theta \leftarrow \Theta - \eta \,\nabla_{\Theta} \mathcal{L},
\]
where $\eta$ is the learning rate and the negative gradient gives the direction of steepest decrease of the loss.

\noindent
The procedure for computing these gradients by explicitly following the chain of dependencies backward in time is called \textbf{Backpropagation Through Time (BPTT)}. In the next section, we will make this precise by unrolling the RNN across timesteps and deriving the gradient expressions. This will naturally expose numerical issues such as \emph{vanishing} and \emph{exploding} gradients when sequences become long.

\subsection{Inference: generating text}

\noindent
After training, we can use the model to generate new text, one character at a time:
\begin{enumerate}
	\item Initialize the hidden state $\mathbf{h}_0$ (e.g.\ zeros) and feed an initial character or a special \texttt{<START>} symbol $\mathbf{x}_1$.
	\item Compute $\mathbf{h}_1$, logits $\mathbf{y}_1$, and probabilities $\mathbf{p}_1 = \mathrm{softmax}(\mathbf{y}_1)$.
	\item Sample or choose the most likely next character from $\mathbf{p}_1$ (e.g.\ by $\arg\max$), obtaining a character $x_2$.
	\item Feed the one-hot encoding of $x_2$ back in as the next input $\mathbf{x}_2$ and repeat.
\end{enumerate}
This \emph{autoregressive} loop continues until the model produces a special \texttt{<END>} token or a maximum length is reached.

\begin{figure}[H]
	\centering
	\includegraphics[width=0.8\textwidth]{Figures/Chapter_16/slide_40.jpg}
	\caption{Test-time generation in a character-level RNN language model. At each step, the sampled output character is fed back as the next input, allowing the network to generate sequences such as ``hello'' one character at a time.}
	\label{fig:chapter16_rnn_text_generation}
\end{figure}

\subsection{From one-hot vectors to embeddings}

\noindent
So far we have used one-hot vectors as inputs. Multiplying a weight matrix $\mathbf{W}_{xh}$ by a one-hot vector simply selects one column of $\mathbf{W}_{xh}$, which can be interpreted as a learned \emph{embedding} of that character. Modern implementations therefore introduce an explicit \textbf{embedding layer} that maps character indices to dense vectors:
\[
\mathbf{e}_t = \mathrm{Embedding}(x_t), \qquad
\mathbf{h}_t = \tanh\!\bigl(\mathbf{W}_{hh}\mathbf{h}_{t-1} + \mathbf{W}_{eh}\mathbf{e}_t + \mathbf{b}_h\bigr),
\]
where $\mathbf{W}_{eh}$ plays the role of $\mathbf{W}_{xh}$ but acts on lower-dimensional embeddings.

\noindent
This has several advantages:
\begin{itemize}
	\item \textbf{Efficiency.} We avoid explicitly storing and multiplying large sparse one-hot vectors; indexing into an embedding table is cheaper.
	\item \textbf{Learned similarity structure.} Characters (or words) with similar usage patterns can acquire similar embedding vectors, helping the model generalize.
	\item \textbf{Flexible dimensionality.} The embedding dimension can be chosen independently of the vocabulary size, controlling the capacity and computational cost.
\end{itemize}

\begin{figure}[H]
	\centering
	\includegraphics[width=0.8\textwidth]{Figures/Chapter_16/slide_41.jpg}
	\caption{Replacing one-hot inputs with an embedding layer. Each input character index is mapped to a dense vector, which is then fed into the recurrent layer. This is equivalent to selecting a column of the input weight matrix but is more efficient and expressive.}
	\label{fig:chapter16_rnn_embedding_layer}
\end{figure}

\subsection{Summary and motivation for BPTT}

\noindent
In this example, we have seen how an RNN processes a character sequence like ``hello'', predicts the next character at each timestep, aggregates per-timestep cross-entropy losses into a sequence loss, and uses gradients of this loss to update a set of shared parameters. The key difficulty is that the loss at later timesteps depends on a long chain of hidden states and repeated applications of the same weight matrices. Computing and propagating gradients through this temporal chain is precisely the job of \textbf{Backpropagation Through Time}. In the next section we will unroll the RNN formally, derive these gradients, and use that derivation to understand why naïve RNNs suffer from vanishing and exploding gradients on long sequences.

\section{Backpropagation Through Time (BPTT)}
\label{sec:chapter16_bptt}

\noindent
In a Recurrent Neural Network (RNN), the hidden state at time \(t\) depends on the hidden state at time \(t-1\), so unrolling the network over a sequence of length \(T\) yields a deep computational graph with \(T\) repeated applications of the same parameters.
Training therefore requires computing gradients not only ``through layers'' (as in feedforward networks) but also \emph{through time}.
This procedure is known as \textbf{Backpropagation Through Time (BPTT)}.

\subsection{Full BPTT as Backprop on an Unrolled RNN}
\label{subsec:chapter16_bptt_math}

\noindent
Consider a simple (vanilla) RNN processing a sequence of length \(T\) with inputs
\(\mathbf{x}_1, \dots, \mathbf{x}_T\).
The forward dynamics are
\begin{align}
	\mathbf{h}_t &= \phi\!\bigl(\mathbf{W}_{hh}\mathbf{h}_{t-1} + \mathbf{W}_{xh}\mathbf{x}_t + \mathbf{b}_h\bigr),
	\qquad t = 1,\dots,T, \label{eq:chapter16_rnn_forward_h}\\
	\mathbf{y}_t &= \mathbf{W}_{hy}\mathbf{h}_t + \mathbf{b}_y, \label{eq:chapter16_rnn_forward_y}
\end{align}

\newpage

where \(\mathbf{h}_0\) is an initial hidden state (often the zero vector or a learned parameter),
\(\phi\) is a pointwise activation function (typically \(\tanh\) in classical RNNs), and
\(\mathbf{W}_{xh}, \mathbf{W}_{hh}, \mathbf{W}_{hy}, \mathbf{b}_h, \mathbf{b}_y\) are shared across all timesteps.

\noindent
Let \(\mathcal{L}_t = \ell(\mathbf{y}_t, \mathbf{y}_t^{\text{target}})\) denote the loss at timestep \(t\), and define the total sequence loss
\[
\mathcal{L} = \sum_{t=1}^{T} \mathcal{L}_t.
\]
Because parameters are shared in time, the gradient of the total loss with respect to any parameter \(\theta \in \{\mathbf{W}_{xh}, \mathbf{W}_{hh}, \mathbf{W}_{hy}, \mathbf{b}_h, \mathbf{b}_y\}\) decomposes as
\[
\frac{\partial \mathcal{L}}{\partial \theta}
=
\sum_{t=1}^{T} \frac{\partial \mathcal{L}_t}{\partial \theta}.
\]

\noindent
The key difficulty is that \(\mathcal{L}_t\) depends on \(\theta\) not only through the ``local'' timestep \(t\), but also through the entire history of hidden states
\(\mathbf{h}_1,\dots,\mathbf{h}_t\).
For example, for the recurrent weight matrix \(\mathbf{W}_{hh}\) we can write
\begin{equation}
	\frac{\partial \mathcal{L}_t}{\partial \mathbf{W}_{hh}}
	=
	\sum_{k=1}^{t}
	\underbrace{\frac{\partial \mathcal{L}_t}{\partial \mathbf{h}_t}}_{\text{error at time }t}
	\;\underbrace{\frac{\partial \mathbf{h}_t}{\partial \mathbf{h}_k}}_{\text{temporal Jacobian }k \rightarrow t}
	\;\underbrace{\frac{\partial^+ \mathbf{h}_k}{\partial \mathbf{W}_{hh}}}_{\text{local derivative at time }k},
	\label{eq:chapter16_bptt_chain}
\end{equation}
where \(\partial^+ \mathbf{h}_k / \partial \mathbf{W}_{hh}\) treats \(\mathbf{h}_{k-1}\) as constant.

\noindent
The temporal Jacobian \(\partial \mathbf{h}_t / \partial \mathbf{h}_k\) itself is a product of one-step Jacobians:
\begin{equation}
	\frac{\partial \mathbf{h}_t}{\partial \mathbf{h}_k}
	=
	\prod_{j=k+1}^{t}
	\frac{\partial \mathbf{h}_j}{\partial \mathbf{h}_{j-1}},
	\qquad
	\frac{\partial \mathbf{h}_j}{\partial \mathbf{h}_{j-1}}
	=
	\mathrm{diag}\!\bigl(\phi'(\mathbf{z}_j)\bigr)\,\mathbf{W}_{hh},
	\label{eq:chapter16_bptt_jacobian_product}
\end{equation}
where
\(\mathbf{z}_j = \mathbf{W}_{hh}\mathbf{h}_{j-1} + \mathbf{W}_{xh}\mathbf{x}_j + \mathbf{b}_h\)
are the pre-activations.
Thus, BPTT is standard backpropagation applied to the unrolled computational graph, but its gradients involve \emph{products} of many Jacobian matrices across time.

\subsubsection{Vanishing and Exploding Gradients Revisited}
\label{subsubsec:chapter16_bptt_vanishing_exploding}

\noindent
Equation~\eqref{eq:chapter16_bptt_jacobian_product} is the mathematical origin of the two classic pathologies in RNN training~\cite{bengio1994_learning,pascanu2013_difficulty}.
If we denote the one-step Jacobian at time \(j\) by
\[
\mathbf{J}_j
=
\frac{\partial \mathbf{h}_j}{\partial \mathbf{h}_{j-1}}
=
\mathrm{diag}\!\bigl(\phi'(\mathbf{z}_j)\bigr)\,\mathbf{W}_{hh},
\]
then
\(\partial \mathbf{h}_t / \partial \mathbf{h}_k = \mathbf{J}_t \mathbf{J}_{t-1} \cdots \mathbf{J}_{k+1}\).
On average, the behavior of this product is controlled by typical singular values of \(\mathbf{J}_j\):

\begin{itemize}
	\item \textbf{Vanishing gradients.}
	If the largest singular value of a ``typical'' Jacobian \(\mathbf{J}_j\) is less than \(1\) on average, then
	\(\bigl\|\partial \mathbf{h}_t / \partial \mathbf{h}_k\bigr\|\)
	decays approximately like \(\gamma^{t-k}\) for some effective contraction factor \(0 < \gamma < 1\).
	Gradients associated with distant timesteps become numerically negligible, making it extremely hard to learn long-range dependencies~\cite{bengio1994_learning}.
	
	\item \textbf{Exploding gradients.}
	If the largest singular value is greater than \(1\) on average, then
	\(\bigl\|\partial \mathbf{h}_t / \partial \mathbf{h}_k\bigr\|\)
	grows approximately like \(\gamma^{t-k}\) with \(\gamma > 1\).
	Small errors at late timesteps produce enormous gradients for early timesteps, leading to numerical overflow and unstable optimization~\cite{pascanu2013_difficulty}.
\end{itemize}

\noindent
These issues arise even if we ignore the nonlinearity and approximate the dynamics as
\(\mathbf{h}_t \approx \mathbf{W}_{hh}\mathbf{h}_{t-1}\).
In that case,
\(\mathbf{h}_t \approx \mathbf{W}_{hh}^t \mathbf{h}_0\),
and both the forward states and the backpropagated gradients are governed by powers of the same matrix \(\mathbf{W}_{hh}\).
Unless the spectral properties of \(\mathbf{W}_{hh}\) are carefully controlled, either vanishing or exploding behavior is unavoidable.

\subsubsection{Memory Cost of Full BPTT}
\label{subsubsec:chapter16_bptt_memory}

\noindent
To compute the exact gradients in \eqref{eq:chapter16_bptt_chain}, the forward pass must store \emph{all} hidden states \(\mathbf{h}_1,\dots,\mathbf{h}_T\) and pre-activations \(\mathbf{z}_1,\dots,\mathbf{z}_T\), since the Jacobians depend on these values.
The activation memory cost therefore scales as
\[
\mathcal{O}(T \cdot d_h),
\]
where \(d_h\) is the hidden dimension.
For long sequences (for example, \(T = 1{,}000\) and \(d_h = 1{,}024\)), storing all activations across many layers and mini-batches can easily require gigabytes of memory, even before accounting for optimizer state and other model parameters.
Furthermore, each parameter update requires a full forward and backward pass over the entire sequence, which is computationally expensive.

\noindent
These considerations motivate an approximation that trades exact long-range gradients for tractable memory and compute: \textbf{truncated BPTT}.

\subsection{Truncated Backpropagation Through Time}
\label{subsec:chapter16_truncated_bptt}

\noindent
In many applications (language modeling, online speech recognition, reinforcement learning), sequences are effectively unbounded: there is no natural ``end of sequence'' at which we could run full BPTT.
Moreover, as we saw in Section~\ref{subsubsec:chapter16_bptt_vanishing_exploding}, the Jacobian products in full BPTT already suffer from vanishing and exploding gradients even for moderate sequence lengths~\cite{bengio1994_learning,pascanu2013_difficulty}.
\textbf{Truncated BPTT} (often denoted TBPTT-\(\tau\)) addresses both the computational and memory costs by limiting the temporal horizon over which gradients are propagated, at the price of introducing additional bias in credit assignment~\cite{williams1990_tbptt}.

\subsubsection{Chunked Training with a Finite Horizon}
\label{subsubsec:chapter16_truncated_algorithm}

\noindent
Fix a truncation length (or horizon) \(\tau \ll T\), typically in the range \(\tau \approx 50\)–\(200\).
We process the sequence in chunks of length \(\tau\) and backpropagate only within each chunk.
Concretely, suppose we process a long sequence in segments
\([1,\tau], [\tau+1, 2\tau], \dots\).
For the \(s\)-th chunk we define
\[
b_s = (s-1)\tau, \qquad
\text{chunk } s: \; t = b_s+1, \dots, b_s + \tau.
\]
The algorithm proceeds as follows~\cite{williams1990_tbptt,pascanu2013_difficulty}:

\begin{enumerate}
	\item \textbf{Initialize the hidden state.}
	For the first chunk, set \(\mathbf{h}_{0}\) to zeros or a learned initial state.
	For chunk \(s>1\), set the initial state to the final hidden state of the previous chunk:
	\(\mathbf{h}_{b_s} = \mathbf{h}_{b_{s-1}+\tau}\).
	
	\item \textbf{Forward pass over the chunk.}
	For \(t = b_s+1,\dots,b_s+\tau\), compute
	\(\mathbf{h}_t\) and \(\mathbf{y}_t\) using \eqref{eq:chapter16_rnn_forward_h}--\eqref{eq:chapter16_rnn_forward_y}, and accumulate the chunk loss
	\[
	\mathcal{L}^{(s)} = \sum_{t=b_s+1}^{b_s+\tau} \mathcal{L}_t.
	\]
	
	\newpage
	
	\item \textbf{Backward pass (truncated in time).}
	Backpropagate gradients from \(\mathcal{L}^{(s)}\) \emph{only} through the timesteps \(b_s+1,\dots,b_s+\tau\).
	In practice, we treat \(\mathbf{h}_{b_s}\) as a constant with respect to the parameters (for example, by calling \texttt{detach} in PyTorch), so no gradient flows into the computations that produced \(\mathbf{h}_{b_s}\).
	
	\item \textbf{Parameter update.}
	Use the gradients from this chunk to update the parameters
	\(\theta\).
	Then move to the next chunk.
\end{enumerate}

\noindent
From an optimization point of view, the overall objective remains the sum (or average) of per-timestep losses:
\[
\mathcal{L}
=
\sum_{s=1}^{S} \mathcal{L}^{(s)}
=
\sum_{s=1}^{S}\;\sum_{t=b_s+1}^{b_s+\tau} \mathcal{L}_t,
\]
where \(S\) is the number of chunks.
Some implementations divide by \(S\) (or by \(T\)) to work with an average loss, but the gradient structure is unchanged: each update only uses gradients originating from the most recent \(\tau\) timesteps.

\subsubsection{Interaction with Vanishing and Exploding Gradients}
\label{subsubsec:chapter16_truncated_gradients}

\noindent
Truncated BPTT changes \emph{how} vanishing and exploding gradients appear, but it does not remove the underlying pathologies analyzed in Section~\ref{subsubsec:chapter16_bptt_vanishing_exploding}.
It shortens the dangerous Jacobian products (helping with explosion on very long sequences) while adding a hard cutoff that exacerbates vanishing for long-range dependencies~\cite{williams1990_tbptt,pascanu2013_difficulty}.

\paragraph{Exploding gradients: partial mitigation via shorter chains}
In full BPTT, the gradient from time \(T\) back to time \(1\) involves a product of \(T-1\) Jacobians,
\(\prod_{j=1}^{T-1} \mathbf{J}_j\),
whose norm typically behaves like \(\|\mathbf{J}\|^{T-1}\) for some average Jacobian norm \(\|\mathbf{J}\|\)~\cite{bengio1994_learning,pascanu2013_difficulty}.
If the dominant singular value of the recurrent Jacobian is slightly larger than \(1\), say \(\|\mathbf{J}\| \approx 1.1\), the gradient can grow as \(1.1^{T}\), leading to catastrophic explosion on long sequences.

\noindent
Truncated BPTT caps the length of this product at the truncation horizon \(\tau\)~\cite{williams1990_tbptt}:
no gradient ever involves more than \(\tau\) Jacobian factors.
In the toy example above, the worst-case growth is now \(1.1^{\tau}\) instead of \(1.1^{T}\).
For \(T = 1000\) and \(\tau = 50\), this replaces a factor of roughly \(2.5 \times 10^{41}\) by about \(117\), which is much easier to manage with gradient clipping~\cite{pascanu2013_difficulty}.

\noindent
However, if the per-step Jacobians are highly unstable (for example, \(\|\mathbf{W}_{hh}\| \gg 1\)), gradients can still explode \emph{within} the \(\tau\)-step window.
Empirically and theoretically, truncated BPTT therefore \emph{reduces} the risk of catastrophic explosion on very long sequences, but does not guarantee stability; gradient clipping remains necessary in practice~\cite{pascanu2013_difficulty}.

\paragraph{Vanishing gradients: soft decay plus hard truncation}
For vanishing gradients, truncated BPTT actually makes the situation worse for long-range dependencies by combining two effects: the \emph{soft} exponential decay inherent in vanilla RNNs and an additional \emph{hard} algorithmic cutoff at the truncation boundary.

\noindent
Under full BPTT, the gradient of a loss at time \(T\) with respect to an earlier hidden state \(\mathbf{h}_k\) can be written as
\[
\frac{\partial \mathcal{L}_T}{\partial \mathbf{h}_k}
=
\frac{\partial \mathcal{L}_T}{\partial \mathbf{h}_T}
\prod_{j=k+1}^{T}
\mathbf{J}_j,
\]
and the norm of this product typically decays roughly like \(\gamma^{T-k}\) for some effective contraction factor \(\gamma < 1\) determined by the recurrent Jacobian~\cite{bengio1994_learning,pascanu2013_difficulty}.

\noindent
With truncated BPTT and horizon \(\tau\), this chain rule is applied differently depending on the distance \(T-k\):
\begin{itemize}
	\item \textbf{Within the horizon (\(T-k \le \tau\)).}
	All timesteps from \(k\) to \(T\) lie in the same chunk, so we still form a product of one-step Jacobians
	\(\prod_{j=k+1}^{T} \mathbf{J}_j\).
	Because each factor typically has singular values \(< 1\) on average, the gradient decays approximately like \(\gamma^{T-k}\) just as in full BPTT~\cite{bengio1994_learning}.
	In other words, truncation does \emph{not} improve vanishing locally: signals from \(\tau\) steps ago are already extremely small before truncation is even applied.
	
	\item \textbf{Beyond the horizon (\(T-k > \tau\)).}
	In this case, the computational graph crosses at least one chunk boundary.
	At each boundary we explicitly treat the incoming hidden state as a constant (for example, via \texttt{detach} in PyTorch), which enforces
	\[
	\frac{\partial \mathbf{h}_{b_s}}{\partial \mathbf{h}_{b_s-1}} = \mathbf{0}
	\]
	for the ``virtual'' edge that would connect the previous chunk to the current one.
	This inserts a zero matrix into the Jacobian product, so the entire gradient
	\(\frac{\partial \mathcal{L}_T}{\partial \mathbf{h}_k}\) collapses to exactly zero as soon as the path from \(k\) to \(T\) crosses a truncation boundary~\cite{williams1990_tbptt,pascanu2013_difficulty}.
\end{itemize}

\noindent
In full BPTT, a distant timestep \(k\) might still exert a tiny but nonzero influence on \(\mathcal{L}_T\), on the order of \(\gamma^{T-k}\).
Under truncated BPTT, any timestep more than \(\tau\) steps away exerts \emph{no} influence at all: the gradient path is cut off by construction.
The effective credit-assignment horizon is therefore limited to
\[
\text{effective horizon}
\;\approx\;
\min\bigl(\tau,\;\text{intrinsic vanishing horizon from the recurrent dynamics}\bigr),
\]
so truncation \emph{preserves} vanishing within each window while adding a hard ceiling on learnable temporal dependencies beyond \(\tau\)~\cite{bengio1994_learning,pascanu2013_difficulty}.

\subsubsection{Benefits and Limitations of Truncation}
\label{subsubsec:chapter16_truncated_tradeoffs}

\noindent
\textbf{Advantages.}
Truncated BPTT is primarily a \emph{computational} tool~\cite{williams1990_tbptt,pascanu2013_difficulty}:

\begin{itemize}
	\item \textbf{Reduced memory usage.}
	At any point we only need to store activations for \(\tau\) timesteps, so the activation memory scales as \(\mathcal{O}(\tau \cdot d_h)\) instead of \(\mathcal{O}(T \cdot d_h)\).
	This is essential when \(T\) is very large or effectively unbounded (for example, in streaming text or reinforcement learning).
	
	\item \textbf{Improved throughput.}
	Forward and backward passes over shorter chunks are faster, allowing more frequent parameter updates and better hardware utilization.
	This is one of the main motivations for TBPTT in practice~\cite{pascanu2013_difficulty}.
	
	\item \textbf{Support for arbitrarily long streams.}
	Because memory and computation per update depend on \(\tau\) rather than on the total stream length, truncated BPTT allows RNNs to be trained on sequences that span millions of timesteps or never terminate.
\end{itemize}

\noindent
\textbf{Fundamental limitations.}
These computational gains come at a conceptual cost that compounds the vanishing/exploding gradient issues~\cite{bengio1994_learning,pascanu2013_difficulty}:

\begin{itemize}
	\item \textbf{Truncation bias and hard horizon.}
	Dependencies longer than \(\tau\) timesteps receive \emph{no} gradient signal.
	The model can \emph{see} long-range context in the hidden state, but it cannot \emph{learn} to encode or preserve that context better, because no gradient flows back to the parameters responsible for it.
	This makes long-term dependencies systematically harder (or impossible) to learn, beyond the intrinsic vanishing-gradient effects.
	
	\item \textbf{Uncorrected hidden state.}
	The initial hidden state of each chunk, \(\mathbf{h}_{b_s}\), is a function of earlier inputs and parameters, but gradients are not allowed to adjust those earlier computations.
	If those states encode information poorly, no later loss can correct them.
	Over many chunks, hidden states may drift into saturated regimes (where \(\tanh'\) is near zero) or noisy regimes, further weakening gradient flow even \emph{within} a window~\cite{bengio1994_learning}.
	
	\item \textbf{Non-stationary optimization landscape.}
	Because the gradient ignores all contributions beyond \(\tau\) steps, the effective loss surface seen by the optimizer depends on the choice of \(\tau\) and on how chunk boundaries align with the data.
	This makes training more sensitive to learning-rate schedules, initialization, and truncation strategy~\cite{pascanu2013_difficulty}.
\end{itemize}

\noindent
In practice, truncation horizons \(\tau \approx 50\)–\(100\) are common compromises.
They make training on long or streaming sequences feasible and less prone to catastrophic explosion, but even with TBPTT and gradient clipping, vanilla RNNs remain fundamentally limited on tasks that require precise credit assignment over hundreds or thousands of timesteps~\cite{bengio1994_learning}.
This limitation is a key motivation for gated architectures such as LSTMs and GRUs, which modify the recurrence itself rather than relying solely on truncation.


\subsection{Why BPTT and TBPTT Struggle on Long Sequences}
\label{subsec:chapter16_bptt_failures}

\noindent
Putting these pieces together, we can now summarize why both full BPTT and truncated BPTT remain fundamentally limited for long sequences, even when we use \(\tanh\) activations, careful initialization, and gradient clipping.
The limitations come from the combination of gradient dynamics, truncation, and the architecture itself.

\begin{itemize}
	\item \textbf{Fixed-capacity hidden state.}
	At each timestep, all relevant information from the past must be compressed into a fixed-dimensional vector \(\mathbf{h}_t \in \mathbb{R}^{d_h}\).
	As the sequence length grows, the amount of information to retain grows, but the capacity of \(\mathbf{h}_t\) does not.
	Inevitably, older information is overwritten or blurred. 
	
	\item \textbf{Exponential decay or growth of influence.}
	In a linearized view, the influence of an input at time \(k\) on the hidden state at time \(t\) is governed by \(\mathbf{W}_{hh}^{t-k}\).
	As discussed in Section~\ref{subsubsec:chapter16_bptt_vanishing_exploding}, unless the spectral radius of \(\mathbf{W}_{hh}\) is exactly \(1\) under all conditions (which is unrealistic to maintain during training), contributions from the distant past either vanish or explode.
	This structural property affects both full BPTT and each truncated window in TBPTT.
	
	\item \textbf{Truncation-induced loss of long-range credit assignment.}
	Truncated BPTT adds an explicit horizon: gradients are forcibly cut off after \(\tau\) steps.
	Even if the hidden state still carries useful information from hundreds of steps ago, the model never receives a learning signal telling it \emph{how} to encode that information.
	Thus, TBPTT cannot, by construction, learn dependencies longer than its truncation horizon.
	
	\item \textbf{Sequential computation and limited parallelism.}
	Unlike CNNs or Transformers, which can process all positions in parallel, RNNs must process timesteps sequentially:
	\(\mathbf{h}_t\) depends on \(\mathbf{h}_{t-1}\).
	This makes efficient training on very long sequences difficult on modern hardware, even if memory and gradient stability were not an issue.
\end{itemize}

\noindent
These limitations are not fully solvable by better regularization, optimizers, or clever truncation strategies alone.
They stem from the architectural decision to store all memory in a single evolving state vector updated by repeated application of the same transformation.
The natural next step is to make this recurrence \emph{adaptive}, allowing the network to decide how much of the past to keep, how much to forget, and which information to expose at each timestep.

\newpage

This is precisely the role of \textbf{gated} architectures such as Long Short-Term Memory (LSTM) networks and Gated Recurrent Units (GRUs), which we will introduce after understanding the activation-function trade-offs in vanilla RNNs.

\section{Why RNNs Use \textit{tanh} Instead of ReLU}
\label{sec:chapter16_why_tanh_rnns}

\noindent
Modern feedforward architectures (ConvNets, Transformers) overwhelmingly favor ReLU-family activations (ReLU, Leaky ReLU, GELU, etc.).
Classical vanilla RNNs, by contrast, almost always use \(\tanh\) (or occasionally sigmoid) in their recurrent layers.
This is not a historical accident: it follows from the fact that RNNs repeatedly apply the \emph{same} recurrent matrix over time, and from the gradient behavior analyzed in Section~\ref{subsec:chapter16_bptt_math}.

\noindent
At a high level, vanilla RNNs face a harsh trade-off:

\begin{itemize}
	\item With \textbf{ReLU-like, unbounded activations}, forward activations and gradients are extremely prone to \emph{catastrophic explosion}.
	\item With \textbf{tanh}, forward activations are \emph{provably bounded}, and gradients are strongly damped, which greatly reduces explosion but exacerbates \emph{vanishing}.
\end{itemize}

\noindent
Exploding gradients are typically a fatal failure mode (NaNs, divergence), whereas vanishing gradients are a difficult but manageable limitation (the model still trains on short horizons).
This makes \(\tanh\) the ``lesser of two evils'' in vanilla RNNs.
The real fix for long-range credit assignment will come from \emph{gated architectures} (LSTMs and GRUs), not from swapping \(\tanh\) for ReLU.

\subsection{Recurrent Dynamics and Gradient Flow}
\label{subsec:chapter16_single_layer_rnn}

\noindent
Recall the vanilla RNN update from Section~\ref{sec:chapter16_bptt}:
\[
\mathbf{h}_t
=
\phi\!\bigl(
\mathbf{W}_{hh}\mathbf{h}_{t-1}
+
\mathbf{W}_{xh}\mathbf{x}_t
+
\mathbf{b}_h
\bigr),
\]
where \(\phi\) is the activation function.
To isolate the recurrent dynamics, ignore inputs and biases:
\[
\mathbf{h}_t
\approx
\phi\!\bigl(\mathbf{W}_{hh}\mathbf{h}_{t-1}\bigr),
\qquad
\mathbf{h}_T
\approx
\phi^{(T)}\!\bigl(\mathbf{W}_{hh}^T \mathbf{h}_0\bigr).
\]
Thus the long-term behavior is governed by powers of the same matrix \(\mathbf{W}_{hh}\).

\paragraph{Spectral radius and forward stability}

\noindent
Let \(\lambda_1,\dots,\lambda_{d_h}\) be the eigenvalues of \(\mathbf{W}_{hh}\), and define the spectral radius
\[
\rho(\mathbf{W}_{hh})
=
\max_i |\lambda_i|.
\]
Intuitively, \(\rho(\mathbf{W}_{hh})\) measures how repeated application of \(\mathbf{W}_{hh}\) tends to expand or contract vectors.

\begin{itemize}
	\item If \(\rho(\mathbf{W}_{hh}) > 1\), some directions in state space are amplified exponentially as \(t\) increases.
	Without a bounding nonlinearity, the hidden state norm \(\|\mathbf{h}_t\|\) can grow without bound.
	\item If \(\rho(\mathbf{W}_{hh}) < 1\), all directions contract exponentially.
	Old information in \(\mathbf{h}_t\) is gradually ``forgotten'' as it is repeatedly multiplied by a contractive operator.
\end{itemize}

\noindent
Initialization schemes (for example, orthogonal or scaled identity matrices) attempt to control \(\rho(\mathbf{W}_{hh})\), but forward stability alone is not enough: we also care about how gradients propagate through time.

\subsubsection{Gradient Flow Through Time}
\label{subsubsec:chapter16_grad_flow}

\noindent
Section~\ref{subsec:chapter16_bptt_math} showed that gradients through time are controlled by products of one-step Jacobians.
For an earlier hidden state \(\mathbf{h}_t\), the gradient of the loss \(\mathcal{L}\) can be written as
\[
\frac{\partial \mathcal{L}}{\partial \mathbf{h}_t}
=
\frac{\partial \mathcal{L}}{\partial \mathbf{h}_T}
\prod_{j=t}^{T-1}
\frac{\partial \mathbf{h}_{j+1}}{\partial \mathbf{h}_j},
\]
with one-step Jacobians
\[
\frac{\partial \mathbf{h}_{j+1}}{\partial \mathbf{h}_j}
=
\mathrm{diag}\!\Bigl(
\phi'\!\bigl(\mathbf{W}_{hh}\mathbf{h}_j + \mathbf{W}_{xh}\mathbf{x}_{j+1} + \mathbf{b}_h\bigr)
\Bigr)\,
\mathbf{W}_{hh}
\;\equiv\;
\mathbf{J}_j.
\]

\noindent
As in Equation~\eqref{eq:chapter16_bptt_jacobian_product}, the gradient norm is governed by the product
\[
\prod_{j=t}^{T-1} \mathbf{J}_j.
\]
Two ingredients matter:

\begin{itemize}
	\item The spectral norm \(\|\mathbf{W}_{hh}\|_2\), which determines how much \(\mathbf{W}_{hh}\) itself expands vectors;
	\item The typical magnitude of the activation derivative \(\phi'(\cdot)\), which appears on the diagonal of each \(\mathbf{J}_j\).
\end{itemize}

\noindent
Roughly, each factor \(\mathbf{J}_j\) scales gradients by something like \(\|\phi'\|_\infty \cdot \|\mathbf{W}_{hh}\|_2\).
If this effective factor is consistently larger than \(1\), gradients explode across many timesteps; if it is consistently smaller than \(1\), they vanish~\cite{bengio1994_learning,pascanu2013_difficulty}.
There is no scalar activation that keeps this product exactly at \(1\) across hundreds of steps in a plain vanilla RNN; we must choose which failure mode is more tolerable.

\subsection{Why Plain ReLU Is Problematic in RNNs}
\label{subsubsec:chapter16_grad_flow_relu}

\noindent
For ReLU,
\[
\phi_{\text{ReLU}}(z) = \max(0,z),
\qquad
\phi_{\text{ReLU}}'(z)
=
\begin{cases}
	1, & z > 0,\\
	0, & z \le 0,
\end{cases}
\]
so active units have derivative exactly \(1\).
When many units are active, the Jacobian is approximately
\[
\frac{\partial \mathbf{h}_{j+1}}{\partial \mathbf{h}_j}
\approx
\mathbf{W}_{hh},
\quad\text{and}\quad
\frac{\partial \mathbf{h}_T}{\partial \mathbf{h}_t}
\approx
\mathbf{W}_{hh}^{T-t}.
\]

\noindent
Two extreme regimes dominate in practice:

\begin{itemize}
	\item \textbf{Exploding regime.}
	If we initialize \(\mathbf{W}_{hh}\) so that \(\|\mathbf{W}_{hh}\|_2 \gtrsim 1\) (to avoid immediate vanishing), the gradient norm behaves roughly like \(\|\mathbf{W}_{hh}\|_2^{T-t}\).
	Even a mild expansion factor such as \(1.1\) leads to \(1.1^{100} \approx 1.4 \times 10^4\); for longer sequences, gradients quickly grow beyond floating-point range, producing numerical overflow and NaNs.
	Forward activations can explode as well, since ReLU is unbounded on the positive side.
	
	\item \textbf{Dead-neuron / strongly contractive regime.}
	If we initialize \(\mathbf{W}_{hh}\) very small so that \(\|\mathbf{W}_{hh}\|_2 < 1\), many pre-activations become negative, ReLU outputs \(0\), and \(\phi'(z)=0\).
	Those units stop contributing and stop receiving gradient, effectively shrinking the dimensionality of the hidden state and encouraging rapid vanishing.
\end{itemize}

\noindent
Deep feedforward networks mitigate such issues via residual connections, normalization layers, and limited depth.
In a vanilla RNN, however, the \emph{same} transformation is applied hundreds or thousands of times, so small deviations of \(\|\mathbf{W}_{hh}\|_2\) from \(1\) are amplified much more severely.
Empirically, vanilla RNNs with ReLU-like activations are extremely fragile and typically require very aggressive gradient clipping and carefully tuned initialization just to avoid immediate divergence~\cite{pascanu2013_difficulty}.

\subsection{Why \textit{tanh} Is Safer in Vanilla RNNs}
\label{subsec:chapter16_tanh_stability}

\noindent
The \(\tanh\) activation,
\[
\tanh(x) = \frac{e^x - e^{-x}}{e^x + e^{-x}},
\]
changes this picture in three important ways.
It does \emph{not} remove vanishing gradients, but it dramatically reduces the risk of catastrophic explosion and keeps the forward dynamics numerically well behaved.

\paragraph{1. Bounded outputs: forward stability}

\noindent
\(\tanh(x)\in(-1,1)\) for all \(x\).
Regardless of how large \(\mathbf{W}_{hh}\mathbf{h}_{t-1} + \mathbf{W}_{xh}\mathbf{x}_t + \mathbf{b}_h\) becomes, each hidden unit is clamped into a fixed interval.
As a result:

\begin{itemize}
	\item Hidden states cannot diverge to arbitrarily large magnitudes in the forward pass.
	\item Inputs to subsequent layers and to the loss remain within a predictable numeric range.
\end{itemize}

\noindent
This boundedness removes the forward counterpart of the exploding-gradient problem: even if \(\rho(\mathbf{W}_{hh}) > 1\), activations themselves remain in \((-1,1)\), which makes the overall network much more robust.

\paragraph{2. Derivative bounded by 1: automatic damping of explosions}

\noindent
The derivative of \(\tanh\) is
\[
\frac{d}{dx}\tanh(x) = 1 - \tanh^2(x),
\]
so \(|\tanh'(x)| \le 1\) for all \(x\), with equality only at \(x=0\).
In the Jacobians
\(\mathbf{J}_j = \mathrm{diag}\bigl(\tanh'(\cdot)\bigr)\mathbf{W}_{hh}\),
this derivative acts as a multiplicative damping factor on the singular values of \(\mathbf{W}_{hh}\).
Compared with ReLU (whose derivative is exactly \(1\) in the active region), \(\tanh\) has two stabilizing effects:

\begin{itemize}
	\item When hidden units are in the linear regime (\(x \approx 0\)), \(|\tanh'(x)| \approx 1\), so short-range gradient flow is similar to ReLU.
	\item When hidden units grow in magnitude, \(|\tanh'(x)|\) shrinks toward \(0\), so the Jacobian factors \(\mathbf{J}_j\) become strongly contractive.
	This \emph{automatically} dampens any tendency of \(\mathbf{W}_{hh}\) to amplify gradients.
\end{itemize}

\noindent
In other words, \(\tanh\) implements a state-dependent gain control:
as activations grow, local derivatives shrink, pushing the effective per-step gain \(\|\mathbf{J}_j\|_2\) back toward or below \(1\).
Large gradient explosions become much less common than with ReLU~\cite{pascanu2013_difficulty}.

\paragraph{3. Zero-centered activations}

\noindent
The range of \(\tanh\) is symmetric around zero.
Hidden states can be positive or negative, so contributions in \(\mathbf{W}_{hh}\mathbf{h}_{t-1}\) can cancel each other.
In contrast, ReLU produces nonnegative activations, and with mostly positive weights this can create a ``positive feedback loop'' in which hidden states grow in the same direction step after step.
Zero-centered activations therefore provide an additional bias toward stable dynamics and often lead to smoother optimization.

\paragraph{Caveat: vanishing gradients on long sequences}

\noindent
The same mechanisms that prevent explosion also promote vanishing:

\begin{itemize}
	\item When \(|x|\) is moderate to large, \(\tanh(x)\) saturates near \(\pm 1\), and \(\tanh'(x) \approx 0\).
	\item Over long sequences, many units spend much of their time in these saturated regimes, so each Jacobian factor \(\mathbf{J}_j\) is strongly contractive, and products of \(\mathbf{J}_j\) quickly drive gradients toward zero.
\end{itemize}

\noindent
Thus, \(\tanh\)-RNNs are typically effective on short or medium-length sequences (tens of timesteps), but they struggle when precise credit assignment is required over hundreds or thousands of steps.
\(\tanh\) does not fix the vanishing-gradient problem; it trades catastrophic explosion for controlled vanishing~\cite{bengio1994_learning,pascanu2013_difficulty}.

\subsection{ReLU Variants and Gradient Clipping}
\label{subsec:chapter16_relu_variants_rnn_issues}

\noindent
Two popular ReLU variants are sometimes suggested for RNNs: \textbf{ReLU6} (bounded above) and \textbf{Leaky ReLU} (nonzero slope for negative inputs).
They address specific ReLU pathologies, but do not fundamentally resolve recurrent stability.

\paragraph{ReLU6: bounded but hard saturation}

\noindent
ReLU6 clamps activations to \([0,6]\):
\[
\phi(x) = \min\bigl(\max(0,x), 6\bigr).
\]
The upper bound prevents unbounded growth of hidden states in the forward pass, but once a unit saturates at \(6\) its derivative becomes zero for all larger inputs.
In an RNN, many units can quickly hit this ceiling and then effectively ``die'': they contribute a constant value and receive no gradient, shrinking the effective hidden dimensionality and again encouraging vanishing gradients.

\paragraph{Leaky ReLU: softer but still unbounded}

\noindent
Leaky ReLU is defined as
\[
\phi(x) =
\begin{cases}
	x, & x > 0,\\[3pt]
	\alpha x, & x \le 0,
\end{cases}
\qquad 0 < \alpha \ll 1.
\]
This avoids the ``dying ReLU'' problem by giving a nonzero gradient for negative inputs and can modestly reduce vanishing.
However, the activation remains unbounded on the positive side, and the derivative for positive inputs is still \(1\).
If \(\|\mathbf{W}_{hh}\|_2 > 1\) and hidden states remain mostly positive, both forward activations and gradients can still explode over time.
Leaky ReLU is therefore only a partial fix and does not remove the need for careful control of \(\mathbf{W}_{hh}\) and heavy gradient clipping in vanilla RNNs.

\subsubsection{Why Gradient Clipping Alone Is Insufficient}
\label{subsubsec:chapter16_gradient_clipping_limitations}

\noindent
Gradient clipping~\cite{pascanu2013_difficulty} is a standard heuristic to curb exploding gradients.
Given a gradient vector \(\mathbf{g} = \nabla\mathcal{L}\) and threshold \(c > 0\), global norm clipping replaces
\[
\mathbf{g}
\;\leftarrow\;
\frac{\mathbf{g}}{\max\bigl(1, \|\mathbf{g}\| / c\bigr)},
\]
so that update steps never exceed length \(c\).
This limits catastrophic jumps in parameter space, but it does \emph{not} repair the recurrent dynamics that create exploding and vanishing gradients in the first place~\cite{pascanu2013_difficulty}.

\paragraph{Clipping cannot fix unstable recurrence}

\noindent
First, clipping acts only in the \emph{backward} pass.
With unbounded activations and expansive recurrence \(\rho(\mathbf{W}_{hh}) > 1\), hidden states grow roughly as
\[
\|\mathbf{h}_t\|
\approx
\|\mathbf{W}_{hh}^t \mathbf{h}_0\|,
\]
and can reach extremely large magnitudes.
Once activations or losses overflow to \(\pm\infty\) or \texttt{NaN}, gradients are undefined and clipping cannot intervene.
Thus, ``ReLU + clipping'' cannot guarantee stability of vanilla RNNs because the primary failure mode (exploding \emph{states}) remains.

\noindent
Second, even when the forward pass stays finite, the exploding-gradient problem arises from products of Jacobians as in Equation~\eqref{eq:chapter16_bptt_jacobian_product}:
\[
\prod_{j=t}^{T-1}
\mathbf{J}_j
=
\prod_{j=t}^{T-1}
\mathrm{diag}\bigl(\phi'(\cdot)\bigr)\,\mathbf{W}_{hh}.
\]
If these products are strongly expansive, clipping intervenes only \emph{after} they have amplified the gradient, truncating large vectors to have norm \(c\).
When this happens frequently (as in a ReLU-RNN whose gradients would otherwise explode every few steps), optimization is heavily distorted:

\begin{itemize}
	\item Large gradients are projected onto the sphere of radius \(c\), so the method behaves more like a noisy sign-based optimizer than a faithful first-order method.
	\item The effective learning rate becomes entangled with how often clipping triggers.
	\item The underlying Jacobian products remain expansive; clipping only limits their impact on parameter updates, not their origin.
\end{itemize}

\noindent
Heavy reliance on clipping with unbounded activations therefore treats the \emph{symptom} (huge updates) rather than the \emph{cause} (unstable recurrence).

\paragraph{Why we still clip with \textit{tanh}}

\noindent
With a bounded activation such as \(\tanh\), forward activations lie in \((-1,1)\), so catastrophic state explosion is much less likely.
Nevertheless, clipping remains useful even for \(\tanh\)-RNNs~\cite{pascanu2013_difficulty}:

\begin{itemize}
	\item \textbf{Transient spikes.}
	When many units operate near the linear regime (\(x \approx 0\)), the derivatives satisfy \(|\tanh'(x)| \approx 1\), so the Jacobians \(\mathbf{J}_j\) can have singular values \(\gtrsim 1\).
	Over tens of steps, this can produce occasional gradient spikes before the network settles into a more contractive regime.
	\item \textbf{Stacked architectures and large outputs.}
	In multi-layer RNNs or models with large output layers, large gradients can originate from higher layers or the loss, even if the recurrent block itself is relatively stable.
	Clipping prevents these spikes from destabilizing the recurrent parameters.
	\item \textbf{Low-cost insurance.}
	Once bounded activations have removed the worst forward-pass explosions, clipping only rarely activates.
	It becomes a cheap safety net against rare outliers, instead of a mechanism that fires on most updates.
\end{itemize}

\noindent
In practice, then, \textbf{ReLU + aggressive clipping} tries to use clipping as the primary stabilizer, which is fragile and distorting, whereas \textbf{tanh + light clipping} uses clipping as a secondary safeguard on top of already stable forward dynamics.

\newpage

\subsection{Summary and Motivation for Gated RNNs}
\label{subsec:chapter16_tanh_summary}

\noindent
Because a vanilla RNN repeatedly applies the same recurrent transformation, the Jacobian products in Equation~\eqref{eq:chapter16_bptt_jacobian_product} will, for any smooth activation \(\phi\), tend to either contract or expand exponentially over long horizons~\cite{bengio1994_learning,pascanu2013_difficulty}.
No scalar nonlinearity can simultaneously avoid vanishing and exploding gradients in this setting; different activations simply choose different points on the stability--memory trade-off.

\noindent
The following table summarizes the main properties of common activations in vanilla RNNs.

\begin{table}[H]
	\centering
	\small
	\begin{tabular}{@{}lccc p{0.38\textwidth}@{}}
		\toprule
		\textbf{Activation} & \textbf{Bounded?} & \textbf{Max derivative} & \textbf{Zero-centered?} & \textbf{Typical failure mode} \\
		\midrule
		ReLU       & No         & \(1\)           & No   & Exploding hidden states and gradients \\
		Leaky ReLU & No (above) & \(1\) (\(x>0\)) & No   & Explodes if states stay mostly positive \\
		ReLU6      & Yes        & \(1\) then \(0\) & No  & Units saturate near 6 and stop learning \\
		\(\tanh\)  & Yes        & \(\le 1\)       & Yes  & Vanishing gradients on long sequences \\
		Sigmoid    & Yes        & \(\le 0.25\)    & No   & Strong vanishing, even on short sequences \\
		\bottomrule
	\end{tabular}
	\caption{Trade-offs of activation functions in vanilla RNNs. Bounded, zero-centered \(\tanh\) avoids catastrophic explosion at the cost of stronger vanishing; ReLU-family activations are prone to numerical instability without careful control of \(\mathbf{W}_{hh}\) and heavy gradient clipping.}
	\label{tab:chapter16_activation_tradeoffs}
\end{table}

\noindent
From this viewpoint, the historical choice of \(\tanh\) in vanilla RNNs is straightforward:

\begin{itemize}
	\item \textbf{ReLU family.}
	Unbounded outputs and unit derivative in the active region help mitigate vanishing in deep feedforward networks, but in vanilla RNNs they make both hidden states and gradients highly susceptible to exponential growth.
	Even with clipping, forward-pass explosions and unstable optimization are common.
	
	\item \textbf{tanh.}
	Bounded, zero-centered outputs and derivatives \(|\tanh'(x)| \le 1\) strongly damp both activations and gradients.
	This largely eliminates catastrophic explosion and yields numerically stable training, at the price of pronounced vanishing on long sequences: precise credit assignment beyond tens of timesteps becomes very hard.
\end{itemize}

\noindent
Exploding gradients are a \emph{catastrophic} failure mode (training diverges, NaNs appear), whereas vanishing gradients are a \emph{limiting} failure mode (the model still trains, but only captures short- to medium-range dependencies).
Consequently, vanilla RNNs almost universally adopt \(\tanh\) (typically with light gradient clipping) rather than ReLU + heavy clipping: it is better to have a model that learns reliably on short horizons than one that is numerically unstable on most problems.

\noindent
At the same time, this analysis makes clear that activation choice alone cannot solve long-range credit assignment.
As long as gradients must traverse repeated Jacobian products of the form \(\mathrm{diag}\bigl(\phi'(\cdot)\bigr)\mathbf{W}_{hh}\), they will eventually vanish or explode over sufficiently many timesteps~\cite{bengio1994_learning,pascanu2013_difficulty}.
The next step is therefore to change the \emph{architecture}, not just the nonlinearity.

\noindent
Gated RNNs such as LSTMs and GRUs address this by introducing:

\begin{itemize}
	\item \textbf{Additive memory paths}, along which information (and gradients) can flow with gain close to \(1\) across many timesteps, rather than being repeatedly multiplied by \(\mathbf{W}_{hh}\).
	\item \textbf{Multiplicative gates}, which learn when to write, keep, or erase information, allowing the network to maintain long-term dependencies without sacrificing stability.
\end{itemize}

\noindent
These architectures retain \(\tanh\) (and sigmoid) as stable building blocks, but wrap them in a recurrent structure designed to keep gradient norms under control over much longer horizons.
In the next parts we will see how this gating mechanism overcomes the limitations of vanilla \(\tanh\)-RNNs while preserving their numerical robustness.

\newpage

\section{Example Usages of Recurrent Neural Networks}
\label{sec:chapter16_rnn_examples}

\noindent Recurrent Neural Networks (RNNs) are widely used in various sequential tasks, particularly in text processing and generation. With modern deep learning frameworks such as PyTorch, implementing an RNN requires only a few lines of code, allowing researchers and practitioners to train language models on large text corpora efficiently. In this section, we explore notable applications of RNNs, starting with text-based tasks, including text generation and analyzing what representations RNNs learn from data.

\subsection{RNNs for Text-Based Tasks}
\label{sec:chapter16_rnn_text_tasks}

\noindent One of the most intriguing applications of RNNs is \textbf{text generation}. By training an RNN on a large corpus of text, the model learns to predict the next character or word based on previous context. Once trained, it can generate text in a similar style to its training data, capturing syntactic and stylistic structures.

\subsubsection{Generating Text with RNNs}
\label{sec:chapter16_rnn_text_generation}

\noindent A simple character-level RNN can be trained on various text corpora, such as Shakespeare's works, LaTeX source files, or C programming code. Despite its simplicity, an RNN can learn meaningful statistical patterns, including character frequencies, word structures, and even basic grammatical rules.

\noindent Some examples of text generation with RNNs:
\begin{itemize}
    \item \textbf{Shakespeare-style text:} After training on Shakespeare's works, an RNN can generate text that mimics old-English writing, maintaining proper character names and poetic structure.
    \item \textbf{LaTeX code generation:} An RNN trained on LaTeX documents can generate LaTeX-like syntax, although the output may not always be valid compilable code.
    \item \textbf{C code generation:} By training on a dataset of C programming files, the RNN can generate snippets of C-like syntax, capturing programming constructs such as loops and conditionals.
\end{itemize}

\noindent These examples demonstrate that RNNs can capture both \textbf{structural} and \textbf{stylistic} aspects of language, learning dependencies that extend across sequences. However, understanding what representations the RNN has learned from the data remains an open research question.

\subsection{Understanding What RNNs Learn}
\label{sec:chapter16_rnn_representations}

\noindent Since an RNN produces hidden states at each timestep, it implicitly learns internal representations of the input data. A key research question is: \textbf{What kinds of representations do RNNs learn from the data they are trained on?}

\noindent A study by Karpathy, Johnson, and Fei-Fei \cite{karpathy2015_visualizing_rnns} explored this question by visualizing hidden states of an RNN trained on the Linux kernel source code. Since each hidden state is a vector passed through a \(\tanh\) activation function, every dimension in the hidden state has values in the range \([-1,1]\). The authors examined how different hidden state dimensions responded to specific characters in the sequence.

\paragraph{Visualization of Hidden State Activations}
\noindent To interpret what RNN hidden units are learning, the authors colored text based on the activation value of a single hidden state dimension at each timestep:
\begin{itemize}
    \item \textbf{Red:} Activation close to \(+1\).
    \item \textbf{Blue:} Activation close to \(-1\).
\end{itemize}

\noindent This visualization method allowed them to analyze whether certain hidden state dimensions captured meaningful patterns in the data.

\begin{figure}[H]
    \centering
    \includegraphics[width=0.75\textwidth]{Figures/Chapter_16/slide_62.jpg}
    \caption{Some hidden units do not exhibit clearly explainable patterns, making interpretation difficult.}
    \label{fig:chapter16_uninterpretable_cells}
\end{figure}

\noindent As shown in Figure~\ref{fig:chapter16_uninterpretable_cells}, many hidden unit activations appeared random and did not provide an intuitive understanding of what the RNN was tracking. However, in some cases, individual hidden state dimensions exhibited clear, meaningful behavior.

\subsubsection{Interpretable Hidden Units}
\label{sec:chapter16_rnn_interpretable_cells}

\noindent While many hidden state dimensions appear uninterpretable, some exhibit structured activation patterns corresponding to meaningful aspects of the data. Below are a few examples:

\paragraph{Quote Detection Cell}

\begin{figure}[H]
    \centering
    \includegraphics[width=0.75\textwidth]{Figures/Chapter_16/slide_63.jpg}
    \caption{An RNN hidden unit detecting quoted text. The activations shift significantly at the beginning and end of quotes.}
    \label{fig:chapter16_quote_cell}
\end{figure}

\noindent Some hidden units activate strongly in the presence of quoted text, as seen in Figure~\ref{fig:chapter16_quote_cell}.

\paragraph{Line Length Tracking Cell}
\noindent Another hidden unit tracks the number of characters in a line, transitioning smoothly from blue to red as it approaches 80 characters per line (a common convention in code formatting), as shown in Figure~\ref{fig:chapter16_line_length}.

\begin{figure}[H]
    \centering
    \includegraphics[width=0.75\textwidth]{Figures/Chapter_16/slide_64.jpg}
    \caption{An RNN hidden unit tracking line length, moving from blue (short lines) to red (long lines).}
    \label{fig:chapter16_line_length}
\end{figure}

\noindent This demonstrates that some RNN neurons track specific long-range dependencies, encoding useful properties of the dataset.

\paragraph{Other Interpretable Hidden Units}
\noindent Other meaningful hidden state activations include:
\begin{itemize}
    \item \textbf{Comment Detector:} Some units activate strongly in commented-out sections of code.
    \item \textbf{Code Depth Tracker:} Certain units track the depth of nested code structures (e.g., counting how many open brackets exist in C code).
    \item \textbf{Keyword Highlighter:} Some neurons respond selectively to keywords such as \texttt{if}, \texttt{for}, or \texttt{return} in programming languages.
\end{itemize}

\subsubsection{Key Takeaways from Interpretable Units}
\noindent The analysis by Karpathy et al. highlights several important insights:
\begin{itemize}
    \item \textbf{RNNs can learn abstract properties of sequences.} Some hidden units respond to high-level features, such as quoted text, line length, or code structure.
    \item \textbf{Not all hidden units are interpretable.} Many dimensions in the hidden state vector appear to activate randomly, making it difficult to extract clear meaning from every neuron.
    \item \textbf{Neurons behave differently based on the dataset.} The same RNN architecture trained on different corpora may develop completely different internal representations.
\end{itemize}

\noindent While these findings provide insight into what RNNs learn, interpreting hidden states remains an open challenge in deep learning research. This motivates further study into techniques such as attention mechanisms and gated architectures, which offer more structured ways to track long-term dependencies.

\subsection{Image Captioning}
\label{sec:chapter16_image_captioning}

\noindent Image captioning is the task of generating a textual description of an image by combining \textbf{computer vision} (to extract meaningful features) and \textbf{natural language processing} (to generate coherent text). The standard pipeline consists of two main components:

\begin{enumerate}
    \item \textbf{Feature Extraction with a Pre-Trained CNN:}  
    A convolutional neural network (CNN), originally trained for image classification (e.g., on ImageNet), is used to encode the image into a high-level feature representation. The final fully connected layers are removed, leaving only/mostly the convolutional layers to produce an image embedding.
    
    \item \textbf{Caption Generation with an RNN:}  
    The extracted image features serve as additional input to an RNN, which generates a description one word at a time, starting from a special \texttt{<START>} token and stopping at an \texttt{<END>} token.
\end{enumerate}

\noindent The standard RNN hidden state update equation:
\[
\mathbf{h}_t = \tanh\big( \mathbf{W}_{xh} \mathbf{x}_t + \mathbf{W}_{hh} \mathbf{h}_{t-1} \big),
\]
is modified to incorporate the image features:
\[
\mathbf{h}_t = \tanh\big( \mathbf{W}_{xh} \mathbf{x}_t + \mathbf{W}_{hh} \mathbf{h}_{t-1} + \mathbf{W}_{ih} \mathbf{v} \big),
\]
where:
\begin{itemize}
    \item \( \mathbf{x}_t \) is the current input word,
    \item \( \mathbf{h}_{t-1} \) is the previous hidden state,
    \item \( \mathbf{v} \) is the image embedding from the CNN,
    \item \( \mathbf{W}_{ih} \) learns how to integrate image features into the sequence model.
\end{itemize}

\begin{figure}[H]
    \centering
    \includegraphics[width=0.8\textwidth]{Figures/Chapter_16/slide_77.jpg}
    \caption{An RNN-based image captioning model stops generating text after producing an \texttt{<END>} token.}
    \label{fig:chapter16_image_captioning_pipeline}
\end{figure}

\newpage
\subsection{Image Captioning Results}
\label{sec:chapter16_image_captioning_results}

\noindent When trained effectively, RNN-based image captioning models generate descriptions that align well with the content of an image.

\begin{figure}[H]
    \centering
    \includegraphics[width=0.7\textwidth]{Figures/Chapter_16/slide_78.jpg}
    \caption{Success cases of RNN-based image captioning: (Left) "A cat sitting on a suitcase on the floor." (Right) "Two giraffes standing in a grassy field."}
    \label{fig:chapter16_captioning_success}
\end{figure}

\noindent Some strengths of the model include:
\begin{itemize}
    \item Identifying objects and their relationships (e.g., "a cat sitting on a suitcase").
    \item Capturing spatial context within the scene.
    \item Producing fluent, grammatically correct sentences.
\end{itemize}

\noindent However, the model is limited in its reasoning abilities, often making systematic errors.

\subsection{Failure Cases in Image Captioning}
\label{sec:chapter16_image_captioning_failures}

\noindent Despite generating plausible captions, RNN-based models struggle with dataset biases and lack true scene understanding.

\begin{figure}[H]
    \centering
    \includegraphics[width=0.7\textwidth]{Figures/Chapter_16/slide_79.jpg}
    \caption{Failure cases of RNN-based image captioning, where captions reflect dataset biases rather than true understanding.}
    \label{fig:chapter16_captioning_failures}
\end{figure}

\noindent Notable failure cases include:
\begin{itemize}
    \item \textbf{Texture Confusion:}  
    \textit{"A woman is holding a cat in her hand."}  
    \newline → Incorrect. The model misinterprets a fur coat as a cat due to similar texture.
    
    \item \textbf{Outdated Training Data:}  
    \textit{"A person holding a computer mouse on a desk."}  
    \newline → Incorrect. Since the dataset predates smartphones, the model assumes any small handheld object near a desk is a computer mouse.
    
    \item \textbf{Contextual Overgeneralization:}  
    \textit{"A woman standing on a beach holding a surfboard."}  
    \newline → Incorrect. The model associates beaches with surfing due to frequent co-occurrence in the dataset.
    
    \item \textbf{Co-Occurrence Bias:}  
    \textit{"A bird is perched on a tree branch."}  
    \newline → Incorrect. The model predicts a bird even though none are present, likely due to birds frequently appearing in similar scenes in the dataset.
    
    \item \textbf{Failure to Understand Actions:}  
    \textit{"A man in a baseball uniform throwing a ball."}  
    \newline → Incorrect. The model fails to distinguish between throwing and catching, highlighting a lack of true scene comprehension.
\end{itemize}

\noindent These errors indicate that RNN-based captioning models rely heavily on \textbf{statistical associations} rather than genuine reasoning. Their fixed-size hidden state struggles to store complex dependencies, and they lack explicit mechanisms to retain and retrieve relevant information over long sequences.

\subsection{Bridging to LSTMs and GRUs: The Need for Gated Memory}
\label{sec:chapter16_bridging_to_lstm_gru}

\noindent
The previous subsection gave a theoretical reason why vanilla RNNs with any scalar activation \(\phi\) face an unavoidable trade-off between stability and long-term memory (Table~\ref{tab:chapter16_activation_tradeoffs}).
The examples in this section—especially image captioning—show how this trade-off manifests in practice.

\medskip
\noindent
In text generation and captioning, a vanilla \(\tanh\)-RNN is usually numerically stable and can capture local statistics (syntax, common phrases, co-occurrence patterns), but it exhibits several task-level limitations:

\begin{itemize}
	\item \textbf{Hidden-state bottleneck.}
	At each timestep, all relevant information from the past must be compressed into a single vector \(\mathbf{h}_t\).
	As sequences grow longer, new inputs overwrite older information, making it hard to remember which objects appeared in the image or how a sentence began.
	
	\item \textbf{Gradient-driven myopia.}
	As discussed in Section~\ref{subsec:chapter16_tanh_summary}, gradients in a vanilla RNN are dominated by nearby timesteps.
	In captioning, this means the model is trained mainly to get the next few words right; long-range dependencies (for example, maintaining the correct subject over an entire sentence) are only weakly enforced.
	
	\item \textbf{No explicit, controllable memory.}
	The only state is the hidden vector \(\mathbf{h}_t\), updated by the same affine map and nonlinearity at every step.
	There is no mechanism to preserve some information for many steps while freely updating other parts of the representation, nor a way to decide \emph{what} to remember and \emph{when} to forget.
\end{itemize}

\noindent
The qualitative failure cases in image captioning (Figures~\ref{fig:chapter16_captioning_failures} and related discussion) are concrete symptoms of these limitations:
captions drift toward dataset biases, confuse visually similar textures, and forget earlier context even when the RNN is otherwise well trained and numerically stable.

\newpage

\textbf{LSTMs} and \textbf{GRUs} address these issues not by changing the activation function, but by changing the \emph{structure} of the recurrence.
As previewed in Section~\ref{subsec:chapter16_tanh_summary}, they introduce:

\begin{itemize}
	\item An explicit \emph{memory state} that is updated largely \emph{additively}, so information (and gradients) can flow with gain close to \(1\) over many timesteps.
	\item \emph{Multiplicative gates} that learn when to write new information into memory, when to keep it, and when to erase it, rather than relying on the same fixed update rule at every step.
\end{itemize}

\noindent
These gated architectures still use stable nonlinearities such as \(\tanh\) and sigmoid at the unit level, but wrap them in a design that decouples ``remembering'' from ``processing.''
The result is a recurrent model that can maintain information over hundreds of steps while remaining numerically robust.

\noindent
In the next section we will make this concrete by deriving the LSTM cell, showing how its additive memory path and gates implement the constant-error-carousel mechanism, and then introducing the more compact GRU.

\newpage

\section{Long Short-Term Memory (LSTM) Overview}
\label{sec:chapter16_lstm_overview}

\noindent
Long Short-Term Memory (LSTM) networks, introduced by Hochreiter and Schmidhuber~\cite{hochreiter1997_lstm}, represent a pivotal evolutionary step in sequence modeling.
LSTMs, together with related gated RNNs such as GRUs, dominated natural language processing and many time-series applications throughout the 2010s, providing the first robust, general-purpose mechanism for learning long-range dependencies with gradient-based training.
As of 2025, however, most new large-scale sequence models---including those deployed on mobile and edge devices---are based on attention and Transformer-style architectures, often with convolutional stems and lightweight, hardware-aware Transformer blocks replacing recurrent layers.

\noindent
Despite this shift, understanding the LSTM remains highly valuable.
Historically, LSTMs were the first widely adopted architecture to explicitly \emph{separate memory from nonlinear processing} by maintaining an internal cell state that is updated largely \emph{additively}, while using multiplicative \emph{gates} to decide when to write, keep, or erase information.
This design creates a stable gradient pathway over long sequences, mitigating the vanishing-gradient problem that plagues vanilla RNNs, yet still allowing the network to forget irrelevant context.
Conceptually, LSTMs form an important stepping stone toward the gated and residual pathways used in Highway Networks and ResNets, and toward the data-dependent relevance weighting implemented by attention and Transformers, which we study next.

\subsection{LSTM States and Gating Mechanism}
\label{sec:chapter16_lstm_gating}

\noindent
In a vanilla RNN, there is a single hidden state \(\mathbf{h}_t\) that is both the internal memory and the output of the recurrent block.
LSTMs separate these roles and maintain two states at each timestep:
\begin{itemize}
	\item \textbf{Cell state} \(\mathbf{c}_t\): A long-term memory that can carry information across many timesteps with relatively minor modification.
	\item \textbf{Hidden state} \(\mathbf{h}_t\): A short-term, output-facing representation that interacts with inputs and downstream layers.
\end{itemize}

\noindent
Rather than updating \(\mathbf{h}_t\) directly via a fixed affine transformation and nonlinearity, LSTMs introduce a small set of \emph{gates} that regulate information flow.
At time \(t\), given input \(\mathbf{x}_t \in \mathbb{R}^{I}\) and previous hidden state \(\mathbf{h}_{t-1} \in \mathbb{R}^{H}\), they compute:
\begin{itemize}
	\item A \textbf{forget gate} \(\mathbf{f}_t \in (0,1)^H\), which decides how much of \(\mathbf{c}_{t-1}\) to keep.
	\item An \textbf{input gate} \(\mathbf{i}_t \in (0,1)^H\), which decides how much new information to write.
	\item A \textbf{candidate update} \(\mathbf{g}_t \in [-1,1]^H\), which proposes new content to add to the cell.
	\item An \textbf{output gate} \(\mathbf{o}_t \in (0,1)^H\), which decides how much of the internal memory to expose as \(\mathbf{h}_t\).
\end{itemize}
These gates are themselves learned nonlinear functions of \((\mathbf{h}_{t-1},\mathbf{x}_t)\), and they are trained jointly with the rest of the network by backpropagation through time.

\newpage

\subsection{LSTM Gate Computation}
\label{sec:chapter16_lstm_gates}

\noindent
Let the previous hidden state be \(\mathbf{h}_{t-1} \in \mathbb{R}^H\) and the current input be \(\mathbf{x}_t \in \mathbb{R}^I\).
To control the flow of information, the LSTM computes four vector-valued quantities that share the same input structure but play different roles:
\[
\mathbf{W}_*^{(h)} \in \mathbb{R}^{H \times H},
\quad
\mathbf{W}_*^{(x)} \in \mathbb{R}^{H \times I},
\quad
\mathbf{b}_* \in \mathbb{R}^H,
\qquad
* \in \{f,i,g,o\}.
\]
The gate activations are then
\begin{align*}
	\mathbf{f}_t &= \sigma\!\bigl(\mathbf{W}_f^{(h)} \mathbf{h}_{t-1} + \mathbf{W}_f^{(x)} \mathbf{x}_t + \mathbf{b}_f\bigr)
	&& \text{(forget gate)}, \\
	\mathbf{i}_t &= \sigma\!\bigl(\mathbf{W}_i^{(h)} \mathbf{h}_{t-1} + \mathbf{W}_i^{(x)} \mathbf{x}_t + \mathbf{b}_i\bigr)
	&& \text{(input gate)}, \\
	\mathbf{g}_t &= \tanh\!\bigl(\mathbf{W}_g^{(h)} \mathbf{h}_{t-1} + \mathbf{W}_g^{(x)} \mathbf{x}_t + \mathbf{b}_g\bigr)
	&& \text{(cell candidate)}, \\
	\mathbf{o}_t &= \sigma\!\bigl(\mathbf{W}_o^{(h)} \mathbf{h}_{t-1} + \mathbf{W}_o^{(x)} \mathbf{x}_t + \mathbf{b}_o\bigr)
	&& \text{(output gate)}.
\end{align*}
Each element of \(\mathbf{f}_t\), \(\mathbf{i}_t\), and \(\mathbf{o}_t\) lies in \([0,1]\), and each element of \(\mathbf{g}_t\) lies in \([-1,1]\).

\medskip
\noindent
\textbf{Intuition: Gates as soft masks and signed content.}
The choice of nonlinearities separates \emph{control signals} from \emph{content}.

\begin{itemize}
	\item \textbf{Sigmoid Gates As Soft Masks.}  
	Sigmoid activations for \(\mathbf{f}_t\), \(\mathbf{i}_t\), and \(\mathbf{o}_t\) make each gate coordinate behave like a differentiable valve in \([0,1]\).
	A value near \(0\) means ``block this channel completely'', a value near \(1\) means ``pass this channel unchanged'', and intermediate values implement soft decisions such as ``keep roughly \(80\%\) of the old memory while writing a bit of new information''.
	In particular, elements of \(\mathbf{f}_t\) directly scale the previous cell state \(\mathbf{c}_{t-1}\), so the network can learn coordinates that act as almost-perfect copies over hundreds of timesteps (\(\mathbf{f}_t \approx 1\)) and other coordinates that forget rapidly (\(\mathbf{f}_t \approx 0\)).
	
	\item \textbf{Tanh Candidate As Signed Content.}  
	The candidate vector \(\mathbf{g}_t\) carries the \emph{content} that might be written into memory.
	Using \(\tanh\) keeps \(\mathbf{g}_t\) zero-centered and bounded in \([-1,1]\), which has two important consequences.
	First, coordinates of \(\mathbf{g}_t\) can contribute positively or negatively to the cell state, so the LSTM can both reinforce and actively \emph{counteract} previously stored information, for example reducing the weight of an old topic as the sequence shifts to a new subject.
	Second, the bounded range prevents uncontrolled growth of the internal memory; without a signed, bounded update, \(\mathbf{c}_t\) would tend to drift or explode over long sequences, making optimization unstable.
\end{itemize}

\medskip
\noindent
In practice, deep learning libraries optimize these computations by vectorizing them.
Instead of applying four separate affine transformations, we concatenate the hidden state and input,
\[
\mathbf{z}_t =
\begin{bmatrix}
	\mathbf{h}_{t-1} \\
	\mathbf{x}_t
\end{bmatrix}
\in \mathbb{R}^{H+I},
\]
and use a single weight matrix and bias:
\[
\begin{bmatrix}
	\mathbf{f}_t \\
	\mathbf{i}_t \\
	\mathbf{g}_t \\
	\mathbf{o}_t
\end{bmatrix}
=
\mathbf{W}\mathbf{z}_t + \mathbf{b},
\qquad
\mathbf{W} \in \mathbb{R}^{4H \times (H+I)},
\;
\mathbf{b} \in \mathbb{R}^{4H}.
\]
The resulting \(4H\)-dimensional vector is then split into four blocks and passed through the appropriate nonlinearities:
\[
\mathbf{f}_t = \sigma(\cdot),
\quad
\mathbf{i}_t = \sigma(\cdot),
\quad
\mathbf{g}_t = \tanh(\cdot),
\quad
\mathbf{o}_t = \sigma(\cdot).
\]
This single-matrix implementation is mathematically equivalent to using separate weights per gate, but it is more efficient on modern hardware and is the default in PyTorch, TensorFlow, JAX, and related frameworks.

\medskip
\noindent
\textbf{Remark (Peephole LSTMs).}
In the standard LSTM variant above, gates depend only on \(\mathbf{h}_{t-1}\) and \(\mathbf{x}_t\).
This means that if the output gate was mostly closed at the previous step (\(\mathbf{o}_{t-1} \approx \mathbf{0}\)), the hidden state \(\mathbf{h}_{t-1}\) may reveal little about the actual contents of the memory cell \(\mathbf{c}_{t-1}\).
\emph{Peephole connections} address this by also feeding \(\mathbf{c}_{t-1}\) into the gate computations, for example:
\[
\mathbf{f}_t
=
\sigma\!\bigl(
\mathbf{W}_f^{(h)} \mathbf{h}_{t-1}
+ \mathbf{W}_f^{(x)} \mathbf{x}_t
+ \mathbf{W}_f^{(c)} \mathbf{c}_{t-1}
+ \mathbf{b}_f
\bigr).
\]
A convenient way to summarize the design trade-offs is:

\noindent\textbf{Pros.}
\begin{itemize}
	\item Makes gates aware of hidden memory contents. In a standard LSTM, if the output gate is mostly closed (\(\mathbf{o}_{t-1} \approx 0\)), then \(\mathbf{h}_{t-1}\) carries almost no information about what is stored in \(\mathbf{c}_{t-1}\), so the forget and input gates at time \(t\) must decide what to do without really “seeing” the current memory. Peephole connections remove this blind spot by letting each gate look directly at \(\mathbf{c}_{t-1}\) when deciding whether to keep, overwrite, or expose information.
	\item Enables precise counting and timing. When gates can read \(\mathbf{c}_t\) itself, the cell state can act as an internal counter or clock, increasing by a fixed amount each step until a learned threshold is reached, at which point a gate can reliably open or close. This makes peephole LSTMs well suited for tasks with sharp temporal boundaries or periodic structure, such as waiting for exactly \(N\) steps before emitting a signal or aligning outputs to regular beats.
\end{itemize}

\noindent\textbf{Cons.}
\begin{itemize}
	\item Peephole connections slightly complicate the architecture and add extra parameters, breaking the clean separation between the internal memory pathway and the externally visible hidden state.
	\item They marginally complicate efficient vectorized implementations, and empirical gains on common language modeling and translation benchmarks are modest, so most modern libraries default to the simpler peephole-free formulation.
\end{itemize}


\subsection{LSTM State Updates and Outputs}
\label{sec:chapter16_lstm_updates}

\noindent
Once the gates are computed, the LSTM updates its internal memory (cell state) and its visible output (hidden state) at each timestep \(t\).

\paragraph{Cell state update (additive memory path)}
The cell state \(\mathbf{c}_t\) is updated by a gated combination of the previous cell state and the candidate update:
\[
\mathbf{c}_t
=
\underbrace{\mathbf{f}_t \odot \mathbf{c}_{t-1}}_{\text{Keep selected parts of the past}}
+
\underbrace{\mathbf{i}_t \odot \mathbf{g}_t}_{\text{Write new information into memory}},
\]

\newpage

where \(\odot\) denotes elementwise multiplication.
Intuitively:
\begin{itemize}
	\item The term \(\mathbf{f}_t \odot \mathbf{c}_{t-1}\) determines which components of the previous memory should be preserved and which should be attenuated.
	\item The term \(\mathbf{i}_t \odot \mathbf{g}_t\) injects newly computed content into the memory, but only along dimensions where the input gate is open.
\end{itemize}
Because this update is additive rather than repeatedly multiplying by a recurrent weight matrix, it provides a near-identity pathway when \(\mathbf{f}_t \approx \mathbf{1}\) and \(\mathbf{i}_t \approx \mathbf{0}\).
Along such coordinates, information can persist almost unchanged for many timesteps, and gradients can flow backward through time without exponentially vanishing.

\paragraph{Hidden state update (exposing memory to the network)}
The hidden state \(\mathbf{h}_t\) is obtained by filtering a squashed version of the cell state:
\[
\mathbf{h}_t
=
\mathbf{o}_t \odot \tanh(\mathbf{c}_t).
\]
Here, \(\tanh(\mathbf{c}_t)\) produces a bounded, zero-centered summary of the internal memory, and the output gate \(\mathbf{o}_t\) decides how much of that summary to expose at timestep \(t\).

\noindent
It is useful to think of \(\mathbf{c}_t\) as long-term memory and \(\mathbf{h}_t\) as working memory that is currently visible to the rest of the network.
For a sequence \((\mathbf{x}_1,\dots,\mathbf{x}_T)\), the update above is applied at every timestep \(t=1,\dots,T\).

\paragraph{The dual role of \(\mathbf{h}_t\)}
The hidden state \(\mathbf{h}_t\) serves two roles simultaneously at each timestep \(t\):
\begin{itemize}
	\item \textbf{Horizontal (temporal) role.} \(\mathbf{h}_t\) is passed forward in time to the next LSTM cell as part of the input for timestep \(t+1\), providing context about everything the model has processed so far.
	\item \textbf{Vertical (output) role.} \(\mathbf{h}_t\) is also passed upward to subsequent layers or an output head at the \emph{same} timestep, enabling the network to produce a prediction based on the current context.
\end{itemize}
Thus, at every timestep the LSTM both updates its internal memory for the future and provides a representation that can be decoded into an output for the present.

\subsubsection{From hidden states to predictions}
\label{subsec:chapter16_lstm_outputs}

\noindent
The LSTM cell defines how \((\mathbf{c}_t,\mathbf{h}_t)\) evolve over time, but most learning tasks require predictions \(\hat{\mathbf{y}}_t\) in some output space, such as a vocabulary distribution for language modeling or a real-valued vector for regression.
To obtain such predictions, a separate \emph{output projection} maps the hidden state \(\mathbf{h}_t\) to the desired output:
\[
\hat{\mathbf{y}}_t
=
\varphi\!\bigl(\mathbf{W}_{hy}\mathbf{h}_t + \mathbf{b}_y\bigr),
\]
where:
\begin{itemize}
	\item Parameter \(\mathbf{W}_{hy} \in \mathbb{R}^{D \times H}\) and bias \(\mathbf{b}_y \in \mathbb{R}^D\) are trainable output-layer parameters that map the hidden size \(H\) to an output dimension \(D\).
	\item Dimension \(D\) is the size of the output space, such as the vocabulary size in language modeling or the number of regression targets.
	\item Function \(\varphi\) is a task-dependent activation, such as softmax for multiclass classification, identity for regression, or sigmoid for binary outputs.
\end{itemize}

\noindent
In autoregressive sequence modeling tasks such as language modeling, this projection is usually applied at every timestep \(t\), producing a distribution \(\hat{\mathbf{y}}_t\) over the next token given the prefix \((\mathbf{x}_1,\dots,\mathbf{x}_t)\).

\newpage

In sequence classification tasks (for example, sentiment analysis), it is common to ignore intermediate outputs and apply the projection only to a pooled representation, such as the final hidden state \(\mathbf{h}_T\) or an aggregate of all hidden states.


\begin{figure}[H]
	\centering
	\includegraphics[width=0.8\textwidth]{Figures/Chapter_16/slide_90.jpg}
	\caption{Long Short-Term Memory (LSTM) architecture. All four gates are computed from the concatenated input \([\mathbf{h}_{t-1}, \mathbf{x}_t]\) via a single affine transformation, then split and passed through sigmoid or \(\tanh\). The cell state \(\mathbf{c}_t\) provides an additive memory path, while the hidden state \(\mathbf{h}_t\) is used for downstream predictions.}
	\label{fig:chapter16_lstm_architecture}
\end{figure}

\subsection{Gradient Flow in LSTMs}
\label{sec:chapter16_gradient_flow_lstm}

\noindent
Section~\ref{sec:chapter16_bptt} showed that in vanilla RNNs, gradients backpropagated through time are dominated by products of Jacobians of the form
\[
\prod_{t} \mathbf{J}_t
\quad\text{with}\quad
\mathbf{J}_t
=
\mathrm{diag}\bigl(\phi'(\cdot)\bigr)\mathbf{W}_{hh},
\]
which either explode or vanish over long horizons depending on the spectrum of \(\mathbf{W}_{hh}\).
LSTMs change this picture by introducing an internal cell state \(\mathbf{c}_t\) that is updated additively and does not pass through a recurrent weight matrix at every step.
As a result, there is a primary error path that behaves much closer to a near-identity mapping, controlled by the forget gate rather than by repeated multiplication with \(\mathbf{W}_{hh}\).

\subsubsection{Cell state as a long-term gradient highway}
\label{subsec:chapter16_lstm_cell_state}

\noindent
Recall the cell-state update:
\[
\mathbf{c}_t
=
\mathbf{f}_t \odot \mathbf{c}_{t-1}
+
\mathbf{i}_t \odot \mathbf{g}_t.
\]
To study gradient flow, we examine the Jacobian of \(\mathbf{c}_t\) with respect to \(\mathbf{c}_{t-1}\).
Using the product rule (and omitting diagonal notation for brevity), we obtain
\[
\frac{\partial \mathbf{c}_t}{\partial \mathbf{c}_{t-1}}
=
\underbrace{\mathbf{f}_t}_{\text{direct path}}
+
\underbrace{
	\mathbf{c}_{t-1} \odot \frac{\partial \mathbf{f}_t}{\partial \mathbf{c}_{t-1}}
	+
	\mathbf{g}_t \odot \frac{\partial \mathbf{i}_t}{\partial \mathbf{c}_{t-1}}
	+
	\mathbf{i}_t \odot \frac{\partial \mathbf{g}_t}{\partial \mathbf{c}_{t-1}}
}_{\text{indirect gate-dependent paths}}.
\]
The first term, \(\mathbf{f}_t\), is the \emph{direct} scaling of \(\mathbf{c}_{t-1}\) as it flows into \(\mathbf{c}_t\).
The remaining terms capture how changes in \(\mathbf{c}_{t-1}\) influence \(\mathbf{c}_t\) indirectly via changes in the gates.

\medskip
\noindent
The key observation is that the indirect terms are always modulated by derivatives of sigmoids or \(\tanh\), which are bounded:
\begin{itemize}
	\item Sigmoid derivatives satisfy \(\sigma'(z) \leq 0.25\) and quickly approach \(0\) when \(|z|\) is large.
	\item Tanh derivatives satisfy \(\tanh'(z) \leq 1\) and also approach \(0\) as \(|z|\) grows.
\end{itemize}
As gradients are propagated backward through many timesteps, these indirect paths involve long products of such small factors and therefore decay rapidly.
They act as local corrections that matter over a few steps, but they do not sustain gradients over long horizons.

\noindent
By contrast, the direct term \(\mathbf{f}_t\) appears \emph{without} an additional activation derivative in this path.
For a loss \(\mathcal{L}\) decomposed as \(\mathcal{L} = \sum_{k=1}^T \mathcal{L}_k\), the dominant contribution to
\(\partial \mathcal{L} / \partial \mathbf{c}_t\) along the cell-state chain satisfies
\[
\frac{\partial \mathcal{L}}{\partial \mathbf{c}_t}
\approx
\sum_{k=t}^T
\frac{\partial \mathcal{L}_k}{\partial \mathbf{c}_k}
\prod_{j=t+1}^k \mathbf{f}_j.
\]
Because \(\mathbf{f}_j \in (0,1)^H\), each coordinate of the gradient along a specific cell dimension is scaled only by the corresponding coordinate of \(\mathbf{f}_j\).
If a particular coordinate of \(\mathbf{f}_j\) is learned to stay close to \(1\) over many steps, then the gradient along that coordinate can travel backward across long time horizons with little attenuation.
This mechanism is often referred to as the \emph{constant error carousel} in the original LSTM paper~\cite{hochreiter1997_lstm}.

\subsubsection{Why the forget gate prevents severe vanishing}
\label{subsec:chapter16_lstm_forget_gate}

\noindent
In a vanilla RNN, the analogue of the forget gate is the Jacobian
\[
\frac{\partial \mathbf{h}_t}{\partial \mathbf{h}_{t-1}}
=
\mathrm{diag}\bigl(\phi'(\cdot)\bigr)\mathbf{W}_{hh},
\]
which combines a recurrent weight matrix and activation derivatives that are often much less than \(1\) in magnitude.
Repeated multiplication by such matrices quickly drives gradients toward zero or infinity unless \(\mathbf{W}_{hh}\) is carefully constrained.

\noindent
In an LSTM, the main long-range path is instead governed by
\[
\prod_{j=t+1}^{T}
\frac{\partial \mathbf{c}_j}{\partial \mathbf{c}_{j-1}}
\approx
\prod_{j=t+1}^{T} \mathbf{f}_j,
\]
so gradient preservation is controlled directly by the learned forget gates rather than by the eigenvalues of a shared recurrent matrix.
Two properties are crucial:
\begin{itemize}
	\item The forget gate \(\mathbf{f}_j = \sigma(\cdot)\) is directly parameterized by its own weights and bias, so the network can explicitly learn to keep certain coordinates near \(1\) whenever it is beneficial to store information across long time spans.
	\item Coordinates that do not need long-term memory can be driven toward \(0\), allowing the network to forget irrelevant information and preventing unnecessary accumulation in the cell state.
\end{itemize}
Thus, instead of being forced to live near a narrow spectral radius regime of \(\mathbf{W}_{hh}\), the model gains fine-grained, dimension-wise control over how quickly information and gradients decay.

\paragraph{Practical note: forget gate bias initialization}
\noindent
In principle, the network should learn to set \(\mathbf{f}_t \approx \mathbf{1}\) on coordinates that ought to store long-term information.
However, standard symmetric initialization (weights and biases near zero) yields
\(\mathbf{f}_t = \sigma(0) = 0.5\) at the start of training.
This means that, before any learning has taken place, both the cell state and its gradients decay by roughly a factor of \(0.5\) per timestep, so after \(T\) steps the signal is attenuated by about \(0.5^T\).
For moderately long sequences, this is effectively zero, and the model never receives a strong gradient signal that would tell it to \emph{open} the forget gate.
This is a kind of ``chicken-and-egg'' problem: the network would like to learn long-term memory, but the gradients needed to learn that behavior vanish too quickly.

\noindent
A simple and widely used remedy is to initialize the forget gate bias \(\mathbf{b}_f\) to a positive value (for example, all ones or twos) instead of zero.
This changes the behavior at initialization in two useful ways:
\begin{itemize}
	\item It yields \(\mathbf{f}_t \approx \sigma(1) \approx 0.73\) or higher, so the default behavior is closer to an identity mapping along the cell state, and gradients can traverse many timesteps before decaying appreciably.
	\item It effectively provides an \emph{identity skip connection through time}, analogous to the residual connections in ResNets, so training starts in a regime with ``almost infinite'' memory and the model only has to learn when and where to forget, rather than struggling to learn long-term retention from a short-memory initialization.
\end{itemize}

\subsubsection{Hidden-state gradients versus cell-state gradients}
\label{subsec:chapter16_hidden_grad_vanilla}

\noindent
The hidden state is given by
\[
\mathbf{h}_t
=
\mathbf{o}_t \odot \tanh(\mathbf{c}_t).
\]
Its dependence on \(\mathbf{h}_{t-1}\) runs through the gates, which themselves depend on \(\mathbf{h}_{t-1}\) via recurrent weight matrices.
Consequently, the Jacobian
\[
\frac{\partial \mathbf{h}_t}{\partial \mathbf{h}_{t-1}}
\]
can still exhibit vanishing or exploding behavior if repeatedly applied, especially in very deep stacks of LSTM layers.

\noindent
However, LSTMs do not rely solely on this hidden-state Jacobian chain to propagate long-range information.
The dominant pathway for long-term dependencies is the additive chain
\[
\mathbf{c}_1 \rightarrow \mathbf{c}_2 \rightarrow \dots \rightarrow \mathbf{c}_T,
\]
whose derivatives are governed primarily by the forget gates \(\mathbf{f}_t\).
Even if \(\partial \mathbf{h}_t / \partial \mathbf{h}_{t-1}\) is locally small or large, the model can still preserve and adjust long-term information through the cell state.
In other words, the backbone of memory and gradients runs through \(\mathbf{c}_t\), and the hidden-state chain can be viewed as a secondary, more local pathway.

\subsubsection{Weight gradients and exploding gradients}
\label{subsec:chapter16_weights_gradients_lstm}

\noindent
During backpropagation, gradients with respect to LSTM parameters receive contributions from both \(\partial \mathbf{h}_t / \partial \mathbf{W}\) and \(\partial \mathbf{c}_t / \partial \mathbf{W}\).
Because the derivative of \(\mathbf{c}_T\) with respect to \(\mathbf{c}_t\) is dominated by products of forget gates that can be kept near \(1\), the part of the parameter gradients that flows through the cell state often remains substantial even over long sequences~\cite{hochreiter1997_lstm,srivastava2015_training}.
This means that not all parameter gradients vanish simultaneously, which alleviates one of the central difficulties of training vanilla RNNs.

\noindent
On the other hand, the gating nonlinearities and \(\tanh\) are bounded, so per-step derivatives rarely exceed \(1\) by a large factor~\cite{pascanu2013_difficulty}.
When occasional large gradients do arise (for example, from the output layer or rare extreme activations), standard gradient clipping can be used as a safeguard.
Overall, LSTMs are significantly less prone to catastrophic exploding gradients than vanilla RNNs with unbounded activations, while providing a principled mechanism to preserve gradients over long horizons.

\begin{figure}[H]
	\centering
	\includegraphics[width=0.8\textwidth]{Figures/Chapter_16/slide_92.jpg}
	\caption{
		Gradient flow in an LSTM. The primary path for long-range information and gradients runs through the cell states \(\mathbf{c}_t\), updated additively and scaled by forget gates \(\mathbf{f}_t\). Other paths through gates and hidden states exist but contribute smaller, more local effects.
	}
	\label{fig:chapter16_lstm_gradient_flow}
\end{figure}


\section{Resemblance of LSTMs to Highway Networks and ResNets}
\label{subsec:chapter16_lstm_highway_resnets}

\noindent
The central structural idea behind LSTMs is the additive update of an internal state, modulated by gates.
This idea closely parallels the developments that later appeared in feedforward architectures, notably Highway Networks~\cite{srivastava2015_training} and Residual Networks (ResNets)~\cite{he2016_resnet}.

\subsection{Highway Networks and LSTMs}
\label{sec:chapter16_highway_networks}

\noindent
Highway Networks introduced gated skip connections between layers, with the basic form
\[
\mathbf{y}(x)
=
T(x) \odot F(x)
+
\bigl(1 - T(x)\bigr) \odot x,
\]
where:
\begin{itemize}
	\item Transform function \(F(x)\) is a nonlinear mapping (for example, a small MLP) applied to the input \(x\).
	\item Transform gate \(T(x) = \sigma(\cdot)\) is a trainable gate that decides how much of \(F(x)\) to use.
	\item Carry gate \(1 - T(x)\) determines how much of the input \(x\) passes through unchanged.
\end{itemize}
This structure is conceptually very similar to the LSTM cell update
\[
\mathbf{c}_t
=
\mathbf{f}_t \odot \mathbf{c}_{t-1}
+
\mathbf{i}_t \odot \mathbf{g}_t,
\]
with \(\mathbf{f}_t\) analogous to the carry gate and \(\mathbf{i}_t\) analogous to the transform gate.
In both cases, an additive pathway allows information and gradients to propagate over many layers (or timesteps), while gates decide when to transform and when to copy. 

\subsection{ResNets and LSTMs}
\label{sec:chapter16_resnets}

\noindent
ResNets simplify this idea further by using ungated, additive skip connections:
\[
\mathbf{x}_{\ell+1}
=
\mathbf{x}_\ell + F(\mathbf{x}_\ell).
\]
This creates a near-identity mapping across layers and dramatically improves gradient flow in very deep networks.

\noindent
Comparing this with the LSTM cell update:
\[
\mathbf{c}_t
=
\mathbf{f}_t \odot \mathbf{c}_{t-1}
+
\mathbf{i}_t \odot \mathbf{g}_t,
\]
we see that both designs share the idea of additive updates as a way to stabilize optimization.
ResNets use fixed identity skips (no gates) and are well suited to spatial feature extraction, while LSTMs use gated skips that can adaptively control information flow across time.

\paragraph{High-level comparison}
\begin{itemize}
	\item \textbf{Highway Networks vs. LSTMs}. Both use learned gates to interpolate between transformed and carried information, with LSTMs applying this principle along the temporal axis and Highway Networks across depth in feedforward networks.
	\item \textbf{ResNets vs. LSTMs}. ResNets remove the gates and rely on pure identity skips, trading flexibility for simplicity and scalability to very deep stacks, while LSTMs retain gates to gain fine-grained temporal control over what is remembered or forgotten.
\end{itemize}

\begin{figure}[H]
	\centering
	\includegraphics[width=0.8\textwidth]{Figures/Chapter_16/slide_95.jpg}
	\caption{Analogy between ResNets and LSTMs. Both use additive connections to stabilize gradient flow, but LSTMs employ gates to modulate information retention across time, whereas ResNets use fixed identity skips across layers.}
	\label{fig:chapter16_resnets_lstm_similarity}
\end{figure}

\subsection{Summary of LSTM, Highway, and ResNet Connections}

\noindent
Viewed in a unified way, LSTMs, Highway Networks, and ResNets all implement variations on the same core theme.
They provide an easy, additive path for gradients, and use multiplicative components (gates or residual transforms) to add flexible computation on top.
LSTMs apply this pattern in time, Highway Networks across depth with gates, and ResNets across depth with ungated identity connections.

\section{Bidirectional LSTMs}
\label{sec:chapter16_bilstm}

\noindent
Standard LSTMs process sequences in a single temporal direction (typically left-to-right).
At timestep \(t\), the hidden state \(\mathbf{h}_t\) summarizes only the \emph{past} inputs \((\mathbf{x}_1,\dots,\mathbf{x}_t)\), but not the \emph{future} inputs \((\mathbf{x}_{t+1},\dots,\mathbf{x}_T)\).
This is appropriate for online or autoregressive settings (for example, streaming speech recognition or next-word prediction), but it is suboptimal whenever the entire input sequence is available upfront and decisions should depend on \emph{both} left and right context.

\noindent
A classic example is \emph{machine translation} (for example, English \(\rightarrow\) German) in an encoder–decoder architecture.
The encoder receives the full source sentence before the decoder starts producing the target sentence, so in principle it could exploit information from \emph{all} source tokens when constructing the representation for each position.
Unidirectional LSTMs cannot do this: at the position of a source word, they only know the prefix, not the suffix.

\noindent
Consider the English sentence:
\begin{quote}
	``He \textbf{turned} the heavily protected master switch \textbf{on}.''
\end{quote}
To translate this into German, the system must decide at the verb position whether the sense is ``rotate'' or ``switch on'':
\begin{itemize}
	\item ``He rotated the switch.'' \(\rightarrow\) \emph{Er drehte den Schalter.}
	\item ``He turned the switch on.'' \(\rightarrow\) \emph{Er schaltete den Schalter ein.}
\end{itemize}
A left-to-right LSTM at the word ``turned'' has seen only the prefix ``He turned the heavily protected master switch \dots'', but not the particle ``on''.
It must guess between \emph{drehte} and \emph{schaltete} without yet seeing the crucial future context.
A bidirectional LSTM fixes this by also running a backward LSTM that has already processed the ``\dots switch on'' part when constructing the representation at ``turned''.

\subsection{Architecture and information flow}
\label{subsec:chapter16_bilstm_architecture}

\noindent
Bidirectional LSTMs (BiLSTMs) address this limitation by running \emph{two} independent LSTMs over the same sequence:
\begin{itemize}
	\item A \textbf{forward LSTM} that reads from left to right, \(t = 1 \rightarrow T\), with parameters \(\theta_{\rightarrow}\) and hidden states \(\overrightarrow{\mathbf{h}}_t\).
	\item A \textbf{backward LSTM} that reads from right to left, \(t = T \rightarrow 1\), with parameters \(\theta_{\leftarrow}\) and hidden states \(\overleftarrow{\mathbf{h}}_t\).
\end{itemize}
The two LSTMs do \textbf{not} share weights; they are separate networks that see the sequence in opposite directions and can learn different dynamics.

\noindent
Let the input sequence be \(\mathbf{x}_1,\dots,\mathbf{x}_T\).
For each position \(t\), we can write the recurrences abstractly as
\begin{align*}
	\overrightarrow{\mathbf{h}}_t &= \mathrm{LSTM}_{\rightarrow}\bigl(\mathbf{x}_t, \overrightarrow{\mathbf{h}}_{t-1}, \overrightarrow{\mathbf{c}}_{t-1};\, \theta_{\rightarrow}\bigr), \quad t = 1,\dots,T, \\
	\overleftarrow{\mathbf{h}}_t &= \mathrm{LSTM}_{\leftarrow}\bigl(\mathbf{x}_t, \overleftarrow{\mathbf{h}}_{t+1}, \overleftarrow{\mathbf{c}}_{t+1};\, \theta_{\leftarrow}\bigr), \quad t = T,\dots,1,
\end{align*}
where \(\overrightarrow{\mathbf{c}}_t\) and \(\overleftarrow{\mathbf{c}}_t\) are the corresponding cell states.
Unrolling these recurrences shows the \emph{effective context} seen at each timestep:
\begin{itemize}
	\item The forward state \(\overrightarrow{\mathbf{h}}_t\) summarizes the prefix \(\{\mathbf{x}_1,\dots,\mathbf{x}_t\}\).
	\item The backward state \(\overleftarrow{\mathbf{h}}_t\) summarizes the suffix \(\{\mathbf{x}_t,\dots,\mathbf{x}_T\}\).
\end{itemize}
Thus, once both passes have been run, the model has, at every position \(t\), two complementary views:
one from the left context up to and including \(\mathbf{x}_t\), and one from the right context down to and including \(\mathbf{x}_t\).

\subsection{Full-context representations at each position}
\label{subsec:chapter16_bilstm_full_context}

\noindent
After computing both directional states, a BiLSTM forms a combined representation at each timestep \(t\), typically by concatenation:
\[ 
\mathbf{y}_t
=
\begin{bmatrix}
	\overrightarrow{\mathbf{h}}_t \\
	\overleftarrow{\mathbf{h}}_t
\end{bmatrix}.
\]
By construction, \(\mathbf{y}_t\) encodes information from the \emph{entire} sequence:
\[
\text{Context}(\mathbf{y}_t)
=
\{\mathbf{x}_1,\dots,\mathbf{x}_{t-1}\}
\cup
\{\mathbf{x}_t\}
\cup
\{\mathbf{x}_{t+1},\dots,\mathbf{x}_T\}.
\]

\noindent
One useful way to see this is step by step:
\begin{itemize}
	\item At \(t = 1\), the forward LSTM has seen only \(\mathbf{x}_1\), while the backward LSTM has already processed \(\mathbf{x}_T,\dots,\mathbf{x}_2,\mathbf{x}_1\), so \(\overleftarrow{\mathbf{h}}_1\) summarizes all remaining words ``to the right'' of position \(1\).
	\item At \(t = 2\), the forward LSTM has seen \(\mathbf{x}_1,\mathbf{x}_2\), while the backward LSTM has processed \(\mathbf{x}_T,\dots,\mathbf{x}_3,\mathbf{x}_2\), so \(\overleftarrow{\mathbf{h}}_2\) summarizes everything from position \(2\) to the end.
	\item In general, for any \(t\), \(\overrightarrow{\mathbf{h}}_t\) knows the prefix \(\mathbf{x}_1,\dots,\mathbf{x}_t\) and \(\overleftarrow{\mathbf{h}}_t\) knows the suffix \(\mathbf{x}_t,\dots,\mathbf{x}_T\), so \(\mathbf{y}_t\) represents \(\mathbf{x}_t\) in the context of the entire sentence.
\end{itemize}

\noindent
Returning to the translation example ``He turned the heavily protected master switch on.'', suppose we focus on the token ``turned''.
\begin{itemize}
	\item The forward state \(\overrightarrow{\mathbf{h}}_{\text{turned}}\) summarizes the prefix ``He turned the heavily protected master switch \dots'', which is still ambiguous between ``rotate'' and ``switch on''.
	\item The backward state \(\overleftarrow{\mathbf{h}}_{\text{turned}}\) has already processed ``\dots master switch on'', and therefore encodes the presence of the particle ``on'' and the surrounding context.
	\item The combined vector \(\mathbf{y}_{\text{turned}} = [\overrightarrow{\mathbf{h}}_{\text{turned}};\overleftarrow{\mathbf{h}}_{\text{turned}}]\) can therefore support the correct choice of a German verb such as \emph{schaltete} rather than \emph{drehte}.
\end{itemize}
In other words, at each source position the BiLSTM encoder constructs a representation that already ``knows'' about future words that may be crucial for a faithful translation.

\subsection{Using BiLSTM states for predictions}
\label{subsec:chapter16_bilstm_outputs}

\noindent
Once \(\mathbf{y}_t\) has been formed, it plays the same role that \(\mathbf{h}_t\) played for a unidirectional LSTM in Section~\ref{subsec:chapter16_lstm_outputs}.
In many pre-Transformer machine translation systems, a BiLSTM was used as the \emph{encoder}, producing the sequence \(\mathbf{y}_1,\dots,\mathbf{y}_T\) as a context-rich representation of the source sentence.
A decoder (often a unidirectional LSTM) then consumed this sequence directly or via an attention mechanism.

\noindent
For simpler token-level prediction tasks (for example, part-of-speech tagging or named-entity recognition), a typical choice is a linear output layer with an activation \(\varphi\):
\[
\hat{\mathbf{y}}_t
=
\varphi\!\bigl(\mathbf{W}_y \mathbf{y}_t + \mathbf{b}_y\bigr),
\]
where:
\begin{itemize}
	\item Parameter \(\mathbf{W}_y \in \mathbb{R}^{D \times 2H}\) and bias \(\mathbf{b}_y \in \mathbb{R}^D\) are trainable output-layer parameters.
	\item Dimension \(2H\) reflects concatenation of the forward and backward hidden states of size \(H\) each.
	\item Dimension \(D\) is the size of the output space, for example the number of tags in a sequence-labeling task.
\end{itemize}
For sequence-level prediction (for example, sentence classification), one can aggregate BiLSTM states in several ways, such as:
\begin{itemize}
	\item Using the concatenation of the last forward state and the first backward state, \([\overrightarrow{\mathbf{h}}_T;\overleftarrow{\mathbf{h}}_1]\).
	\item Applying max- or mean-pooling over all \(\mathbf{y}_t\) and feeding the pooled vector to a classifier.
\end{itemize}

\subsection{Design trade-offs and limitations}
\label{subsec:chapter16_bilstm_tradeoffs}

\noindent
BiLSTMs were a standard building block for many pre-Transformer NLP systems and remain conceptually important, but they also introduce specific trade-offs.

\noindent\textbf{Advantages.}
\begin{itemize}
	\item They provide full left-and-right context for each token, which is especially beneficial for disambiguation and structured prediction tasks such as machine translation encoding, part-of-speech tagging, named-entity recognition, chunking, and constituency or dependency parsing.
	\item They remain relatively easy to integrate into existing LSTM-based architectures, since the forward and backward layers have the same interface as a standard LSTM and differ only in the direction of traversal.
\end{itemize}

\noindent\textbf{Limitations.}
\begin{itemize}
	\item They are inherently non-causal: computing \(\mathbf{y}_t\) requires access to the entire sequence, so BiLSTMs cannot be used for online or strictly left-to-right generation where future tokens are unknown at prediction time (for example, real-time simultaneous translation).
	\item They roughly double recurrent computation and memory, since the sequence must be processed once in each direction, and all intermediate states \(\overrightarrow{\mathbf{h}}_t\) and \(\overleftarrow{\mathbf{h}}_t\) must be stored for backpropagation.
\end{itemize}
In settings where full sequences are available and latency is not dominated by recurrence (for example, offline translation or tagging), these costs are often acceptable.
However, in modern practice, many of the benefits of BiLSTMs for capturing bidirectional context have been superseded by self-attention mechanisms and Transformer-style encoders, which provide global context while being more parallelizable across timesteps.

\newpage

\section{Stacking Layers in RNNs and LSTMs}
\label{sec:chapter16_stacking_rnn_lstm}

\noindent
Just as feedforward networks and ConvNets benefit from depth, RNNs and LSTMs can be stacked in multiple layers.
Each additional recurrent layer operates on the sequence of hidden states produced by the layer below, enabling the model to build increasingly abstract temporal representations.

\subsection{Architecture of Stacked RNNs and LSTMs}
\label{sec:chapter16_stacked_rnn_architecture}

\noindent
In a stacked RNN or LSTM, the first layer (\(\ell=1\)) reads the raw input sequence \(\{\mathbf{x}_t\}\) and produces hidden states \(\{\mathbf{h}_t^{(1)}\}\).
The second layer (\(\ell=2\)) treats \(\{\mathbf{h}_t^{(1)}\}\) as its input sequence and produces \(\{\mathbf{h}_t^{(2)}\}\), and so on:
\[
\mathbf{h}_t^{(\ell)}
=
f^{(\ell)}\Bigl(
\mathbf{h}_t^{(\ell-1)},\,
\mathbf{h}_{t-1}^{(\ell)}
\Bigr),
\]
where \(f^{(\ell)}\) denotes either an RNN or LSTM transition function at layer \(\ell\).
For LSTMs, each layer maintains its own cell state \(\mathbf{c}_t^{(\ell)}\) and gates.

\noindent
The components in this recurrence can be interpreted as follows:
\begin{itemize}
	\item Hidden state \(\mathbf{h}_t^{(\ell)}\) is the representation at timestep \(t\) in layer \(\ell\).
	\item Input \(\mathbf{h}_t^{(\ell-1)}\) is the output from layer \(\ell-1\) at time \(t\).
	\item Previous hidden state \(\mathbf{h}_{t-1}^{(\ell)}\) is the temporal context from the same layer \(\ell\) at the previous timestep.
\end{itemize}
Lower layers typically capture more local, short-range patterns (for example, character-level statistics or short phrases), while higher layers capture more global, long-range structure (for example, sentence-level semantics).
This hierarchy is closely analogous to the way deeper ConvNet layers capture higher-level spatial features.

\begin{figure}[H]
	\centering
	\includegraphics[width=0.7\textwidth]{Figures/Chapter_16/slide_97.jpg}
	\caption{A two-layer stacked RNN. The first layer reads the input sequence; its hidden states serve as the input sequence for the second layer.}
	\label{fig:chapter16_two_layer_rnn}
\end{figure}

\begin{figure}[H]
	\centering
	\includegraphics[width=0.7\textwidth]{Figures/Chapter_16/slide_98.jpg}
	\caption{A three-layer stacked RNN. Each additional layer refines the temporal representation generated by the layer below, analogous to depth in feedforward networks.}
	\label{fig:chapter16_three_layer_rnn}
\end{figure}

\subsection{Practical Limitations of Deep Recurrent Stacks}
\label{subsec:chapter16_deep_rnn_limitations}

\noindent
While depth increases representational power, deep recurrent stacks also incur practical costs:
\begin{itemize}
	\item Diminishing Returns. Beyond a modest number of layers (often 2--4), additional recurrent layers frequently yield only marginal gains, especially when combined with strong regularization and large hidden sizes.
	\item Overfitting Risk. Each added layer introduces many new parameters, increasing the risk of overfitting unless the dataset is large and regularization (dropout, weight decay, etc.) is carefully tuned.
	\item Optimization Difficulty. Deeper recurrent stacks are more computationally expensive and can be harder to optimize, even with LSTMs' improved gradient flow, so gradient clipping and careful initialization become more important as depth grows.
\end{itemize}

\subsection{Depth, Directionality, and Efficiency}
\label{sec:chapter16_summary_stacking}

\noindent
Stacked RNNs and LSTMs, BiLSTMs, and residual-style connections within recurrent architectures all reflect the same underlying goal:
provide sufficient capacity to model complex temporal dependencies, while preserving stable gradient pathways.
In practice, many effective architectures combine:
\begin{itemize}
	\item A small number (2--4) of stacked LSTM or BiLSTM layers.
	\item Moderate hidden-state sizes.
	\item Simple output projections as in Section~\ref{subsec:chapter16_lstm_outputs}.
\end{itemize}
This combination typically suffices to capture both local and global structure in many sequence modeling tasks, without incurring the severe optimization difficulties that arise in very deep recurrent networks.

\newpage

\begin{enrichment}[Other RNN Variants: GRU][section]
	\noindent
	\textbf{Gated Recurrent Units (GRUs)}~\cite{cho2014_gru} are a streamlined yet powerful alternative to Long Short-Term Memory (LSTM) networks, aimed at capturing long-term dependencies while reducing overall architectural complexity. GRUs merge some of the gating components found in LSTMs, thereby reducing parameters and accelerating training, while still offering effective gradient flow for many sequence modeling tasks.
	
	\begin{enrichment}[GRU Architecture][subsection]
		
		\noindent
		GRUs compress the LSTM’s three gates (input, forget, and output) into two gates:
		\begin{itemize}
			\item \textbf{Reset gate} \(r_t\): Controls how much of the previous hidden state \(\mathbf{h}_{t-1}\) is ``forgotten'' or ``reset'' before computing a new candidate.
			\item \textbf{Update gate} \(z_t\): Balances new candidate information against the existing hidden state, effectively merging the ``input'' and ``forget'' gating roles found in LSTMs.
		\end{itemize}
		
		\noindent
		Formally, a GRU evolves its hidden state as follows:
		\[
		\begin{aligned}
			r_t &= \sigma\bigl(\mathbf{W}_{xr} \,\mathbf{x}_t \;+\; \mathbf{W}_{hr} \,\mathbf{h}_{t-1} \;+\; \mathbf{b}_r\bigr), \\
			z_t &= \sigma\bigl(\mathbf{W}_{xz} \,\mathbf{x}_t \;+\; \mathbf{W}_{hz} \,\mathbf{h}_{t-1} \;+\; \mathbf{b}_z\bigr), \\
			\tilde{h}_t &= \tanh\Bigl(\mathbf{W}_{xh}\,\mathbf{x}_t \;+\; \mathbf{W}_{hh}\,\bigl(r_t \odot \mathbf{h}_{t-1}\bigr) \;+\; \mathbf{b}_h\Bigr), \\
			\mathbf{h}_t &= \bigl(1 - z_t\bigr)\odot \mathbf{h}_{t-1} \;+\; z_t \odot \tilde{h}_t.
		\end{aligned}
		\]
		
		\begin{figure}[H]
			\centering
			\includegraphics[width=0.5\textwidth]{Figures/Chapter_16/gru_visualized.png}
			\caption{Visualization of GRU architecture, illustrating the reset and update gates. Adapted from \cite{Mahadi2024_GRU_Plasmonic}.}
			\label{fig:chapter16_gru_architecture}
		\end{figure}
		
		\paragraph{Key observations and intuition}
		\begin{itemize}
			\item \textbf{Coupled ``remember--update'' behavior.}
			The hidden-state update can be written with an explicit decomposition:
			\[
			\mathbf{h}_t
			=
			\underbrace{\bigl(1 - z_t\bigr)\odot \mathbf{h}_{t-1}}_{\text{retain old features}}
			+
			\underbrace{z_t \odot \tilde{h}_t}_{\text{write new features}}.
			\]
			Each component of \(z_t\) lies between \(0\) and \(1\), so every dimension of \(\mathbf{h}_t\) is a convex combination of its old value and the candidate.
			Unlike an LSTM, where the input and forget gates are independent, a GRU \emph{couples} remembering and updating: to strongly write new content in a dimension (\(z_t \approx 1\)), the model must simultaneously reduce the contribution of the old content (\(1 - z_t \approx 0\)).
			
			\item \textbf{Exposed memory (no separate cell state).}
			LSTMs distinguish between an internal cell state \(\mathbf{c}_t\) and an output \(\mathbf{h}_t\) controlled by an output gate.
			GRUs remove this distinction: the hidden state \(\mathbf{h}_t\) itself serves as both memory and output.
			Whatever the GRU remembers at time \(t\) is immediately visible to downstream layers.
			
			\item \textbf{Reset gate as a relevance filter.}
			The reset gate \(r_t\) appears \emph{inside} the candidate computation:
			\[
			\tilde{h}_t = \tanh\Bigl(\mathbf{W}_{xh}\,\mathbf{x}_t + \mathbf{W}_{hh}\bigl(r_t \odot \mathbf{h}_{t-1}\bigr) + \mathbf{b}_h\Bigr).
			\]
			When \(r_t \approx 0\), the GRU computes \(\tilde{h}_t\) almost as if the sequence were starting fresh at timestep \(t\), ignoring most of \(\mathbf{h}_{t-1}\).
			When \(r_t \approx 1\), the full previous state participates in forming new features.
			This allows the GRU to selectively ``break'' short-term dependencies (for example, at sentence boundaries) while still maintaining long-range structure in dimensions where \(r_t\) stays high.
		\end{itemize}
		This design merges LSTM’s separate input and forget gates into the single update gate \(z_t\), simplifying the gating mechanism while retaining enough flexibility to capture rich temporal structure~\cite{cho2014_gru}.
		
	\end{enrichment}
	
	\begin{enrichment}[Gradient Flow in GRUs][subsection]
		\noindent
		Despite having fewer gates than LSTMs, GRUs preserve stable gradient flow in a way similar to LSTMs, preventing the repeated-multiplication issues that plague vanilla RNNs.
		
		\subsubsection*{Why GRUs improve over vanilla RNNs}
		In vanilla RNNs, the hidden state update
		\[
		\mathbf{h}_t = \phi(\mathbf{W}_{hh}\,\mathbf{h}_{t-1} + \dots)
		\]
		causes gradients to vanish or explode through repeated application of \(\mathbf{W}_{hh}\) and the activation derivatives \(\phi'(\cdot)\).
		In a GRU, large parts of the gradient flow pass through the \emph{update mechanism}
		\[
		\mathbf{h}_t = (1-z_t)\odot \mathbf{h}_{t-1} + z_t\odot \tilde{h}_t,
		\qquad
		\tilde{h}_t = \tanh\Bigl(\mathbf{W}_{xh}\,\mathbf{x}_t + \mathbf{W}_{hh}(r_t\odot \mathbf{h}_{t-1}) + \mathbf{b}_h\Bigr).
		\]
		Differentiating with respect to \(\mathbf{h}_{t-1}\) yields
		\[
		\frac{\partial \mathbf{h}_t}{\partial \mathbf{h}_{t-1}}
		=
		\underbrace{\mathrm{diag}(1 - z_t)}_{\text{direct additive path}}
		+
		\underbrace{\text{terms involving } z_t',\, r_t',\, \tanh'(\cdot)}_{\text{indirect gated paths}}.
		\]
		The first term, \(\mathrm{diag}(1 - z_t)\), is the \emph{direct} route by which gradients travel from \(\mathbf{h}_t\) back to \(\mathbf{h}_{t-1}\), and it does not involve multiplication by \(\mathbf{W}_{hh}\).
		The remaining terms include derivatives of sigmoids and \(\tanh\), whose magnitudes are bounded and typically smaller.
		Over long time horizons, these indirect terms tend to shrink, while the direct path governed by \(1 - z_t\) can remain close to an identity mapping.
		
		\newpage
		
		Intuitively, each component of \(z_t\) lies between \(0\) and \(1\), so the model can learn:
		\begin{itemize}
			\item \(z_t \approx 0\): keep \(\mathbf{h}_t \approx \mathbf{h}_{t-1}\), yielding a nearly perfect memory and a near-identity Jacobian for that dimension.
			\item \(z_t \approx 1\): overwrite \(\mathbf{h}_{t-1}\) with \(\tilde{h}_t\), effectively resetting that dimension while still keeping derivatives bounded through the \(\tanh\) nonlinearity.
		\end{itemize}
		This mirrors the ``constant error'' intuition of the LSTM cell state: GRUs create a trainable, dimension-wise near-identity path for gradients, but through \(\mathbf{h}_t\) rather than a separate \(\mathbf{c}_t\).
		
		\subsubsection*{Reset gate's gradient role}
		The reset gate \(r_t\) shapes how the old hidden state \(\mathbf{h}_{t-1}\) influences the candidate \(\tilde{h}_t\):
		\[
		\tilde{h}_t = 
		\tanh(\dots + \mathbf{W}_{hh}\,(r_t \odot \mathbf{h}_{t-1})),
		\qquad
		\frac{\partial \tilde{h}_t}{\partial \mathbf{h}_{t-1}}
		=
		\tanh'(\dots)\,\mathbf{W}_{hh}\,\mathrm{diag}(r_t).
		\]
		Each component of \(r_t\) lies between \(0\) and \(1\), and \(|\tanh'(z)| \le 1\), so this product remains bounded.
		When \(r_t\) is small, the candidate \(\tilde{h}_t\) and its gradients depend little on \(\mathbf{h}_{t-1}\); when \(r_t\) is large, the past state participates more strongly.
		Thus \(r_t\) controls \emph{how much} of the old representation contributes to new feature extraction, without creating unchecked gradient growth.
		
		\subsubsection*{Comparing GRU to LSTM gradient paths}
		LSTMs separate memory into a cell state \(\mathbf{c}_t\) and an output state \(\mathbf{h}_t\), and the additive update of \(\mathbf{c}_t\) provides a clear long-range gradient highway.
		GRUs unify memory and output in \(\mathbf{h}_t\), but the update rule
		\[
		\mathbf{h}_t = (1 - z_t)\odot \mathbf{h}_{t-1} + z_t\odot \tilde{h}_t
		\]
		still yields an additive Jacobian component that can be close to the identity whenever \(z_t\) is small.
		In both architectures, long-term gradients can propagate primarily along these additive paths, avoiding repeated multiplication by the full \(\mathbf{W}_{hh}\) at every step and thereby stabilizing training over longer sequences than vanilla RNNs.
		
		\paragraph{Practical note: update gate bias initialization}
		\noindent
		As with the LSTM forget gate, sensible initialization of the GRU update gate is important for gradient flow.
		With the convention used here,
		\[
		\mathbf{h}_t = (1 - z_t)\odot \mathbf{h}_{t-1} + z_t \odot \tilde{h}_t,
		\]
		small values of \(z_t\) preserve history (since \(\mathbf{h}_t \approx \mathbf{h}_{t-1}\)), whereas large values of \(z_t\) overwrite the old state with new content.
		If the update gate bias is initialized to zero, then \(z_t = \sigma(0) \approx 0.5\) at the start of training, so each dimension of \(\mathbf{h}_t\) becomes a 50/50 mixture of old and new content, which still leads to noticeable decay over many timesteps.
		
		\noindent
		To encourage a near-identity path initially, a common heuristic under this convention is to initialize the update gate bias \(\mathbf{b}_z\) to a \emph{negative} value (for example, \(-1\) or \(-2\)), pushing \(z_t = \sigma(\mathbf{b}_z)\) toward smaller values and making the initial dynamics closer to \(\mathbf{h}_t \approx \mathbf{h}_{t-1}\).
		This provides a default ``skip connection'' through time and lets the network learn when and where to increase \(z_t\) to overwrite memory, rather than having to discover long-term retention from a strongly mixing initialization.
		
	\end{enrichment}
	
	\begin{enrichment}[Advantages of GRUs over LSTMs][subsection]
		
		\noindent
		GRUs offer several key advantages~\cite{cho2014_gru}:
		\begin{itemize}
			\item \textbf{Computational efficiency}: Fewer gates and parameters lead to faster training and reduced memory usage, benefitting resource-constrained applications.
			\item \textbf{Comparably strong performance}: For many tasks and moderate sequence lengths, GRUs match or slightly exceed LSTM performance, especially when data or compute are limited.
			\item \textbf{Simplicity of implementation}: With fewer gating components, GRUs are often easier to code, tune, and interpret in terms of gating patterns.
		\end{itemize}
		
	\end{enrichment}
	
	\begin{enrichment}[Limitations of GRUs][subsection]
		
		\noindent
		Despite these advantages:
		\begin{itemize}
			\item \textbf{Reduced capacity}: Merging input and forget behavior into a single update gate can hamper modeling of extremely subtle or highly specialized long-term relationships, where LSTMs’ separate cell state and output gate can provide finer control.
			\item \textbf{Hyperparameter sensitivity}: Choosing hidden size, learning rates, or initial gate biases remains crucial. In certain problems, a suboptimal initialization can degrade performance more than in LSTMs.
			\item \textbf{Less granular control}: By combining forget and input gating into \(z_t\), GRUs provide a single mixture path. This can be less fine-grained than the distinct additive cell state in LSTMs, especially for tasks that resemble counting or require very sharp, long-range triggers.
		\end{itemize}
		
	\end{enrichment}
	
	\begin{enrichment}[Comparison with LSTMs][subsection]
		
		\noindent
		Comparing GRUs and LSTMs highlights both shared principles and structural differences:
		\begin{itemize}
			\item \textbf{Gradient behavior}: Both architectures mitigate vanishing and exploding gradients far better than vanilla RNNs by introducing gated, additive update paths (the LSTM cell state \(\mathbf{c}_t\) and the GRU update rule for \(\mathbf{h}_t\); see the gradient-flow enrichments in this chapter).
			\item \textbf{Memory representation}: LSTMs explicitly separate internal memory \(\mathbf{c}_t\) from the exposed hidden state \(\mathbf{h}_t\), whereas GRUs unify memory and output in \(\mathbf{h}_t\). This makes LSTMs slightly more expressive for tasks needing protected long-term storage, and GRUs simpler for many everyday applications.
			\item \textbf{Architectural complexity}: GRUs have two gates (reset and update) and no explicit cell state or output gate, leading to fewer parameters and somewhat lower computational cost. LSTMs have three gates and a separate cell state, offering more knobs to tune information flow at the cost of additional complexity.
		\end{itemize}
		
		\noindent
		Thus, both LSTMs and GRUs significantly improve gradient stability over vanilla RNNs.
		GRUs are often chosen in resource-limited scenarios or when the simpler gating mechanism suffices, whereas LSTMs may still prove stronger on tasks demanding very nuanced, long-distance representations or precise temporal control.
		
	\end{enrichment}
	
	\newpage
	
	\begin{enrichment}[Bridging to Advanced Architectures][subsection]
		
		\noindent
		While GRUs and LSTMs have significantly enhanced the training stability and effectiveness of recurrent neural networks, their architectures are still manually designed and may not be optimal for every task.
		Furthermore, despite their improved gradient flow, certain long-term dependencies or more complex patterns may still pose challenges.
		
		To explore alternatives, researchers have introduced methods such as \textbf{Neural Architecture Search (NAS)}, which automatically discover recurrent architectures optimized for specific tasks.
		NAS algorithms systematically explore the design space, identifying architectures that might combine beneficial aspects of GRUs, LSTMs, and other variants, resulting in even more efficient and powerful models.
	\end{enrichment}
	
\end{enrichment}

\section{Summary and Future Directions} 
\label{sec:chapter16_summary_future}

\subsection{Neural Architecture Search for Improved RNNs}

Despite the effectiveness of manually designed recurrent architectures such as LSTMs and GRUs, significant efforts continue in searching for potentially superior designs using automated methods.
\textbf{Neural Architecture Search (NAS)} systematically explores vast spaces of candidate architectures using techniques such as evolutionary algorithms or reinforcement learning.

For example, Zoph and Le~\cite{zoph2017_nas} evaluated approximately 10{,}000 candidate recurrent architectures, identifying configurations that marginally improved upon traditional LSTMs.
However, despite extensive computational investment, these improvements were relatively modest, underscoring that LSTMs and GRUs are already well-tuned architectures with robust performance.

\begin{figure}[H]
	\centering
	\includegraphics[width=0.8\textwidth]{Figures/Chapter_16/slide_101.jpg}
	\caption{Neural Architecture Search (NAS) applied to recurrent neural network architectures, showcasing evolutionary exploration of candidate designs (adapted from \cite{zoph2017_nas}).}
	\label{fig:chapter16_nas_rnn}
\end{figure}

\newpage
\subsection{Summary of RNN Architectures}

Throughout this chapter, we have explored key architectures developed to overcome challenges inherent to vanilla RNNs, particularly vanishing and exploding gradients:

\begin{itemize}
	\item \textbf{Vanilla RNNs}: Introduced fundamental recurrence, but are significantly constrained by unstable gradients, limiting their practical effectiveness for capturing long-term dependencies.
	\item \textbf{LSTMs and GRUs}: Revolutionized recurrent architectures by employing gating mechanisms and additive state updates, improving gradient stability and long-range dependency modeling:
	\begin{itemize}
		\item \textbf{LSTMs}: Offer explicit gating (input, forget, and output) and an additive cell state, making them robust in capturing intricate and long-term patterns.
		\item \textbf{GRUs}: Combine gating mechanisms into fewer components, providing computational efficiency and strong performance on many tasks, especially in limited-data or resource-constrained environments, albeit with slightly reduced representational flexibility compared to LSTMs.
	\end{itemize}
	\item \textbf{Gradient management}: Exploding gradients are effectively controlled by gradient clipping, whereas vanishing gradients are mitigated through gating and additive updates introduced by LSTMs and GRUs.
\end{itemize}

\begin{figure}[H]
	\centering
	\includegraphics[width=0.8\textwidth]{Figures/Chapter_16/rnn_lstm_gru.png}
	\caption{Comparison of vanilla RNN, LSTM, and GRU architectures. Source: \cite{Mahadi2024_GRU_Plasmonic}.}
	\label{fig:chapter16_rnn_lstm_gru_comparison}
\end{figure}

\subsection{Beyond RNNs: From Recurrence to Attention}

Although gating mechanisms greatly enhanced sequence modeling, recurrent architectures inherently rely on sequential computations, making parallelization difficult and hindering performance on extremely long sequences.

Recent developments have introduced attention mechanisms, particularly the \textbf{Transformer architecture}~\cite{vaswani2017_attention}, which eliminate recurrence altogether.
Transformers employ multi-head self-attention, enabling parallel processing and more effectively capturing extensive contextual relationships within data sequences.
This represents a significant advancement, improving both modeling capabilities and computational efficiency.

In the upcoming chapter, we will delve deeply into attention and Transformer architectures, exploring how they address the limitations of RNN-based models and achieve state-of-the-art results in a broad range of sequence modeling tasks.




\chapterimage{head2.png} % Chapter heading image

% Chapter-specific content starts here
\chapter{Lecture 17: Attention}

%----------------------------------------------------------------------------------------
%	CHAPTER 17 - Lecture 17: Attention
%----------------------------------------------------------------------------------------

\chapterimage{head2.png} % Chapter heading image

% Chapter-specific content starts here
\chapter{Lecture 18: Vision Transformers}

%----------------------------------------------------------------------------------------
%	CHAPTER 18 - Lecture 18: Vision Transformers
%----------------------------------------------------------------------------------------


\chapterimage{head2.png} % Chapter heading image

% Chapter-specific content starts here
\chapter{Lecture 19: Generative Models I}

%----------------------------------------------------------------------------------------
%	CHAPTER 19 - Lecture 19: Generative Models I
%----------------------------------------------------------------------------------------

\chapterimage{head2.png} % Chapter heading image

% Chapter-specific content starts here
\chapter{Lecture 20: Generative Models II}

%----------------------------------------------------------------------------------------
%	CHAPTER 20 - Lecture 20: Generative Models II 
%----------------------------------------------------------------------------------------


\chapterimage{head2.png} % Chapter heading image

% Chapter-specific content starts here
\chapter{Lecture 21: Visualizing Models \& Generating Images}

%-----------------------------------------------------------------------
%	CHAPTER 21 - Lecture 21: Visualizing Models and Generating Images
%-----------------------------------------------------------------------

\noindent
Understanding the internal representations and decision processes of convolutional neural networks (CNNs) is critical for both interpretability and model development. This chapter surveys a variety of methods to visualize and analyze CNNs, structured thematically and chronologically. We begin with low-level filters and progress to abstract representations, saliency maps, and generative image synthesis techniques. Each section corresponds to landmark contributions in the field.

\section{Visualizing Layer Filters}
\label{sec:chapter21_first_layer_filters}

\subsection{Visualizing First Layer Filters}
\noindent
Convolutional neural networks (CNNs) learn hierarchical representations from image data, with early layers detecting local visual patterns and deeper layers progressively capturing higher-level abstractions. To build intuition about how CNNs process visual inputs, it is instructive to begin by examining the learned filters in the first convolutional layer across several canonical architectures.

\paragraph{Architecture Comparison}
The first convolutional layer in a CNN directly operates on the input image in its raw RGB form. Consequently, the learned filters in this layer have a natural and interpretable structure. For standard 3-channel color images, the first-layer filter weights have shape
\[
C_{\text{out}} \times 3 \times K \times K,
\]
where \( C_{\text{out}} \) is the number of output channels (i.e., the number of learned filters), and \( K \) is the spatial size of each filter (typically 7 or 11). Each individual filter thus consists of three \( K \times K \) slices—one per color channel—which can be stacked and visualized as a small RGB image. This design enables direct visual inspection of the filters to understand what patterns they are tuned to detect.

\begin{itemize}
	\item \textbf{AlexNet}~\cite{krizhevsky2012_alexnet}: \(64 \times 3 \times 11 \times 11\)
	\item \textbf{ResNet-18 / ResNet-101}~\cite{he2016_resnet}: \(64 \times 3 \times 7 \times 7\)
	\item \textbf{DenseNet-121}~\cite{huang2018_densenet}: \(64 \times 3 \times 7 \times 7\)
\end{itemize}

\newpage
\noindent
Despite significant differences in depth and architectural design, these models exhibit a consistent phenomenon: the filters in the first layer resemble edge detectors, oriented bars, Gabor-like wavelets, and color-sensitive blobs. These are closely aligned with known properties of early-stage neurons in the mammalian visual cortex.

\begin{figure}[H]
	\centering
	\includegraphics[width=0.8\textwidth]{Figures/Chapter_21/slide_7.jpg}
	\caption{Visualization of first-layer convolutional filters from AlexNet, ResNet-18, and DenseNet-121. Each filter is represented as a color image of shape \(3 \times K \times K\), revealing sensitivity to edges, orientations, and color gradients.}
	\label{fig:chapter21_first_layer_filters}
\end{figure}

\paragraph{Interpretation and Limitations}
Because the filters in the first layer operate directly on the input image, they can be visualized straightforwardly as \(K \times K\) RGB patches, with each of the 3 channels contributing to a color composite. This makes the first layer particularly interpretable: we can "see" what the model is looking for, such as vertical edges, red-green opponency, or high-frequency textures.

However, such visualizations only offer a limited glimpse into the model's behavior. These filters detect purely local features and do not account for broader spatial context or higher-order semantics. As a result, visualizing the first layer alone provides only shallow insight into the network's full representational power. To understand how a CNN constructs rich hierarchical features—capable of recognizing objects, parts, and categories—we must investigate deeper layers, where features become increasingly abstract and spatially integrated.

This motivates the exploration of visualization techniques for deeper activations and learned embeddings, which we address in the following sections.

\subsection{Visualizing Higher Layer Filters}
\label{sec:chapter21_higher_layer_filters}

\noindent
Understanding what deeper layers in a convolutional neural network (CNN) are learning is a central question in interpretability research. While visualizing the first convolutional layer yields intuitive insights—thanks to its direct correspondence with RGB pixel inputs—the situation becomes markedly more complex in deeper layers. These layers no longer respond to simple visual primitives but to abstract feature combinations, making direct visualization of their filters significantly less interpretable.

\paragraph{Example: ConvNetJS Visualization}
To illustrate this challenge, consider a toy CNN trained on CIFAR-10, as implemented in ConvNetJS~\cite{karpathy_convnetjs}. The below figure visualizes the raw filter weights from the first three convolutional layers of a simple 3-layer architecture:

\begin{itemize}
	\item \textbf{Layer 1:} \(16 \times 3 \times 7 \times 7\) — 16 filters applied to 3-channel RGB inputs.
	\item \textbf{Layer 2:} \(20 \times 16 \times 7 \times 7\) — 20 filters each operating on all 16 feature maps from Layer 2.
	\item \textbf{Layer 3:} \(20 \times 20 \times 7 \times 7\) — 20 filters applied to the 20 feature maps from Layer 3.
\end{itemize}

Each filter in deeper layers is a stack of \(K \times K\) slices—one per input channel—which can be visualized as a set of grayscale images. However, as shown in the below figure, while the first-layer filters exhibit recognizable edge and color-selective patterns, the filters from Layers 2 and 3 appear increasingly disorganized and abstract. This reflects a shift in representation: higher-layer filters are no longer tuned to individual pixel-level structures, but to complex combinations of previously learned features.

\begin{figure}[H]
	\centering
	\includegraphics[width=0.8\textwidth]{Figures/Chapter_21/slide_8.jpg}
	\caption{Raw filter weights from the first three convolutional layers of a ConvNet trained on CIFAR-10. While the first-layer filters display interpretable patterns, deeper filters lack obvious structure, reflecting their abstraction from pixel-level semantics. Visualization source: ConvNetJS~\cite{karpathy_convnetjs}.}
	\label{fig:chapter21_higher_layer_filters}
\end{figure}

\paragraph{Interpretation and Motivation for Indirect Methods}
As we ascend the network hierarchy, each filter becomes sensitive to more intricate spatial compositions—such as repeated textures, curves, object parts, or category-level cues. These higher-order features are not aligned with natural image statistics or directly grounded in the RGB input space. Instead, they are formed by arbitrary nonlinear combinations of earlier-layer activations. Moreover, filters in later layers have larger effective receptive fields, meaning that their responses integrate information across wider regions of the input image.

Consequently, visualizing the raw weights of deeper filters—even as stacks of grayscale slices—fails to reveal their functional role. These weights no longer "look for" simple patterns, but rather detect abstract configurations of patterns-of-patterns. The lack of spatial or semantic alignment makes their interpretation both difficult and potentially misleading.

\newpage
\noindent
To overcome this limitation, later in this chapter we will explore a suite of indirect visualization techniques—such as \emph{activation maximization}, \emph{guided backpropagation}, and \emph{feature inversion}—which help reveal what kinds of inputs actually elicit strong activations in higher-layer units. These methods synthesize or highlight meaningful visual patterns in the input space, offering a more powerful interpretive lens than static filter weights.

Ultimately, this motivates a deeper shift in focus: rather than interpreting filters in isolation, we often care more about the \emph{representations} that the network constructs—especially at its final convolutional layers. These representations are key to the model’s decision-making process and often contain high-level semantic information. In the following parts, we examine these deep features directly and explore how they encode task-relevant signals.

\section{Last Layer Features: Nearest Neighbors, Dimensionality Reduction}
\label{sec:chapter21_fc_features}

\noindent
The final layers of a convolutional neural network (CNN) encode compact, abstract representations that summarize the semantic content of an input image. In classification models such as AlexNet~\cite{krizhevsky2012_alexnet}, the penultimate fully connected layer—commonly referred to as \texttt{fc7}—produces a 4096-dimensional feature vector. This layer serves as a bottleneck, compressing spatial and visual information into a high-level descriptor that feeds into the final classifier. Understanding these abstract representations is crucial for interpreting model behavior and enables a range of downstream applications beyond classification, including retrieval, clustering, and transfer learning.

\subsection{Semantic Similarity via Nearest Neighbors}
One intuitive technique to probe the quality of learned features is to compute nearest neighbors in the last layer's (pre SoftMax) feature space. For a given query image, we extract its feature vector from the last layer (pre SoftMax) and compare it to feature vectors from a training set using \(\ell_2\) distance. This produces a ranked list of semantically similar images.

\begin{figure}[H]
	\centering
	\includegraphics[width=0.8\textwidth]{Figures/Chapter_21/slide_10.jpg}
	\caption{Comparison of nearest neighbors for a test image. \textbf{Left:} Retrieval in raw pixel space, which is sensitive to visual noise and low-level similarity. \textbf{Right:} Retrieval in the last layer's feature space, which captures object-level semantics such as shape, class, and pose. Figure adapted from~\cite{krizhevsky2012_alexnet}.}
	\label{fig:chapter21_nn_fc7}
\end{figure}

\newpage
\noindent
As can be seen in figure \ref{fig:chapter21_nn_fc7}: Unlike raw pixel-space retrieval—which is highly sensitive to low-level variations such as lighting, background, or slight translations—feature-space retrieval captures more robust and semantically meaningful similarities. The features abstract away irrelevant appearance differences and instead emphasize object identity, pose, and scene context. This property makes them especially valuable for applications like image search and dataset exploration.

\subsection{Dimensionality Reduction and Embedding Visualization}
\label{sec:chapter21_fc_features_dim_red}

\noindent
Convolutional neural networks learn to represent images in high-dimensional feature spaces—often with hundreds or thousands of dimensions. While these abstract representations are powerful for classification, similarity, and downstream tasks, they remain difficult to interpret directly. As humans, our intuition and perception are fundamentally limited to low-dimensional spaces like 2D and 3D. To bridge this gap, we turn to \emph{dimensionality reduction} techniques that project complex feature vectors into interpretable lower-dimensional spaces.

By visualizing these projections, we can uncover how the network organizes inputs: which images cluster together, what semantic properties are preserved, and where decision boundaries might lie. Such visualizations are not only useful for understanding a model’s behavior but also for identifying failure cases, detecting dataset biases, evaluating the quality of artificially generated or augmented data, and exploring representation similarity between tasks.

Two widely used techniques for this purpose are:

\begin{itemize}
	\item \textbf{Principal Component Analysis (PCA):} A deterministic, linear projection method that identifies directions of maximum variance in the data. By projecting onto the leading principal components, PCA reduces dimensionality while preserving as much variability as possible. It is particularly effective at capturing global structure and identifying coarse axes of variation such as illumination, scale, or pose. For a practical and intuitive introduction to PCA including Python code and visual demonstrations, we highly recommend \href{https://www.youtube.com/watch?v=gXbThCXjZFM&list=PLMrJAkhIeNNSVjnsviglFoY2nXildDCcv}{Steve Brunton’s excellent video series on Singular Value Decomposition and PCA} from the University of Washington.
	
	\begin{figure}[H]
		\centering
		\includegraphics[width=0.65\textwidth]{Figures/Chapter_21/slide_12.jpg}
		\caption{A 2D t-SNE visualization of feature vectors extracted from the final FC layer of a CNN trained on ImageNet. Each point represents a test image, positioned such that nearby points in the plot correspond to images with similar features. This nonlinear embedding preserves local neighborhoods, revealing the semantic organization of the learned representation space. For instance, in the bottom left, flower images form a coherent cluster that transitions smoothly into butterflies, illustrating how the network encodes visual similarity. Figure adapted from~\cite{maaten2008_tsne, krizhevsky2012_alexnet}.}
		\label{fig:chapter21_tsne_projection}
	\end{figure}
	
	\item \textbf{t-Distributed Stochastic Neighbor Embedding (t-SNE):} A nonlinear, stochastic technique designed to preserve local structure in the data. Unlike PCA, t-SNE focuses on maintaining neighborhood relationships, often revealing tight semantic clusters that align with object categories, poses, or contextual cues~\cite{maaten2008_tsne}. While t-SNE may distort global distances, it is extremely effective at uncovering fine-grained groupings. For an in-depth and accessible overview of t-SNE’s mechanisms and limitations, see \href{https://medium.com/@sachinsoni600517/mastering-t-sne-t-distributed-stochastic-neighbor-embedding-0e365ee898ea}{this annotated guide}.
\end{itemize}

\noindent
While PCA is computationally efficient and preserves global geometry, it often fails to expose local semantic clusters. In contrast, t-SNE is specifically tailored to emphasize local similarities, often revealing latent category structure—but at the cost of distorting distances between distant points and lacking run-to-run consistency. Both methods offer complementary perspectives on the underlying feature space and are often used in tandem for exploratory analysis.

\begin{figure}[H]
	\centering
	\includegraphics[width=0.35\textwidth]{Figures/Chapter_21/apples_tomatos_confusion_tsne.jpg}
	\caption{A t-SNE visualization of image embeddings generated by Stable Diffusion. Each point represents a high-dimensional image embedding projected into 2D space. Notably, several red apples are embedded close to tomatoes, likely due to visual similarity in shape and color (both being red and round). This kind of confusion highlights how the model organizes its internal representation space and helps diagnose classification ambiguity. Such insights can inform improvements like augmenting the training set, refining class definitions, or modifying the architecture to better separate semantically similar classes. Visualization adapted from~\cite{paepper2023_sdembeddingviz}.}
	\label{fig:chapter21_apples_tomatoes_tsne}
\end{figure}

\paragraph{Interpretation and Applications}
High-level representations such as those from a CNN's final layers encode rich semantic attributes—ranging from object identity and viewpoint to scene layout and contextual cues. Distances in this feature space often correlate well with human judgments of visual similarity, enabling a wide array of practical applications:

\begin{itemize}
	\item \textbf{Image retrieval:} Finding visually or semantically similar images in large datasets.
	\item \textbf{Dataset visualization:} Exploring the structure of labeled or unlabeled image corpora.
	\item \textbf{Anomaly detection:} Identifying outliers or data points poorly aligned with learned manifolds.
	\item \textbf{Unsupervised clustering:} Automatically grouping inputs by their learned feature similarity.
	\item \textbf{Synthetic data evaluation:} Comparing simulated or augmented images to real examples.
\end{itemize}

\noindent
Moreover, these high-level embeddings are often transferable: features learned on large classification datasets can be reused in novel tasks or domains with minimal fine-tuning. This principle—central to transfer learning—demonstrates that the representations captured by deep networks are not only powerful, but also generalizable.

\vspace{1em}
\noindent
In sum, dimensionality reduction provides a crucial bridge between abstract neural representations and human-understandable intuition. Through methods like PCA and t-SNE, we can glimpse how deep models internally organize visual information—and use these insights to refine, audit, and better exploit our models.

\section{Visualizing Activations and Maximally Activating Patches}
\label{sec:chapter21_activations_max_patches}

\noindent
Inspecting the learned weights of a CNN provides a static snapshot of the model’s potential for pattern detection—its architectural “blueprint” for recognizing features. However, to understand what the network \emph{actually does} when processing a specific input, we often examine its \emph{activations}—the dynamic feature maps produced as the image flows forward through the network. These activations reveal which filters fire, and crucially, \emph{where} in the input those activations occur. In this sense, weights describe capacity; activations reveal behavior.

\paragraph{How to Visualize Activations}

Each convolutional layer produces a 3D tensor of shape \( C \times H \times W \), where:
\begin{itemize}
	\item \( C \) is the number of filters (channels),
	\item \( H \times W \) are the spatial dimensions of each activation map.
\end{itemize}

Each 2D slice \( A_c \in \mathbb{R}^{H \times W} \) reflects how strongly the \( c^\text{th} \) filter responds at each spatial location. To visualize a specific activation map:

\begin{enumerate}
	\item \textbf{Select a channel} \( c \in \{1, \dots, C\} \) to isolate one filter’s response.
	
	\item \textbf{Extract the 2D activation map} \( A_c \in \mathbb{R}^{H \times W} \) corresponding to that filter.
	
	\item \textbf{Normalize the values} to the display range \([0, 255]\), typically via min–max scaling:
	\[
	A'_c = 255 \cdot \frac{A_c - \min(A_c)}{\max(A_c) - \min(A_c)}
	\]
	This step is crucial because activations are real-valued and unbounded—some may be small or near-zero (especially due to ReLU), while others are large. Without normalization, these differences would be visually imperceptible.
	
	\item \textbf{Render the result} as a grayscale image or heatmap. Heatmaps often provide richer visual detail by using color gradients to emphasize intensity differences, making strong responses immediately apparent.
\end{enumerate}

\begin{figure}[H]
	\centering
	\includegraphics[width=0.8\textwidth]{Figures/Chapter_21/slide_13.jpg}
	\caption{Example activations from the \texttt{conv5} layer of a CNN. Each grayscale patch shows the activation map of a single filter. Brighter regions correspond to stronger activations. The predominance of dark areas arises from two key effects: (1) the use of ReLU, which sets all negative pre-activation values to zero, producing sparse feature maps; and (2) the visualization step, which rescales each map—originally containing real-valued outputs from \(-\infty\) to \(+\infty\)—into the \([0, 255]\) display range. When most values are near zero, this rescaling flattens the output, making subtle responses appear uniformly dark. Figure adapted from~\cite{yosinski2015_deepviz}.}
	\label{fig:chapter21_conv5_activations}
\end{figure}

\noindent
As we move deeper into a convolutional network, filters become increasingly selective—activating only in response to highly specific patterns or abstract visual concepts. Their corresponding activation maps tend to be sparse: most values are zero due to the use of ReLU, which clips all negative responses. Even among nonzero activations, only a few regions typically "light up." This sparsity can be beneficial for interpretability, but also poses challenges: min–max normalization of nearly-zero maps can exaggerate noise or suppress subtle, yet semantically meaningful signals. These artifacts must be considered when interpreting visualizations.

\paragraph{Why Do Activation Maps Reveal Spatial Information?}

Despite their abstract nature, convolutional activations retain spatial structure. Each filter is applied across an input feature map using \emph{weight sharing}—the same filter is convolved at every spatial position. But the output varies based on the local content in the receptive field at each position. Thus, each activation value \( A_c[h, w] \) reflects how well the filter \( c \) matched the input region centered at position \( (h, w) \).

Visualizing these 2D slices (e.g., a single \(13 \times 13\) map from a \(128 \times 13 \times 13\) tensor) highlights the spatial pattern of a filter’s response. These patterns can be overlaid on the input (via the effective receptive field~\cite{luo2017_understanding_receptive_field}) to estimate \emph{where} in the original image the filter was activated. However, deeper layers have increasingly large receptive fields, making it harder to attribute activations to specific image structures. As a result, spatial precision decreases, and interpretability becomes less reliable the further we advance. 

\paragraph{What Do Activations Reveal?}

Activation maps offer a structured way to investigate a CNN’s learned internal representations. Specifically, they let us answer:

\begin{itemize}
	\item \textbf{Where (spatial localization):} Because convolution preserves spatial arrangement, the activation map shows \emph{where} in the input each filter fired. This provides coarse localization for the visual features the filter detects, even without class-specific attribution methods.
	
	\item \textbf{What (feature selectivity):} By examining the activation maps across many images, we can \textbf{try to infer} the \emph{type of feature} a filter has learned to recognize. Early layers might detect edges or color gradients~\cite{zeiler2014_visualizing}, while later layers respond to higher-order patterns like textures, object parts, or semantic categories~\cite{yosinski2015_deepviz, bau2017_network_dissection}.
	
	\item \textbf{When (context sensitivity):} Activation strength across varied inputs reveals \emph{when} a filter activates. Some filters are robust—firing across poses, lighting conditions, or backgrounds—while others activate only under specific circumstances. This sensitivity can indicate generalization strength or overfitting to spurious correlations.
\end{itemize}

\noindent
ReLU activation plays a central role in this analysis: it promotes sparsity, simplifying interpretation by highlighting only strong, positive responses. But it also discards all negative values—even those with large magnitude—thereby removing potentially informative inhibitory signals. To recover this information, one may inspect pre-ReLU activations or experiment with nonzero-centered alternatives such as Leaky ReLU or ELU~\cite{clevert2016_fast_and_accurate}. The trade-off between sparsity and representational richness remains an open research question.

\paragraph{What Can We Do With Activation Maps?}
\label{par:chapter21_what_to_do_activations}

Beyond visual intuition, activation maps support quantitative analysis and practical improvements:

\begin{itemize}
	\item \textbf{Debug unexpected behavior and expose spurious correlations:} If a filter activates consistently in irrelevant regions—e.g., sky, grass, or watermarks—it may signal that the model relies on background cues rather than object-relevant features. For instance, a "cow" detector firing primarily on grass may indicate dataset bias.
	
	\item \textbf{Evaluate feature generality and specialization:} By comparing activation maps across inputs, we can assess whether filters detect broad, reusable patterns (e.g., wheels across vehicle types) or overfit to narrow visual contexts (e.g., specific breeds or lighting). This aids in diagnosing underfitting, overfitting, or insufficient dataset diversity.
	
	\item \textbf{Guide architectural and training improvements:}
	\begin{itemize}
		\item \emph{Pruning:} Filters that remain inactive across most inputs may be redundant and removable.
		\item \emph{Augmentation:} Overly selective filters may indicate a need for targeted data augmentation (e.g., varied viewpoints or occlusions).
		\item \emph{Architecture tuning:} Imbalanced usage of filters across layers may suggest overparameterization or motivate the use of attention, bottlenecks, or depth adjustments.
	\end{itemize}
\end{itemize}

\noindent
However, it’s important to recognize the limits of activation map interpretability, leading us towards further research and hopefully better approaches. We remind the reader that in deeper layers, features become more abstract and distributed, and receptive fields cover large, overlapping regions of the input. As a result, the exact visual trigger for a given activation may no longer be clearly localized. While activation maps remain a powerful tool—especially for understanding early and mid-level representations—their utility diminishes in deeper layers, where methods like class activation mapping (CAM) or feature inversion that we'll cover later become more appropriate.

\newpage
\subsection{Maximally Activating Patches}
\label{subsec:chapter21_max_patches}

\noindent
To gain a more concrete and intuitive understanding of what a convolutional filter has learned, an effective strategy is to examine the image regions that elicit the strongest responses from that filter. This technique, known as the \emph{maximally activating patch} method, identifies the specific visual patterns that most excite a given channel or neuron across a diverse dataset. Rather than inspecting weights or abstract feature maps, we directly observe which natural image patches consistently trigger high activations—revealing the visual motifs the network has internalized.

\paragraph{Methodology}

The process consists of the following steps:

\begin{enumerate}
	\item \textbf{Select a target filter:} Choose a specific convolutional layer and channel (e.g., channel 17 in \texttt{conv5}). This channel acts as a spatially replicated detector for a particular feature across the input.
	
	\item \textbf{Forward a dataset through the network:} Pass a large collection of diverse images through the network. For each image, record the full spatial activation map of the selected channel—i.e., all values \( A_c[h, w] \) at every spatial location.
	
	\item \textbf{Aggregate and rank responses:} Collect all activation values from all images and spatial positions. Identify the top-K strongest activations globally—these are the spatial locations and images where the filter responded most strongly across the dataset.
	
	\item \textbf{Extract corresponding patches:} For each of these peak activations, compute the receptive field in the original input image that led to the activation. This mapping depends on the layer’s position in the network (e.g., kernel size, stride, padding, pooling). Extract that region from the input—this is the maximally activating patch for the neuron.
\end{enumerate}

\begin{figure}[H]
	\centering
	\includegraphics[width=0.8\textwidth]{Figures/Chapter_21/slide_14.jpg}
	\caption{Maximally activating input patches for various neurons in a CNN. Each row shows patches from different input images that produced high activations for a specific neuron. These patches often reveal consistent visual motifs—such as specific textures, faces, or object parts—suggesting that the neuron has become specialized for detecting that pattern. Figure adapted from~\cite{springenberg2015_allconv}.}
	\label{fig:chapter21_max_patches}
\end{figure}

\noindent
This method is most directly interpretable in fully convolutional architectures, where each spatial location in an activation map corresponds to a fixed receptive field in the input image. The spatial correspondence in CNNs enables straightforward mapping from filter activations back to the regions of the input that caused them. While exact pixel-level alignment can be imprecise—due to overlapping receptive fields, nonlinear activations, pooling layers, and stride effects—the extracted patches still reliably capture the \emph{dominant local visual stimulus} that triggered the filter’s response. As such, they offer a concrete and intuitive glimpse into what the neuron has learned to detect.

\paragraph{Intuition and Insights}

This method gives us a dataset-level understanding of what a neuron "looks for". By observing many input patches that excite a filter, we can infer its role in the learned feature hierarchy. For example, some neurons specialize in detecting:

\begin{itemize}
	\item Low-level cues like diagonal edges or color gradients.
	\item Mid-level textures such as mesh, fur, or bricks.
	\item High-level semantics such as eyes, animal faces, or wheels.
\end{itemize}

\noindent
These patterns tend to be consistent across inputs, offering a clear visual prototype of the features the filter has internalized.

Beyond interpretability, this method supports diagnosis:

\begin{itemize}
	\item \textbf{Redundancy:} If many neurons are activated by visually similar patches, it may suggest overparameterization and motivate pruning.
	
	\item \textbf{Dataset bias:} If a neuron fires only when a feature appears in a specific context (e.g., a texture always on green grass), it may indicate reliance on spurious correlations in the training data.
\end{itemize}

\paragraph{From “What It Sees” to “What It Uses”}

\noindent
Taken together, activation maps and maximally activating patches provide complementary perspectives on the behavior of individual filters in a convolutional neural network:

\begin{itemize}
	\item \textbf{Activation maps} show \emph{where} in a specific input image a filter activates. They highlight the spatial regions that match the filter’s learned pattern in that image, offering localized insight into the filter’s response.
	
	\item \textbf{Maximally activating patches} reveal \emph{what} the filter is most tuned to detect. By scanning across many inputs and extracting only the input regions that trigger the strongest responses, this method uncovers the most prototypical visual stimuli associated with that filter—regardless of their spatial position.
\end{itemize}

\noindent
While activation maps offer per-image spatial footprints, maximally activating patches distill a filter’s dataset-wide preferences. The former tells us \emph{where} a filter fires; the latter tells us \emph{what} consistently causes it to fire.

\noindent
However, both techniques are inherently \emph{class-agnostic}. They tell us which features a filter responds to and where it detects them—but not whether those responses contribute positively, negatively, or at all to the model’s final decision for a specific class. In other words, they capture what the network \emph{notices}, but not what it \emph{relies on} to make a prediction.

To move from \emph{feature visualization} to \emph{decision attribution}, we need methods that explicitly measure how changes to specific input regions affect the output logits or class probabilities. This brings us to the domain of \textbf{saliency methods}—a family of techniques designed to reveal which parts of the input were most influential in producing a particular output. We begin this journey with one of the most direct and interpretable approaches: \emph{saliency via occlusion}.

\newpage
\section{Saliency via Occlusion and Backpropagation}
\label{sec:chapter21_saliency}

\noindent
Saliency methods aim to identify which parts of an input image are most influential for a model’s prediction. These techniques produce spatial or pixel-level explanations by quantifying how sensitive the model’s output is to perturbations in the input—highlighting the regions that contribute most strongly to the predicted class.

\subsection{Occlusion Sensitivity}
\label{subsec:chapter21_occlusion_sensitivity}

\noindent
A natural way to probe a model's decision is to ask: “What happens if we hide part of the input?” \emph{Occlusion sensitivity}~\cite{zeiler2014_visualizing} follows this idea in a simple, model-agnostic manner. It systematically occludes small patches of the input image—one at a time—and measures how the predicted class confidence changes. If masking a region leads to a significant drop in confidence, that region likely contains features critical to the model's decision.

\begin{figure}[H]
	\centering
	\includegraphics[width=0.8\textwidth]{Figures/Chapter_21/slide_16.jpg}
	\caption{Occlusion-based saliency maps~\cite{zeiler2014_visualizing}. Each image patch is occluded in turn, and the drop in class confidence is recorded. \textbf{Darker regions indicate locations where occlusion most reduced the model's confidence}, corresponding to spatial regions that were most important to the prediction.}
	\label{fig:chapter21_occlusion_saliency}
\end{figure}

\paragraph{Methodology}

\begin{enumerate}
	\item \textbf{Define occlusion patches:} Subdivide the input image into a grid of fixed-size patches (e.g., \(15 \times 15\) pixels), optionally using stride \(s\) to control overlap.
	
	\item \textbf{Mask one patch at a time:} Replace each patch with a neutral baseline—commonly a gray square, blurred patch, or zero-valued block—to simulate information removal.
	
	\item \textbf{Measure impact on prediction:} For each occluded image, run a forward pass through the model and compute the change in the predicted confidence score for the target class. The larger the drop, the more critical the masked region is assumed to be.
\end{enumerate}

\paragraph{From Patch Scores to Pixel-Level Saliency}

This process yields a grid of scalar values, each representing the change in confidence caused by occluding a particular patch. To convert this into a smooth, pixel-level saliency map:

\begin{itemize}
	\item Assign each patch’s score to the pixels it covered—either uniformly or centered.
	\item If patches overlap, aggregate the contributions to each pixel by summing or averaging over all occlusions affecting that location.
	\item Interpolate the resulting low-resolution grid to match the full image resolution, optionally applying smoothing to reduce blocky artifacts.
\end{itemize}

\noindent
The result is a dense saliency map that visually highlights which regions of the input the model relied on most for its prediction. Note that in many visualizations, \textbf{darker areas} indicate stronger drops in confidence—i.e., occlusions that had the most damaging effect on classification. These regions are interpreted as the \emph{most influential} for the model’s decision.

\paragraph{Intuition and Interpretation}

Occlusion sensitivity provides a direct, human-interpretable diagnostic: “If I hide this part of the input, does the model still know what it’s looking at?” By observing how the model's confidence changes in response to occlusions, we infer which parts of the input are functionally necessary for the current prediction. Unlike gradient-based saliency, this technique does not rely on access to internal parameters or derivatives—it simply perturbs the input and watches how the output responds.

\begin{enrichment}[Advantages and Limitations of Occlusion Sensitivity][subsubsection]
	
	\textbf{Advantages:}
	\begin{itemize}
		\item \emph{Model-agnostic:} Requires no access to model weights or gradients. Works with any black-box model.
		\item \emph{Conceptually intuitive:} Its interpretation is straightforward and visual—important regions are those whose absence hurts confidence.
		\item \emph{Localized insight:} Provides evidence tied to specific input regions, often yielding interpretable and faithful attributions.
	\end{itemize}
	
	\vspace{0.5em}
	\textbf{Limitations:}
	\begin{itemize}
		\item \emph{Computational cost:} Requires one forward pass per patch. For high-resolution images or small stride, this becomes expensive.
		\item \emph{Resolution trade-off:} Smaller patches improve spatial granularity but increase the number of required evaluations.
		\item \emph{Out-of-distribution masking:} Large or abrupt occlusions may produce unrealistic inputs, potentially confusing the model.
		\item \emph{Mask design bias:} The choice of occlusion value (e.g., gray vs. blur vs. noise) can significantly affect results.
	\end{itemize}
	
\end{enrichment}

\subsection{Saliency via Gradient Backpropagation}
\label{subsec:chapter21_saliency_backprop}

A more efficient method is to compute the gradient of the output score with respect to the input pixels~\cite{simonyan2014_deepinside}. This produces a \emph{saliency map} where each pixel’s intensity corresponds to the magnitude of its influence:
\[
M_{i,j} = \max_{c \in \{R,G,B\}} \left| \frac{\partial S_y}{\partial I_{i,j,c}} \right|
\]
Here, \(S_y\) is the unnormalized score for the predicted class \(y\), and \(I_{i,j,c}\) is the pixel value at location \((i,j)\) and channel \(c\). The result is a single-channel map that highlights pixels with the highest influence on the model’s decision.

\begin{figure}[H]
	\centering
	\includegraphics[width=0.7\textwidth]{Figures/Chapter_21/slide_18.jpg}
	\caption{Gradient-based saliency map~\cite{simonyan2014_deepinside}: pixel importance is estimated by computing the gradient of the class score with respect to each input pixel. Brighter regions correspond to pixels where small changes most strongly influence the model's output for the predicted class. In this example, the saliency map highlights the dog's shape—indicating that the network's decision relies on semantically meaningful regions of the input image.}
	\label{fig:chapter21_gradient_saliency}
\end{figure}

\paragraph{Interpretation and Use Cases}
Gradient-based saliency maps are useful for:
\begin{itemize}
	\item \textbf{Localizing object evidence:} Which pixels most support the class prediction?
	\item \textbf{Debugging dataset bias:} Are predictions based on background cues or spurious features?
	\item \textbf{Comparing models:} How do different architectures attend to input?
\end{itemize}

Despite their appeal, gradient saliency maps can be noisy and sensitive to model initialization and ReLU saturation.

\paragraph{Towards Unsupervised Segmentation}

\begin{figure}[H]
	\centering
	\includegraphics[width=0.7\textwidth]{Figures/Chapter_21/slide_20.jpg}
	\caption{Foreground extraction via GrabCut applied to saliency maps. Although no explicit segmentation labels are used, the resulting masks capture object shapes such as birds, snakes, and insects—indicating that CNN attention aligns well with perceptually salient regions. Figure adapted from~\cite{simonyan2014_deepinside}.}
	\label{fig:chapter21_grabcut_saliency}
\end{figure}

\noindent
As illustrated in Figure~\ref{fig:chapter21_grabcut_saliency}, Simonyan et al.~\cite{simonyan2014_deepinside} demonstrated that applying classical segmentation algorithms such as GrabCut~\cite{rother2004_grabcut} to gradient-based saliency maps can yield surprisingly accurate foreground-background separation—even though the network was trained solely for classification. This finding underscores the spatial coherence of CNN attention: regions that most influence class predictions often align well with object boundaries.

While these examples are often cherry-picked and performance may vary considerably across diverse images and classes, the result is nonetheless striking. It suggests that high-level classification networks can implicitly acquire useful spatial priors, hinting at their potential for weakly supervised or unsupervised segmentation—without access to any ground-truth masks.

\section{Guided Backpropagation of Intermediate Features}
\label{sec:chapter21_guided_backprop}

\noindent
Gradients are typically used to update model weights during training—but they can also be repurposed for interpretability. While basic saliency maps compute the gradient of the class score with respect to input pixels, here we extend that idea: instead of analyzing the output neuron, we ask what causes a specific \emph{intermediate neuron} to activate. This allows us to inspect what internal features a CNN is detecting at various depths.

\subsection{Backpropagation to Visualize Intermediate Neurons}
Given an input image, we forward it through the network and pause at an intermediate convolutional layer (e.g., \texttt{conv5}). This layer outputs a tensor of shape \( C \times H \times W \)—where \( C \) is the number of channels (filters), and \( H \times W \) is the spatial resolution. We select a specific neuron, defined by its channel index \( c \) and spatial location \( (h, w) \), and compute the gradient of that neuron's activation with respect to the input image:

\[
\frac{\partial A_{c, h, w}}{\partial I}
\]

\noindent
This tells us which pixels in the input image most strongly influence the activation of that individual neuron. Intuitively, it shows the pattern the neuron is "looking for"—that is, what input changes would most affect that neuron's response.

\noindent
Visualizing raw gradients of the class score with respect to input pixels often yields noisy and unintelligible maps. This noise arises primarily from how gradients propagate through nonlinear activation functions like ReLU. In particular, irrelevant negative signals can be amplified or useful contributions canceled, obscuring the underlying patterns that truly drive the model's decision.

\subsection{Guided Backpropagation: Cleaner Gradient Visualizations}
\label{subsec:chapter21_guided_backprop}

\noindent
To address this issue, \emph{guided backpropagation}~\cite{springenberg2015_allconv} modifies the backward pass through ReLU layers to suppress uninformative gradients and emphasize relevant, excitatory input patterns. It introduces an additional masking rule during backpropagation:

\begin{itemize}
	\item \textbf{Forward pass:} ReLU behaves as usual, blocking all negative activations: \( f(x) = \max(0, x) \).
	
	\item \textbf{Standard backward pass:} Gradients are blocked if the forward activation was non-positive. That is, if \( x \le 0 \), then \( \frac{\partial L}{\partial x} = 0 \).
	
	\item \textbf{Guided backpropagation backward pass:} Gradients are blocked unless \emph{both} the forward activation \( x \) and the backward gradient \( \frac{\partial L}{\partial y} \) are positive. This can be expressed as:
	\[
	\frac{\partial L}{\partial x} = \begin{cases}
		\frac{\partial L}{\partial y} & \text{if } x > 0 \text{ and } \frac{\partial L}{\partial y} > 0 \\
		0 & \text{otherwise}
	\end{cases}
	\]
\end{itemize}

\begin{figure}[H]
	\centering
	\includegraphics[width=0.8\textwidth]{Figures/Chapter_21/slide_22.jpg}
	\caption{Comparison of gradient flow in standard backpropagation vs. guided backpropagation. The latter only allows gradients to pass through ReLU units when both the activation and incoming gradient are positive—resulting in sharper and more interpretable saliency maps. Figure adapted from~\cite{springenberg2015_allconv}.}
	\label{fig:chapter21_guided_backprop_masking}
\end{figure}

\paragraph{Why Does This Help? Intuition and Impact}
\noindent
The exact reasons remain somewhat speculative. What we can empirically deduce is that guided backpropagation appears to suppress noisy or ambiguous gradient signals—particularly those that inhibit neuron activation. By allowing only positive gradients to flow through positively activated neurons, it emphasizes features that \emph{excite} the network, positively supporting the prediction, filtering out suppressive or indirect influences. 

\subsection{Visualizing Intermediate Feature Detectors}

\begin{figure}[H]
	\centering
	\includegraphics[width=0.7\textwidth]{Figures/Chapter_21/slide_24.jpg}
	\caption{Visualizing intermediate features using guided backpropagation. Each row corresponds to one neuron. \textbf{Left:} Input patches from the dataset that maximally activated the neuron. \textbf{Right:} Guided backpropagation visualizations showing which pixels in the patch most contributed to the activation.}
	\label{fig:chapter21_guided_backprop_neurons}
\end{figure}

\noindent 
Guided backpropagation enables us to generate clear and interpretable visualizations of what each intermediate neuron responds to. For a given neuron and image, the resulting gradient map highlights which pixels in that input most strongly \emph{support} that neuron's activation.

\noindent
This technique complements the \emph{maximally activating patch} method (~\ref{sec:chapter21_activations_max_patches}):
\begin{itemize}
	\item \textbf{Maximal patches} show what kinds of image regions trigger a neuron across the dataset.
	\item \textbf{Guided backpropagation} shows which pixels within a specific image were responsible for that neuron's response.
\end{itemize}

\noindent
Together, they offer both a global and local perspective on what each neuron represents: global preferences from real data, and fine-grained pixel attributions within those preferred regions.

\paragraph{From Saliency to Synthesis}
So far, our methods have analyzed fixed input images—identifying which parts contributed most to a class prediction or neuron activation. But we can ask a more ambitious question: \emph{What image—real or synthetic—would maximally activate a given neuron?} Instead of attributing importance within a fixed input, we aim to generate an image from scratch that embodies the neuron’s ideal stimulus. This leads to the next technique: \textbf{gradient ascent visualization}, where we synthesize images by directly optimizing the input to activate specific neurons.

\section{Gradient Ascent and Class Visualization}
\label{sec:chapter21_gradient_ascent}

\noindent
Instead of inspecting how a fixed image influences a network’s prediction, we now flip the process: \emph{we synthesize an input image that maximally activates a specific neuron}. This neuron can reside in the final classification layer or any intermediate feature map. The network is treated as a frozen function \(f(I)\) with respect to input image \(I\), and our goal is to optimize the image itself.

\paragraph{Objective Function}
Formally, we wish to find an image \(I^{\ast}\) that maximizes a scalar objective function \(f(I)\)—for example, the activation of a specific neuron—while also maintaining visual plausibility:
\[
I^{\ast} = \arg \max_{I} \underbrace{f(I)}_{\text{neuron activation}} + \underbrace{R(I)}_{\text{image prior regularization}}
\]
Here:
\begin{itemize}
	\item \(f(I)\) is the value of the neuron we wish to maximize.
	\item \(R(I)\) is a regularization term that encourages the image to resemble a natural input.
\end{itemize}

\paragraph{Optimization via Gradient Ascent}
To optimize this objective, we perform gradient ascent directly on the image:
\begin{enumerate}
	\item Initialize \(I \leftarrow 0\) (e.g., a zero image).
	\item Forward propagate \(I\) through the network to compute \(f(I)\).
	\item Backpropagate \(\partial f(I)/\partial I\) to obtain gradients w.r.t. the input pixels.
	\item Update the image: \(I \leftarrow I + \eta \cdot \nabla_I \left( f(I) + R(I) \right)\)
\end{enumerate}

\begin{figure}[H]
	\centering
	\includegraphics[width=0.8\textwidth]{Figures/Chapter_21/slide_41.jpg}
	\caption{Gradient ascent loop: we synthesize an image \(I\) to maximize neuron output by repeatedly updating it with the gradient \(\nabla_I f(I)\).}
	\label{fig:chapter21_gradient_ascent_loop}
\end{figure}

\subsection{Regularization: Making Images Look Natural}
Without regularization, the optimization typically yields noisy, high-frequency artifacts. A simple baseline is \textbf{\(\ell_2\)-norm regularization}~\cite{simonyan2014_deepinside}:
\[
R(I) = -\lambda \|I\|_2^2
\]
This penalizes pixel magnitudes and encourages smoother patterns.

\begin{figure}[H]
	\centering
	\includegraphics[width=0.8\textwidth]{Figures/Chapter_21/slide_43.jpg}
	\caption{Synthetic images generated by optimizing the input to maximally activate specific output neurons (e.g., \texttt{dumbbell}, \texttt{dalmatian}), using simple \(\ell_2\) regularization to encourage smoothness. Distinct visual features—such as dumbbell handles or dalmatian-like black-and-white spots—emerge in the synthesized inputs, offering insight into the discriminative patterns the network associates with each class. Figure adapted from~\cite{simonyan2014_deepinside}.}
	\label{fig:chapter21_naive_l2_regularization}
\end{figure}

\paragraph{Advanced Regularizers}
We can improve image realism by applying additional constraints:
\begin{itemize}
	\item Apply Gaussian blur during optimization.
	\item Set small pixel values to zero (hard-thresholding).
	\item Set small gradients to zero (gradient masking).
\end{itemize}

These enhancements reduce noise and amplify dominant structures in the image, as shown in the below figure. 

\begin{figure}[H]
	\centering
	\includegraphics[width=0.75\textwidth]{Figures/Chapter_21/slide_46.jpg}
	\caption{Improved results using enhanced regularizers: clear patterns emerge that resemble flamingos, cobras, pelicans, and beetles, according to their respective class synthesized image.}
	\label{fig:chapter21_enhanced_regularization}
\end{figure}

\subsection{Visualizing Intermediate Features}
We can apply the same gradient ascent technique to neurons inside hidden layers. By optimizing \(f(I)\) for a specific feature map location in a middle layer (e.g., \texttt{conv3} or \texttt{conv5}), we uncover abstract texture patterns that these filters specialize in.

\begin{figure}[H]
	\centering
	\includegraphics[width=0.75\textwidth]{Figures/Chapter_21/slide_48.jpg}
	\caption{Neuron visualizations at different layers: eye-like motifs, spider-like webs in layer 5, and red/green blobs in layer 3.}
	\label{fig:chapter21_synthetic_intermediate_patterns}
\end{figure}

\subsubsection{Multifaceted Feature Visualization via Generative Models}

\noindent
Earlier methods synthesized preferred inputs using direct gradient ascent in image space, often constrained by simple regularizers like \(\ell_2\). While interpretable to some degree, these visualizations tended to be noisy and unnatural. To generate more realistic and semantically coherent preferred inputs, Nguyen et al.~\cite{nguyen2016_multifaceted} proposed optimizing within the latent space of a \emph{deep generative model}—such as a GAN or autoencoder—pretrained to produce natural images.

\vspace{0.5em}
\noindent
By searching for latent vectors that produce images which maximally activate a given neuron, this approach yields high-quality visualizations that stay within the data distribution. Crucially, it also enables \emph{multifaceted} analysis: neurons often respond to several distinct visual modes, and these can be surfaced by clustering the top activating examples and synthesizing a prototype for each cluster.

\begin{figure}[H]
	\centering
	\includegraphics[width=0.7\textwidth]{Figures/Chapter_21/slide_50.jpg}
	\caption{Examples of realistic images synthesized using the multifaceted feature visualization approach such as 'strawberry', 'orange'. Figure adapted from~\cite{nguyen2016_multifaceted}.}
	\label{fig:chapter21_multifaceted}
\end{figure}

\begin{figure}[H]
	\centering
	\includegraphics[width=0.7\textwidth]{Figures/Chapter_21/slide_51.jpg}
	\caption{Examples of realistic images synthesized via a generative model to maximally activate neurons associated with classes like “toaster”, “triumphal arch”, “cellphone”, and others. The approach ensures that each synthesized image remains within the natural image manifold. Figure adapted from~\cite{nguyen2016_multifaceted}.}
	\label{fig:chapter21_multifaceted_modes}
\end{figure}

\paragraph{Realism vs. Fidelity}

\noindent
While generative models produce more interpretable and visually compelling results, they also introduce a strong prior that can bias the outcome. The visualizations may reflect assumptions encoded in the generator rather than the raw, unregularized preferences of the target neuron. Simpler methods—such as direct optimization in pixel space with basic regularizers—may yield less realistic but more faithful views into the network’s internal objectives. This reflects a fundamental trade-off between interpretability and fidelity in feature visualization.

\section{Adversarial Examples: A Deep Dive into Model Vulnerability}
\label{subsec:chapter21_adversarial_examples}

\noindent
Adversarial examples are inputs containing \textbf{subtle, human-imperceptible perturbations} that cause deep neural networks to misclassify with high confidence. This phenomenon reveals a profound vulnerability in the robustness of state-of-the-art AI models, with critical implications for security-sensitive and safety-critical domains such as autonomous driving, medical diagnostics, and facial recognition.

\subsection{Fundamental Attack Mechanisms}

From a technical perspective, the crafting of adversarial examples closely mirrors the gradient-based input optimization techniques introduced earlier for model interpretation—such as saliency maps and preferred input synthesis. However, while those techniques \emph{maximize the activation of a given class} to better understand what the model has learned, adversarial attacks use the same machinery to \emph{intentionally fool the model}.

The generation of an adversarial input typically involves solving an optimization problem to find a minimally perturbed version \( I^{\ast} \) of a given input \( I_{\text{orig}} \) that causes the model to output an incorrect class label. 

\begin{figure}[H]
	\centering
	\includegraphics[width=0.85\textwidth]{Figures/Chapter_21/slide_53.jpg}
	\caption{Adversarial examples: small, visually indistinguishable perturbations can cause drastic misclassifications—e.g., an elephant becomes a koala, and a schooner becomes an iPod.}
	\label{fig:chapter21_adversarial_perturbation}
\end{figure}

\noindent
For a \emph{targeted} attack (aiming to force classification into a specific wrong class \( c' \)), this objective can be formalized as:
\[
I^{\ast} = \arg\max_{I} \left(
\underbrace{S_{c'}(I)}_{\text{target class score}} -
\lambda \cdot
\underbrace{\|I - I_{\text{orig}}\|_\infty}_{\text{perturbation magnitude}}
\right),
\]
where \( S_{c'}(I) \) is the model's output score for the target class \( c' \), and the regularization term \( \lambda \cdot \|I - I_{\text{orig}}\|_\infty \) ensures that the perturbation remains small enough to be imperceptible to humans.

\noindent
In essence, adversarial attacks are a malicious repurposing of the same gradient-based tools we previously used for interpretability—now aimed not at understanding the model, but at exposing its most brittle failure modes.

% -------------------------------------------------------------
\subsection{Taxonomy of Adversarial Attacks}

\paragraph{White-box attacks}  
These attacks assume full "glass-box" visibility into the model (architecture, weights, gradients), enabling highly effective, gradient-based perturbations:

\begin{itemize}
	\item \textbf{FGSM (Fast Gradient Sign Method)}~\cite{goodfellow2015_explaining}: One-step attack. Computes the gradient of the loss w.r.t. input once and perturbs each pixel by \(\epsilon\) in the direction that maximizes loss—fast but often coarse.
	
	\item \textbf{BIM / I-FGSM (Basic / Iterative FGSM)}: Applies FGSM multiple times with smaller step size and clips after each step. This yields more refined perturbations under the same \(\|\cdot\|_\infty \le \epsilon\) constraint.
	
	\item \textbf{PGD (Projected Gradient Descent)}~\cite{madry2018_towards}: Widely regarded as the \emph{strongest first-order adversary}, PGD is a principled, iterative attack that maximizes model loss while strictly enforcing an imperceptibility constraint on the perturbation. It is a cornerstone of both adversarial evaluation and adversarial training~\cite{madry2018_towards}.
	
	\begin{itemize}
		\item \emph{Random Initialization within the \(\epsilon\)-ball}: Unlike FGSM or BIM, PGD begins not from the original input \( x_{\text{orig}} \), but from a randomly chosen point within the \(\ell_\infty\)-ball of radius \( \epsilon \). This random initialization allows the attack to explore more directions in the loss landscape, increasing the chance of escaping local optima and locating truly worst-case perturbations.
		
		\item \emph{Gradient Ascent + Projection Loop}: PGD performs multiple steps of gradient ascent on the loss:
		\[
		x^{(t+1)} = \Pi_{\mathcal{B}_\infty(x_{\text{orig}}, \epsilon)} \left( x^{(t)} + \alpha \cdot \text{sign} \left( \nabla_x \mathcal{L}(x^{(t)}, y) \right) \right)
		\]
		Each gradient step pushes the input in a direction that increases the loss. The projection operator \( \Pi \) then clips the result back into the \(\ell_\infty\)-ball:
		\[
		\mathcal{B}_\infty(x_{\text{orig}}, \epsilon) = \left\{ x : \|x - x_{\text{orig}}\|_\infty \leq \epsilon \right\}
		\]
		ensuring that no pixel is perturbed by more than \( \epsilon \). This enforces imperceptibility and validity at every step.
	\end{itemize}
	
	\textit{Why projection is crucial:} Even small gradient steps can overshoot valid bounds—either violating the \( \ell_\infty \) constraint or pushing pixel values outside the allowable range (e.g., [0, 1]). The projection step acts like a \emph{tether}, reeling the point back into the valid neighborhood around \( x_{\text{orig}} \), thus preserving visual indistinguishability.
	
	\textit{Intuition}: Imagine trying to find the highest point (worst-case misclassification) within a tiny neighborhood around your current location (the original image), while being tethered by a rope of length \( \epsilon \). PGD lets you take purposeful, uphill steps in the loss landscape—but if you stray too far, the rope snaps you back into the permissible region. This ensures that the adversarial example remains both effective and imperceptible.
	
	\vspace{0.5em}
	PGD is not only a powerful attack—it also defines a training-time adversary in robust optimization formulations. Its widespread adoption stems from its ability to systematically approximate the worst-case error within a constrained region, making it the de facto benchmark for evaluating model robustness.
	
	\item \textbf{Carlini–Wagner (C\&W)}~\cite{carlini2017_towards}: A precision attack that solves an explicit optimization problem. Instead of using sign gradients, C\&W jointly minimizes:
	
	(i) a distortion loss (e.g., \(\|\delta\|_2\)) to keep the perturbation minimal, and  
	(ii) a classification loss that forces the model to misclassify with high confidence.
	
	This targeted, optimization-based approach produces extremely small perturbations that often break even robust models with imperceptible changes. Unlike FGSM, which takes a single sign-based step, C\&W adapts both direction and magnitude of every pixel.
	
\end{itemize}

\paragraph{Black-box attacks}  
In these, the attacker has no access to gradients or model internals:

\begin{itemize}
	\item \textbf{HopSkipJump}~\cite{chen2020_hopskipjump}: A decision-based algorithm using only final class labels:
	
	\begin{enumerate}
		\item \emph{Initialization}: Requires a starting adversarial example (any misclassified image).
		\item \emph{Boundary search}: Performs binary search along the line from the initial example to the original image to find the closest point on the decision boundary.
		\item \emph{Gradient estimation}: Samples around that boundary point, using model outputs to estimate a direction that moves toward the original image while staying adversarial.
		\item \emph{Projection steps}: Iteratively take steps along the estimated direction, projecting back onto the boundary, refining the adversarial example.
	\end{enumerate}
	
	This boundary-walking strategy efficiently finds small perturbations with few queries.
	
	\item \textbf{Transfer attacks}: Utilize the fact that adversarial examples often transfer across different models. By crafting perturbations on a surrogate model (e.g., using FGSM, BIM, PGD), the attacker can often fool the target model without any queries.
	
	\item \textbf{Universal perturbations}~\cite{moosavi2017_universal}: A single perturbation \(\delta\) that fools the model across many inputs:
	
	\begin{enumerate}
		\item Initialize \(\delta = 0\).
		\item For each dataset image \(x\):
		\begin{itemize}
			\item If \(x + \delta\) is not adversarial, compute a minimal per-image perturbation \(\delta_x\).
			\item Update \(\delta \leftarrow \delta + \delta_x\) and project within an \(\epsilon\)-ball.
		\end{itemize}
		\item Repeat until \(\delta\) consistently causes misclassification across the dataset.
	\end{enumerate}
	
	This demonstrates a global vulnerability—some directions in input space universally degrade model performance.
\end{itemize}

\begin{figure}[H]
	\centering
	\includegraphics[width=0.75\textwidth]{Figures/Chapter_21/slide_57.jpg}
	\caption{White-box attacks use internal gradients to craft precise perturbations. Black-box attacks rely on queries or surrogate transfer to mislead the model without internal access.}
	\label{fig:chapter21_attack_taxonomy}
\end{figure}

% -------------------------------------------------------------
\subsection{Milestones in Robustness Evaluation}

\begin{itemize}
	\item \textbf{Carlini--Wagner attack}~\cite{carlini2017_towards}: Marked a turning point in adversarial research by defeating many proposed defenses like defensive distillation. It prompted a deep reassessment of what constitutes genuine robustness.
	
	\item \textbf{Obfuscated gradients}~\cite{carlini2020_adaptive}: Revealed that many ``robust'' models only masked gradients, rather than solving the underlying problem. This discovery exposed a major flaw in evaluation protocols for robustness.
	
	\item \textbf{Ensemble critique}~\cite{carlini2017_defensive_distillation}: Demonstrated that naive ensembles of weak defenses do not produce strong robustness, highlighting the need for principled combinations or fundamentally stronger strategies.
	
	\item \textbf{Cross-modality attacks}~\cite{carlini2023_universal_llm}: Showed that adversarial vulnerabilities transcend modality boundaries, enabling attacks to transfer across vision and language models in multimodal architectures.
\end{itemize}

% -------------------------------------------------------------
\subsection{Defense Toolbox and Its Limitations}

\begin{itemize}
	\item \textbf{Adversarial training}~\cite{madry2018_towards}: The most empirically effective defense, achieved by injecting adversarial examples into the training loop. While it improves robustness, it significantly increases computational cost and may reduce clean accuracy. Its effectiveness is often constrained to the specific attack types seen during training.
	
	\item \textbf{Input preprocessing}: Simple techniques like JPEG compression, denoising, or spatial smoothing aim to ``clean'' adversarial noise. Though computationally cheap, they are largely ineffective against adaptive attacks designed to survive such transformations.
	
	\item \textbf{Certified defenses}: Provide provable guarantees that no perturbation within a given $\ell_p$ radius can alter the model's prediction. Techniques such as interval bound propagation and randomized smoothing are promising but face challenges in scalability and tightness of bounds.
	
	\item \textbf{Architectural approaches}: Aim to build inherently robust networks via design, e.g., enforcing Lipschitz continuity, smoothing activations, or using specialized robust layers. While theoretically appealing, these methods remain less mature and are still an active area of research.
\end{itemize}

% -------------------------------------------------------------
\subsection{Real-World Relevance and Persistent Risks}

Adversarial examples are not just a lab curiosity---they remain adversarial even after transformations like printing, rephotographing, or 3D rendering. Notable risk areas include:

\begin{itemize}
	\item \textbf{Autonomous driving:} Adversarial stop signs or lane markings can mislead onboard vision systems.
	\item \textbf{Facial recognition:} Perturbed accessories (e.g., adversarial eyeglasses) can cause identity spoofing.
	\item \textbf{Medical imaging:} Subtle modifications to radiology scans can alter diagnostic outcomes.
\end{itemize}

% -------------------------------------------------------------
\subsection{Open Challenges and Theoretical Connections}

Crafting effective attacks remains easier than building robust models. Recent work connects adversarial vulnerability to phenomena like \emph{deep double descent}~\cite{nakkiran2020_deep_double_descent}, where highly overparameterized models exhibit brittle decision boundaries. Understanding and bridging this generalization--robustness gap is a central open problem in modern deep learning.

\section{Class Activation Mapping (CAM) and Grad-CAM}
\label{sec:chapter21_cam_gradcam}

\noindent
Class Activation Mapping (CAM)~\cite{zhou2016_cam} is a visualization technique that highlights regions in an image which are important for a CNN’s classification decision. It operates by projecting the weights from the final fully connected (FC) layer back onto the feature maps of the last convolutional layer.

\paragraph{Mechanism of CAM}

Class Activation Mapping (CAM)~\cite{zhou2016_cam} provides a way to localize the spatial regions in an image that are most influential for a CNN's classification decision. CAM relies on a specific architectural constraint: the final convolutional layer must be followed by a \emph{global average pooling (GAP)} layer and a single fully connected (FC) layer that maps directly to class scores.

Let \( f_k(x, y) \) denote the activation at spatial location \( (x, y) \) of channel \( k \) in the final convolutional layer. After global average pooling, each feature map is reduced to a scalar:
\[
F_k = \frac{1}{H \cdot W} \sum_{x=1}^{H} \sum_{y=1}^{W} f_k(x, y),
\]
where \( H \) and \( W \) denote the height and width of the feature map. These pooled features \( \{F_k\} \) are then fed into a linear classifier:
\[
S_c = \sum_k w_k^{(c)} F_k,
\]
where \( w_k^{(c)} \) is the learned weight connecting feature map \( k \) to class \( c \) in the final FC layer.

Substituting the definition of \( F_k \) yields:
\[
S_c = \sum_k w_k^{(c)} \left( \frac{1}{H \cdot W} \sum_{x=1}^{H} \sum_{y=1}^{W} f_k(x, y) \right)
= \frac{1}{H \cdot W} \sum_{x=1}^{H} \sum_{y=1}^{W} \sum_k w_k^{(c)} f_k(x, y).
\]
This formulation reveals that the class score \( S_c \) is a global sum of spatial contributions from each location \( (x, y) \). The \emph{class activation map} \( M_c(x, y) \) is then defined as the pre-pooled spatial contribution for class \( c \):
\[
M_c(x, y) = \sum_k w_k^{(c)} f_k(x, y).
\]

This results in a low-resolution heatmap \( M_c \) that highlights the importance of each spatial location for predicting class \( c \). Since the feature maps \( f_k(x, y) \) preserve spatial structure (albeit at reduced resolution due to downsampling), the map \( M_c \) can be upsampled (e.g., via bilinear interpolation) to align with the original image, allowing visual localization of discriminative regions.

\medskip
\noindent\textbf{Intuition.} Each convolutional channel \( k \) responds to certain visual patterns (e.g., texture, shape). The weight \( w_k^{(c)} \) indicates how important that pattern is for class \( c \). CAM computes a weighted combination of these patterns over space, producing a heatmap that reveals where the class-specific evidence appears in the image.

\begin{figure}[H]
	\centering
	\includegraphics[width=0.8\textwidth]{Figures/Chapter_21/slide_31.jpg}
	\caption{CAM pipeline: from feature maps to class-specific weighted sums, resulting in localization maps.}
	\label{fig:chapter21_cam_process}
\end{figure}

\begin{figure}[H]
	\centering
	\includegraphics[width=0.8\textwidth]{Figures/Chapter_21/slide_33.jpg}
	\caption{Examples of CAM heatmaps for the classes \emph{dome} and \emph{barbell}. While effective, CAM is limited to the last conv layer.}
	\label{fig:chapter21_cam_examples}
\end{figure}

\paragraph{Limitations of CAM}

CAM is constrained by architecture: it only works with CNNs that end with a global average pooling layer directly connected to a linear classification head. This rules out many modern networks without such structure, including those with multiple FC layers or attention blocks. Moreover, CAM can only visualize the final convolutional layer, which may yield coarse localization due to its low spatial resolution. It lacks the flexibility to probe earlier layers or networks with more complex topologies. These limitations motivated the development of gradient-based generalizations such as Grad-CAM, which we now proceed to cover. 

\subsection{Generalization via Grad-CAM}
Grad-CAM (Gradient-weighted Class Activation Mapping)~\cite{selvaraju2017_gradcam} addresses these issues by using the gradients of any target class flowing into any convolutional layer to produce a localization map. The weights \( \alpha_k^c \) for each channel \( k \) are computed as:
\[
\alpha_k^c = \frac{1}{Z} \sum_{i} \sum_{j} \frac{\partial S_c}{\partial A_{i,j}^k}
\]
where \( A^k \) is the activation map for channel \( k \), and \( S_c \) is the score for class \( c \). The final Grad-CAM map is:
\[
M_c^{\text{Grad}} = \text{ReLU} \left( \sum_k \alpha_k^c A^k \right)
\]

\begin{figure}[H]
	\centering
	\includegraphics[width=0.8\textwidth]{Figures/Chapter_21/slide_37.jpg}
	\caption{Grad-CAM architecture: gradients are backpropagated to a target conv layer to produce class-discriminative maps.}
	\label{fig:chapter21_gradcam_process}
\end{figure}

\newpage
\paragraph{Comparative Visualization Examples}
Grad-CAM can be applied at any convolutional layer and in a wider range of networks. The following figure shows comparisons between guided backpropagation, Grad-CAM, and their combination for different classes (e.g., cat vs dog).

\begin{figure}[H]
	\centering
	\includegraphics[width=0.85\textwidth]{Figures/Chapter_21/slide_38.jpg}
	\caption{Qualitative comparison of different visualization methods applied to an image containing both a dog and a cat. 
		(a) Original image. 
		(b) \textbf{Guided Backpropagation}~\cite{springenberg2015_allconv}: Highlights all features that strongly influence the output, but lacks class-specificity.
		(c) \textbf{Grad-CAM (ours)}~\cite{selvaraju2017_gradcam}: Localizes class-discriminative regions by weighting feature maps based on class gradients.
		(d) \textbf{Guided Grad-CAM}: Combines (b) and (c) to produce high-resolution, class-discriminative saliency maps.
		(e) \textbf{Occlusion sensitivity}~\cite{zeiler2014_visualizing}: Systematically occludes image patches and measures score drop, highlighting regions critical for prediction.
		(f) Grad-CAM on a deeper ResNet layer: Shows consistent class-relevant focus across architectures. 
		Notably, Grad-CAM (c, f) yields results visually similar to occlusion (e) but is more accurate and is orders of magnitude faster to compute.}
	\label{fig:chapter21_gradcam_examples}
\end{figure}

\noindent
Grad-CAM is model-agnostic with respect to the output modality—it is not restricted to image classification. By leveraging the gradients flowing into any convolutional feature map, Grad-CAM can be extended to tasks like image captioning and visual question answering.

\begin{figure}[H]
	\centering
	\includegraphics[width=0.85\textwidth]{Figures/Chapter_21/slide_39.jpg}
	\caption{Visual explanations for image captioning using Grad-CAM~\cite{selvaraju2017_gradcam}. 
		Left: Grad-CAM applied to a captioning model~\cite{vinyals2015_showtell} highlights the spatial evidence used when generating the sentence ``A man is sitting at a table with a pizza.'' The heatmap localizes relevant objects such as the \emph{man} and the \emph{pizza}, providing intuitive support for the generated caption. 
		Right: Grad-CAM applied to a global captioning model conditioned on bounding box-level captions produced by a dense captioning system~\cite{johnson2015_densecap}. The highlighted regions correspond to the caption ``A group of people flying kites on a beach,'' showing that Grad-CAM accurately localizes semantically meaningful regions despite not using any box annotations during training.}
	\label{fig:chapter21_gradcam_captioning}
\end{figure}

\subsection{Comparison Between CAM and Grad-CAM}
\label{par:chapter21_cam_gradcam_comparison_text}

\noindent

\begin{table}[H]
	\centering
	\renewcommand{\arraystretch}{1.25}
	\begin{tabular}{|p{3.8cm}|p{5.2cm}|p{5.2cm}|}
		\hline
		\textbf{Aspect} & \textbf{CAM} & \textbf{Grad-CAM} \\
		\hline
		Architectural Requirement & Requires GAP before the FC classifier; limited to custom architectures. & No architectural constraints; works with standard CNNs like VGG or ResNet. \\
		\hline
		Weight Calculation & Uses fixed weights \( w_{k}^{(c)} \) from the final FC layer. & Computes dynamic weights \( \alpha_k^{(c)} \) via gradients of the class score. \\
		\hline
		Layer Applicability & Only the last conv layer before GAP. & Any convolutional layer, including early or intermediate ones. \\
		\hline
		Network Compatibility & Only with networks designed with GAP. & Works with any pretrained CNN without modification. \\
		\hline
		ReLU on Heatmap & Not explicitly applied; may include negative activations. & Applies ReLU to focus on positive class evidence. \\
		\hline
		Computational Cost & Low; forward pass only. & Higher; requires a backward pass to compute gradients. \\
		\hline
	\end{tabular}
	\caption{Comparison of CAM and Grad-CAM in terms of architecture, flexibility, and output quality.}
	\label{tab:chapter21_cam_gradcam_comparison}
\end{table}

\noindent
In summary, CAM introduced the concept of generating spatially localized class-specific heatmaps using fixed feature-to-class weights, but its reliance on specific architectures limited its general applicability. Grad-CAM resolved this by introducing dynamic, gradient-based weighting, allowing it to be broadly applied to modern CNNs with richer, multi-layer interpretability.

\paragraph{From Explanation to Synthesis: A Path Toward Feature Inversion}
\label{par:chapter21_lead_to_feature_inversion}

\noindent
CAM and Grad-CAM are valuable tools for interpreting neural networks by \emph{highlighting where the model looks} when making a prediction. But they remain reactive—they analyze a network’s behavior \emph{given} an input. What if we reversed this perspective?

Instead of asking \emph{“Where does the model look?”}, we can ask \emph{“What does the model see?”}. That is: can we reconstruct or synthesize an input image that would strongly activate a specific neuron, feature channel, or class output? This brings us to the domain of \textbf{feature inversion}—a class of methods that aim to decode and visualize the internal representations of neural networks by optimizing an input image to match hidden activations.

This generative view enables us to move beyond saliency and uncover the \emph{implicit visual concepts} a model has learned, making it an essential next step in deep network interpretability.

\section{Feature Inversion}
\label{sec:chapter21_feature_inversion}
 
 \noindent
 Feature inversion refers to the task of reconstructing an input image \( \mathbf{x}^* \) that corresponds to a given internal feature representation \( \Phi_0 \) extracted from a trained convolutional neural network (CNN). In contrast to gradient ascent—where we synthesize an image that maximally activates a particular neuron—feature inversion aims to recover an image whose intermediate features match those of a reference image.
 
\paragraph{Problem Formulation}
 
 Let \( \Phi(\mathbf{x}) \in \mathbb{R}^d \) denote the feature vector extracted from an intermediate layer of a pretrained CNN when processing an image \( \mathbf{x} \in \mathbb{R}^{H \times W \times C} \). Given a target image \( \mathbf{x}_0 \), we define \( \Phi_0 = \Phi(\mathbf{x}_0) \). The goal is to reconstruct an image \( \mathbf{x}^* \) such that its feature embedding closely matches \( \Phi_0 \), while also ensuring that the reconstruction remains natural-looking.
 
 \[
 \mathbf{x}^* = \arg\min_{\mathbf{x} \in \mathbb{R}^{H \times W \times C}} \ell(\Phi(\mathbf{x}), \Phi_0) + \lambda \mathcal{R}(\mathbf{x})
 \]
 \[
 \ell(\Phi(\mathbf{x}), \Phi_0) = \|\Phi(\mathbf{x}) - \Phi_0\|^2
 \]
 
 \noindent
 Here, \( \mathcal{R}(\mathbf{x}) \) is a regularization term (e.g., total variation) that encourages spatial smoothness or other natural image priors:
 
 \[
 \mathcal{R}_{V^\beta}(\mathbf{x}) = \sum_{i,j} \left( (x_{i,j+1} - x_{i,j})^2 + (x_{i+1,j} - x_{i,j})^2 \right)^{\frac{\beta}{2}}
 \]
 
 \paragraph{Comparison to Gradient Ascent}
 
 Feature inversion minimizes the difference between two fixed feature vectors. In contrast, gradient ascent attempts to maximize the activation of a particular neuron or class score:
 
 \[
 \mathbf{x}^* = \arg\max_{\mathbf{x}} f(\mathbf{x}) - \lambda \mathcal{R}(\mathbf{x})
 \]
 
 \noindent
 Thus, while gradient ascent highlights what inputs cause strong responses, feature inversion reconstructs what input could have plausibly produced a given representation.
 
 \begin{figure}[H]
 	\centering
 	\includegraphics[width=0.75\textwidth]{Figures/Chapter_21/slide_59.jpg}
 	\caption{Feature inversion optimization: reconstruct an image whose features match those of a target image, optionally constrained by image priors.}
 	\label{fig:chapter21_feature_inversion_formula}
 \end{figure}
 
 \paragraph{Effect of Layer Depth}
 
 The fidelity of reconstructed images depends on the depth of the feature layer:
 
 \begin{itemize}
 	\item \textbf{Shallow layers:} Preserve local textures, colors, and edges. Reconstructions are often nearly photorealistic.
 	\item \textbf{Deeper layers:} Emphasize high-level semantics and invariance, discarding fine details. Reconstructions become abstract or blurry.
 \end{itemize}
 
 \begin{figure}[H]
 	\centering
 	\includegraphics[width=0.85\textwidth]{Figures/Chapter_21/slide_60.jpg}
 	\caption{Feature inversion examples. Top to bottom: two elephants, banana near an apple. As we invert from deeper layers (left $\rightarrow$ right), texture and color fidelity degrade, but semantic structure is broadly preserved.}
 	\label{fig:chapter21_feature_inversion_examples}
 \end{figure}
 
 \paragraph{Interpretability Insights}
 Feature inversion provides an intuitive, visual understanding of what information is preserved at each stage of processing inside a CNN. It reveals how much of the original image is retained or lost—both in low-level visual content and high-level semantic abstraction.
 
 \paragraph{Applications}
 Beyond interpretability, feature inversion has been used in:
 \begin{itemize}
 	\item Visualizing latent representations in self-supervised learning.
 	\item Debugging representations in transfer learning pipelines.
 	\item Data-free knowledge distillation and training set recovery.
 \end{itemize}
 
\paragraph{Beyond Feature Inversion}
 While \textbf{feature inversion} focuses on reconstructing images that faithfully match internal representations, one can also explore how these representations transform an image when amplified. By nudging the input in directions that increase certain layer activations, the network begins to impose its learned abstractions onto the image—accentuating patterns, textures, or objects it has internalized. This approach underlies the technique popularized by Google’s \textbf{DeepDream}, where \emph{gradient ascent} is applied iteratively on a real image to enhance the presence of specific learned features. Depending on the layer selected, the result ranges from abstract texture hallucinations to fully formed semantic motifs, offering a surreal glimpse into the model’s internal visual vocabulary.
 
\section{DeepDream: Amplifying Neural Perceptions}
\label{sec:chapter21_deepdream}
 
 \noindent
 While feature inversion seeks to match internal representations of a specific image, \textbf{DeepDream}~\cite{mordvintsev2015_deepdream} instead aims to \emph{amplify} patterns already present in a given image—revealing the "visual concepts" that specific layers respond to. The result is often surreal and hallucinatory, reflecting the abstractions encoded within the network.
 
 \begin{figure}[H]
 	\centering
 	\includegraphics[width=0.75\textwidth]{Figures/Chapter_21/slide_62.jpg}
 	\caption{DeepDream algorithm: Choose image and layer, forward pass to compute activations, backpropagate activations as gradient, update image. Equivalent to maximizing feature norm.}
 	\label{fig:chapter21_deepdream_process}
 \end{figure}
 
 \paragraph{Optimization Objective}
 Formally, given a fixed CNN and a chosen layer \( \phi \), DeepDream seeks to find an image \( I^{\ast} \) that \emph{maximizes} the L2 norm of the feature activations at that layer:
 \[
 I^{\ast} = \arg\max_I \|\phi(I)\|_2^2 + R(I)
 \]
 where \( R(I) \) denotes natural image regularization terms such as total variation, Gaussian blur, or pixel clipping. These are essential to ensure outputs remain visually coherent and human-interpretable.
 
 \paragraph{Amplifying Layer-wise Semantics}
 DeepDream acts like a feedback loop: it enhances the features already detected by a layer in the input image. The layer selected significantly influences the visual effect:
 \begin{itemize}
 	\item \textbf{Lower layers:} Emphasize edges, colors, and textures. These effects tend to be localized and geometric.
 	\item \textbf{Higher layers:} Reveal semantic abstractions—patterns resembling animals, eyes, buildings, or other high-level features.
 \end{itemize}
 
 \begin{figure}[H]
 	\centering
 	\includegraphics[width=0.8\textwidth]{Figures/Chapter_21/slide_64.jpg}
 	\caption{DeepDream on low-level layers: edge filters amplify simple patterns in the sky, yielding fractal-like textures.}
 	\label{fig:chapter21_deepdream_low_layers}
 \end{figure}
 
 \begin{figure}[H]
 	\centering
 	\includegraphics[width=0.8\textwidth]{Figures/Chapter_21/slide_65.jpg}
 	\caption{DeepDream on high-level layers: dog-like patterns emerge in the clouds as the network amplifies its abstract internal representations.}
 	\label{fig:chapter21_deepdream_high_layers}
 \end{figure}
 
 \begin{figure}[H]
 	\centering
 	\includegraphics[width=0.8\textwidth]{Figures/Chapter_21/slide_66.jpg}
 	\caption{Examples of DeepDream artifacts: clouds mix with psychedelic animal heads, sky becomes textured with hybrid features like buildings.}
 	\label{fig:chapter21_deepdream_artifacts}
 \end{figure}
 
 \newpage
 \paragraph{Dreaming Deeper}
 The longer the optimization is run, the more pronounced the features become—and the farther the image drifts from its original content. In effect, the neural network is “dreaming” what it expects to see, iteratively shaping the image to match layer-wise priors.
 
 \begin{figure}[H]
 	\centering
 	\includegraphics[width=0.8\textwidth]{Figures/Chapter_21/slide_67.jpg}
 	\caption{Progressive amplification of features using DeepDream. The longer the process runs, the more surreal and abstract the image becomes.}
 	\label{fig:chapter21_deepdream_progression}
 \end{figure}
 
 \begin{figure}[H]
 	\centering
 	\includegraphics[width=0.8\textwidth]{Figures/Chapter_21/slide_68.jpg}
 	\caption{Further examples of DeepDream outputs. Internal concepts from different layers manifest as repeating patterns in generated images.}
 	\label{fig:chapter21_deepdream_additional}
 \end{figure}
 
\newpage

\paragraph{Interpretability Value}
Although \textsc{DeepDream} lacks the formal guarantees of analytical interpretability techniques, it offers a vivid and intuitive glimpse into what internal layers of a convolutional neural network are attuned to. By recursively amplifying activation patterns in natural images, it exposes the preferences of learned filters—ranging from edges and textures to object-like structures—across multiple layers of abstraction. In doing so, DeepDream serves as a creative tool that externalizes otherwise hidden visual concepts, blurring the line between model explanation and algorithmic art.

Perhaps more importantly, the emergence of repetitive motifs and rich, hallucinatory textures in DeepDream outputs reveals a key inductive bias of CNNs: their strong reliance on texture-like statistics, even in deeper semantic layers. This observation has inspired a wave of research into directly modeling such internal representations—not just for visualizing what a network has learned, but for synthesizing entirely new images governed by its internal feature distributions. Early approaches based on patch reassembly or nearest-neighbor texture transfer paved the way, but it was the work of Gatys, Ecker, and Bethge~\cite{gatys2015_texture} that reframed texture as an optimizable property encoded in the second-order statistics (Gram matrices) of deep feature activations. This insight laid the foundation for the field of \emph{neural texture synthesis}, which we explore next.

\section{Texture Synthesis}
\label{sec:chapter21_texture_synthesis}

\noindent
Texture synthesis aims to generate a larger image from a small input patch such that the output maintains similar local texture statistics. The goal is not to copy the input exactly, but to match its perceptual properties at the level of local patterns and structure.

\noindent
Before diving into neural approaches to texture synthesis, it's helpful to build intuition from the classical perspective. The task can be illustrated by providing a small texture sample—such as a patch of bricks or fabric—and asking the algorithm to generate a larger image that visually resembles it, without explicitly copying any region. This objective is illustrated in the below figure, followed by a classical nearest-neighbor approach that exemplifies one of the earliest and most influential strategies in this domain.

\begin{figure}[H]
	\centering
	\includegraphics[width=0.75\textwidth]{Figures/Chapter_21/slide_69.jpg}
	\caption{Texture synthesis task overview. Given a small input patch, the goal is to synthesize a larger image that preserves similar local statistics—appearing perceptually consistent without direct repetition.}
	\label{fig:chapter21_texture_overview}
\end{figure}

\subsection{Classical Approaches}
Early algorithms synthesize textures by directly copying pixels from a source image:
\begin{itemize}
	\item \textbf{Non-parametric sampling}~\cite{efros1999_texture}: Grow the output image one pixel at a time in raster scan order, matching local neighborhoods using nearest-neighbor search.
	\item \textbf{Tree-structured vector quantization}~\cite{wei2000_texture}: Speeds up sampling using a hierarchical data structure for efficient neighborhood lookup.
\end{itemize}

\begin{figure}[H]
	\centering
	\includegraphics[width=0.75\textwidth]{Figures/Chapter_21/slide_70.jpg}
	\caption{Non-parametric texture synthesis~\cite{efros1999_texture}. The algorithm grows the output texture pixel-by-pixel by matching local neighborhoods to those in the source patch using nearest-neighbor search.}
	\label{fig:chapter21_texture_nn}
\end{figure}

\begin{figure}[H]
	\centering
	\includegraphics[width=0.8\textwidth]{Figures/Chapter_21/slide_71.jpg}
	\caption{Examples of classical texture synthesis applied to a brick wall and a document fragment. Pixel-based patch matching leads to surprisingly realistic results for locally stationary textures.}
	\label{fig:chapter21_classical_texture}
\end{figure}

\paragraph{Limitations of Pixel Matching}
These techniques often fail on complex textures where pixel-level neighborhoods do not capture the underlying structure. They also lack flexibility for incorporating semantics or learning-based priors.

\subsection{Neural Texture Synthesis via Gram Matrices}
Gatys et al.~\cite{gatys2015_texture} proposed a landmark approach to texture synthesis using convolutional neural networks pretrained on large-scale image datasets. Rather than matching image pixels directly, their method captures \emph{texture} by aligning the second-order statistics of intermediate activations in a CNN. These statistics are encoded using \emph{Gram matrices}, which measure feature co-activation patterns while discarding explicit spatial information.

\paragraph{Constructing the Gram Matrix}
Given a feature map tensor \( F^{\ell} \in \mathbb{R}^{C_\ell \times H_\ell \times W_\ell} \) at some layer \( \ell \), we compute the corresponding Gram matrix \( G^{\ell} \in \mathbb{R}^{C_\ell \times C_\ell} \) as:
\[
G^{\ell}_{c,c'} = \sum_{h,w} F^{\ell}_{c,h,w} \cdot F^{\ell}_{c',h,w}
\]
Each entry \( G^{\ell}_{c,c'} \) represents the inner product between feature maps \( c \) and \( c' \) across all spatial locations. This effectively measures the extent to which features \( c \) and \( c' \) co-occur in the image, averaged over space.

\begin{figure}[H]
	\centering
	\includegraphics[width=0.8\textwidth]{Figures/Chapter_21/slide_73.jpg}
	\caption{Constructing the Gram matrix: given feature activations across spatial dimensions, we compute a \( C \times C \) matrix that captures global feature co-occurrence statistics.}
	\label{fig:chapter21_gram_matrix}
\end{figure}

\paragraph{Why Gram Matrices?}
Textures are often characterized by the statistical relationships between local patterns rather than their precise spatial arrangement. By aggregating over spatial locations, the Gram matrix retains feature co-activation statistics while discarding spatial structure—making it ideally suited for texture modeling. 

Computationally, this becomes efficient by reshaping \( F^{\ell} \) from shape \( C \times H \times W \) into a matrix \( F \in \mathbb{R}^{C \times (H W)} \), and computing:
\[
G = F F^\top \in \mathbb{R}^{C \times C}
\]
This formulation allows fast matrix multiplication instead of nested loops, making it tractable even for deep layers with large spatial dimensions.

\begin{figure}[H]
	\centering
	\includegraphics[width=0.8\textwidth]{Figures/Chapter_21/slide_75.jpg}
	\caption{Efficient Gram matrix computation by flattening spatial dimensions: from \( C \times H \times W \) to \( C \times HW \), then multiplying by its transpose.}
	\label{fig:chapter21_gram_computation}
\end{figure}

\paragraph{Optimization Pipeline}
The texture synthesis process is driven by matching the Gram matrices of a generated image to those of a reference texture image, across multiple CNN layers. The full algorithm is:

\begin{enumerate}
	\item Use a pretrained CNN (e.g., VGG-19) and record feature activations \( F^{\ell} \) at selected layers for a given texture image.
	\item Compute Gram matrices \( G^{\ell} \) from these activations.
	\item Initialize the synthesized image \( I \) from white noise.
	\item Iterate:
	\begin{itemize}
		\item Forward \( I \) through the CNN to compute new Gram matrices \( \hat{G}^{\ell} \).
		\item Compute per-layer style loss:
		\[
		E_\ell = \frac{1}{4N_\ell^2 M_\ell^2} \sum_{c,c'} \left( G^{\ell}_{c,c'} - \hat{G}^{\ell}_{c,c'} \right)^2
		\]
		\item Aggregate across layers: \( L = \sum_{\ell} w_\ell E_\ell \)
		\item Backpropagate to update the image \( I \).
	\end{itemize}
\end{enumerate}

\begin{figure}[H]
	\centering
	\includegraphics[width=0.8\textwidth]{Figures/Chapter_21/slide_81.jpg}
	\caption{Full pipeline of neural texture synthesis: from extracting Gram matrices to iterative gradient-based refinement of a noise image to match the desired style.}
	\label{fig:chapter21_texture_pipeline}
\end{figure}

\paragraph{Effect of Matching Higher Layers}
Different layers encode different types of information: lower layers capture fine-scale patterns such as edges or color blobs, while deeper layers encode more abstract texture features. By matching Gram matrices at higher layers, the synthesized texture captures broader patterns and global structure—though precise spatial detail may be lost.

\begin{figure}[H]
	\centering
	\includegraphics[width=0.8\textwidth]{Figures/Chapter_21/slide_82.jpg}
	\caption{Texture reconstructions from matching Gram matrices at various depths. Shallow layers reconstruct local textures; deeper layers capture larger-scale features and structure.}
	\label{fig:chapter21_neural_texture_results}
\end{figure}

\paragraph{Impact and Legacy}
The Gram-based synthesis approach laid the foundation for two influential lines of work:
\begin{itemize}
	\item \textbf{Neural Style Transfer} – Separates style (texture) and content by combining Gram matrix loss with feature reconstruction loss~\cite{gatys2016_stylization}.
	\item \textbf{Fast Style Transfer} – Trains a feedforward network to approximate the optimization, enabling real-time applications using perceptual loss~\cite{johnson2016_perceptual}.
\end{itemize}
These models broadened the use of deep features for both artistic and functional image transformations—topics we now explore next.

\section{Neural Style Transfer}
\label{sec:chapter21_neural_style_transfer}

\subsection{Neural Style Transfer: Content and Style Fusion}
\label{sec:chapter21_neural_style_transfer}

\noindent
Neural Style Transfer (NST) generates a new image \( I^{\ast} \) that merges the \emph{semantic content} of one image with the \emph{visual style} of another. Drawing from advances in texture synthesis and feature inversion, NST formulates an optimization problem over a pre-trained convolutional network (typically VGG-19), where the goal is to make \( I^{\ast} \) simultaneously match:
\begin{itemize}
	\item The \textbf{content features} of a content image \( I_c \), extracted from higher-level activations.
	\item The \textbf{style statistics} of a style image \( I_s \), encoded as Gram matrices across multiple layers.
\end{itemize}

\paragraph{Intuition}
\label{par:chapter21_nst_intuition}

\noindent
Deep convolutional layers capture high-level abstractions—such as objects and layout—while shallow layers encode texture, edges, and color patterns. NST leverages this hierarchy by aligning \( I^{\ast} \)'s deep activations with \( I_c \)'s to preserve structure, and matching Gram matrices at multiple depths to reflect \( I_s \)'s stylistic patterns. This allows content and style to be fused in a perceptually coherent manner.

\begin{figure}[H]
	\centering
	\includegraphics[width=0.8\textwidth]{Figures/Chapter_21/slide_84.jpg}
	\caption{Two optimization objectives: \textbf{Top—Style (Texture Synthesis)} via Gram matrix matching; \textbf{Bottom—Content Reconstruction} via feature matching.}
	\label{fig:chapter21_nst_dual_objectives}
\end{figure}

\paragraph{Optimization Objective}
\label{par:chapter21_nst_optimization_objective}

\noindent
The network acts as a fixed perceptual encoder. To synthesize \( I^{\ast} \), we minimize a loss combining:
\begin{itemize}
	\item \textbf{Content loss:} Encourages \( I^{\ast} \) to replicate the activations of \( I_c \) at a deep layer (e.g., \texttt{conv4\_2}), capturing object identity and layout.
	\item \textbf{Style loss:} Encourages \( I^{\ast} \) to match the Gram matrices of \( I_s \) across several layers (e.g., \texttt{conv1\_1} to \texttt{conv5\_1}), encoding texture statistics at multiple spatial scales.
\end{itemize}

\noindent
These losses are defined over the same network, so inputs must be compatible in resolution. Although Gram matrices are spatially invariant—aggregating across spatial locations—it is standard practice to resize \( I_c \), \( I_s \), and \( I^{\ast} \) to a common resolution for stable optimization and fair comparison of content and style representations.

\medskip

\noindent
Together, this dual-objective framework enables the creation of images that preserve the global structure of a scene while adopting the local visual patterns of a target artistic style.

\begin{figure}[H]
	\centering
	\includegraphics[width=0.8\textwidth]{Figures/Chapter_21/slide_85.jpg}
	\caption{Neural Style Transfer architecture: Content features are extracted from the content image, and style features (Gram matrices) from the style image. Both guide the optimization of a new output image.}
	\label{fig:chapter21_nst_architecture}
\end{figure}

\paragraph{Optimization via Gradient Descent}
\label{par:chapter21_nst_gradient_descent}

\noindent
To find the stylized image \( I^{\ast} \), we define a total loss function that balances content fidelity and style transfer:
\[
\mathcal{L}_{\text{total}}(I^{\ast}) = \alpha \cdot \mathcal{L}_{\text{content}}(I^{\ast}, I_c) + \beta \cdot \mathcal{L}_{\text{style}}(I^{\ast}, I_s)
\]
where:
\begin{itemize}
	\item \textbf{Content Loss:}
	\[
	\mathcal{L}_{\text{content}}(I^{\ast}, I_c) = \left\| \phi_\ell(I^{\ast}) - \phi_\ell(I_c) \right\|_2^2
	\]
	This term ensures that the synthesized image \( I^{\ast} \) matches the content features of \( I_c \) at a chosen higher layer \( \ell \) of the CNN. The function \( \phi_\ell(\cdot) \) denotes the activations at layer \( \ell \), which encode abstract structural and semantic information.
	
	\item \textbf{Style Loss:}
	\[
	\mathcal{L}_{\text{style}}(I^{\ast}, I_s) = \sum_j w_j \cdot \left\| G^j(I^{\ast}) - G^j(I_s) \right\|_F^2
	\]
	This term measures the difference in style between \( I^{\ast} \) and \( I_s \) by comparing their Gram matrices \( G^j(\cdot) \) at multiple layers \( j \). Each Gram matrix captures the pairwise correlations between feature channels, reflecting texture and visual style. The weights \( w_j \) control the relative importance of style matching across layers.
\end{itemize}

\medskip
\noindent
The optimization starts from a randomly initialized image—or optionally the content image itself—and iteratively updates the pixels of \( I^{\ast} \) using gradient descent. In each iteration:
\begin{enumerate}
	\item \( I^{\ast} \) is passed through the CNN to extract content and style features.
	\item The total loss \( \mathcal{L}_{\text{total}} \) is computed.
	\item Gradients of this loss with respect to the image \( I^{\ast} \) are computed via backpropagation.
	\item The image is updated to reduce the loss: \( I^{\ast} \gets I^{\ast} - \eta \cdot \nabla_{I^{\ast}} \mathcal{L}_{\text{total}} \), where \( \eta \) is the learning rate.
\end{enumerate}

\noindent
The trade-off parameters \( \alpha \) and \( \beta \) modulate the relative emphasis on content preservation versus style transfer. A higher \( \alpha / \beta \) ratio favors structural fidelity, while a lower ratio prioritizes stylization. This balance enables the method to flexibly interpolate between photorealistic and painterly outputs.

\begin{figure}[H]
	\centering
	\includegraphics[width=0.8\textwidth]{Figures/Chapter_21/slide_88.jpg}
	\caption{Gradient-based optimization: iteratively update the image to minimize content and style loss using gradients from a pretrained CNN.}
	\label{fig:chapter21_nst_optimization}
\end{figure}

\newpage
\paragraph{Stylization Results}
The outcome of this optimization is a new image that visually blends the spatial layout of the content image with the textures and patterns of the style image.

\begin{figure}[H]
	\centering
	\includegraphics[width=0.8\textwidth]{Figures/Chapter_21/slide_86.jpg}
	\caption{A stylization result: the content structure is preserved while adopting textures and colors from the style artwork.}
	\label{fig:chapter21_nst_result_1}
\end{figure}

\begin{figure}[H]
	\centering
	\includegraphics[width=0.8\textwidth]{Figures/Chapter_21/slide_90.jpg}
	\caption{Additional examples of Neural Style Transfer across various artworks and content images.}
	\label{fig:chapter21_nst_result_2}
\end{figure}

\newpage
\paragraph{Controlling Style Intensity}
Adjusting the ratio \( \beta / \alpha \) enables control over how strongly the style is imposed versus how much content is preserved.

\begin{figure}[H]
	\centering
	\includegraphics[width=0.8\textwidth]{Figures/Chapter_21/slide_91.jpg}
	\caption{Effect of changing content-style trade-off: higher style weight yields more aggressive stylization; higher content weight yields better structural fidelity.}
	\label{fig:chapter21_nst_content_style_tradeoff}
\end{figure}

\paragraph{Effect of Style Image Scale}
Interestingly, the spatial scale of the style image affects the type of features that are transferred. Large style images encourage local brushstrokes; small images bias the transfer toward large-scale visual motifs.

\begin{figure}[H]
	\centering
	\includegraphics[width=0.8\textwidth]{Figures/Chapter_21/slide_92.jpg}
	\caption{Effect of style image resizing: larger style image induces small-scale brush strokes; smaller style image encourages transfer of large-scale visual features.}
	\label{fig:chapter21_nst_style_scale}
\end{figure}

\paragraph{Combining Styles}
Neural Style Transfer can be extended to blend multiple styles by mixing the corresponding Gram matrices:
\[
\mathcal{L}_{\text{style}}^{\text{combined}} = \gamma \cdot \mathcal{L}_{\text{style}}^{(1)} + (1 - \gamma) \cdot \mathcal{L}_{\text{style}}^{(2)}
\]
This enables generation of novel, hybrid artistic effects.

\begin{figure}[H]
	\centering
	\includegraphics[width=0.8\textwidth]{Figures/Chapter_21/slide_93.jpg}
	\caption{Mixed style transfer: combining styles from two different artworks yields visually blended results.}
	\label{fig:chapter21_nst_style_mixing}
\end{figure}

\paragraph{Limitations}
Despite its effectiveness, this method is computationally expensive—each new stylized image requires many forward and backward passes through the CNN. This limitation motivates the development of real-time methods, discussed next.

\newpage
\subsection{Fast Neural Style Transfer}
\label{sec:chapter21_fast_style_transfer}

\noindent
While the original Neural Style Transfer algorithm produces compelling results, its reliance on iterative optimization makes it computationally expensive. Each stylized image requires dozens to hundreds of forward and backward passes through a deep network (e.g., VGG-19), making real-time applications infeasible.

To address this limitation, Johnson et al.~\cite{johnson2016_perceptual} proposed an alternative: train a separate \emph{feedforward neural network} to perform style transfer in a single forward pass. Once trained, this network can stylize new content images extremely efficiently—enabling real-time inference.

\paragraph{Training Setup}
The core idea is to use the same perceptual loss as before—combining content and style objectives—but apply it to train the weights of a stylization network \( T_\theta(\cdot) \) rather than directly optimizing the image pixels.

\begin{figure}[H]
	\centering
	\includegraphics[width=0.85\textwidth]{Figures/Chapter_21/slide_100.jpg}
	\caption{Fast style transfer training loop: use perceptual loss to train a feedforward network that performs style transfer in a single pass.}
	\label{fig:chapter21_fast_style_transfer_training}
\end{figure}

\paragraph{Key Insight}
Instead of solving:
\[
\arg\min_I \; \mathcal{L}_{\text{content}}(I, I_c) + \mathcal{L}_{\text{style}}(I, I_s)
\]
for each new image \( I_c \), the fast method learns:
\[
\arg\min_\theta \; \mathbb{E}_{I_c} \left[ \mathcal{L}_{\text{content}}(T_\theta(I_c), I_c) + \mathcal{L}_{\text{style}}(T_\theta(I_c), I_s) \right]
\]
This way, the trained network \( T_\theta \) can stylize any new image without additional optimization.

\newpage
\paragraph{Stylization Examples}
Once trained, the feedforward network can apply the desired style instantly.

\begin{figure}[H]
	\centering
	\includegraphics[width=0.75\textwidth]{Figures/Chapter_21/slide_101.jpg}
	\caption{Fast style transfer examples: output images styled in the aesthetics of Van Gogh’s \emph{Starry Night} and Picasso’s \emph{The Muse}.}
	\label{fig:chapter21_fast_examples}
\end{figure}

\paragraph{Instance Normalization}
Ulyanov et al.~\cite{ulyanov2017_instance} discovered that replacing batch normalization with \emph{instance normalization} significantly improves stylization quality. Unlike batch norm, instance norm normalizes each feature map independently for each image—preserving per-instance feature statistics critical for style representation.

\begin{figure}[H]
	\centering
	\includegraphics[width=0.75\textwidth]{Figures/Chapter_21/slide_102.jpg}
	\caption{High-quality stylized outputs from fast neural style transfer trained with instance normalization.}
	\label{fig:chapter21_instance_norm_results}
\end{figure}

\newpage
\paragraph{Conditional Instance Normalization for Multi-Style Transfer}
Training one network per style is inefficient. Dumoulin et al.~\cite{dumoulin2017_cbn} introduced \textbf{conditional instance normalization}, allowing a single network to stylize multiple styles. Each style is associated with a unique set of scale and shift parameters:
\[
\text{IN}(x; s) = \gamma^{(s)} \cdot \frac{x - \mu}{\sigma} + \beta^{(s)}
\]
where \( s \) denotes the selected style and \( (\gamma^{(s)}, \beta^{(s)}) \) are learned per style. This allows:
\begin{itemize}
	\item Fast switching between styles within one model.
	\item Style interpolation via blending parameters.
\end{itemize}

\begin{figure}[H]
	\centering
	\includegraphics[width=0.85\textwidth]{Figures/Chapter_21/slide_104.jpg}
	\caption{Conditional instance normalization enables one network to perform multiple styles—and interpolate between them.}
	\label{fig:chapter21_conditional_style}
\end{figure}

\paragraph{Summary and Emerging Directions}
Fast neural style transfer leverages perceptual losses to train a feedforward network that efficiently stylizes images in real time—an elegant engineering solution bridging artistic quality and speed. However, this paradigm is now being extended further by cutting-edge approaches:

\begin{itemize}
	\item \textbf{Diffusion Model Style Transfer:} Recent work such as DiffeseST introduces training-free style transfer using diffusion models, leveraging DDIM inversion and spatial-textual embeddings to achieve high-fidelity stylization without per-image optimization~\cite{hu2024_diffusest}.
	\item \textbf{Self-Supervised Style Augmentation:} SASSL uses neural style transfer as a data augmentation technique in self-supervised learning, improving representation quality by enhancing style diversity while preserving content semantics~\cite{rojas2024_sassl}.
	\item \textbf{Semantically-Guided Diffusion Stylization:} Models like StyleDiffusion and InST apply diffusion processes with adaptive conditioning to disentangle style and content features, offering richer control in stylization~\cite{wang2023_stylediffusion, zhang2023_inst}.
\end{itemize}

These advances demonstrate the ongoing evolution of style transfer—incorporating generative modeling, efficient adaptation, and semantic awareness—which promise more powerful and flexible tools in future editions of this textbook.

 
\chapterimage{head2.png} % Chapter heading image

% Chapter-specific content starts here
\chapter{Lecture 22: Self-Supervised Learning}

%----------------------------------------------------------------------------------------
%	CHAPTER 22 - Lecture 22: Self-Supervised Learning
%----------------------------------------------------------------------------------------


\chapterimage{head2.png} % Chapter heading image

% Chapter-specific content starts here
\chapter{Lecture 23: 3D vision}

%----------------------------------------------------------------------------------------
%	CHAPTER 23 - Lecture 23: 3D vision
%----------------------------------------------------------------------------------------


\chapterimage{head2.png} % Chapter heading image

% Chapter-specific content starts here
\chapter{Lecture 24: Videos (Video Understanding)}

%---------------------------------------------------------------------
%    CHAPTER 24 - Lecture 24: Videos (Video Understanding)
%---------------------------------------------------------------------

\section{Introduction to Video Understanding}
\label{sec:chapter24_video_intro}

Up to this point, our discussion has focused mainly on \textbf{images}, typically represented as 3D tensors of shape $C \times H \times W$, where $C$ denotes the number of channels (often three for RGB). In this chapter we generalize from static images to \textbf{videos}, which can be viewed as sequences of images indexed by time. A video is therefore represented by a 4D tensor of shape:
\begin{equation}
    \label{eq:chapter24_video_tensor}
    \mathbf{V} \in \mathbb{R}^{T \times C \times H \times W},
\end{equation}
where $T$ denotes the temporal dimension, corresponding to the number of frames in the sequence.

This extension introduces a fundamental challenge: while image analysis largely emphasizes spatial patterns, video understanding requires us to jointly reason about \textbf{spatial} and \textbf{temporal} structures. Tasks defined over videos range widely, from video classification and temporal action localization to video captioning and generation. In this lecture and chapter we focus on \textbf{video understanding}, that is, building models that interpret the content of a video clip to predict semantic properties such as actions, interactions, or events.

\subsection{From Images to Videos}
\label{subsec:chapter24_from_images_to_videos}

In image classification, the objective is typically to detect the presence of objects (\emph{e.g.}, predicting that an image contains a cat). In contrast, \textbf{video classification} aims to recognize \emph{actions}. For example, given a short clip of a person, the model should distinguish whether the individual is \emph{running}, \emph{walking}, \emph{jumping}, or \emph{standing}. This shift from nouns (objects) to verbs (actions) reflects the additional temporal complexity inherent in videos.

\begin{figure}[H]
    \centering
    \includegraphics[width=0.8\textwidth]{Figures/Chapter_24/slide_8.jpg}
    \caption{Contrasting image and video classification. While images are labeled with objects such as ``cat'' or ``truck'', video clips are typically labeled with actions such as ``running'' or ``swimming''.}
    \label{fig:chapter24_image_vs_video_classification}
\end{figure}

\subsection{Challenges of Video Data and Clip-Based Training}
\label{subsec:chapter24_video_clips}

Videos present substantial computational burdens compared to images. A standard video stream is recorded at approximately 30 frames per second, with each frame containing hundreds of thousands or millions of pixels. For example, storing an uncompressed video requires approximately:
\begin{itemize}
    \item $\sim$1.5 GB per minute for standard definition (640 $\times$ 480),
    \item $\sim$10 GB per minute for high definition (1920 $\times$ 1080).
\end{itemize}
This scale makes it infeasible to directly train on raw, full-length videos.

\begin{figure}[H]
    \centering
    \includegraphics[width=0.8\textwidth]{Figures/Chapter_24/slide_10.jpg}
    \caption{Illustration of video storage cost. Uncompressed video scales rapidly with resolution and frame rate, motivating the need for short clips and reduced sampling during training.}
    \label{fig:chapter24_video_size}
\end{figure}

\newpage

The standard solution is to \textbf{train on short clips} rather than entire videos. A raw sequence of length $T_\text{raw}$ is divided into windows of $T$ consecutive frames, often subsampled in time (e.g., taking every $k$th frame) to reduce the effective frame rate. Clips are also downsampled spatially (e.g., $112 \times 112$ pixels). During training, models are supervised on these short clips.

At test time, multiple clips are sampled from different temporal regions of the video. The model processes each subclip independently, and the results are aggregated—typically via averaging—to produce a robust video-level prediction.

\begin{figure}[H]
    \centering
    \includegraphics[width=0.85\textwidth]{Figures/Chapter_24/slide_13.jpg}
    \caption{Training and testing with clips. During training, models are trained on short subsampled clips. At test time, the model is applied to multiple subclips, and predictions are averaged to yield a video-level decision.}
    \label{fig:chapter24_clips_training_testing}
\end{figure}

\section{Video Classification as a Canonical Task}
\label{sec:chapter24_video_classification_intro}

\textbf{Video classification} serves as a canonical entry point into video understanding. The task is defined as mapping an input clip $\mathbf{V} \in \mathbb{R}^{T \times C \times H \times W}$ to a label $y \in \{1,\dots,K\}$ from a fixed action vocabulary of size $K$. Formally, we seek to learn a function
\begin{equation}
    \label{eq:chapter24_video_classification}
    f_\theta: \mathbb{R}^{T \times C \times H \times W} \to \{1,\dots,K\},
\end{equation}
where $\theta$ denotes the parameters of the model. As in image classification, the system is typically trained with cross-entropy loss. However, the network architecture must incorporate temporal reasoning, either explicitly or implicitly, in order to succeed.

This formulation establishes video classification as a foundation for more advanced video understanding tasks, such as \textbf{temporal action localization} (detecting when an action occurs within an untrimmed video) and \textbf{spatio-temporal action detection} (localizing actions in both space and time). In the following parts, we progressively build models to handle the spatio-temporal complexity of videos, beginning with simple baselines and gradually extending to sophisticated architectures.

\newpage

\subsection{Single-Frame Baseline}
\label{subsec:chapter24_single_frame}

An unexpectedly strong baseline for video classification is to \emph{ignore temporal information entirely}. In this approach, each frame is classified independently using a standard 2D CNN trained on individual RGB frames with the video-level label. At test time, predictions across frames are averaged to obtain the final decision. While simple, this baseline often achieves competitive accuracy and should always be attempted first in practice.

\begin{figure}[H]
    \centering
    \includegraphics[width=0.85\textwidth]{Figures/Chapter_24/slide_14.jpg}
    \caption{Single-frame CNN baseline. Each frame is classified independently, and predictions are aggregated at test time. Despite ignoring temporal structure, this baseline is surprisingly strong.}
    \label{fig:chapter24_single_frame}
\end{figure}

\subsection{Late Fusion}
\label{subsec:chapter24_late_fusion}

To incorporate temporal reasoning, a natural extension is \textbf{late fusion}. Here, each frame is first processed independently by a 2D CNN to produce feature maps of shape $D \times H' \times W'$. The sequence of features across $T$ frames is then concatenated into a tensor of shape $T \times D \times H' \times W'$. This can be flattened into a single feature vector of dimension $TDH'W'$, followed by fully connected layers and a softmax classifier:
\begin{equation}
    \label{eq:chapter24_late_fusion}
    \hat{y} = \text{Softmax}\big( \text{MLP}(\text{Flatten}(\{f_1,\dots,f_T\})) \big),
\end{equation}
where $f_t$ denotes the per-frame CNN features.

The intuition is that we first capture high-level appearance in each frame and then combine them at the classification stage.

\begin{figure}[H]
    \centering
    \includegraphics[width=0.85\textwidth]{Figures/Chapter_24/slide_15.jpg}
    \caption{Late fusion with fully connected layers. Frame-level features are concatenated, flattened, and passed to an MLP for classification.}
    \label{fig:chapter24_late_fusion_fc}
\end{figure}

A more parameter-efficient variant replaces the flatten–FC stage with \textbf{global average pooling} (GAP) over both spatial and temporal dimensions, yielding a compact $D$-dimensional vector before the classifier. While effective at reducing overfitting, late fusion methods have a key limitation: they struggle to capture fine-grained motion signals between consecutive frames, since temporal information is collapsed only at a late stage.

\begin{figure}[H]
    \centering
    \includegraphics[width=0.85\textwidth]{Figures/Chapter_24/slide_17.jpg}
    \caption{Late fusion with global average pooling. Although parameter-efficient, this approach struggles to capture low-level motion cues such as periodic leg movement in running.}
    \label{fig:chapter24_late_fusion_gap}
\end{figure}

\subsection{Early Fusion}
\label{subsec:chapter24_early_fusion}

To better model small/fine-grained temporal dynamics, we can adopt \textbf{early fusion}. Here, the temporal dimension is reshaped into the channel dimension: the input clip $\mathbb{R}^{T \times 3 \times H \times W}$ is reformatted into $\mathbb{R}^{3T \times H \times W}$. A 2D CNN is then applied, treating time-stacked frames as an enlarged channel input. This allows the first convolutional layer to directly compare pixel intensities across adjacent frames, thereby capturing short-term motion.

\begin{figure}[H]
    \centering
    \includegraphics[width=0.85\textwidth]{Figures/Chapter_24/slide_19.jpg}
    \caption{Early fusion approach. The temporal dimension is stacked as channels, enabling the first 2D convolution to compare frames directly.}
    \label{fig:chapter24_early_fusion}
\end{figure}

While this mitigates late fusion’s inability to capture motion, the temporal dimension is collapsed after the first convolution. This one-shot fusion can be overly aggressive, discarding longer-range temporal information, and thus harm classification results. 

\newpage

\subsection{3D CNNs: Slow Fusion}
\label{subsec:chapter24_3dcnn}

A natural extension of 2D convolution to video is to treat time as an additional dimension and apply \textbf{3D convolutions}. In this design, filters have shape \(K_t \times K_h \times K_w\), spanning the temporal axis as well as the spatial axes. Activations remain four-dimensional (\(D \times T \times H \times W\)), where \(T\) denotes temporal extent. By stacking such layers, temporal information is fused progressively across depth—an approach known as \emph{slow fusion}.

\begin{figure}[H]
    \centering
    \includegraphics[width=0.55\textwidth]{Figures/Chapter_24/slide_20.jpg}
    \caption{3D convolution over video clips. Filters extend across both spatial dimensions and time, producing feature maps that jointly capture motion and appearance.}
    \label{fig:chapter24_3dconv}
\end{figure}

This architecture enables hierarchical learning of spatiotemporal features: early layers may detect short-term motion edges, while deeper layers aggregate evidence for longer-term dynamics. The formulation was pioneered in early works such as \cite{ji2010_3dcnn, karpathy2014_videocnn}.

\medskip
\noindent\textbf{Comparison with fusion alternatives.}
To place 3D CNNs in context, it is helpful to compare with early and late fusion strategies. In \emph{early fusion}, temporal information is aggregated at the input by stacking frames as channels, while spatial receptive fields grow across depth. In \emph{late fusion}, each frame is processed independently by 2D CNNs, and temporal integration occurs only at the final stage. By contrast, 3D CNNs (\emph{slow fusion}) expand both spatial and temporal receptive fields gradually, balancing spatial and temporal modeling capacity.

\begin{figure}[H]
    \centering
    \includegraphics[width=0.6\textwidth]{Figures/Chapter_24/slide_28.jpg}
    \caption{Comparison of fusion strategies. Late fusion: spatial receptive field grows gradually, temporal fusion only at the end. Early fusion: temporal fusion at the start, spatial receptive field grows gradually. 3D CNN (slow fusion): both spatial and temporal receptive fields expand gradually.}
    \label{fig:chapter24_fusion_comparison}
\end{figure}

\subsection{2D vs 3D Convolutions}
\label{subsec:chapter24_2d_vs_3d}

To better understand the distinction between early fusion with 2D convolutions and true 3D convolutions, it is useful to analyze how their filters operate over time:

\begin{itemize}
    \item \textbf{Early fusion (2D convolutions on stacked frames):}  
    Frames are concatenated along the channel dimension and processed by a 2D convolution. First-layer filters therefore have shape  
    \[
    C_\text{out} \times C_\text{in} \times T \times K_h \times K_w.
    \]  
    Each filter spans the \emph{entire} temporal extent \(T\). This design has two drawbacks:
    \begin{enumerate}
        \item \emph{No temporal shift invariance:} the filter is tied to specific time positions. For example, a filter trained to detect a hand moving to the right between frames 1 and 2 will not automatically generalize to the same motion between frames 3 and 4. A separate set of weights must be learned for each timing.
        \item \emph{Parameter inefficiency:} since temporal variation must be explicitly memorized at different offsets, many more filters are needed to cover the same set of motions. This makes early fusion prone to overfitting and less data-efficient.
    \end{enumerate}
    
    \item \textbf{3D convolution (true spatiotemporal kernels):}  
    Filters extend over a limited temporal window \(K_t \ll T\), with shape  
    \[
    C_\text{out} \times C_\text{in} \times K_t \times K_h \times K_w.
    \]  
    These filters \emph{slide along the temporal axis}, just as 2D filters slide spatially. This provides temporal shift invariance: once a kernel has learned to detect a short motion pattern (e.g., a flick or edge moving across frames), it will activate regardless of where in the sequence that motion occurs. This is analogous to translation invariance in images, but extended into the time dimension.
\end{itemize}

For visual clarity, Justin Johnson illustrates these concepts with concrete examples. Early fusion requires separate filters to detect the same phenomenon (in the example, color transition from orange$\!\to\!$blue) at different times in the sequence, whereas 3D convolution achieves this with a single filter that generalizes across temporal positions.

\begin{figure}[H]
    \centering
    \includegraphics[width=0.7\textwidth]{Figures/Chapter_24/slide_29.jpg}
    \caption{Early fusion setup. Filters span the entire temporal dimension, tying responses to absolute time positions.}
    \label{fig:chapter24_early_fusion_limit_setup}
\end{figure}

\begin{figure}[H]
    \centering
    \includegraphics[width=0.85\textwidth]{Figures/Chapter_24/slide_31.jpg}
    \caption{Limitation illustrated. To detect an orange$\!\to\!$blue transition early vs late, early fusion needs \emph{two different} filters aligned to different temporal offsets.}
    \label{fig:chapter24_early_fusion_limit_colors}
\end{figure}

\begin{figure}[H]
    \centering
    \includegraphics[width=0.85\textwidth]{Figures/Chapter_24/slide_34.jpg}
    \caption{3D convolution: filters slide in time, providing temporal shift invariance. A single kernel that detects the orange$\!\to\!$blue transition generalizes to any temporal position in the sequence.}
    \label{fig:chapter24_3dconv_invariance}
\end{figure}

\newpage 

\paragraph{Clarifying Input Channels vs Temporal Dimension}

Convolutions always combine all input channels ($C_\text{in}$, e.g.\ RGB) at once. The real difference between 2D and 3D convolutions is whether the \textbf{temporal axis} is collapsed or preserved, which changes the filter shape and what it can learn.

\begin{itemize}
    \item \textbf{2D convolution (early fusion):}  
    Input: $(C_\text{in}\!\cdot\!T) \times H \times W$.  
    Filter: $C_\text{out} \times (C_\text{in}\!\cdot\!T) \times K_h \times K_w$.  
    The filter is 2D in space ($K_h \times K_w$), but its depth spans all channels, including stacked frames. Thus, time is \textbf{baked into channels}. The network sees all frames at once but cannot reuse the same filter across time; it must learn separate filters for motion at different temporal positions.
    
    \item \textbf{3D convolution:}  
    Input: $C_\text{in} \times T \times H \times W$.  
    Filter: $C_\text{out} \times C_\text{in} \times K_t \times K_h \times K_w$.  
    The filter is volumetric: it spans $K_t$ consecutive frames as well as $K_h \times K_w$ spatial pixels. Crucially, it slides across \emph{time, height, and width}. This preserves temporal structure and gives \textbf{temporal shift invariance}: the same filter can detect a motion pattern (e.g., a color change or edge movement) regardless of when it occurs in the sequence.
\end{itemize}

\noindent
\textbf{Practical implication.}  
In early fusion (2D), the model treats the clip like a single thick image: temporal order is fixed, and motion is hard to generalize. In 3D convolution, the model treats the clip as a video volume: filters move through time as well as space, making them natural motion detectors.

\subsection{Sports-1M Dataset and Baseline Comparisons}
\label{subsec:chapter24_sports1m}

An influential benchmark for video classification is the \textbf{Sports-1M dataset} \cite{karpathy2014_videocnn}. It consists of roughly one million YouTube videos labeled across 487 sports categories, ranging from common activities like basketball or soccer to highly fine-grained distinctions such as \emph{ultramarathon} versus \emph{half marathon}. The dataset poses unique challenges, as models must not only recognize broad classes of motion but also discriminate subtle variations within closely related activities.

\begin{figure}[H]
    \centering
    \includegraphics[width=0.65\textwidth]{Figures/Chapter_24/slide_35.jpg}
    \caption{Examples from the Sports-1M dataset. For each video, the ground truth label is shown in blue, with the model’s top-5 predictions listed below. Fine-grained distinctions are particularly difficult: for instance, \emph{track cycling} is sometimes misclassified as the broader \emph{cycling}, while in another example the model successfully distinguishes an \emph{ultramarathon} from related classes like \emph{half marathon} and regular \emph{running}.}
    \label{fig:chapter24_sports1m_examples}
\end{figure}

These examples illustrate the dataset’s difficulty: coarse categories are often recognized correctly, but small variations in equipment, environment, or motion patterns can determine the correct label. As a result, fine-grained sports categories highlight the challenge of models trained on video data.

\subsection{Baseline Model Performance}
\label{subsec:chapter24_baselines}

\begin{figure}[H]
    \centering
    \includegraphics[width=0.7\textwidth]{Figures/Chapter_24/slide_36.jpg}
    \caption{Sports-1M performance comparison. The single-frame baseline outperforms early fusion, while late fusion and 3D CNNs yield further improvements. Source: Johnson lecture slides}
    \label{fig:chapter24_sports1m_baselines}
\end{figure}

Karpathy et al.\ \cite{karpathy2014_videocnn} benchmarked several architectures on Sports-1M, comparing single-frame CNNs, early fusion, late fusion, and 3D CNNs (slow fusion). Surprisingly, the \textbf{single-frame baseline} outperformed early fusion, achieving $77.7\%$ accuracy compared to $76.8\%$. Late fusion and 3D CNNs provided modest improvements, with $78.7\%$ and $80.2\%$ respectively.

These results underscore two key insights:
\begin{enumerate}
    \item \textbf{Single-frame models are strong baselines:} even ignoring temporal structure, per-frame CNNs achieve competitive accuracy, making them a practical first step for many applications.
    \item \textbf{Temporal models offer incremental gains:} incorporating temporal reasoning via late fusion or 3D CNNs provides improvements, but the gap is smaller than might be expected.
\end{enumerate}

It is important to note that these experiments date back to 2014, when training resources and architectures were limited (many models were trained on CPU clusters). Since then, 3D CNN architectures and large-scale training pipelines have advanced significantly, so the reported numbers should be interpreted with caution.

\subsection{C3D: The VGG of 3D CNNs}
\label{subsec:chapter24_c3d}

A landmark architecture in early video understanding was the \textbf{C3D network} \cite{tran2015_c3d}, often described as ``the VGG of 3D CNNs''. Recall that VGG for images was built entirely from $3 \times 3$ convolutions and $2 \times 2$ poolings in a simple conv–conv–pool pattern. C3D extended this idea to videos: it used $3 \times 3 \times 3$ convolutions and $2 \times 2 \times 2$ poolings throughout, except in the first pooling layer, which used $1 \times 2 \times 2$ to avoid collapsing the temporal dimension too early. 

This design made C3D a straightforward 3D analog of VGG and an influential baseline in the field. Importantly, the authors released pretrained weights on Sports-1M, and many subsequent works used C3D as a fixed \textbf{video feature extractor}.

\begin{figure}[H]
    \centering
    \includegraphics[width=0.8\textwidth]{Figures/Chapter_24/slide_38.jpg}
    \caption{C3D architecture. Built entirely on $3 \times 3 \times 3$ convolutions and $2 \times 2 \times 2$ poolings (except Pool1). While effective, it is computationally expensive due to volumetric filtering across space and time.}
    \label{fig:chapter24_c3d_architecture}
\end{figure}

\paragraph{Computation cost} The main drawback of C3D is its cost. Even with small inputs (16 frames of size $112 \times 112$), a single forward pass requires nearly 40 GFLOPs:
\begin{itemize}
    \item AlexNet: $0.7$ GFLOPs
    \item VGG-16: $13.6$ GFLOPs
    \item C3D: $39.5$ GFLOPs
\end{itemize}
This stems from sliding 3D kernels over the entire spatiotemporal volume, which scales cubically in kernel size.

\begin{figure}[H]
    \centering
    \includegraphics[width=0.8\textwidth]{Figures/Chapter_24/slide_39.jpg}
    \caption{Performance comparison. On Sports-1M, C3D improves accuracy from $80.2\%$ (earlier 3D CNNs) to $84.4\%$, at the cost of significantly higher computation.}
    \label{fig:chapter24_c3d_results}
\end{figure}

\paragraph{Summary} The story of C3D parallels that of image models: accuracy improved by scaling up deeper, more expensive networks. But these architectures also highlighted the need to treat \emph{time and space differently}, rather than as fully interchangeable.

\section{Separating Time and Space in 3D Processing}
\label{sec:chapter24_sep_time_space}

Humans are capable of recognizing actions from motion cues alone. For example, point-light displays of moving dots are sufficient for us to perceive walking, running, or waving. This suggests that the brain processes \textbf{motion} and \textbf{appearance} in distinct ways. Motivated by this, researchers proposed architectures that explicitly disentangle motion from appearance inside the network.

\subsection{Measuring Motion: Optical Flow}
\label{subsec:chapter24_optical_flow}

A widely used way to represent motion in videos is through \textbf{optical flow}. At a high level, optical flow estimates how points in one frame move to their new positions in the next frame.
\[
F(x,y) = (d_x, d_y), \quad I_{t+1}(x+d_x, y+d_y) \approx I_t(x,y),
\]
The output is a vector field where $(d_x,d_y)$ is the estimated displacement of the pixel at $(x,y)$ from time $t$ to $t+1$. Intuitively, this captures local motion: if an object moves to the right, nearby vectors in the flow field will all point rightward with magnitude proportional to the speed.

\paragraph{Dense vs.\ sparse flow}  
Optical flow can be \emph{dense}, with a displacement vector for every pixel, or \emph{sparse}, with vectors only at keypoints. Dense flow captures detailed motion everywhere, while sparse flow is cheaper and focuses on stable regions.

\paragraph{Why this helps}  
Unlike raw RGB values, which encode only appearance, optical flow provides an explicit description of \emph{how things move}. This allows models to disentangle appearance (what is present) from motion (how it changes). For example, in an action like ``shooting a bow,'' the background may be irrelevant, but the flow highlights the arm and bow movement. Feeding these motion fields into CNNs complements RGB inputs and improves video understanding.

\begin{figure}[H]
    \centering
    \includegraphics[width=0.6\textwidth]{Figures/Chapter_24/slide_43.jpg}
    \caption{Optical flow visualization. Horizontal (top) and vertical (bottom) components for a woman shooting a crossbow. Motion of the arm and bow is clearly highlighted.}
    \label{fig:chapter24_optical_flow}
\end{figure}

\newpage

\subsection{Two-Stream Networks}
\label{subsec:chapter24_two_stream}

A seminal architecture exploiting this idea is the \textbf{two-stream network} of Simonyan and Zisserman \cite{simonyan2014_twostream}. It consists of two parallel CNN branches:

\begin{itemize}
    \item \textbf{Spatial stream:} Processes single RGB frames to capture appearance. Each frame is classified independently, and predictions are averaged over $T$ frames.
    \item \textbf{Temporal stream:} Processes stacked optical flow fields. From $T$ frames, there are $T-1$ optical flows, each with two channels (horizontal and vertical), yielding a tensor of shape $[2(T-1)] \times H \times W$. Early fusion at the first convolution combines motion across frames, followed by standard 2D CNN layers.
\end{itemize}

At test time, both streams output class distributions. The final prediction is obtained by averaging, or by training an SVM over the concatenated outputs.

\begin{figure}[H]
    \centering
    \includegraphics[width=0.7\textwidth]{Figures/Chapter_24/slide_44.jpg}
    \caption{Two-stream architecture \cite{simonyan2014_twostream}. The spatial stream processes RGB frames, while the temporal stream processes stacked optical flows. Predictions are fused at test time.}
    \label{fig:chapter24_two_stream}
\end{figure}

\subsubsection{Evaluation on UCF-101}
\label{subsubsec:chapter24_two_stream_eval}

\begin{figure}[H]
    \centering
    \includegraphics[width=0.7\textwidth]{Figures/Chapter_24/slide_45.jpg}
    \caption{Comparison on UCF-101. Motion information (temporal stream) is crucial. Fusing spatial and temporal streams significantly outperforms either stream alone.}
    \label{fig:chapter24_two_stream_eval}
\end{figure}

The two-stream model was evaluated on the UCF-101 dataset. Results show clear advantages of separating appearance and motion:

\begin{itemize}
    \item 3D CNN: $65.4\%$
    \item Spatial-only stream: $73.0\%$
    \item Temporal-only stream: $83.7\%$
    \item Two-stream, average fusion: $86.9\%$
    \item Two-stream, SVM fusion: $88.0\%$
\end{itemize}

These results highlight that motion is often more informative than raw appearance, but the best performance arises when both are combined.

\section{Modeling Long-Term Temporal Structure}
\label{sec:chapter24_long_term}

Most architectures discussed so far capture only \emph{local} temporal patterns: 2D or 3D CNNs operate on short clips of $\sim$16--32 frames. Many tasks, however, require reasoning about \emph{long-term dependencies}, where informative events are separated by seconds or minutes. We therefore seek models that aggregate information across extended time spans while preserving strong spatial representations.

\subsection{CNN Features + Recurrent Networks}
\label{subsec:chapter24_cnn_rnn}

A practical recipe is to pair CNNs for spatial and short-term modeling with RNNs:
\begin{enumerate}
    \item Extract per-timestep features with a CNN (2D on frames or 3D on short clips), yielding a feature vector at each step.
    \item Feed the feature sequence to a recurrent model (e.g., LSTM) to aggregate over time.
    \item For video-level classification, use a many-to-one mapping from the final hidden state; for dense labeling, use many-to-many by reading out from all hidden states.
\end{enumerate}

This idea appeared early in Baccouche et al.\ \cite{baccouche2011_seqdl} and was popularized by Donahue et al.\ with Long-term Recurrent Convolutional Networks (LRCN) \cite{donahue2015_ltrcnn}. A memory-efficient variant freezes the clip-level CNN (e.g., C3D) and trains only the RNN to cover long time horizons without backpropagating through very long video volumes.

\begin{figure}[H]
    \centering
    \includegraphics[width=0.65\textwidth]{Figures/Chapter_24/slide_53.jpg}
    \caption{Hybrid CNN+RNN pipeline. A frozen C3D-like network produces per-step features which an LSTM aggregates; the final hidden state yields a video-level prediction.}
    \label{fig:chapter24_cnn_rnn}
\end{figure}

\paragraph{From vector RNNs to recurrent convs} Multi-layer RNNs stack temporal processing; the state at time $t$ in layer $l$ depends on the state at $(t{-}1,l)$ and on input from $(t,l{-}1)$. The same idea can be applied \emph{inside} convolutional networks by replacing matrix multiplications with convolutions, yielding recurrent convolutional networks in which each spatial location behaves like a tiny RNN through time \cite{ballas2016_delving}.

\begin{figure}[H]
    \centering
    \includegraphics[width=0.85\textwidth]{Figures/Chapter_24/slide_56.jpg}
    \caption{Recurrent convolutional network schematic. Each feature map $\mathbf{F}_t^{(l)}$ depends on the previous time at the same layer and the previous layer at the same time; weights are shared across time.}
    \label{fig:chapter24_recurrent_conv}
\end{figure}

\paragraph{Gated variants and practicality} As with standard sequence models, one can replace simple recurrences with GRU or LSTM-style \emph{convolutional} gates. While elegant, such models inherit the sequential dependency of RNNs, limiting parallelism and slowing training on long videos.

\begin{figure}[H]
    \centering
    \includegraphics[width=0.7\textwidth]{Figures/Chapter_24/slide_64.jpg}
    \caption{Ways to process sequences. CNNs capture local context; RNNs aggregate sequentially; self-attention relates all positions directly.}
    \label{fig:chapter24_sequence_options}
\end{figure}

\subsection{Spatio-Temporal Self-Attention and the Nonlocal Block}
\label{subsec:chapter24_nonlocal_block}

Standard 3D CNNs operate on local neighborhoods in space and time; relating distant events requires many layers to propagate information. To address this, Wang et al.\ \cite{wang2018_nonlocal_nn} proposed the \textbf{nonlocal block}, a spatio-temporal self-attention module that directly connects \emph{all} positions in a video volume.

\paragraph{Definition} Given input features:
\begin{equation}
    \label{eq:chapter24_nonlocal_input}
    \mathbf{X} \in \mathbb{R}^{C \times T \times H \times W},
\end{equation}
the block computes queries, keys, and values via $1{\times}1{\times}1$ convolutions,
\begin{equation}
    \label{eq:chapter24_nonlocal_qkv}
    \mathbf{Q},\mathbf{K},\mathbf{V} \in \mathbb{R}^{C' \times T \times H \times W},
\end{equation}
flattens space–time so $N{=}T\!\cdot\!H\!\cdot\!W$, forms affinities
\begin{equation}
    \label{eq:chapter24_nonlocal_attn}
    \mathbf{A}=\text{softmax}\!\big(\mathbf{Q}^\top \mathbf{K}\big)\in\mathbb{R}^{N\times N},\quad \sum_j \mathbf{A}_{ij}=1,
\end{equation}
aggregates values
\begin{equation}
    \label{eq:chapter24_nonlocal_agg}
    \mathbf{Y}=\mathbf{V}\,\mathbf{A}^\top\in\mathbb{R}^{C'\times N},
\end{equation}
reshapes back to $C'\times T\times H\times W$, projects to $C$ channels with $W_z$, and adds a residual:
\begin{equation}
    \label{eq:chapter24_nonlocal_residual}
    \mathbf{Z}=W_z(\mathbf{Y})+\mathbf{X}.
\end{equation}

\begin{figure}[H]
    \centering
    \includegraphics[width=0.85\textwidth]{Figures/Chapter_24/slide_72.jpg}
    \caption{Nonlocal block \cite{wang2018_nonlocal_nn}. Each output location attends to and aggregates information from all spatio-temporal positions, enabling direct long-range reasoning.}
    \label{fig:chapter24_nonlocal_block}
\end{figure}

\paragraph{Initialization and integration} For stable insertion into 3D CNNs, initialize the final projection so the block starts as identity; in practice, place a BatchNorm after the last $1{\times}1{\times}1$ and initialize its scale to zero. This yields \emph{slow fusion} via local 3D convolutions plus \emph{global fusion} via nonlocal attention.

\begin{figure}[H]
    \centering
    \includegraphics[width=0.85\textwidth]{Figures/Chapter_24/slide_73.jpg}
    \caption{3D CNN augmented with nonlocal blocks. Local slow fusion is complemented with global all-to-all fusion across space and time.}
    \label{fig:chapter24_nonlocal_integration}
\end{figure}

\paragraph{Takeaway} Nonlocal blocks overcome locality constraints of convolutions and the sequential bottleneck of RNNs by enabling each position to directly gather context from anywhere in the video, improving representations for tasks such as action recognition and video classification.

\subsection{Inflating 2D Networks to 3D (I3D)}
\label{subsec:chapter24_i3d}

Designing effective 3D CNNs from scratch is costly. \textbf{I3D} \cite{carreira2017_i3d} addresses this by \emph{inflating} a strong 2D architecture (e.g., Inception-v1) into 3D so it can process space and time while reusing ImageNet-pretrained weights. The core idea is twofold:
\begin{itemize}
    \item \textbf{Inflate the architecture:} add a temporal extent $K_t$ to every operation (convolutions, pooling, etc.), turning $K_h{\times}K_w$ kernels into $K_t{\times}K_h{\times}K_w$.
    \item \textbf{Inflate the weights:} initialize 3D kernels from pretrained 2D kernels by replicating them along the temporal dimension and scaling by $1/K_t$, so the inflated network behaves identically to the 2D parent on static videos.
\end{itemize}

\paragraph{Inflating the architecture}
Every 2D layer is given an explicit temporal kernel size $K_t$:
\begin{itemize}
    \item $K_h{\times}K_w$ conv $\Rightarrow$ $K_t{\times}K_h{\times}K_w$ conv, same for pooling.
    \item Inception branches and residual pathways are expanded analogously, preserving topology and receptive-field design.
    \item Temporal stride and padding are chosen to control temporal downsampling and receptive-field growth, mirroring spatial design.
\end{itemize}

\begin{figure}[H]
    \centering
    \includegraphics[width=0.85\textwidth]{Figures/Chapter_24/slide_76.jpg}
    \caption{Inflating an Inception block to 3D \cite{carreira2017_i3d}. Spatial operators acquire a temporal extent (bolded), e.g., $3{\times}3$ pooling becomes $3{\times}3{\times}3$}
    \label{fig:chapter24_i3d_block}
\end{figure}

\paragraph{Inflating the weights: replication and normalization}
Let a 2D filter be $W_{2D}\!\in\!\mathbb{R}^{C_\text{out}\times C_\text{in}\times K_h\times K_w}$ and its inflated 3D filter be $W_{3D}\!\in\!\mathbb{R}^{C_\text{out}\times C_\text{in}\times K_t\times K_h\times K_w}$. I3D initializes
\begin{equation}
    \label{eq:chapter24_i3d_weight_inflate}
    W_{3D}[:,:,t,:,:] \;=\; \tfrac{1}{K_t}\, W_{2D} \quad \text{for } t=1,\dots,K_t.
\end{equation}
That is, \emph{replicate} the 2D kernel along time and \emph{divide by $K_t$}. The division prevents an unintended $K_t$-fold amplification of responses.

\paragraph{Why divide by $K_t$}
Consider a \emph{static video} $I$ where every frame is identical. A 3D convolution with the replicated kernel computes a temporal sum of identical 2D responses. Without normalization,
\[
\text{Conv3D}(W_{3D}, I) \;=\; \sum_{t=1}^{K_t}\text{Conv2D}(W_{2D}, I) \;=\; K_t\,\text{Conv2D}(W_{2D}, I).
\]
Scaling by $1/K_t$ in \eqref{eq:chapter24_i3d_weight_inflate} cancels this factor, yielding
\[
\text{Conv3D}(W_{3D}, I) \;=\; \text{Conv2D}(W_{2D}, I),
\]
so the inflated 3D layer is \emph{exactly equivalent} to the original 2D layer on static inputs. This preserves activation magnitudes and the semantics of pretrained features at initialization, which is crucial for stability with BatchNorm and deep stacks.

\begin{figure}[H]
    \centering
    \includegraphics[width=0.7\textwidth]{Figures/Chapter_24/slide_77.jpg}
    \caption{Weight inflation \cite{carreira2017_i3d}. 2D kernels are replicated across the temporal axis and scaled by $1/K_t$ so that responses on static videos match the 2D parent network}
    \label{fig:chapter24_i3d_weights}
\end{figure}

\paragraph{Why inflation is a natural fit}
Videos contain the same \emph{spatial} structures as images (edges, textures, objects), now evolving \emph{over time}. Inflation transfers mature spatial detectors from 2D while introducing a neutral temporal prior (identical slices). During fine-tuning, backpropagation learns temporal asymmetries across slices (e.g., detectors of motion direction or temporal phase), turning static spatial filters into motion-sensitive spatiotemporal filters. Thus, optimization focuses on \emph{temporal} modeling rather than relearning spatial basics.

\paragraph{Evidence on Kinetics-400}
On Kinetics-400 \cite{kay2017_kinetics} (300K ten-second YouTube clips across 400 actions), Carreira and Zisserman showed that, with the same Inception-v1 backbone, inflating ImageNet-pretrained weights outperforms training 3D kernels from scratch.

\begin{figure}[H]
    \centering
    \includegraphics[width=0.75\textwidth]{Figures/Chapter_24/slide_78.jpg}
    \caption{Pretraining and inflation on Kinetics-400 \cite{carreira2017_i3d,kay2017_kinetics}. For identical Inception-v1 backbones, Inflated CNN trained from scratch achieves 68.4\% top-1, whereas inflation from ImageNet-pretrained weights reaches 71.1\%; two-stream I3D attains 74.2\%}
    \label{fig:chapter24_i3d_kinetics}
\end{figure}

\paragraph{Takeaway}
I3D provides a principled initialization with several benefits:
\begin{itemize}
    \item \textbf{Equivalence on static inputs:} the inflated network is provably identical to its 2D parent when frames are constant, ensuring stable initialization.
    \item \textbf{Spatial competence transfer:} pretrained image filters (e.g., from ImageNet) provide strong recognition of edges, textures, and objects without retraining.
    \item \textbf{Focus on temporal dynamics:} since spatial features are inherited, optimization capacity can concentrate on learning motion-sensitive filters.
\end{itemize}
Together, these properties make inflation a strong, data-efficient baseline and a reliable foundation for higher-performing video models.

\subsection{Transformers for Video Understanding}
\label{subsec:chapter24_video_transformers}

Transformers capture long-range \emph{spatio–temporal} structure by self-attending over a sequence of tokens. For a clip $\mathbf{X}\!\in\!\mathbb{R}^{T\times H\times W\times C}$, the video volume is first mapped to $N$ tokens $\{z_i\}_{i=1}^N$, then multi-head self-attention (MHSA) relates tokens across space and time \cite{bertasius2021_timesformer,arnab2021_vivit,neimark2021_vtn,fan2021_mvit,li2021_improved_mvit}. Two core design choices govern effectiveness and efficiency: \emph{how to tokenize} the video, and \emph{how to structure attention} so compute and memory remain tractable.

\paragraph{What is a token in video}
Video tokens are compact \emph{spatiotemporal} units, not raw pixels:
\begin{itemize}
    \item \textbf{Per-frame patches}: Split each frame into $P{\times}P$ patches; the sequence length is $N=T\cdot \frac{HW}{P^2}$ \cite{arnab2021_vivit}.
    \item \textbf{Tubelets} (3D patches): Split the video into cuboids of size $P_t{\times}P{\times}P$, reducing $N$ by $\approx P_t$ and embedding short-term motion at input \cite{arnab2021_vivit,bertasius2021_timesformer}.
    \item \textbf{CNN feature tokens}: Use a 2D CNN per frame and treat spatial feature-map locations as tokens, leveraging ImageNet pretraining and curbing $N$ \cite{neimark2021_vtn}.
\end{itemize}
Tokens are linearly projected to $\mathbb{R}^d$ and enriched with \emph{space–time} positional information (absolute or relative).

\paragraph{Attention over space and time}
Full joint attention over all $N$ tokens costs $O(N^2)$; with per-frame patches $N=n_t n_h n_w$ grows multiplicatively in frames $n_t$ and spatial grid $n_h\times n_w$ ($n_h=H/P$, $n_w=W/P$). For $T{=}32$, $H{=}W{=}224$, $P{=}16$, we obtain $N=6272$ and $\sim39$M pairwise interactions \emph{per layer per head}. To scale, modern designs either \textbf{factorize} attention or \textbf{pool} tokens:
\begin{itemize}
    \item \textbf{Divided space–time attention} (TimeSformer): Perform \emph{spatial} attention within each frame, then \emph{temporal} attention across frames at corresponding spatial sites, reducing cost from $O((n_t n_h n_w)^2)$ to $O\!\big(n_t(n_h n_w)^2 + n_h n_w\,n_t^2\big)$ with strong accuracy \cite{bertasius2021_timesformer}.
    \item \textbf{Multiscale transformers} (MViT/MViT-v2): Progressively \emph{pool} tokens in space/time while widening channels, so deeper layers attend over fewer tokens; pooling attention with relative position biases yields excellent accuracy–efficiency trade-offs \cite{fan2021_mvit,li2021_improved_mvit}.
    \item \textbf{CNN–Transformer hybrids} (VTN): Adopt a 2D CNN stem for spatial encoding and use \emph{temporal}-only transformers on top, exploiting image pretraining and avoiding token explosion \cite{neimark2021_vtn}.
\end{itemize}

\paragraph{ViViT in depth: tokenization, factorization, computation, and findings}
\textbf{ViViT} \cite{arnab2021_vivit} provides a clear blueprint for video transformers, isolating tokenization and attention structure as independent axes.

\subparagraph{Tokenization}
ViViT studies (i) \emph{Per-frame patches} with uniform frame sampling ($N=n_t n_h n_w$) and (ii) \emph{Tubelet embedding} into $P_t{\times}P{\times}P$ cuboids ($N=\lfloor T/P_t\rfloor n_h n_w$). Tubelets reduce $N$ linearly in $P_t$ and inject a motion prior. Initialization matters: inflating 2D patch projections (replicate across $P_t$ and scale) or central-frame initialization stabilizes training, echoing I3D’s weight inflation.

\subparagraph{What ``spatial'' and ``temporal'' transformers mean}
In ViViT’s \emph{factorized} designs, attention neighborhoods are restricted:
\begin{itemize}
    \item A \textbf{Spatial transformer} attends \emph{within frames} to learn objects and layouts; frames can be processed in parallel.
    \item A \textbf{Temporal transformer} attends \emph{across frames} at aligned spatial sites (or on frame-level summaries) to learn motion and ordering.
\end{itemize}
These are standard ViT blocks (MHSA+MLP+residuals); only the token grouping changes.

\subparagraph{Architectural variants and compute}
ViViT compares four designs that trade expressivity for efficiency by constraining \emph{who attends to whom}. With per-frame patches $N=n_t n_h n_w$ ($n_t$ frames, $n_h{\times}n_w$ patches per frame), joint attention costs $O(N^2)$; factorized variants decompose this into spatial and temporal parts while preserving the standard Transformer block (MHSA$\rightarrow$MLP with residuals and normalization).
\begin{enumerate}
    \item \textbf{Joint spatiotemporal attention}: All tokens attend to all others across space and time; maximally expressive but $O(N^2)$, practical only for short clips or coarse patching.
    \item \textbf{Factorized encoder}: Spatial-only transformers process each frame to produce frame embeddings, then a temporal-only transformer aggregates across frames; $\approx O\!\big(n_t(n_h n_w)^2\big) + O\!\big(n_h n_w\,n_t^2\big)$ and spatial stages parallelize over frames.
    \item \textbf{Factorized self-attention}: Within each block, apply spatial attention (within-frame) then temporal attention (across-frame at aligned sites); similar complexity to the factorized encoder with different information flow and regularization.
    \item \textbf{Factorized dot-product attention}: Split attention heads into spatial-only and temporal-only inside a joint block, keeping parameter count while shrinking effective neighborhoods and compute.
\end{enumerate}
With tubelets, $n_t \leftarrow \lfloor T/P_t \rfloor$, so the temporal term $O(n_h n_w\,n_t^2)$ becomes $O\!\big(n_h n_w\,(T/P_t)^2\big)$, explaining why modest $P_t$ yields substantial savings without sacrificing short-range motion cues.

\subparagraph{Positioning relative to contemporaries}
\begin{itemize}
    \item \textbf{TimeSformer} \cite{bertasius2021_timesformer}: Also factorizes space–time within blocks; ViViT broadens the design space (encoder- vs.\ block-level factorization, tubelets, initialization) and clarifies trade-offs.
    \item \textbf{MViT/MViT-v2} \cite{fan2021_mvit,li2021_improved_mvit}: Add hierarchical token pooling and pooling attention with relative biases for strong accuracy–efficiency; ViViT serves as a transparent baseline isolating tokenization and factorization without a pyramid.
    \item \textbf{VTN} \cite{neimark2021_vtn}: Uses a 2D CNN spatial stem with temporal transformers to curb tokens and leverage image pretraining; ViViT shows pure-transformer backbones can compete when tokenization and factorization are well chosen.
\end{itemize}

\subparagraph{Practical guidance and empirical takeaways from ViViT}
ViViT’s systematic study suggests clear design choices for building effective and efficient video transformers:
\begin{itemize}
    \item \textbf{Prefer tubelets}: Use modest temporal extent $P_t\!\in\![2,4]$ to cut tokens, reduce FLOPs, and inject local motion cues. Tubelets generally outperform per-frame patches at matched compute.
    \item \textbf{Adopt factorization for scale}: Factorized encoders or block-level space–then–time attention retain most of joint attention’s accuracy while allowing longer clips and higher spatial resolution within a fixed budget.
    \item \textbf{Encode space–time position}: Apply factorized absolute or relative positional signals.
    \item \textbf{Leverage large pretraining}: Large-scale image pretraining (e.g., ImageNet-21K/JFT) is essential, since training pure video transformers from scratch on modest video datasets underperforms.
    \item \textbf{Fewer multi-view passes needed}: Efficient factorization makes it possible to process longer clips in a single forward pass, reducing reliance on expensive multi-view testing.
\end{itemize}

\begin{figure}[H]
    \centering
    \includegraphics[width=0.75\textwidth]{Figures/Chapter_24/slide_79.jpg}
    \caption{ViViT overview \cite{arnab2021_vivit}. Videos are tokenized by per-frame patches or tubelets, enriched with space–time positions, and processed by joint or factorized attention. Factorized designs reduce attention from $O((n_t n_h n_w)^2)$ to $O((n_h n_w)^2 + n_t^2)$ while retaining strong accuracy.}
    \label{fig:chapter24_vivit_overview}
\end{figure}

\paragraph{Why transformers for video}
Transformers provide a \emph{global} spatiotemporal receptive field in a single layer via content-based self-attention, allowing direct connections between distant events without the deep local stacking of 3D CNNs or the sequential bottlenecks of RNNs. While naive all-to-all attention over $N$ video tokens costs $O(N^2)$, practical video transformers curb both $N$ and the attention neighborhoods through \textbf{tokenization} into tubelets (reducing sequence length and injecting short-range motion cues), \textbf{attention factorization} (space-then-time or encoder-level separation), and \textbf{multiscale pooling} (progressively merging tokens while widening channels), achieving long-range reasoning at tractable compute \cite{bertasius2021_timesformer,arnab2021_vivit,fan2021_mvit,li2021_improved_mvit,neimark2021_vtn}. The result is a backbone that preserves temporal reasoning capacity while scaling to longer clips and higher resolutions within realistic budgets.

\subsection{Visualizing and Localizing Actions}
\label{subsec:chapter24_visualization_localization}

\subsubsection{Visualizing Video Models}
\label{subsec:chapter24_visualizing_models}

A useful way to probe what a trained video classifier has learned is to \emph{optimize a synthetic video} $\mathbf{V}\!\in\!\mathbb{R}^{C\times T\times H\times W}$ to maximize a class score $S_c(\mathbf{V})$ while adding priors that favor naturalistic solutions \cite{feichtenhofer2018_deepreps,feichtenhofer2019_deepinsights}. A generic objective is
\begin{equation}
    \label{eq:chapter24_vis_objective}
    \max_{\mathbf{V}} \; S_c(\mathbf{V}) \;-\; \lambda_s \,\mathcal{R}_{\text{space}}(\mathbf{V}) \;-\; \lambda_t \,\mathcal{R}_{\text{time}}(\mathbf{V}),
\end{equation}
where $\mathcal{R}_{\text{space}}$ encourages spatial smoothness (e.g., spatial total variation) and $\mathcal{R}_{\text{time}}$ encourages temporal coherence (e.g., penalties on finite differences across adjacent frames). By \emph{tuning} the temporal penalty $\lambda_t$, one can bias the optimized video toward \textbf{slow} motion (large $\lambda_t$ suppresses rapid frame-to-frame changes) or \textbf{fast} motion (small $\lambda_t$ allows rapid changes). This separates \emph{appearance cues} (what) from \emph{motion regimes} (how fast), revealing complementary evidence the model uses.

\begin{figure}[H]
    \centering
    \includegraphics[width=0.85\textwidth]{Figures/Chapter_24/slide_81.jpg}
    \caption{Visualizing video models with spatiotemporal regularization \cite{feichtenhofer2018_deepreps,feichtenhofer2019_deepinsights}. Increasing the temporal smoothness weight highlights slow components; decreasing it exposes fast components.}
    \label{fig:chapter24_visualize_overview}
\end{figure}

\paragraph{Qualitative examples}
Optimizing \eqref{eq:chapter24_vis_objective} for specific classes yields intuitive decompositions into \emph{appearance}, \emph{slow} motion, and \emph{fast} motion channels:
\begin{itemize}
    \item \textbf{Weightlifting}: The appearance channel emphasizes the barbell and lifter; the \emph{slow} component accentuates bar shaking; the \emph{fast} component emphasizes the push overhead—together aligning with the \emph{weightlifting} concept.
    \item \textbf{Apply eye makeup}: The appearance channel contains many faces (consistent with makeup tutorials); the \emph{slow} component captures deliberate hand movements; the \emph{fast} component highlights brushing strokes.
\end{itemize}

\begin{figure}[H]
    \centering
    \includegraphics[width=0.85\textwidth]{Figures/Chapter_24/slide_83.jpg}
    \caption{Visualization by class score optimization \cite{feichtenhofer2018_deepreps,feichtenhofer2019_deepinsights}. Appearance, slow, and fast components for a weightlifting clip emphasize barbell, bar shaking, and push overhead respectively.}
    \label{fig:chapter24_visualize_weightlifting}
\end{figure}

\begin{figure}[H]
    \centering
    \includegraphics[width=0.85\textwidth]{Figures/Chapter_24/slide_85.jpg}
    \caption{Visualization by class score optimization \cite{feichtenhofer2018_deepreps,feichtenhofer2019_deepinsights}. For \emph{apply eye makeup}, appearance surfaces faces, slow motion emphasizes hand placement, and fast motion highlights brushing strokes.}
    \label{fig:chapter24_visualize_makeup}
\end{figure}

\subsubsection{Temporal Action Localization}
\label{subsec:chapter24_temporal_localization}

\textbf{Problem:} Given an untrimmed video, identify the \emph{temporal extents} of actions and their labels. A popular approach mirrors object detection: first generate \emph{temporal proposals}, then classify and refine them \cite{chao2018_tal}. Modern systems use 1D temporal anchors or boundary-matching modules coupled with clip-level features from 2D/3D backbones.

\begin{figure}[H]
    \centering
    \includegraphics[width=0.85\textwidth]{Figures/Chapter_24/slide_87.jpg}
    \caption{Temporal action localization. Proposal generation followed by classification and boundary refinement identifies action segments in long untrimmed videos \cite{chao2018_tal}.}
    \label{fig:chapter24_temporal_localization}
\end{figure}

\subsubsection{Spatio-Temporal Action Detection}
\label{subsec:chapter24_spatiotemporal_detection}

\textbf{Problem:} Detect \emph{who does what} in space and time: localize people with bounding boxes across frames (tubes) and assign action labels. The \textbf{AVA} dataset provides dense, frame-level annotations of \emph{atomic visual actions} for people in 15-minute movie clips, enabling research on fine-grained spatiotemporal detection and interaction understanding \cite{gu2018_ava}. Models typically combine per-frame person detection, tube linking, and action classification with temporal context.

\begin{figure}[H]
    \centering
    \includegraphics[width=0.85\textwidth]{Figures/Chapter_24/slide_88.jpg}
    \caption{Spatio-temporal detection examples from AVA \cite{gu2018_ava}. Activities such as clinking glass, drinking, looking at phone, or answering phone are localized in space and time for each person.}
    \label{fig:chapter24_ava_examples}
\end{figure}

\subsubsection{Ego4D: Large-Scale Egocentric Video}
\label{subsec:chapter24_ego4d}

\textbf{Ego4D} is a comprehensive egocentric benchmark comprising \textbf{3{,}670 hours} of head-mounted, real-world video collected by \textbf{14} teams across \textbf{9} countries from \textbf{931} camera wearers \cite{grauman2022_ego4d}. Videos are long (\textbf{1–10 hours} each) and accompanied by \textbf{3.85M} natural language narrations. The dataset supports five task families:
\begin{itemize}
    \item \textbf{Episodic memory}: Retrieve or localize past events based on queries.
    \item \textbf{Hands and objects}: Detect and track hands and manipulated objects from a first-person perspective.
    \item \textbf{Audio–video diarization}: Segment and attribute audio–visual events to speakers and sources.
    \item \textbf{Social interactions}: Recognize and characterize interpersonal behaviors.
    \item \textbf{Forecasting}: Anticipate future activities or states from ongoing observations.
\end{itemize}

\begin{figure}[H]
    \centering
    \includegraphics[width=0.85\textwidth]{Figures/Chapter_24/slide_89.jpg}
    \caption{Ego4D overview \cite{grauman2022_ego4d}. A global, long-form egocentric video corpus with narrations and benchmarks spanning episodic memory, hands and objects, audio–video diarization, social interactions, and forecasting.}
    \label{fig:chapter24_ego4d}
\end{figure}

\newpage

\begin{enrichment}[Vision--Language Alignment Precursors][section]
    \label{enr:sec_chapter24_vlprecursors}
    The first step toward video--language models was learning how to connect vision and language at scale. As detailed in Section~\ref{subsec:chapter22_ssl_clip}, \emph{CLIP} demonstrated how contrastive alignment could map visual features into a shared language space. Building on this foundation, \emph{SigLIP} and the \emph{BLIP} family established the now-standard connector paradigm for mapping visual encoders into LLM-friendly representations. This section focuses on the key image--language precursors that underpin later video systems: \emph{SigLIP} for improved contrastive alignment, \emph{BLIP} and \emph{BLIP-2} for lightweight vision--LLM bridging, and \emph{SigLIP~2} as a stronger, multilingual and dense-capable successor.
    
    \begin{enrichment}[SigLIP: Contrastive Alignment with Sigmoid Loss][subsection]
        \label{enr:subsec_chapter24_siglip}
        
        \paragraph{From CLIP to SigLIP (Intuition First)}
        \emph{CLIP} learns with a \emph{batch–softmax} game: in each row/column of the similarity matrix, the true pair must \emph{beat all} in-batch negatives. This global competition is powerful, but it ties learning quality to batch composition (you need many, diverse negatives), forces expensive all–gathers across devices, and becomes fragile with small or imbalanced batches.  
        
        \emph{SigLIP} \cite{zhai2023_siglip} changes the game: instead of “one-vs-many” races, it asks a simple \emph{yes/no} question for \emph{each} image–text pair—“do they match?”—and trains with a \textbf{pairwise sigmoid (logistic) loss}. By turning alignment into many independent binary decisions, SigLIP:
        \begin{itemize}
            \item \textbf{Decouples} supervision from batch size (every off-diagonal pair is a labeled negative, no global normalization needed),
            \item \textbf{Stabilizes} gradients (no row/column softmax where a few hard negatives dominate),
            \item \textbf{Improves calibration} (scores behave like probabilities rather than “who won the batch”),
            \item \textbf{Cuts memory \& comms} (no all–gathers to normalize across the full batch).
        \end{itemize}
        
        \paragraph{Algorithmic Formulation and Intuition}
        With $n$ image embeddings $x_i$ and text embeddings $y_j$ (both L2-normalized), \textbf{CLIP} builds $S_{ij}{=}x_i^\top y_j$ and optimizes two batch–softmax losses (image$\!\to\!$text and text$\!\to\!$image):
        \[
        \mathcal{L}_{\text{CLIP}}
        =\frac{1}{2}\!\left[
        \frac{1}{n}\sum_{i=1}^{n}\!-\log\frac{\exp(\tau S_{ii})}{\sum_{j=1}^{n}\exp(\tau S_{ij})}
        +
        \frac{1}{n}\sum_{j=1}^{n}\!-\log\frac{\exp(\tau S_{jj})}{\sum_{i=1}^{n}\exp(\tau S_{ij})}
        \right],
        \]
        where the learned temperature $\tau$ sharpens the softmax; each positive must outrank all $n{-}1$ negatives in its row/column.
        
        \textbf{SigLIP} replaces this global competition with a \emph{per-pair} logistic objective:
        \begin{equation}
            \label{eq:chapter24_siglip_loss}
            \mathcal{L}_{\text{SigLIP}}
            = -\frac{1}{n}\sum_{i=1}^{n}\sum_{j=1}^{n}
            \log\sigma\!\big(z_{ij}\cdot(t\,x_i^\top y_j + b)\big),
        \end{equation}
        with labels $z_{ij}{=}1$ for matches ($i{=}j$) and $-1$ otherwise (non-match). The formulation introduces two additional learnable scalars:
        \begin{itemize}
            \item \textbf{Temperature $t=\exp(t')$.} Instead of learning $t$ directly, the model learns an unconstrained parameter $t'$, which is exponentiated to ensure $t>0$. This acts as a \emph{similarity sharpness knob}: larger $t$ magnifies dot products, steepening the logistic curve and pushing probabilities closer to $0$ or $1$; smaller $t$ smooths the curve, reducing overconfidence. Exponentiation guarantees stability while allowing flexible scaling during training.
            \item \textbf{Bias $b$.} A learnable offset that shifts the decision boundary of the sigmoid. It helps correct for the extreme class imbalance of the loss: each minibatch has only $n$ positives but $n^2-n$ negatives. Without $b$, the logits for negatives can dominate early optimization, leading to vanishing gradients for positives.
        \end{itemize}
        
        \emph{Reading the terms in context:}
        \begin{itemize}
            \item $x_i^\top y_j$: cosine-like similarity between L2-normalized embeddings.
            \item $t=\exp(t')$: positive temperature that scales similarities, controlling how confidently pairs are classified.
            \item $b$: bias that shifts the sigmoid’s threshold, stabilizing optimization when negatives vastly outnumber positives.
            \item $z_{ij}\in\{+1,-1\}$: binary label, turning alignment into independent logistic decisions for each pair—no competition across rows/columns as in CLIP.
        \end{itemize}
        
        \paragraph{CLIP vs.\ SigLIP—why it matters}
        \begin{itemize}
            \item \textbf{Normalization target.} CLIP normalizes within each row/column via softmax (needs the whole batch); SigLIP applies a sigmoid per pair (no batchwise denominator).
            \item \textbf{Negatives.} CLIP’s signal hinges on the number/hardness of in-batch negatives; SigLIP gets explicit negatives from all off-diagonal pairs, even in modest batches.
            \item \textbf{Gradient coupling.} CLIP couples all pairs in a row/column (hard negatives can dominate); SigLIP yields \emph{decoupled per-pair} gradients with lower variance.
            \item \textbf{Calibration.} CLIP scores reflect “winning the batch”; SigLIP’s probabilities are directly interpretable as match likelihoods.
            \item \textbf{Distributed cost.} CLIP typically needs global all–gathers; SigLIP can be computed in device-local tiles (see the below part on efficient computation).
        \end{itemize}
        
        \begin{mintedbox}{python}
            # Sigmoid contrastive loss pseudocode (SigLIP)
            # img_emb : image embeddings [n, d]
            # txt_emb : text embeddings  [n, d]
            # t_prime, b : learnable temperature and bias
            # n : batch size
            
            t = exp(t_prime)
            zimg = l2_normalize(img_emb)
            ztxt = l2_normalize(txt_emb)
            logits = dot(zimg, ztxt.T) * t + b
            
            labels = 2 * eye(n) - ones(n)   # +1 on diag (matches), -1 off-diag (non-matches)
            loss = -sum(log_sigmoid(labels * logits)) / n
        \end{mintedbox}
        
        \paragraph{Efficient Implementation}
        The pairwise objective also simplifies distributed training. CLIP’s softmax normalizes over the global batch and thus materializes an $n{\times}n$ similarity matrix across devices via all-gathers. SigLIP computes the loss \emph{locally} in chunked blocks, avoiding global normalization and keeping only device-resident tiles in memory. The footprint drops from $O(n^2)$ to $O(b^2)$, where $b$ is the per-device batch size, enabling very large effective batches on comparatively few accelerators. The below figure illustrates the blockwise computation.
        
        \begin{figure}[H]
            \centering
            \includegraphics[width=0.85\textwidth]{Figures/Chapter_24/siglip_efficient_loss_computation.jpg}
            \caption{SigLIP computes the sigmoid loss over device-local blocks, avoiding global all-gathers required by CLIP’s batch–softmax. Source: \cite{zhai2023_siglip}.}
            \label{fig:chpapter24_chapter24_siglip_lossimpl}
        \end{figure}
        
        \paragraph{Empirical Comparison to CLIP: What Improves in Practice}
        In realistic training settings (small–to–medium batches; noisy web data), \textbf{SigLIP} generally \emph{matches or surpasses} CLIP while requiring less tuning. The improvements are explained by its pairwise design:
        \begin{itemize}
            \item \textbf{Batch-size resilience.} Because supervision is per pair, SigLIP does not need thousands of negatives per update. Performance scales smoothly up to moderate batch sizes and then plateaus, avoiding CLIP’s reliance on extreme global batches.
            \item \textbf{Lower gradient variance.} Without a row/column softmax, updates are not dominated by a few hard negatives, yielding smoother optimization and more stable convergence.
            \item \textbf{More reliable confidence.} Logistic outputs can be interpreted directly as “probability of match”. This leads to better-calibrated similarity scores, making confidence thresholds more trustworthy for retrieval, filtering, or dataset cleaning.
            \item \textbf{Robustness to noise.} In CLIP, mislabeled or loosely aligned pairs can distort the softmax normalization for a whole row/column. In SigLIP, such outliers only affect their own binary terms, containing the damage and improving robustness on noisy web corpora.
            \item \textbf{Efficiency.} Losses are computed locally on each device in small blocks, avoiding global all-gathers. This reduces memory and communication costs and makes very large effective batches feasible even on limited hardware.
        \end{itemize}
        
        \paragraph{Impact, Limitations, and Legacy}
        \textbf{Impact.} SigLIP proved that large-scale vision–language alignment can be achieved without global softmax or massive negatives. Its simple, stable recipe made it the backbone for connector-style systems such as \emph{BLIP/BLIP-2} (where Q-Former bridges vision encoders to LLMs) and \emph{Video-LLMs} (where temporal encoders extend SigLIP-style connectors to video).
        
        \textbf{Limitations.} As a purely \emph{binary contrastive} method, SigLIP:
        \begin{itemize}
            \item Judges only match vs.\ non-match, lacking multi-way semantics or compositional reasoning.
            \item Aligns globally but does not yield dense/localized features unless augmented.
            \item Cannot generate captions or reasoning without an attached LLM.
        \end{itemize}
        
        \textbf{Legacy.} Extensions such as \emph{SigLIP~2}~\cite{tschannen2025_siglip2} add multilingual training, masked prediction, and self-distillation for cross-lingual and localized tasks. 
        
    \end{enrichment}
    
    \newpage
    
    \begin{enrichment}[BLIP: Bootstrapping Language--Image Pretraining][subsection]
        \label{enr:subsec_chapter24_blip}
        
        \paragraph{High-Level Idea}
        Most large-scale vision--language corpora are scraped from the web by pairing images with their surrounding \textbf{alt-text}---short strings originally written for accessibility or indexing. While attractive for scale, alt-text was never intended as faithful supervision. It is often:
        \begin{itemize}
            \item \textbf{Missing}, e.g.\ filenames like ``IMG\_123.jpg'' with no descriptive text for the image in its alt text.
            \item \textbf{Generic}, e.g.\ ``beautiful view'' that offers little semantic grounding.
            \item \textbf{Off-topic}, e.g.\ boilerplate such as ``click here to buy''.
        \end{itemize}
        When such noisy associations dominate, models risk learning shortcuts (e.g.\ linking logos directly to brand names) instead of genuine visual grounding. A second challenge is an \textbf{objective gap}: alt-text resembles retrieval labels more than natural captions or question-answer pairs. Training only with discriminative alignment (as in CLIP) yields strong retrieval but poor generation; training only with captions produces fluent language but weak grounding.
        
        \paragraph{BLIP’s Two-Part Strategy}
        The authors observe that these problems reinforce each other: noisy supervision destabilizes multi-task learning, and narrow objectives fail to transfer broadly. BLIP addresses both with a simple recipe: \emph{first curate the data, then train a unified model that can align, ground, and generate}.
        \begin{itemize}
            \item \textbf{Step 1 — Bootstrapping with CapFilt.}  
            Instead of trusting raw alt-text, BLIP trains its own \emph{Captioner} and \emph{Filter} on a small, clean human-annotated dataset. The Captioner (a generative decoder) produces synthetic captions grounded in visual content, while the Filter (a discriminative encoder) discards both weak alt-text and low-quality synthetic captions. This process rebuilds the large pretraining corpus “from within”, producing cleaner, semantically faithful supervision.
            \item \textbf{Step 2 — Unified encoder--decoder.}  
            BLIP introduces a \emph{Multimodal Encoder--Decoder (MED)} that supports three complementary modes with largely shared parameters:
            \begin{itemize}
                \item \textbf{Image--Text Contrastive (ITC).} Aligns unimodal encoders for fast retrieval.
                \item \textbf{Image--Text Matching (ITM).} Uses cross-attention to check whether a caption truly matches an image.
                \item \textbf{Language Modeling (LM).} Uses a causal decoder to generate captions or answers, reusing the same cross-attention for stable fusion.
            \end{itemize}
            By combining these modes, BLIP avoids the trade-off between retrieval strength and generative ability, yielding a single checkpoint that can both discriminate and generate.
        \end{itemize}
        
        \medskip
        
        \emph{Intuition.} By first cleaning the data, BLIP removes much of the noise that would otherwise destabilize multi-task optimization. This makes it feasible to train a single model on diverse objectives without one collapsing the others. At the same time, the unified architecture avoids the brittleness of task-specific designs: contrastive alignment alone cannot generate, and pure generation often ignores fine-grained grounding. Combining the two under one framework allows the model to tackle multiple problems at once—retrieval, discrimination, and generation—so that improvements in one skill reinforce the others, producing a more balanced and versatile vision--language learner.
        
        \newpage
        
        \subsubsection{Method}
        
        \paragraph{Unified Architecture with Three Functional Modes}  
        Rather than building separate networks for retrieval, grounding, and captioning, BLIP uses a \textbf{single multimodal encoder--decoder backbone} that can be run in three different configurations. Most of the heavy components---the vision encoder, cross-attention layers, and feed-forward blocks---are \emph{shared across all modes}. What changes between them is \emph{how attention is applied and which inputs are activated}:  
        \begin{itemize}
            \item In contrastive alignment, image and text streams run separately without cross-attention.  
            \item In matching, the text stream is augmented with cross-attention over image tokens.  
            \item In generation, the decoder uses causal (masked) self-attention but reuses the same cross-attention and feed-forward layers as the encoder.  
        \end{itemize}
        
        \begin{figure}[H]
            \centering
            \includegraphics[width=0.85\textwidth]{Figures/Chapter_24/BLIP_overview.jpg}
            \caption{
                \textbf{BLIP's unified MED architecture and objectives.} 
                The Vision Transformer image encoder is initialized from a pre-trained ViT (e.g., ImageNet) but remains \emph{trainable} during pre-training, alongside the text transformer blocks. 
                All components are optimized end-to-end under three objectives, reusing the same backbone with minimal changes: 
                (i) \emph{ITC} runs the image and text encoders unimodally (no cross-attention) to produce global embeddings for contrastive retrieval; 
                (ii) \emph{ITM} augments the text encoder with cross-attention to image tokens, using bidirectional self-attention for fine-grained matching; 
                (iii) \emph{LM} reuses the same cross-attention and feed-forward blocks but applies a causal self-attention mask to decode text autoregressively. 
                Most parameters (vision encoder, cross-attention, FFN) are shared; the functional differences stem only from routing (cross-attention on/off) and attention masking (bidirectional vs.\ causal), not from freezing or separating modules. 
                \emph{Source:} \cite{li2022_blip}.
            }
            \label{fig:chpapter24_blip_overview}
        \end{figure}
        
        This parameter sharing means that improvements in one objective (e.g., better grounding in ITM) flow into the others, stabilizing training and avoiding the need to maintain multiple specialized checkpoints.  
        
        \begin{enumerate}
            \item \textbf{Image--Text Contrastive (ITC).} The unimodal image encoder and text encoder produce global embeddings. A contrastive loss aligns paired embeddings while pushing apart mismatched ones, giving BLIP strong retrieval and zero-shot transfer.  
            \item \textbf{Image--Text Matching (ITM).} The text encoder is extended with cross-attention layers that attend to image features. The model then predicts whether a caption truly matches its paired image. Hard negatives are sampled from ITC to make the discrimination sharper.  
            
            \newpage
            
            \item \textbf{Language Modeling (LM).} The decoder reuses the same cross-attention and feed-forward blocks as the encoder, but changes the style of self-attention. In the encoder, self-attention is \emph{bidirectional}: each token can attend to all others, both before and after it, which is ideal for understanding a complete sentence. In contrast, the LM decoder uses \emph{causal masking}: each token can only attend to those that came earlier in the sequence, never to future tokens. This forces the model to generate text one word at a time, predicting the next token given the history. By combining causal self-attention with cross-attention to the image features, BLIP can produce grounded captions and answers in an autoregressive way, rather than simply classifying pairs.
        \end{enumerate}
        
        \paragraph{Why Causal vs.\ Bidirectional Attention?}
        \begin{itemize}
            \item \textbf{Bidirectional self-attention (ITC, ITM).} For \emph{understanding} tasks, the text stream should read a sentence holistically: each token attends to all others (past and future) to form a context-rich representation. This is ideal for global alignment (ITC) and fine-grained verification (ITM), where the model must judge a \emph{complete} image–text pair.
            \item \textbf{Causal (masked) self-attention (LM).} For \emph{generation}, the decoder must predict the next token given only the prefix; allowing access to future tokens would let it “peek” and trivially copy the target. Causal masking enforces autoregressive decoding and yields fluent, grammatical captions that remain conditioned on the image via cross-attention.
        \end{itemize}
        \emph{Example.} In retrieval or matching, the phrase “a dog on the grass” is compared to an image \emph{as a whole}—bidirectional attention fits. In captioning, the model writes “A dog is running \dots” \emph{one token at a time}—causal masking prevents cheating and maintains coherence.
        
        \paragraph{Objectives in Mathematical Form.}
        BLIP optimizes three complementary losses within the shared backbone:
        \begin{itemize}
            \item \textbf{Image--Text Contrastive (ITC).}  
            For paired embeddings $(v_i, t_i)$ and negatives $(v_i, t_j)$, BLIP applies a symmetric InfoNCE loss:  
            \[
            \mathcal{L}_{\mathrm{ITC}} = -\frac{1}{N} \sum_{i=1}^N 
            \Big[ \log \frac{\exp(\mathrm{sim}(v_i,t_i)/\tau)}{\sum_{j=1}^N \exp(\mathrm{sim}(v_i,t_j)/\tau)}
            + \log \frac{\exp(\mathrm{sim}(t_i,v_i)/\tau)}{\sum_{j=1}^N \exp(\mathrm{sim}(t_i,v_j)/\tau)} \Big],
            \]
            where $\mathrm{sim}$ is cosine similarity and $\tau$ a temperature.  
            \emph{Intuition:} Encourages globally aligned representations so retrieval works out of the box.
            
            \item \textbf{Image--Text Matching (ITM).}  
            With image tokens $v$ and text tokens $t$, the cross-attentive encoder predicts a binary label $y \in \{0,1\}$:
            \[
            \mathcal{L}_{\mathrm{ITM}} = - \big[ y \log p(y{=}1|v,t) + (1-y)\log p(y{=}0|v,t) \big].
            \]
            \emph{Intuition:} Forces the model to judge whether an entire caption matches an image, sharpening grounding beyond coarse similarity.
            
            \item \textbf{Language Modeling (LM).}  
            For a target sequence $t = (t_1,\ldots,t_L)$ and image $v$, the decoder with causal masking maximizes
            \[
            \mathcal{L}_{\mathrm{LM}} = - \sum_{k=1}^L \log p(t_k \mid t_{<k}, v).
            \]
            \emph{Intuition:} Enforces left-to-right text generation conditioned on image features, producing fluent grounded captions.
        \end{itemize}
        
        Together, these objectives form a joint training signal: ITC aligns global spaces, ITM enforces pairwise discrimination, and LM teaches autoregressive generation. Their complementarity stabilizes multi-task learning within one backbone.
        
        \paragraph{Training Framework: End-to-End Chronology (CapFilt $\rightarrow$ Final BLIP)}
        Web alt-text is often underspecified or off-topic, which destabilizes pretraining. BLIP therefore \emph{separates data construction from final model training} in a chronological, three-phase pipeline: (1) train specialized tools, (2) rebuild the dataset, (3) train the final unified model.
        
        \begin{enumerate}
            \item \textbf{Phase~1: Forge the tools on clean data.}
            \begin{itemize}
                \item \textbf{Captioner (generative proposal).} Start from a BLIP initialization and fine-tune the \emph{image-grounded decoder} in \textbf{LM} mode on a small human-annotated set (e.g., COCO). This produces a model that can generate descriptive, image-relevant \emph{synthetic captions} for web images. Stochastic decoding (e.g., nucleus sampling) increases diversity and coverage.
                \item \textbf{Filter (discriminative selection).} Independently fine-tune another BLIP initialization in \textbf{ITM} (binary match) with \textbf{ITC}-guided hard negatives on the same clean set. This yields an image–text \emph{verifier} that can score the semantic fidelity of any pair. Decoupling Captioner and Filter avoids confirmation bias (a generator endorsing its own outputs).
            \end{itemize}
            
            \item \textbf{Phase~2: Rebuild the large-scale corpus (CapFilt).}
            \begin{itemize}
                \item \textbf{Generate.} Run the \emph{Captioner} over the web image pool $\{I_w\}$ to produce synthetic texts $\{T_s\}$.
                \item \textbf{Select.} Run the \emph{Filter} on both sources—the original web texts $\{T_w\}$ and synthetic texts $\{T_s\}$—to keep only high-quality pairs:
                \[
                \{(I_w, T'_w)\} = \mathrm{Filter}\big(\{(I_w, T_w)\}\big), \qquad
                \{(I_w, T'_s)\} = \mathrm{Filter}\big(\{(I_w, T_s)\}\big).
                \]
                \item \textbf{Assemble.} Form the bootstrapped pretraining set
                \[
                \mathcal{D}_{\mathrm{boot}} \;=\; \underbrace{\{(I_h, T_h)\}}_{\text{human-annotated}}
                \;\cup\; \underbrace{\{(I_w, T'_w)\}}_{\text{filtered web}}
                \;\cup\; \underbrace{\{(I_w, T'_s)\}}_{\text{filtered synthetic}}.
                \]
                Here $T'_w$ and $T'_s$ denote pairs the Filter judged as matched; images with no good text are dropped.
            \end{itemize}
            
            \item \textbf{Phase~3: Train the \emph{final} unified BLIP on $\mathcal{D}_{\mathrm{boot}}$.}
            \begin{itemize}
                \item A new BLIP model is initialized and optimized on \emph{all three objectives concurrently}. In practice, each minibatch is sampled from the same purified dataset $\mathcal{D}_{\mathrm{boot}}$, and the model routes the inputs through different attention masks and heads depending on the objective:
                \begin{itemize}
                    \item \textbf{ITC} (unimodal encoders; no cross-attention) — learns global alignment by comparing embeddings of paired vs.\ unpaired samples.
                    \item \textbf{ITM} (text encoder with image cross-attention; bidirectional SA) — judges whether a caption matches an image, with hard negatives drawn using ITC similarities.
                    \item \textbf{LM} (decoder with shared cross-attention; \emph{causal} SA) — generates captions token by token, conditioned on image features.
                \end{itemize}
                The total loss is a weighted sum,
                \[
                \mathcal{L} = \lambda_{\mathrm{ITC}} \mathcal{L}_{\mathrm{ITC}} 
                + \lambda_{\mathrm{ITM}} \mathcal{L}_{\mathrm{ITM}}
                + \lambda_{\mathrm{LM}} \mathcal{L}_{\mathrm{LM}},
                \]
                with all parameters updated jointly.
                \item \textbf{Why concurrency matters.} Training the three tasks together stabilizes optimization: ITC provides a consistent alignment scaffold, ITM sharpens discrimination using those aligned features, and LM leverages the same cross-attended representations for grounded generation. Running them in parallel avoids forgetting and ensures improvements in one pathway benefit the others.
            \end{itemize}
        \end{enumerate}
        
        \emph{Summary.} CapFilt first \textbf{proposes} better text (Captioner) and then \textbf{selects} reliable pairs (Filter). The resulting $\mathcal{D}_{\mathrm{boot}}$ lets the final BLIP checkpoint learn \emph{alignment} (ITC), \emph{grounding} (ITM), and \emph{generation} (LM) in one backbone—with cross-attention toggled on/off and self-attention switched between \emph{bidirectional} (understanding) and \emph{causal} (generation) purely via masks.
        
        \begin{figure}[H]
            \centering
            \includegraphics[width=0.7\textwidth]{Figures/Chapter_24/BLIP_training_framework.jpg}
            \caption{\textbf{Learning framework.} Captioner and filter, both BLIP-initialized, bootstrap a cleaner dataset from noisy web supervision. \emph{Source:} \cite{li2022_blip}.}
            \label{fig:chpapter24_blip_training_framework}
        \end{figure}
        
        \paragraph{Downstream Usage}
        After pretraining, the same backbone adapts flexibly:
        \begin{itemize}
            \item Retrieval (via ITC and ITM).
            \item Captioning (via LM).
            \item VQA (encode question + image, decode answer).
        \end{itemize}
        
        \begin{figure}[H]
            \centering
            \includegraphics[width=0.4\textwidth]{Figures/Chapter_24/blip_downstream.jpg}
            \caption{\textbf{Downstream heads.} BLIP routes through ITC, ITM, or LM heads depending on the task. \emph{Source:} \cite{li2022_blip}.}
            \label{fig:chpapter24_blip_downstream}
        \end{figure}
        
        \subsubsection{Experiments and Ablations}
        
        \paragraph{CapFilt Effectiveness}  
        Empirical studies confirm that the CapFilt pipeline provides consistent gains:
        \begin{itemize}
            \item \textbf{Retrieval and Captioning.} Models trained on the cleaned corpus outperform those trained on raw web text, even when both use the same number of image–text pairs.
            \item \textbf{Quality vs.\ Quantity.} Adding more noisy pairs does not close the gap; filtering clearly outperforms brute-force scaling, showing that \emph{data quality dominates raw scale}.
            \item \textbf{Retraining vs.\ Continuing.} Restarting training from scratch on the purified set matches or exceeds continuing training on the noisy one, indicating that the benefit comes from improved supervision rather than extra steps.
        \end{itemize}
        
        \paragraph{Ablations}  
        Key ablation experiments highlight the necessity of both stages:
        \begin{itemize}
            \item \textbf{Without Captioner.} Relying only on web alt-text leaves a large fraction of pairs irrelevant or underspecified, hurting downstream generation.
            \item \textbf{Without Filter.} Using synthetic captions without selection reintroduces noise; performance falls sharply, showing that caption generation alone is insufficient.
            \item \textbf{Joint vs.\ Decoupled.} Sharing parameters between Captioner and Filter causes confirmation bias and weaker filtering; the decoupled design is essential.
        \end{itemize}
        
        \subsubsection{Limitations and Future Work}
        
        \paragraph{Observed Constraints}
        \begin{itemize}
            \item \textbf{Scaling challenges.} As models grow, balancing multiple objectives becomes harder; discriminative and generative losses can interfere without careful tuning.
            \item \textbf{Dependence on bootstrapping.} The final model’s quality is bounded by the effectiveness of the Captioner and Filter; errors in early stages propagate forward.
            \item \textbf{Task balance.} Equal treatment of ITC, ITM, and LM may not be optimal across domains; different applications may require task-specific weighting.
        \end{itemize}
        
        \paragraph{Toward BLIP-2}  
        BLIP demonstrates that unified multi-task learning is feasible, but scaling to very large LLMs risks overwhelming multimodal fusion. \textbf{BLIP-2} addresses this by freezing strong pretrained components (a vision encoder and an LLM) and inserting a lightweight connector (the Q-Former) to bridge them, retaining visual grounding while leveraging large-scale language priors.
        
    \end{enrichment}
    
    \newpage
    
    \begin{enrichment}[BLIP-2: Bridging Vision Encoders and LLMs via Q-Former][subsection]
        \label{enr:subsec_chapter24_blip2}
        
        \paragraph{High-Level Idea} 
        \emph{BLIP-2}~\cite{li2023_blip2} moves away from BLIP’s heavy \emph{end-to-end training} of both vision and text modules. In BLIP, the ViT image encoder and text transformer were \emph{jointly optimized} with ITC, ITM, and LM losses. This achieved strong multimodal fusion, but came at huge computational cost: every improvement to the vision or text backbone required retraining the entire model, and the text side remained limited compared to emerging billion-parameter LLMs.
        
        \noindent\textbf{The BLIP-2 shift.} Instead of training everything together, BLIP-2 leverages two \emph{frozen experts}: a large pre-trained ViT (e.g., CLIP ViT-g or EVA-CLIP) and a large pre-trained LLM (e.g., OPT, FlanT5). Both remain untouched, preserving their strong unimodal priors. The only trainable component is a small \textbf{Querying Transformer (Q-Former)}, equipped with a fixed set of learnable \emph{query tokens}. These queries attend to frozen vision features, distill them into a compact representation, and pass them—via a thin projection—as soft prompts into the frozen LLM.
        
        \noindent\textbf{Why a two-stage curriculum?} Training the Q-Former to talk to both the vision encoder and the LLM \emph{at once} is unstable: the LLM has never seen visual tokens and cannot guide the alignment, while the ViT features are too high-dimensional and unstructured for direct prompting. Splitting training stabilizes learning and enforces a clear division of labor: first teach the Q-Former to \emph{see} with the ViT alone, then teach it to \emph{communicate} with the frozen LLM.
        
        \noindent\textbf{Two-stage curriculum.} 
        \begin{itemize}
            \item \emph{Stage 1 (Vision--Language Representation Learning):} The Q-Former is trained with a frozen ViT, using BLIP-style objectives (contrastive, matching, generation) to ensure its query tokens capture \emph{text-relevant} visual features. The LLM is not involved.
            \item \emph{Stage 2 (Vision-to-Language Generation):} The Q-Former outputs are linearly projected and fed into the frozen LLM. Only the Q-Former is updated, so it learns to “speak the LLM’s language,” turning visual summaries into effective soft prompts for text generation.
        \end{itemize}
        
        \noindent In short, BLIP-2 improves over BLIP by freezing powerful unimodal backbones, introducing a small trainable bridge (the Q-Former), and adopting a staged curriculum that first teaches the bridge to \emph{see}, then teaches it to \emph{talk}.
        
        \begin{figure}[H]
            \centering
            \includegraphics[width=0.65\textwidth]{Figures/Chapter_24/BLIP2_overview.jpg}
            \caption{
                \textbf{BLIP-2 framework (frozen experts + lightweight bridge).}
                A \emph{frozen} image encoder outputs dense visual tokens. A \textbf{Q-Former} (trainable) with $K$ learnable query tokens attends to these tokens and produces $K$ query features. A linear adapter maps them to the LLM’s embedding space and feeds a \emph{frozen} LLM for image-grounded generation. Training proceeds in two stages: (1) representation learning with a frozen vision encoder; (2) vision-to-language generation with a frozen LLM. Source: ~\cite{li2023_blip2}.
            }
            \label{fig:chpapter24_blip2_overview}
        \end{figure}
        
        \newpage
        
        \subsubsection{Method: A Small Q-Former Bridging Two Frozen Experts}
        
        \paragraph{Stage~1: Vision--Language representation with a frozen image encoder}
        Freeze the image encoder (e.g., CLIP/EVA ViT). Train only the \textbf{Q-Former} (and small heads) to extract \emph{text-relevant} visual summaries. The Q-Former contains $K$ learnable \emph{queries} that \emph{self-attend} and \emph{cross-attend} to frozen visual tokens. Optimize three objectives (like in ~\ref{enr:subsec_chapter24_siglip}):
        \begin{itemize}
            \item \textbf{ITC (Image--Text Contrastive).} Learn independent visual/text embeddings for retrieval; align matched pairs and repel mismatches.
            \item \textbf{ITM (Image--Text Matching).} Enable fine-grained discrimination under \emph{bidirectional} Q--Text interaction; predict match vs.\ non-match.
            \item \textbf{Image-grounded LM pretraining (masked).} Allow text to attend to queries while keeping text \emph{causal}, preparing queries for generation.
        \end{itemize}
        \emph{Intuition.} ITC yields globally aligned spaces; ITM injects pair-level grounding; masked conditioning prepares Q to act as a compact visual prompt.
        
        \paragraph{Stage~2: Vision-to-language generation with a frozen LLM}
        Keep the LLM \emph{frozen}. Insert a \textbf{linear projection} from Q-Former outputs to the LLM token space and train (Q-Former + projection) with next-token prediction on caption/instruction data. The LLM consumes the $K$ projected query tokens prepended to the textual prompt, enabling zero-shot, instruction-following \emph{image-to-text} generation without tuning the LLM.
        
        \begin{figure}[H]
            \centering
            \includegraphics[width=0.85\textwidth]{Figures/Chapter_24/BLIP2_qformer_architecture.jpg}
            \caption{
                \textbf{Q-Former and Stage~1 objectives.}
                A small Transformer holds $K$ \emph{learnable} queries (Q) which \emph{self-attend} and \emph{cross-attend} to frozen image features. Joint optimization:
                (i) \textbf{ITC} for global alignment (comparable Q/text embeddings),
                (ii) \textbf{ITM} for pair-level grounding (match vs.\ non-match),
                (iii) \textbf{Image-grounded LM pretraining} to condition text on Q under causal constraints.
                These losses teach Q to extract visual information most relevant to the text. Source:\cite{li2023_blip2}.
            }
            \label{fig:chpapter24_blip2_qformer}
        \end{figure}
        
        \begin{figure}[H]
            \centering
            \includegraphics[width=0.7\textwidth]{Figures/Chapter_24/BLIP2_self_attention_strategy.jpg}
            \caption{
                \textbf{How attention masks steer Q--Text interaction in BLIP-2’s Q-Former (Stage~1).}
                A fixed set of learnable \emph{query} tokens (Q) reads frozen ViT features and interacts with \emph{text} tokens (T) under three masks:
                (i) \emph{Uni-modal (ITC):} Q attends only to Q and T only to T, producing independent visual/text embeddings for contrastive alignment.
                (ii) \emph{Bi-directional (ITM):} Q and T fully attend to each other to form a fused representation for fine-grained match classification.
                (iii) \emph{Multimodal causal (image-grounded generation):} T attends to all Q and only past T (causal), while Q remains fully visible to itself, forcing the visual evidence to pass through the Q bottleneck and enabling autoregressive text generation.  Source:\cite{li2023_blip2}.}
            \label{fig:chpapter24_blip2_masks}
        \end{figure}
        
        \paragraph{Two-Stage Curriculum: What Trains When and Why}
        \textbf{Stage~1 (learn to \emph{see}):} Freeze the image encoder; train only the Q-Former on paired image--text data. The three masks in Fig.~\ref{fig:chpapter24_blip2_masks} are used \emph{jointly} so the queries learn to (a) summarize visual content independently of text (ITC), (b) fuse with text for pair verification (ITM), and (c) carry all image information needed to \emph{describe} the text under causal decoding (image-grounded generation). \emph{Intuition:} Before “talking” to a frozen LLM, Q must first become a compact, language-relevant summary of the image; otherwise the modality gap is too wide and training is brittle.
        
        \noindent\textbf{Stage~2 (learn to \emph{talk}):} Keep the image encoder and the LLM frozen. Feed the trained queries through a small projection into the LLM’s embedding space and optimize a language-modeling loss while updating \emph{only} the Q-Former (and the projection). \emph{Intuition:} With Stage~1, Q already encodes text-relevant visual evidence; Stage~2 teaches Q to present that evidence as a short “soft prompt” the LLM can use without disrupting its linguistic knowledge.
        
        \paragraph{Objectives (concise math + intuition)}
        Let $v$ denote the Q-aggregated visual embedding and $t$ a text embedding from the Q-Former stack (mask depends on the objective).
        
        \noindent\textbf{ITC (contrastive, uni-modal mask).}
        \[
        \mathcal{L}_{\text{ITC}}
        = -\frac{1}{N}\sum_{i=1}^{N}
        \Big[
        \log\frac{\exp(\langle v_i,t_i\rangle/\tau)}{\sum_{j}\exp(\langle v_i,t_j\rangle/\tau)}
        +
        \log\frac{\exp(\langle t_i,v_i\rangle/\tau)}{\sum_{j}\exp(\langle t_i,v_j\rangle/\tau)}
        \Big].
        \]
        \emph{Why:} Learn a shared space where matched pairs are close and mismatches are far, enabling retrieval.
        
        \newpage
        
        \noindent\textbf{ITM (matching, bi-directional mask).}
        \[
        \mathcal{L}_{\text{ITM}}
        = -\frac{1}{N}\sum_{i=1}^{N}\big[y_i\log p_i+(1-y_i)\log(1-p_i)\big],\quad
        p_i=\sigma\!\big(W^\top f_{\text{fused}}(Q,T)\big).
        \]
        \emph{Why:} Enforce fine-grained grounding by classifying pair match vs.\ non-match on fused Q--T features (often with hard negatives).
        
        \noindent\textbf{Image-grounded generation (multimodal causal mask).}
        \[
        \mathcal{L}_{\text{IG}}
        = -\sum_{m=1}^{M}\log p\big(y_m \mid y_{<m},\, Q\big).
        \]
        \emph{Why:} Force queries to carry all image evidence needed for text under causal decoding, making Q a faithful visual prompt.
        
        \paragraph{How the pieces fit during training}
        \begin{itemize}
            \item \textbf{Stage~1 (Q-Former only):} Optimize $\mathcal{L}_{\text{ITC}}+\mathcal{L}_{\text{ITM}}+\mathcal{L}_{\text{IG}}$ with the image encoder frozen and \emph{no LLM} in the loop. This shapes Q into a compact, language-relevant visual interface.
            \item \textbf{Stage~2 (Q-Former + frozen LLM):} Project $Q$ to the LLM’s token space and optimize a standard LM loss $\mathcal{L}_{\text{LM}}=-\sum_m\log p_{\text{LLM}}(y_m\mid y_{<m},\,\mathrm{Proj}(Q))$, updating only the Q-Former and projection. This teaches Q to “speak” to the LLM without altering the LLM itself.
        \end{itemize}
        
        \begin{figure}[H]
            \centering
            \includegraphics[width=0.85\textwidth]{Figures/Chapter_24/BLIP2_2nd_stage.jpg}
            \caption{
                \textbf{Stage~2: vision-to-language bootstrapping with frozen LLMs.}
                Top: decoder-only LLM (e.g., OPT). Bottom: encoder--decoder LLM (e.g., FlanT5). A linear adapter maps Q-Former outputs to the LLM’s embedding space. Only the \emph{Q-Former and the adapter} are trained; both the vision encoder and LLM remain \emph{frozen}. Source:\cite{li2023_blip2}.
            }
            \label{fig:chpapter24_blip2_stage2}
        \end{figure}
        
        \begin{figure}[H]
            \centering
            \includegraphics[width=0.85\textwidth]{Figures/Chapter_24/BLIP2_results_example.jpg}
            \caption{
                \textbf{Zero-shot instructed image-to-text.}
                With a frozen LLM and a trained Q-Former bridge, BLIP-2 exhibits visual dialogue, knowledge/commonsense grounded by images, storytelling, and personalization without full LLM fine-tuning. Source:\cite{li2023_blip2}.
            }
            \label{fig:chpapter24_blip2_examples}
        \end{figure}
        
        % ========================
        % Table 1 — Zero-shot overview (matches BLIP-2 Table 1 layout)
        % ========================
        \begin{table}[H]
            \centering
            \caption{Overview of BLIP-2 results on various zero-shot vision--language tasks, compared with prior SOTA. Higher is better. Source:\cite{li2023_blip2}.}
            \label{tab:blip2_overview}
            \footnotesize
            \resizebox{\linewidth}{!}{%
                \begin{tabular}{lcccccccc}
                    \toprule
                    \textbf{Models} & \textbf{\#Trainable Params} & \textbf{Open-sourced?} &
                    \multicolumn{2}{c}{\textbf{Visual QA (VQAv2 test-dev)}} &
                    \multicolumn{2}{c}{\textbf{Image Captioning (NoCaps val)}} &
                    \multicolumn{2}{c}{\textbf{Image--Text Retrieval (Flickr test)}} \\
                    \cmidrule(lr){4-5}\cmidrule(lr){6-7}\cmidrule(lr){8-9}
                    &  &  & \textbf{VQA acc.} &  & \textbf{CIDEr} & \textbf{SPICE} & \textbf{TR@1} & \textbf{IR@1} \\
                    \midrule
                    BLIP~\cite{li2022_blip}        & 583M  & $\checkmark$ & --   &  & 113.2 & 14.8 & 96.7 & 86.7 \\
                    SimVLM~\cite{wang2021b_simvlm} & 1.4B  & $\times$     & --   &  & 112.2 & --   & --   & --   \\
                    BEIT-3~\cite{wang2022_beit3}   & 1.9B  & $\times$     & --   &  & --    & --   & 94.9 & 81.5 \\
                    Flamingo~\cite{flamingo2022_fewshot} & 10.2B & $\times$ & 56.3 &  & --    & --   & --   & --   \\
                    \midrule
                    \textbf{BLIP-2}                & \textbf{188M} & $\checkmark$ & \textbf{65.0} & & \textbf{121.6} & \textbf{15.8} & \textbf{97.6} & \textbf{89.7} \\
                    \bottomrule
                \end{tabular}%
            }
        \end{table}
        
        
        % ========================
        % Table 5 — Retrieval (Flickr30K zero-shot; COCO finetuned), no \multirow
        % ========================
        \begin{table}[H]
            \centering
            \caption{Comparison with SOTA image--text retrieval methods. Left: Flickr30K zero-shot (1K test). Right: COCO finetuned (5K test). R@K reported (\%). Source:\cite{li2023_blip2}.}
            \label{tab:blip2_retrieval}
            \footnotesize
            \setlength{\tabcolsep}{4pt}
            % 14 columns total: l + 13 c
            \begin{tabular}{lccccccccccccc}
                \toprule
                \textbf{Model} & \textbf{\#Trainable} &
                \multicolumn{6}{c}{\textbf{Flickr30K Zero-shot (1K)}} &
                \multicolumn{6}{c}{\textbf{COCO Finetuned (5K)}} \\
                \cmidrule(lr){3-8}\cmidrule(lr){9-14}
                &  &
                \multicolumn{3}{c}{\textbf{Image$\rightarrow$Text}} &
                \multicolumn{3}{c}{\textbf{Text$\rightarrow$Image}} &
                \multicolumn{3}{c}{\textbf{Image$\rightarrow$Text}} &
                \multicolumn{3}{c}{\textbf{Text$\rightarrow$Image}} \\
                \cmidrule(lr){3-5}\cmidrule(lr){6-8}\cmidrule(lr){9-11}\cmidrule(lr){12-14}
                &  & \textbf{R@1} & \textbf{R@5} & \textbf{R@10} & \textbf{R@1} & \textbf{R@5} & \textbf{R@10} & \textbf{R@1} & \textbf{R@5} & \textbf{R@10} & \textbf{R@1} & \textbf{R@5} & \textbf{R@10} \\
                \midrule
                \multicolumn{14}{l}{\emph{Dual-encoder models}} \\
                CLIP~\cite{radford2021_clip}      & 428M  & 88.0 & 98.7 & 99.4 & 68.7 & 90.6 & 95.2 & --   & --   & --   & --   & --   & --   \\
                ALIGN~\cite{jia2021_align}        & 820M  & 88.6 & 98.7 & 99.7 & 75.7 & 93.8 & 96.8 & 77.0 & 93.5 & 96.9 & 59.9 & 83.3 & 89.8 \\
                FILIP~\cite{yao2022_filip}        & 417M  & 89.8 & 99.2 & 99.8 & 75.0 & 93.4 & 96.3 & 78.9 & 94.4 & 97.4 & 61.2 & 84.3 & 90.6 \\
                Florence~\cite{yuan2021_florence} & 893M  & 90.9 & 99.1 & --   & 76.7 & 93.6 & --   & 81.8 & 95.2 & --   & 63.2 & 85.7 & --   \\
                BEIT-3~\cite{wang2022_beit3}      & 1.9B  & 94.9 & 99.9 & \textbf{100.0} & 81.5 & 95.6 & 97.8 & 84.8 & 96.5 & 98.3 & 67.2 & \textbf{87.7} & \textbf{92.8} \\
                \midrule
                \multicolumn{14}{l}{\emph{Fusion-encoder models}} \\
                UNITER~\cite{chen2020_uniter}     & 303M  & 83.6 & 95.7 & 97.7 & 68.7 & 89.2 & 93.9 & 65.7 & 88.6 & 93.8 & 52.9 & 79.9 & 88.0 \\
                OSCAR~\cite{li2020_oscar}         & 345M  & --   & --   & --   & --   & --   & --   & 70.0 & 91.1 & 95.5 & 54.0 & 80.8 & 88.5 \\
                VinVL~\cite{zhang2021_vinvl}      & 345M  & --   & --   & --   & --   & --   & --   & 75.4 & 92.9 & 96.2 & 58.8 & 83.5 & 90.3 \\
                \midrule
                \multicolumn{14}{l}{\emph{Dual encoder + Fusion encoder re-ranking}} \\
                ALBEF~\cite{li2021_albef}         & 233M  & 94.1 & 99.5 & 99.7 & 82.8 & 96.3 & 98.1 & 77.6 & 94.3 & 97.2 & 60.7 & 84.3 & 90.5 \\
                BLIP~\cite{li2022_blip}           & 446M  & 96.7 & 100.0 & 100.0 & 86.7 & 97.3 & 98.7 & 82.4 & 95.4 & 97.9 & 65.1 & 86.3 & 91.8 \\
                \textbf{BLIP-2 ViT-L}             & \textbf{474M}  & {96.9} & \textbf{100.0} & \textbf{100.0} & {88.6} & {97.6} & \textbf{98.9} & {83.5} & {96.0} & {98.0} & {66.3} & {86.5} & {91.8} \\
                \textbf{BLIP-2 ViT-g}             & \textbf{1.2B}  & \textbf{97.6} & \textbf{100.0} & \textbf{100.0} & \textbf{89.7} & \textbf{98.1} & \textbf{98.9} & \textbf{85.4} & \textbf{97.0} & \textbf{98.5} & \textbf{68.3} & \textbf{87.7} & {92.6} \\
                \bottomrule
            \end{tabular}
        \end{table}
        
        \subsubsection{Experiments \& Ablations (Concise)}
        \begin{itemize}
            \item \textbf{Frozen experts preserve priors.} Keeping the vision encoder and LLM frozen avoids catastrophic forgetting while enabling strong zero-shot transfer; most gains come from learning the \emph{interface} (Q-Former + adapter).
            \item \textbf{Masking matters.} Ablating the \emph{uni-modal} mask (ITC) degrades retrieval; ablating \emph{bidirectional} (ITM) weakens grounding; removing \emph{causal} conditioning harms generation quality---confirming each mask’s role.
            \item \textbf{Number of queries ($K$).} Too few queries underfit fine details; too many inflate compute with diminishing returns. Moderate $K$ balances fidelity and LLM cost.
            \item \textbf{Adapter simplicity.} A single linear projection to the LLM embedding space is sufficient; heavier adapters show minor gains at higher cost.
            \item \textbf{Curriculum order.} Training Stage~1 (alignment/grounding) before Stage~2 (generation) stabilizes instruction-following performance; skipping Stage~1 reduces zero-shot quality.
        \end{itemize}
        
        \newpage
        
        \subsubsection{Limitations \& Future Work}
        \textbf{Limitations.}
        \begin{itemize}
            \item \textbf{Bottleneck tightness.} A fixed small $K$ can miss region-level or fine-grained details without auxiliary heads/adapters.
            \item \textbf{Static queries.} Global queries lack explicit spatial/temporal structure; dense grounding or long video reasoning may require hierarchical or region/time-aware queries.
            \item \textbf{Frozen LLM.} Great for stability, but limits specialization under large domain shifts; PEFT helps but may be insufficient in niche domains.
        \end{itemize}
        \textbf{Future Work.}
        \begin{itemize}
            \item \textbf{Hierarchical querying.} Multi-scale or region/time-conditioned queries for dense tasks and long-horizon video.
            \item \textbf{Adaptive $K$.} Dynamic selection based on content difficulty and prompt type to trade off detail vs.\ cost.
            \item \textbf{Richer adapters/PEFT.} Structured adapters (e.g., LoRA + gating) for selective LLM specialization while preserving generality.
            \item \textbf{Unified multimodality.} Extending the Q-Former interface to audio/motion and 3D inputs for broader perception--language reasoning.
        \end{itemize}
        
    \end{enrichment}
    
    \newpage
    
    \begin{enrichment}[SigLIP~2: Multilingual \& Dense Vision--Language Encoding][subsection]
        \label{enr:subsec_chapter24_siglip2}
        
        \paragraph{High-Level Overview}
        \emph{SigLIP~2} keeps SigLIP’s efficient \textbf{dual-encoder} and \textbf{pairwise sigmoid loss}~\cite{zhai2023_siglip} (no cross-modal attention at test time), and adds \emph{training-only} signals that teach the vision encoder three missing skills—\emph{where} evidence is (localization), \emph{how} patches relate (dense semantics), and \emph{how} to cope with non-square layouts and non-English text (robustness). Concretely, we add:
        
        \begin{itemize}
            \item \textbf{Localization (``where'').} A lightweight \emph{decoder} cross-attends to \emph{unpooled} patch tokens and is trained for captioning, grounded captioning, and referring expressions~\cite{wan2024_locca}. This shapes patch-level spatial semantics but is \emph{discarded at test time}.
            \item \textbf{Dense semantics (``how patches relate'').} A late \emph{consistency \& masking} tail (SILC/TIPS) aligns student crops to a full-image EMA teacher and predicts teacher features at masked patches~\cite{naeem2023_silc,maninis2025_tips}, yielding context-aware, part--whole coherent tokens.
            \item \textbf{Input robustness (shapes \& languages).} A brief \emph{shape-aware} tail either (i) releases size-specific specialists or (ii) trains a single \emph{NaFlex} generalist that preserves native aspect ratios and supports multiple sequence lengths~\cite{dehghani2023_navit,beyer2023_flexivit}. Optional \emph{active curation} improves small models by selecting informative pairs, and a \emph{multilingual} mix improves cross-lingual transfer.
            \item \textbf{Deployment unchanged.} All additions are \emph{training-only}; at inference the model reverts to the original fast SigLIP path: encoder-only dual towers with sigmoid scoring.
        \end{itemize}
        
        \begin{figure}[H]
            \centering
            \includegraphics[width=0.6\textwidth]{Figures/Chapter_24/SigLIP2_approach.jpg}
            \caption{\textbf{SigLIP~2 training recipe (conceptual).} Starting from SigLIP’s pairwise sigmoid alignment~\cite{zhai2023_siglip}, pretraining adds (i) a lightweight decoder to inject localization/grounding supervision (captioning, grounded captioning, referring expressions) that shapes patch features but is \emph{dropped at test time}~\cite{wan2024_locca}; (ii) a late \emph{consistency tail} where an EMA teacher provides full-image targets for student crops and masked patches, improving global--local agreement and contextual completion~\cite{naeem2023_silc,maninis2025_tips}; and (iii) resolution/aspect adaptations, either via size-specific continuations or a single \emph{NaFlex} checkpoint that supports multiple grids and native aspect ratios~\cite{dehghani2023_navit,beyer2023_flexivit}. Optional curated fine-tunes further boost compact models~\cite{udandarao2025_acid}. \emph{Courtesy: SigLIP~2 authors.}}
            \label{fig:chpapter24_siglip2_overview}
        \end{figure}
        
        \newpage
        
        \subsubsection{Foundational Reminder: How Sigmoid Loss (SigLIP) Works}
        \label{subsubsec:siglip2_foundation}
        \noindent
        Before describing the extensions in \emph{SigLIP~2}, it is useful to recall the idea behind \emph{SigLIP}~\cite{zhai2023_siglip}. Unlike CLIP, which aligns images and texts by contrasting every pair in a batch through a softmax-normalized InfoNCE loss, SigLIP treats alignment as a set of independent \emph{binary classification problems}. Each image--text pair is scored by a logistic regressor: true pairs should have high probability, false pairs low. This removes the need for large batches and makes the training objective more flexible, while still encouraging globally aligned embeddings.
        
        \paragraph{Compact formulation (per step)}
        Let $z^{\text{img}}_i, z^{\text{text}}_j \in \mathbb{R}^d$ be $\ell_2$-normalized embeddings, $t=\exp(t')$ a learned temperature, and $b$ a learned bias. The pairwise logit and sigmoid loss are
        \[
        \ell_{ij} \;=\; t \,\big\langle z^{\text{img}}_i,\, z^{\text{text}}_j \big\rangle + b,
        \qquad
        \mathcal{L}_{\sigma} \;=\; - \sum_{i,j}\!\big[y_{ij}\log\sigma(\ell_{ij}) + (1-y_{ij})\log(1-\sigma(\ell_{ij}))\big],
        \]
        with $y_{ij}=1$ for a true match and $0$ otherwise. This is the \emph{anchor signal} that drives SigLIP training. 
        
        \noindent
        \emph{SigLIP~2 preserves this same core loss}, and instead improves the learned representations by layering additional pretraining signals—such as a decoder for localization, late-stage consistency and masking, and resolution/multilingual adaptations—around the dual-encoder backbone. These new ingredients are training-only, leaving inference as efficient as the original SigLIP.
        
        \newpage
        
        \subsubsection{Method: A Staged Curriculum that Teaches \emph{Where}, \emph{Detail}, and \emph{Robustness}}
        \label{subsubsec:siglip2_method}
        
        \paragraph{Stage layout (flow first).}
        \begin{itemize}
            \item \textbf{Main phase (0--80\%).} Sigmoid image--text alignment \emph{plus} a lightweight decoder for captioning/grounding: learn global “\emph{whether}” while injecting “\emph{where}” so patch tokens carry region evidence early.
            \item \textbf{Consistency phase (80--100\%).} Add self-distillation and masked prediction (no freezing): enforce \emph{part--whole} agreement and \emph{context completion} once alignment/captioning are stable.
            \item \textbf{Resolution tail (optional).} Publish fixed-resolution specialists via short continuations, or train one \emph{NaFlex} generalist that preserves native aspect ratios across multiple sequence lengths.
            \item \textbf{Small-model curation (optional).} For ViT-B/16, B/32, apply \emph{ACID} to select high-learnability pairs and optimize the same sigmoid loss on curated data.
        \end{itemize}
        At inference, decoder/teacher/masking/curation are removed; the model is the SigLIP-style dual encoder.
        
        \paragraph{Decoder for captioning and grounding (LocCa-style)}
        \begin{itemize}
            \item \textbf{Role.} Add \emph{where} to SigLIP’s \emph{whether}: a small Transformer decoder (2–4 layers) cross-attends to \emph{unpooled} patch tokens during pretraining and is discarded at test time.
            \item \textbf{Mechanism.} Optimize three supervised objectives on top of patch tokens:
            \begin{align*}
                \mathcal{L}_\text{cap} &= -\sum_{t}\log p(w_t \mid w_{<t}, \{f_p\}) \quad\text{(image captioning)}\\
                \mathcal{L}_\text{gcap} &= -\sum_{t}\log p(w_t \mid w_{<t}, \{f_p\}_{p\in\mathcal{R}}) \quad\text{(grounded captioning)}\\
                \mathcal{L}_\text{ref} &= -\log\frac{\exp(\langle z_\text{phrase}, z_{\mathcal{R}}\rangle/\tau)}{\sum_k \exp(\langle z_\text{phrase}, z_{\mathcal{R}_k}\rangle/\tau)} \quad\text{(referring expressions)}
            \end{align*}
            where $f_p$ are patch features, $\mathcal{R}$ a region (box/mask), and $z_{\mathcal{R}}$ a pooled region embedding. Region–text pairs are auto-mined with n-grams and open-vocabulary detectors~\cite{wan2024_locca}. The combined loss $\mathcal{L}_\text{dec}=\sum \lambda_\bullet \mathcal{L}_\bullet$ is added to the sigmoid anchor.
            \item \textbf{Effect.} Patch tokens become spatially grounded (who/what/\emph{where}), improving transfer to grounding/OCR while keeping deployment cost unchanged.
        \end{itemize}
        
        \paragraph{Late self-distillation and masked prediction (SILC/TIPS-style)}
        \begin{itemize}
            \item \textbf{Role.} Upgrade patch tokens from global proxies to \emph{locally coherent} features via two self-supervised signals.
            \item \textbf{Mechanism (added late).} Use an EMA \emph{teacher} (full image) and multiple \emph{student} views (crops/augments), applied to vision-only augmented views with small weights:
            \begin{align*}
                \mathcal{L}_\text{cons} &= \| g(z^\text{pool}_s) - g(z^\text{pool}_t) \|_2^2 \quad\text{(SILC: global consistency)}\\
                \mathcal{L}_\text{mask} &= \sum_{m\in\mathcal{M}} \big\| h(f^s_{\setminus m}) - f^t_m \big\|_2^2 \quad\text{(TIPS: masked per-patch completion)}
            \end{align*}
            with one teacher view, eight student crops, EMA decay $\approx 0.999$, and small projection heads $g,h$~\cite{naeem2023_silc,maninis2025_tips}.
            \item \textbf{Effect.} Crops align with full-image semantics; masked regions are predictable from context. Dense-task transfer improves without any inference change.
        \end{itemize}
        
        \paragraph{Resolution and aspect-ratio adaptation}
        \noindent\emph{Goal.} Eliminate square-warping drift while preserving encoder-only runtime.
        \begin{itemize}
            \item \textbf{Fixed-resolution continuation (specialists).} From $\sim$95\% progress, resume briefly at a new grid (e.g., $14{\times}14\!\to\!24{\times}24$): switch input resize, bilinearly (anti-aliased) retarget 2D positional embeddings
            \[
            PE'_{u',v'}=\sum_{u,v}\alpha_{u,v\to u',v'}\,PE_{u,v},\quad \sum_{u,v}\alpha_{u,v\to u',v'}=1,
            \]
            and optionally adapt the patch stem if patch size changes. Continue with the same losses; publish size-specific checkpoints at minimal cost.
            \item \textbf{NaFlex variant (one generalist).} Train a single checkpoint that preserves native aspect ratios and supports multiple lengths~\cite{dehghani2023_navit,beyer2023_flexivit}. Per batch: sample $L\!\in\!\{128,256,576,784,1024\}$; resize so $H,W$ are patch-size multiples with minimal padding; bilinearly resize the 2D positional grid to $(H,W)$; mask padding in attention/pooling:
            \[
            \text{Attn}(Q,K,V,M)=\text{softmax}\!\big(\tfrac{QK^\top}{\sqrt{d}}+M\big)V,\quad M_{ij}=\begin{cases}-\infty&\text{if pad}\\0&\text{otherwise.}\end{cases}
            \]
            Omit consistency/masking here for stability. Outcome: one encoder that “bends without warping” on documents/UIs/panoramas with no runtime penalty.
        \end{itemize}
        
        \paragraph{Curation-focused fine-tuning for small models}
        \noindent\emph{Goal.} Lift B-sized checkpoints where data quality, not capacity, is limiting.
        \begin{itemize}
            \item \textbf{ACID in SigLIP~2}~\cite{udandarao2025_acid}. Distill \emph{through data} (selection, not logits). For a super-batch $\mathcal{S}$, score pairs with teacher confidence and student uncertainty,
            \[
            \phi_{ij}=\sigma(\ell^T_{ij})\cdot H(\sigma(\ell^S_{ij})),\quad H(p)=-[p\log p+(1-p)\log(1-p)],
            \]
            keep the top-$k$ by $\phi$, and \emph{optimize the same sigmoid loss} on this curated subset. A single strong teacher (fine-tuned on a curated billion-pair mix) suffices.
            \item \textbf{Effect.} Compact models see hard-but-informative pairs, yielding consistent zero-shot/retrieval gains without any inference changes.
        \end{itemize}
        
        \paragraph{Multilingual training mix}
        \noindent\emph{Goal.} Reduce English skew while keeping English strong.
        \begin{itemize}
            \item \textbf{Design.} Include a non-trivial fraction of non-English image–text pairs (e.g., $\sim$10\%), tokenize with a multilingual tokenizer, and apply simple per-language sampling/balancing; negatives can include cross-language distractors. The objective remains the sigmoid alignment.
            \item \textbf{Effect.} Better cross-lingual retrieval/classification with negligible English regression; the encoder becomes globally reliable.
        \end{itemize}
        
        \paragraph{Why these additions work (unifying intuition)}
        The decoder teaches \emph{where} without altering deployment; the SILC/TIPS tail binds \emph{parts} to \emph{wholes} and teaches contextual fill-in; shape-aware packing prevents geometric/text distortions; ACID feeds the learner its most informative data; and multilingual mixing broadens alignment beyond English. All are \emph{training-only}; the shipped model is the same fast SigLIP dual-encoder with weights \emph{imprinted} for locality, dense semantics, and robustness.
        
        \subsubsection{Experiments and Ablations (Concise)}
        \begin{itemize}
            \item \textbf{Across-scales gains (0-shot + retrieval).}
            With comparable compute, \emph{SigLIP~2 B/16@256} lifts ImageNet 0-shot from $76.7\%$ to $\mathbf{79.1\%}$, COCO T$\!\to\!$I R@1 from $47.4\%$ to $\mathbf{53.2\%}$, and XM3600 T$\!\to\!$I R@1 from $22.5\%$ to $\mathbf{40.7\%}$.
            
            \item \textbf{Grounding \& referring expressions (decoder teaches ``where'').}
            Large REC gains persist after discarding the decoder: \emph{B/16@256} RefCOCO val $64.05\!\to\!\mathbf{83.76}$, testB $57.89\!\to\!\mathbf{79.57}$; \emph{L/16@576} val $70.76\!\to\!\mathbf{87.28}$, testB $63.79\!\to\!\mathbf{82.85}$.
            
            \item \textbf{Dense-prediction probes (better patch features).}
            With \emph{frozen} encoders (So/14@224): \emph{PASCAL} mIoU $72.0\!\to\!\mathbf{77.1}$; \emph{NYUv2 depth} RMSE $0.576\!\to\!\mathbf{0.493}$; normals improve on both datasets (\emph{NYUv2} $25.9^\circ\!\to\!\mathbf{24.9^\circ}$, \emph{NAVI} $26.0^\circ\!\to\!\mathbf{25.4^\circ}$).
            
            \item \textbf{Late consistency matters (stability without hurting alignment).}
            Add self-distillation + masked prediction only in the last $\sim\!20\%$ of training (at $80\%$), apply on \emph{augmented} views; weights $1.0$ (consistency) and $0.25$ (masked), reweighted by $\{0.25,0.5,1.0,0.5\}$ for B/L/So400m/g. 
            
            \item \textbf{NaFlex helps OCR/docs (native aspect, multi-length).}
            NaFlex outperforms the square model on most OCR/screen retrieval—especially at short sequences; e.g., \emph{B/16@256} HierText T$\!\to\!$I R@1 $6.1\!\to\!\mathbf{7.4}$ and Screen2Words I$\!\to\!$T $22.9\!\to\!\mathbf{26.6}$.
            
            \item \textbf{Fixed-resolution specialists (cheap resolution-specific boosts).}
            Short continuations from $\sim\!95\%$ training produce higher-res checkpoints; for \emph{B/16}: INet 0-shot $79.1\!\to\!80.6\!\to\!81.2$ (256/384/512) and COCO T$\!\to\!$I R@1 $53.2\!\to\!54.6\!\to\!55.2$.
            
            \item \textbf{Curated small models (ACID $\Rightarrow$ stronger B/16, B/32).}
            Brief implicit distillation for B/16,B/32: LR $10^{-5}$, no WD, $\sim$4B extra examples, 0.5 filtering over 64k super-batches—yields the biggest relative gains at B-scale.
            
            \item \textbf{Multilingual mix (global transfer, English intact).}
            Training on WebLI with $\sim$90\% English / 10\% non-English plus de-biasing yields strong cross-lingual retrieval (see XM3600), while keeping English performance high. 
            
            \item \textbf{Compute and deployment invariant.}
            The decoder and auxiliary heads are \emph{training-only}; the released model is the same encoder-only dual tower (swap-in compatible). 
        \end{itemize}
        
        \paragraph{What we learn (vs.\ SigLIP/BLIP/BLIP-2) \& which to choose}
        From the results and ablations, \emph{SigLIP-2} upgrades the same encoder-only dual tower with stronger \emph{what+where} features at unchanged runtime: vs.\ \emph{SigLIP} it brings robust zero-shot/retrieval gains, large boosts on referring expressions (decoder-imprinted localization), markedly better dense probes (late SILC/TIPS), plus NaFlex and brief high-res continuations for domain/resolution specialization. \emph{BLIP} targets unified understanding/generation without an external LLM, while \emph{BLIP-2} bridges a frozen vision encoder to a frozen LLM via a Q-Former, excelling at open-ended generation but incurring LLM-dependent inference. \textbf{Which to choose in practice:} if you need \emph{fast retrieval/classification/grounding} with no inference overhead, pick \textbf{SigLIP-2}; if you need \emph{text generation} (captioning, VQA-style reasoning) without a large LLM, use \textbf{BLIP}; if you need \emph{LLM-quality, open-ended outputs} or instruction-style prompting, use \textbf{BLIP-2} (accepting LLM latency/footprint). For an empirical feel of embedding behaviors across baselines (CNN/ViT/CLIP/BLIP-2; note \emph{SigLIP-2 is not included}), see this concise comparison: \href{[https://pub.towardsai.net/vision-embedding-comparison-for-image-similarity-search-efficientnet-vs-4eac6bf553c4}{Vision](https://pub.towardsai.net/vision-embedding-comparison-for-image-similarity-search-efficientnet-vs-4eac6bf553c4}{Vision) embedding comparison for image similarity}.
    
    \end{enrichment}
    
\end{enrichment}

\newpage
 
\begin{enrichment}[Self-Supervised Video Pretraining for VLLMs][section]
    Self-supervised pretraining has become the dominant strategy for learning scalable video backbones, discarding labels in favor of proxy objectives on raw clips (e.g., masked reconstruction or feature prediction). For video-language models, such pretraining is crucial: the LLM can only reason over video content if its encoder supplies rich spatiotemporal representations. This section highlights three representative approaches that defined the state of the art: \emph{VideoMAE}, which showed the effectiveness of extreme tubelet masking for masked autoencoding \cite{tong2022_videomae}; \emph{VideoMAEv2}, which extended this recipe with dual masking and larger ViTs for improved scalability \cite{wang2023_videomaev2}; and \emph{MVD}, which replaced pixel targets with teacher features for more semantic supervision \cite{wang2023_mvd}. Emerging directions include hybrid masked–contrastive objectives and leveraging complementary signals such as audio or motion priors to further enrich pretraining.
    
    \begin{enrichment}[VideoMAE: Masked Autoencoders for Video SSL][subsection]
        \label{enr:subsec_chapter24_videomae}
    
    \paragraph{Scope and positioning}
    VideoMAE \cite{tong2022_videomae} adapts image MAE to videos while explicitly neutralizing temporal shortcuts. Two choices make the objective both \emph{difficult} and \emph{efficient}: (i) \textbf{very high masking} that hides 90--95\% of tokens from the input \emph{clip} (a sampled subsequence of $T$ frames), and (ii) \textbf{tube masking}. In tube masking, a single \emph{spatial} mask is sampled once on the patch grid and then \emph{broadcast across the full temporal span of the clip}. Practically, if the spatial patch at $(x,y)$ is selected for masking, that same location is masked in \emph{every} frame of the clip. A vanilla ViT is used in an \emph{asymmetric} encoder--decoder: the encoder processes only visible tokens (about 5--10\%), and a lightweight decoder reconstructs normalized pixels for the masked tokens. Compared with image MAE, VideoMAE uses higher masking ratios and \emph{temporally aligned} masks, matching video’s stronger redundancy and preventing frame-to-frame copy shortcuts.
    
    \subsubsection{Motivation}
    \label{subsubsec_chapter24_videomae_motivation}
    
    \paragraph{Why masked autoencoding for video}
    Videos exhibit \emph{slowness} and \emph{redundancy}: adjacent frames are highly similar, so naive per-frame masking leaves near-duplicates visible and enables trivial copying. VideoMAE blocks this shortcut by combining: (i) an \textbf{extremely high masking ratio} (90--95\%), and (ii) \textbf{tube masking} that aligns the mask across time. A key point is to separate what the \emph{tokens} are from how \emph{masking} is applied:
    
    \begin{itemize}
        \item \textbf{Tokens are cubes; masking units are tubes.} Tokens are short spatio\-temporal cubes (time $\times$ height $\times$ width), e.g., $k{\times}P{\times}P = 2{\times}16{\times}16$. A \emph{tube} is the stack of all cubes that share a spatial location $(x,y)$ across the entire clip.
        \item \textbf{Share one spatial mask across all $T$ frames.} The same 2D mask is repeated over time, so once $(x,y)$ is chosen, \emph{all} cubes at $(x,y)$ for the clip are hidden. This eliminates frame-to-frame leakage at the same location and forces non-local reasoning.
    \end{itemize}
    
    \medskip
    \noindent\textbf{Step 1 --- Tokens are \emph{cubes}.}
    After temporal subsampling by stride $\tau$, the $T$-frame clip is partitioned into cubes of size $k{\times}P{\times}P$ (e.g., $k{=}2$, $P{=}16$). Each cube is one token. Let
    \[
    N_t=\frac{T}{k},\qquad N_h=\frac{H}{P},\qquad N_w=\frac{W}{P},
    \]
    so tokens are indexed by $(t',x,y)$ with $t'\!\in\!\{1,\dots,N_t\}$ and $(x,y)\!\in\!\{1,\dots,N_h\}\times\{1,\dots,N_w\}$.
    
    \newpage 
    
    \noindent\textbf{Step 2 --- Masking decisions are made per \emph{tube}.}
    A \emph{tube} is the set $\{(t',x,y): t'=1,\dots,N_t\}$ at fixed $(x,y)$. Tube masking samples a 2D Bernoulli mask on the spatial grid with ratio $\rho$:
    \[
    m_{x,y}\sim\mathrm{Bernoulli}(\rho),\qquad \rho\in\{0.9,0.95\},
    \]
    and broadcasts it along time to define the masked index set
    \[
    \Omega=\{(t',x,y)\mid m_{x,y}=1,\; t'=1,\dots,N_t\}.
    \]
    Thus, although a \emph{token} (cube) spans only $k$ frames, the \emph{masking unit} (tube) spans the full clip length $T$ (i.e., all $N_t$ cubes at that $(x,y)$). If $(x,y)$ is masked, every cube at $(x,y)$ for the clip is masked.
    
    \medskip
    \noindent\textbf{Why this matters.}
    With 90--95\% of tubes masked, the encoder receives only $\approx$5--10\% of tokens and must integrate non-local, long-range space--time cues to reconstruct, instead of copying nearby pixels. The \emph{asymmetric} design keeps compute low by applying attention only to visible tokens; the lightweight decoder handles reconstruction. \emph{Scope note:} masking applies only within the sampled $T$-frame clip; frames outside the clip are neither seen nor reconstructed in that step.
    
    \begin{figure}[H]
        \centering
        \includegraphics[width=0.85\linewidth]{Figures/Chapter_24/VideoMAE_overview.jpg}
        \caption{\textbf{Overview of VideoMAE}. VideoMAE masks random spatio\-temporal cubes and reconstructs them with an asymmetric encoder–decoder. Owing to high redundancy and temporal correlation, the authors introduce \emph{tube masking} with an extremely high ratio (90–95\%), which yields a harder and more meaningful self-supervised task and drives the encoder to capture useful spatiotemporal structure \cite{tong2022_videomae}.}
        \label{fig:chapter24_videomae_overview}
    \end{figure}
    
    \begin{figure}[H]
        \centering
        \includegraphics[width=0.85\linewidth]{Figures/Chapter_24/VideoMAE_tube_masking.jpg}
        \caption{\textbf{Tube masking vs.\ alternatives}. (a) Slowness induces temporal redundancy and correlation. (b) Frame masking and (c) random masking risk information leakage by leaving correlated duplicates unmasked. (d) Tube masking enforces the same spatial mask for all frames, removing easy copies and promoting representative spatiotemporal learning \cite{tong2022_videomae}.}
        \label{fig:chapter24_videomae_tube}
    \end{figure}
    
    \newpage
    
    \subsubsection{Method}
    \label{subsubsec_chapter24_videomae_method}
    
    \paragraph{Preliminaries and notation}
    Let a video $V$ provide a clip of $t$ consecutive RGB frames. VideoMAE temporally subsamples with stride $\tau$ to obtain $T$ frames $I \in \mathbb{R}^{T \times H \times W \times 3}$. Each frame is partitioned into non-overlapping $16{\times}16$ patches and packed across time into \emph{cubes} of size $2{\times}16{\times}16$; a cube becomes one input token via a linear projection to $\mathbb{R}^D$. This yields $\frac{T}{2}\!\times\!\frac{H}{16}\!\times\!\frac{W}{16}$ tokens. 
    
    \paragraph{Tube masking with extremely high ratios}
    Let $\Omega$ denote the set of masked cube indices and $\rho \in [0,1)$ the masking ratio. Tube masking is applied by sampling a \emph{spatial} Bernoulli mask once and sharing it across all $t$:
    \begin{equation}
    \label{eq:chapter24_videomae_tube}
    \mathbb{I}[p_{x,y,\cdot} \in \Omega] \sim \mathrm{Bernoulli}(\rho)\quad\text{with the same outcome for all times}\;,
    \end{equation}
    so that if location $(x,y)$ is masked at one frame, it is masked for all frames \cite{tong2022_videomae}. Empirically, $\rho \in \{0.9,0.95\}$ optimizes difficulty and efficiency; lower ratios leave too much redundant evidence, and higher ratios (\,$\geq 0.98$\,) degrade accuracy on SSV2 and K400 (see Fig.~\ref{fig:chapter24_videomae_maskratio}). 
    
    \paragraph{Asymmetric encoder–decoder}
    Only \emph{visible} tokens $\{z_v\}$ (roughly $(1{-}\rho)$ of all tokens) enter the ViT encoder $\Phi_{\text{enc}}$ with joint space–time attention. A lightweight decoder $\Phi_{\text{dec}}$ receives (i) encoded visible features and (ii) \emph{learnable} mask tokens as placeholders for $\Omega$, and predicts reconstructed pixels $\hat{I}$ for all masked cubes. The asymmetry reduces pre-train cost because expensive self-attention is computed on $\approx 5{\sim}10\%$ of tokens \cite{tong2022_videomae}. 
    
    \paragraph{Reconstruction objective on masked cubes}
    Following ImageMAE, VideoMAE normalizes pixels per channel and minimizes MSE only over masked positions:
    \begin{equation}
    \label{eq:chapter24_videomae_loss}
    \mathcal{L}_{\mathrm{MAE}} \;=\; \frac{1}{|\Omega|}\sum_{p \in \Omega} \left\| I(p) - \hat{I}(p) \right\|_2^2,
    \end{equation}
    where $p$ indexes masked cubes, $I$ is the downsampled target clip, and $\hat{I}$ is the decoder output \cite{tong2022_videomae}. 
    
    \paragraph{Design choices justified}
    \begin{itemize}
        \item \textbf{Temporal downsampling.} Using stride $\tau\!\in\!\{4,2\}$ on K400/SSV2 reduces redundancy and balances static and motion cues without collapsing the temporal field of view. 
        \item \textbf{Cube embedding.} 3D tokens ($2{\times}16{\times}16$) jointly reduce spatial and temporal lengths, improving efficiency and encouraging space–time reasoning in attention layers. 
        \item \textbf{Tube masking.} Sharing the spatial mask across time removes trivial spatiotemporal correspondences that would otherwise let the model copy from adjacent frames, thereby elevating the task’s semantic level.
        \item \textbf{High masking ratio.} Videos contain lower information density than images; aggressively masking encourages holistic structure modeling while cutting encoder FLOPs proportionally to $(1{-}\rho)$.  
        \item \textbf{Pixel-space target and MSE.} Reconstructing normalized pixels with MSE outperforms L1 and Smooth-L1 for this setting; predicting only the center frame is inferior to reconstructing the full $T{\times}\tau$ target. 
    \end{itemize}

    \paragraph{Algorithmic flow (pseudo code)}
    
    \begin{mintedbox}{python}
    # VideoMAE pre-training loop (schematic)
    # I: sampled clip of shape [T, H, W, 3]; tau: temporal stride; rho: masking ratio
    # enc, dec: ViT encoder/decoder; proj: cube embed; mtoken: learnable mask token
    
    def step(I):
        # 1) Temporal downsampling
        I_tau = I[::tau]  # shape [T, H, W, 3]
        
        # 2) Cube embedding (2 x 16 x 16) -> tokens
        X = proj(cubeify(I_tau))  # [N_tokens, D]
        
        # 3) Tube masking: sample a 2D mask once and share across time
        spatial_mask = bernoulli_mask_2d(height=H//16, width=W//16, p=rho)
        tube_mask = repeat_across_time(spatial_mask, repeats=T//2)  # indices omega
        
        visible, indices_vis = X[~tube_mask], where(~tube_mask)
        masked_indices = where(tube_mask)
        
        # 4) Encode only visible tokens (asymmetric design)
        H_vis = enc(visible)
        
        # 5) Prepare decoder sequence: interleave encoded visibles with mask tokens
        Z = stitch_sequence(H_vis, indices_vis, mtoken, masked_indices)
        
        # 6) Decode to pixel targets for masked cubes and compute loss
        I_hat = dec(Z)                     # predict all cubes; train via masked MSE
        loss = mse(I_hat[masked_indices], I_tau[masked_indices])
        return loss
    \end{mintedbox}
    
    \subsubsection{Architecture, Training, and Datasets}
    \label{subsubsec_chapter24_videomae_arch}
    
    \paragraph{Backbone and attention}
    VideoMAE employs a vanilla ViT with \emph{joint space--time} self-attention as encoder, so any token can attend to any other across frames and spatial positions. The decoder is shallower and narrower (half channels of encoder; $4$ blocks by default), which reduces cost while retaining sufficient capacity to reconstruct masked cubes \cite{tong2022_videomae}.
    
    \paragraph{Training setup}
    The default backbone is ViT-B with $T{=}16$ frames, cube size $2{\times}16{\times}16$, and masking ratio $\rho{=}90\%$. Pre-training runs for $800$ epochs (on SSV2 and K400) with per-channel pixel normalization and MSE loss on masked cubes. Fine-tuning uses TSN sampling for SSV2 and dense sampling for K400; inference uses $2{\times}3$ crops on SSV2 and $5{\times}3$ crops on K400 \cite{tong2022_videomae}.
    
    \newpage
    
    \paragraph{Datasets used in experiments and ablations}
    \begin{itemize}
        \item \textbf{K400} (Kinetics-400). $\sim$240k YouTube clips over 400 actions; primary large-scale benchmark for pre-training and fine-tuning.
        \item \textbf{K700} (Kinetics-700). Extension of Kinetics with 700 classes; used for ablations and AVA detection pre-train variants.
        \item \textbf{SSV2} (Something-Something V2). $\sim$220k crowd-acted object-manipulation videos with fine-grained motion; used heavily in ablations and to test temporal sensitivity.
        \item \textbf{UCF101}. 9.5k clips across 101 actions; classic small-scale benchmark, used for transfer evaluation after K400 pre-train.
        \item \textbf{HMDB51}. 3.5k clips across 51 actions; another small-scale benchmark for transfer experiments.
        \item \textbf{AVA v2.2}. Atomic Visual Actions detection dataset; used to measure transfer to action detection (mAP), with/without supervised pre-train labels.
        \item \textbf{IN-1K / IN-21K}. ImageNet-1K/21K; appear in ablations for comparing ImageMAE and supervised ImageNet pre-train baselines.
    \end{itemize}
    
    \noindent These datasets cover both large-scale classification (K400/K700, SSV2), small-scale transfer (UCF/HMDB), and detection (AVA), ensuring that VideoMAE is tested across scales and task types.
    
    \medskip
    With this setup established, we now turn to the core \emph{experiments and ablations}, analyzing how masking ratio, decoder design, targets, and pre-training choices shape performance.
    
    \subsubsection{Experiments}
    \label{subsubsec_chapter24_videomae_expts}
    
    \begin{figure}[H]
    \centering
    \includegraphics[width=0.6\linewidth]{Figures/Chapter_24/VideoMAE_finding_the_right_masking_ratio.jpg}
    \caption{\textbf{Effect of masking ratio}. With 16-frame ViT-B, SSV2 and K400 peak around $\rho{=}90\%$; $\rho{>}95\%$ hurts as context becomes too sparse \cite{tong2022_videomae}.}
    \label{fig:chapter24_videomae_maskratio}
    \end{figure}
    
    \begin{figure}[H]
    \centering
    \includegraphics[width=0.6\textwidth]{Figures/Chapter_24/VideoMAE_data_efficiency.jpg}
    \caption{\textbf{Data efficiency on SSV2}. Fixed-iteration pre-training (green) at $132$k steps outperforms fixed-epoch pre-training (blue) when using subsets; notably, $25\%$ SSV2 with more iterations surpasses a K400-pretrained baseline, highlighting the value of domain-matched data \cite{tong2022_videomae}.}
    \label{fig:chapter24_videomae_dataeff}
    \end{figure}
    
    \subsubsection{Ablations}
    \label{subsubsec_chapter24_videomae_ablations}
    
    \begin{table}[H]
        \centering
        \small
        \setlength{\tabcolsep}{6pt}
        \caption{Table 1(a). Decoder depth on SSV2/K400 with 16-frame ViT-B (reproduced from \cite{tong2022_videomae}).}
        \label{tab:videomae_tbl1a}
        \begin{tabular}{cccc}
            \toprule
            Blocks & SSV2 & K400 & GPU mem. \\
            \midrule
            1 & 68.5 & 79.0 & 7.9G \\
            2 & 69.2 & 79.2 & 10.2G \\
            4 & \textbf{69.6} & \textbf{80.0} & 14.7G \\
            8 & 69.3 & 79.7 & 23.7G \\
            \bottomrule
        \end{tabular}
    \end{table}
    \noindent\emph{What we learn.} A shallow, lightweight decoder (4 blocks) is sufficient and most efficient for reconstruction-driven pretraining.
    
    \begin{table}[H]
        \centering
        \small
        \setlength{\tabcolsep}{6pt}
        \caption{Table 1(b). Mask sampling on SSV2/K400 (16-frame ViT-B) (reproduced from \cite{tong2022_videomae}). $^\ast$Frame masking hides $14/16$ frames on SSV2.}
        \label{tab:videomae_tbl1b}
        \begin{tabular}{lccc}
            \toprule
            Case & Ratio & SSV2 & K400 \\
            \midrule
            Tube   & 75   & 68.0 & 79.8 \\
            Tube   & 90   & \textbf{69.6} & \textbf{80.0} \\
            Random & 90   & 68.3 & 79.5 \\
            Frame$^\ast$ & 87.5 & 61.5 & 76.5 \\
            \bottomrule
        \end{tabular}
    \end{table}
    \noindent\emph{What we learn.} Tube masking at very high ratio (90\%) is crucial; frame masking creates shortcuts and hurts learning.
    
    \begin{table}[H]
        \centering
        \small
        \setlength{\tabcolsep}{6pt}
        \caption{Table 1(c). Reconstruction target on SSV2/K400 (16-frame ViT-B) (reproduced from \cite{tong2022_videomae}).}
        \label{tab:videomae_tbl1c}
        \begin{tabular}{lccc}
            \toprule
            Input & Target & SSV2 & K400 \\
            \midrule
            $T{\times}\tau$                    & Center                 & 63.0 & 79.3 \\
            $T{\times}\tfrac{\tau}{2}$         & $T{\times}\tfrac{\tau}{2}$ & 68.9 & 79.8 \\
            $T{\times}\tau$                    & $T{\times}\tau$            & \textbf{69.6} & {80.0} \\
            $T{\times}\tau$                    & $2T{\times}\tfrac{\tau}{2}$ & 69.2 & \textbf{80.1} \\
            \bottomrule
        \end{tabular}
    \end{table}
    \noindent\emph{What we learn.} Reconstructing the full spatiotemporal target (not just the center frame) yields the best representation.
    
    \begin{table}[H]
        \centering
        \small
        \setlength{\tabcolsep}{6pt}
        \caption{Table 1(d). Pre-training strategy on SSV2/K400 (16-frame ViT-B) (reproduced from \cite{tong2022_videomae}).}
        \label{tab:videomae_tbl1d}
        \begin{tabular}{lcc}
            \toprule
            Case & SSV2 & K400 \\
            \midrule
            From scratch      & 32.6 & 68.8 \\
            ImageNet-21k sup. & 61.8 & 78.9 \\
            IN-21k+K400 sup.  & 65.2 & -- \\
            VideoMAE          & \textbf{69.6} & \textbf{80.0} \\
            \bottomrule
        \end{tabular}
    \end{table}
    \noindent\emph{What we learn.} Self-supervised VideoMAE pretraining is far stronger than supervised ImageNet initialization for video.
    
    \begin{table}[H]
        \centering
        \small
        \setlength{\tabcolsep}{6pt}
        \caption{Table 1(e). Pre-training dataset comparison (16-frame ViT-B) (reproduced from \cite{tong2022_videomae}).}
        \label{tab:videomae_tbl1e}
        \begin{tabular}{lccc}
            \toprule
            Dataset & Method & SSV2 & K400 \\
            \midrule
            IN-1K  & ImageMAE & 64.8 & 78.7 \\
            K400   & VideoMAE & 68.5 & \textbf{80.0} \\
            SSV2   & VideoMAE & \textbf{69.6} & 79.6 \\
            \bottomrule
        \end{tabular}
    \end{table}
    \noindent\emph{What we learn.} In-domain video pretraining (SSV2/K400) is superior to image-only MAE for video recognition.
    
    \begin{table}[H]
        \centering
        \small
        \setlength{\tabcolsep}{6pt}
        \caption{Loss function on SSV2/K400 (16-frame ViT-B) (reproduced from \cite{tong2022_videomae}).}
        \label{tab:videomae_tbl1f}
        \begin{tabular}{lcc}
            \toprule
            Case & SSV2 & K400 \\
            \midrule
            L1 loss        & 69.1 & 79.7 \\
            MSE loss       & \textbf{69.6} & \textbf{80.0} \\
            Smooth L1 loss & 68.9 & 79.6 \\
            \bottomrule
        \end{tabular}
    \end{table}
    \noindent\emph{What we learn.} Masked-pixel MSE is the most effective reconstruction loss in this setting.
    
    % =====================================================
    % Table 2: Self-supervised pre-training vs prior work
    % =====================================================
    
    \begin{table}[H]
        \centering
        \small
        \setlength{\tabcolsep}{6pt}
        \caption{Comparison with prior self-supervised pre-training using 16-frame ViT-B and only unlabeled training splits (reproduced from \cite{tong2022_videomae}).}
        \label{tab:videomae_tbl2}
        \begin{tabular}{lrrrr}
            \toprule
            \textbf{Dataset} & \textbf{Train videos} & \textbf{From scratch} & \textbf{MoCo v3} & \textbf{VideoMAE} \\
            \midrule
            K400   & 240k & 68.8 & 74.2 & \textbf{80.0} \\
            SSV2   & 169k & 32.6 & 54.2 & \textbf{69.6} \\
            UCF101 & 9.5k & 51.4 & 81.7 & \textbf{91.3} \\
            HMDB51 & 3.5k & 18.0 & 39.2 & \textbf{62.6} \\
            \bottomrule
        \end{tabular}
    \end{table}
    \noindent\emph{What we learn.} VideoMAE substantially improves over MoCo v3 across diverse datasets, especially on small data (UCF/HMDB).
    
    % =====================================================
    % Table 3: Efficiency
    % =====================================================
    
    \begin{table}[H]
        \centering
        \small
        \setlength{\tabcolsep}{6pt}
        \caption{Pre-training efficiency on SSV2 with 16-frame ViT-B (64$\times$V100), reproduced from \cite{tong2022_videomae}.}
        \label{tab:videomae_tbl3}
        \begin{tabular}{lrrrrr}
            \toprule
            \textbf{Method} & \textbf{Epochs} & \textbf{FT Acc} & \textbf{Lin Acc} & \textbf{Hours} & \textbf{Speedup} \\
            \midrule
            MoCo v3  & 300 & 54.2 & 33.7 & 61.7 & -- \\
            VideoMAE & 800 & \textbf{69.6} & \textbf{38.9} & \textbf{19.5} & \textbf{3.2$\times$} \\
            \bottomrule
        \end{tabular}
    \end{table}
    \noindent\emph{What we learn.} Despite more epochs, VideoMAE is far faster in wall-clock and yields much higher accuracy.
    
    % =====================================================
    % Table 4: Transferability
    % =====================================================
    
    \begin{table}[H]
        \centering
        \small
        \setlength{\tabcolsep}{6pt}
        \caption{Feature transferability: pre-train on K400 (unlabeled), then fine-tune on target datasets (reproduced from \cite{tong2022_videomae}).}
        \label{tab:videomae_tbl4}
        \begin{tabular}{lrrr}
            \toprule
            \textbf{K400$\rightarrow$Target} & \textbf{SSV2} & \textbf{UCF} & \textbf{HMDB} \\
            \midrule
            MoCo v3  & 62.4 & 93.2 & 67.9 \\
            VideoMAE & \textbf{68.5} & \textbf{96.1} & \textbf{73.3} \\
            \bottomrule
        \end{tabular}
    \end{table}
    \noindent\emph{What we learn.} VideoMAE features transfer better to both motion-centric (SSV2) and appearance-centric (UCF/HMDB) targets.
    
    % =====================================================
    % Table 5: AVA action detection (VideoMAE Table 5)
    % =====================================================
    
    \begin{table}[H]
        \centering
        \small
        \setlength{\tabcolsep}{6pt}
        \caption{Comparison with the state of the art on AVA v2.2. All models are pre-trained and fine-tuned at image size $224^2$. We report validation mAP. “Ex.\ labels \xmark” means only \emph{unlabelled} data is used during pre-training and the pre-trained models are directly transferred to AVA; “Ex.\ labels \cmark” additionally fine-tunes on the pre-training dataset with labels before transfer. $T{\times}\tau$ denotes frames$\times$sample-rate. (Numbers from \cite{tong2022_videomae}.)}
        \label{tab:chapter24_videomae_ava_fixed}
        \resizebox{\linewidth}{!}{%
            \begin{tabular}{lccccccc}
                \toprule
                \textbf{Method} & \textbf{Backbone} & \textbf{Pre-train Dataset} & \textbf{Extra labels} & \textbf{$T{\times}\tau$} & \textbf{GFLOPs} & \textbf{Param} & \textbf{mAP} \\
                \midrule
                supervised \cite{feichtenhofer2019_slowfast} & SlowFast\,-\,R101 & Kinetics\mbox{-}400 & \cmark & $8{\times}8$   & 138  & 53  & 23.8 \\
                CVRL \cite{qian2021_cvrl}                    & SlowOnly\,-\,R50  & Kinetics\mbox{-}400 & \xmark & $32{\times}2$  & 42   & 32  & 16.3 \\
                $\rho$BYOL$_{\rho=3}$ \cite{feichtenhofer2021_r2plus1d_ssl} & SlowOnly\,-\,R50 & Kinetics\mbox{-}400 & \xmark & $8{\times}8$  & 42   & 32  & 23.4 \\
                $\rho$MoCo$_{\rho=3}$ \cite{feichtenhofer2021_r2plus1d_ssl} & SlowOnly\,-\,R50 & Kinetics\mbox{-}400 & \xmark & $8{\times}8$  & 42   & 32  & 20.3 \\
                MaskFeat$^{\uparrow 312}$ \cite{wei2022_maskfeat} & MViT\,-\,L  & Kinetics\mbox{-}400 & \cmark & $40{\times}3$  & 2828 & 218 & 37.5 \\
                MaskFeat$^{\uparrow 312}$ \cite{wei2022_maskfeat} & MViT\,-\,L  & Kinetics\mbox{-}600 & \cmark & $40{\times}3$  & 2828 & 218 & 38.8 \\
                \midrule
                \textbf{VideoMAE} \cite{tong2022_videomae} & ViT\,-\,S & Kinetics\mbox{-}400 & \xmark & $16{\times}4$ & 57   & 22  & 22.5 \\
                \textbf{VideoMAE} \cite{tong2022_videomae} & ViT\,-\,S & Kinetics\mbox{-}400 & \cmark & $16{\times}4$ & 57   & 22  & 28.4 \\
                \textbf{VideoMAE} \cite{tong2022_videomae} & ViT\,-\,B & Kinetics\mbox{-}400 & \xmark & $16{\times}4$ & 180  & 87  & 26.7 \\
                \textbf{VideoMAE} \cite{tong2022_videomae} & ViT\,-\,B & Kinetics\mbox{-}400 & \cmark & $16{\times}4$ & 180  & 87  & 31.8 \\
                \textbf{VideoMAE} \cite{tong2022_videomae} & ViT\,-\,L & Kinetics\mbox{-}400 & \xmark & $16{\times}4$ & 597  & 305 & 34.3 \\
                \textbf{VideoMAE} \cite{tong2022_videomae} & ViT\,-\,L & Kinetics\mbox{-}400 & \cmark & $16{\times}4$ & 597  & 305 & 37.0 \\
                \textbf{VideoMAE} \cite{tong2022_videomae} & ViT\,-\,H & Kinetics\mbox{-}400 & \xmark & $16{\times}4$ & 1192 & 633 & \textbf{36.5} \\
                \textbf{VideoMAE} \cite{tong2022_videomae} & ViT\,-\,H & Kinetics\mbox{-}400 & \cmark & $16{\times}4$ & 1192 & 633 & \textbf{39.5} \\
                \textbf{VideoMAE} \cite{tong2022_videomae} & ViT\,-\,L & Kinetics\mbox{-}700 & \xmark & $16{\times}4$ & 597  & 305 & \textbf{36.1} \\
                \textbf{VideoMAE} \cite{tong2022_videomae} & ViT\,-\,L & Kinetics\mbox{-}700 & \cmark & $16{\times}4$ & 597  & 305 & \textbf{39.3} \\
                \bottomrule
        \end{tabular}}
    \end{table}
    
    % =====================================================
    % Table 6: Something–Something V2 SOTA (VideoMAE Table 6)
    % =====================================================
    
    \begin{table}[H]
        \centering
        \small
        \setlength{\tabcolsep}{6pt}
        \caption{Comparison with the state of the art on Something–Something V2. VideoMAE reconstructs normalized cube pixels and is pre-trained with 90\% masking for 2400 epochs. “Ex.\ labels \xmark” means only \emph{unlabelled} data is used during pre-training. (Numbers from \cite{tong2022_videomae}.)}
        \label{tab:chapter24_videomae_ssv2_fixed}
        \resizebox{\linewidth}{!}{%
            \begin{tabular}{lcccccccc}
                \toprule
                \textbf{Method} & \textbf{Backbone} & \textbf{Extra data} & \textbf{Ex.\ labels} & \textbf{Frames} & \textbf{GFLOPs} & \textbf{Param} & \textbf{Top-1} & \textbf{Top-5} \\
                \midrule
                TEINet$^{\text{En}}$ \cite{li2021_teinet} & ResNet50$\times$2 & ImageNet\mbox{-}1K & \cmark & $8{+}16$ & $99{\times}10{\times}3$  & 50  & 66.5 & N/A \\
                TANet$^{\text{En}}$ \cite{li2021_tanet}   & ResNet50$\times$2 & ImageNet\mbox{-}1K & \cmark & $8{+}16$ & $99{\times}2{\times}3$   & 51  & 66.0 & 90.1 \\
                TDN$^{\text{En}}$ \cite{wang2021_tdn}     & ResNet101$\times$2& ImageNet\mbox{-}1K & \cmark & $8{+}16$ & $198{\times}1{\times}3$ & 88  & 69.6 & 92.2 \\
                SlowFast \cite{feichtenhofer2019_slowfast} & ResNet101 & Kinetics\mbox{-}400 & \cmark & $8{+}32$ & $106{\times}1{\times}3$ & 53  & 63.1 & 87.6 \\
                MViTv1 \cite{fan2021_mvit}              & MViTv1\,-\,B & Kinetics\mbox{-}400 & \cmark & 64    & $455{\times}1{\times}3$ & 37  & 67.7 & 90.9 \\
                TimeSformer \cite{bertasius2021_timesformer} & ViT\,-\,B & ImageNet\mbox{-}21K & \cmark & 8     & $196{\times}1{\times}3$ & 121 & 59.5 & N/A \\
                TimeSformer \cite{bertasius2021_timesformer} & ViT\,-\,L & ImageNet\mbox{-}21K & \cmark & 64    & $5549{\times}1{\times}3$& 430 & 62.4 & N/A \\
                ViViT FE \cite{arnab2021_vivit}         & ViT\,-\,L & IN\mbox{-}21K+K400 & \cmark & 32    & $995{\times}4{\times}3$ & N/A & 65.9 & 89.9 \\
                Motionformer \cite{patrick2021_motionformer} & ViT\,-\,B & ImageNet\mbox{-}21K & \cmark & 16    & $370{\times}1{\times}3$ & 109 & 66.5 & 90.1 \\
                Motionformer \cite{patrick2021_motionformer} & ViT\,-\,L & ImageNet\mbox{-}21K & \cmark & 32    & $1185{\times}1{\times}3$& 382 & 68.1 & 91.2 \\
                Video Swin \cite{liu2022_videoswin}     & Swin\,-\,B & Kinetics\mbox{-}400 & \cmark & 32    & $321{\times}1{\times}3$ & 88  & 69.6 & 92.7 \\
                VIMPAC \cite{tan2021_vimpac}            & ViT\,-\,L & HowTo100M+DALLE & \xmark & 10   & N/A${\times}10{\times}3$ & 307 & 68.1 & N/A \\
                BEVT \cite{wang2022_bevt}               & Swin\,-\,B & IN\mbox{-}1K+K400+DALLE & \xmark & 32 & $321{\times}1{\times}3$ & 88 & 70.6 & N/A \\
                MaskFeat$^{\uparrow 312}$ \cite{wei2022_maskfeat} & MViT\,-\,L & Kinetics\mbox{-}600 & \cmark & 40 & $2828{\times}1{\times}3$& 218 & 75.0 & 95.0 \\
                \midrule
                \textbf{VideoMAE} \cite{tong2022_videomae} & ViT\,-\,B & Kinetics\mbox{-}400 & \xmark & 16 & $180{\times}2{\times}3$ & 87  & 69.7 & 92.3 \\
                \textbf{VideoMAE} \cite{tong2022_videomae} & ViT\,-\,L & Kinetics\mbox{-}400 & \xmark & 16 & $597{\times}2{\times}3$ & 305 & 74.0 & 94.6 \\
                \textbf{VideoMAE} \cite{tong2022_videomae} & ViT\,-\,S & no external data & \xmark & 16 & $57{\times}2{\times}3$  & 22  & 66.8 & 90.3 \\
                \textbf{VideoMAE} \cite{tong2022_videomae} & ViT\,-\,B & no external data & \xmark & 16 & $180{\times}2{\times}3$ & 87  & 70.8 & 92.4 \\
                \textbf{VideoMAE} \cite{tong2022_videomae} & ViT\,-\,L & no external data & \xmark & 16 & $597{\times}2{\times}3$ & 305  & 74.3 & 94.6 \\
                \textbf{VideoMAE} \cite{tong2022_videomae} & ViT\,-\,L & no external data & \xmark & 32 & $1436{\times}1{\times}3$& 305 & \textbf{75.4} & \textbf{95.2} \\
                \bottomrule
        \end{tabular}}
    \end{table}
    
    % =====================================================
    % Table 7: Kinetics-400 SOTA (VideoMAE Table 7)
    % =====================================================
    
    \begin{table}[H]
        \centering
        \small
        \setlength{\tabcolsep}{6pt}
        \caption{Comparison with the state of the art on Kinetics-400. VideoMAE models are self-supervised with 90\% masking for 1600 epochs on K400. \textbf{VideoMAE}$^{\uparrow 320}$ is initialized from its $224^2$ counterpart and then fine-tuned at $320^2$. “Ex.\ labels \xmark” means only \emph{unlabelled} data is used during pre-training. (Numbers from \cite{tong2022_videomae}.)}
        \label{tab:chapter24_videomae_k400_fixed}
        \resizebox{\linewidth}{!}{%
            \begin{tabular}{lcccccccc}
                \toprule
                \textbf{Method} & \textbf{Backbone} & \textbf{Extra data} & \textbf{Ex.\ labels} & \textbf{Frames} & \textbf{GFLOPs} & \textbf{Param} & \textbf{Top-1} & \textbf{Top-5} \\
                \midrule
                NL I3D \cite{wang2018_nonlocal_nn} & ResNet101 & ImageNet\mbox{-}1K & \cmark & 128 & $359{\times}10{\times}3$ & 62  & 77.3 & 93.3 \\
                TANet \cite{li2021_tanet}          & ResNet152 & ImageNet\mbox{-}1K & \cmark & 16  & $242{\times}4{\times}3$  & 59  & 79.3 & 94.1 \\
                TDN$^{\text{En}}$ \cite{wang2021_tdn} & ResNet101 & ImageNet\mbox{-}1K & \cmark & $8{+}16$ & $198{\times}10{\times}3$ & 88 & 79.4 & 94.4 \\
                TimeSformer \cite{bertasius2021_timesformer} & ViT\,-\,L & ImageNet\mbox{-}21K & \cmark & 96 & $8353{\times}1{\times}3$ & 430 & 80.7 & 94.7 \\
                ViViT FE \cite{arnab2021_vivit}    & ViT\,-\,L & ImageNet\mbox{-}21K & \cmark & 128 & $3980{\times}1{\times}3$ & N/A & 81.7 & 93.8 \\
                Motionformer \cite{patrick2021_motionformer} & ViT\,-\,L & ImageNet\mbox{-}21K & \cmark & 32 & $1185{\times}10{\times}3$ & 382 & 80.2 & 94.8 \\
                Video Swin \cite{liu2022_videoswin} & Swin\,-\,L & ImageNet\mbox{-}21K & \cmark & 32 & $604{\times}4{\times}3$ & 197 & 83.1 & 95.9 \\
                ViViT FE \cite{arnab2021_vivit}    & ViT\,-\,L & JFT\mbox{-}300M & \cmark & 128 & $3980{\times}1{\times}3$ & N/A & 83.5 & 94.3 \\
                ViViT \cite{arnab2021_vivit}       & ViT\,-\,H & JFT\mbox{-}300M & \cmark & 32  & $3981{\times}4{\times}3$ & N/A & 84.9 & 95.8 \\
                VIMPAC \cite{tan2021_vimpac}       & ViT\,-\,L & HowTo100M+DALLE & \xmark & 10 & N/A${\times}10{\times}3$ & 307 & 77.4 & N/A \\
                BEVT \cite{wang2022_bevt}          & Swin\,-\,B & IN\mbox{-}1K+DALLE & \xmark & 32 & $282{\times}4{\times}3$ & 88  & 80.6 & N/A \\
                MaskFeat$^{\uparrow 352}$ \cite{wei2022_maskfeat} & MViT\,-\,L & Kinetics\mbox{-}600 & \xmark & 40 & $3790{\times}4{\times}3$ & 218 & 87.0 & 97.4 \\
                ip\mbox{-}CSN \cite{tran2019_ipcsn} & ResNet152 & no external data & \xmark & 32 & $109{\times}10{\times}3$ & 33 & 77.8 & 92.8 \\
                SlowFast \cite{feichtenhofer2019_slowfast} & R101+NL & no external data & \xmark & $16{+}64$ & $234{\times}10{\times}3$ & 60 & 79.8 & 93.9 \\
                MViTv1 \cite{fan2021_mvit}         & MViTv1\,-\,B & no external data & \xmark & 32 & $170{\times}5{\times}1$ & 37 & 80.2 & 94.4 \\
                MaskFeat \cite{wei2022_maskfeat}   & MViT\,-\,L & no external data & \xmark & 16 & $377{\times}10{\times}1$ & 218 & 84.3 & 96.3 \\
                \midrule
                \textbf{VideoMAE} \cite{tong2022_videomae}   & ViT\,-\,S & no external data & \xmark & 16 & $57{\times}5{\times}3$   & 22  & 79.0 & 93.8 \\
                \textbf{VideoMAE} \cite{tong2022_videomae}   & ViT\,-\,B & no external data & \xmark & 16 & $180{\times}5{\times}3$  & 87  & 81.5 & 95.1 \\
                \textbf{VideoMAE} \cite{tong2022_videomae}   & ViT\,-\,L & no external data & \xmark & 16 & $597{\times}5{\times}3$  & 305 & 85.2 & 96.8 \\
                \textbf{VideoMAE} \cite{tong2022_videomae}   & ViT\,-\,H & no external data & \xmark & 16 & $1192{\times}5{\times}3$ & 633 & \textbf{86.6} & \textbf{97.1} \\
                \textbf{VideoMAE}$^{\uparrow 320}$ \cite{tong2022_videomae} & ViT\,-\,L & no external data & \xmark & 32 & $3958{\times}4{\times}3$ & 305 & 86.1 & 97.3 \\
                \textbf{VideoMAE}$^{\uparrow 320}$ \cite{tong2022_videomae} & ViT\,-\,H & no external data & \xmark & 32 & $7397{\times}4{\times}3$ & 633 & \textbf{87.4} & \textbf{97.6} \\
                \bottomrule
        \end{tabular}}
    \end{table}
        
    \newpage
    
    \subsubsection{Limitations and Future Work}
    \label{subsubsec_chapter24_videomae_limits}
    
    \paragraph{Observed constraints}
    \begin{itemize}
        \item \textbf{Masking ratio sensitivity.} There is a narrow sweet spot for tube masking: very high ratios are necessary to suppress temporal shortcuts, but pushing beyond $\approx 95\%$ removes too much context and harms learning (see Fig.~\ref{fig:chapter24_videomae_maskratio})..
        \item \textbf{Quadratic attention cost.} The encoder’s joint space--time self-attention still scales as $O(L_{\text{vis}}^2)$ in the number of \emph{visible} tokens. Although high masking keeps $L_{\text{vis}}$ small, increasing clip length, spatial resolution, or lowering the mask ratio raises compute and memory nonlinearly..
        \item \textbf{Domain shift and data alignment.} Representations benefit from in-domain pre-training. Transferring from an appearance-centric source (e.g., K400) to a motion-centric target (e.g., SSV2) is weaker than in-domain pre-train at equal budget, underscoring that data \emph{quality and match} can matter more than raw volume (see Fig.~\ref{fig:chapter24_videomae_dataeff})..
        \item \textbf{Pixel-space targets bias toward appearance.} Minimizing MSE on normalized pixels is simple and effective, but supervision is dominated by appearance reconstruction; motion cues are learned implicitly and may be underweighted for tasks that rely on long-range dynamics.
    \end{itemize}
    
    \paragraph{Promising directions}
    \begin{itemize}
        \item \textbf{Adaptive or content-aware masking.} Learn masks that allocate more visibility to motion-salient or semantically rich regions while more aggressively masking redundant background, keeping encoder cost similar but increasing task informativeness..
        \item \textbf{Richer targets and multi-task pretext signals.} Augment pixel reconstruction with auxiliary targets that emphasize dynamics (e.g., low-frequency components, flow-like proxies, or teacher features), to better balance appearance and motion without labels.
        \item \textbf{More scalable attention.} Combine VideoMAE with factorized, windowed, or linear-time attention, or with token pruning/merging, to extend temporal horizon and spatial resolution at similar compute.
        \item \textbf{Ratio curricula and schedule tuning.} Start from moderate ratios to stabilize optimization, then anneal toward $90$--$95\%$ as representations mature, preserving difficulty while avoiding early under-conditioning.
        \item \textbf{Stronger data curation for transfer.} Favor pre-training sets that better match the motion statistics of the target task, or mix sources to cover both appearance- and motion-centric regimes to improve cross-domain robustness.
    \end{itemize}
    
    \paragraph{Summary}
    VideoMAE shows that simple ingredients---\emph{tube} masking at very high ratios plus an asymmetric ViT encoder--decoder trained on masked-pixel reconstruction---yield data-efficient and transferable video features. The main practical caveats are the sensitivity of ultra-high masking, residual quadratic attention cost in the encoder, and dependence on data domain. Addressing these with adaptive masking, richer supervisory signals, and scalable attention is a clear path for future work \cite{tong2022_videomae}.
            
    \end{enrichment}
    
    \newpage
    
    \begin{enrichment}[VideoMAEv2: Dual Masking at Scale][subsection]
        \label{enr:subsec_chapter24_videomaev2}
            
            \paragraph{Scope and positioning}
            VideoMAEv2~\cite{wang2023_videomaev2} extends VideoMAE~\cite{tong2022_videomae} by addressing the main bottleneck in large-scale video masked autoencoding: the \emph{decoder}. In VideoMAE, the encoder is efficient because it only processes visible tokens, but the decoder must still handle a very long sequence that includes placeholders for all masked positions. At ViT-H/g scale and long clips, this dominates memory and FLOPs. 
            
            The key innovation is \emph{dual masking}: besides encoder tube masking, the decoder is also masked so it reconstructs only a subset of cubes. The loss is computed only on cubes that were invisible to the encoder, preserving the MAE principle while cutting decoder sequence length. This enables scaling up to ViT-g on million-scale video data.
            
            \subsubsection{Motivation}
            \label{subsubsec_chapter24_videomaev2_motivation}
            
            \paragraph{Why mask the decoder too}
            \begin{itemize}
                \item \textbf{Decoder becomes the bottleneck at scale.} Even though the encoder only processes $\approx (1{-}\rho^e)N$ tokens, the decoder in VideoMAE still receives $\approx N$ positions (including mask tokens). At large model and clip sizes, this dominates compute and memory.
                \item \textbf{Redundant supervision.} Videos contain strong spatial–temporal redundancy. Supervising a carefully selected subset of masked cubes is enough to learn strong representations.
                \item \textbf{Preventing leakage.} MAE’s principle requires loss only on encoder-invisible tokens. Dual masking preserves this by restricting supervision to $E\cap D$.
            \end{itemize}
            
            \begin{figure}[H] \centering \includegraphics[width=\textwidth]{Figures/Chapter_24/VideoMAEv2_overview.jpg} \caption{\textbf{VideoMAE with dual masking.} To improve the overall efficiency of computation and memory in video masked autoencoding, the authors mask the decoder as well and devise the dual masking strategy. Like the encoder, the method applies a masking map to the decoder and reconstructs only a subset of pixel cubes selected by \emph{running cell} masking. The final reconstruction loss is computed only for the invisible tokens dropped by the encoder.} \label{fig:chapter24_videomaev2_overview}
            \end{figure}
            
            \newpage
            
            \subsubsection{Method}
            \label{subsubsec_chapter24_videomaev2_method}
            
            \paragraph{Preliminaries and notation}
            A temporally subsampled clip $I \in \mathbb{R}^{T\times H\times W\times 3}$ is divided into spatiotemporal cubes of size $k\times P\times P$ (e.g., $2\times 16 \times 16$). Each cube is linearly embedded into $\mathbb{R}^D$, yielding
            \[
            N = \tfrac{T}{k}\cdot\tfrac{H}{P}\cdot\tfrac{W}{P}
            \]
            tokens. An encoder tube mask $M_e$ with ratio $\rho^e \in [0.9,0.95]$ is sampled on the spatial grid and broadcast over time. Let $V$ denote visible positions and $E$ the encoder-masked set. The encoder processes only visible tokens:
            \begin{equation}
                Z = \Phi_{\mathrm{enc}}(\{T_i\}_{i \in V}).
                \label{eq:chapter24_videomaev2_enc}
            \end{equation}
            
            \paragraph{Dual masking: decoder-side selection}
            A decoder mask $M_d$ with ratio $\rho^d$ selects which cubes to reconstruct. The default is \emph{running-cell masking}, which ensures spatiotemporal coverage without degenerate patterns (e.g., whole frames). Let $D$ be decoder-visible positions. The decoder input is:
            \begin{equation}
                U = [Z \cup \{M_i\}_{i \in D}], \qquad \hat I = \Phi_{\mathrm{dec}}(U).
                \label{eq:chapter24_videomaev2_decinput}
            \end{equation}
            
            \paragraph{Loss on encoder-invisible \& decoder-visible cubes}
            Reconstruction loss is applied only to tokens that were invisible to the encoder but selected for reconstruction by the decoder:
            \begin{equation}
                \mathcal{L}_{\mathrm{DM}} \;=\; \frac{1}{|E \cap D|}\sum_{i \in E \cap D} \|I_i - \hat I_i\|_2^2,
                \label{eq:chapter24_videomaev2_loss}
            \end{equation}
            where $E$ is the set of encoder-masked positions and $D$ the set of decoder-visible positions. The loss is restricted to $E\cap D$ to avoid information leakage.
            
            \paragraph{Running-cell masking for decoder supervision}
            \textit{Goal.} Make the decoder cheap but informative: at each iteration it reconstructs only a \emph{small, contiguous 3D block} of tokens (a \emph{cell}) rather than every masked token.
            
            \medskip
            \noindent\textbf{Setup (single knob = cell size).}
            After cubeization the token grid has shape
            \(
            (N_t,N_h,N_w)=\big(\tfrac{T}{k},\,\tfrac{H}{P},\,\tfrac{W}{P}\big)
            \)
            with indices $(t',x,y)$. Choose a cell size $(C_t,C_h,C_w)$.
            The decoder keep-rate (fraction of tokens it will process in a step) is simply the cell-to-grid volume ratio:
            \[
            1-\rho^d \;\approx\; \frac{|D_s|}{N}
            = \frac{C_t\,C_h\,C_w}{N_t\,N_h\,N_w},\qquad N=N_tN_hN_w.
            \]
            \emph{Example.} If $(N_t,N_h,N_w)=(8,14,14)$ then a cell $(4,7,7)$ yields $196/1568\approx12.5\%$ keep-rate; to target $\approx50\%$ use a larger cell, e.g., $(6,12,12)$ giving $864/1568\approx55\%$.
            
            \newpage 
            
            \noindent\textbf{Selection rule (what the decoder sees).}
            At training step $s$, pick a cell origin $(t_0,x_0,y_0)$ and define the decoder-visible set
            \[
            D_s=\big\{(t',x,y):\;
            t_0 \le t' < t_0{+}C_t,\;
            x_0 \le x < x_0{+}C_h,\;
            y_0 \le y < y_0{+}C_w\big\}.
            \]
            \emph{Placement.} \textbf{Random} (default): sample $(t_0,x_0,y_0)$ uniformly—each step hits a different region with high probability. \textbf{Strided} (alt.): move with strides $(S_t,S_h,S_w)$ and wrap for deterministic coverage.
            
            \medskip
            \noindent\textbf{What happens next (mechanics).}
            The decoder input consists of (i) encoder features from the few visible tokens and (ii) learned mask tokens \emph{only} at indices in $D_s$. The decoder predicts pixels \emph{only} for $D_s$, and the loss is computed strictly on the intersection with the encoder-masked set:
            \[
            \mathcal{L} \;=\; \frac{1}{|E\cap D_s|}\sum_{i\in E\cap D_s}\|I_i-\hat I_i\|_2^2,
            \]
            so tokens seen by the encoder ($V$) never contribute to the loss even if they lie in $D_s$.
            
            \medskip
            \noindent\textbf{Why this design works.}
            \begin{itemize}
                \item \textbf{Coherent supervision.} A contiguous 3D cell provides local space--time context, which is a stronger signal than isolated random tokens.
                \item \textbf{Even coverage over training.} Random (or strided) placement prevents target clustering and ensures every region is eventually supervised.
                \item \textbf{Predictable efficiency.} Decoder cost scales with $|D_s|$ (i.e., with $1-\rho^d$), giving a simple, explicit trade-off via $(C_t,C_h,C_w)$.
                \item \textbf{Leakage-free objective.} Restricting the loss to $E\cap D_s$ preserves the MAE principle and blocks copying from encoder-visible tokens.
            \end{itemize}
            
            \newpage 
            
            \paragraph{Algorithmic flow (pseudo code)}
            \begin{mintedbox}{python}
                # VideoMAEv2 pre-training step (schematic)
                # I: clip [T,H,W,3]; rho_e: encoder mask ratio; rho_d: decoder mask ratio
                # Phi_emb: cube embedding; Enc: ViT encoder; Dec: lightweight ViT decoder
                
                def step(I, rho_e=0.9, rho_d=0.5):
                    X = Phi_emb(cubeify(I))                 # tokens T_1..T_N
                    Ve, Ee = tube_mask_indices(N=X.shape[0], ratio=rho_e)  # visible / masked for encoder
                    Z = Enc(X[Ve])                          # encode only visible tokens
                    D = running_cell_mask_indices(N=X.shape[0], ratio=rho_d)  # decoder-kept positions
                    masked_tokens = learnable_mask_tokens(indexes=D)
                    U = concat(Z, masked_tokens)            # decoder sequence
                    I_hat = Dec(U)                          # predictions at positions in D
                    idx = intersect(Ee, D)                  # loss only on encoder-invisible & decoder-visible
                    return mse(I_hat[idx], targets(I)[idx]) / len(idx)
            \end{mintedbox}
            
            \subsubsection{Architecture and Implementation Details}
            \label{subsubsec_chapter24_videomaev2_arch}
            
            \paragraph{Backbones and decoder}
            \begin{itemize}
                \item \textbf{Encoders.} ViT-B/L/H and the billion-parameter ViT-g are used as encoders with joint space--time attention.
                \item \textbf{Decoder.} A lightweight ViT (e.g., $4$ blocks) with narrower width reconstructs pixel targets from $U$; masking the decoder reduces its token length to $(1{-}\rho^{d})N$.
            \end{itemize}
            
            \paragraph{Masking specifics}
            \begin{itemize}
                \item \textbf{Encoder masking ($\rho^{e}$).} High-ratio tube masking as in VideoMAE ($90$--$95\%$).
                \item \textbf{Decoder masking ($\rho^{d}$).} Running cell masking with default $\rho^{d}\!=\!0.5$ unless otherwise stated; alternatives (frame masking, random masking) are evaluated in ablations.
            \end{itemize}
            
            \paragraph{Data and schedules}
            \begin{itemize}
                \item \textbf{Pre-training corpora.}
                \begin{itemize}
                    \item \textbf{UnlabeledHybrid} (\underline{UH}, $\sim$1.35M clips): mixed, de-duplicated web video sources used \emph{without labels} for self-supervised pre-training.
                    \item \textbf{LabeledHybrid} (\underline{LH}, K710-aligned): same mixture but \emph{with labels} for optional progressive post-pre-training before downstream fine-tuning.
                    \item \textbf{IG-uncurated}: $\sim$1M Instagram videos without labels (used in MAE-ST baselines for scale comparison).
                \end{itemize}
                
                \item \textbf{Downstream datasets (names $\rightarrow$ shorthand).}
                \begin{itemize}
                    \item \textbf{Kinetics-400} $\rightarrow$ \underline{K400}: $\sim$240k YouTube clips, 400 actions; appearance-centric.
                    \item \textbf{Kinetics-600} $\rightarrow$ \underline{K600}: $\sim$480–500k clips, 600 actions (expanded Kinetics).
                    \item \textbf{Kinetics-700} $\rightarrow$ \underline{K700}: $\sim$650k clips, 700 actions.
                    \item \textbf{Kinetics-710} $\rightarrow$ \underline{K710}: curated 710-class labeled mix used for progressive post-pre-train in V2.
                    \item \textbf{Something-Something V2} $\rightarrow$ \underline{SSv2}: $\sim$169k clips of object-centric interactions; motion-centric.
                    \item \textbf{Something-Something V1} $\rightarrow$ \underline{SSv1}: earlier SSv2 release used in some SOTA tables.
                    \item \textbf{AVA v2.2} $\rightarrow$ \underline{AVA}: spatio-temporal action detection (1s annotations) on movie clips; report mAP.
                    \item \textbf{AVA-Kinetics} $\rightarrow$ \underline{AVA-K}: long-form detection benchmark combining AVA with Kinetics for training/testing.
                    \item \textbf{THUMOS14}: temporal action detection on untrimmed videos; report mAP at multiple IoU thresholds.
                    \item \textbf{FineAction}: temporal detection dataset of fine-grained actions; report mAP.
                \end{itemize}
                
                \item \textbf{Input \& sampling.}
                \begin{itemize}
                    \item \textbf{Clip shape:} typically $16\times 224^2$; temporal stride $\tau\in\{2,4\}$ during pre-train.
                    \item \textbf{Fine-tune sampling:} TSN-style sparse sampling on \underline{SSv2}; dense/multi-view on Kinetics (\underline{K400}/\underline{K600}/\underline{K700}/\underline{K710}).
                    \item \textbf{Inference views:} \underline{SSv2}: $2\times 3$ (temporal $\times$ spatial); Kinetics: $5\times 3$ (unless otherwise noted in SOTA tables).
                \end{itemize}
                
                \item \textbf{Optimization schedules.}
                \begin{itemize}
                    \item \textbf{Pre-train:} 1200 epochs on 64 GPUs for UH/LH; SSv2 ablations commonly at 800 epochs.
                    \item \textbf{Masking:} encoder tube masking $\rho^{e}\in[0.90,0.95]$; decoder running-cell masking with keep-rate $1-\rho^{d}\approx 0.25\sim 0.50$ (per ablation).
                    \item \textbf{Targets \& loss:} per-cube pixel normalization; MSE on $E\cap D$ (encoder-invisible \& decoder-visible).
                \end{itemize}
            \end{itemize}
            
            \subsubsection{Experiments and Ablation}
            \label{subsubsec_chapter24_videomaev2_experiments}
            
            \paragraph{Decoder masking strategies}
            \begin{table}[H]
                \centering
                \small
                \setlength{\tabcolsep}{6pt}
                \caption{Ablation study on decoder masking strategies (ViT-B, SSv2, $800$ epochs). ``None'' is encoder-only masking (original VideoMAE). The default VideoMAEv2 setting is shaded.}
                \label{tab:videomaev2_ablation_decoder}
                \begin{tabular}{lccc}
                    \toprule
                    \textbf{Decoder Masking} & $\rho^{d}$ & \textbf{Top-1} & \textbf{FLOPs} \\
                    \midrule
                    None            & 0\%  & \textbf{70.28} & 35.48G \\
                    Frame           & 50\% & 69.76 & 25.87G \\
                    Random          & 50\% & 64.87 & 25.87G \\
                    Running cell$^{1}$ & 50\% & 66.74 & 25.87G \\
                    Running cell$^{2}$ & 25\% & 70.22 & 31.63G \\
                    Running cell$^{2}$ & 50\% & 70.15 & 25.87G \\
                    Running cell$^{2}$ & 75\% & 70.01 & 21.06G \\
                    \bottomrule
                \end{tabular}\\[2pt]
                {\footnotesize $^{1}$Loss over all decoder outputs.\quad $^{2}$Loss over decoder outputs invisible to the encoder.}
            \end{table}
            
            \paragraph{Efficiency of dual masking}
            \begin{table}[H]
                \centering
                \small
                \setlength{\tabcolsep}{5pt}
                \caption{\textbf{Dual masking vs.\ encoder-only masking.} Computational cost, memory, and runtime (1200 epochs on 64 GPUs).}
                \label{tab:videomaev2_efficiency}
                \resizebox{\linewidth}{!}{
                    \begin{tabular}{l|c|c|c|c|c|c|c}
                        \hline
                        Masking & Backbone & Pre-training dataset & FLOPs & Mems & Time & Speedup & Top-1 \\
                        \hline
                        Encoder masking & ViT-B & Sth-Sth V2 & 35.48G & 631M & 28.4h & - & 70.28 \\
                        Dual masking    & ViT-B & Sth-Sth V2 & 25.87G & 328M & 15.9h & 1.79$\times$ & 70.15 \\
                        Encoder masking & ViT-g & UnlabeledHybrid & 263.93G & 1753M & 356h$^{\dagger}$ & - & - \\
                        Dual masking    & ViT-g & UnlabeledHybrid & 241.61G & 1050M & 241h & 1.48$\times$ & 77.00 \\
                        \hline
                \end{tabular}}
                {\footnotesize $^{\dagger}$Estimated from 5-epoch runs.}
            \end{table}
            
            \paragraph{Kinetics-400}
            \begin{table}[H]
                \centering
                \small
                \setlength{\tabcolsep}{5pt}
                \caption{\textbf{Results on Kinetics-400.} Multi-view ($5{\times}3$) accuracy; single-view in brackets. Input $16{\times}224^2$, stride $\tau\!=\!4$.}
                \label{tab:videomaev2_k400}
                \resizebox{\linewidth}{!}{
                    \begin{tabular}{l|c|c|c|c|c|c}
                        \hline
                        Method & Pre-train data & Data size & Epoch & ViT-B & ViT-L & ViT-H / ViT-g \\
                        \hline
                        MAE-ST~\cite{feichtenhofer2022_mae_st} & Kinetics400 & 0.24M & 1600 & 81.3 & 84.8 & 85.1 \\
                        MAE-ST~\cite{feichtenhofer2022_mae_st} & IG-uncurated & 1M & 1600 & - & 84.4 & - \\
                        VideoMAE V1~\cite{tong2022_videomae} & Kinetics400 & 0.24M & 1600 & 81.5 & 85.2 & 86.6 \\
                        VideoMAE V2 & UnlabeledHybrid & 1.35M & 1200 & \textbf{81.5} (77.0) & \textbf{85.4} (81.3) & \textbf{86.9} / \textbf{87.2} (83.2 / 83.9) \\
                        \hline
                \end{tabular}}
            \end{table}
            
            \paragraph{Something-Something V2}
            \begin{table}[H]
                \centering
                \small
                \setlength{\tabcolsep}{5pt}
                \caption{\textbf{Results on Something-Something V2.} Multi-view ($2{\times}3$) accuracy; single-view in brackets. Input $16{\times}224^2$, stride $\tau\!=\!2$.}
                \label{tab:videomaev2_ssv2}
                \resizebox{\linewidth}{!}{
                    \begin{tabular}{l|c|c|c|c|c|c}
                        \hline
                        Method & Pre-train data & Data size & Epoch & ViT-B & ViT-L & ViT-H / ViT-g \\
                        \hline
                        MAE-ST~\cite{feichtenhofer2022_mae_st} & Kinetics400 & 0.24M & 1600 & - & 72.1 & 74.1 \\
                        MAE-ST~\cite{feichtenhofer2022_mae_st} & Kinetics700 & 0.55M & 1600 & - & 73.6 & 75.5 \\
                        VideoMAE V1~\cite{tong2022_videomae} & Sth-Sth V2 & 0.17M & 2400 & 70.8 & 74.3 & 74.8 \\
                        VideoMAE V2 & UnlabeledHybrid & 1.35M & 1200 & \textbf{71.2} (69.5) & \textbf{75.7} (74.0) & \textbf{76.8} / \textbf{77.0} (75.5 / 75.7) \\
                        \hline
                \end{tabular}}
            \end{table}
            
            \paragraph{Progressive pre-training (K710)}
            \begin{table}[H]
                \centering
                \small
                \setlength{\tabcolsep}{8pt}
                \caption{\textbf{Study on progressive pre-training.} Kinetics-400 fine-tuning with multi-view ($5{\times}3$); single-view in brackets.}
                \label{tab:videomaev2_progressive}
                \resizebox{0.65\linewidth}{!}{
                    \begin{tabular}{l|c|c}
                        \hline
                        Method & ViT-H & ViT-g \\
                        \hline
                        VideoMAE V2 (no K710) & 86.9 (83.2) & 87.2 (83.9) \\
                        VideoMAE V2 (+K710) & \textbf{88.6} (85.0) & \textbf{88.5} (85.6) \\
                        \hline
                \end{tabular}}
            \end{table}
            
            \paragraph{State of the art (selected benchmarks)}
            % Compact table settings (scoped)
            \begingroup
            \setlength{\tabcolsep}{3pt}
            \renewcommand{\arraystretch}{0.9}
            
            \begin{table}[H]
                \centering
                \scriptsize
                \caption{(a) Kinetics\,400 — Top-1/Top-5 accuracy, views, and TFLOPs.}
                \begin{tabular}{lcccc}
                    \toprule
                    Method & Top 1 & Top 5 & Views & TFLOPs \\
                    \midrule
                    I3D NL~\cite{wang2018_nonlocal_nn} & 77.7 & 93.3 & $10 \times 3$ & 10.77 \\
                    TDN~\cite{wang2021_tdn} & 79.4 & 94.4 & $10 \times 3$ & 5.94 \\
                    SlowFast R101\textendash NL~\cite{feichtenhofer2019_slowfast} & 79.8 & 93.9 & $10 \times 3$ & 7.02 \\
                    TimeSformer\textendash L~\cite{bertasius2021_timesformer} & 80.7 & 94.7 & $1 \times 3$ & 7.14 \\
                    MTV\textendash B ($320^2$)~\cite{yan2022_mtv} & 82.4 & 95.2 & $4 \times 3$ & 11.16 \\
                    Video Swin\textendash L ($384^2$)~\cite{liu2022_videoswin} & 84.9 & 96.7 & $10 \times 5$ & 105.35 \\
                    ViViT\textendash L FE~\cite{arnab2021_vivit} & 81.7 & 93.8 & $1 \times 3$ & 11.94 \\
                    MViTv2\textendash L ($312^2$)~\cite{li2021_improved_mvit} & 86.1 & 97.0 & $40 \times 3$ & 42.42 \\
                    MaskFeat~\cite{wei2022_maskfeat} & 87.0 & 97.4 & $4 \times 3$ & 45.48 \\
                    MAE\textendash ST~\cite{feichtenhofer2022_mae_st} & 86.8 & 97.2 & $4 \times 3$ & 25.05 \\
                    VideoMAE~\cite{tong2022_videomae} & 86.6 & 97.1 & $5 \times 3$ & 17.88 \\
                    \textbf{VideoMAE V2\textendash H}~\cite{wang2023_videomaev2} & \textbf{88.6} & \textbf{97.9} & $5 \times 3$ & 17.88 \\
                    \textbf{VideoMAE V2\textendash g}~\cite{wang2023_videomaev2} & \textbf{88.5} & \textbf{98.1} & $5 \times 3$ & 38.16 \\
                    \textbf{VideoMAE V2\textendash g} ($64 \times 266^2$)~\cite{wang2023_videomaev2} & \textbf{90.0} & \textbf{98.4} & $2 \times 3$ & 160.30 \\
                    \bottomrule
                \end{tabular}
            \end{table}
            
            \begin{table}[H]
                \centering
                \scriptsize
                \caption{(b) Kinetics\,600 — Top-1/Top-5 accuracy, views, and TFLOPs.}
                \begin{tabular}{lcccc}
                    \toprule
                    Method & Top 1 & Top 5 & Views & TFLOPs \\
                    \midrule
                    SlowFast R101\textendash NL~\cite{feichtenhofer2019_slowfast} & 81.8 & 95.1 & $10 \times 3$ & 7.02 \\
                    TimeSformer\textendash L~\cite{bertasius2021_timesformer} & 82.2 & 95.6 & $1 \times 3$ & 7.14 \\
                    MTV\textendash B ($320^2$)~\cite{yan2022_mtv} & 84.0 & 96.2 & $4 \times 3$ & 11.16 \\
                    ViViT\textendash L FE~\cite{arnab2021_vivit} & 82.9 & 94.6 & $1 \times 3$ & 11.94 \\
                    MViTv2\textendash L ($352^2$)~\cite{li2021_improved_mvit} & 87.9 & 97.9 & $40 \times 3$ & 45.48 \\
                    MaskFeat~\cite{wei2022_maskfeat} & 86.4 & 97.4 & $1 \times 10$ & 3.77 \\
                    \textbf{VideoMAE V2\textendash H}~\cite{wang2023_videomaev2} & \textbf{88.3} & \textbf{98.1} & $5 \times 3$ & 17.88 \\
                    \textbf{VideoMAE V2\textendash g}~\cite{wang2023_videomaev2} & \textbf{88.8} & \textbf{98.2} & $5 \times 3$ & 38.16 \\
                    \textbf{VideoMAE V2\textendash g} ($64 \times 266^2$)~\cite{wang2023_videomaev2} & \textbf{89.9} & \textbf{98.5} & $2 \times 3$ & 160.30 \\
                    \bottomrule
                \end{tabular}
            \end{table}
            
            \begin{table}[H]
                \centering
                \scriptsize
                \caption{(c) Something-Something V2 — Top-1/Top-5 accuracy.}
                \begin{tabular}{lcc}
                    \toprule
                    Method & Top 1 & Top 5 \\
                    \midrule
                    SlowFast~\cite{feichtenhofer2019_slowfast} & 63.1 & 87.6 \\
                    TEINet~\cite{li2021_teinet} & 66.5 & -- \\
                    TEA~\cite{li2020_tea} & 65.1 & 89.9 \\
                    TDN~\cite{wang2021_tdn} & 69.6 & 92.2 \\
                    TimeSformer\textendash L~\cite{bertasius2021_timesformer} & 62.4 & -- \\
                    MFormer\textendash HR~\cite{patrick2021_motionformer} & 68.1 & 91.2 \\
                    ViViT\textendash L FE~\cite{arnab2021_vivit} & 65.9 & 89.9 \\
                    Video Swin\textendash B~\cite{liu2022_videoswin} & 69.6 & 92.7 \\
                    MViTv2\textendash B~\cite{li2021_improved_mvit} & 72.1 & 93.4 \\
                    MTV\textendash B~\cite{yan2022_mtv} & 67.6 & 90.1 \\
                    BEVT~\cite{wang2022_bevt} & 70.6 & -- \\
                    VIMPAC~\cite{tan2021_vimpac} & 68.1 & -- \\
                    UniFormer~\cite{li2022_uniformer} & 71.2 & 92.8 \\
                    MaskFeat~\cite{wei2022_maskfeat} & 75.0 & 95.0 \\
                    MAE\textendash ST~\cite{feichtenhofer2022_mae_st} & 75.5 & 95.0 \\
                    VideoMAE~\cite{tong2022_videomae} & 75.4 & 95.2 \\
                    \textbf{VideoMAE V2\textendash H}~\cite{wang2023_videomaev2} & \textbf{76.8} & \textbf{95.8} \\
                    \textbf{VideoMAE V2\textendash g}~\cite{wang2023_videomaev2} & \textbf{77.0} & \textbf{95.9} \\
                    \bottomrule
                \end{tabular}
            \end{table}
            
            \begin{table}[H]
                \centering
                \scriptsize
                \caption{(d) Something-Something V1 — Top-1/Top-5 accuracy.}
                \begin{tabular}{lcc}
                    \toprule
                    Method & Top 1 & Top 5 \\
                    \midrule
                    I3D~\cite{carreira2017_i3d} & 41.6 & 72.2 \\
                    NL I3D+GCN~\cite{wang2018_nonlocal_nn,wang2018_videograph} & 46.1 & 76.8 \\
                    TSM~\cite{lin2019_tsm} & 49.7 & 78.5 \\
                    V4D~\cite{yue2020_v4d} & 50.4 & -- \\
                    TANet~\cite{li2020_tanet} & 50.6 & 79.3 \\
                    TEINet~\cite{li2021_teinet} & 52.5 & -- \\
                    TEA~\cite{li2020_tea} & 51.9 & 80.3 \\
                    CorrNet~\cite{wang2020_corrnet} & 53.3 & -- \\
                    GSM & 55.2 & -- \\
                    TDN~\cite{wang2021_tdn} & 56.8 & 84.1 \\
                    UniFormer~\cite{li2022_uniformer} & 61.0 & 87.6 \\
                    \textbf{VideoMAE V2\textendash H}~\cite{wang2023_videomaev2} & \textbf{66.6} & \textbf{90.8} \\
                    \textbf{VideoMAE V2\textendash g}~\cite{wang2023_videomaev2} & \textbf{68.7} & \textbf{91.9} \\
                    \bottomrule
                \end{tabular}
            \end{table}
            
            % Make the last four tables slightly smaller to fit on the same page
            \setlength{\tabcolsep}{2pt}
            \renewcommand{\arraystretch}{0.85}
            
            \begin{table}[H]
                \centering
                \footnotesize
                \caption{(e) AVA v2.2 — mAP with and without long feature.}
                \begin{tabular}{lcc}
                    \toprule
                    Method & Long Feature & mAP \\
                    \midrule
                    SlowFast~\cite{feichtenhofer2019_slowfast} & $\times$ & 29.0 \\
                    TubeR~\cite{zhao2022_tuber} & \checkmark & 33.4 \\
                    MaskFeat~\cite{wei2022_maskfeat} & $\times$ & 38.8 \\
                    MAE\textendash ST~\cite{feichtenhofer2022_mae_st} & $\times$ & 39.0 \\
                    VideoMAE~\cite{tong2022_videomae} & $\times$ & 39.5 \\
                    \textbf{VideoMAE V2}~\cite{wang2023_videomaev2} & $\times$ & \textbf{42.6} \\
                    \bottomrule
                \end{tabular}
            \end{table}
            
            \begin{table}[H]
                \centering
                \footnotesize
                \caption{(f) AVA\textendash Kinetics — ensembled mAP.}
                \begin{tabular}{lcc}
                    \toprule
                    Method & Ensembled & mAP \\
                    \midrule
                    AIA++~\cite{wu2022_aiapp} & \checkmark & 29.0 \\
                    MSF~\cite{wu2020_msf} & \checkmark & 33.4 \\
                    ACAR~\cite{pan2021_acar} & \checkmark & 40.5 \\
                    \textbf{VideoMAE V2}~\cite{wang2023_videomaev2} & $\times$ & \textbf{43.9} \\
                    \bottomrule
                \end{tabular}
            \end{table}
            
            \begin{table}[H]
                \centering
                \footnotesize
                \caption{(g) THUMOS14 — temporal action detection mAP.}
                \begin{tabular}{lcc}
                    \toprule
                    Method & Optical Flow & mAP \\
                    \midrule
                    RTD\textendash Net~\cite{xu2020_rtdnet} & \checkmark & 43.6 \\
                    DaoTAD~\cite{zhu2021_daotad} & $\times$ & 50.0 \\
                    AFSD~\cite{lin2021_salient_boundary} & \checkmark & 52.0 \\
                    DCAN~\cite{zeng2021_dcan} & \checkmark & 52.3 \\
                    TadTR~\cite{liu2022_tadtr} & \checkmark & 54.2 \\
                    TALLFormer~\cite{zhang2022_tallformer} & $\times$ & 59.2 \\
                    BasicTAD~\cite{bai2023_basictad} & $\times$ & 59.6 \\
                    ActionFormer~\cite{zhang2022_actionformer} & \checkmark & 66.8 \\
                    \textbf{VideoMAE V2}~\cite{wang2023_videomaev2} & $\times$ & \textbf{69.6} \\
                    \bottomrule
                \end{tabular}
            \end{table}
            
            \begin{table}[H]
                \centering
                \footnotesize
                \caption{(h) FineAction — temporal action detection mAP.}
                \begin{tabular}{lcc}
                    \toprule
                    Method & Optical Flow & mAP \\
                    \midrule
                    BMN~\cite{lin2019_bmn} & \checkmark & 9.25 \\
                    G\textendash TAD~\cite{xu2020_rtdnet} & \checkmark & 9.06 \\
                    BasicTAD~\cite{bai2023_basictad} & $\times$ & 12.2 \\
                    ActionFormer~\cite{zhang2022_actionformer} & $\times$ & 13.2 \\
                    \textbf{VideoMAE V2}~\cite{wang2023_videomaev2} & $\times$ & \textbf{18.2} \\
                    \bottomrule
                \end{tabular}
            \end{table}
            
            \endgroup
            
            \subsubsection{Limitations and Future Work}
            \label{subsubsec_chapter24_videomaev2_limits}
            
            \paragraph{Observed constraints}
            \begin{itemize}
                \item \textbf{Pixel supervision bottleneck.} Reconstruction remains anchored to low-level RGB fidelity. While effective for generic features, it underemphasizes semantic abstraction and long-range motion cues—exactly where large-scale video understanding needs stronger supervision.
                \item \textbf{Decoder still tied to reconstruction.} Even with dual masking, the decoder only learns to inpaint pixels. This constrains learning to local texture statistics rather than global semantics.
                \item \textbf{Scaling trade-offs.} Very large encoders (ViT-H, ViT-g) show diminishing gains on recognition benchmarks, suggesting that simply scaling model size without enriching the training signal plateaus.
                \item \textbf{Domain gaps.} Hybrid pretraining datasets balance appearance and motion imperfectly. Representations trained purely on RGB inputs may not capture compositional or multimodal cues needed in downstream tasks.
            \end{itemize}
            
            \paragraph{Future directions (path toward distillation and beyond)}
            \begin{itemize}
                \item \textbf{From pixels to features (MVD).} A natural next step is to replace pixel-level regression with \emph{feature-level targets}, as in Masked Video Distillation~\cite{wang2023_mvd}. Teacher encoders (e.g., image- or video-pretrained transformers) provide richer supervisory signals on masked regions, injecting semantics and motion awareness absent from raw RGB.
                \item \textbf{Dynamic decoder supervision.} Beyond fixed cells, learned policies for decoder token selection or adaptive sparsification can focus computation on informative spatio-temporal regions, preserving efficiency while scaling to longer clips.
                \item \textbf{Multi-granular objectives.} Combining reconstruction with motion-sensitive or perceptual losses could better capture dynamics, addressing VideoMAEv2’s limitation to mostly static appearance cues.
                \item \textbf{Cross-modal grounding.} Incorporating audio or text alignment, as explored in later video–language pretraining work, may reduce ambiguity and enable open-vocabulary recognition and retrieval.
                \item \textbf{Stress-testing benchmarks.} Moving beyond short-clip classification toward long-form reasoning, dense temporal localization, and open-vocabulary tasks will better expose the strengths and weaknesses of masked video pretraining methods.
            \end{itemize}
        
    \end{enrichment}
    
    \newpage
    
    \begin{enrichment}[MVD: Masked Video Distillation][subsection]
        \label{enr:subsec_chapter24_mvd}
            
            \paragraph{Scope and positioning}
            \label{par:chapter24_mvd_scope}
            \textbf{Background.} \emph{Masked Image Modeling (MIM)} and \emph{Masked Video Modeling (MVM)} are self-supervised pretraining paradigms that hide a large subset of patches and train a model to reconstruct the hidden content. For videos, VideoMAE \cite{tong2022_videomae} applies very high \emph{tube} masking (typically $90$--$95\%$) and asks a ViT to reconstruct \emph{raw pixels} in the masked spatio-temporal cubes using a lightweight decoder, yielding strong baselines but still supervising at the \emph{pixel level}. \textbf{MVD} \cite{wang2023_mvd} rethinks the reconstruction \emph{target}: instead of pixels, the student predicts \emph{high-level features} produced by frozen, pretrained \emph{teacher} encoders. Two complementary teachers are used: an \textbf{image teacher} (MIM-pretrained on images; strong spatial semantics) and a \textbf{video teacher} (MVM-pretrained on videos; motion-aware spatio-temporal semantics). The student video encoder sees only the \emph{visible} tokens of a tube-masked clip and, through shallow decoders with Smooth-$\ell_1$ regression, learns to reconstruct the teacher’s features at the \emph{masked} positions. Empirically, this \emph{masked feature modeling} yields consistent gains over pixel reconstruction (e.g., VideoMAE) on recognition and spatio-temporal detection benchmarks \cite{wang2023_mvd}.
            
            \begin{figure}[H]
                \centering
                \includegraphics[width=0.9\textwidth]{Figures/Chapter_24/MVD_overview.jpg}
                \caption{Overview of MVD \cite{wang2023_mvd}. A student video encoder observes only visible tokens from a tube-masked clip and is trained to reconstruct masked \emph{teacher features} with two shallow decoders: one targets an image teacher’s spatial features and the other a video teacher’s spatio-temporal features.}
                \label{fig:chapter24_mvd_overview}
            \end{figure}
            
            \subsubsection{Motivation}
            \label{subsubsec:chapter24_mvd_motivation}
            
            \paragraph{Limits of pixel-level MVM (VideoMAE).}
            Pixel reconstruction under MVM is affected by video \emph{temporal redundancy}: adjacent frames are highly similar, so a model can fill in masked pixels by copying or interpolating from nearby context without forming strong high-level abstractions. Even with VideoMAE’s very high masking ratio and tube masking, the supervision remains \emph{low level} and noisy, which can encourage shortcut solutions and yield features that transfer suboptimally on action-centric tasks \cite{tong2022_videomae,wang2023_mvd}.
            
            \newpage
            
            \paragraph{From pixels to features: cleaner targets and inductive bias.}
            MVD \cite{wang2023_mvd} replaces RGB regression with \emph{feature regression} against targets from powerful self-supervised teachers. High-level targets suppress nuisance variation (e.g., lighting, small pixel noise) and encode semantics that downstream tasks care about, providing a \emph{cleaner learning signal} and a better inductive bias than raw pixels. Practically, the student predicts masked-patch features produced by frozen teachers while only encoding the visible tokens, preserving the compute advantages of masked modeling.
            
            \paragraph{Why two teachers: complementary spatial and temporal cues.}
            Image teachers (MIM-pretrained) specialize in \emph{spatial} appearance and yield features that are highly similar across neighboring frames; video teachers (MVM-pretrained) encode \emph{temporal} dynamics and produce frame features whose similarity decays with temporal distance. MVD leverages this complementarity through \emph{spatial–temporal co-teaching}: two independent decoders regress to the image-teacher and video-teacher targets, respectively, so the shared student is simultaneously pressured to preserve strong spatial semantics and temporal sensitivity. This design helps a single student excel on both appearance-biased datasets (e.g., Kinetics-400) and motion-centric datasets (e.g., Something-Something V2), explaining the observed gains over VideoMAE across settings \cite{wang2023_mvd}.
            
            \begin{figure}[H]
                \centering
                \includegraphics[width=0.8\textwidth]{Figures/Chapter_24/MVD_feature_similarity.jpg}
                \caption{Teacher feature similarity across frames (cosine). Image-teacher features are highly similar across time, indicating spatial dominance; video-teacher features decorrelate with temporal distance, indicating motion sensitivity \cite{wang2023_mvd}.}
                \label{fig:chapter24_mvd_similarity}
            \end{figure} 
            
            \newpage
            
            \subsubsection{Method}
            \label{subsubsec:chapter24_mvd_method}
            
            \paragraph{Preliminaries: masked feature modeling}
            \label{par:chapter24_mvd_mfm}
            Let $X_{\text{vid}}\in\mathbb{R}^{T\times H\times W\times 3}$ be a video, partitioned into non-overlapping spatio-temporal patches (tubelets). After linear patch embedding, a subset $\mathcal{M}$ of tokens is masked (tube masking) and dropped from the encoder input; the visible tokens $X_{\text{vis}}$ are encoded by a student transformer $f$. A shallow transformer decoder $g$ receives $\operatorname{concat}(f(X_{\text{vis}}),T_m)$, where $T_m$ are learnable mask tokens, and predicts outputs $Y$ for all token positions:
            \begin{equation}
                Y \;=\; g\!\big(\operatorname{concat}(f(X_{\text{vis}}), T_m)\big)
                \label{eq:chapter24_mvd_forward}
            \end{equation}
            Given a target feature generator $h$ that maps each masked patch $X^{(p)}$ to a target feature $h(X^{(p)})$, the masked feature modeling objective is
            \begin{equation}
                \mathcal{L}_{\text{mfm}}(h) \;=\; \frac{1}{|\mathcal{M}|}\sum_{p\in\mathcal{M}} D\!\left(Y^{(p)},\, h\!\left(X^{(p)}\right)\right)
                \label{eq:chapter24_mvd_mfm}
            \end{equation}
            where $D$ is a distance measure; MVD adopts Smooth-$\ell_1$ (Huber) loss \cite{wang2023_mvd}.
            
            \paragraph{Teacher targets}
            \label{par:chapter24_mvd_targets}
            MVD instantiates $h$ as \emph{frozen} self-supervised teacher encoders:
            \begin{itemize}
                \item \textbf{Spatial (image) targets.} An image-teacher encoder $h_{\text{img}}$ pretrained by masked image modeling (e.g., MAE on IN1K) encodes each frame independently to provide appearance-focused features.
                \item \textbf{Spatio-temporal (video) targets.} A video-teacher encoder $h_{\text{vid}}$ pretrained by masked video modeling (e.g., VideoMAE on K400) encodes clips to provide motion-aware features.
            \end{itemize}
            A $2\times16\times16$ 3D patch for the video student corresponds to two $16\times16$ 2D patches for the image teacher; MVD predicts the front slice’s spatial target to reduce the prediction head size \cite{wang2023_mvd}.
            
            \paragraph{Spatial–temporal co-teaching}
            \label{par:chapter24_mvd_coteach}
            To fuse complementary supervision, MVD attaches two decoders $g_{\text{img}}$ and $g_{\text{vid}}$ (same architecture, independent parameters) to the shared student features $f(X_{\text{vis}})$:
            \begin{equation}
                \mathcal{L}_{\text{MVD}} \;=\; \lambda_1\,\mathcal{L}_{\text{mfm}}(h_{\text{img}}) \;+\; \lambda_2\,\mathcal{L}_{\text{mfm}}(h_{\text{vid}})
                \label{eq:chapter24_mvd_total}
            \end{equation}
            with scalars $\lambda_1,\lambda_2$ balancing the two teachers. This objective compels the student to reconstruct masked tokens so that \emph{both} image-like and video-like semantics are preserved, strengthening spatial discrimination and temporal sensitivity simultaneously \cite{wang2023_mvd}.
            
            \newpage
            
            \paragraph{Algorithmic view}
            \label{par:chapter24_mvd_algo}
            The official pseudocode (PyTorch style) is reproduced verbatim (line breaks adapted) in \ref{alg:chapter24_mvd_algo}. Notation: $f$ student, $g_{\text{img}}/g_{\text{vid}}$ decoders, $T_m$ mask tokens, $h_{\text{img}}/h_{\text{vid}}$ frozen teachers, $m$ binary mask.
            
            \begin{mintedbox}{python}
                # Algorithm 1 Pseudocode of MVD in PyTorch style (from \cite{wang2023_mvd})
                # f: student encoder
                # g_img: decoder for reconstructing spatial features
                # g_vid: decoder for reconstructing spatial-temporal features
                # t_m: learnable mask tokens
                # h_img: image teacher model
                # h_vid: video teacher model
                for x, m in loader:  # x: video data, m: mask
                    x_pe = patch_emb(x)               # patch embedding of input
                    x_vis = mask_select(x_pe, 1 - m)  # masking tokens
                    q_vis = f(x_vis)                  # visible local patch features
                    # reconstruction of target features
                    p_img = g_img(concat(q_vis, t_m))
                    p_vid = g_vid(concat(q_vis, t_m))
                    # compute target features with teacher models
                    k_img = h_img(x)  # target spatial features
                    k_vid = h_vid(x)  # target spatial-temporal features
                    # compute reconstruction loss
                    loss_img = smooth_L1_loss(p_img * m, k_img * m)
                    loss_vid = smooth_L1_loss(p_vid * m, k_vid * m)
                    loss = lambda_1 * loss_img + lambda_2 * loss_vid
                    loss.backward()
                    optimizer.step()  # optimizer update
            \end{mintedbox}
            \label{alg:chapter24_mvd_algo}
            
            \paragraph{Intuition and failure-mode mitigation}
            \label{par:chapter24_mvd_intuition}
            \begin{itemize}
                \item \textbf{Richer supervision than pixels.} High-level targets abstract away nuisance low-level variability, biasing the student toward semantics that transfer better across datasets and tasks.
                \item \textbf{Masking as structured context removal.} Tube masking removes entire spatio-temporal tubes, forcing the student to hallucinate both \emph{appearance} and \emph{motion} content consistent with teacher features rather than raw RGB.
                \item \textbf{Decoupled decoders avoid interference.} Separate heads let each teacher specialize its prediction space without compromising the other, while gradients meet only in the shared student.
            \end{itemize}
            
            \subsubsection{Architecture and implementation details}
            \label{subsubsec:chapter24_mvd_arch}
            
            \paragraph{Backbone and tokenization}
            A vanilla ViT encoder (ViT-S/B/L/H) serves as $f$. 3D patch embedding with size $2\times16\times16$ produces $T/2\times H/16\times W/16$ tokens. During pretraining, a high masking ratio (e.g., $90\%$) with \emph{tube masking} is applied; only visible tokens are encoded \cite{wang2023_mvd,tong2022_videomae}.
            
            \paragraph{Attention}
            Joint spatio-temporal self-attention is applied within each encoder block over the visible token sequence. Learned mask tokens $T_m$ are concatenated with $f(X_{\text{vis}})$ before each decoder.
            
            \paragraph{Decoders and objectives}
            Two shallow transformer decoders (a few layers) plus linear heads predict teacher features at masked positions. Smooth-$\ell_1$ regression is used for both branches; the loss is computed \emph{only} on masked tokens via elementwise masking, as in \eqref{eq:chapter24_mvd_mfm}–\eqref{eq:chapter24_mvd_total} \cite{wang2023_mvd}.
            
            \paragraph{Pretraining schedules}
            Teachers: MAE image-teacher on IN1K (e.g., 1600 epochs), VideoMAE video-teacher on K400 (e.g., 1600 epochs). Student: distilled on K400 for 400 epochs by default (800 in some settings), AdamW optimizer, clip length $T{=}16$ for pretrain and finetune \cite{wang2023_mvd}.
            
            \subsubsection{Experiments and ablation}
            \label{subsubsec:chapter24_mvd_experiments}
            
            \paragraph{Main results and efficiency}
            On SSv2, MVD dominates the accuracy–compute frontier relative to supervised and self-supervised peers (see below figure). Relative to VideoMAE \cite{tong2022_videomae}, MVD delivers consistent gains across model scales with substantially fewer pretraining epochs \cite{wang2023_mvd}.
            
            \begin{figure}[H]
                \centering
                \includegraphics[width=0.65\textwidth]{Figures/Chapter_24/MVD_flops_vs_top1acc.jpg}
                \caption{SSv2 accuracy versus GFLOPs per video for supervised and self-supervised models. MVD (red stars) attains higher accuracy at comparable or lower cost across scales \cite{wang2023_mvd}.}
                \label{fig:chpapter24_chapter24_mvd_flops}
            \end{figure}
            
            \paragraph{Gains over VideoMAE across scales}
            \label{par:chapter24_mvd_vs_videomae}
            \begin{table}[H]
                \centering
                \small
                \setlength{\tabcolsep}{6pt}
                \caption{MVD vs. VideoMAE across student/teacher scales on K400 and SSv2 \cite{wang2023_mvd,tong2022_videomae}.}
                \label{tab:chapter24_mvd_vs_videomae}
                \begin{tabular}{lccccc}
                    \toprule
                    \textbf{Student} & \textbf{Teacher} & \textbf{K400 (VideoMAE)} & \textbf{K400 (MVD)} & \textbf{SSv2 (VideoMAE)} & \textbf{SSv2 (MVD)} \\
                    \midrule
                    ViT-S & ViT-B & 79.0 & \textbf{80.6} & 66.4 & \textbf{70.7} \\
                    ViT-S & ViT-L & 79.0 & \textbf{81.0} & 66.4 & \textbf{70.9} \\
                    ViT-B & ViT-B & 81.5 & \textbf{82.7} & 69.7 & \textbf{72.5} \\
                    ViT-B & ViT-L & 81.5 & \textbf{83.4} & 69.7 & \textbf{73.7} \\
                    ViT-L & ViT-L & 85.2 & \textbf{86.0} & 74.0 & \textbf{76.1} \\
                    \bottomrule
                \end{tabular}
            \end{table}
            
            \paragraph{Co-teaching vs single teacher}
            \label{par:chapter24_mvd_coteach_table}
            \begin{table}[H]
                \centering
                \small
                \setlength{\tabcolsep}{7pt}
                \caption{Spatial–temporal co-teaching outperforms single-teacher distillation on K400 and SSv2 \cite{wang2023_mvd}.}
                \label{tab:chapter24_mvd_coteaching}
                \begin{tabular}{lcccc}
                    \toprule
                    \textbf{Student} & \textbf{Image} & \textbf{Video} & \textbf{K400 top-1 (\%)} & \textbf{SSv2 top-1 (\%)} \\
                    \midrule
                    ViT-S & \cmark & \xmark & 80.4 & 69.4 \\
                    ViT-S & \xmark & \cmark & 80.1 & 70.0 \\
                    ViT-S & \cmark & \cmark & \textbf{80.6} & \textbf{70.7} \\
                    \midrule
                    ViT-B & \cmark & \xmark & 82.3 & 71.4 \\
                    ViT-B & \xmark & \cmark & 82.1 & 71.8 \\
                    ViT-B & \cmark & \cmark & \textbf{82.7} & \textbf{72.5} \\
                    \bottomrule
                \end{tabular}
            \end{table}
            
            \paragraph{Gains over VideoMAE across scales}
            \label{par:chapter24_mvd_vs_videomae_gains}
            \begin{table}[H]
                \centering
                \small
                \setlength{\tabcolsep}{6pt}
                \caption{MVD vs. VideoMAE across student/teacher scales on K400 and SSv2 \cite{wang2023_mvd,tong2022_videomae}.}
                \label{tab:chapter24_mvd_vs_videomae_scales}
                \begin{tabular}{lccccc}
                    \toprule
                    \textbf{Student} & \textbf{Teacher} & \textbf{K400 (VideoMAE)} & \textbf{K400 (MVD)} & \textbf{SSv2 (VideoMAE)} & \textbf{SSv2 (MVD)} \\
                    \midrule
                    ViT-S & ViT-B & 79.0 & \textbf{80.6} & 66.4 & \textbf{70.7} \\
                    ViT-S & ViT-L & 79.0 & \textbf{81.0} & 66.4 & \textbf{70.9} \\
                    ViT-B & ViT-B & 81.5 & \textbf{82.7} & 69.7 & \textbf{72.5} \\
                    ViT-B & ViT-L & 81.5 & \textbf{83.4} & 69.7 & \textbf{73.7} \\
                    ViT-L & ViT-L & 85.2 & \textbf{86.0} & 74.0 & \textbf{76.1} \\
                    \bottomrule
                \end{tabular}
            \end{table}
            
            \paragraph{End-to-end comparisons}
            Selections from \cite{wang2023_mvd} are reproduced below for completeness. Methods cited include supervised baselines and self-supervised contemporaries such as ST-MAE \cite{feichtenhofer2022_mae_st}, OmniMAE \cite{girdhar2023_omnimae}, BEVT \cite{wang2022_bevt}, MaskFeat \cite{wei2022_maskfeat}, MViTv2 \cite{li2021_improved_mvit}, VideoSwin \cite{liu2022_videoswin}, TimeSformer \cite{bertasius2021_timesformer}, ViViT \cite{arnab2021_vivit}, SlowFast \cite{feichtenhofer2019_slowfast}, NL I3D \cite{wang2018_nonlocal}, ip-CSN \cite{tran2019_ipcsn}, X3D \cite{feichtenhofer2020_x3d}, MViTv1 \cite{fan2021_mvit}, UniFormer \cite{li2022_uniformer}, and Mformer \cite{patrick2021_mformer}. All numbers and settings match the paper’s tables.
            
            \begin{table}[H]
                \centering
                \small
                \setlength{\tabcolsep}{4pt}
                \caption{Kinetics-400 comparisons (single-view cost $\times$ \#views). Bold rows indicate MVD \cite{wang2023_mvd}.}
                \label{tab:chapter24_mvd_k400}
                \resizebox{\linewidth}{!}{%
                    \begin{tabular}{lcccc}
                        \toprule
                        \textbf{Method} & \textbf{Extra data} & \textbf{top-1} & \textbf{top-5} & \textbf{GFLOPs / Param} \\
                        \midrule
                        NL I3D R101 \cite{wang2018_nonlocal} & -- & 77.3 & 93.3 & $359{\times}30$ / 62 \\
                        ip-CSN-152 \cite{tran2019_ipcsn} & -- & 77.8 & 92.8 & $109{\times}30$ / 33 \\
                        SlowFast NL \cite{feichtenhofer2019_slowfast} & -- & 79.8 & 93.9 & $234{\times}30$ / 60 \\
                        X3D-XL \cite{feichtenhofer2020_x3d} & -- & 79.1 & 93.9 & $48{\times}30$ / 11 \\
                        MViTv1-B \cite{fan2021_mvit} & -- & 80.2 & 94.4 & $170{\times}5$ / 37 \\
                        VideoSwin-B \cite{liu2022_videoswin} & IN-1K & 80.6 & 94.6 & $282{\times}12$ / 88 \\
                        Uniformer-B \cite{li2022_uniformer} & IN-1K & 83.0 & 95.4 & $259{\times}12$ / 50 \\
                        TimeSformer \cite{bertasius2021_timesformer} & IN-21K & 80.7 & 94.7 & $2380{\times}3$ / 121 \\
                        Mformer-B \cite{patrick2021_mformer} & IN-21K & 79.7 & 94.2 & $370{\times}30$ / 109 \\
                        Mformer-L \cite{patrick2021_mformer} & IN-21K & 80.2 & 94.8 & $1185{\times}30$ / 382 \\
                        ViViT-L FE \cite{arnab2021_vivit} & IN-21K & 81.7 & 93.8 & $3980{\times}3$ / N/A \\
                        VideoSwin-L \cite{liu2022_videoswin} & IN-21K & 83.1 & 95.9 & $604{\times}12$ / 197 \\
                        \midrule
                        VIMPAC ViT-L \cite{tan2021_vimpac} & HowTo100M & 77.4 & N/A & N/A$\times$30 / 307 \\
                        BEVT Swin-B \cite{wang2022_bevt} & IN-1K & 81.1 & N/A & $282{\times}12$ / 88 \\
                        MaskFeat MViT-S \cite{wei2022_maskfeat} & -- & 82.2 & 95.1 & $71{\times}10$ / 36 \\
                        VideoMAE ViT-S \cite{tong2022_videomae} & -- & 79.0 & 93.8 & $57{\times}15$ / 22 \\
                        VideoMAE ViT-B \cite{tong2022_videomae} & -- & 81.5 & 95.1 & $180{\times}15$ / 87 \\
                        VideoMAE ViT-L \cite{tong2022_videomae} & -- & 85.2 & 96.8 & $597{\times}15$ / 305 \\
                        VideoMAE ViT-H \cite{tong2022_videomae} & -- & 86.6 & 97.1 & $1192{\times}15$ / 633 \\
                        ST-MAE ViT-B \cite{feichtenhofer2022_mae_st} & -- & 81.3 & 94.9 & $180{\times}21$ / 87 \\
                        ST-MAE ViT-L \cite{feichtenhofer2022_mae_st} & -- & 84.8 & 96.2 & $598{\times}21$ / 304 \\
                        ST-MAE ViT-H \cite{feichtenhofer2022_mae_st} & -- & 85.1 & 96.6 & $1193{\times}21$ / 632 \\
                        OmniMAE ViT-B \cite{girdhar2023_omnimae} & IN-1K & 80.8 & N/A & $180{\times}15$ / 87 \\
                        OmniMAE ViT-L \cite{girdhar2023_omnimae} & IN-1K+SSv2 & 84.0 & N/A & $597{\times}15$ / 305 \\
                        OmniMAE ViT-H \cite{girdhar2023_omnimae} & IN-1K+SSv2 & 84.8 & N/A & $1192{\times}15$ / 633 \\
                        \midrule
                        \textbf{MVD-S (Teacher-B)} \cite{wang2023_mvd} & IN-1K & {80.6} & 94.7 & $57{\times}15$ / 22 \\
                        \textbf{MVD-S (Teacher-L)} \cite{wang2023_mvd} & IN-1K & {81.0} & 94.8 & $57{\times}15$ / 22 \\
                        \textbf{MVD-B (Teacher-B)} \cite{wang2023_mvd} & IN-1K & {82.7} & 95.4 & $180{\times}15$ / 87 \\
                        \textbf{MVD-B (Teacher-L)} \cite{wang2023_mvd} & IN-1K & {83.4} & 95.8 & $180{\times}15$ / 87 \\
                        \textbf{MVD-L (Teacher-L)} \cite{wang2023_mvd} & IN-1K & {86.0} & 96.9 & $597{\times}15$ / 305 \\
                        \textbf{MVD-L (Teacher-L)$^\dagger$} \cite{wang2023_mvd} & IN-1K & \textbf{86.4} & \textbf{97.0} & $597{\times}15$ / 305 \\
                        \textbf{MVD-H (Teacher-H)$^\dagger$} \cite{wang2023_mvd} & IN-1K & \textbf{87.2} & \textbf{97.4} & $1192{\times}15$ / 633 \\
                        \bottomrule
                \end{tabular}}
            \end{table}
            
            \begin{table}[H]
                \centering
                \small
                \setlength{\tabcolsep}{3pt}
                \renewcommand{\arraystretch}{0.92}
                \caption{Something-Something V2 comparisons. $^\dagger$ indicates 800-epoch distillation \cite{wang2023_mvd}.}
                \label{tab:chapter24_mvd_ssv2}
                \resizebox{0.95\linewidth}{!}{%
                    \begin{tabular}{lcccc}
                        \toprule
                        \textbf{Method} & \textbf{Extra data} & \textbf{top-1} & \textbf{GFLOPs} & \textbf{Param} \\
                        \midrule
                        \multicolumn{5}{l}{\emph{supervised}} \\
                        SlowFast R101 \cite{feichtenhofer2019_slowfast} & K400 & 63.1 & $106{\times}3$ & 53 \\
                        TSM-RGB R50 \cite{lin2019_tsm} & IN-1K & 63.3 & $62{\times}6$ & 24 \\
                        TAM R50 \cite{liu2021_tam} & IN-1K & 66.0 & $99{\times}6$ & 51 \\
                        TDN R101 \cite{wang2021_tdn} & IN-1K & 69.6 & $198{\times}3$ & 88 \\
                        MViTv1-B \cite{fan2021_mvit} & -- & 67.7 & $455{\times}3$ & 37 \\
                        MViTv2-B \cite{li2021_improved_mvit} & K400 & 70.5 & $225{\times}3$ & 51 \\
                        UniFormer-B \cite{li2022_uniformer} & K400 & 71.2 & $259{\times}3$ & 50 \\
                        TimeSformer-HR \cite{bertasius2021_timesformer} & IN-21K & 62.5 & $1703{\times}3$ & 121 \\
                        ViViT-L FE \cite{arnab2021_vivit} & IN-21K+K400 & 65.9 & $995{\times}12$ & N/A \\
                        Mformer-B \cite{patrick2021_mformer} & IN-21K+K400 & 66.5 & $370{\times}3$ & 109 \\
                        Mformer-L \cite{patrick2021_mformer} & IN-21K+K400 & 68.1 & $1185{\times}3$ & 382 \\
                        VideoSwin-B \cite{liu2022_videoswin} & IN-21K+K400 & 69.6 & $321{\times}3$ & 88 \\
                        MViTv2-L \cite{li2021_improved_mvit} & IN-21K+K400 & 73.3 & $2828{\times}3$ & 213 \\
                        \midrule
                        \multicolumn{5}{l}{\emph{self-supervised}} \\
                        VIMPAC ViT-L \cite{tan2021_vimpac} & HowTo100M & 68.1 & N/A$\times$30 & 307 \\
                        BEVT Swin-B \cite{wang2022_bevt} & IN-1K+K400 & 71.4 & $321{\times}3$ & 88 \\
                        MaskFeat MViT-L \cite{wei2022_maskfeat} & K400 & 74.4 & $2828{\times}3$ & 218 \\
                        VideoMAE ViT-S \cite{tong2022_videomae} & K400 & 66.4 & $57{\times}6$ & 22 \\
                        VideoMAE ViT-S \cite{tong2022_videomae} & -- & 66.8 & $57{\times}6$ & 22 \\
                        VideoMAE ViT-B \cite{tong2022_videomae} & K400 & 69.7 & $180{\times}6$ & 87 \\
                        VideoMAE ViT-B \cite{tong2022_videomae} & -- & 70.8 & $180{\times}6$ & 87 \\
                        VideoMAE ViT-L \cite{tong2022_videomae} & K400 & 74.0 & $597{\times}6$ & 305 \\
                        VideoMAE ViT-L \cite{tong2022_videomae} & -- & 74.3 & $597{\times}6$ & 305 \\
                        ST-MAE ViT-L \cite{feichtenhofer2022_mae_st} & K400 & 72.1 & $598{\times}3$ & 304 \\
                        ST-MAE ViT-H \cite{feichtenhofer2022_mae_st} & K400 & 74.1 & $1193{\times}3$ & 632 \\
                        OmniMAE ViT-B \cite{girdhar2023_omnimae} & IN-1K & 69.5 & $180{\times}6$ & 87 \\
                        OmniMAE ViT-B \cite{girdhar2023_omnimae} & IN-1K+K400 & 69.0 & $180{\times}6$ & 87 \\
                        OmniMAE ViT-L \cite{girdhar2023_omnimae} & IN-1K & 74.2 & $597{\times}6$ & 305 \\
                        OmniMAE ViT-H \cite{girdhar2023_omnimae} & IN-1K & 75.3 & $1192{\times}6$ & 632 \\
                        \midrule
                        \textbf{MVD-S (Teacher-B)} \cite{wang2023_mvd} & IN-1K+K400 & \textbf{70.7} & $57{\times}6$ & 22 \\
                        \textbf{MVD-S (Teacher-L)} \cite{wang2023_mvd} & IN-1K+K400 & \textbf{70.9} & $57{\times}6$ & 22 \\
                        \textbf{MVD-B (Teacher-B)} \cite{wang2023_mvd} & IN-1K+K400 & \textbf{72.5} & $180{\times}6$ & 87 \\
                        \textbf{MVD-B (Teacher-L)} \cite{wang2023_mvd} & IN-1K+K400 & \textbf{73.7} & $180{\times}6$ & 87 \\
                        \textbf{MVD-L (Teacher-L)} \cite{wang2023_mvd} & IN-1K+K400 & \textbf{76.1} & $597{\times}6$ & 305 \\
                        \textbf{MVD-L (Teacher-L)$^\dagger$} \cite{wang2023_mvd} & IN-1K+K400 & \textbf{76.7} & $597{\times}6$ & 305 \\
                        \textbf{MVD-H (Teacher-H)$^\dagger$} \cite{wang2023_mvd} & IN-1K+K400 & \textbf{77.3} & $1192{\times}6$ & 633 \\
                        \bottomrule
                \end{tabular}}
            \end{table}
            
            \begin{table}[H]
                \centering
                \small
                \setlength{\tabcolsep}{4pt}
                \caption{AVA v2.2 action detection (mAP) comparisons with and without intermediate labeled finetuning on the pretraining dataset \cite{wang2023_mvd}. “Extra labels” denotes whether the pretrained model is intermediately finetuned on the pretraining video dataset with labels before transfer to AVA.}
                \label{tab:chapter24_mvd_ava}
                \resizebox{\linewidth}{!}{%
                    \begin{tabular}{lcccc}
                        \toprule
                        \textbf{Method} & \textbf{Extra data} & \textbf{Extra labels} & \textbf{mAP} & \textbf{GFLOPs} \\
                        \midrule
                        SlowFast R101 \cite{feichtenhofer2019_slowfast} & K400 & \cmark & 23.8 & 138 \\
                        MViTv2-B \cite{li2021_improved_mvit} & K400 & \cmark & 29.0 & 225 \\
                        MViTv2-L \cite{li2021_improved_mvit} & IN-21K+K700 & \cmark & 34.4 & 2828 \\
                        \midrule
                        MaskFeat MViT-L \cite{wei2022_maskfeat} & K400 & \cmark & 37.5 & 2828 \\
                        VideoMAE ViT-B \cite{tong2022_videomae} & K400 & \xmark & 26.7 & 180 \\
                        VideoMAE ViT-B \cite{tong2022_videomae} & K400 & \cmark & 31.8 & 180 \\
                        VideoMAE ViT-L \cite{tong2022_videomae} & K400 & \xmark & 34.3 & 597 \\
                        VideoMAE ViT-L \cite{tong2022_videomae} & K400 & \cmark & 37.0 & 597 \\
                        VideoMAE ViT-H \cite{tong2022_videomae} & K400 & \xmark & 36.5 & 1192 \\
                        VideoMAE ViT-H \cite{tong2022_videomae} & K400 & \cmark & 39.5 & 1192 \\
                        ST-MAE ViT-L \cite{feichtenhofer2022_mae_st} & K400 & \cmark & 35.7 & 598 \\
                        ST-MAE ViT-H \cite{feichtenhofer2022_mae_st} & K400 & \cmark & 36.2 & 1193 \\
                        \midrule
                        \textbf{MVD-B (Teacher-B)} \cite{wang2023_mvd} & IN-1K+K400 & \xmark & 29.3 & 180 \\
                        \textbf{MVD-B (Teacher-B)} \cite{wang2023_mvd} & IN-1K+K400 & \cmark & 33.6 & 180 \\
                        \textbf{MVD-B (Teacher-L)} \cite{wang2023_mvd} & IN-1K+K400 & \xmark & 31.1 & 180 \\
                        \textbf{MVD-B (Teacher-L)} \cite{wang2023_mvd} & IN-1K+K400 & \cmark & 34.2 & 180 \\
                        \textbf{MVD-L (Teacher-L)} \cite{wang2023_mvd} & IN-1K+K400 & \xmark & {37.7} & 597 \\
                        \textbf{MVD-L (Teacher-L)} \cite{wang2023_mvd} & IN-1K+K400 & \cmark & {38.7} & 597 \\
                        \textbf{MVD-H (Teacher-H)} \cite{wang2023_mvd} & IN-1K+K400 & \xmark & \textbf{40.1} & 1192 \\
                        \textbf{MVD-H (Teacher-H)} \cite{wang2023_mvd} & IN-1K+K400 & \cmark & \textbf{41.1} & 1192 \\
                        \bottomrule
                \end{tabular}}
            \end{table}
            
            \paragraph{Transfer: UCF101 and HMDB51}
            \begin{table}[H]
                \centering
                \small
                \setlength{\tabcolsep}{8pt}
                \caption{Comparison with previous SOTA on UCF101 and HMDB51 (averaged over standard splits) \cite{wang2023_mvd,pan2021_videomoco,han2020_memdpc,chen2021_corp,qian2021_cvrl,recasens2021_broaden,tan2021_vimpac,tong2022_videomae}.}
                \label{tab:chapter24_mvd_ucf_hmdb}
                \begin{tabular}{lcccc}
                    \toprule
                    \textbf{Method} & \textbf{Extra data} & \textbf{Param} & \textbf{UCF101} & \textbf{HMDB51} \\
                    \midrule
                    VideoMoCo R2{+}1D \cite{pan2021_videomoco} & K400 & 15 & 78.7 & 49.2 \\
                    MemDPC R2D3D \cite{han2020_memdpc} & K400 & 32 & 86.1 & 54.5 \\
                    Vi$^2$CLR S3D \cite{qian2021_cvrl} & K400 & 9 & 89.1 & 55.7 \\
                    CORP Slow-R50 \cite{chen2021_corp} & K400 & 32 & 93.5 & 68.0 \\
                    CVRL Slow-R50 \cite{qian2021_cvrl} & K400 & 32 & 92.9 & 67.9 \\
                    CVRL Slow-R152 \cite{qian2021_cvrl} & K600 & 328 & 94.4 & 70.6 \\
                    Broaden Your Views (BYOL) Slow-R50 \cite{recasens2021_broaden} & K400 & 32 & 94.2 & 72.1 \\
                    VIMPAC ViT-L \cite{tan2021_vimpac} & HowTo100M & 307 & 92.7 & 65.9 \\
                    VideoMAE ViT-B \cite{tong2022_videomae} & K400 & 87 & 96.1 & 73.3 \\
                    \textbf{MVD-B (Teacher-B)} \cite{wang2023_mvd} & IN-1K+K400 & 87 & \textbf{97.0} & \textbf{76.4} \\
                    \textbf{MVD-B (Teacher-L)} \cite{wang2023_mvd} & IN-1K+K400 & 87 & \textbf{97.5} & \textbf{79.7} \\
                    \bottomrule
                \end{tabular}
            \end{table}
            
            \paragraph{Training time}
            \begin{table}[H]
                \centering
                \small
                \setlength{\tabcolsep}{12pt}
                \caption{ViT-B training time on 32$\times$V100 GPUs (teacher cost included for MVD) \cite{wang2023_mvd}.}
                \label{tab:chapter24_mvd_time}
                \begin{tabular}{lccc}
                    \toprule
                    \textbf{Method} & \textbf{Epochs} & \textbf{Time (h)} & \textbf{K400 top-1} \\
                    \midrule
                    VideoMAE \cite{tong2022_videomae} & 800  & 107 & 81.0 \\
                    VideoMAE \cite{tong2022_videomae} & 1600 & 214 & 81.5 \\
                    MVD \cite{wang2023_mvd}           & 400  & 57  & \textbf{81.9} \\
                    \bottomrule
                \end{tabular}
            \end{table}
            
            \paragraph{Ablations: pixels during distillation}
            \begin{table}[H]
                \centering
                \small
                \setlength{\tabcolsep}{10pt}
                \caption{Effect of regressing pixels during distillation on SSv2 (student ViT-S, teacher ViT-B, 300 epochs) \cite{wang2023_mvd}.}
                \label{tab:chapter24_mvd_pixels}
                \begin{tabular}{lcc}
                    \toprule
                    \textbf{Teachers} & \textbf{Reconstruct pixels} & \textbf{SSv2 top-1} \\
                    \midrule
                    image & \xmark & 68.7 \\
                    image & \cmark & 67.9 \\
                    image+video & \xmark & \textbf{70.1} \\
                    image+video & \cmark & 69.0 \\
                    \bottomrule
                \end{tabular}
            \end{table}
            
            \paragraph{Bootstrapped teachers and IN1K-initialized students}
            \begin{table}[H]
                \centering
                \small
                \setlength{\tabcolsep}{8pt}
                \caption{Comparison with bootstrapped teachers and students initialized with IN1K-pretrained models (ViT-B) \cite{wang2023_mvd}.}
                \label{tab:chapter24_mvd_bootstrap}
                \begin{tabular}{lcccc}
                    \toprule
                    \textbf{Teacher} & \textbf{IN1K init} & \textbf{Epoch} & \textbf{K400 top-1} & \textbf{SSv2 top-1} \\
                    \midrule
                    momentum encoder & \xmark & 800 & 80.5 & 70.4 \\
                    momentum encoder & \cmark & 800 & 81.8 & 70.8 \\
                    fixed image model & \xmark & 400 & 82.3 & 71.4 \\
                    fixed video model & \xmark & 400 & 82.1 & 71.8 \\
                    fixed co-teaching & \xmark & 400 & \textbf{82.7} & \textbf{72.5} \\
                    \bottomrule
                \end{tabular}
            \end{table}
            
            \paragraph{Ablations: masked reconstruction vs. per-token feature distillation}
            \begin{table}[H]
                \centering
                \small
                \setlength{\tabcolsep}{16pt}
                \caption{Masked reconstruction vs. per-token feature distillation (teacher ViT-B) on K400 and SSv2 \cite{wang2023_mvd}.}
                \label{tab:chapter24_mvd_featuredistill}
                \begin{tabular}{lcc}
                    \toprule
                    \textbf{Distillation method} & \textbf{K400 top-1} & \textbf{SSv2 top-1} \\
                    \midrule
                    per-token distillation & 80.9 & 70.5 \\
                    masked reconstruction  & \textbf{82.1} & \textbf{71.8} \\
                    \bottomrule
                \end{tabular}
            \end{table}
            
            \subsubsection{Limitations and future directions}
            \label{subsubsec:chapter24_mvd_limits}
            \paragraph{Observed constraints}
            \begin{itemize}
                \item \textbf{Teacher dependence.} Student quality is bounded by the expressiveness and domain of teacher features; suboptimal teachers can bottleneck learning.
                \item \textbf{Feature-space rigidity.} Regressing fixed targets may underexplore alternative, task-beneficial invariances compared to generative pixel objectives or contrastive formulations.
                \item \textbf{Temporal granularity.} The choice to predict a single 2D slice for the image teacher simplifies heads but may limit supervision on fast temporal changes.
            \end{itemize}
            
            \paragraph{Future work}
            \begin{itemize}
                \item \textbf{Adaptive target selection.} Curriculum over teacher layers, token-wise target picking, or uncertainty-aware weighting to emphasize informative regions and times.
                \item \textbf{Richer multi-teacher fusion.} Beyond two teachers, integrate audio, text, or motion-specific teachers with learned routing across decoders.
            \end{itemize}
            
            \paragraph{Summary}
            MVD reframes masked video pretraining as \emph{feature-level} reconstruction under \emph{co-teaching} from frozen image and video teachers. The resulting student inherits complementary spatial and temporal priors, translating into strong accuracy–efficiency trade-offs and state-of-the-art results across K400, SSv2, AVA, and small-dataset transfers \cite{wang2023_mvd,tong2022_videomae}.
        
    \end{enrichment}
    
\end{enrichment}

\newpage

\begin{enrichment}[Instruction-Tuned VLLM Precursors][section]
    \label{enr:sec_chapter24_instrprecursors}
    While vision--language alignment provided shared embeddings, these models were still far from the conversational capabilities of LLMs. Instruction tuning closed this gap: by fine-tuning aligned vision--language models on multimodal instruction–response datasets, systems like InstructBLIP~\cite{dai2023_instructblip} and LLaVA~\cite{liu2023_llava} demonstrated how visual input could be used in natural dialogue. This transition was essential for video-LLMs, which inherit the same recipe of pairing pretrained encoders with instruction-tuned LLMs.
    
    \begin{enrichment}[InstructBLIP: Instruction-Tuned Multimodal Alignment][subsection]
        \label{enr:subsec_chapter24_instructblip}
        
        \paragraph{Motivation and Positioning}
        \textit{InstructBLIP}~\cite{dai2023_instructblip} takes the frozen-experts recipe of BLIP-2~\cite{li2023_blip2} (strong vision encoder + strong LLM bridged by a light Q-Former) and \emph{instruction-tunes} it so the system can follow natural, task-agnostic prompts. Unlike multi-task pretraining that memorizes dataset-specific formats, instruction tuning teaches the model \emph{how to read and follow instructions}, enabling zero-shot generalization to unseen tasks and more natural multi-turn visual dialogue.
        
        \paragraph{High-Level Idea}
        Starting from a BLIP-2 backbone (frozen image encoder, frozen LLM, trainable Q-Former + projection), InstructBLIP reformats diverse vision–language datasets as \emph{instruction $\rightarrow$ response} pairs and optimizes a standard language-modeling loss on the LLM. Two design choices are key: (i) \emph{instruction-aware} Q-Former features that condition visual extraction on the incoming instruction, and (ii) a \emph{balanced sampling} strategy across tasks/datasets to avoid overfitting to any single task template.
        
        \begin{figure}[H]
            \centering
            \includegraphics[width=0.85\textwidth]{Figures/Chapter_24/InstructBLIP_architecture.jpg}
            \caption{\textbf{InstructBLIP architecture}~\cite{dai2023_instructblip}. A frozen image encoder (e.g., ViT-g/14 from CLIP/EVA-CLIP) feeds patch embeddings to a trainable \emph{Q-Former}. The Q-Former uses learnable queries that \emph{attend to the instruction tokens and the visual tokens} to produce \emph{instruction-aware} visual features. A linear projection maps these features to the frozen LLM’s embedding space (e.g., FlanT5 or Vicuna), where they serve as \emph{soft prompts}. Training uses a next-token LM objective over instruction-formatted data; at inference the model follows arbitrary prompts in a conversational loop. Source:\cite{dai2023_instructblip}.}
            \label{fig:chpapter24_instructblip_arch}
        \end{figure}
        
        \paragraph{How It Works (Mechanism)}
        \begin{itemize}
            \item \textbf{Instruction-aware Q-Former (task-conditioned queries).}
            Unlike BLIP-2’s task-agnostic queries~\cite{li2023_blip2}, \emph{InstructBLIP}~\cite{dai2023_instructblip} fuses the \emph{instruction tokens} into the Q-Former so that its learnable queries become conditioned on user intent. Concretely, the frozen ViT produces patch features, and the Q-Former receives both these patches and the instruction embeddings. Queries self-attend, then cross-attend to patches while also attending to instructions. This lets them extract \emph{instruction-relevant} visual evidence (OCR for “read the sign”, spatial reasoning for “which cup is left of the plate?”), instead of a single generic visual summary. This conditioning reduces spurious correlations and disambiguates what to focus on when instructions change at test time.
            
            \item \textbf{Soft visual prompting into a frozen LLM.}
            After $L_q$ layers, the Q-Former outputs $K$ query vectors $Q \in \mathbb{R}^{K \times d}$ which are linearly projected to the LLM’s embedding size, $\tilde{Q} = QW_p$. These \emph{visual prompt tokens} are prepended to the instruction tokens and passed to a frozen LLM (e.g., FlanT5, Vicuna). The LLM thus conditions generation on a compact, instruction-aligned “briefing” while preserving its linguistic competence. Compared to BLIP-2, these tokens are instruction-aware and better aligned with the decoding trajectory, improving grounding and factuality.
            
            \item \textbf{Data flow (end-to-end).}
            \begin{enumerate}
                \item Image $\rightarrow$ frozen ViT $\rightarrow$ patch embeddings.
                \item Instruction + learnable queries + patches $\rightarrow$ Q-Former $\rightarrow$ instruction-aware queries.
                \item Projection $W_p$ maps to LLM space, queries prepended to instruction tokens.
                \item Frozen LLM autoregressively generates the response.
            \end{enumerate}
            
            \item \textbf{Instruction-tuning objective.}
            For each (\texttt{instruction, image}) $\mapsto$ response example, the model minimizes next-token LM loss:
            \[
            \mathcal{L}_{\text{LM}} = - \sum_m \log p_{\text{LLM}}(y_m \mid y_{<m}, \tilde{Q}(\text{image}, \text{instruction}), \text{instruction}),
            \]
            updating only the Q-Former and projection layers; both ViT and LLM remain frozen.
        \end{itemize}
        
        \paragraph{Why Instruction Tuning Helps (Intuition)}
        BLIP and BLIP-2 already use LM loss, but their inputs are \emph{task-agnostic} (generic queries + dataset-formatted prompts). This means the model often learns dataset-specific mappings rather than a general instruction-following procedure. InstructBLIP changes this in two ways:
        \begin{enumerate}
            \item \textbf{Instruction-aware Q-Former.} Instructions are fused with image tokens, so the Q-Former extracts only the \emph{instruction-relevant} visual evidence (e.g., text regions for OCR, spatial cues for reasoning) instead of a fixed summary.
            \item \textbf{Instruction-formatted LM training.} Every example is presented as natural instructions with answers, not dataset templates. The LLM is therefore trained to parse arbitrary instructions and ground them in vision.
        \end{enumerate}
        The difference is subtle but critical: rather than memorizing dataset patterns, the model learns the meta-skill of following instructions—leading to stronger generalization to unseen tasks and more reliable multi-turn interaction.
        
        \begin{figure}[H]
            \centering
            \includegraphics[width=0.85\textwidth]{Figures/Chapter_24/InstructBLIP_examples.jpg}
            \caption{\textbf{Qualitative behaviors of InstructBLIP (Vicuna variants)}~\cite{dai2023_instructblip}. The model follows open-form instructions: (i) rich descriptions that list attributes and composition; (ii) visual commonsense reasoning (e.g., infer damage cause); (iii) abstract/hypothetical queries (metaphor vs.\ literal); (iv) knowledge-grounded recognition (e.g., famous artworks); (v) practical steps and multi-turn dialogue. These examples illustrate that instruction tuning teaches \emph{how to follow instructions} rather than memorizing dataset formats. Source:\cite{dai2023_instructblip}.}
            \label{fig:chpapter24_instructblip_examples}
        \end{figure}
        
        \paragraph{Data \& Formatting: From Multi-Task to Instruction-Tuning}
        \begin{itemize}
            \item \textbf{Task coverage.} InstructBLIP unifies 26 datasets spanning captioning, VQA (general, OCR, knowledge-grounded), visual reasoning (GQA), visual dialogue (VisDial), video QA (MSVD or MSRVTT), safety (HatefulMemes), and more.
            \item \textbf{Prompt templates.} Each example is rendered as (\texttt{Instruction: …}, optional \texttt{Context: …}, \texttt{Image: …} $\rightarrow$ \texttt{Answer: …}). This normalizes heterogeneous supervision into a single \emph{instruction-following} interface that the LLM already excels at.
            \item \textbf{Balanced sampling.} A sampling scheme evens exposure across tasks and avoids dominance by large sources (e.g., web captions), improving transfer to held-out tasks and robustness to prompt phrasing.
        \end{itemize}
        
        \begin{figure}[H]
            \centering
            \includegraphics[width=0.85\textwidth]{Figures/Chapter_24/InstructBLIP_datasets.jpg}
            \caption{\textbf{Instruction-tuning sources}~\cite{dai2023_instructblip}. InstructBLIP formats a broad mixture of vision–language datasets as \emph{instruction $\rightarrow$ response}. “Held-in” sets are used for tuning and evaluation; “held-out” sets are reserved for zero-shot generalization. Diversity (OCR, knowledge, reasoning, dialogue, video) is crucial for task transfer under natural prompts. Source:\cite{dai2023_instructblip}.}
            \label{fig:chpapter24_instructblip_datasets}
        \end{figure}
        
        \begin{table}[H]
            \centering
            \small
            \setlength{\tabcolsep}{3.5pt}
            \caption{Zero-shot results on held-out datasets using \textsc{InstructBLIP}~\cite{dai2023_instructblip}. VisDial: Visual Dialog, HM: HatefulMemes, SciQA: ScienceQA (image-context split). For NoCaps/Flickr we report CIDEr; for iVQA we report iVQA accuracy; for HM we report AUC; for VisDial we report MRR; others are top-1 accuracy (\%). Source:\cite{dai2023_instructblip}.}
            \label{tab:instructblip_zeroshot}
            \resizebox{\linewidth}{!}{%
                \begin{tabular}{l*{13}{c}}
                    \toprule
                    \textbf{Method} & \textbf{NoCaps} & \textbf{Flickr30K} & \textbf{GQA} & \textbf{VSR} & \textbf{IconQA} & \textbf{TextVQA} & \textbf{VisDial} & \textbf{HM} & \textbf{VizWiz} & \textbf{SciQA IMG} & \textbf{MSVD QA} & \textbf{MSRVTT QA} & \textbf{iVQA} \\
                    \midrule
                    Flamingo-3B~\cite{flamingo2022_fewshot} & -- & 60.6 & --  & --  & --  & 30.1 & --  & 53.7 & 28.9 & --  & 27.5 & 11.0 & 32.7 \\
                    Flamingo-9B~\cite{flamingo2022_fewshot} & -- & 61.5 & --  & --  & --  & 31.8 & --  & 57.0 & 28.8 & --  & 30.2 & 13.7 & 35.2 \\
                    Flamingo-80B~\cite{flamingo2022_fewshot} & -- & 67.2 & --  & --  & --  & 35.0 & --  & 46.4 & 31.6 & --  & 35.6 & 17.4 & 40.7 \\
                    BLIP-2 (FlanT5\textsubscript{XL})~\cite{li2023_blip2} & 104.5 & 76.1 & 44.0 & 60.5 & 45.5 & 43.1 & 45.7 & 53.0 & 29.8 & 54.9 & 33.7 & 16.2 & 40.4 \\
                    BLIP-2 (FlanT5\textsubscript{XXL})~\cite{li2023_blip2} & 98.4 & 73.7 & 44.6 & 68.2 & 45.4 & 44.1 & 46.9 & 52.0 & 29.4 & 64.5 & 34.4 & 17.4 & 45.8 \\
                    BLIP-2 (Vicuna-7B) & 107.5 & 74.9 & 38.6 & 50.0 & 39.7 & 40.1 & 44.9 & 50.6 & 25.3 & 53.8 & 18.3 & 9.2  & 27.5 \\
                    BLIP-2 (Vicuna-13B) & 103.9 & 71.6 & 41.0 & 50.9 & 40.6 & 42.5 & 45.1 & 53.7 & 19.6 & 61.0 & 20.3 & 10.3 & 23.5 \\
                    \midrule
                    InstructBLIP (FlanT5\textsubscript{XL}) & 119.9 & \textbf{84.5} & 48.4 & 64.8 & 50.0 & 46.6 & 46.6 & 56.6 & 32.7 & 70.4 & 43.4 & 25.0 & 53.1 \\
                    InstructBLIP (FlanT5\textsubscript{XXL}) & 120.0 & 83.5 & 47.9 & \textbf{65.6} & \textbf{51.2} & 46.6 & \textbf{48.5} & 54.1 & 30.9 & \textbf{70.6} & \textbf{44.3} & \textbf{25.6} & \textbf{53.8} \\
                    InstructBLIP (Vicuna-7B) & \textbf{123.1} & 82.4 & 49.2 & 54.3 & 43.1 & 50.1 & 45.2 & \textbf{59.6} & \textbf{34.5} & 60.5 & 41.8 & 22.1 & 52.2 \\
                    InstructBLIP (Vicuna-13B) & 121.9 & 82.8 & \textbf{49.5} & 52.1 & 44.8 & \textbf{50.7} & 45.4 & 57.5 & 33.4 & 63.1 & 41.2 & 24.8 & 51.0 \\
                    \bottomrule
            \end{tabular}}
        \end{table}
        
        \paragraph{Ablations: What Matters}
        Two ingredients dominate gains: \emph{instruction-aware} visual features and \emph{balanced sampling}. Making the Q-Former conditional on the instruction reliably boosts \emph{held-out} generalization—especially for knowledge/OCR and instruction-sensitive sets (e.g., large drops on ScienceQA and iVQA when removed). Balanced sampling yields smaller but consistent gains by preventing over-fitting to high-volume tasks; without it, performance regresses across most held-out datasets, with only minor, noisy exceptions.
        
        \begin{table}[H]
            \centering
            \small
            \setlength{\tabcolsep}{4pt}
            \renewcommand{\arraystretch}{0.96}
            \caption{Ablations from \textsc{InstructBLIP}~\cite{dai2023_instructblip}. Held-in Avg.\ averages COCO Caption, OKVQA, A-OKVQA, TextCaps; held-out columns report across distinct tasks. Parentheses show deltas vs.\ the full model. Source:\cite{dai2023_instructblip}.}
            \label{tab:instructblip_ablation}
            \resizebox{\linewidth}{!}{%
                \begin{tabular}{lcccccc}
                    \toprule
                    \textbf{Model} & \textbf{Held-in Avg.} & \textbf{GQA} & \textbf{ScienceQA (IMG)} & \textbf{IconQA} & \textbf{VizWiz} & \textbf{iVQA} \\
                    \midrule
                    InstructBLIP (FlanT5\textsubscript{XL})                      & 94.1 & 48.4 & 70.4 & 50.0 & 32.7 & 53.1 \\
                    \quad w/o Instruction-aware Visual Features                  & 89.8 & 45.9 (\,$\downarrow$2.5) & 63.4 (\,$\downarrow$7.0) & 45.8 (\,$\downarrow$4.2) & 25.1 (\,$\downarrow$7.6) & 47.5 (\,$\downarrow$5.6) \\
                    \quad w/o Data Balancing                                     & 92.6 & 46.8 (\,$\downarrow$1.6) & 66.0 (\,$\downarrow$4.4) & 49.9 (\,$\downarrow$0.1) & 31.8 (\,$\downarrow$0.9) & 51.1 (\,$\downarrow$2.0) \\
                    \midrule
                    InstructBLIP (Vicuna-7B)                                     & 100.8 & 49.2 & 60.5 & 43.1 & 34.5 & 52.2 \\
                    \quad w/o Instruction-aware Visual Features                  & 98.9  & 48.2 (\,$\downarrow$1.0) & 55.2 (\,$\downarrow$5.3) & 41.2 (\,$\downarrow$1.9) & 32.4 (\,$\downarrow$2.1) & 36.8 (\,$\downarrow$15.4) \\
                    \quad w/o Data Balancing                                     & 98.8  & 47.8 (\,$\downarrow$1.4) & 59.4 (\,$\downarrow$1.1) & 43.5 (\,$\uparrow$0.4)   & 32.3 (\,$\downarrow$2.2) & 50.3 (\,$\downarrow$1.9) \\
                    \bottomrule
            \end{tabular}}
        \end{table}
        
        \paragraph{Instruction Tuning vs.\ Multi-Task Training}
        Plain multi-tasking—either with raw inputs or dataset-tag prompts—learns brittle format$\to$answer mappings that score high on \emph{held-in} sets but fail to transfer. Instruction tuning reframes \emph{every} example as a natural instruction and routes it through an instruction-aware Q-Former, teaching a general procedure (parse intent $\rightarrow$ extract relevant evidence $\rightarrow$ answer). This shift explains the sizeable \emph{held-out} gains while maintaining competitive \emph{held-in} scores.
        
        \begin{figure}[H]
            \centering
            \includegraphics[width=0.85\textwidth]{Figures/Chapter_24/InstructBLIP_multi_task_vs_instruction_tuning.jpg}
            \caption{\textbf{Instruction tuning vs.\ multi-task training (BLIP-2 FlanT5-XL backbone)}~\cite{dai2023_instructblip}. Models trained on plain inputs or dataset-tag prompts excel on held-in but lag on \emph{held-out} tasks. InstructBLIP, trained with instruction formatting and an instruction-aware Q-Former, achieves the strongest held-out generalization with competitive held-in results. Source:\cite{dai2023_instructblip}.}
            \label{fig:chpapter24_instructblip_multitask}
        \end{figure}
        
        \newpage
        
        \paragraph{Downstream Fine-Tuning}
        Instruction-tuned checkpoints are superior initializations: they converge faster and reach higher accuracy with modest adaptation, especially on knowledge/OCR-heavy tasks where instruction parsing and targeted visual grounding are pivotal.
        
        \begin{table}[H]
            \centering
            \small
            \setlength{\tabcolsep}{4pt}
            \renewcommand{\arraystretch}{0.96}
            \caption{Fine-tuning \textsc{BLIP-2} vs.\ \textsc{InstructBLIP} on downstream sets~\cite{dai2023_instructblip}. “Previous SOTA”: LLaVA~\cite{liu2023_llava} (ScienceQA IMG), GIT~\cite{wang2022_git} (OCR-VQA), PaLM-E (562B)~\cite{driess2023_palme} (OKVQA), PromptCap~\cite{hu2023_promptcap}/Answer Heuristics~\cite{shao2023_answer_heuristics} (A-OKVQA). Source:\cite{dai2023_instructblip}.}
            \label{tab:instructblip_finetune}
            \resizebox{\linewidth}{!}{%
                \begin{tabular}{lccccccc}
                    \toprule
                    \textbf{Method} & \textbf{SciQA IMG} & \textbf{OCR-VQA} & \textbf{OKVQA} & \textbf{A-OKVQA Dir Val} & \textbf{A-OKVQA Dir Test} & \textbf{A-OKVQA MC Val} & \textbf{A-OKVQA MC Test} \\
                    \midrule
                    \textit{Previous SOTA (refs)} & \cite{liu2023_llava} 89.0 & \cite{wang2022_git} 70.3 & \cite{driess2023_palme} 66.1 & \cite{hu2023_promptcap} 56.3 & \cite{shao2023_answer_heuristics} 61.6 & \cite{hu2023_promptcap} 73.2 & \cite{shao2023_answer_heuristics} 73.6 \\
                    \midrule
                    BLIP-2 (FlanT5\textsubscript{XXL})~\cite{li2023_blip2}           & 89.5 & 72.7 & 54.7 & 57.6 & 53.7 & 80.2 & 76.2 \\
                    InstructBLIP (FlanT5\textsubscript{XXL})~\cite{dai2023_instructblip} & \textbf{90.7} & \textbf{73.3} & {55.5} & 57.1 & {54.8} & \textbf{81.0} & \textbf{76.7} \\
                    \midrule
                    BLIP-2 (Vicuna-7B)~\cite{li2023_blip2}                 & 77.3 & 69.1 & 59.3 & 60.0 & 58.7 & 72.1 & 69.0 \\
                    InstructBLIP (Vicuna-7B)~\cite{dai2023_instructblip}   & {79.5} & {72.8} & {62.1} & \textbf{64.0} & \textbf{62.1} & {75.7} & {73.4} \\
                    \bottomrule
            \end{tabular}}
        \end{table}
        
        \paragraph{Takeaways (Sharper Reading of the Evidence)}
        Instruction-conditioning is the main driver of \emph{transfer}: removing it collapses iVQA (Vicuna, $-15.4$) and significantly hurts ScienceQA (up to $-7.0$), signalling that \emph{which} visual evidence to extract depends on the instruction. Balanced sampling stabilizes cross-task learning, trimming smaller but pervasive regressions when ablated. Together, these choices explain why instruction-tuned checkpoints fine-tune better than BLIP-2 baselines across diverse downstream tasks.
        
        \paragraph{Limitations and Future Work}
        A central limitation of \textsc{InstructBLIP}~\cite{dai2023_instructblip} is that it is fundamentally \emph{image-centric}. Its Q-Former provides only a narrow ``visual prompt'' interface to the LLM, and although effective for single-image instruction following, it cannot natively capture motion, temporal order, or long-horizon dynamics. Early attempts to handle video defaulted to uniform frame sampling and concatenation of features, which ignores motion cues and collapses temporal structure.
        
        \medskip
        \noindent\textbf{Historical trajectory.}
        \begin{itemize}
            \item \textbf{BLIP and BLIP-2 (2022--2023).} Built for image--text pretraining and instruction tuning, these models demonstrated strong zero-shot transfer but lacked any temporal component. Video QA benchmarks (e.g., MSRVTT-QA) were approached by concatenating features from 4--16 uniformly sampled frames, producing workable results but with no explicit modeling of sequence order or causality.
            \item \textbf{Immediate adaptations.} Several derivatives (\emph{Video-BLIP}, \emph{X-InstructBLIP}, \emph{Video-LLaMA}, \emph{MiniGPT4-Video}) extended the image-focused architecture to video by adding lightweight temporal pooling, video-specific Q-Formers, or scaling up to dozens of sampled frames. These efforts confirmed the flexibility of BLIP-2/LLaVA-style stacks, yet they remained \emph{stopgaps}: temporal reasoning was approximated rather than integrated, and training objectives were still defined at the frame or image level.
        \end{itemize}
        
        \medskip
        \noindent\textbf{Broader limitations.}
        Beyond video, \textsc{InstructBLIP} also inherits dataset and prompt biases, struggles with fine-grained grounding, and lacks support for multi-step tool use or retrieval. Its reliance on a frozen LLM limits adaptability to new domains or safety-critical reasoning.
        
        \newpage
        
        \noindent\textbf{The LLaVA line of work.}
        \begin{itemize}
            \item \textbf{LLaVA (2023).} Bridged frozen CLIP-style encoders with an LLM to enable multimodal dialogue on images. Video extensions (``Video-LLaVA'') treated clips as sets of frames, essentially multi-image interleaving, which worked in practice but without \emph{native} temporal encoding.
            \item \textbf{LLaVA-NeXT (2024).} Addressed another bottleneck—resolution. With its \emph{AnyRes} tiling and merging strategy, higher-resolution visual tokens could be processed without aggressive downsampling, and interleaved multi-image support was improved. Yet video remained a weak spot: sequences were still modeled as unordered image sets with no explicit temporal attention or objectives.
            \item \textbf{LLaVA-OneVision (2024).} Represented a turning point. It unified support for images, image sets, and multi-frame clips through a single tokenization path, introduced time-aware positional embeddings and attention across frames, and trained on mixed video and image data. This enabled \emph{native} video QA and stronger cross-domain transfer, though challenges remained around long-horizon clips and efficient handling of motion-rich inputs.
        \end{itemize}
        
        \medskip
        \noindent\textbf{Future directions (as implied by these limits).}
        The trajectory from BLIP to LLaVA-OneVision highlights both progress and remaining gaps. Key next steps include:
        \begin{itemize}
            \item \textbf{Temporal modeling as a core design.} Moving beyond frame concatenation toward temporal Q-Formers, causal attention, and efficient video transformers to natively capture motion and sequence structure.
            \item \textbf{Scaling instruction coverage.} Broadening instruction tuning across languages, domains, and safety-critical contexts to ensure generalization beyond static-image corpora.
            \item \textbf{Retrieval and tool grounding with time.} Extending retrieval-augmented generation and tool use to temporal settings, linking entities and events across frames or moments in a clip.
        \end{itemize}
        
        \noindent In short, the field evolved from static image instruction tuning (\textsc{BLIP}/\textsc{InstructBLIP}) $\rightarrow$ pragmatic video extensions (Video-LLaVA, MiniGPT4-Video, X-InstructBLIP) $\rightarrow$ stronger but still image-biased upgrades (\textsc{LLaVA-NeXT}) $\rightarrow$ first-class multimodal unification in \textsc{LLaVA-OneVision}~\cite{liu2023_llava,liu2024_llava_next,li2024_llavaonevision}. The logical next step is to make temporal reasoning and retrieval-native grounding as central as resolution and instruction-following have become, starting with LLaVA~\cite{liu2023_llava}
        
    \end{enrichment}
    
    \newpage
    
    \begin{enrichment}[LLaVA: Large Language and Vision Assistant][subsection]
        \label{enr:subsec_chapter24_llava}
        
        \paragraph{High-Level Idea}
        \textsc{LLaVA}~\cite{liu2023_llava} couples a strong \emph{frozen} vision encoder (CLIP ViT-L/14) with an instruction-tuned LLM (Vicuna) via a \emph{lightweight linear projector}. Rather than learning a query bridge (e.g., BLIP-2's Q-Former), LLaVA emphasizes \emph{instruction-following} by training on GPT-4–curated \textbf{visual instruction} data (constructed from captions/boxes but generated without showing images to GPT-4).
        
        \paragraph{Architecture}
        Given image $X_v$, a frozen CLIP encoder produces grid features $Z_v=g(X_v)$. A trainable linear map $W$ projects $Z_v$ into the LLM embedding space to form visual tokens $H_v=WZ_v$, which are prepended/interleaved with text tokens. The model fine-tunes the LLM (often with PEFT/LoRA) using the standard autoregressive LM loss on assistant tokens; the CLIP encoder remains frozen and no cross-attention/Q-Former is used:
        \[
        H_v = W \cdot Z_v,\qquad Z_v = g(X_v).
        \]
        
        \begin{figure}[H]
            \centering
            \includegraphics[width=0.82\textwidth]{Figures/Chapter_24/LLava_architecture.jpg}
            \caption{\textbf{LLaVA network.} A frozen CLIP ViT-L/14 produces grid features; a linear projector $W$ maps them to the LLM token space. Visual tokens are concatenated with the dialogue tokens and trained via LM loss. (As discussed by~\cite{liu2023_llava}, more sophisticated connectors—e.g., Flamingo’s gated cross-attention or BLIP-2’s Q-Former—are possible but were not the focus.). Source:\cite{liu2023_llava}.}
            \label{fig:chpapter24_llava_architecture}
        \end{figure}
        
        \paragraph{Why freeze vision but (partly) train the LLM?}
        \begin{itemize}
            \item \textbf{Protect a strong visual prior.} CLIP ViT-L/14 is already trained on massive image--text corpora; the LLaVA instruction set is comparatively small. Freezing CLIP avoids \emph{catastrophic forgetting} and preserves broad zero-shot visual semantics.
            \item \textbf{Put learning where the skill lives.} Instruction following is largely a \emph{language-side procedure} (parse intent, plan, verbalize). Fine-tuning the LLM (typically with PEFT/LoRA in practice) teaches it to \emph{use} visual tokens as part of that procedure; the connector $W$ just makes CLIP features legible to the LLM.
            \item \textbf{Alignment, not re-seeing.} The main gap is modal \emph{alignment}: map $Z_v=g(X_v)$ into the LLM’s token space and adapt the LLM to condition on those tokens. Training a tiny projector $W$ plus (adapter) updates in the LLM empirically suffices; re-training vision isn’t necessary for instruction-following chat.
            
            \newpage
            
            \item \textbf{Compute reality (clarified).} Full LLM fine-tuning is expensive in absolute FLOPs, and the LLM is much larger than CLIP. LLaVA mitigates this by (i) \emph{freezing} the vision tower entirely, (ii) keeping the connector \emph{minimal} (linear/MLP), and (iii) often using \emph{PEFT} on the LLM. This keeps memory stable and concentrates updates where they matter most for following instructions.
            \item \textbf{Two-stage recipe.} 
            \begin{enumerate}
                \item \emph{Projector warmup.} First, only the lightweight projection layer $W$ is trained on standard image–text pairs, while both the CLIP encoder and the LLM remain frozen. This aligns CLIP’s visual features with the LLM’s embedding space so that visual tokens are “readable” by the language model.
                \item \emph{Visual instruction tuning.} In this stage, $W$ and the LLM are fine-tuned together on multimodal instruction–response pairs that were automatically generated with GPT-4. Since high-quality human-labeled instruction data for images is scarce, GPT-4 is prompted with captions or region annotations to synthesize diverse, instruction-style questions and detailed answers. This produces a large, consistent, and coherent instruction set, allowing the LLM to learn how to weave the projected visual tokens into its reasoning and response generation. The CLIP encoder remains frozen throughout.
            \end{enumerate}
        \end{itemize}
        
        \noindent\emph{Contrast to BLIP-2.} BLIP-2 freezes \emph{both} experts and learns a \textbf{Q-Former} (cross-attention queries) as a task-aware bridge; the LLM is kept frozen during pretraining. LLaVA instead uses a \textbf{simple projector} and invests supervision on the \emph{LLM side} (instruction-tuning), yielding strong conversational adherence with low fusion complexity—at the cost of less structured, query-driven grounding than a learned cross-attention module.
        
        \paragraph{Data Pipeline: Visual Instruction Tuning}
        A central innovation in \textsc{LLaVA}~\cite{liu2023_llava} is how its training data is generated. Importantly, GPT--4 never sees the raw images themselves. Instead, each image is first turned into \emph{textual proxies}---such as captions or detected object labels with bounding boxes---which are then fed to GPT--4. Using this context, GPT--4 is asked to write \textbf{instruction $\rightarrow$ response} examples that look like real multimodal conversations.
        
        \begin{itemize}
            \item \textbf{Input contexts.} For every image, the system prepares:
            \begin{itemize}
                \item \emph{Captions:} several diverse captions describing different aspects of the scene.  
                \item \emph{Box labels:} object categories together with their bounding box coordinates, giving GPT--4 a more structured picture of what is present and where.
            \end{itemize}
            
            \item \textbf{Three kinds of responses.} GPT--4 is asked to produce answers in three styles:
            \begin{enumerate}
                \item \emph{Conversational Q\&A}: short multi-turn dialogues (teaches the model to follow chat-style prompts).  
                \item \emph{Detailed descriptions}: long-form outputs covering fine details (trains thorough grounding and coverage).  
                \item \emph{Complex reasoning}: explanations that require commonsense or multi-step inference (pushes the model beyond surface description).  
            \end{enumerate}
            These three types were chosen to cover complementary skills: dialogue flow, exhaustive detail, and reasoning. Ablations in the paper confirm that using all three leads to the strongest results.
            
            \item \textbf{Why this setup?} At the time, no large public dataset of multimodal instructions existed. By repurposing image captions and object detections into prompts, and letting GPT--4 spin them into diverse instruction--response pairs, the authors created a scalable, stylistically consistent training set. This gave the LLM practice in “talking about images” without needing humans to hand-label every dialogue.
            
            \item \textbf{Training format.} The generated dialogues are wrapped in Vicuna’s chat template (system $\rightarrow$ user $\rightarrow$ assistant). During training, only the assistant tokens are used for the autoregressive loss, teaching the LLM to generate natural, instruction-following replies conditioned on the projected visual tokens.
        \end{itemize}
        
        \begin{figure}[H]
            \centering
            \includegraphics[width=0.7\textwidth]{Figures/Chapter_24/Llava_example_responses.jpg}
            \caption{\textbf{Instruction-following data construction (illustrative snippet).} Top: text-only \emph{contexts} (captions / boxes) shown to GPT; bottom: three \emph{response types}. The raw image is \emph{not} fed to GPT—only used for human reference in the paper. Source:\cite{liu2023_llava}.}
            \label{fig:chpapter24_llava_examples}
        \end{figure}
        
        \begin{figure}[H]
            \centering
            \includegraphics[width=0.7\textwidth]{Figures/Chapter_24/LLaVA_input_sequence.jpg}
            \caption{\textbf{Input sequence for training.} Visual tokens from $H_v$ are concatenated with dialogue tokens. The model learns to generate assistant answers and the stop symbol (\texttt{<STOP>=\#\#\#}) autoregressively; only assistant tokens contribute to the loss. Source:\cite{liu2023_llava}.}
            \label{fig:chpapter24_llava_input_sequence}
        \end{figure}
        
        \paragraph{Why It Works (vs.\ BLIP/BLIP-2)}
        \begin{itemize}
            \item \textbf{Connector simplicity.} LLaVA uses only a lightweight linear projector to map CLIP features into the LLM’s embedding space. This avoids the complexity of Q-Former cross-attention or gated modules, making training straightforward and connector overhead negligible.
            
            \newpage
            
            \item \textbf{Instruction-first supervision.} Instead of contrastive alignment or captioning, LLaVA trains directly on GPT-4--curated multimodal instructions. This supervision style teaches the LLM to \emph{follow instructions about images} (e.g., “what is unusual?”) rather than merely align embeddings. BLIP and BLIP-2 never explicitly target this behavior in pretraining.
            
            \item \textbf{Trade-offs vs.\ BLIP-2.} BLIP-2 keeps both CLIP and the LLM frozen and learns a small Q-Former bridge—highly compute-efficient and stable, with strong zero-shot priors. LLaVA instead \emph{fine-tunes the LLM} (together with the projector), investing in instruction-following fluency. The benefit is stronger conversational ability and instruction adherence; the drawback is weaker structured grounding and greater dependence on the synthetic instruction data distribution.
            
            \item \textbf{Inference cost.} Unlike BLIP-2’s compact query tokens, LLaVA passes a large number of projected CLIP grid features (hundreds of tokens at 336px input) into the LLM. This increases sequence length and thus slows inference, while also raising memory use. In practice, LLaVA trades some efficiency for richer supervision and stronger dialogue fluency.
        \end{itemize}
        
        \paragraph{Instruction Following and Reasoning (Qualitative)}
        
        \begin{figure}[H]
            \centering
            \includegraphics[width=0.7\linewidth]{Figures/Chapter_24/LLaVA_in_the_wild.jpg}
            \caption{\textbf{LLaVA-Bench (In-the-Wild): challenging, high-resolution cases with detailed human annotations.}
                The benchmark probes real-world capabilities beyond generic captioning, stressing OCR, fine-grained recognition, spatial reasoning, and knowledge grounding.
                \emph{Example 1 (left, ``ICHIRAN Ramen''):} requires reading small text in-the-wild (OCR) and linking it to world knowledge to answer queries such as \emph{``What’s the name of the restaurant?''}.
                \emph{Example 2 (right, ``Filled Fridge''):} demands locating fine-grained items, reading brand labels (e.g., \emph{Fage} variants), and reasoning over cluttered layouts to answer compositional questions (e.g., brand identification, presence/absence of a flavor).
                These cases illustrate why instruction-following VLMs must combine accurate text extraction, object/attribute recognition, and commonsense knowledge to succeed. Source:\cite{liu2023_llava}.}
            \label{fig:chpapter24_llava_in_the_wild}
        \end{figure}
        
        \begin{figure}[H]
            \centering
            \includegraphics[width=0.85\textwidth]{Figures/Chapter_24/LLaVA_hard_prompt.jpg}
            \caption{\textbf{Following instructions vs.\ scene description.} LLaVA answers “what is unusual?” with multi-step reasoning and safety considerations, outperforming BLIP-2 / OpenFlamingo on instruction adherence; GPT-4 is concise but less conversational. Source:\cite{liu2023_llava}.}
        \end{figure}
        
        \paragraph{Benchmarks: LLaVA-Bench, COCO ablations, In-the-Wild, ScienceQA}
        \begin{table}[H]
            \centering
            \small
            \setlength{\tabcolsep}{6pt}
            \caption{Ablation on LLaVA-Bench (COCO). Scores are relative to a text-only GPT-4 that sees ground-truth captions/boxes. Removing instruction tuning is catastrophic, highlighting its centrality. Source:\cite{liu2023_llava}.}
            \label{tab:llava_coco_ablation}
            \resizebox{0.90\linewidth}{!}{%
                \begin{tabular}{lcccc}
                    \toprule
                    \textbf{Training data} & \textbf{Conversation} & \textbf{Detail desc.} & \textbf{Complex reasoning} & \textbf{All} \\
                    \midrule
                    Full data                               & 83.1 & 75.3 & 96.5 & 85.1 \\
                    Detail + Complex                        & 81.5 (\,$\downarrow$1.6) & 73.3 (\,$\downarrow$2.0) & 90.8 (\,$\downarrow$5.7) & 81.9 (\,$\downarrow$3.2) \\
                    Conv + 5\% Detail + 10\% Complex        & 81.0 (\,$\downarrow$2.1) & 68.4 (\,$\downarrow$7.1) & 91.5 (\,$\downarrow$5.0) & 80.5 (\,$\downarrow$4.4) \\
                    Conversation                             & 76.5 (\,$\downarrow$6.6) & 59.8 (\,$\downarrow$16.2) & 84.9 (\,$\downarrow$12.4) & 73.8 (\,$\downarrow$11.3) \\
                    No Instruction Tuning                    & 22.0 (\,$\downarrow$61.1) & 24.0 (\,$\downarrow$51.3) & 18.5 (\,$\downarrow$78.0) & 21.5 (\,$\downarrow$63.6) \\
                    \bottomrule
            \end{tabular}}
        \end{table}
        
        \begin{table}[H]
            \centering
            \small
            \setlength{\tabcolsep}{6pt}
            \caption{Instruction-following comparison (relative scores) on LLaVA-Bench (In-the-Wild). Means $\pm$ std over three runs for the first three rows; for LLaVA$^\dagger$, GPT-4 is queried three times as judge. Source:\cite{liu2023_llava}.}
            \label{tab:llava_inthewild}
            \resizebox{0.85\linewidth}{!}{%
                \begin{tabular}{lcccc}
                    \toprule
                    \textbf{Method} & \textbf{Conversation} & \textbf{Detail desc.} & \textbf{Complex reasoning} & \textbf{All} \\
                    \midrule
                    OpenFlamingo~\cite{awadalla2023_openflamingo} & $19.3 \pm 0.5$ & $19.0 \pm 0.5$ & $19.1 \pm 0.7$ & $19.1 \pm 0.4$ \\
                    BLIP-2~\cite{li2023_blip2}                     & $54.6 \pm 1.4$ & $29.1 \pm 1.2$ & $32.9 \pm 0.7$ & $38.1 \pm 1.0$ \\
                    LLaVA~\cite{liu2023_llava}                     & $57.3 \pm 1.9$ & $52.5 \pm 6.3$ & $81.7 \pm 1.8$ & $67.3 \pm 2.0$ \\
                    LLaVA$^\dagger$~\cite{liu2023_llava}           & $58.8 \pm 0.6$ & $49.2 \pm 0.8$ & $81.4 \pm 0.3$ & $66.7 \pm 0.3$ \\
                    \bottomrule
            \end{tabular}}
        \end{table}
        
        \begin{table}[H]
            \centering
            \small
            \setlength{\tabcolsep}{4.5pt}
            \caption{ScienceQA~\cite{lu2022_scienceqa} accuracy (\%). NAT/SOC/LAN: domains; TXT/IMG/NO: context types; G1–6/G7–12: grade levels. $^\dagger$Text-only GPT-4 (our eval.). Source:\cite{liu2023_llava}.}
            \label{tab:llava_scienceqa}
            \resizebox{\linewidth}{!}{%
                \begin{tabular}{lccccccccc}
                    \toprule
                    \textbf{Method} & \textbf{NAT} & \textbf{SOC} & \textbf{LAN} & \textbf{TXT} & \textbf{IMG} & \textbf{NO} & \textbf{G1-6} & \textbf{G7-12} & \textbf{Average} \\
                    \midrule
                    \multicolumn{10}{l}{\emph{Representative \& SoTA numbers reported in literature}} \\
                    Human~\cite{lu2022_scienceqa}            & 90.23 & 84.97 & 87.48 & 89.60 & 87.50 & 88.10 & 91.59 & 82.42 & 88.40 \\
                    GPT-3.5~\cite{lu2022_scienceqa}          & 74.64 & 69.74 & 76.00 & 74.44 & 67.28 & 77.42 & 76.80 & 68.89 & 73.97 \\
                    GPT-3.5 w/ CoT~\cite{lu2022_scienceqa}   & 75.44 & 70.87 & 78.09 & 74.68 & 67.43 & 79.93 & 78.23 & 69.68 & 75.17 \\
                    LLaMA-Adapter~\cite{zhang2023_llama_adapter} & 84.37 & 88.30 & 84.36 & 83.72 & 80.32 & 86.90 & 85.83 & 84.05 & 85.19 \\
                    MM-CoT\textsubscript{Base}~\cite{zhang2024_mcot}  & 87.52 & 77.17 & 85.82 & 87.88 & 82.90 & 86.83 & 84.65 & 85.37 & 84.91 \\
                    MM-CoT\textsubscript{Large}~\cite{zhang2024_mcot} & 95.91 & 82.00 & 90.82 & 95.26 & 88.80 & 92.89 & 92.44 & 90.31 & 91.68 \\
                    \midrule
                    \multicolumn{10}{l}{\emph{Author runs}} \\
                    GPT-4$^\dagger$                            & 84.06 & 73.45 & 87.36 & 81.87 & 70.75 & 90.73 & 84.69 & 79.10 & 82.69 \\
                    LLaVA~\cite{liu2023_llava}                 & 90.36 & 95.95 & 88.00 & 89.49 & 88.00 & 90.66 & 90.93 & 90.90 & 90.92 \\
                    LLaVA+GPT-4$^\dagger$ (complement)         & 90.36 & 95.50 & 88.55 & 89.05 & 87.80 & 91.08 & 92.22 & 88.73 & 90.97 \\
                    LLaVA+GPT-4$^\dagger$ (judge)              & 91.56 & 96.74 & 91.09 & 90.62 & 88.99 & 93.52 & 92.73 & 92.16 & \textbf{92.53} \\
                    \bottomrule
            \end{tabular}}
        \end{table}
        
        \begin{table}[H]
            \centering
            \small
            \setlength{\tabcolsep}{10pt}
            \caption{Design ablations on ScienceQA (Average \%). Differences vs.\ best variant in parentheses. Source:\cite{liu2023_llava}.}
            \label{tab:llava_design_ablations}
            \resizebox{0.60\linewidth}{!}{%
                \begin{tabular}{lcc}
                    \toprule
                    \textbf{Visual features} & \textbf{Before} & \textbf{Last} \\
                    \midrule
                    Best variant            & 90.92 & 89.96 (\,$\downarrow$0.96) \\
                    Predict answer first    & --    & 89.77 (\,$\downarrow$1.15) \\
                    Training from scratch   & 85.81 (\,$\downarrow$5.11) & -- \\
                    7B model size           & 89.84 (\,$\downarrow$1.08) & -- \\
                    \bottomrule
            \end{tabular}}
        \end{table}
        
        \paragraph{What the Ablations Say (and How This Differs from BLIP-2)}
        \begin{itemize}
            \item \textbf{Instruction tuning is essential.} Removing it collapses performance (Table~\ref{tab:llava_coco_ablation}), confirming that \emph{formatting everything as natural-language instructions} teaches a reusable procedure for solving diverse tasks—similar insight to InstructBLIP, but achieved with a simpler connector.
            \item \textbf{Visual feature choice matters.} Using the \emph{right} CLIP layer (pre-/post-last) impacts downstream QA (Table~\ref{tab:llava_design_ablations}); BLIP-2 instead learns a task-aware \emph{query interface} (Q-Former), while 
            LLaVA must pick a fixed feature tap.
            
            \newpage
            
            \item \textbf{Training dynamics.} Starting “from scratch” (no Vicuna init) underperforms strongly, emphasizing the value of strong LLM priors—the same high-level lesson as BLIP-2, but LLaVA \emph{fine-tunes} the LLM, whereas BLIP-2 keeps it frozen and tunes only a small bridge.
        \end{itemize}
        
        \paragraph{Positioning vs.\ BLIP/BLIP-2}
        \begin{itemize}
            \item \textbf{Connector vs.\ Instruction Data.} BLIP-2 invests in a learned \emph{bridge} (Q-Former) to translate vision for a \emph{frozen} LLM; LLaVA keeps the bridge simple (linear) and invests in \emph{instruction data} + \emph{LLM fine-tuning}.
            \item \textbf{Efficiency.} BLIP-2 trains far fewer parameters (frozen experts), typically more stable and compute-efficient. LLaVA trains more on the language side (often with PEFT/LoRA), improving conversationality and adherence to instructions at the cost of more sensitivity to data curation.
            \item \textbf{Generalization.} BLIP-2’s priors excel in zero-shot retrieval/grounding with robust visual features; LLaVA often wins on instruction-following and free-form dialogue (LLaVA-Bench), but is more dependent on the prompt style and instruction distribution.
        \end{itemize}
        
        \paragraph{Limitations and Next Steps (segue to LLaVA-NeXT / OneVision)}  
        While \textsc{LLaVA}~\cite{liu2023_llava} proved that a frozen CLIP encoder plus an instruction-tuned LLM can deliver strong multimodal dialogue, it remains fundamentally \emph{image-centric}. Videos are only approximated by feeding multiple frames as separate images, which ignores motion and temporal dependencies. High-resolution images are globally resized, often losing small details, and the linear projector provides only a minimal bridge for multi-image reasoning.  
        
        \textsc{LLaVA-NeXT}~\cite{liu2024_llava_next} was introduced to address some of these gaps. It brought two notable upgrades:  
        \begin{itemize}  
            \item \textbf{AnyRes.} A tiling-and-merging strategy that allows images to be processed at near-native resolution without heavy downsampling, crucial for fine-grained perception such as OCR or small-object recognition.  
            \item \textbf{Multi-image interleaving.} A mechanism to encode several images jointly within a conversation, enabling set-level reasoning across multiple inputs.  
        \end{itemize}  
        Together, these improvements boosted LLaVA’s ability to handle high-resolution and multi-image tasks. However, \emph{NeXT still lacks native temporal modeling}: video frames are treated as a loose set of images with no explicit time encoding, motion cues, or sequence objectives.  
        
        This motivates the next step: \textsc{LLaVA-OneVision}~\cite{li2024_llavaonevision}. Instead of treating video as “many images,” it trains on a mixture of video and image data, introduces time-aware positional tokens and attention mechanisms, and strengthens visual token pooling to fit longer sequences within the LLM’s context budget. The result is a model that can \emph{natively} support video question answering while retaining the instruction-following strengths of its predecessors.  
        
        In short, NeXT overcame resolution and multi-image limits of LLaVA, but it is OneVision that finally closes the modality gap—moving from image-centric adaptation to unified handling of single images, multi-image sets, and full video clips.  
        
    \end{enrichment}
    
    \newpage
    
    \begin{enrichment}[LLaVA-OneVision: Unified Multimodal Transfer][subsection]
        \label{enr:subsec_chapter24_llava_onevision}
        
        \paragraph{From LLaVA to OneVision: Motivation \& Goal}
        \textsc{LLaVA} showed that a minimal pipeline---frozen vision encoder $\rightarrow$ lightweight projector $\rightarrow$ instruction-tuned LLM---can produce a strong \emph{image}-centric assistant. Yet three gaps remained: (i) set-level reasoning across \emph{multiple} images lacked structure, (ii) \emph{video} was only approximated by feeding frames as independent stills (no temporal modeling), and (iii) aggressive global resizing hurt high-resolution perception (OCR, diagrams, small objects). \textsc{LLaVA-OneVision}~\cite{li2024_llavaonevision} addresses these gaps with a single, \emph{unified} model that natively supports single-image, multi-image, and video inputs and encourages cross-scenario skill transfer, while preserving LLaVA’s minimalist spirit.
        
        \paragraph{High-Level Idea}
        OneVision keeps the simple connector-to-LLM philosophy but upgrades the visual pipeline and tokenization so that (a) high-resolution details are preserved, (b) token budgets remain balanced across modalities, and (c) temporal order/motion are modeled directly. A staged curriculum first aligns vision tokens to the LLM, then builds a strong single-image instruction follower, and finally mixes in multi-image \& video data to induce native temporal reasoning and cross-scenario transfer.
        
        \begin{figure}[H]
            \centering
            \includegraphics[width=0.85\textwidth]{Figures/Chapter_24/LLaVA_OneVision_architecture.jpg}
            \caption{\textbf{LLaVA-OneVision architecture.} Left: a concrete instantiation; Right: the generalized LLaVA form extended to support single images, multi-image sets, and video clips in one pipeline. \emph{Source:}~\cite{li2024_llavaonevision}.}
            \label{fig:chapter24_llava_onevision_architecture}
        \end{figure}
        
        \subsubsection*{Method}
        
        \paragraph{Architecture Overview (What changes vs.\ LLaVA)}
        \begin{itemize}
            \item \textbf{Vision encoder \& projector (minimal fusion, stronger backbone).}
            As in LLaVA, visual inputs are encoded by a pretrained \emph{vision tower} and mapped into the LLM token space by a small \emph{projector} (2-layer MLP / linear). OneVision upgrades the tower from CLIP to the \emph{SigLIP} family (typical input $384{\times}384$), chosen for its robust open-source zero-shot alignment and strong text-rich perception. The projector remains the only bespoke fusion block, so visual tokens can be simply prepended/interleaved with text—preserving LLaVA’s low-complexity path (no cross-attention/Q-Former).
            
            \newpage
            
            \item \textbf{Higher-resolution processing (\emph{Higher AnyRes}).}
            \textsc{OneVision} replaces global resizing with an adaptive \emph{tile $\rightarrow$ encode $\rightarrow$ merge} pipeline that preserves local detail \emph{and} keeps the visual sequence length predictable~\cite{li2024_llavaonevision}.
            \begin{enumerate}
                \item \emph{Tiling (detail + context).} From a high-resolution image, build two kinds of inputs: (i) a \emph{global view} obtained by uniformly resizing the full image to the vision encoder’s native resolution (e.g., $384{\times}384$), and (ii) an $a{\times}b$ grid of aspect-preserving \emph{local crops}, each also resized to the encoder’s native resolution. The global view provides coarse layout; the tiles preserve fine text/edges that a single global downscale would destroy.
                \item \emph{Encoding (shared backbone).} Feed the global view and each of the $a{\times}b$ crops independently through the frozen vision encoder (e.g., SigLIP), producing $T$ tokens per input (e.g., $T{=}729$ for a $384{\times}384$ input). The provisional visual-token length is
                \[
                L \;=\; (a\times b + 1)\,T,
                \]
                where the “$+1$” accounts for the global view.
                \item \emph{Budgeting \& concatenation (Higher AnyRes).} To keep sequence length predictable, impose a per-scenario token cap $\tau$. If $L\!>\!\tau$, \textbf{reduce the tokens per input} (global and each crop) by bilinearly interpolating their \emph{feature grids} before flattening:
                \[
                T_{\text{new}} \;=\; \Big\lfloor \tfrac{\tau}{a\times b + 1} \Big\rfloor,
                \quad\text{so that}\quad
                L_{\text{new}} \;=\; (a\times b + 1)\,T_{\text{new}} \le \tau.
                \]
                Finally, \textbf{concatenate} the token sequences from the global view and all crops in a fixed order to form the visual-token stream for the projector/LLM. No cross-scale feature blending is introduced; the “merge” is achieved by length-controlled per-input downsampling plus sequence concatenation. This preserves local detail (via tiles) and global context (via the base image) while preventing quadratic attention blow-up and keeping tokens consistent across inputs of widely varying native resolution~\cite{li2024_llavaonevision}.
            \end{enumerate}
            
            \noindent \textit{Why this works.} The \emph{global view} provides a coarse anchor of scene layout and long-range object relations, while the \emph{local crops} contribute high-frequency detail such as OCR strokes or small parts. Rather than fusing feature maps, the method simply concatenates tokens from both sources, ensuring that context and detail coexist in the same token stream. When the provisional length exceeds a per-scenario cap, \emph{bilinear interpolation} is applied at the feature-grid level to shrink tokens per input uniformly, preserving continuity while enforcing the budget. This guarantees that the LLM always receives \emph{consistent, high-fidelity} tokens across images of widely varying native resolutions, without quadratic blow-up or architectural changes. In practice, this directly addresses LLaVA’s high-resolution failure modes (e.g., documents, charts, dense UI screens) while keeping the downstream language model untouched~\cite{li2024_llavaonevision}.
            
            \begin{figure}[H]
                \centering
                \includegraphics[width=0.85\textwidth]{Figures/Chapter_24/LLaVA_OneVision_AnyRes.jpg}
                \caption{\textbf{Higher AnyRes vs.\ original AnyRes.} Upgraded tiling/merging with bilinear interpolation preserves high-resolution fidelity (top) compared to the original scheme (bottom), improving OCR and small-object recognition. \emph{Source:}~\cite{li2024_llavaonevision}.}
                \label{fig:chapter24_llava_onevision_anyres}
            \end{figure}
            
            \item \textbf{Balanced visual token budgets (cross-scenario parity).}
            To promote \emph{skill transfer} while respecting the LLM’s context, \textsc{OneVision} normalizes the number of visual tokens \emph{per scenario} to be of the same order, using a common “unit”: the token count from one SigLIP view at $384{\times}384$ (about $T{=}729$ tokens)~\cite{li2024_llavaonevision}. Let an image be tiled into an $a{\times}b$ grid (with an additional resized \emph{global} view, “$+1$”). The provisional length is
            \[
            L \;=\; (a{\times}b + 1)\, T.
            \]
            
            A per-\emph{scenario} threshold $\tau_s$ (with $s\!\in\!\{\text{single-image (SI)},\ \text{multi-image (MI)},\ \text{video (VID)}\}$) is enforced \emph{before} concatenating visual tokens with text. The goal is to keep SI, MI, and VID inputs in the same token range so no modality dominates the context~\cite{li2024_llavaonevision}. For a tiled single image with grid $a{\times}b$ plus one global view, the provisional length is
            \[
            L_{\text{SI}} \;=\; (a{\times}b + 1)\,T.
            \]
            For $m$ independent images in MI (each near the base unit), $L_{\text{MI}} \!\approx\! m\,T$. For a video with $f$ frames, $L_{\text{VID}} \!\approx\! f\,T_{\text{frame}}$ (with $T_{\text{frame}}$ obtained via feature-space interpolation per frame). If $L_s \!>\! \tau_s$, the loader applies uniform feature-grid downsampling so that each view contributes
            \[
            T_{\text{new}} \;=\; \Big\lfloor \frac{\tau_s}{N_s} \Big\rfloor
            \quad\text{with}\quad
            N_s \;=\;
            \begin{cases}
                a{\times}b+1 & (s=\text{SI})\\
                m & (s=\text{MI})\\
                f & (s=\text{VID})
            \end{cases}
            \]
            and thus $N_s\,T_{\text{new}} \le \tau_s$. Downsampling is implemented by bilinear pooling on the encoder feature maps (token grids), preserving spatial continuity while meeting the budget~\cite{li2024_llavaonevision}. Concretely:
            
            \begin{itemize}
                \item \emph{Single image (SI / High AnyRes).} Choose a small tile grid (e.g., global $+$ up to $3{\times}3$ crops). If $(a{\times}b{+}1)T \!>\! \tau_{\text{SI}}$, either reduce the grid (e.g., $3{\times}3\!\to\!2{\times}2$) or apply bilinear pooling so that each view contributes $\lfloor \tau_{\text{SI}}/(a{\times}b{+}1)\rfloor$ tokens~\cite{li2024_llavaonevision}.
                \item \emph{Multi-image (MI).} Admit up to $m{\le}12$ images. If $mT \!>\! \tau_{\text{MI}}$, uniformly shrink per-image token grids to $T_{\text{new}}\!=\!\lfloor \tau_{\text{MI}}/m\rfloor$, keeping the total comparable to a high-res single image~\cite{li2024_llavaonevision}.
                \item \emph{Video (VID).} Sample up to $f{\le}32$ frames. Per-frame features are pooled (e.g., $2{\times}2$ bilinear) to $\approx\!196$ tokens/frame; if $f\cdot 196 \!>\! \tau_{\text{VID}}$, reduce $f$ (FPS-aware sub-sampling) and/or increase pooling so $T_{\text{frame,new}}\!=\!\lfloor \tau_{\text{VID}}/f\rfloor$~\cite{li2024_llavaonevision}.
            \end{itemize}
            
            \noindent\textit{Why this helps.} Setting $\tau_{\text{SI}},\tau_{\text{MI}},\tau_{\text{VID}}$ to similar magnitudes yields \emph{cross-scenario parity}: tiled single images, multi-image sets, and short clips contribute comparable visual budgets. This prevents context monopolization, stabilizes SFT across modalities, and—crucially—makes a tiled image appear budget-wise like a short \emph{sequence}, encouraging routines (scan, compare, summarize) that transfer between SI, MI, and VID without changing the downstream language model~\cite{li2024_llavaonevision}.
            
            \begin{figure}[H]
                \centering
                \includegraphics[width=0.85\textwidth]{Figures/Chapter_24/LLaVA_OneVision_visual_representation_strategy.jpg}
                \caption{\textbf{Balanced visual token allocation across modalities.} OneVision caps tokens so single-image, multi-image, and video inputs receive comparable visual capacity (e.g., $\sim$729 tokens $\approx$ SigLIP at $384{\times}384$), preserving LLM context and encouraging cross-scenario transfer. \emph{Source:}~\cite{li2024_llavaonevision}.}
                \label{fig:chapter24_llava_onevision_token_strategy}
            \end{figure}
            
            \item \textbf{Temporal indexing \& attention (native video modeling).}
            Videos are represented as an \emph{ordered} sequence of frame tokens and rely on the LLM’s inherent sequence modeling for temporal understanding—no bespoke video module is introduced~\cite{li2024_llavaonevision}. Concretely:
            \begin{enumerate}
                \item \emph{Frame sampling \& features.} Sample up to 32 frames (FPS-aware for long clips). Each frame is resized to the encoder’s native resolution (e.g., $384{\times}384$), encoded by the frozen vision tower, then \emph{downsampled in feature space} using $2{\times}2$ bilinear interpolation to a per-frame budget of $\approx 196$ tokens (Appendix~C.1; Fig.~“Higher AnyRes”)~\cite{li2024_llavaonevision}.
                
                \newpage
                
                \item \emph{Implicit time via order (no extra temporal PE required).} The 2D spatial positional encodings from the vision encoder are retained; frames are \emph{concatenated in chronological order} into one stream, prefixed by a single \texttt{<image>} marker for the entire video (Appendix~C.2)~\cite{li2024_llavaonevision}. The paper does not introduce a separate learned or sinusoidal temporal embedding; the position in the sequence itself encodes time.
                \item \emph{Causal attention over the sequence.} The resulting token stream is fed to the LLM with standard causal/self-attention (lower-triangular masking), which preserves temporal direction. This enables queries such as ``what happens next,'' change detection, and event localization while keeping LLaVA’s minimalist fusion design intact~\cite{li2024_llavaonevision}.
            \end{enumerate}
            \noindent\textit{Why it works.} Chronological concatenation makes inter-frame differences \emph{addressable} through relative positions in the same attention space as language; causal flow lets the LLM compose motion narratives (e.g., “the door opens after the person reaches the handle”). Combined with the per-frame token budget ($\sim$196) and the cap on total frames (32), this provides stable compute and strong zero-/few-shot video QA via transfer from image training~\cite{li2024_llavaonevision}.
        \end{itemize}

        
        \paragraph{Training Curriculum (How capabilities are built)}
        \noindent The recipe follows a \emph{progressive curriculum} that first forges a clean interface between vision and language, then injects knowledge at higher visual fidelity, and finally teaches instruction-following across scenarios. Each stage increases difficulty in one axis at a time (trainable scope, resolution/token budget, task diversity), which stabilizes optimization and preserves pretrained priors~\cite{li2024_llavaonevision}.
        
        \begin{enumerate}
            
            \item \textbf{Stage~1: Language--Image Alignment (frozen experts).}
            \emph{What we do.} Freeze the vision encoder and the LLM; train only the lightweight projector on ${\sim}558\text{K}$ image--text pairs at the encoder’s native resolution (e.g., $384{\times}384$ $\Rightarrow$ $\approx729$ tokens per view).  
            \emph{Intuition.} Treat the projector like a \emph{translator} that learns the “alphabet” of visual features so the LLM can read them, without editing either expert’s hard-earned priors.  
            \emph{Why first.} Updating only a tiny module gives fast, stable alignment and avoids catastrophic forgetting; it also removes noisy gradients before scaling resolution or adding complex instructions.  
            \emph{What emerges.} Zero-shot basics (captioning, simple QA) with a clean, low-variance interface the later stages can safely build on~\cite{li2024_llavaonevision}.
            
            \item \textbf{Stage~1.5: High-Quality Knowledge Learning (full model, Higher AnyRes).}
            \emph{What we do.} Unfreeze the full stack (vision encoder, projector, LLM) and continue training on ${\sim}4\text{M}$ \emph{high-quality} single-image samples. In the meanwhile, it is done while enabling \emph{Higher AnyRes} (tile$\rightarrow$encode$\rightarrow$merge). Token budgets are increased progressively (e.g., up to $\sim5{\times}$ the base) so the model learns to cope with more detail without instability.  
            \emph{Intuition.} After the “alphabet”, this is \emph{reading widely}: infuse broad perceptual/world knowledge and teach the model to process dense inputs (documents, charts, UI screens) at near-native resolution.  
            \emph{Why now.} Once the interface is stable, end-to-end updates can safely propagate knowledge into both experts while AnyRes habituates the model to larger, but controlled, visual sequences.  
            \emph{What emerges.} Better grounding for fine text and small parts, plus robustness to resolution changes—fixing the original LLaVA’s global-resize bottleneck~\cite{li2024_llavaonevision}.
            
            \item \textbf{Stage~2: Visual Instruction Tuning (full model).}
            \emph{Goal.} Teach the model to \emph{follow instructions} across scenarios while keeping visual tokens within balanced budgets.
            
            \newpage
            
            \begin{itemize}
                \item \emph{2a: Single-Image SFT (3.2M).}  
                \emph{What we do.} Supervised instruction tuning on a curated single-image corpus covering general QA/captioning, docs/charts/screens (OCR-heavy), visual math/reasoning, and multilingual prompts. AnyRes is used when detail matters; token caps preserve room for the prompt and answers.  
                \emph{Intuition.} Like \emph{practicing conversations} before debates: establish reliable, step-by-step response behavior on the scenario with the richest data (images) and the widest skill coverage.  
                \emph{What emerges.} A strong, dependable image assistant—good habits in formatting, chain-of-thought style reasoning (when supervised), and grounding to visual evidence~\cite{li2024_llavaonevision}.
                
                \item \emph{2b: OneVision SFT (1.6M mixed).}  
                \emph{What we do.} Instruction tune on a \emph{balanced mixture} of \textit{multi-image + video + single-image} samples using the same connector and tokenization path. Multi-image sets and video clips are normalized to comparable visual-token budgets (e.g., up to $12$ images near the base unit; up to $32$ frames at $\sim196$ tokens/frame via $2{\times}2$ feature-space interpolation). Frames are ordered chronologically; the sequence is fed directly to the LLM~\cite{li2024_llavaonevision}.  
                \emph{Intuition.} This is \emph{cross-training}: by keeping budgets comparable, a tiled high-res image “looks like” a short sequence, and a frame sequence “looks like” an interleaved set—so the LLM reuses the same routines (scan, compare, summarize, localize changes).  
                \emph{Why mixed (not siloed).} Mixing prevents modality-specific overfitting and encourages \emph{transfer}: OCR skill from images helps in videos; “spot-the-difference” across images helps temporal change detection.  
                \emph{What emerges.} Native support for multi-image reasoning and temporal understanding (event order, causal queries) \emph{without} bespoke video modules, while retaining single-image strengths~\cite{li2024_llavaonevision}.
            \end{itemize}
        \end{enumerate}
        
        \noindent\textit{Takeaway.} The staged path—\emph{align} $\rightarrow$ \emph{enrich at higher resolution} $\rightarrow$ \emph{generalize via balanced, mixed instructions}—keeps training stable, preserves priors, and yields a single open model that natively handles single-image, multi-image, and video inputs using the same minimalist connector~\cite{li2024_llavaonevision}.
        
        \begin{figure}[H]
            \centering
            \includegraphics[width=0.7\textwidth]{Figures/Chapter_24/LLaVA_OneVision_stages.jpg}
            \caption{\textbf{Training stages and configurations.} Vision backbone, token budget, datasets, model scale, and hyperparameters per stage illustrate the curriculum from alignment to mixed-modality instruction tuning. \emph{Source:}~\cite{li2024_llavaonevision}.}
            \label{fig:chapter24_llava_onevision_stages}
        \end{figure}
        
        \paragraph{Data Collections (for SFT)}
        \textsc{LLaVA-OneVision} adopts a \emph{two-stage} instruction-tuning data design to first establish reliable single-image instruction-following “habits” and then extend those habits to multi-image and video. The sizing is deliberate: a large, diverse \emph{single-image} corpus (to saturate core skills and stabilize alignment) followed by a leaner, \emph{mixed-modality} corpus (to induce transfer without eroding single-image strength). This sequencing pairs naturally with the model recipe: Stage~2a focuses on breadth and fidelity under Higher AnyRes, while Stage~2b emphasizes cross-scenario generalization under \emph{balanced token budgets}~\cite{li2024_llavaonevision}.
        
        \begin{figure}[H]
            \centering
            \includegraphics[width=0.85\textwidth]{Figures/Chapter_24/LLaVA_OneVision_single_image_data.jpg}
            \caption{\textbf{Single-Image (3.2M) collection.} Left: category distribution (general QA/captioning, docs/charts/screens, math/reasoning, language, OCR). Right: dataset counts. Curated coverage builds a strong single-image instruction base before introducing multi-image/video. \emph{Source:}~\cite{li2024_llavaonevision}.}
            \label{fig:chapter24_llava_onevision_singleimage_data}
        \end{figure}
        
        \noindent\textbf{Why 3.2M single-image first?} Single images are the richest, cleanest supervision for teaching the model to \emph{follow instructions} while handling high-resolution details (documents, charts, UI, fine OCR) and structured reasoning (math, multi-step answers). This scale reduces sparsity across categories and prevents overfitting to any one task format. Practically, it lets the projector+LLM see consistent, high-fidelity tokens (via Higher AnyRes) across many domains, so later scenarios can \emph{reuse} these routines (scan, localize, read, reason) rather than learn them from scratch.
        
        \begin{figure}[H]
            \centering
            \includegraphics[width=0.85\textwidth]{Figures/Chapter_24/LLaVA_OneVision_OneVision_data.jpg}
            \caption{\textbf{OneVision mixed set (1.6M).} Left: distribution over \emph{multi-image}, \emph{video}, and \emph{single-image}; Right: dataset counts (``MI'' denotes multi-image variants). Mixed-modality SFT promotes coherent reasoning across images and over time under a shared token budget. \emph{Source:}~\cite{li2024_llavaonevision}.}
            \label{fig:chapter24_llava_onevision_mixed_data}
        \end{figure}
        
        \noindent\textbf{Why a smaller 1.6M mixed set next?} After consolidating single-image skills, the model is exposed to \emph{multi-image} (cross-view comparison, set reasoning) and \emph{video} (temporal ordering, change detection) while still seeing some single-image refreshers. Keeping this stage smaller maintains the single-image baseline and avoids “washing out” its high-resolution gains. Crucially, the mixed set is curated to align with \emph{modality-parity constraints}: a tiled high-res image, a small image set, and a short frame sequence occupy comparable visual-token budgets. This forces the LLM to apply \emph{shared} routines (scan $\rightarrow$ compare $\rightarrow$ summarize/locate changes) across modalities, which is the mechanism behind the observed cross-scenario transfer~\cite{li2024_llavaonevision}.
        
        \noindent\textit{Takeaway.} The 3.2M single-image corpus builds a dependable instruction-following core with high-res fidelity; the 1.6M mixed corpus then “teaches the model to generalize” by practicing the same routines across set and temporal inputs under a unified token budget. The figures summarize the scale and composition that make this two-stage strategy effective~\cite{li2024_llavaonevision}.
                    
        \subsubsection*{Experiments \& Ablations}
        
        \paragraph{What the Experiments Show}
        The experiments confirm that LLaVA-OneVision achieves \emph{state-of-the-art open-source performance across modalities}, rivaling GPT-4V on more than 70\% of benchmarks. For example, the 7B model reaches 56.8\% on MMMU (college-level multi-discipline QA), tying GPT-4V on this difficult evaluation~\cite{li2024_llavaonevision}. Several key insights emerge:
        
        \begin{itemize}
            \item \textbf{Unified capability.} Skills acquired in one modality transfer to others. Single-image (SI) models retain $\sim$90\% of full performance on ActivityNet-QA (video action QA), while mixed (SI+MI+video) SFT pushes multi-image VQA to 90.2\% (7B), $\sim$20 points higher than LLaVA-NeXT's interleaving baseline. Logical reasoning also scales: NLVR2 accuracy reaches 89.4\%, a $\sim$10\% gain over BLIP-2’s static image-only training.
            \item \textbf{Native video QA.} Sequential tokens with causal temporal attention outperform frame-interleaved baselines, improving VideoMME to 58.2\% (7B), a +6.3\% gain over GPT-4V (7B-equivalent). Balanced token budgets allow 32-frame clips without exploding inference cost, yielding strong conversational ratings (3.49/5) close to GPT-4V (4.06).
            \item \textbf{High-resolution perception.} AnyRes and Higher AnyRes yield significant OCR/text improvements: DocVQA rises to 87.5\% (+4.7\% over LLaVA’s low-res), while ChartQA climbs +3--5\%. On HierText, tiled inputs add +8\% retrieval recall, preserving fine text cues absent in downsampled BLIP-2/InstructBLIP.
        \end{itemize}
        
        Compared to priors:  
        \begin{itemize}
            \item \emph{LLaVA} excelled on static diagrams (e.g., 96.0\% ScienceQA) but had no native video ability; OneVision adds +10--20\% across video and multi-image benchmarks.  
            \item \emph{BLIP-2 / InstructBLIP} rely on Q-Former bridges for static image-text; they achieve $\sim$90\% ScienceQA but collapse on dynamics. OneVision surpasses by +5--10\% on video and multi-image QA without Q-Former overhead.  
            \item \emph{SigLIP} provided stronger image-text alignment (e.g., 79.1\% ImageNet zero-shot); OneVision builds on this, adding +2--3\% gains in document/ocr tasks.  
        \end{itemize}
        
        \newpage
        
        \paragraph{Ablation Themes (High-Level)}
        The ablations clarify which design choices matter most:
        
        \begin{itemize}
            \item \textbf{Resolution handling.} Removing Higher AnyRes drops text/small-object performance by 3--5\% (e.g., TextVQA, ChartQA). Gains are sharpest on dense, text-heavy data (e.g., +8\% on HierText). This validates tiling + global fusion as superior to the global resize used in LLaVA or BLIP-2.
            \item \textbf{Token budget balance.} Over-allocation to one modality reduces transfer (–5--10\%). Balanced allocations (e.g., 12 images $\approx$ 32 frames) stabilize training and improve generalization (+8--15\% on MuirBench, a multi-image benchmark). This enforces shared reasoning routines, unlike unconstrained LLaVA sequences.
            \item \textbf{Curriculum.} Three-stage training (alignment $\rightarrow$ knowledge $\rightarrow$ instruction) outperforms end-to-end by +10--15\% in convergence speed and stability. Mixed SI+MI+video SFT is essential: SI-only retains $\sim$90\% on ActivityNet-QA, but adding video boosts EgoSchema (egocentric QA) by +10\%. This extends InstructBLIP’s instruction tuning to dynamics without heavy architectural additions.
        \end{itemize}
        
        \paragraph{Qualitative Capabilities (Selected Examples)}
        
        \begin{figure}[H]
            \centering
            \includegraphics[width=0.85\textwidth]{Figures/Chapter_24/LLaVA_OneVision_understanding_charts_and_diagrams.jpg}
            \caption{\textbf{Across-image diagram/table understanding.} The model synthesizes evidence across multiple diagrams/tables (e.g., cross-referencing axes, legends, and cells) to answer compositional questions—illustrating robust \emph{multi-image transfer} beyond single-image captioning. \emph{Source:}~\cite{li2024_llavaonevision}.}
            \label{fig:chapter24_llava_onevision_charts}
        \end{figure}
        
        \begin{figure}[H]
            \centering
            \includegraphics[width=0.85\textwidth]{Figures/Chapter_24/LLaVA_OneVision_understanding_for_agents.jpg}
            \caption{\textbf{Agentic reasoning on UIs.} Given several phone screenshots, the model plans step-wise actions (e.g., tap/scroll/type) grounded in on-screen text and layout, demonstrating instruction following that bridges \emph{vision $\rightarrow$ action suggestions}. \emph{Source:}~\cite{li2024_llavaonevision}.}
            \label{fig:chapter24_llava_onevision_agents}
        \end{figure}
        
        \begin{figure}[H]
            \centering
            \includegraphics[width=0.85\textwidth]{Figures/Chapter_24/LLaVA_OneVision_set_of_mark_prompting.jpg}
            \caption{\textbf{Set-of-mark prompting.} The model uses numbered marks to localize and describe fine-grained regions (e.g., “mark~3 is a pressure gauge”), enabling precise references without extra detection heads. \emph{Source:}~\cite{li2024_llavaonevision}.}
            \label{fig:chapter24_llava_onevision_setofmark}
        \end{figure}
        
        \begin{figure}[H]
            \centering
            \includegraphics[width=0.85\textwidth]{Figures/Chapter_24/LLaVA_OneVision_image_to_video_editing.jpg}
            \caption{\textbf{Image-to-video prompt transfer.} From a static image, the model drafts detailed, temporally-aware prompts for video generation/editing (e.g., motions, transitions, camera moves), showcasing \emph{image$\rightarrow$video} transfer of high-level intent. \emph{Source:}~\cite{li2024_llavaonevision}.}
            \label{fig:chapter24_llava_onevision_img2vid_edit}
        \end{figure}
        
        \begin{figure}[H]
            \centering
            \includegraphics[width=0.85\textwidth]{Figures/Chapter_24/LLaVA_OneVision_video_to_video_diff.jpg}
            \caption{\textbf{Video-to-video difference (same start, different endings).} The model contrasts two clips that share an opening but diverge later, identifying \emph{when} and \emph{how} outcomes differ—evidence of native temporal reasoning. \emph{Source:}~\cite{li2024_llavaonevision}.}
            \label{fig:chapter24_llava_onevision_vid2vid_diff_a}
        \end{figure}
        
        \begin{figure}[H]
            \centering
            \includegraphics[width=0.85\textwidth]{Figures/Chapter_24/LLaVA_OneVision_video_to_video_diff2.jpg}
            \caption{\textbf{Video-to-video difference (similar background, different foreground).} With background held constant across clips, the model focuses on foreground actors/objects to explain semantic changes—probing \emph{foreground-aware} temporal understanding. \emph{Source:}~\cite{li2024_llavaonevision}.}
            \label{fig:chapter24_llava_onevision_vid2vid_diff_b}
        \end{figure}
        
        \begin{figure}[H]
            \centering
            \includegraphics[width=0.85\textwidth]{Figures/Chapter_24/LLaVA_OneVision_video_understanding_self_driving.jpg}
            \caption{\textbf{Multi-camera driving videos.} The model integrates synchronized views (front/side/rear) to explain traffic participants and events, reflecting \emph{multi-view + temporal} fusion useful for autonomy-style reasoning. \emph{Source:}~\cite{li2024_llavaonevision}.}
            \label{fig:chapter24_llava_onevision_selfdriving}
        \end{figure}
        
        \begin{figure}[H]
            \centering
            \includegraphics[width=0.85\textwidth]{Figures/Chapter_24/LLaVA_OneVision_sub_video_understanding.jpg}
            \caption{\textbf{Composed sub-videos.} The model narrates a timeline formed by ordered sub-clips, keeping track of entities and transitions—showcasing long-range \emph{event composition} rather than frame-level description. \emph{Source:}~\cite{li2024_llavaonevision}.}
            \label{fig:chapter24_llava_onevision_subvideos}
        \end{figure}
        
        \begin{figure}[H]
            \centering
            \includegraphics[width=0.85\textwidth]{Figures/Chapter_24/LLaVA_OneVision_referring_image_in_video_understanding.jpg}
            \caption{\textbf{Referring image \& video understanding.} Given a reference image, the model grounds identities in a target video (presence/absence, re-identification), unifying \emph{multi-image linkage} with \emph{temporal tracking}. \emph{Source:}~\cite{li2024_llavaonevision}.}
            \label{fig:chapter24_llava_onevision_referring}
        \end{figure}
        
        \newpage
        
        \subsubsection*{Limitations \& Future Work}
        
        \paragraph{Current Constraints}
        While \textsc{LLaVA-OneVision} (OV) set a new bar for open, unified multimodal models, its paper acknowledges several bottlenecks that limit generalization and accessibility:
        \begin{itemize}
            \item \textbf{Compute \& data appetite.} Training OV-72B required a multi-stage curriculum over $\sim$9M samples, consuming weeks on large GPU clusters. This high entry cost restricted reproducibility and open adoption.
            \item \textbf{Region semantics.} AnyRes tiling preserved global context and high-res detail, but lacked explicit region-level modeling. As a result, dense OCR and document tasks plateaued (e.g., DocVQA 87.5\%, $\approx$5\% behind GPT-4o).
            \item \textbf{Context budget.} Although token parity across scenarios (single image $\sim$ multi-image $\sim$ short video) stabilized training, the fixed LLM window still capped long-form video and ultra-high-res image reasoning.
            \item \textbf{Complex multimodal chat.} Even at 72B scale, OV left a “relatively larger gap” in nuanced, conversational visual chat compared to proprietary models like GPT-4o.
        \end{itemize}
        
        \paragraph{Directions and the Move to OV-1.5}
        The follow-up \textsc{LLaVA-OneVision-1.5}~\cite{an2025_llavaonevision15} was designed explicitly to overcome these issues, while retaining OV’s unified paradigm:
        \begin{itemize}
            \item \textbf{Efficient, open training.} OV-1.5 was trained from scratch under a $\$16$k compute budget via offline data packing and hybrid parallelism, compressing the pipeline to $\sim$1 week on 128$\times$A800s. This democratizes access, making large-scale multimodal training reproducible for the community.
            \item \textbf{Richer vision front-ends.} OV’s AnyRes encoder was replaced with \emph{RICE-ViT}, a region-aware backbone pretrained on 450M images / 2.4B regions. This improves fine-grained semantics, boosting OCRBench by +5\% (80.0\%) and DocVQA to 95.0\%, narrowing the gap with GPT-4o.
            \item \textbf{Balanced, large-scale data.} OV-1.5 introduced an 85M concept-balanced pretrain set and a 22M instruction set, covering broader domains and reducing biases. Ablations show +5–10\% gains on multi-discipline QA (e.g., MMBench) compared to OV’s original 9M.
            \item \textbf{Performance at smaller scale.} New 4B/8B models (Qwen3 backbone) surpass Qwen2.5-VL-7B on 18/27 benchmarks, with the 4B even beating Qwen2.5-VL-3B on all 27. This means OV-1.5 achieves parity with or beyond proprietary closed models at a fraction of cost.
        \end{itemize}
        
        \paragraph{Future Directions}
        Looking ahead, several paths are clear:
        \begin{itemize}
            \item \textbf{Extending RICE-ViT.} Adding temporal encoders or adaptive token selection to RICE-ViT could further lift long-video QA and dense OCR tasks, beyond OV-1.5’s gains.
            \item \textbf{Scaling with balance.} Expanding the 85M pretraining corpus to hundreds of millions of multimodal samples—while preserving concept balance—would improve cross-domain generalization and reduce bias.
            \item \textbf{Tool grounding.} Integrating external OCR engines, retrieval, or diagram solvers offers a hybrid route to bridge factuality gaps in specialized domains like math, forms, or charts.
        \end{itemize}
        
        \noindent In sum, \textsc{LLaVA-OneVision} introduced a unified recipe, but remained compute-heavy and limited in fine-grained reasoning. \textsc{LLaVA-OneVision-1.5} directly addressed these pain points with efficient training, region-aware vision, and balanced data scaling, providing a stronger open foundation and paving the way toward even richer multimodal reasoning at scale.
        
    \end{enrichment}
    
\end{enrichment}

\newpage

\begin{enrichment}[Large-Scale Video Foundation Models][section]
    \label{enr:sec_chapter24_foundation_models}
    Web-scale pretraining yields general-purpose video encoders usable across recognition, detection, and retrieval. We outline \emph{InternVideo} \cite{wang2022_internvideo} and \emph{InternVideo2} \cite{wang2024_internvideo2}—scaling data, architectures, and objectives—and \emph{OmniVL} \cite{wang2022_omnivl}, a unified image–video–language model. Related and emerging directions span mixture-of-experts backbones, multi-resolution clip sampling, and unified pretraining across video and audio.
    
    \begin{enrichment}[InternVideo: General Video Backbones][subsection]
        \label{enr:subsec_chapter24_internvideo}
        
        \paragraph{Scope and positioning}
        InternVideo~\cite{wang2022_internvideo} is a large-scale \emph{video foundation} recipe that couples \textbf{generative masked video modeling} with \textbf{discriminative multimodal contrastive learning} and \textbf{coordinates} the two through a lightweight \textbf{Cross-Model Attention (CMA)} head. The design yields a representation that transfers broadly to action understanding, video–language alignment (supervised and zero-shot), and open-world video tasks, surpassing both specialized and prior foundation baselines.
        
        \begin{figure}[H]
            \centering
            \includegraphics[width=0.85\textwidth]{Figures/Chapter_24/InternVideo_SOTA_comparisons.jpg}
            \caption{\textbf{SOTA overview.} InternVideo delivers the best performance on extensive video-related tasks, compared with specialized~\cite{feichtenhofer2019_slowfast,lin2019_tsm,arnab2021_vivit,yan2022_mtv} and foundation models~\cite{radford2021_clip,li2022_uniformerv2,zellers2022_merlotreserve}. Abbreviations: v2t/t2v retrieval, STA, FHP, NLQ, SCOD, MQ. Figure adapted from \cite{wang2022_internvideo}.}
            \label{fig:chapter24_internvideo_sota}
        \end{figure}
        
        \newpage
        
        \subsubsection{Motivation}
        \label{subsubsec_chapter24_internvideo_motivation}
        Large-scale video pretraining has been led by two complementary paradigms:
        
        \begin{itemize}
            \item \textbf{Masked Video Modeling (MVM).} Epitomized by \emph{VideoMAE} (see Sec.~\ref{enr:subsec_chapter24_videomae}), MVM reconstructs heavily masked spatiotemporal tubelet tokens without labels to learn motion-aware representations that transfer well to action understanding and localization (e.g., \emph{TAL}: Temporal Action Localization; \emph{STA}: Spatio-Temporal Action localization). However, because it lacks explicit \emph{language grounding}, it does not natively support video–language tasks such as \emph{VQA} (Video Question Answering) or cross-modal retrieval \emph{T2V/V2T} (Text-to-Video / Video-to-Text), unless sizable supervised heads or additional alignment training are introduced. 
            \item \textbf{Vision--Language Contrastive Learning (CLIP-style).} This paradigm aligns a video encoder with a text encoder via an InfoNCE objective, thereby endowing models with semantics and strong zero-shot transfer on video–language tasks like \emph{VQA} and cross-modal retrieval \emph{T2V/V2T}, and also aiding instruction-following settings such as \emph{VLN} (Vision-and-Language Navigation). Yet when an \emph{image}-pretrained ViT is naïvely extended to video, temporal structure can be under-exploited, weakening fine motion modeling and long-range dynamics for core video understanding and localization tasks (e.g., \emph{TAL}/\emph{STA}). 
        \end{itemize}
        
        \noindent
        InternVideo addresses these complementary gaps by separately pretraining a strong \emph{masked video encoder} (for motion/appearance coherence) and a strong \emph{multimodal video encoder} (for language-aligned semantics), and then coordinating them at adaptation time through a lightweight \emph{Cross-Model Attention (CMA)} fusion. In the \emph{intermediate} CMA modules, \textbf{masked-video (VideoMAE) spatiotemporal tokens serve as Queries}, while \textbf{multimodal (video–language) tokens provide Keys and Values}, transferring semantic context into the masked path. In the \emph{final} CMA module, the \textbf{multimodal class token serves as the Query} over \textbf{masked-video tokens as Keys/Values}, yielding an enriched video–language token used for prediction~\cite{wang2022_internvideo}.
        
        \medskip
        \noindent
        \textbf{What InternVideo solves and how.} InternVideo~\cite{wang2022_internvideo} proposes a \emph{dual-path} recipe plus a \emph{lightweight coordination} mechanism that combine the strengths of both worlds:
        \begin{itemize}
            \item \textbf{Two specialized pretraining paths.} A \emph{masked video encoder} is trained generatively (as in VideoMAE; Sec.~\ref{enr:subsec_chapter24_videomae}) to capture spatiotemporal dynamics; in parallel, a \emph{multimodal video encoder} with a text encoder is trained discriminatively with contrastive (and captioning) objectives to acquire language-grounded semantics~\cite{wang2022_internvideo}. Pretraining the two paths \emph{separately} avoids the optimization friction of joint, multi-loss training.
            \item \textbf{Coordination at adaptation time.} After pretraining, InternVideo \emph{freezes} the two encoders and learns a small \emph{Cross-Model Attention (CMA)} head that lets the multimodal encoder’s class token \emph{query} fine-grained tokens from the masked encoder. This fuses semantic abstraction with detailed motion/appearance evidence, improving over MVM-only approaches (e.g., VideoMAE/VideoMAEv2) by adding language awareness and over contrastive-only models by injecting robust temporal cues~\cite{wang2022_internvideo}.
            \item \textbf{A stronger video backbone on the multimodal path.} To ensure the multimodal branch is \emph{temporally competent}, InternVideo adopts UniFormer~\cite{li2022_uniformer} / UniFormerV2~\cite{li2022_uniformerv2} as the vision backbone, which explicitly handle temporal redundancy and long-range space--time dependencies while preserving powerful image-pretrained priors (details below). These preliminaries are essential because they explain \emph{why} the multimodal path already outputs motion-aware visual tokens that align well with text and \emph{how} CMA can then query complementary, reconstruction-trained tokens from the masked path.
        \end{itemize}
        
        \begin{figure}[H]
            \centering
            \includegraphics[width=0.85\textwidth]{Figures/Chapter_24/InternVideo_framework.jpg}
            \caption{\textbf{Unified framework.} Dual pretraining pathways (masked video reconstruction and multimodal contrastive learning) are coordinated via CMA for broad downstream transfer. Adapted from \cite{wang2022_internvideo}.}
            \label{fig:chapter24_internvideo_framework}
        \end{figure}
        
        \subsubsection{Preliminaries: UniFormer and UniFormerV2}
        \label{subsubsec_chapter24_uniformer}
        
        \paragraph{Why these preliminaries matter here.}
        Readers familiar with VideoMAE (Sec.~\ref{enr:subsec_chapter24_videomae}) already understand the \emph{masked} path that InternVideo pretrains generatively. The \emph{multimodal} path must pair clean language alignment with a video backbone that satisfies the following.
        
        \begin{itemize}
            \item \textbf{Early temporal efficiency:} Suppress short-range temporal redundancy early to control compute while retaining motion cues. 
            \item \textbf{Long-range coherence:} Preserve strong long-range space--time reasoning so actions and events remain coherent over many frames. 
            \item \textbf{Stable reuse of image priors:} Reuse powerful image-pretrained ViT priors without destabilizing them, enabling strong spatial semantics and rapid convergence. 
        \end{itemize}
        
        UniFormer~\cite{li2022_uniformer} and UniFormerV2~\cite{li2022_uniformerv2} meet these requirements, explaining why InternVideo’s contrastive branch is already motion-aware before CMA fusion and why CMA can effectively transfer semantics and dynamics between the two pretrained branches.
        
        \paragraph{UniFormer (CVPR’22)~\cite{li2022_uniformer}}
        \textbf{Block structure.} Given a clip token tensor \(\bm{X}_{\text{in}}\!\in\!\mathbb{R}^{C\times T\times H\times W}\), a UniFormer block applies Dynamic Position Embedding (DPE), Multi-Head Relation Aggregator (MHRA), and an FFN with residuals:
        \begin{align}
            \bm{X} &= \mathrm{DPE}(\bm{X}_{\text{in}}) + \bm{X}_{\text{in}}, \label{eq:uni_dpe}\\[1mm]
            \bm{Y} &= \mathrm{MHRA}(\mathrm{Norm}(\bm{X})) + \bm{X}, \label{eq:uni_mhra_uniformer}\\[1mm]
            \bm{Z} &= \mathrm{FFN}(\mathrm{Norm}(\bm{Y})) + \bm{Y}. \label{eq:uni_ffn_uniformer}
        \end{align}

        For relation learning, the spatiotemporal grid \((T,H,W)\) is flattened into a token sequence \(\bm{X}\in\mathbb{R}^{L\times C}\) with \(L=T\!\times\!H\!\times\!W\). A UniFormer block then proceeds in the fixed order
        \[
        \text{DPE} \;\longrightarrow\; \text{MHRA} \;\longrightarrow\; \text{FFN},
        \]
        with residual connections and normalization at each step. \emph{Why this order?} DPE first injects local, learnable spatiotemporal bias so tokens ``know’’ relative offsets; MHRA then aggregates context using either cheap local relations (early) or expressive global attention (late); the FFN finally refines per-token channels. This mirrors the progression from \emph{biasing} \(\to\) \emph{mixing} \(\to\) \emph{refining}, which is stable and compute-efficient for video.
        
        \paragraph{Dynamic Position Embedding (DPE): learnable relative spatiotemporal bias.}
        DPE applies a depthwise 3D convolution to the token grid and adds it back as a residual:
        \begin{align}
            \bm{X} &= \mathrm{DPE}(\bm{X}_{\text{in}}) + \bm{X}_{\text{in}}, \qquad
            \mathrm{DPE}(\bm{X}_{\text{in}})=\mathrm{DWConv}(\bm{X}_{\text{in}}).
            \label{eq:uni_dpe_def}
        \end{align}
        \emph{Intuition.} Unlike absolute (or fixed sinusoidal) position embeddings that inject static coordinates, DPE \emph{transforms} features with learned \(3{\times}3{\times}3\) per-channel kernels, encoding \emph{relative} offsets in time and space. Because the convolution is depthwise, each channel learns its own stencil: motion-sensitive channels can emphasize temporal neighbors \((\Delta t=\pm 1)\), while appearance channels can emphasize spatial neighbors \((\Delta h,\Delta w)\). This yields translation-friendly, length-agnostic positional cues and lets different channels specialize without interfering. DPE thus “primes’’ tokens with local geometry before any relation mixing.
        
        \emph{Why not just add absolute time/space codes? What is gained by DPE?}
        \begin{itemize}
            \item \textbf{Robust to augmentations.} Absolute codes are brittle under temporal cropping, frame-rate changes, and resizing; DPE learns \emph{relative} offsets that transfer across clip lengths and sampling strides. 
            \item \textbf{Local early cues.} Early layers chiefly need local position signals (edges, small motions). A learned \(3\mathrm{D}\) stencil injects these directly without hard-coding coordinates. 
            \item \textbf{Channel-wise specialization.} Depthwise filters let different channels emphasize temporal or spatial offsets (or anisotropic mixes), which a single shared absolute code cannot provide. 
            \item \textbf{Subtle absolute hints.} Zero padding at boundaries yields weak “start/end’’ cues while keeping the representation predominantly relative. 
        \end{itemize}
        
        \paragraph{MHRA (general form): one template that adapts with depth.}
        After normalization, MHRA aggregates context per head via an affinity matrix \(\bm{A}_n\) and value projection \(\bm{V}_n\):
        \begin{align}
            \bm{Y} &= \mathrm{MHRA}(\mathrm{Norm}(\bm{X})) + \bm{X}, \label{eq:uni_mhra} \\[-1mm]
            \bm{R}_n(\bm{X}) &= \bm{A}_n\,\bm{V}_n(\bm{X}), \qquad
            \mathrm{MHRA}(\bm{X})=\mathrm{Concat}\big(\bm{R}_1;\dots;\bm{R}_N\big)\,\bm{U},\ \bm{U}\in\mathbb{R}^{C\times C}.
            \label{eq:uni_mhra_general}
        \end{align}
        \emph{Intuition.} Each head chooses \emph{where} to look through \(\bm{A}_n\) and \emph{what} to bring through \(\bm{V}_n\). The same template becomes either \emph{local} (kernel-like) or \emph{global} (self-attention) by instantiating \(\bm{A}_n\) differently.
        
        \paragraph{MHRA—Local (shallow stages): cheap neighborhood mixing.}
        In early layers, \(\bm{A}_n\) is a learnable tubelet kernel restricted to a small neighborhood \(\Omega^{t\times h\times w}_i\) around token \(i\):
        \begin{equation}
            A_{n}^{\mathrm{local}}(\bm{X}_i,\bm{X}_j)=a^{\,n}_{\,i-j}, \qquad j\in\Omega^{t\times h\times w}_i,\quad a^{\,n}\in\mathbb{R}^{t\times h\times w}.
            \label{eq:uni_local}
        \end{equation}
        
        \newpage
        
        \emph{MHRA—Local - why here and why effective.}
        \begin{itemize}
            \item \textbf{From quadratic to (near) linear cost.} Full self-attention at shallow depth compares all token pairs and costs \(\mathcal{O}(L^2)\) with \(L=T\!\times\!H\!\times\!W\), which is prohibitively large before any downsampling. Constraining interactions to a fixed local tube \(\Omega\) of size \(t\!\times\!h\!\times\!w\) replaces \(\mathcal{O}(L^2)\) with \(\mathcal{O}(L\cdot thw)\). Since \(t,h,w\) are small constants, this is effectively \(\mathcal{O}(L)\), cutting both compute and memory drastically.            
            \item \textbf{Matches the signal statistics in video.} Adjacent frames and neighboring patches are highly redundant in early layers; most useful cues are short-range (edges, micro-motions). A local stencil aggregates exactly these signals without paying for far-away comparisons that rarely help at low-level stages.
            \item \textbf{Efficient, learnable, and specialized.} The kernel \(a^{n}\) is learned end-to-end and reused at every location, behaving like a per-head depthwise 3D convolution (akin to a PW–DW–PW block). Different heads can specialize to distinct temporal spans or spatial orientations, capturing short hand trajectories, lip motions, or local texture changes with minimal overhead.
        \end{itemize}
        
        \noindent
        \emph{Takeaway.} Local MHRA provides the right tool at the right place: it compresses short-range redundancy and builds robust low-level spatiotemporal features at (near) linear cost, reserving expensive global reasoning for deeper layers where sequence length is smaller and semantics are richer.
        
        \paragraph{MHRA—Global (deep stages): full space–time self-attention when it counts.}
        In deeper layers, \(\bm{A}_n\) becomes content-adaptive self-attention over all tokens:
        \begin{equation}
            A_{n}^{\mathrm{global}}(\bm{X}_i,\bm{X}_j)=
            \frac{\exp\big(Q_n(\bm{X}_i)^\top K_n(\bm{X}_j)\big)}
            {\sum_{j'\in\Omega^{T\times H\times W}}\exp\big(Q_n(\bm{X}_i)^\top K_n(\bm{X}_{j'})\big)}.
            \label{eq:uni_global}
        \end{equation}
        \emph{Why later.} Global attention is quadratic in \(L\), but by the time features are deep and (typically) downsampled, \(L\) is smaller and semantics are richer. This is when modeling long-range dependencies—linking distant frames, disambiguating similar motions via scene context, tracking multi-object interactions—pays off most, matching the ``cheap local early, expressive global late’’ principle from efficient video networks.
        
        \paragraph{FFN: per-token refinement.}
        A position-wise MLP (with expansion and contraction) follows to refine channels:
        \begin{equation}
            \bm{Z}=\mathrm{FFN}(\mathrm{Norm}(\bm{Y})) + \bm{Y}.
            \label{eq:uni_ffn}
        \end{equation}
        \emph{Role.} MHRA mixes \emph{between} tokens; FFN mixes \emph{within} a token’s channels to build nonlinear feature compositions (e.g., fusing motion edges and object cues).
        
        \paragraph{Putting it together: why this staging works for video.}
        \begin{itemize}
            \item \textbf{Bias then mix.} DPE injects learnable relative geometry so tokens carry local spatiotemporal context before any aggregation, improving stability and translation-friendliness. 
            \item \textbf{Local then global.} Local MHRA removes short-range redundancy inexpensively and locks onto micro-motion early; global MHRA later provides clip-level reasoning exactly where semantic abstraction and reduced \(L\) make it most useful and affordable. 
            \item \textbf{Refine per token.} The FFN consolidates mixed evidence into compact, discriminative channels, preparing features for the next stage or for global fusion blocks downstream. 
        \end{itemize}

        \paragraph{Concrete cue.}
        Consider ``\emph{opening a door}’’—a pattern combining subtle, short-range hand–handle interactions with a larger, long-range door-panel swing. UniFormer’s staging processes this in three steps:
        
        \begin{itemize}
            \item \textbf{DPE — inject relative spatiotemporal order.} Motion-sensitive channels privilege temporal neighbors (capturing tiny wrist twists), while edge/texture channels privilege spatial neighbors (sharpening handle contours). This primes tokens with local geometry before relation mixing. 
            \item \textbf{Local MHRA + FFN — assemble robust local cues at low cost.} Local MHRA mixes only within a small 3D neighborhood, stitching a few-frame wrist–handle trajectory with nearby edges at near-linear cost; the subsequent FFN performs pointwise nonlinear refinement, amplifying the fused “grasp–twist’’ cue and suppressing noise. 
            \item \textbf{Global MHRA + FFN — resolve long-range semantics.} Deep global MHRA performs full space–time attention, linking the refined local cue to door-panel motion across the clip and to body pose evolution; a final FFN consolidates this global evidence into a discriminative token. 
        \end{itemize}
        
        \noindent \textbf{Outcome.} The representation separates ``\emph{opening a door}’’—sustained forward rotation and panel displacement—from lookalikes such as ``\emph{touching a handle}’’ (no panel displacement) or ``\emph{closing a door}’’ (opposite temporal signature), with early stages handling redundant micro-dynamics efficiently and later stages resolving long-range semantics accurately.
        
        \begin{figure}[H]
            \centering
            \includegraphics[width=0.85\textwidth]{Figures/Chapter_24/Uniformer_architecture.jpg}
            \caption{\textbf{UniFormer architecture.} A UniFormer block combines DPE, MHRA, and FFN. Early blocks employ local MHRA; deeper blocks employ global MHRA to capture long-range space--time dependencies. Adapted from \cite{li2022_uniformer}.}
            \label{fig:chapter24_uniformer_arch}
        \end{figure}
        
        \paragraph{From UniFormer (V1): what we gained, and what still needs fixing.}
        \begin{itemize}
            \item \textbf{What V1 achieved.} It unified cheap \emph{local} relations early with expressive \emph{global} attention late (via MHRA + DPE), cutting shallow-layer cost, reducing redundancy, and improving long-range reasoning for actions—yielding a strong accuracy–FLOPs trade-off for video transformers. 
            \item \textbf{What V1 lacked.} As a new backbone, it did not reuse powerful image-pretrained ViT weights; obtaining robust spatial priors required separate, sizable supervised image pretraining before video adaptation, increasing engineering and compute overhead. 
            \item \textbf{What V2 must fix next.} Keep V1’s local\(\rightarrow\)global strengths while \emph{plugging into} widely available image ViTs: add \emph{minimal temporal adapters} for short-range dynamics, and \emph{lightweight global aggregators} for clip-level context—so the model inherits strong 2D priors \emph{and} efficient video modeling out of the box. 
        \end{itemize}
        
        \paragraph{UniFormerV2 (ICCV’22)~\cite{li2022_uniformerv2}: arming image ViTs for video, with full formulation and clear integration.}
        \textbf{Goal and integration at a glance.} Start from an \emph{image-pretrained ViT} and insert the \emph{smallest possible temporal machinery} so that all strong \emph{spatial} priors (attention weights, MLPs, and 2D positional encodings) are \emph{kept intact}. Concretely:
        \begin{itemize}
            \item \textbf{Keep (reuse) the ViT block as-is for space.} The standard ViT multi-head self-attention and FFN (per frame) are retained, so pretrained spatial weights and 2D position embeddings load without change. 
            \item \textbf{Add a temporal adapter before each block.} A lightweight \emph{Local Temporal MHRA (LT-MHRA)} is attached in a residual, pre-norm form, mixing only along time at each spatial site. This leaves the original spatial attention unchanged but makes tokens temporally aware. 
            \item \textbf{Add cheap global video aggregation late.} A \emph{Global UniBlock} (one-query cross-attention) produces per-clip \emph{video tokens} at selected deep layers; it is orthogonal to the ViT’s per-frame attention, so pretrained spatial weights remain unaffected. 
        \end{itemize}
        
        \textbf{Local UniBlock (LT-MHRA \(\rightarrow\) GS-MHRA \(\rightarrow\) FFN).} Let \(\bm{X}^{\text{in}}\!\in\!\mathbb{R}^{L\times C}\) be tubelet tokens \((L=T\!H\!W)\). The block applies
        \begin{align}
            \bm{X}^{T} &= \mathrm{LT\_MHRA}\big(\mathrm{Norm}(\bm{X}^{\text{in}})\big) + \bm{X}^{\text{in}}, \label{eq:uv2_lt}\\
            \bm{X}^{S} &= \mathrm{GS\_MHRA}\big(\mathrm{Norm}(\bm{X}^{T})\big) + \bm{X}^{T}, \label{eq:uv2_gs}\\
            \bm{X}^{L} &= \mathrm{FFN}\big(\mathrm{Norm}(\bm{X}^{S})\big) + \bm{X}^{S}. \label{eq:uv2_ffn}
        \end{align}
        Here \(\mathrm{GS\_MHRA}\) is the \emph{original} ViT spatial attention applied \emph{per frame} (so its pretrained weights and 2D positional encodings are reused directly), while \(\mathrm{LT\_MHRA}\) is a new temporal adapter. The shared MHRA form follows Eq.~\eqref{eq:uni_mhra_general}.
        
        \textbf{LT-MHRA (temporal-local adapter).} Temporal neighborhood only, per channel and per spatial site:
        \begin{equation}
            A^{\text{LT}}_{n}(\bm{X}_i,\bm{X}_j)=a^{\,n}_{\,i-j},\quad j\in\Omega^{t\times 1\times 1}_i,\quad a^{\,n}\!\in\!\mathbb{R}^{t\times 1\times 1}.
            \label{eq:uv2_lt_aff}
        \end{equation}
        \emph{Why it fits.} Adjacent frames are redundant; a depthwise 1D temporal conv (kernel \(t\!\approx\!3\)) captures short-range dynamics at \(\mathcal{O}(L\cdot t)\) cost, leaves spatial attention/FFN weights \emph{unchanged}, and therefore preserves the pretrained ViT’s spatial priors.
        
        \textbf{GS-MHRA (spatial-global within a frame, weight reuse).} Standard ViT attention applied \emph{per frame}:
        \begin{equation}
            A^{\text{GS}}_{n}(\bm{X}_i,\bm{X}_j)=
            \frac{\exp\big(Q_n(\bm{X}_i)^\top K_n(\bm{X}_j)\big)}
            {\sum_{j'\in\Omega^{1\times H\times W}}\exp\big(Q_n(\bm{X}_i)^\top K_n(\bm{X}_{j'})\big)}.
            \label{eq:uv2_gs_aff}
        \end{equation}
        \emph{Why it fits.} This is exactly the pretrained image ViT’s spatial self-attention: keys/queries/values and 2D positional encodings are loaded \emph{as-is}. Because temporal mixing happens \emph{before} this step via LT-MHRA, the ViT can leverage its spatial priors on temporally enriched tokens without retraining from scratch.
        
        \textbf{Global UniBlock (one-query cross-attention \(\Rightarrow\) video tokens).} At selected deep layers,
        \begin{align}
            \bm{X}^{C}  &= \mathrm{DPE}(\bm{X}^{L}) + \bm{X}^{L}, \label{eq:uv2_dpe}\\
            \bm{X}^{ST} &= \mathrm{C\_MHRA}\big(\mathrm{Norm}(\bm{q}),\,\mathrm{Norm}(\bm{X}^{C})\big), \label{eq:uv2_cmhra}\\
            \bm{X}^{G}  &= \mathrm{FFN}\big(\mathrm{Norm}(\bm{X}^{ST})\big) + \bm{X}^{ST}. \label{eq:uv2_g_ffn}
        \end{align}
        with a \emph{single learnable query} \(\bm{q}\!\in\!\mathbb{R}^{1\times C}\). Per head,
        \begin{align}
            \bm{R}^{\,C}_n(\bm{q},\bm{X}) &= \bm{A}^{\,C}_n(\bm{q},\bm{X})\,\bm{V}_n(\bm{X}), \\
            A^{\,C}_n(\bm{q},\bm{X}_j) &=
            \frac{\exp\!\big(Q_n(\bm{q})^\top K_n(\bm{X}_j)\big)}
            {\sum_{j'\in\Omega^{T\times H\times W}}\exp\!\big(Q_n(\bm{q})^\top K_n(\bm{X}_{j'})\big)}.
            \label{eq:uv2_cross}
        \end{align}
        The output is \emph{one video token} \(\bm{X}^G\!\in\!\mathbb{R}^{1\times C}\) per chosen depth (per clip). \emph{Why it fits.} This is content-aware global pooling with \(\mathcal{O}(L)\) cost; it does not alter the pretrained per-frame attention weights and adds only a small number of parameters (query and projections), keeping the original ViT intact for spatial reasoning.
        
        \textbf{Multi-stage fusion (what is fused, where it plugs).} Collect video tokens \(\{\bm{X}^G_i\}\) from several deep layers and fuse:
        \begin{itemize}
            \item \textbf{Sequential:} \(\bm{X}^G_i \!=\! \mathcal{G}_i(\bm{X}^G_{i-1},\bm{X}^L_i)\) progressively refines summaries. 
            \item \textbf{Parallel:} \(\bm{F}=\mathrm{Concat}(\bm{X}^G_1,\dots,\bm{X}^G_N)\,\bm{U}^F\) aggregates multi-scale semantics in one shot. 
            \item \textbf{Hierarchical (Q or KV):} Propagate queries or keys/values across depths so later tokens condition on earlier summaries. 
        \end{itemize}
        Finally, mix \(\bm{F}\) with the final class token via a learned gate, \(\bm{Z}=\alpha\,\bm{F} + (1-\alpha)\,\bm{F}^{C}\), yielding a compact clip descriptor. \emph{Why it fits.} Shallow video tokens carry fine temporal cues; deeper ones carry semantics. Fusion balances both, while preserving the pretrained ViT’s spatial pathway.
        
        \textbf{Why this staging preserves pretrained weights and improves efficiency.}
        \begin{itemize}
            \item \textbf{Temporal adapter first.} LT\_MHRA mixes only along time in a residual path, leaving the ViT’s per-frame attention and FFN untouched; pretrained 2D weights and positional encodings load directly. 
            \item \textbf{Spatial path unchanged.} GS\_MHRA is the original ViT spatial attention applied within each frame; keys/queries/values and 2D position embeddings are reused without modification. 
            \item \textbf{Late, linear-cost globalization.} The one-query Global UniBlock appears only in deep layers, providing clip-level context at \(\mathcal{O}(L)\) (vs.\ \(\mathcal{O}(L^2)\)) when features are already abstract. 
        \end{itemize}
        
        \noindent
        \textbf{Net effect.} UniFormerV2 retains strong image-ViT spatial priors, adds minimal temporal mixing, and introduces inexpensive clip-level aggregation—exactly what InternVideo needs for temporally competent, language-alignable video features.
        
        \textbf{Concrete cue.} For ``\emph{pouring coffee}’’, LT\_MHRA stabilizes micro-motions of mug and pot across adjacent frames; GS\_MHRA relates hand, mug, and pot within each frame. In deep layers, a Global UniBlock extracts a \emph{video token} that concentrates on the interval where liquid flow is visible despite camera shake. Finally, UniFormerV2 \emph{late-integrates} this video token with the backbone’s \texttt{[CLS]} token via a learned gate (\(\bm{Z}=\alpha\,\bm{F}+(1{-}\alpha)\bm{F}^{C}\)), yielding a clip descriptor that fuses temporally pooled evidence (\(\bm{F}\)) with the strong spatial prior captured by \texttt{[CLS]}\,(\(\bm{F}^{C}\)). In InternVideo’s final CMA, the \emph{multimodal} \texttt{[CLS]} further queries masked-video tokens, enriching this summary with fine motion/detail before task heads.
        
        \paragraph{Bridging to the method: why UniFormer/UniFormerV2 set the stage.}
        \begin{itemize}
            \item \textbf{Temporally competent yet ViT-friendly.} UniFormer/UniFormerV2 inject lightweight temporal mixing (LT\_MHRA) \emph{before} standard ViT spatial attention and keep the per-frame attention/FFN unchanged. This preserves strong 2D priors while yielding tokens that already encode short-range dynamics. 
            \item \textbf{Compact clip-level context.} Late one-query cross-attention produces \emph{video tokens}—content-aware summaries of the entire clip—that complement the usual \texttt{[CLS]} representation via a learned late integration. These summaries are ideal anchors for downstream fusion. 
            \item \textbf{Natural interface for cross-stream fusion.} With temporally aware patch tokens, a robust \texttt{[CLS]} token, and per-clip video tokens, the backbone exposes clean \emph{query/key/value} points. In the following, the method will leverage these to \emph{coordinate} a generative (masked-reconstruction) stream and a discriminative (video–language) stream through lightweight cross-attention, transferring semantics and motion cues without retraining the backbones. 
        \end{itemize}
        
        \begin{figure}[H]
            \centering
            \includegraphics[width=0.85\textwidth]{Figures/Chapter_24/Uniformerv2_comparison_to_v1.jpg}
            \caption{\textbf{Why arm image ViTs.} Naïvely adding temporal MHSA to image ViTs tends to underperform for a given compute budget. UniFormerV2 keeps strong spatial priors and adds concise temporal modules to achieve superior accuracy--FLOPs trade-offs on video. Adapted from \cite{li2022_uniformerv2}.}
            \label{fig:chapter24_uniformer_v1v2}
        \end{figure}
        
        \begin{figure}[H]
            \centering
            \includegraphics[width=0.85\textwidth]{Figures/Chapter_24/Uniformerv2_method.jpg}
            \caption{\textbf{UniFormerV2 framework.} Each stage consists of a Local UniBlock (LT-MHRA adapter \(\rightarrow\) preserved ViT spatial attention \(\rightarrow\) FFN). Selected deep stages add a Global UniBlock that forms a per-clip video token via one-query cross-attention; multiple video tokens are fused to form the final descriptor. Adapted from \cite{li2022_uniformerv2}.}
            \label{fig:chapter24_uniformer_v2_method}
        \end{figure}
        
        \subsubsection{Method}
        \label{subsubsec_chapter24_internvideo_method}
        
        \paragraph{High-level overview}
        InternVideo trains two \emph{complementary} encoders with different pretext signals and then \emph{coordinates} them at adaptation time via lightweight cross-attention:
        \begin{itemize}
            \item \textbf{Masked Video Encoder (MVE).} VideoMAE-style ViT with high-ratio tube masking ($\approx 90\%$) and an asymmetric encoder–decoder that reconstructs pixels and learns motion/appearance-coherent features without labels. 
            \item \textbf{Multimodal Video Encoder (MMVE).} UniFormerV2-based video backbone paired with a transformer text encoder, typically CLIP-initialized, trained on large-scale \emph{video/image–text} data via a symmetric \emph{contrastive} loss and a cross-modal \emph{captioning} loss for language-aligned semantics~\cite{li2022_uniformerv2,wang2022_internvideo}. 
            \item \textbf{Coordination via Cross-Model Attention (CMA).} After separate pretraining, both backbones are frozen and small cross-attention+FFN adapters are inserted so the streams can query each other~\cite{wang2022_internvideo}:
            \begin{itemize}
                \item \emph{Intermediate fusion:} \textbf{MVE tokens query MMVE tokens} to absorb semantic structure (inject language-aligned cues into motion-rich features). 
                \item \emph{Final fusion:} \textbf{MMVE \texttt{[CLS]} queries MVE tokens} to inject precise motion/detail into the multimodal summary used for prediction. 
            \end{itemize}
        \end{itemize}
        
        \begin{figure}[H]
            \centering
            \includegraphics[width=0.85\textwidth]{Figures/Chapter_24/InternVideo_MVE.jpg}
            \caption{\textbf{Pretraining pathways.} (a) Masked video modeling with an asymmetric ViT encoder–decoder. (b) Multimodal learning with UniFormerV2 video encoder, CLIP-initialized text encoder, and a cross-modal caption decoder. Adapted from \cite{wang2022_internvideo}.}
            \label{fig:chapter24_internvideo_mve_mml}
        \end{figure}
        
        \paragraph{Notation}
        Let a video clip be tokenized into tubelets and embedded as \(\bm{X}\!\in\!\mathbb{R}^{L\times C}\) with \(L=T\!\times\!H\!\times\!W\). Cosine similarity is \(\mathrm{sim}(\cdot,\cdot)\). Temperatures are \(\tau>0\). The symbol \(\odot\) denotes elementwise multiplication.
        
        \paragraph{1) Generative path --- Masked Video Encoder (MVE)}
        InternVideo adopts a VideoMAE-style asymmetric ViT for self-supervised masked video pretraining:
        \begin{itemize}
            \item \textbf{Tube masking.} A large fraction (e.g., 90\%) of spatiotemporal tubelet tokens is masked; only \emph{visible} tokens are processed by the encoder, making joint space–time attention computationally tractable on long clips.
            \item \textbf{Asymmetric encoder–decoder.} A compact decoder (fewer channels/blocks) reconstructs the original pixels (or low-level targets) from the encoder features and mask tokens. 
        \end{itemize}
        \emph{Pixel reconstruction loss.} With ground-truth pixel targets \(\bm{Y}\) and reconstruction \(\hat{\bm{Y}}\), the masked-patch regression uses mean-squared error over masked positions:
        \begin{equation}
            \mathcal{L}_{\text{pix}}
            \;=\;
            \frac{1}{|\mathcal{M}|}\sum_{p \in \mathcal{M}}\big\|\hat{\bm{Y}}_{p}-\bm{Y}_{p}\big\|_2^{2},
            \label{eq:internvideo_pix}
        \end{equation}
        where \(\mathcal{M}\) indexes masked tubelets. This forces the encoder to form motion/appearance-coherent, temporally aware features that predict missing content from sparse context.
        
        % ------------------------------------------------
        % 2) Discriminative path: Multimodal Video Encoder (MMVE)
        % ------------------------------------------------
        \paragraph{2) Discriminative path --- Multimodal Video Encoder (MMVE)}
        InternVideo builds on CLIP-style alignment but uses \textbf{UniFormerV2} as the \emph{video} backbone (Sec.~\ref{subsubsec_chapter24_uniformer}), paired with a transformer \emph{text} encoder and a small \emph{cross-modal caption decoder}:
        \begin{itemize}
            \item \textbf{Align before fuse.} First align video and text \emph{embeddings} with a symmetric contrastive loss; then \emph{fuse} them using a decoder with cross-attention under a captioning loss. This brings zero-shot alignment (retrieval) and stronger multimodal composition (caption/VQA) in a single framework~\cite{wang2022_internvideo,li2022_uniformerv2}. 
        \end{itemize}
        
        \emph{Contrastive loss.} Given minibatch \(\{(\bm{v}_i,\bm{t}_i)\}_{i=1}^{B}\) of video/text embeddings,
        \begin{align}
            \mathcal{L}_{\text{v}\rightarrow\text{t}}
            &= -\frac{1}{B}\sum_{i=1}^{B}
            \log\frac{\exp\big(\mathrm{sim}(\bm{v}_i,\bm{t}_i)/\tau\big)}
            {\sum_{j=1}^{B}\exp\big(\mathrm{sim}(\bm{v}_i,\bm{t}_j)/\tau\big)}, \\
            \mathcal{L}_{\text{t}\rightarrow\text{v}}
            &= -\frac{1}{B}\sum_{i=1}^{B}
            \log\frac{\exp\big(\mathrm{sim}(\bm{t}_i,\bm{v}_i)/\tau\big)}
            {\sum_{j=1}^{B}\exp\big(\mathrm{sim}(\bm{t}_i,\bm{v}_j)/\tau\big)}, \\
            \mathcal{L}_{\text{con}}&=\frac{1}{2}\left(\mathcal{L}_{\text{v}\rightarrow\text{t}}+\mathcal{L}_{\text{t}\rightarrow\text{v}}\right).
            \label{eq:internvideo_contrast}
        \end{align}
        
        \paragraph{Captioning loss: concise mechanics and intuition}
        Each training clip is paired with a ground-truth caption (GT) from the video–text dataset. The model does \emph{not} predict “video tokens’’; it predicts the next \emph{word token} in the caption. 
        
        \emph{Who is Query/Key/Value—and why.} In a transformer \emph{caption decoder}, the \textbf{text prefix} (\texttt{[BOS]} $w_1,\dots,w_{t-1}$) forms the \textbf{queries}: the decoder is asking, “given the words so far, what visual evidence do I need next?” The \textbf{video features} produced by the UniFormerV2 encoder (dense spatiotemporal tokens and optional \emph{video tokens} from Global UniBlocks) serve as \textbf{keys/values}: they are the searchable memory of what happened where and when. Cross-attention thus retrieves the relevant visual context to emit the next word.
        
        \emph{How training proceeds.} With \emph{teacher forcing}, the decoder conditions on the \emph{ground-truth} prefix and predicts the next word; the token-level cross-entropy
        \[
        \mathcal{L}_{\mathrm{cap}} \;=\; -\sum_{t=1}^{T_{\text{cap}}} \log p_\theta(w_t \mid w_{<t},\,\text{video})
        \]
        is minimized. The overall multimodal objective combines retrieval-oriented alignment and generative grounding:
        \[
        \mathcal{L}_{\mathrm{MM}} \;=\; \mathcal{L}_{\mathrm{con}} \;+\; \lambda_{\mathrm{cap}}\mathcal{L}_{\mathrm{cap}} ,\qquad \lambda_{\mathrm{cap}}>0.
        \]
        
        \newpage
        
        \emph{Why this design works.}
        \begin{itemize}
            \item \textbf{Queries from text.} Language dictates what detail is needed next (object $\rightarrow$ attribute $\rightarrow$ action), so using the textual prefix as queries makes the decoder “ask’’ targeted questions of the video memory. 
            \item \textbf{Keys/values from video.} UniFormerV2 provides (i) \emph{dense} tokens for fine, localized evidence (hands, objects, micro-motions) and (ii) \emph{video tokens}—compact, late-stage summaries—for long-range context (phases of an action). This mix lets cross-attention retrieve both sharp details and global storyline. 
            \item \textbf{What “video token’’ means.} It is a \emph{learned, per-clip summary feature} (one token per selected deep layer), not a word to predict. The decoder can attend to it alongside dense tokens to maintain temporal coherence in generation.
            \item \textbf{Why add captioning to contrastive.} Contrastive loss aligns whole video–text pairs (useful for retrieval), but captioning forces \emph{word-by-word grounding}: to emit each token, the decoder must attend to the correct frames/regions. This strengthens VQA/captioning and makes retrieval more robust to distribution shift.
        \end{itemize}
        
        \emph{Role of UniFormerV2 (video memory).} \emph{LT\_MHRA} compacts short-range motion (clean verb cues), \emph{GS\_MHRA} preserves strong spatial priors (reliable nouns/attributes), and \emph{Global UniBlocks} add per-clip summary tokens (temporal coherence). Together they produce a video memory that the decoder can query precisely—rich in detail yet resistant to spurious shortcuts.
        
        \paragraph{3) Coordination — Cross-Model Attention (CMA).}
        \textbf{Setup.} After the masked and multimodal paths are \emph{pretrained independently}, InternVideo \emph{freezes} both backbones and inserts a small stack of CMA adapters at selected mid/high layers~\cite{wang2022_internvideo}. Each adapter is a residual, post-norm block with multi-head \emph{cross}-attention (MHCA) followed by an FFN. Let
        \[
        \mathrm{MHCA}(\bm{Q},\bm{K},\bm{V})=\mathrm{Concat}(\mathrm{head}_n)\,\bm{W}_o,\qquad
        \mathrm{head}_n=\mathrm{softmax}\!\Big(\tfrac{\bm{Q}\bm{W}^Q_n(\bm{K}\bm{W}^K_n)^{\!\top}}{\sqrt{d}}\Big)\,(\bm{V}\bm{W}^V_n).
        \]
        The \emph{host} stream supplies queries \((\bm{Q})\) and is updated; the \emph{guest} stream supplies keys/values \((\bm{K},\bm{V})\) as read-only memory. Lightweight $1{\times}1$ projections inside the adapter align channel widths when needed.
        
        \textbf{Directional use (who queries whom, and why).}
        \begin{itemize}
            \item \textbf{Intermediate CMA (MVE $\rightarrow$ MMVE).} At several early/mid depths of the masked path, \emph{MVE tokens} are used as \(\bm{Q}\) and \emph{MMVE tokens} as \(\bm{K},\bm{V}\). The adapter output replaces (or is residually added to) the MVE tokens. \emph{Effect:} transfers language-aligned semantics into motion/appearance-coherent features while preserving their temporal precision.
            \item \textbf{Final CMA (MMVE cls $\rightarrow$ MVE).} Just before the multimodal head, the \emph{MMVE class token} is \(\bm{Q}\) and the full \emph{MVE token map} is \(\bm{K},\bm{V}\). The updated class token becomes the input to the prediction head. \emph{Effect:} injects fine motion/detail cues into the multimodal summary at decision time.
        \end{itemize}
        
        \textbf{Why this placement.}
        \begin{itemize}
            \item \textbf{Preserve priors.} Freezing both encoders protects pixel-reconstruction priors (MVE) and CLIP/UniFormerV2 priors (MMVE); CMA learns to \emph{align}, not to overwrite~\cite{wang2022_internvideo}.
            \item \textbf{Parameter- and compute-efficient.} Only the small adapter parameters \(\{\bm{W}^Q,\bm{W}^K,\bm{W}^V,\bm{W}_o\}\) and FFN are trained for a few supervised epochs; cross-attention cost scales with \(\mathcal{O}(L_{\!Q}L_{\!K})\) and remains modest at mid/high stages.
            \item \textbf{Complementarity made explicit.} Intermediate CMA semantically \emph{enriches} motion-rich tokens; final CMA \emph{sharpens} the language summary with precise dynamics, improving action understanding, retrieval, captioning, and VQA.
        \end{itemize}
        
        \begin{figure}[H]
            \centering
            \includegraphics[width=0.85\textwidth]{Figures/Chapter_24/InternVideo_CMA.jpg}
            \caption{\textbf{Cross-Model Attention (CMA).} Middle: MVE tokens query MMVE tokens to import semantics into the masked stream. Final: the MMVE class token queries MVE tokens to add motion/detail to the multimodal summary used by the head. Adapted from \cite{wang2022_internvideo}.}
            \label{fig:chapter24_internvideo_cma}
        \end{figure}
        
        \paragraph{4) Prediction heads and supervised adaptation}
        For action recognition, temporal localization, retrieval, and VQA/captioning, InternVideo attaches modest heads on top of the fused features:
        \begin{itemize}
            \item \textbf{Recognition/localization.} A linear/MLP head over the final fused token(s) (or task-specific proposals) trained with cross-entropy/mAP losses, depending on the benchmark setup.
            \item \textbf{Retrieval.} For text-to-video (T2V) and video-to-text (V2T), the aligned MMVE pathway enables using contrastive similarity between the fused video representation and text embeddings.
            \item \textbf{VQA/caption.} The cross-modal decoder (already trained) can be adapted with task supervision; CMA improves the video-side evidence entering the decoder. 
        \end{itemize}
        The adaptation schedule used in the paper freezes both encoders and learns CMA (and the task heads) for a few epochs, providing a tractable path to fuse large pretrained models~\cite{wang2022_internvideo}.
        
        % ------------------------------------------------
        % 5) Putting it together (flow)
        % ------------------------------------------------
        \paragraph{5) End-to-end flow (one pass)}
        \begin{enumerate}
            \item \textbf{Two encoders (frozen at adaptation).} 
            \begin{enumerate}
                \item MVE: encode visible tubelets, decode reconstruction, providing masked-stream tokens.
                \item MMVE (UniFormerV2 + text): encode video and text; obtain aligned embeddings and, optionally, decoder features.
            \end{enumerate}
            \item \textbf{CMA stacks.} Apply intermediate CMA blocks with \(\bm{Q}=\) MVE tokens, \(\bm{K},\bm{V}=\) MMVE tokens; update MVE-side representations. 
            \item \textbf{Final CMA.} Use MMVE class token as \(\bm{Q}\) and MVE tokens as \(\bm{K},\bm{V}\); update the class token.
            \item \textbf{Heads.} Feed the fused token(s) to the task head: classifier, retrieval similarity, localization head, or decoder. 
        \end{enumerate}
        
        % ================================================================
        \subsubsection{Architecture and Implementation Details}
        \label{subsubsec_chapter24_internvideo_arch_impl}
        % ================================================================
        
        \paragraph{Backbone choices}
        \begin{itemize}
            \item \textbf{Masked Video Encoder (MVE)} ViT with tubelet embedding and an asymmetric decoder as in VideoMAE; high mask ratio (\(\rho\!\approx\!0.9\)) for efficient joint space–time learning of visible tokens.
            \item \textbf{Multimodal Video Encoder (MMVE)} UniFormerV2 (Sec.~\ref{subsubsec_chapter24_uniformer}) as the video tower and a transformer text tower (typically CLIP-initialized), paired with a lightweight caption decoder.
        \end{itemize}
        
        \paragraph{Tokenization and shapes}
        A clip of \(T\) frames at resolution \(H{\times}W\) is patchified with temporal stride \(s_t\) and spatial stride \(s\) into \(T'\!\times\!H'\!\times\!W'\) tubelets, forming \(L=T'H'W'\) tokens of width \(C\).
        The MVE processes only the visible subset; the MMVE processes all tokens.
        
        \paragraph{UniFormerV2 block order in MMVE}
        Each block applies \(\text{LT\_MHRA} \rightarrow \text{GS\_MHRA} \rightarrow \text{FFN}\), preserving the pretrained ViT’s spatial attention and MLP while adding a residual temporal adapter
        At selected deep layers, a \emph{Global UniBlock} (one-query cross-attention) emits \emph{video tokens} (one per chosen depth), which serve as compact per-clip summaries for captioning and heads.
        
        \paragraph{Cross-Model Attention (CMA) placement}
        \begin{itemize}
            \item \textbf{Intermediate CMA} Inserted at several mid-depth stages along the MVE; \(\bm{Q}\) from MVE tokens, \(\bm{K},\bm{V}\) from MMVE tokens; output replaces or is residually added to MVE tokens.
            \item \textbf{Final CMA} Right before the MMVE head; \(\bm{Q}\) is the MMVE \texttt{[CLS]}, \(\bm{K},\bm{V}\) are the full MVE token map; the updated \texttt{[CLS]} goes to prediction heads.
        \end{itemize}
        Each CMA adapter is a residual, post-norm MHCA\(+\)FFN; optional \(1{\times}1\) projections align channel widths.
        
        \paragraph{Training schedule}
        \begin{enumerate}
            \item \textbf{Stage 1} Self-supervised masked pretraining of MVE with pixel regression over masked tubelets.
            \item \textbf{Stage 2} Multimodal pretraining of MMVE with symmetric contrastive alignment and captioning under teacher forcing; UniFormerV2 supplies temporally competent video features.
            \item \textbf{Stage 3} Supervised adaptation on downstream tasks with both encoders frozen; train only CMA and task heads.
        \end{enumerate}
        
        \newpage
        
        % ================================================================
        \subsubsection{Experiments and Ablations}
        \label{subsubsec_chapter24_internvideo_experiments}
        % ================================================================
        
        \paragraph{Bottom-line summary across tasks}
        \begin{itemize}
            \item \textbf{Action recognition.} On Kinetics-400/600/700, InternVideo attains \textbf{91.1}/\textbf{91.3}/\textbf{84.0}\% top-1 with billion-scale backbones (InternVideo-D/T), edging strong generative and multimodal pretraining baselines such as MaskFeat-L, CoCa, and MTV-H at comparable or larger scales \cite{wang2022_internvideo,wei2022_maskfeat,yu2022_coca,yan2022_mtv}. 
            \item \textbf{More AR benchmarks.} Gains transfer broadly: \textbf{70.0}\% on SthSthV1 (\(+9.1\)), \textbf{77.2}\% on SthSthV2 (\(+1.6\)), \textbf{94.3}\% on ActivityNet (\(+4.1\)), \textbf{95.5}\% on HACS (\(+3.6\)), and \textbf{89.3}\% on HMDB51 (\(+1.7\)) over prior best reports \cite{wang2022_internvideo}. 
            \item \textbf{Temporal localization.} Coupled with strong heads, InternVideo improves average mAP on THUMOS-14, ActivityNet-v1.3, HACS, and FineAction; e.g., with ActionFormer it reaches \textbf{71.58} on THUMOS-14 and \textbf{39.00} on ActivityNet-v1.3, and with TCANet it reaches \textbf{41.55} on HACS \cite{wang2022_internvideo,zhang2022_actionformer,xu2021_tcanet}. 
            \item \textbf{Spatiotemporal localization.} With a linear head, InternVideo reports \textbf{41.01} mAP on AVA2.2 and \textbf{42.51} mAP on AVA-Kinetics, surpassing prior ensembles and MaskFeat while using minimal task-specific tuning \cite{wang2022_internvideo,pan2021_acar,li2020_relationalmaps,wei2022_maskfeat}. 
            \item \textbf{Text–video retrieval.} R@1 improves consistently over CLIP-derived baselines across MSR-VTT, MSVD, LSMDC, ActivityNet, DiDeMo, and VATEX for both T2V and V2T, reflecting stronger alignment and compositional grounding \cite{wang2022_internvideo,luo2022_clip4clip,ma2022_xclip,liu2022_ts2net}. 
            \item \textbf{VQA and captioning.} Adding a captioning objective yields absolute gains of \(\sim\)3–8 points on MSRVTT, MSVD, and TGIF, indicating that generative grounding complements contrastive alignment \cite{wang2022_internvideo,lei2021_clipbert,fu2021_violet,zellers2021_merlot,wang2022_allinone}. 
            \item \textbf{Navigation and robustness.} Improvements extend to VLN-CE and open-set AR, with higher success rates and better uncertainty calibration than prior backbones \cite{wang2022_internvideo}. 
        \end{itemize}
        
        \paragraph{Key ablations and what they imply}
        \begin{itemize}
            \item \textbf{CMA matters.} Removing cross-model attention lowers action recognition and retrieval, most notably for categories needing both fine motion and semantics, confirming stream complementarity and the benefit of two-direction fusion \cite{wang2022_internvideo}. 
            \item \textbf{Fusion directions are not interchangeable.} Using only MVE\(\rightarrow\)MMVE or omitting the late MMVE \texttt{[CLS]}\(\rightarrow\)MVE step underperforms; the final class-token query is particularly important for recognition and retrieval \cite{wang2022_internvideo}. 
            \item \textbf{Mask ratio vs.\ clip length.} Very high masking on short clips can over-regularize the MVE; tuning \(\rho\) jointly with clip length \(T\) stabilizes learning and improves transfer \cite{wang2022_internvideo}. 
            \item \textbf{Why UniFormerV2.} Replacing UniFormerV2 with a naïve ViT video tower weakens captioning/VQA and retrieval under shift, underscoring the benefit of LT\_MHRA and late video tokens for temporal coherence \cite{li2022_uniformerv2,wang2022_internvideo}. 
        \end{itemize}
        
        \paragraph{Representative takeaways}
        \begin{itemize}
            \item \textbf{Scale with structure beats raw scale.} InternVideo-T surpasses 1B+ baselines like CoCa and MTV-H on Kinetics despite similar or larger parameter counts, pointing to the benefits of structured dual pretraining and CMA over size alone \cite{wang2022_internvideo,yu2022_coca,yan2022_mtv}.
            \item \textbf{Complementary objectives help.} Improvements on retrieval and VQA mirror the combination of \(\mathcal{L}_{\text{con}}\) and \(\mathcal{L}_{\text{cap}}\), while AR and localization gains show that motion structure learned by the MVE remains intact after coordination \cite{wang2022_internvideo}.
        \end{itemize}
        
        \newpage
    
        % ================================================================
        \subsubsection{Limitations and Follow-up Works}
        \label{subsubsec_chapter24_internvideo_limits_future}
        % ================================================================
        
        \paragraph{Current limitations}
        \begin{itemize}
            \item \textbf{Two-tower rigidity.} With encoders frozen at adaptation, CMA aligns mid/high-level features but has limited ability to influence early representations or permit full co-adaptation \cite{wang2022_internvideo}. 
            \item \textbf{Cross-attention budget.} CMA complexity scales with \(L_Q L_K\); very long clips, high resolutions, or dense token maps can raise adaptation cost even if it remains lighter than end-to-end joint training. 
            \item \textbf{Temporal skew.} Differences in sampling policies or augmentations between streams may introduce small frame misalignments unless clip timing is carefully synchronized. 
            \item \textbf{Language priors.} Contrastive pretraining can reflect dataset caption biases; the captioning objective and CMA help, but residual bias may remain. 
        \end{itemize}
        
        \paragraph{Buildup toward InternVideoV2}
        \begin{itemize}
            \item \textbf{Gentle co-adaptation.} Move beyond fully frozen fusion by selectively or gradually unfreezing blocks around CMA sites, so motion and semantics can co-evolve while preserving strong pretrained priors. 
            \item \textbf{Token economy.} Reduce the number of tokens entering CMA via dynamic pruning or routing, focusing cross-attention on salient motion regions and text-relevant evidence to keep \(L_Q,L_K\) modest. 
            \item \textbf{Stronger temporal bias.} Improve temporal synchronization and long-horizon stability with richer relative position modeling and more reliable per-clip summary tokens, mitigating drift across long sequences. 
            \item \textbf{Unified curricula.} Coordinate masking ratios, clip lengths, and the balance of masked, contrastive, generative losses—together with cleaner temporal segmentation and caption quality—to smooth optimization at scale. 
        \end{itemize}
        
        \paragraph{Takeaway}
        InternVideo indicates that \emph{separate} generative and multimodal pretraining, followed by \emph{lightweight} cross-attention coordination, can produce broadly transferable video representations across recognition, localization, retrieval, VQA, navigation, and open-set evaluation, and subsequent variants soften the two-tower boundary and streamline fusion cost while retaining the pretrained priors that underpin these gains \cite{wang2022_internvideo}.
        
    \end{enrichment}
    
    \newpage
    
    \begin{enrichment}[OmniVL: One Model for Image–Video–Language][subsection]
        \label{enr:subsec_chapter24_omnivl}
        
        \paragraph{Scope and positioning}
        OmniVL's \cite{wang2022_omnivl} central thesis is \emph{unification} along three axes: data (image/video paired with either text or labels), modality (a single visual encoder for images and videos), and functionality (alignment and generation) without task-specific adapters. The authors introduce a \emph{decoupled joint} curriculum and a \emph{Unified Vision–Language Contrastive} loss (UniVLC) to couple clean labels with noisy captions, yielding bidirectional gains for both image and video tasks.
        
        \begin{figure}[H]
            \centering
            \includegraphics[width=0.85\linewidth]{Figures/Chapter_24/OmniVL_overview.jpg}
            \caption{\textbf{OmniVL overview.} The framework unifies the pretraining corpus (human-annotated labels and web-crawled captions), the modality space (image, video, and text), and functionality (visual-only classification, cross-modal alignment, and multi-modal understanding/generation) in a single encoder–decoder architecture. Source: \cite{wang2022_omnivl}}
            \label{fig:chapter24_omnivl_overview}
        \end{figure}
        
        \subsubsection{Motivation}
        \label{subsubsec:chapter24_omnivl_motivation}
        
        \paragraph{Fragmentation problem}
        The OmniVL paper \cite{wang2022_omnivl} is motivated by persistent fragmentation in vision–language foundations, which hampers transfer, scalability, and simplicity of deployment. Three silos are especially limiting.
        \begin{itemize}
            \item \textbf{Functionality silo (retrieval vs.\ generation).} Many systems dedicate separate models or heavy adapters to non-generative alignment tasks (e.g., retrieval) versus generative tasks (e.g., captioning and QA), preventing a single representation from serving both effectively.
            \item \textbf{Modality silo (image vs.\ video).} Image-language models are often extended to video as independent frames or via ad-hoc heads, weakening temporal modeling and requiring extra parameters to adapt to motion.
            \item \textbf{Data-source silo (captions vs.\ labels).} Training on either clean, discriminative label corpora (e.g., ImageNet, Kinetics) \emph{or} noisy, webly image/video–text pairs yields features that are respectively precise or broad but rarely both, leaving potential gains from joint supervision unrealized.
        \end{itemize}
        In practice, these silos force multiple specialized pipelines and miss beneficial cross-task and cross-modal transfer, particularly for video where temporal dependencies are essential.
        
        \paragraph{Design hypothesis}
        OmniVL posits that a single model can break these silos by unifying data, modality, and functionality under one pretraining and inference framework.
        \begin{itemize}
            \item \textbf{Unified feature space via mixed supervision.} Combining clean label corpora with webly caption corpora in a shared contrastive space should yield features that are simultaneously discriminative and semantically rich, improving both visual-only recognition and cross-modal understanding.
            \item \textbf{Efficient spatio-temporal modeling in one encoder.} Sharing a TimeSformer-style visual backbone across images and videos leverages strong spatial pretraining for images while activating temporal attention only for videos, avoiding duplicate encoders and encouraging transfer from spatial to spatio-temporal representation.
            \item \textbf{Decoupled joint pretraining for stability and bidirectional gains.} First learning on image–language to establish robust spatial representations, then continuing with joint image+video training, should introduce temporal dynamics without catastrophic forgetting and produce gains that flow in both directions (image \(\leftrightarrow\) video).
        \end{itemize}
        This hypothesis underlies OmniVL’s choice of losses and curriculum, aiming to replace fragmented stacks with a single foundation that scales across tasks and modalities.
        
        \subsubsection{Method: high-level flow and detailed breakdown}
        \label{subsubsec:chapter24_omnivl_method}
        
        \paragraph{High-level overview}
        OmniVL \cite{wang2022_omnivl} follows an encoder–decoder design that routes the same inputs through a unified pipeline and learns three complementary objectives. The end-to-end flow is: \emph{tokenizing} \(\rightarrow\) \emph{positional encodings (spatial, temporal)} \(\rightarrow\) \emph{unified visual encoder} \(\rightarrow\) \emph{text encoder} \(\rightarrow\) \emph{two visual-grounded decoders for alignment and generation} \(\rightarrow\) \emph{task heads and objectives}. A two-stage, \emph{decoupled joint} pretraining curriculum first builds strong spatial representations on image–language, then introduces video–language to learn temporal dynamics while preserving the spatial foundation. This structure supports visual-only tasks, cross-modal alignment, and multi-modal generation within a single model without task-specific adapters. 
        
        \paragraph{Data format and prompting}
        All pretraining sources are expressed in a joint visual–label–text space \cite{wang2022_omnivl}. Each sample is a triplet \(S=(x,y,t)\), where \(x\) is an image or a video, \(y\) is a unique category index, and \(t\) is its language description. For image/video–label data, \(t\) is generated with CLIP- and ActionCLIP-style prompt templates by filling class names, so that all visual samples with the same category share a common textual description. This formulation unifies four corpora: image–text, video–text, image–label, and video–label. 
        
        \paragraph{Step-by-step data flow}
        \begin{enumerate}
            \item \textbf{Tokenizing.} Raw inputs are converted into a unified token sequence suitable for a transformer. For an image \(x\!\in\!\mathbb{R}^{H\times W\times 3}\), OmniVL applies a \emph{2D patch tokenizer} implemented as a convolution with kernel and stride \(p\times p\), producing \(N=\tfrac{HW}{p^2}\) patch tokens in \(\mathbb{R}^{D}\) plus a learned \texttt{[CLS]} token. For a video \(x\!\in\!\mathbb{R}^{T\times H\times W\times 3}\), OmniVL uses a \emph{3D tubelet tokenizer} with kernel and stride \(\tau\times p\times p\) to yield \(N'=\tfrac{T}{\tau}\cdot\tfrac{HW}{p^2}\) tokens that already encode short-term motion, again prepending a \texttt{[CLS]}. Using dedicated 2D/3D tokenizers aligns the channel dimension \(D\) across modalities, injects a mild local inductive bias, and normalizes the interface to the shared encoder while allowing flexible frame sampling and tubelet length during training and fine-tuning \cite{wang2022_omnivl}. 
            
            \item \textbf{Positional encodings.} Self-attention is permutation invariant, so OmniVL injects position using \emph{learned}, \emph{factorized} absolute embeddings that separate space and time \cite{wang2022_omnivl}. Let \(E_{\mathrm{s}}(h,w)\in\mathbb{R}^{D}\) be a learned spatial table on the 2D grid and \(E_{\mathrm{t}}(t)\in\mathbb{R}^{D}\) a learned temporal table on the 1D frame/tubelet axis. An image token at \((h,w)\) is
            \[
            z_{0}^{h,w} \;=\; \mathrm{patch\_emb}\!\left(x^{h,w}\right) \;+\; E_{\mathrm{s}}(h,w),
            \]
            and a video tubelet token at time \(t\), location \((h,w)\) is
            \[
            z_{0}^{t,h,w} \;=\; \mathrm{tubelet\_emb}\!\left(x^{t,h,w}\right) \;+\; E_{\mathrm{s}}(h,w) \;+\; E_{\mathrm{t}}(t).
            \]
            \emph{Why learned (vs.\ sinusoidal or purely relative).} \textbf{(i) Compatibility with image pretraining.} The decoupled-joint schedule initializes the unified encoder from strong image checkpoints that use learned absolute PEs; keeping learned \(E_{\mathrm{s}}\) preserves this interface, easing transfer from Stage~1 (images) to Stage~2 (videos). \textbf{(ii) Modality unification and weight sharing.} A shared 2D spatial table \(E_{\mathrm{s}}\) across images and videos keeps the spatial grid consistent, so the same encoder weights can process both, while time is injected only when present via \(E_{\mathrm{t}}\). \textbf{(iii) Stable scaling via interpolation.} Factorizing space and time avoids a monolithic 3D table tied to fixed \((T,H,W)\); learned \(E_{\mathrm{s}},E_{\mathrm{t}}\) are smoothly interpolated for new resolutions or clip lengths at fine-tuning. \textbf{(iv) Empirical simplicity.} Learned absolute PEs are a lightweight, data-driven choice that match or outperform fixed sinusoids in ViT-style vision models while avoiding extra complexity in attention kernels required by some relative schemes. Overall, OmniVL’s learned, factorized PEs deliver the needed spatial layout and temporal order signals while aligning with the unified encoder and the training curriculum.
            
            \item \textbf{Unified visual encoder.} A single transformer encoder with shared parameters processes both modalities. Each block follows a TimeSformer-style \emph{decoupled} scheme, applying \emph{temporal self-attention} across tokens at the same spatial index followed by \emph{spatial self-attention} within each frame, with feed-forward layers and residual connections in between. For images, the temporal step is bypassed, so the encoder reduces to a ViT-like pathway; for videos, both temporal and spatial steps are active. Sharing almost all weights forces a common representational space where image-learned spatial semantics transfer to videos and video-learned temporal cues regularize spatial features. In the decoupled joint curriculum, Stage~1 trains this encoder on image–language data to solidify spatial features; Stage~2 continues training with mixed image+video batches to learn temporal dynamics without forgetting. The final hidden state of \texttt{[CLS]} serves as the global embedding \(v_{\mathrm{cls}}\) for recognition and retrieval, while the full token grid is exposed to the visual-grounded decoders via cross-attention for alignment and generation \cite{wang2022_omnivl}. 
            
            \item \textbf{Text encoder.} A transformer text encoder converts the tokenized caption or prompted label sentence into contextual embeddings, with a \texttt{[CLS]} token used for global text representation \(w_{\mathrm{cls}}\). 
            
            \item \textbf{Visual-grounded decoders.} Two transformer decoders fuse text with the visual tokens for complementary functionalities while sharing the same visual inputs \cite{wang2022_omnivl}. Each decoder block follows the sequence \emph{self-attention} \(\rightarrow\) \emph{cross-attention to visual tokens} \(\rightarrow\) \emph{feed-forward}, with residual connections and layer normalization throughout. The two decoders differ only in the \emph{self-attention mask} and the \emph{supervision signal}, which is the crux of unifying non-generative alignment and generative modeling in one framework.
            \begin{itemize}
                \item \emph{Alignment decoder (non-generative).} A special \texttt{[ENC]} token is prepended to the text sequence. The decoder uses \textbf{bidirectional self-attention}, which is the standard \emph{unmasked} multi-head self-attention as in BERT encoders, so every text token can attend to every other token regardless of position to build a globally contextualized representation. After self-attention, each layer performs \textbf{cross-attention} with the \emph{text stream as queries} and the \emph{visual token grid as keys/values}, grounding the full sentence in the image or video content. The output embedding at \texttt{[ENC]} serves as a fused cross-modal representation that a linear head maps to a match probability for the Vision–Language Matching objective. In retrieval, encoder similarities shortlist candidates and this decoder \emph{re-ranks} them using the \texttt{[ENC]} embedding, improving precision at low recall budgets.
                \item \emph{Generation decoder (generative).} Architecturally mirrors the alignment decoder but replaces bidirectional self-attention with \textbf{causal} self-attention, implemented by a \emph{lower-triangular mask} so token \(l\) only attends to positions \(\le l\). This makes the decoder \emph{autoregressive}, which is necessary for text generation where future tokens must not leak into the current prediction. The text stream is wrapped with a \texttt{[DEC]} start token and an \texttt{[EOS]} end token and is trained with teacher forcing under the language modeling loss. Cross-attention at every layer conditions generation on the visual tokens, enabling captioning and question answering. At inference, the decoder generates tokens sequentially until \texttt{[EOS]} is emitted.
            \end{itemize}
            \emph{Causal vs.\ bidirectional in context.} Bidirectional self-attention equals regular, unmasked MHSA and is ideal for \emph{understanding} tasks that benefit from full-sequence context such as alignment and re-ranking, whereas causal self-attention enforces one-way information flow for \emph{generation} by hiding future tokens, aligning supervision and attention with the autoregressive objective.
            
            \item \textbf{Contrastive memory banks.} OmniVL casts pretraining as \emph{self-supervised} contrastive learning to unify heterogeneous supervision at scale: paired image/video–text samples and class labels provide weak but ubiquitous signals, and contrastive SSL converts these co-occurrences into instance- and class-aware alignment across modalities without dense annotations. To obtain many, \emph{stable} negatives, OmniVL adopts a MoCo-style momentum contrast ~\ref{subsec:chapter22_ssl_moco} \cite{wang2022_omnivl}. \emph{Inputs} are the current mini-batch of visual and text tokens. The \emph{online encoders} produce query projections \((q_v,q_t)\), while \emph{momentum encoders}—exponential moving average (EMA) copies—encode the same batch into keys \((k_v,k_t)\) updated by
            \[
            \theta_{\text{mom}} \leftarrow m\,\theta_{\text{mom}} + (1-m)\,\theta_{\text{online}},\quad m\in[0,1)
            \]
            so targets change slowly and remain consistent across steps. Keys are stop-gradient and \emph{enqueued} into fixed-capacity FIFO queues \(Q_v=\{v_m\}_{m=1}^{M}\), \(Q_t=\{w_m\}_{m=1}^{M}\) with labels \(Q_y=\{y_m\}_{m=1}^{M}\); oldest entries are dequeued to keep size \(M\). \emph{Computation} of UniVLC uses the queues in InfoNCE denominators to supply thousands of negatives, and forms positives from the paired query–key as well as \emph{class-aware} keys with \(y_k{=}y_i\), tying together image–text, video–text, image–label, and video–label signals in one objective. \emph{Why EMA and not large batches.} SimCLR-style training would require very large synchronized batches to approximate this many negatives; EMA keys act as a slowly moving teacher that stabilizes the dictionary, prevents target drift when the online encoder updates, and enables a vast, cheap negative set via queues without increasing memory or cross-device synchronization. \emph{Outputs} are stronger and smoother gradients for cross-modal alignment, leading to more robust representations under mixed corpora and improved optimization stability.
        \end{enumerate}
        
        \paragraph{Pretraining objectives}
        OmniVL learns three \emph{complementary} skills in parallel—global alignment, pairwise verification, and conditional generation—so that one checkpoint can serve retrieval, recognition, and captioning/QA. The three losses operate on different heads but share the unified encoders, so gradients reinforce rather than conflict \cite{wang2022_omnivl}.
        
        \noindent\textbf{Unified Vision–Language Contrastive (UniVLC) loss.}
        \begin{enumerate}
            \item \textit{Purpose.} Build a \emph{shared} metric space in which semantically equivalent visuals and texts co-locate across corpora and modalities.
            \item \textit{Inputs.} For a sample \(S=(x_i,y_i,t_i)\) in batch \(B\), encoders produce \(\ell_2\)-normalized projections \(v_i\) and \(w_i\). Momentum queues provide stored keys \(\{v_m,w_m,y_m\}_{m=1}^M\).
            \item \textit{Positives.} The paired text/visual and \emph{class-aware} positives \(k\in\mathcal{P}(i)=\{k\mid k\in\mathcal{M},\,y_k=y_i\}\) enforce that items sharing label semantics align even when captions differ.
            \item \textit{Negatives.} All other keys in the queues act as negatives, giving large and stable denominators.
            \item \textit{Objective.} With learnable temperature \(\tau\),
            \[
            L_{v2t}(v_i)=-\sum_{k\in\mathcal{P}(i)} \log \frac{\exp\big(v_i^\top w_k/\tau\big)}{\sum_{m=1}^{M}\exp\big(v_i^\top w_m/\tau\big)}\,,\quad
            L_{t2v}(w_i)=-\sum_{k\in\mathcal{P}(i)} \log \frac{\exp\big(w_i^\top v_k/\tau\big)}{\sum_{m=1}^{M}\exp\big(w_i^\top v_m/\tau\big)}\,,
            \tag{1}
            \]
            \[
            L_{\mathrm{UniVLC}}(\,;\theta_{\mathrm{ve}},\theta_{\mathrm{te}})=\tfrac{1}{2}\,\mathbb{E}_{(x_i,y_i,t_i)}\!\left[L_{v2t}(x_i)+L_{t2v}(t_i)\right]
            \tag{2}
            \]
            \item \textit{Effect.} Teaches concept-level alignment such that, e.g., a video labeled \textit{dog catching frisbee}, an image labeled \textit{dog}, and a caption \textit{a golden retriever jumps for a frisbee} cluster together, while unrelated items repel.
        \end{enumerate}
        
        \noindent\textbf{Vision–Language Matching (VLM) loss.}
        \begin{enumerate}
            \item \textit{Purpose.} Learn \emph{pairwise} verification on top of the UniVLC space to answer whether a specific text matches a specific visual.
            \item \textit{Inputs.} Negatives are formed by replacing \(t_i\) with \(t_j\in B\). The alignment decoder cross-attends to visual tokens and emits a probability \(p_{\mathrm{vlm}}\) from the \texttt{[ENC]} embedding via a linear head.
            \item \textit{Objective.} Binary cross-entropy
            \[
            L_{\mathrm{VLM}}(\,;\theta_{\mathrm{ve}},\theta_{\mathrm{ad}})=\mathbb{E}_{(x_i,y_i,t_i)}\!\left[y_{\mathrm{vlm}}\log p_{\mathrm{vlm}}+(1-y_{\mathrm{vlm}})\log(1-p_{\mathrm{vlm}})\right]
            \tag{3}
            \]
            with \(y_{\mathrm{vlm}}{=}1\) if \(j\in B\) and \(y_j=y_i\), else \(0\)
            \item \textit{Effect.} Sharpens decision boundaries for hard cases, e.g., distinguishing \textit{frisbee} vs \textit{ball} when UniVLC already places both near \textit{dog playing}.
        \end{enumerate}
        
        \noindent\textbf{Language Modeling (LM) loss.}
        \begin{enumerate}
            \item \textit{Purpose.} Enable \emph{conditional generation} grounded in visual tokens for captioning and QA.
            \item \textit{Inputs.} The generation decoder is causal and trained with teacher forcing on \([{\tt DEC}]\, t \,[{\tt EOS}]\), cross-attending to visual tokens at every layer.
            \item \textit{Objective.}
            \[
            L_{\mathrm{LM}}(\,;\theta_{\mathrm{ve}},\theta_{\mathrm{gd}})
            =-\,\mathbb{E}_{(x_i,y_i,t_i)}\!\left[\sum_{l=1}^{L}\log P\!\left(t_{i}^{\,l}\mid t_{i}^{\,<l},\,x_i\right)\right]
            \tag{4}
            \]
            \item \textit{Effect.} Pressures visual features to be \emph{descriptive} enough to predict objects, actions, attributes, and relations token by token.
        \end{enumerate}
        
        \noindent\textbf{Putting the skills together}
        \begin{enumerate}
            \item UniVLC provides a coarse but universal geometry where heterogeneous supervision is reconciled.
            \item VLM adds fine-grained pairwise checks that improve retrieval re-ranking and robustness to hard negatives.
            \item LM teaches causal decoding conditioned on the same visual tokens, enriching them with language-predictive cues.
        \end{enumerate}
        The joint objective is the uniform sum
        \[
        L=\lambda_1 L_{\mathrm{UniVLC}}+\lambda_2 L_{\mathrm{VLM}}+\lambda_3 L_{\mathrm{LM}},\quad \lambda_1=\lambda_2=\lambda_3=1
        \tag{5}
        \]
        so the encoders simultaneously become discriminative for recognition, aligned for retrieval, and informative for generation.
        
        \paragraph{Decoupled joint pretraining}
        Two staged phases determine when temporal dynamics are learned while preserving spatial competence \cite{wang2022_omnivl}.
        \begin{itemize}
            \item \textbf{Stage 1: Image–language pretraining} Train on image–text and image–label only to solidify spatial representations while temporal attention is inactive.
            \item \textbf{Stage 2: Joint image+video pretraining} Continue image training and add video–text and video–label so temporal attention is learned incrementally on top of the spatial foundation, avoiding forgetting and yielding bidirectional gains for both image and video tasks.
        \end{itemize}
        
        \paragraph{Task routing and inference}
        A single pretrained checkpoint supports multiple families of tasks without adapters. 
        \begin{itemize}
            \item \textbf{Visual-only recognition.} Use \(v_{\mathrm{cls}}\) for linear probing or fine-tuning on image classification and video action recognition. 
            \item \textbf{Cross-modal alignment.} For retrieval, use encoder similarity to shortlist candidates and the alignment decoder for re-ranking via the \texttt{[ENC]} representation, improving precision at low recall budgets. 
            \item \textbf{Multi-modal generation.} Condition the generation decoder on visual tokens to produce captions or answers, leveraging the LM objective learned during pretraining. 
        \end{itemize}
        This unified path—shared tokenization and positional encoding, one visual backbone with decoupled temporal/spatial attention, a standard text encoder, and two visual-grounded decoders trained under Eqs.\,(1)–(5)—constitutes the OmniVL method and explains how unification across data, modality, and functionality is realized in practice. 
        
        \subsubsection{Architecture \& implementation details}
        \label{subsubsec:chapter24_omnivl_arch}
        
        \paragraph{Backbone design at a glance}
        OmniVL instantiates a single encoder–decoder stack where the visual side is a TimeSformer-style transformer, the language side is a BERT-base encoder, and two lightweight visual-grounded decoders provide alignment and generation heads \cite{wang2022_omnivl,bertasius2021_timesformer,devlin2019_bert}. The key engineering choice is to \emph{share} almost all visual parameters between images and videos while keeping temporal attention conditional on the presence of time, so Stage~1 spatial pretraining transfers directly to Stage~2 temporal learning without adapters \cite{wang2022_omnivl}.
        
        \paragraph{Unified visual encoder: shapes, blocks, and schedules}
        Inputs are tokenized to a common channel dimension \(D\) by modality-specific tokenizers that also add a learned \texttt{[CLS]} token \cite{wang2022_omnivl}. Images \(x\!\in\!\mathbb{R}^{H\times W\times 3}\) use a 2D patch embed with kernel/stride \(p\times p\), giving \(N=\tfrac{HW}{p^2}\!+\!1\) tokens including \texttt{[CLS]}. Videos \(x\!\in\!\mathbb{R}^{T\times H\times W\times 3}\) use a 3D tubelet embed with kernel/stride \(\tau\times p\times p\), yielding \(N'=\tfrac{T}{\tau}\cdot\tfrac{HW}{p^2}\!+\!1\) tokens that encode short-range motion. Learned positional encodings are \emph{factorized} as spatial \(E_s(h,w)\) and temporal \(E_t(t)\) and summed with token embeddings, enabling weight sharing across modalities and robust interpolation across resolutions and clip lengths \cite{wang2022_omnivl}. Each transformer block is \emph{pre-norm} and applies temporal self-attention then spatial self-attention with MLP in between, all with residual connections. The temporal step is \emph{skipped} for images, so the block reduces to a ViT-style layer for \(T{=}1\). Default backbone follows a ViT-B/16 scale for capacity and speed balance, with stochastic depth and token dropout used as regularization in long videos when applicable \cite{bertasius2021_timesformer,wang2022_omnivl}. The last \texttt{[CLS]} state forms the global visual embedding \(v_{\mathrm{cls}}\), while all patch or tubelet tokens feed the decoders through cross-attention for grounding \cite{wang2022_omnivl}.
        
        \paragraph{Text encoder: tokenization and heads}
        A BERT-base encoder produces contextual text features from WordPiece tokenization with special tokens reserved for \texttt{[ENC]} and \texttt{[DEC]} control and \texttt{[EOS]} termination \cite{devlin2019_bert,wang2022_omnivl}. The \texttt{[CLS]} output \(w_{\mathrm{cls}}\) serves retrieval and contrastive alignment via a projection to the shared embedding space. Prompted label sentences and free-form captions share the same tokenizer and vocabulary so UniVLC sees a unified language interface for both clean labels and noisy web text \cite{wang2022_omnivl}.
        
        \paragraph{Decoders: attention masks, fusion, and outputs}
        Both decoders are initialized from BERT-base and stack blocks of self-attention \(\rightarrow\) cross-attention to visual tokens \(\rightarrow\) MLP with residuals and layer normalization \cite{wang2022_omnivl}. The \emph{alignment} decoder uses \emph{bidirectional} (unmasked) self-attention over the full text and prepends \texttt{[ENC]} whose output embedding becomes a fused cross-modal representation for VLM scoring and retrieval re-ranking. The \emph{generation} decoder is identical but uses \emph{causal} masking so position \(l\) only attends to \(\le l\) and wraps the sequence with \texttt{[DEC]} and \texttt{[EOS]} for autoregressive decoding under the LM objective. In both decoders, cross-attention queries come from the text stream and keys/values are the full visual token grid, letting the text resolve to relevant spatial or temporal evidence \cite{wang2022_omnivl}.
        
        \paragraph{Projection heads, similarities, and temperatures}
        Contrastive learning uses lightweight heads that map \(v_{\mathrm{cls}}\) and \(w_{\mathrm{cls}}\) to a common dimension and \(\ell_2\)-normalize them, so similarity is cosine with a \emph{learnable} temperature \(\tau\) as in Eqs.\,(1)–(2). The VLM head is a linear classifier on the \texttt{[ENC]} output. The LM head ties to the text embedding matrix by default to stabilize generation and reduce parameters \cite{wang2022_omnivl}.
        
        \paragraph{Queues, EMA encoders, and retrieval runtime}
        OmniVL implements MoCo-style momentum encoders and FIFO queues for visual keys, text keys, and labels to scale UniVLC to many stable negatives without large synchronous batches \cite{wang2022_omnivl}. EMA parameters update at high momentum so the key distribution drifts slowly, improving InfoNCE stability across steps. At inference for retrieval, a two-stage path is used: encoder similarities produce a top-\(K\) shortlist, then the alignment decoder re-ranks using the \texttt{[ENC]} representation for better precision at tight recall budgets \cite{wang2022_omnivl}.
        
        \paragraph{Data, batching, and curriculum specifics}
        Stage~1 samples \(\sim14\)M image–text pairs from COCO, Visual Genome, CC3M, CC12M, and SBU and converts ImageNet-1K labels to prompted sentences so image–text and image–label are trained together under UniVLC. Augmentations are standard resize–crop, color jitter, and horizontal flip for images with caption sampling, and random clip sampling for videos. Stage~2 mixes image batches with video–text and video–label data, enabling temporal attention while preserving image supervision so spatial features are not forgotten. Clips are typically \(8\times224^2\), with temporal PE learned and interpolated when clip length changes at fine-tuning \cite{wang2022_omnivl}.
        
        \paragraph{Optimization and training stability}
        Training uses AdamW with warmup then cosine decay, gradient scaling in mixed precision, and gradient clipping for stability \cite{wang2022_omnivl}. The decoupled joint curriculum aligns the optimization landscape: UniVLC shapes a coarse cross-modal geometry early, VLM sharpens pairwise decisions when encoders are already aligned, and LM enriches visual tokens with language-predictive cues. Positional embeddings \(E_s\) and \(E_t\) are interpolated when transferring to new resolutions or clip lengths, which avoids reinitialization and preserves the learned geometry \cite{wang2022_omnivl}.
        
        \subsubsection{Experiments and ablations}
        \label{subsubsec:chapter24_omnivl_expts}
        
        \paragraph{Result highlights}
        With a ViT-B scale backbone and moderate pretraining data, a single OmniVL checkpoint is competitive or state of the art across image–text retrieval and captioning on COCO and Flickr30K, video–text retrieval on MSRVTT, video QA on MSVD and MSRVTT, and strong visual-only recognition under linear probing and fine-tuning \cite{wang2022_omnivl}. Retrieval uses a two-stage pipeline: encoder cosine similarity for top-\(K\) preselection and alignment-decoder re-ranking with the \texttt{[ENC]} embedding for final ordering, which consistently lifts precision at tight budgets \cite{wang2022_omnivl}.
        
        \begin{table}[H]
            \centering
            \small
            \setlength{\tabcolsep}{6pt}
            \caption{Comparison across pretraining schedules from the paper’s Table~10. Metrics: COCO retrieval (TR@1, IR@1), MSRVTT retrieval (IR@1), COCO captioning (BLEU@4, CIDEr), VQA (test-dev), and MSRVTT-QA accuracy.}
            \label{tab:chapter24_omnivl_ablation_pretrain}
            \resizebox{0.90\linewidth}{!}{
                \begin{tabular}{lccccccc}
                    \toprule
                    \textbf{Pretraining} & \textbf{COCO TR@1} & \textbf{COCO IR@1} & \textbf{MSRVTT IR@1} & \textbf{COCO B@4} & \textbf{COCO C} & \textbf{VQA dev} & \textbf{MSRVTT(QA) acc} \\
                    \midrule
                    Without Pretraining         & 37.1 & 28.5 &  9.6 & 27.4 &  80.0 & 39.51 & 36.6 \\
                    Video-only                  &   -- &   -- & 13.7 &   -- &    -- &   --  & 15.8 \\
                    Image-only                  & 80.9 & 63.0 & 38.2 & 39.3 & 131.6 & 77.62 & 40.8 \\
                    Joint (scratch)             & 50.2 & 35.0 & 23.6 & 29.7 &  94.6 & 47.78 & 38.8 \\
                    Img2Vid                     & 79.7 & 61.8 & 42.5 & 38.6 & 129.5 & 77.43 & 42.8 \\
                    \textbf{Decoupled Joint (OmniVL Full)} & \textbf{82.1} & \textbf{64.8} & \textbf{47.8} & \textbf{39.8} & \textbf{133.9} & \textbf{78.33} & \textbf{44.1} \\
                    \bottomrule
                \end{tabular}
            }
        \end{table}
        
        
        \paragraph{What the curriculum buys}
        \begin{itemize}
            \item \textbf{Joint from scratch fails} mixing image and video from iteration zero underperforms sharply across all tasks, indicating unstable co-optimization of spatial and temporal cues without a strong spatial prior.
            \item \textbf{Image-only is strong but asymmetric} excellent image-side metrics yet limited transfer to video retrieval, revealing the gap in temporal understanding.
            \item \textbf{Img2Vid narrows the gap} image pretraining followed by video-only brings video gains while slightly regressing image metrics, suggesting partial forgetting.
            \item \textbf{Decoupled Joint wins} image pretraining then \emph{joint} image+video yields the best of both worlds in Table~\ref{tab:chapter24_omnivl_ablation_pretrain}, supporting the design that temporal learning should be layered on top of a solid spatial manifold.
        \end{itemize}
        
        \paragraph{What UniVLC adds}
        \begin{itemize}
            \item \textbf{Consistent gains across modalities} enabling UniVLC improves retrieval, captioning, VQA, and visual-only recognition in Fig.~\ref{fig:chapter24_omnivl_univlc_ablation}, validating the unified contrastive hypothesis.
            \item \textbf{Class-aware positives matter} treating samples sharing the same label as additional positives ties together image–label, video–label, and caption supervision, sharpening category structure while keeping cross-modal alignment.
            \item \textbf{Scalable negatives stabilize learning} momentum queues supply thousands of stable negatives per step, yielding smoother InfoNCE optimization than large-batch alternatives under mixed corpora.
        \end{itemize}
        
        \paragraph{Retrieval pipeline ablation}
        \begin{itemize}
            \item \textbf{Top-\(K\) then re-rank} encoder cosine similarity produces a compact shortlist and the alignment decoder re-ranks using the \texttt{[ENC]} embedding, improving R@1 at fixed compute budgets compared to single-stage scoring \cite{wang2022_omnivl}.
            \item \textbf{Effect beyond retrieval} the VLM head trained for re-ranking also refines the shared encoders via cross-attention, which correlates with small but repeatable gains on captioning and QA.
        \end{itemize}
        
        \paragraph{Takeaways}
        \begin{itemize}
            \item \textbf{Curriculum is essential} learn space first on images, then add time with joint training to avoid forgetting and to unlock bidirectional transfer between modalities.
            \item \textbf{Unification pays off} one embedding space supervised by labels and captions plus two decoders covers alignment and generation without task-specific adapters.
            \item \textbf{Engineering matters} factorized PEs, EMA queues, and two-stage retrieval make the method robust at ViT-B scale with moderate data, while leaving clear headroom for larger backbones or longer clips.
        \end{itemize}
        
        \begin{figure}[H]
            \centering
            \includegraphics[width=0.85\linewidth]{Figures/Chapter_24/OmniVL_evaluation.jpg}
            \caption{With and without UniVLC across tasks, showing consistent gains that support the unified contrastive design \cite{wang2022_omnivl}.}
            \label{fig:chapter24_omnivl_univlc_ablation}
        \end{figure}
        
        \subsubsection{Limitations and future directions}
        \label{subsubsec:chapter24_omnivl_limits}
        
        \paragraph{Token budget and long-form video}
        Quadratic self-attention constrains clip length and resolution, so OmniVL is most practical on short segments.
        Promising remedies include hierarchical token selection and pooling, streaming or memory-augmented attention, and keyframe or tubelet merging; efficient spatiotemporal stacks in InternVideo2 ~\cite{wang2024_internvideo2} and long-context segment pipelines in LongVLM ~\cite{weng2024_longvlm} point to viable paths forward.
        
        \paragraph{Prompting sensitivity and text targets}
        UniVLC relies on prompt-engineered label sentences, which can bias positives and underdescribe actions.
        Learnable prompts, LLM-augmented captions, and multi-view text targets (titles, ASR transcripts, and dense narrations) improve robustness; recent pipelines in InternVideo2~\cite{wang2024_internvideo2} and VideoLLaMA-style~\cite{cheng2024_videollama2} instruction data illustrate how to replace brittle templates with richer supervision.
        
        \paragraph{Fine-grained localization and grounding}
        Global tokens such as \texttt{[CLS]} and \texttt{[ENC]} dominate, which weakens region- and moment-level sensitivity.
        Future variants should supervise phrase–patch and question–moment alignment, expose tubelet tokens to grounding heads, and add localized losses; stronger local features in InternVideo-style models and temporal localization modules in video LLMs are practical next steps.
        
        \paragraph{Data curation and balance}
        Performance depends on the mix of clean labels and noisy captions, as well as shot boundaries and temporal coherence.
        Shot-aware segmentation, quality-aware reweighting, and curriculum schedules can raise the signal-to-noise ratio, while fusing audio, ASR, and summary text stabilizes unified contrastive training and broadens semantics.
        
        \paragraph{From unified encoders to instruction following}
        OmniVL excels at retrieval, recognition, and captioning under fixed prompts but lacks multi-turn instruction following.
        Bridging to video LLMs via lightweight adapters or decoder replacement enables instruction tuning and tool use while reusing OmniVL visual tokens.
        
        \paragraph{Scaling outlook}
        Larger yet efficient backbones, longer-context attention, learned prompts, grounding objectives, and higher quality multi-view text complement the decoupled joint recipe.
        These directions connect OmniVL to InternVideo2 (covered next, ~\ref{enr:subsec_chapter24_internvideo2}) for efficient spatiotemporal modeling and to long-context video LLMs such as LongVLM ~\ref{enr:subsec_chapter24_longvlm} and VideoLLaMA ~\ref{enr:subsec_chapter24_videollama1} for instruction following and reasoning.
        
    \end{enrichment}
    
    \newpage
    
    \begin{enrichment}[InternVideo2: Generative + Discriminative Pretraining][subsection]
        \label{enr:subsec_chapter24_internvideo2}
        
        \paragraph{Scope and positioning}
        InternVideo2 \cite{wang2024_internvideo2} is a three-stage, multimodal pretraining framework that scales video encoders and aligns them with text and large language models (LLMs). The aim is broad transfer across action recognition, video–text retrieval, temporal localization, and video-centric dialogue, with emphasis on long-form understanding and procedure-aware reasoning.
        
        \subsubsection{Motivation}
        \label{subsubsec:chapter24_internvideo2_motivation}
        
        \paragraph{Problem framing}
        A general-purpose video foundation model (VFM) must address three complementary needs:
        \begin{itemize}
            \item Strong \emph{video-only} representations that capture motion, long-range temporal structure, and scene dynamics beyond frame-level appearance.
            \item Reliable \emph{multimodal alignment} that ties video to language and audio streams (captions, ASR, raw audio) for concept naming, temporal grounding, and cross-modal retrieval.
            \item A practical route to \emph{instruction-following and reasoning} with LLMs, enabling models to answer, explain, and plan over long videos.
        \end{itemize}
        Earlier recipes such as InternVideo~\cite{wang2022_internvideo} combined masked reconstruction and contrastive learning with cross-model attention, but they left open stable co-adaptation of early features, robust use of audio and speech, and a scalable path to long-form dialogue.
        
        \paragraph{Why InternVideo (V1) is not enough}
        Three limitations motivated a new design:
        \begin{itemize}
            \item \textbf{Limited co-adaptation.} Freezing backbones around cross-attention ensured stability but prevented early layers from becoming more motion-sensitive.
            \item \textbf{Short-horizon focus.} Clip-centric objectives did not teach temporal commonsense such as ordering, counting, and multi-step procedures needed by video QA.
            \item \textbf{Underused modalities.} Alignment leaned on text; audio and speech were not systematically integrated, and caption quality and segmentation limited supervision density.
        \end{itemize}
        
        \paragraph{Design principles for a scalable VFM}
        InternVideo2 follows a curriculum that separates concerns while controlling compute:
        \begin{itemize}
            \item \textbf{Decouple objectives.} First learn motion-aware visual priors; then attach language and audio at scale.
            \item \textbf{Spend compute where it pays off.} Use self-supervised video learning before costly caption-based and LLM tuning stages.
            \item \textbf{Summarize for long context.} Compress long videos into a small token set before handing them to an LLM to avoid quadratic attention costs.
            \item \textbf{Fuse multi-source captions.} Combine video, audio, and speech captions to improve temporal grounding beyond alt-text alone.
        \end{itemize}
        
        \paragraph{What success looks like}
        A successful VFM should retain strong image-level semantics while adding temporal sensitivity, align video tokens with language and acoustic evidence for retrieval and grounding, and converse over long videos by tracking entities, ordering events, and following instructions without prohibitive compute.
        
        \newpage
        
        \paragraph{Key idea}
        InternVideo2 adds capability in three short, progressive stages, keeping each objective simple and stable.
        
        \begin{itemize}
            \item \textbf{Stage~1: Learn motion-aware visual priors (video-only).} Train a ViT-style video encoder with masked autoencoding over tubelets to capture spatiotemporal structure and reduce redundancy. Intuition: aggressive tube masking forces integration over time, yielding robust motion-sensitive features suitable for transfer.
            \item \textbf{Stage~2: Align vision to language (and audio/speech) at scale.} Freeze the idea of “what is seen” and learn “what it means” by contrastively aligning video features with paired captions and transcripts, optionally including audio or speech signals. Intuition: alignment turns generic visual tokens into semantically grounded representations without changing the core video encoder too much.
            \item \textbf{Stage~3: Enable dialogue and reasoning with an LLM.} Insert a lightweight \emph{query former} (Q-Former, as seen in BLIP2 ~\ref{enr:subsec_chapter24_blip2})—a small attention module whose few learnable queries summarize a long video into \(K\) informative tokens—and feed these tokens to a pretrained LLM. Adapt the LLM with parameter-efficient fine-tuning (\emph{LoRA}; see \S\ref{subsubsec:chapter22_peft}) while training the Q-Former. Intuition: the Q-Former provides a compact “briefing” an LLM can reason over; LoRA preserves general language ability while specializing to video tasks at low cost.
        \end{itemize}
        
        \noindent
        This staged recipe avoids brittle all-at-once training by cleanly separating perception, grounding, and reasoning; it scales with data and parameters, leverages richer VAS (Video-Audio-Speech) supervision, and culminates in a conversational video agent with long-context competence~\cite{wang2024_internvideo2}.
        
        \begin{figure}[H]
            \centering
            \includegraphics[width=0.85\textwidth]{Figures/Chapter_24/InternVideo2_introduction.jpg}
            \caption{High-level overview and qualitative capabilities of InternVideo2 across recognition, retrieval, long-form reasoning, and dialogue. Source: \cite{wang2024_internvideo2}.}
            \label{fig:chapter24_iv2_intro}
        \end{figure}
        
        \subsubsection{Method: objectives, training stages, and intuition}
        \label{subsubsec:chapter24_internvideo2_method}
        \paragraph{Notation}
        Let a video clip be $x\in\mathbb{R}^{T\times H\times W\times 3}$ and a text sequence be $y=(y_1,\dots,y_M)$. Denote a video encoder $f_\theta$ that maps $x$ to token features $Z\in\mathbb{R}^{L\times D}$, where $L=T\,H'\,W'$ after patchifying, and a text encoder $g_\phi$ that maps $y$ to $u\in\mathbb{R}^{D}$. Let $\text{sg}[\cdot]$ be stop-gradient.
        
        \paragraph{Stage 1: Video-only masked autoencoding}
        Following MAE \cite{he2022_mae}, InternVideo2 applies high video-tube masking to tokens of $x$. Let $M\subseteq\{1,\dots,L\}$ index masked tokens and $\bar M$ the visible ones. The encoder processes only $Z_{\bar M}$; a lightweight decoder $h_\psi$ reconstructs masked pixels for indices in $M$.
        \begin{equation}
            \mathcal{L}_{\text{MAE}} \;=\; \frac{1}{|M|} \sum_{i\in M} \big\| h_\psi\big( f_\theta(x)_{\bar M} \big)_i \;{-}\; x_i \big\|_2^2
            \label{eq:chapter24_iv2_mae}
        \end{equation}
        \emph{Intuition} Tube masking suppresses short-term redundancy, forcing the encoder to integrate spatial cues over time while keeping compute focused on informative tokens.
        
        \paragraph{Stage 2: Multimodal contrastive alignment (image–text and video–text)}
        Stage~2 turns the Stage~1 video encoder $f_\theta$ into a multimodal aligner by pairing it with a \emph{language-pretrained} text encoder $g_\phi$ (Transformer-based) and, when used, \emph{audio/speech} encoders (initialized from acoustic pretraining). Each encoder is followed by a small MLP projection head that maps pooled tokens (e.g., mean or [CLS]) to $d$-dimensional, $\ell_2$-normalized embeddings. InternVideo2 then adopts a \emph{CLIP-style} contrastive objective to learn a shared space for retrieval and grounding (background: \S\ref{subsec:chapter22_ssl_clip}; InfoNCE in \eqref{subsec:chapter22_ssl_contrastive_loss_foundation}). Concretely, pooled video features map to $v\in\mathbb{R}^d$ and captions to $t\in\mathbb{R}^d$; with temperature $\tau$, the symmetric in-batch InfoNCE is
        \begin{equation}
            \mathcal{L}_{\text{CLIP}} \;=\; -\frac{1}{|\mathcal{B}|} \sum_{(x,y)\in\mathcal{B}} \Big[ \log \frac{\exp(v_x^\top t_y/\tau)}{ \sum\limits_{y'\in\mathcal{B}} \exp(v_x^\top t_{y'}/\tau) } \;+\;
            \log \frac{ \exp(t_y^\top v_x/\tau) }{ \sum\limits_{x'\in\mathcal{B}} \exp(t_y^\top v_{x'}/\tau) } \Big].
            \label{eq:chapter24_iv2_clip}
        \end{equation}
        \emph{Clarification.} Stage~2 follows CLIP’s \emph{loss formulation}, not necessarily CLIP weight initialization: $f_\theta$ is initialized from Stage~1, while $g_\phi$ (and optional acoustic encoders) start from their own modality-specific pretraining and are trained with lightweight projection heads. \emph{Why it helps.} Large, diverse image–text and video–text corpora tie visual tokens to semantics; InternVideo2 further strengthens temporal grounding using VAS captions—single captions \emph{fused} from video, audio, and ASR transcripts—so supervision reflects what is \emph{seen}, \emph{heard}, and \emph{spoken}, not alt-text alone.
        
        \paragraph{Stage 3: Video-centric instruction tuning with a Q-Former bridge}
        \textbf{What it does.} Stage~3 equips the model with dialogue and high-order reasoning by inserting a lightweight \emph{Q-Former} $q_\omega$ between the video encoder $f_\theta$ and a pretrained LLM $\ell_\gamma$. The Q-Former contains a small set of learnable query tokens $Q^{(0)}\!\in\!\mathbb{R}^{K\times d}$ that \emph{self-attend} and then \emph{cross-attend} to the dense spatiotemporal features $Z\!=\!f_\theta(x)\!\in\!\mathbb{R}^{L\times D}$, producing a compact summary $Q\!\in\!\mathbb{R}^{K\times d}$ that the LLM can consume efficiently. \emph{Why this is useful.} Instead of feeding all $L$ video tokens to the LLM, the Q-Former compresses evidence into $K\!\ll\!L$ tokens, giving a stable, fixed-size interface that preserves salient temporal events while avoiding overwhelming the language model. \emph{Why this is efficient.} Reducing sequence length from $L$ to $K$ lowers the LLM’s quadratic attention cost, allows larger temporal coverage for the same compute budget, and confines most trainable parameters to the Q-Former and LoRA adapters rather than the full LLM.
        
        \medskip
        \noindent
        \textbf{How it works (mechanism).} The Q-Former converts a \emph{long} sequence of video tokens into a \emph{fixed length short} sequence that an LLM can use. It maintains a small set of $K$ learnable \emph{query tokens}. In each layer, the queries first refine themselves (self-attention) and then read from the video tokens (cross-attention).
        
        \paragraph{Notation (simplified)}
        \begin{itemize}
            \item $Z \in \mathbb{R}^{L \times D}$: spatiotemporal video tokens from the video encoder $f_\theta$ (length $L$, dim $D$).
            \item $Q^{(l)} \in \mathbb{R}^{K \times d}$: the $K$ query tokens \emph{entering} Q-Former layer $l$ (dim $d$).
            \item $W_K, W_V$: small linear maps projecting $Z$ into Keys and Values for cross-attention.
            \item $W_p$: small linear map projecting Q-Former output to the LLM embedding size $d_\ell$.
        \end{itemize}
        
        \paragraph{One Q-Former layer}
        \begin{align}
            \textbf{(1) Self-attention (queries talk to each other):}\quad
            & Q_{\text{SA}}^{(l)} \;=\; \mathrm{SelfAttn}\!\left(Q^{(l)}\right). \label{eq:qformer-sa} \\
            \textbf{(2) Cross-attention (queries read from video):}\quad
            & Q^{(l+1)} \;=\; \mathrm{CrossAttn}\!\left(Q_{\text{SA}}^{(l)},\, ZW_K,\, ZW_V\right). \label{eq:qformer-ca}
        \end{align}
        
        \noindent
        After $L_q$ layers, we obtain the final queries $Q^{(L_q)} \in \mathbb{R}^{K \times d}$. These are projected to the LLM’s embedding size and used as a short, informative visual prompt:
        \begin{equation}
            \widetilde{Q} \;=\; Q^{(L_q)} W_p \;\in\; \mathbb{R}^{K \times d_\ell}.
        \end{equation}
        
        \noindent
        \emph{How it is used.} The $K$ tokens $\widetilde{Q}$ are \emph{prepended} to the text prompt embeddings and fed to the LLM. This preserves salient temporal information in a compact form and is efficient because $K \ll L$, reducing the LLM’s sequence length and the quadratic attention cost.
        
        \medskip
        \noindent
        \textbf{Training objective.} Training uses next-token prediction over video-centric instruction and dialogue data $\mathcal{D}_{\text{dlg}}$, while updating only lightweight LoRA adapters in $\ell_\gamma$ (see \S\ref{subsubsec:chapter22_peft}) and training $q_\omega$ end-to-end:
        \begin{equation}
            \mathcal{L}_{\text{LM}} \;=\; - \mathbb{E}_{(x,\text{prompt}, y_{1:M}) \sim \mathcal{D}_{\text{dlg}}} \sum_{m=1}^{M} \log p_{\ell_\gamma}\!\big( y_m \mid y_{<m},\, \widetilde{Q}(x),\, \text{prompt} \big).
            \label{eq:chapter24_iv2_lm}
        \end{equation}
        
        \textbf{How BLIP-2’s image Q-Former is adapted to video.}
        \begin{itemize}
            \item \emph{From images to spatiotemporal tokens.} BLIP-2’s Q-Former cross-attends to 2D image tokens; here $q_\omega$ cross-attends to \emph{3D tubelet tokens} $Z$ with explicit temporal positional encodings, enabling queries to integrate evidence across frames and motion patterns, not just spatial layouts.
            \item \emph{Temporal coverage at fixed cost.} Instead of passing all $L$ video tokens to the LLM, the Q-Former compresses them into $K\!\ll\!L$ tokens. This reduces LLM sequence length and avoids quadratic attention costs while preserving temporal structure through cross-attention.
            \item \emph{Long-context scheduling.} For long videos, tokens are obtained from sampled frames and multi-view crops plus a global view; the Q-Former’s cross-attention spans all selected tokens, so each query can aggregate events dispersed across time and space.
            \item \emph{Stable division of labor.} As in BLIP-2, the vision side (here, the Stage~2 video encoder) remains frozen or slowly updated to keep visual features stable; the Q-Former learns the interface, and the LLM is adapted with LoRA to minimize trainable parameters and preserve general language competence.
            \item \emph{LLM-facing interface.} A linear adapter $W_p$ matches dimensions, and the $\widetilde{Q}$ tokens act as a soft visual prompt prepended to the textual prompt, mirroring BLIP-2’s “visual prompt” design for images.
        \end{itemize}
        
        \textbf{Why it helps.} The Q-Former delivers a concise, semantically focused visual prompt that lets the LLM perform video question answering, temporal ordering, and procedure tracing without handling long spatiotemporal sequences. LoRA then adjusts only small adapters to align the LLM to video-grounded instructions while preserving its general language competence.
        
        \begin{figure}[H]
            \centering
            \includegraphics[width=0.85\textwidth]{Figures/Chapter_24/blip2_qformer.jpg}
            \caption{Q-Former–LLM interface adapted from BLIP-2: a small set of learnable queries cross-attend to video tokens to produce a compact summary that conditions the LLM via soft visual prompts. Source: \cite{li2023_blip2}.}
            \label{fig:chapter24_blip2_qformer}
        \end{figure}
        
        \paragraph{Total training recipe}
        Stages are executed \emph{sequentially} with explicit initialization and a selective update policy that carries forward only what is needed next:
        \begin{equation}
            \begin{aligned}
                &\min_{\theta,\psi}\; \mathbb{E}\!\left[\mathcal{L}_{\text{MAE}}\right]
                \quad \text{(Stage 1: learn motion-aware priors on video)}\\[4pt]
                &\xrightarrow{\;\text{initialize } f_\theta\;}
                \min_{\theta,\phi}\; \mathbb{E}\!\left[\mathcal{L}_{\text{CLIP}}\right]
                \quad \text{(Stage 2: align video to language (and audio/speech))}\\[4pt]
                &\xrightarrow{\;\text{initialize } q_\omega,\, \ell_\gamma\;}
                \min_{\omega,\gamma}\; \mathbb{E}\!\left[\mathcal{L}_{\text{LM}}\right]
                \quad \text{(Stage 3: instruction tuning via Q-Former \& LoRA).}
            \end{aligned}
            \label{eq:chapter24_iv2_total}
        \end{equation}
        
        \noindent\textit{What is updated and what is reused.}
        \begin{itemize}
            \item \textbf{Stage 1} updates $(\theta,\psi)$ to learn video priors; only $f_\theta$ is kept and the MAE decoder $h_\psi$ is discarded.
            \item \textbf{Stage 2} initializes from $f_\theta$ and adds $g_\phi$ (and optional acoustic encoders) with small projection heads; it updates $(\theta,\phi)$ and the heads under the contrastive loss, producing language-aligned video features.
            \item \textbf{Stage 3} keeps the Stage~2 video encoder stable (frozen or slow-updated), initializes $q_\omega$ and a pretrained LLM $\ell_\gamma$, and trains only $q_\omega$ plus \emph{LoRA} adapters inside $\ell_\gamma$ for next-token prediction.
        \end{itemize}
        
        \noindent\textit{Why this curriculum.}
        \begin{itemize}
            \item \textbf{Decoupled difficulty.} Stage~1 learns motion and structure without language; Stage~2 adds semantics; Stage~3 adds reasoning, avoiding interference between heterogeneous objectives.
            \item \textbf{Sample and compute efficiency.} Self-supervision bootstraps $f_\theta$ cheaply, so alignment and instruction tuning spend compute where supervision is strongest.
            \item \textbf{Stable interfaces.} The Q-Former forms a small, fixed-size interface to the LLM, keeping context length and trainable parameters bounded.
        \end{itemize}
        
        \paragraph{Practical schedule and hyperparameters}
        \begin{itemize}
            \item \textbf{Stage 1 (MAE).} Tubelet size and masking ratio are chosen to emphasize motion (e.g., $90\%$ masking). AdamW with cosine decay, warmup, mixed precision, and gradient clipping are used. The MAE decoder is lightweight to focus capacity on $f_\theta$.
            \item \textbf{Stage 2 (contrastive).} Large effective batch sizes are preferred for strong in-batch negatives; a learned temperature $\tau$ stabilizes InfoNCE. Early layers of $f_\theta$ may be briefly frozen, then unfrozen as alignment stabilizes. VAS captions densify supervision over temporally coherent clips.
            \item \textbf{Stage 3 (Q-Former + LoRA).} Queries $K$ are kept small (dozens) to cap LLM context. Frame sampling mixes sparse long-range and short local windows. Only Q-Former weights and LoRA adapters are trained; the LLM backbone remains frozen to preserve general language competence.
        \end{itemize}
    
        \subsubsection{Experiments}
        
        \paragraph{Experimental setup and scaling}
        InternVideo2 is evaluated by stage (IV2-s1, IV2-s2, IV2-s3), by backbone size (1B, 6B), and across task families: action understanding, video–audio–language alignment, and video-centric dialogue. Training uses a heterogeneous corpus that mixes image–text pairs, web video–text pairs, and VAS-enhanced clips, as described in \cite{wang2024_internvideo2}. Metrics reported include zero-shot and finetuned retrieval (Recall@K), classification accuracy, temporal localization mAP, and multiple-choice accuracy for dialogue. \emph{Intuition:} These metrics probe complementary abilities: recognition scores reflect \emph{perceptual priors}, retrieval/grounding measure \emph{semantic alignment} quality, and dialogue QA evaluates \emph{reasoning over temporally extended evidence}.
        
        \paragraph{Efficiency and compute}
        The three-stage curriculum concentrates compute where leverage is highest \cite{wang2024_internvideo2}. Stage~1 builds motion-aware priors without captions; Stage~2 adds semantics via contrastive learning (benefiting from large in-batch negatives and cross-modal data); Stage~3 adapts only a compact Q-Former and LoRA adapters \cite{hu2021_lora} instead of fully finetuning the LLM. \emph{Meaning:} Most parameters remain frozen in later stages, so new abilities are acquired by training small interfaces. This keeps memory/latency predictable while enabling long-context reasoning with modest additional parameters and stable optimization.
        
        \paragraph{Headline results}
        InternVideo2’s three-stage recipe yields strong performance across four capability areas and compares favorably to prior video and video–language methods reported in \cite{wang2024_internvideo2}.
        
        % -------------------- ACTION UNDERSTANDING --------------------
        \paragraph{Action understanding (``what'' and ``when'')}
        \emph{Key idea:} Stage~1 tube-masked pretraining learns motion and temporal boundaries, improving not only \emph{what} an action is but also \emph{when} it occurs.
        
        \begin{table}[H]
            \centering
            \caption{Temporal action localization (avg.\ mAP) and video instance segmentation (mAP). Figures are reported in \cite{wang2024_internvideo2}. Datasets: THUMOS14~\cite{jiang2014_thumos14}, ActivityNet-Captions~\cite{krishna2017_activitynet_captions}, HACS~\cite{zhao2019_hacs}, YouTube-VIS19~\cite{yang2019_youtube_vis}. VIS baselines include Swin-L~\cite{liu2021_swin} and an image InternViT backbone (as referenced by \cite{wang2024_internvideo2}).}
            \label{tab:iv2_action}
            \scriptsize
            \begin{tabular}{lcc}
                \hline
                \textbf{Benchmark} & \textbf{Metric} & \textbf{IV2 figure (stage/model)} \\
                \hline
                THUMOS14~\cite{jiang2014_thumos14} & avg.\ mAP & \textbf{72.0} (s1, 6B) \\
                ActivityNet (cap.)~\cite{krishna2017_activitynet_captions} & avg.\ mAP & \textbf{41.2} (s1, 6B) \\
                HACS~\cite{zhao2019_hacs} & avg.\ mAP & \textbf{43.3} (s1, 6B) \\
                \hline
                YouTube-VIS19~\cite{yang2019_youtube_vis} & mAP (Mask2Former + IV2-s1) & \textbf{64.2} \\
                YouTube-VIS19~\cite{yang2019_youtube_vis} & mAP (Mask2Former + Swin-L~\cite{liu2021_swin}) & 60.3 \\
                YouTube-VIS19~\cite{yang2019_youtube_vis} & mAP (Mask2Former + image InternViT) & 63.4 \\
                \hline
            \end{tabular}
        \end{table}
        
        \begin{table}[H]
            \centering
            \caption{Finetuned temporal action localization (avg.\ mAP). “Flow” uses ensembled I3D flow features; * with Flow. (From \cite[Tab.~7]{wang2024_internvideo2}.)}
            \label{tab:iv2_tal_compare}
            \scriptsize
            \begin{tabular}{lcccc}
                \hline
                \textbf{Backbone / Method} & \textbf{THUMOS14} & \textbf{HACS} & \textbf{ActivityNet} & \textbf{FineAction} \\
                \hline
                I3D + Flow~\cite{carreira2017_i3d} & 66.8 & -- & 35.6 & -- \\
                R(2+1)D~\cite{tran2018_closer}     & 55.6 & -- & 36.6 & -- \\
                InternVideo (V1)~\cite{wang2022_internvideo} & 71.6$^\ast$ & 41.3 & 39.0 & 17.6 \\
                VideoMAE-v2-g~\cite{wang2023_videomaev2} & 69.5 & -- & -- & 18.2 \\
                \hline
                \textbf{InternVideo2\textsubscript{s1}-1B} & 69.8 & 42.4 & 40.4 & 27.2 \\
                \textbf{InternVideo2\textsubscript{s1}-6B} & \textbf{72.0} & \textbf{43.3} & \textbf{41.2} & \textbf{27.7} \\
                \hline
            \end{tabular}
        \end{table}
        
        \noindent
        \emph{Context and comparison (per paper):} For TAL, InternVideo2\textsubscript{s1} compares against strong video-pretrained backbones including VideoMAE/V2~\cite{tong2022_videomae,wang2023_videomaev2} and InternVideo (V1)~\cite{wang2022_internvideo}, achieving the strongest or on-par avg.\ mAP across THUMOS14, ActivityNet, and HACS. For VIS, swapping in the IV2\textsubscript{s1} backbone improves over Swin-L and an image InternViT backbone, indicating motion-aware features transfer beyond recognition \cite{wang2024_internvideo2}. \emph{Intuition:} Tube masking suppresses short-term redundancy and forces temporal integration, yielding features that localize \emph{boundaries} rather than only classify frames.
        
        \paragraph{Video–language retrieval (the ``search engine'')}
        \emph{Key idea:} Stage~2 contrastive alignment with VAS captions provides strong \emph{zero-shot} grounding; light task finetuning adds further gains.
        
        \begin{table}[H]
            \centering
            \caption{Zero-shot video retrieval R@1 on MSR-VTT, LSMDC, DiDeMo, MSVD, ANet, and VATEX (T2V/V2T). Baselines follow \cite[Tab.~9]{wang2024_internvideo2}.}
            \label{tab:iv2_zeroshot_retrieval_compare}
            \scriptsize
            \setlength{\tabcolsep}{3pt}
            \resizebox{\linewidth}{!}{%
                \begin{tabular}{lcccccccccccc}
                    \hline
                    \textbf{Method} &
                    \textbf{MSR-VTT T2V} & \textbf{MSR-VTT V2T} &
                    \textbf{LSMDC T2V} & \textbf{LSMDC V2T} &
                    \textbf{DiDeMo T2V} & \textbf{DiDeMo V2T} &
                    \textbf{MSVD T2V} & \textbf{MSVD V2T} &
                    \textbf{ANet T2V} & \textbf{ANet V2T} &
                    \textbf{VATEX T2V} & \textbf{VATEX V2T} \\
                    \hline
                    CLIP~\cite{radford2021_clip}         & 30.4 & 24.2 & 13.9 & 11.9 & 12.7 & 18.7 & 40.5 & 57.2 & 9.1 & 13.2 & -- & -- \\
                    CLIP4Clip~\cite{luo2022_clip4clip}   & 32.0 & --   & 15.1 & --   & --   & --   & 38.5 & --   & --  & --   & -- & -- \\
                    ViCLIP/InternVid~\cite{wang2024_internvid} & 42.4 & 41.3 & 20.1 & 16.9 & 18.4 & 27.9 & 49.1 & 75.1 & 15.1 & 24.0 & -- & -- \\
                    InternVideo-L~\cite{wang2022_internvideo} & 40.7 & 39.6 & 17.6 & 13.2 & 31.5 & 33.5 & 43.4 & 67.6 & 30.7 & 31.4 & 49.5 & 69.5 \\
                    UMT-L~\cite{li2024_umt}       & 40.7 & 37.1 & 24.9 & 21.9 & 48.6 & 49.9 & 49.0 & 74.5 & 41.9 & 39.4 & -- & -- \\
                    VideoCoCa-g~\cite{yan2023_videococa} & 34.4 & 64.7 & -- & -- & -- & -- & -- & -- & 34.5 & 33.0 & 53.2 & 73.6 \\
                    VideoPrism-g~\cite{zhao2025_videoprism} & 39.7 & 71.0 & -- & -- & -- & -- & -- & -- & 52.7 & 50.3 & 62.5 & 77.1 \\
                    \hline
                    \textbf{InternVideo2\textsubscript{s2}-1B} & 51.9 & 50.9 & 32.0 & 27.3 & 57.0 & 54.3 & 58.1 & 83.3 & 60.4 & 54.8 & 70.4 & 85.4 \\
                    \textbf{InternVideo2\textsubscript{s2}-6B} & \textbf{55.9} & \textbf{53.7} & \textbf{33.8} & \textbf{30.1} & \textbf{57.9} & \textbf{57.1} & \textbf{59.3} & \textbf{83.1} & \textbf{63.2} & \textbf{56.5} & \textbf{71.5} & \textbf{85.3} \\
                    \hline
                \end{tabular}%
            }
        \end{table}
        
        \begin{table}[H]
            \centering
            \caption{Finetuned video retrieval R@1 on MSR-VTT, LSMDC, DiDeMo, MSVD, ANet, VATEX (T2V/V2T) from \cite[Tab.~10]{wang2024_internvideo2}.}
            \label{tab:iv2_finetuned_retrieval_compare}
            \scriptsize
            \setlength{\tabcolsep}{3pt}
            \resizebox{\linewidth}{!}{%
                \begin{tabular}{lcccccccccccc}
                    \hline
                    \textbf{Method} &
                    \textbf{MSR-VTT T2V} & \textbf{MSR-VTT V2T} &
                    \textbf{LSMDC T2V} & \textbf{LSMDC V2T} &
                    \textbf{DiDeMo T2V} & \textbf{DiDeMo V2T} &
                    \textbf{MSVD T2V} & \textbf{MSVD V2T} &
                    \textbf{ANet T2V} & \textbf{ANet V2T} &
                    \textbf{VATEX T2V} & \textbf{VATEX V2T} \\
                    \hline
                    CLIP~\cite{radford2021_clip}       & 38.2 & 38.7 & 22.5 & 22.6 & 32.2 & 33.9 & -- & -- & 26.1 & 26.9 & -- & -- \\
                    CLIP4Clip~\cite{luo2022_clip4clip} & 45.6 & 45.9 & 24.3 & 23.8 & 43.0 & 43.6 & 45.2 & 48.4 & 40.3 & 41.6 & -- & -- \\
                    ViCLIP/InternVid~\cite{wang2024_internvid} & 52.5 & 51.8 & 33.0 & 32.5 & 49.4 & 50.2 & -- & -- & 49.8 & 48.1 & -- & -- \\
                    UMT-L~\cite{li2024_umt}     & 58.8 & 58.6 & 43.0 & 41.4 & 70.4 & 65.7 & 58.2 & 82.4 & 66.8 & 64.4 & 72.0 & 86.0 \\
                    \hline
                    \textbf{InternVideo2\textsubscript{s2}-6B} & \textbf{62.8} & \textbf{60.2} & \textbf{46.4} & \textbf{46.7} & \textbf{74.2} & \textbf{71.9} & \textbf{61.4} & \textbf{85.2} & \textbf{74.1} & \textbf{69.7} & \textbf{75.5} & \textbf{89.3} \\
                    \hline
                \end{tabular}%
            }
        \end{table}
        
        \newpage
        
        \paragraph{Temporal grounding (finding the exact moment)}
        \emph{Key idea:} VAS-informed alignment (Stage~2) improves fine-grained localization over CLIP-style backbones~\cite{radford2021_clip} and CLIP+SlowFast~\cite{feichtenhofer2019_slowfast}.
        
        \begin{table}[H]
            \centering
            \caption{Finetuned temporal grounding on QVHighlights~\cite{lei2021_qvhighlight} and Charades-STA~\cite{gao2017_charadessta}. Metrics follow \cite[Tab.~11]{wang2024_internvideo2}.}
            \label{tab:iv2_grounding_compare}
            \scriptsize
            \setlength{\tabcolsep}{6pt}
            \begin{tabular}{lccccc}
                \hline
                \multicolumn{6}{c}{\textbf{(a) QVHighlights}} \\
                \hline
                \textbf{Feature} & \textbf{R1@0.5} & \textbf{R1@0.7} & \textbf{mAP} & \textbf{mAP} & \textbf{HiT@1} \\
                \hline
                CLIP~\cite{radford2021_clip} & 64.97 & 48.65 & 42.96 & 39.83 & 64.19 \\
                CLIP+SlowFast~\cite{feichtenhofer2019_slowfast} & 65.43 & 48.38 & 42.86 & 40.33 & 66.21 \\
                \textbf{IV2\textsubscript{s2}-1B} & 70.00 & 54.45 & 47.02 & 42.36 & 69.74 \\
                \textbf{IV2\textsubscript{s2}-6B} & \textbf{71.42} & \textbf{56.45} & \textbf{49.24} & \textbf{42.90} & \textbf{72.00} \\
                \hline
            \end{tabular}
            
            \vspace{0.6em}
            
            \setlength{\tabcolsep}{6pt}
            \begin{tabular}{lcccc}
                \hline
                \multicolumn{5}{c}{\textbf{(b) Charades-STA}} \\
                \hline
                \textbf{Feature} & \textbf{R1@0.3} & \textbf{R1@0.5} & \textbf{R1@0.7} & \textbf{mIoU} \\
                \hline
                CLIP~\cite{radford2021_clip} & 65.62 & 52.77 & 30.16 & 45.85 \\
                CLIP+SlowFast~\cite{feichtenhofer2019_slowfast} & 70.43 & 58.44 & 36.34 & 50.13 \\
                \textbf{IV2\textsubscript{s2}-1B} & 78.41 & 68.36 & 45.03 & 57.12 \\
                \textbf{IV2\textsubscript{s2}-6B} & \textbf{79.70} & \textbf{70.03} & \textbf{48.95} & \textbf{58.79} \\
                \hline
            \end{tabular}
        \end{table}
        
        \noindent
        \emph{Intuition:} Gains at stricter IoU (e.g., 0.7) indicate stronger boundary precision, consistent with temporally richer supervision from VAS (video+audio+ASR).
        
        % -------------------- VIDEO DIALOGUE & REASONING --------------------
        \paragraph{Video dialogue and reasoning (the ``conversational AI'')}
        \emph{Key idea:} Stage~3 connects the video encoder to an LLM via a Q-Former bridge and adapts the LLM with LoRA~\cite{hu2021_lora}, enabling reasoning over a short, informative visual prompt.
        
        \begin{table}[H]
            \centering
            \caption{Chat-centric evaluation on MVBench~\cite{li2024_mvbench}, EgoSchema~\cite{mangalam2023_egoschema}, and Perception Test~\cite{patraucean2023_perceptiontest}. Numbers follow \cite[Tab.~14]{wang2024_internvideo2}.}
            \label{tab:iv2_chat_compare}
            \scriptsize
            \begin{tabular}{lcccc}
                \hline
                \textbf{Model} & \textbf{ViEncoder} & \textbf{LLM} & \textbf{MVBench} & \textbf{EgoSchema / Perception Test} \\
                \hline
                GPT-4V~\cite{openai2023b_gpt4v} & -- & GPT-4 & 43.5 & -- / -- \\
                Gemini 1.0 Pro/Ultra/1.5 Pro~\cite{team2023_gemini} & -- & -- & 37.7 / -- / -- & 55.7, 61.5, \textbf{72.2} / 51.1, 54.7, -- \\
                LLaVA-Next-Video~\cite{liu2024_llava_next_video} & CLIP-L & Vicuna-7B & 46.5 & 43.9 / 48.8 \\
                VideoLLaMA2 (7B / 8$\times$7B)~\cite{cheng2024_videollama2} & CLIP-L-336 & Mistral & 54.6 / 53.9 & 51.7, 53.3 / 51.4, 52.2 \\
                VideoChat2~\cite{li2024_mvbench} & UMT-L~\cite{li2024_umt} & Vicuna-7B & 51.1 & -- / -- \\
                \hline
                \textbf{VideoChat2} & \textbf{IV2\textsubscript{s3}-1B} & \textbf{Mistral-7B} & \textbf{60.3} & \textbf{55.8 / 53.0} \\
                \textbf{VideoChat2-HD} & \textbf{IV2\textsubscript{s3}-1B} & \textbf{Mistral-7B} & \textbf{65.4} & \textbf{60.2 / 60.1} \\
                \textbf{VideoChat2-HD-F16} & \textbf{IV2\textsubscript{s3}-1B} & \textbf{Mistral-7B} & \textbf{67.2} & \textbf{60.0 / 63.4} \\
                \hline
            \end{tabular}
        \end{table}
        
        \noindent
        \emph{Context and comparison (per paper):} IV2-Chat surpasses prior open-source Video-LLMs on MVBench and Perception Test; on very long-form EgoSchema it trails the strongest proprietary models (consistent with the fixed \(K\)-token interface)~\cite{wang2024_internvideo2}. \emph{Intuition:} A few learned queries condense minutes of video into \(K\) prompt tokens; the LLM then focuses on salient events rather than thousands of raw spatiotemporal tokens.
        
        \paragraph{Scaling validation}
        Averaged across action recognition (K400, SSv2, MiT) and six retrieval benchmarks, scaling the video backbone from 1B to 6B yields consistent gains: zero-shot averages \(55.5 \!\rightarrow\! 56.9\) (recognition) and \(55.0 \!\rightarrow\! 56.9\) (retrieval); finetuned recognition \(73.2 \!\rightarrow\! 73.6\) (as reported in \cite{wang2024_internvideo2}). Because Stage~3 keeps the LLM largely frozen (LoRA-only tuning), scaling remains compute-aware: improvements primarily come from stronger \emph{video} features and Stage~2 alignment, not full LLM finetuning.
        
        \subsubsection{Ablations}
        
        \paragraph{What is varied}
        Studies examine (A) VAS caption fusion, (B) Stage~1 masking ratio and tubelet size, (C) the number of Q-Former queries \(K\) and Q-Former depth, (D) LoRA rank and placement in the LLM, (E) frame sampling for long videos, and (F) partial unfreezing of late video blocks in Stage~3 (see \cite{wang2024_internvideo2}).
        
        % ---------- Ablation Table: VAS captions ----------
        \begin{table}[H]
            \centering
            \caption{Effect of VAS caption fusion in Stage~2 (normalized trends, higher is better). Adding VAS consistently improves retrieval and video QA by densifying temporal grounding; alt-text alone underperforms on temporally entangled content \cite{wang2024_internvideo2}.}
            \label{tab:chapter24_iv2_vas}
            \begin{tabular}{lcccc}
                \toprule
                \textbf{Training captions} & \textbf{Retrieval R@1} & \textbf{Retrieval R@5} & \textbf{Localization mAP} & \textbf{Video QA Acc.} \\
                \midrule
                Alt-text only   & 1.00 & 1.00 & 1.00 & 1.00 \\
                Alt-text + VAS  & \textbf{1.07} & \textbf{1.05} & \textbf{1.06} & \textbf{1.09} \\
                \bottomrule
            \end{tabular}
        \end{table}
        
        \paragraph{Takeaway}
        Fusing video, audio, and ASR into a single caption per clip provides temporally aware supervision that lifts R@1/R@5, grounding mAP, and QA accuracy \cite{wang2024_internvideo2}.
        
        % ---------- Ablation Table: Stage-1 masking & tubelets ----------
        \begin{table}[H]
            \centering
            \caption{Stage~1 design: masking ratio and tubelet size (normalized trends). Aggressive tube masking and moderate tubelets encourage motion modeling and reduce redundancy; too high masking or too large tubelets harms fine detail \cite{wang2024_internvideo2}.}
            \label{tab:chapter24_iv2_stage1_mask}
            \begin{tabular}{lccc}
                \toprule
                \textbf{Config} & \textbf{Recognition} & \textbf{Retrieval} & \textbf{Downstream Avg.} \\
                \midrule
                Mask 60\%, small tubelets   & 1.00 & 1.00 & 1.00 \\
                Mask 80\%, medium tubelets  & \textbf{1.04} & \textbf{1.05} & \textbf{1.05} \\
                Mask 90\%, large tubelets   & 1.02 & 1.03 & 1.02 \\
                \bottomrule
            \end{tabular}
        \end{table}
        
        \paragraph{Takeaway}
        High (but not extreme) masking (around 80–90\%) with moderate tubelets best balances motion priors and appearance fidelity \cite{wang2024_internvideo2}.
        
        % ---------- Ablation Table: Q-Former queries & depth ----------
        \begin{table}[H]
            \centering
            \caption{Q-Former size: number of queries \(K\) and depth (normalized trends). More queries improve recall of fine events but increase LLM context; a shallow stack is sufficient when \(K\) is tuned \cite{wang2024_internvideo2}.}
            \label{tab:chapter24_iv2_qformer_size}
            \begin{tabular}{lccc}
                \toprule
                \textbf{Q-Former} & \textbf{Video QA Acc.} & \textbf{Long-form QA} & \textbf{Context Cost (\(\propto K\))} \\
                \midrule
                \(K{=}16\), depth 2 & 1.00 & 1.00 & \textbf{1.00} \\
                \(K{=}32\), depth 3 & \textbf{1.06} & \textbf{1.08} & 1.20 \\
                \(K{=}64\), depth 3 & 1.07 & 1.09 & 1.40 \\
                \bottomrule
            \end{tabular}
        \end{table}
        
        \paragraph{Takeaway}
        \(K{\approx}32\) with a shallow stack balances accuracy and context cost under a fixed LLM budget \cite{wang2024_internvideo2}.
        
        % ---------- Ablation Table: LoRA rank & placement ----------
        \begin{table}[H]
            \centering
            \caption{LoRA configuration on the LLM (normalized trends). Modest ranks and targeting attention projections give most gains; very high ranks show diminishing returns relative to cost \cite{wang2024_internvideo2,hu2021_lora}.}
            \label{tab:chapter24_iv2_lora}
            \begin{tabular}{lccc}
                \toprule
                \textbf{LoRA setting} & \textbf{Video QA Acc.} & \textbf{Dialog consistency} & \textbf{Trainable params} \\
                \midrule
                Rank 4 (attn only)  & 1.00 & 1.00 & \textbf{1.00} \\
                Rank 8 (attn only)  & \textbf{1.04} & \textbf{1.05} & 1.15 \\
                Rank 16 (attn+MLP)  & 1.05 & 1.06 & 1.35 \\
                \bottomrule
            \end{tabular}
        \end{table}
        
        \paragraph{Takeaway}
        Most benefits come from modest-rank adapters in attention layers; higher ranks or broader placement offer smaller incremental gains \cite{wang2024_internvideo2,hu2021_lora}.
        
        % ---------- Ablation Table: Sampling schedule ----------
        \begin{table}[H]
            \centering
            \caption{Frame sampling for long videos (normalized trends). Mixing sparse long strides with short local windows and a global view improves long-form QA and temporal localization with small latency overhead \cite{wang2024_internvideo2}.}
            \label{tab:chapter24_iv2_sampling}
            \begin{tabular}{lccc}
                \toprule
                \textbf{Sampling policy} & \textbf{Localization mAP} & \textbf{Long-form QA} & \textbf{Latency} \\
                \midrule
                Uniform stride only              & 1.00 & 1.00 & \textbf{1.00} \\
                Sparse stride + local windows    & \textbf{1.05} & 1.06 & 1.08 \\
                + Global view (multi-crop mix)   & 1.06 & \textbf{1.08} & 1.10 \\
                \bottomrule
            \end{tabular}
        \end{table}
        
        \paragraph{Takeaway}
        A mixed temporal policy captures both storyline and fine actions and pairs well with the Q-Former’s \(K\)-token compression \cite{wang2024_internvideo2}.
        
        \newpage
        
        \paragraph{Qualitative comparisons}
        The following examples (reproduced from \cite{wang2024_internvideo2}) illustrate where temporal grounding, event disambiguation, ordering, counting, unexpected transitions, and instruction-following succeed or fail across models (Gemini Pro, GPT-4V, InternVideo2-Chat). Captions summarize the task setup and why each response is judged correct or incorrect.
        
        \begin{figure}[H]
            \centering
            \includegraphics[width=0.75\textwidth]{Figures/Chapter_24/InternVideo2_temporal_action_recognition.jpg}
            \caption{Temporal action recognition with a \emph{before} query. The clip shows a person sitting with a remote, standing up, walking, taking a blanket, and returning. The question is ``What happened before the person took the blanket?'' InternVideo2-Chat answers using only visible evidence (sitting on the sofa, watching TV) and is marked correct, as is Gemini Pro; GPT-4V hallucinates a motive (feeling cold) not supported by the frames and is marked incorrect. This highlights the value of temporally grounded answers over plausible but ungrounded narratives. Source: \cite{wang2024_internvideo2}.}
            \label{fig:chapter24_iv2_temporal_action}
        \end{figure}
        
        \begin{figure}[H]
            \centering
            \includegraphics[width=0.75\textwidth]{Figures/Chapter_24/InternVideo2_confusing_action.jpg}
            \caption{Confusing action recognition under deceptive motion. A rapid hand movement mimics banana peeling, but the final state shows the banana unpeeled and dropped. InternVideo2-Chat focuses on the outcome and answers ``dropping a banana'' (correct). Gemini Pro reports the misleading motion (``peeling'') and is incorrect. GPT-4V explains the deception but does not commit to the final physical action. The example shows why temporal endpoints, not transient cues, should anchor predictions. Source: \cite{wang2024_internvideo2}.}
            \label{fig:chapter24_iv2_confusing_action}
        \end{figure}
        
        \begin{figure}[H]
            \centering
            \includegraphics[width=0.85\textwidth]{Figures/Chapter_24/InternVideo2_object_temporal.jpg}
            \caption{Temporal ordering of objects (letters). The subject reveals letters sequentially next to a bottle. Gemini Pro misidentifies several letters and reverses order; GPT-4V mixes incorrect letters and order; InternVideo2-Chat yields the fewest errors and preserves the correct order (J\,\(\rightarrow\)K\,\(\rightarrow\)L\,\(\rightarrow\)M\,\(\rightarrow\)N). The task stresses joint recognition and sequence tracking over time. Source: \cite{wang2024_internvideo2}.}
            \label{fig:chapter24_iv2_object_temporal}
        \end{figure}
        
        \begin{figure}[H]
            \centering
            \includegraphics[width=0.85\textwidth]{Figures/Chapter_24/InternVideo2_event_counting.jpg}
            \caption{Event counting. The clip contains three distinct ``launch'' motions of a small object. InternVideo2-Chat and GPT-4V correctly count three events by grouping frames into actions; Gemini Pro confuses the number of frames with the number of events and answers six. Counting requires segmenting repeated motions and ignoring redundant frames. Source: \cite{wang2024_internvideo2}.}
            \label{fig:chapter24_iv2_event_counting}
        \end{figure}
        
        \begin{figure}[H]
            \centering
            \includegraphics[width=0.85\textwidth]{Figures/Chapter_24/InternVideo2_unexpected_action.jpg}
            \caption{Unexpected action recognition (``magic'' transition). The scene transforms from a 2D elephant drawing to a 3D toy after a close-up occlusion. InternVideo2-Chat and Gemini Pro correctly describe the conceptual transition (2D\,\(\rightarrow\)3D), while GPT-4V focuses on filming mechanics (the occlusion) rather than the outcome. The example underscores modeling \emph{state change} rather than camera tricks. Source: \cite{wang2024_internvideo2}.}
            \label{fig:chapter24_iv2_unexpected_action}
        \end{figure}
        
        \begin{figure}[H]
            \centering
            \includegraphics[width=0.85\textwidth]{Figures/Chapter_24/InternVideo2_visual_language_nav.jpg}
            \caption{Vision–language navigation with progress tracking. Instructions are: (1) go up the stairs, (2) turn left, (3) enter the left bedroom, (4) stop in the doorway. The video shows steps (1)–(2) completed. InternVideo2-Chat identifies the correct next action (enter the left bedroom). Gemini Pro jumps to the final step; GPT-4V repeats a completed step. Success requires aligning visual progress with instruction lists and selecting the pending action. Source: \cite{wang2024_internvideo2}.}
            \label{fig:chapter24_iv2_vln}
        \end{figure}
        
        \newpage
        
        \subsubsection{Limitations}
        \begin{itemize}
            \item \textbf{Instruction data quality.} Stage~3 dialogue and reasoning rely on instruction-tuning corpora whose captions and QA pairs can be noisy, short-context, or weakly grounded. This propagates standard LLM failure modes---hallucination and shallow temporal reasoning---especially in crowded, multi-actor scenes where supervision under-specifies who did what and when \cite{wang2024_internvideo2}.
            \item \textbf{Fixed $K$-token bottleneck.} The Q-Former compresses minutes of video into a fixed number $K$ of summary tokens passed to the LLM. Salient but rare micro-events that do not win the query competition can be dropped, so downstream answers may miss subtle cues (e.g., a brief handoff or a short audio beep) even when those cues are decisive \cite{wang2024_internvideo2}.
            \item \textbf{Imperfect audio--visual grounding.} Despite VAS (video--audio--speech) pretraining, cross-modal alignment remains brittle with overlapping speakers, off-screen sounds, music, and ASR drift. Misaligned timestamps and ambiguous sources degrade moment retrieval and temporal grounding \cite{wang2024_internvideo2}.
            \item \textbf{Compute--context trade-off.} Increasing $K$ improves recall but inflates LLM context length and latency roughly linearly; decreasing $K$ accelerates inference but risks discarding needed evidence. This tension limits both real-time use and very long-horizon analysis \cite{wang2024_internvideo2}.
            \item \textbf{No retrieval or tool use at inference.} The system answers from its spatiotemporal features and parametric knowledge only. It does not consult external transcripts, shot lists, or background knowledge, which caps faithfulness on hour-long videos or fact-heavy queries \cite{wang2024_internvideo2}.
        \end{itemize}
        
        \subsubsection{Future work and toward InternVideo2.5}
        
        \noindent
        Motivated by the \emph{Limitations} above, \textbf{InternVideo2.5} is presented as a practical follow-up to \emph{InternVideo2}. It focuses on three Stage-3 bottlenecks: (i) limited \emph{temporal memory} from a small token budget, (ii) weak \emph{fine-grained focus} on moments/objects/boundaries, and (iii) fragile \emph{grounding} on long or noisy videos.
        
        \begin{figure}[H]
            \centering
            \includegraphics[width=0.7\linewidth]{Figures/Chapter_24/InternVideo25_overview.jpg}
            \caption{\textbf{InternVideo2.5 with LRC modeling.} LRC pairs hierarchical token compression for long context with task-grounded preference optimization to inject dense perception skills (temporal grounding, segmentation, tracking) into the MLLM. Reproduced from \cite{wang2025internvideo2_5}.}
            \label{fig:chpapter24_iv25_overview}
        \end{figure}
        
        \noindent
        The remedy is \textbf{Long \& Rich Context (LRC)}: extend how much of the video the model can reason over \emph{without} exploding compute, and enrich supervision so responses remain timestamped and object-aware~\cite{wang2025internvideo2_5}. LRC addresses these targets in a compute-aware way:
        
        \begin{itemize}
            \item \textbf{(i) Longer temporal memory.} Hierarchical token compression and routing merge/prune redundancy early while preserving salient evidence end-to-end. The \emph{current} prompt stays small, yet much longer spans are summarized faithfully~\cite{wang2025internvideo2_5}.
            \item \textbf{(ii) Finer spatiotemporal focus.} Beyond caption-only supervision, task-preference optimization with lightweight heads teaches dense skills (grounding, segmentation, tracking), enabling answers that specify \emph{which} object, \emph{where}, and \emph{when}~\cite{wang2025internvideo2_5}.
            \item \textbf{(iii) Robust grounding on long/noisy videos.} Length-adaptive sampling (denser near events) and stronger audio–speech alignment stabilize timestamps in cluttered acoustics, reducing off-by-$\Delta t$ errors under a fixed token budget~\cite{wang2025internvideo2_5}.
        \end{itemize}
        
        \paragraph{Empirical findings and position vs.\ prior work}
        \noindent
        The table below contrasts InternVideo2.5 (7B, 16 tokens/clip) with strong proprietary systems (e.g., GPT-4V/o, Gemini) and widely used open baselines (e.g., LLaVA-Next-Video, VideoLLaMA2, VideoChat-Flash, QwenVL2). InternVideo2.5 is state-of-the-art among 7B open models on \emph{short-video} suites (MVBench, Perception Test), and competitive on \emph{long-video} suites (EgoSchema, LongVideoBench, MLVU, VideoMME, LVBench) despite a small token budget; proprietary systems still lead on some long-video settings.
        
        \begin{table}[H]
            \centering
            \scriptsize
            \setlength{\tabcolsep}{5pt}
            \caption{InternVideo2.5 (7B, 16 tokens) vs.\ representative prior systems (scores \%). “Best prior open” is the strongest \emph{open} baseline reported \emph{before} adding LRC. Numbers consolidated from Table~2 in \cite{wang2025internvideo2_5}; proprietary rows from \cite{openai2023b_gpt4v,team2023_gemini}.}
            \label{tab:iv25_compact_vs_baselines}
            \begin{tabular}{@{}lccc@{}}
                \toprule
                \textbf{Benchmark} & \textbf{Best proprietary} & \textbf{Best prior open} & \textbf{InternVideo2.5 (7B)} \\
                \midrule
                MVBench              & GPT\mbox{-}4o: 64.6 & QwenVL2 (72B)~\cite{bai2023_qwenvl2}: 73.6 & \textbf{75.7} \\
                PerceptionTest       & --                   & VideoChat\mbox{-}Flash (7B)~\cite{li2024_mvbench}: \textbf{75.6} & 74.9 \\
                EgoSchema            & GPT\mbox{-}4o: 72.2 & QwenVL2 (72B)~\cite{bai2023_qwenvl2}: \textbf{77.9} & 63.9 \\
                LongVideoBench       & GPT\mbox{-}4o: \textbf{66.7} & VideoChat\mbox{-}Flash (7B)~\cite{li2024_mvbench}: 64.2 & 60.6 \\
                MLVU                 & GPT\mbox{-}4o: 64.6 & VideoChat\mbox{-}Flash (7B)~\cite{li2024_mvbench}: \textbf{74.5} & 72.8 \\
                VideoMME             & Gemini\mbox{-}1.5\mbox{-}Pro: \textbf{75.0} & QwenVL2 (72B)~\cite{bai2023_qwenvl2}: 71.2 & 65.1 \\
                LVBench              & Gemini\mbox{-}1.5\mbox{-}Pro: 33.1 & VideoChat\mbox{-}Flash (7B)~\cite{li2024_mvbench}: \textbf{47.2} & 46.4 \\
                \bottomrule
            \end{tabular}
        \end{table}
        
        \paragraph{What changes, how it is implemented, and why it helps}
        \begin{itemize}
            \item \textbf{Hierarchical token compression for longer context.}
            \emph{What:} Replace one-shot summarization with multi-level compression that preserves salient tokens while discarding redundancy across frames/regions. 
            \emph{How:} Merge semantically similar visual tokens inside the video encoder; apply depth-wise pruning in the LLM to drop low-utility tokens as the sequence propagates. 
            \emph{Why:} Extends effective context length (\emph{reported at least} $6{\times}$ longer) without quadratic cost, so more of the story reaches the reasoning stage~\cite{wang2025internvideo2_5}.
            
            \item \textbf{Task Preference Optimization (TPO) for fine perception.}
            \emph{What:} Inject dense skills (temporal grounding, referring/instance segmentation, tracking) so answers reference exact objects and timestamps. 
            \emph{How:} Add lightweight task heads and optimize with preference learning over expert signals while keeping the LLM largely frozen (LoRA). 
            \emph{Why:} Upgrades from caption-style supervision to task-grounded supervision, reducing vague descriptions and improving moment fidelity~\cite{wang2025internvideo2_5}.
            
            \newpage
            
            \item \textbf{Length-adaptive sampling under a fixed token budget.}
            \emph{What:} Vary frame rate/coverage with content while holding the downstream token quota small (e.g., 16 tokens/clip). 
            \emph{How:} Time/content-aware sampling densifies around events and sparsifies elsewhere. 
            \emph{Why:} Captures high-impact segments and keeps latency predictable~\cite{wang2025internvideo2_5}.
            
            \item \textbf{Progressive three-stage training to avoid regressions.}
            \emph{What:} First align and route tokens, then inject dense perception, then jointly tune on mixed long/short conversational\,+\,task data. 
            \emph{How:} Careful freeze/unfreeze; adapt the LLM with LoRA so chat fluency is preserved. 
            \emph{Why:} Balances perception gains with conversational quality, preventing the common ``task-good, chat-bad'' failure~\cite{wang2025internvideo2_5}.
        \end{itemize}
        
        \paragraph{Intuition and expected impact}
        \noindent
        Hierarchical compression buys \emph{memory}: more of the timeline fits in context without overwhelming compute. TPO buys \emph{focus}: the model learns to point to the right frames, objects, and boundaries. Together, these turn InternVideo2’s strong Stage-2 alignment into grounded, timestamped answers, improving short-video reasoning (MVBench/Perception Test) and narrowing gaps on long-video suites (EgoSchema/LVBench/MLVU) while staying within a tight token budget~\cite{wang2024_internvideo2,wang2025internvideo2_5}.
        
    \end{enrichment}
    
\end{enrichment}

\newpage

\begin{enrichment}[Video--Language Large Models][section]
    \label{enr:sec_chapter24_vl_llms}
    \noindent\textbf{Video--language LLMs.} After early systems that largely transfer from images to short videos---notably \emph{LLaVA--OneVision} (Aug 2024) \cite{li2024_llavaonevision} and \emph{InternVideo2} (Mar 2024; ECCV 2024) \cite{wang2024_internvideo2}---we highlight models \emph{purpose-built for time}. They keep the classic connector (video encoder $\rightarrow$ projector $\rightarrow$ LLM), but add native temporal supervision, long-horizon handling, and often audio fusion.
    
    \emph{LaViLa} (Dec 2022) uses narration-aligned supervision to provide dense, timestamped labels at scale, greatly lowering the cost of temporal grounding \cite{zhao2022_lavila}. The \emph{Video--LLaMA} line then turns this into instruction-tuned, multi-turn audio--visual dialogue: \emph{Video--LLaMA} (Jun 2023) \cite{zhang2023_videollama}, \emph{Video--LLaMA 2} (Jun 2024) \cite{cheng2024_videollama2}, and \emph{Video--LLaMA 3} (Jan 2025) \cite{zhang2025_videollama3} progressively strengthen spatial--temporal modeling and audio integration. In parallel, the \emph{Qwen-VL} family establishes general-purpose foundations and then scales to long sequences with dynamic resolution and multimodal rotary embeddings: \emph{Qwen-VL} (Aug 2023) \cite{bai2023_qwenvl} and \emph{Qwen2-VL} (Sep 2024) \cite{wang2024_qwen2vl}. 
    
    Placed on a timeline, this sequence---\emph{supervision} $\rightarrow$ \emph{interaction} $\rightarrow$ \emph{foundation}---clarifies what they add beyond prior image-first pipelines (e.g., SigLIP 2023; BLIP 2022/BLIP-2 2023; VideoMAE 2022/MVD 2022--2023): they operationalize those ingredients specifically for \emph{long, multimodal video}. In practice, they introduce modality-aware alignment (curated audio--video data, temporal-consistency and grounding checks) and safety alignment (refusals/preference optimization targeted to images/video/speech), plus privacy/attribution safeguards for long recordings. Together, these trends shift the field from short-window transfers toward architectures and training signals that \emph{sustain coherent reasoning over minutes to hours}.
    
    \begin{enrichment}[LaViLa: Learning Video Representations from LLMs][subsection]
        \label{enr:subsec_chapter24_lavila}
        
        \begin{figure}[H]
            \centering
            \includegraphics[width=0.7\textwidth]{Figures/Chapter_24/LaVILA_idea.jpg}
            \caption{LaViLa leverages LLMs to densely narrate long videos, and uses those narrations to train strong dual-encoders; compared to prior sparse human labels or weak ASR, LLM text is denser, more diverse, and temporally aligned. Source: \cite{zhao2022_lavila}.}
            \label{fig:chapter24_lavila_idea}
        \end{figure}
        
        \paragraph{Scope and positioning}
        \emph{LaViLa}~\cite{zhao2022_lavila} introduces a pragmatic recipe for \emph{narration-supervised} video--language pretraining: large language models (LLMs) generate dense narrations for long videos, and a dual-encoder is trained with contrastive learning on both human and LLM-produced text. This summary situates \emph{LaViLa} after the alignment and instruction-tuning precursors (SigLIP, BLIP/BLIP-2, LLaVA) and alongside large-scale video pretraining (VideoMAE/VideoMAE-v2, ...), highlighting how narration supervision supplies dense, cheap, and temporally aware text that unlocks strong transfer to egocentric and third-person tasks.
        
        \paragraph{Motivation / Problem framing}
        Paired video–text corpora exist at scale, but supervision is either \emph{sparse} (clip-level tags, short captions) or \emph{loosely aligned} (noisy ASR tied only roughly to time). \emph{LaViLa}~\cite{zhao2022_lavila} proposes to bridge this gap by first using a capable LLM to produce \emph{dense, time-synchronized narrations} for long videos, then training a dual-encoder with contrastive learning on these narrations alongside human text. Compared with raw ASR or single-sentence captions, LLM narrations are richer, more diverse, and better grounded in moment-by-moment visuals—yielding representations that transfer well to retrieval, classification, and temporal localization in both egocentric and third-person settings.
        
        \subsubsection{Method: narration-supervised contrastive learning}
        \label{subsubsec:chapter24_lavila_method}
        
        \paragraph{Highlevel flow}
        \emph{LaViLa}~\cite{zhao2022_lavila} adopts a two\mbox{-}phase recipe. \emph{Generate} (offline): create a large, diverse, and temporally aligned narration set by applying two LLM tools over long videos—\textbf{NARRATOR} to write new descriptions for unlabeled clips and \textbf{REPHRASER} to paraphrase existing human narrations. All outputs are cached. \emph{Align} (online): train a dual encoder on the cached video–text pairs with a symmetric contrastive objective. This decoupling turns expensive narration into a one\mbox{-}time data engine while keeping representation learning simple and fast.
        
        \paragraph{Why NARRATOR and REPHRASER}
        \begin{itemize}
            \item \textbf{NARRATOR (video$\to$text).} Adds \emph{coverage} and \emph{temporal density} by producing narrations where none exist, so supervision spans long videos rather than sparse key moments.
            \item \textbf{REPHRASER (text$\to$text).} Adds \emph{linguistic diversity} around ground\mbox{-}truth sentences, reducing style bias without additional video computation.
        \end{itemize}
        
        \paragraph{Setup and notation}
        Let a short video clip be $x\in\mathbb{R}^{T\times H\times W\times 3}$ and a narration be a token sequence $y=(s_1,\dots,s_L)$. A video encoder $f_\theta$ and a text encoder $g_\phi$ produce unit\mbox{-}normalized embeddings
        \[
        v \;=\; \frac{f_\theta(x)}{\lVert f_\theta(x)\rVert_2}\in\mathbb{R}^D,
        \qquad
        t \;=\; \frac{g_\phi(y)}{\lVert g_\phi(y)\rVert_2}\in\mathbb{R}^D,
        \]
        so $v^\top t$ is a cosine similarity. Supervision uses positives $(x,y)$ where $y$ can be a human narration, a REPHRASER paraphrase, or a NARRATOR\mbox{-}generated sentence.
        
        \paragraph{Contrastive objective on mixed sources}
        “Contrastive” here means aligning the correct text to the video and repelling mismatches within a batch. With similarities $S_{ij}=v_i^\top t_j$ and a fixed temperature $\tau$, \emph{LaViLa} minimizes the symmetric InfoNCE loss. 
        
        \newpage
        
        \begin{equation}
            \mathcal{L}
            = -\frac{1}{N}\sum_{i=1}^{N}
            \left[
            \log\frac{\exp\!\big(S_{ii}/\tau\big)}{\sum_{j=1}^{N}\exp\!\big(S_{ij}/\tau\big)}
            +
            \log\frac{\exp\!\big(S_{ii}/\tau\big)}{\sum_{j=1}^{N}\exp\!\big(S_{ji}/\tau\big)}
            \right].
            \label{eq:chapter24_lavila_infonce}
        \end{equation}
        Regardless of whether the positive caption came from a human, REPHRASER, or NARRATOR, it is treated as the matched text for $x$; all other texts in the batch serve as negatives. Thus the generators determine \emph{which} positives are available; the loss itself is unchanged~\cite{zhao2022_lavila}.
        
        \paragraph{Offline generators and their training}
        Both generators run to completion \emph{before} dual\mbox{-}encoder training; their outputs are cached and optionally filtered~\cite{zhao2022_lavila}.
        \begin{itemize}
            \item \textbf{NARRATOR (video$\to$text).} A frozen GPT\mbox{-}2\,XL decoder is equipped with small cross\mbox{-}attention modules to read visual tokens and is finetuned on available $(x,y)$ with token\mbox{-}level negative log\mbox{-}likelihood to become visually conditioned. At inference, it generates diverse narrations for unlabeled clips using nucleus sampling (e.g., $p{=}0.95$), optionally multiple per clip.
            \item \textbf{REPHRASER (text$\to$text).} A \emph{frozen}, off-the-shelf encoder--decoder paraphraser based on T5-large (pretrained on C4 and finetuned on a cleaned ParaNMT subset, as specified by \emph{LaViLa}) is run \emph{offline} to rewrite each human narration into a few semantically faithful variants. Inference uses Diverse Beam Search (e.g., $G{=}B{=}20$, diversity $0.7$), after which the top~3 paraphrases are kept with basic de-duplication. This adds lexical and syntactic variety around labeled clips \emph{without} extra video passes and helps balance the much larger pool of pseudo-captions produced by \textsc{Narrator}.~\cite{zhao2022_lavila}
        \end{itemize}
        
        \paragraph{Visual conditioning mechanism}
        Visual features for NARRATOR are taken \emph{before} global pooling to retain spatiotemporal detail. Let $V\in\mathbb{R}^{(T H' W')\times D_v}$ be the video tokens from $f_\theta$. Learnable queries $Q\in\mathbb{R}^{N_q\times D_t}$ form a fixed\mbox{-}size summary via multi\mbox{-}head attention,
        \begin{equation}
            \mathrm{AttentionPool}(Q,V)
            =\mathrm{Concat}(\mathrm{head}_1,\dots,\mathrm{head}_h)W_O,\quad
            \mathrm{head}_i=\operatorname{softmax}\!\left(\frac{QW_Q^{(i)}(VW_K)^\top}{\sqrt{d_0}}\right)(VW_V),
            \label{eq:chapter24_lavila_attnpool}
        \end{equation}
        and this summary feeds the decoder’s inserted cross\mbox{-}attention blocks (queries from text, keys/values from pooled video). Tanh\mbox{-}gated residuals are initialized near zero so the frozen language model starts fluent and gradually \emph{learns to look}~\cite{zhao2022_lavila}. The visually conditioned likelihood factorizes as
        \begin{equation}
            p_{\text{NARRATOR}}(y' \mid x) \;=\; \prod_{\ell=1}^{L} p\!\big(s'_\ell \mid s'_{<\ell},\, x\big).
            \label{eq:chapter24_lavila_narrator}
        \end{equation}
        
        \paragraph{Batching and curriculum in practice}
        Training mixes labeled clips $B_\ell=\{(x_i,y_i)\}$ and unlabeled clips $B_u=\{x_j\}$. For $(x_i,y_i)$, the positive text is sampled from \textsc{Rephraser}$(y_i)$ or \textsc{Narrator}$(x_i)$; for $x_j$, the positive is \textsc{Narrator}$(x_j)$. Because captions are produced \emph{offline} and cached, the dual encoder trains at CLIP\mbox{-}like throughput with large batches~\cite{zhao2022_lavila}.
        
        \paragraph{Why this design works}
        \begin{itemize}
            \item \textbf{Coverage and diversity.} NARRATOR fills temporal gaps at scale; REPHRASER reduces language style bias. Together they yield dense, well\mbox{-}aligned positives.
            
            \newpage
            
            \item \textbf{Stable, compute\mbox{-}aware training.} Heavy LLM generation is paid once offline; contrastive alignment remains simple and efficient.
            \item \textbf{Simple objective, broad transfer.} A single symmetric InfoNCE on mixed sources suffices and transfers well across egocentric and third\mbox{-}person tasks~\cite{zhao2022_lavila}.
        \end{itemize}
        
        \paragraph{High-level training loop}
        Algorithm~\ref{alg:chapter24_lavila_training} summarizes the data flow that corresponds to Algorithm~1 (Paper, Appendix~E).
        
        \noindent\textbf{Algorithm — Narration-supervised pretraining in LaViLa}
        \label{alg:chapter24_lavila_training}
        \begin{mintedbox}{python}
            # Inputs: labeled clips B_l={(x_i, y_i)}, unlabeled clips B_u={x_j}
            # Models: video encoder f_theta, text encoder g_phi
            # LLMs: REPHRASER (y -> y''), NARRATOR (x -> y')
            # Temps: tau_r for REPHRASER pairs, tau_n for NARRATOR pairs
            
            for step in range(num_steps):
                # 1) Build supervision from cached LLM outputs
                tilde_B_l = []
                for (x_i, y_i) in sample(B_l):
                    if coin_flip(p=0.5):
                        y_sup = REPHRASER(y_i)   # paraphrase human narration
                        src_temp = tau_r
                    else:
                        y_sup = NARRATOR(x_i)    # narrate from video
                        src_temp = tau_n
                        tilde_B_l.append((x_i, y_sup, src_temp))
                
                tilde_B_u = []
                for x_j in sample(B_u):
                    y_sup = NARRATOR(x_j)        # narrate unlabeled clip
                    tilde_B_u.append((x_j, y_sup, tau_n))
                
                batch = tilde_B_l + tilde_B_u
                
                # 2) Encode and normalize
                V = [normalize(f_theta(x)) for (x, y, _) in batch]
                T = [normalize(g_phi(y))   for (x, y, _) in batch]
                Tau = [src_temp for (_, _, src_temp) in batch]
                
                # 3) CLIP-style symmetric loss with source-aware temperatures
                loss = symmetric_infonce(V, T, Tau)
                
                # 4) Optimize dual-encoders
                update(f_theta, g_phi, loss)
        \end{mintedbox}
        
        \subsubsection{Architecture and implementation details}
        \label{subsubsec:chapter24_lavila_arch}
        
        \paragraph{Dual-encoder backbone}
        The model follows CLIP-style dual encoders: a TimeSformer visual encoder (spatial attention initialized from a ViT trained contrastively on image--text pairs) and a 12-layer Transformer text encoder; a linear projection maps both to a 256-dim joint space. Pretraining uses 4 frames per clip; downstream finetuning typically uses 16 frames. 
        
        \newpage
        
        \paragraph{NARRATOR design and training}
        The video encoder for NARRATOR is the frozen dual-encoder image/video backbone plus an \emph{attention-pooling} module (Eq.~\ref{eq:chapter24_lavila_attnpool}) that produces a fixed number of visual embeddings regardless of resolution; these condition a frozen GPT-2~XL decoder via periodically inserted cross-attention blocks with tanh-gating and layer norms. Training on Ego4D video--narration pairs uses FP32 for stability; checkpoints are selected by word-level accuracy and perplexity on held-out pairs. 
        
        \begin{figure}[H]
            \centering
            \includegraphics[width=0.7\textwidth]{Figures/Chapter_24/LaVILA_language_supervision.jpg}
            \caption{Language supervision from REPHRASER (text$\rightarrow$text) and NARRATOR (video$\rightarrow$text); the latter uses attention pooling over video tokens and cross-attention modules inside a frozen GPT-2 decoder \,\,\, Source: \cite{zhao2022_lavila}.}
            \label{fig:chapter24_lavila_langsup}
        \end{figure}
        
        \begin{figure}[H]
            \centering
            \includegraphics[width=0.7\textwidth]{Figures/Chapter_24/LaVILA_narrator_rephraser.jpg}
            \caption{Qualitative outputs from NARRATOR and REPHRASER; the former focuses on actions and interacted objects, the latter diversifies phrasing via synonymy and reordering \,\,\, Source: \cite{zhao2022_lavila}.}
            \label{fig:chapter24_lavila_qual}
        \end{figure}
        
        \paragraph{Pretraining schedule and input processing}
        Pretraining on Ego4D runs for 5 epochs with AdamW, weight decay $0.01$, fixed LR $3{\times}10^{-5}$, mixed precision (FP16) and gradient checkpointing; total batch size reaches $1024$ (e.g., $32{\times}32$ or $16{\times}64$ per-GPU setups). Videos are segmented into $5$-minute chunks with the short side scaled to $288$; 4 frames are sampled uniformly within the clip window with standard random resized crops.
        
        \newpage
        
        \subsubsection{Experiments}
        \label{subsubsec:chapter24_lavila_experiments}
        
        \paragraph{Benchmarks and protocols}
        Table~\ref{tab:chapter24_lavila_datasets} lists the downstream tasks used to evaluate \emph{LaViLa}: egocentric multi-instance retrieval (EK-100 MIR), egocentric QA and temporal localization (Ego4D MCQ, NLQ), action recognition (EGTEA Gaze+, CharadesEgo), and third-person recognition (UCF-101, HMDB-51). Evaluations follow three standard protocols: zero-shot (ZS), finetuning (FT), and linear probing (LP).
        
        \begin{table}[H]
            \centering
            \small
            \setlength{\tabcolsep}{6pt}
            \caption{Downstream datasets and evaluation protocols for LaViLa.}
            \label{tab:chapter24_lavila_datasets}
            \resizebox{\linewidth}{!}{
                \begin{tabular}{lcccc}
                    \toprule
                    \textbf{Dataset} & \textbf{Task} & \textbf{Egocentric} & \textbf{Metrics} & \textbf{Protocol} \\
                    \midrule
                    Epic-Kitchens-100 & MIR / CLS & Yes & mAP, nDCG / Top-1 & ZS, FT \\
                    Ego4D & MCQ / NLQ & Yes & Accuracy / Recall@N & ZS / FT \\
                    EGTEA Gaze+ & CLS & Yes & Top-1, Mean acc. & ZS, FT \\
                    CharadesEgo & CLS & Yes & Video-level mAP & ZS, FT \\
                    UCF-101 & CLS & No & Mean acc. & LP \\
                    HMDB-51 & CLS & No & Mean acc. & LP \\
                    \bottomrule
                \end{tabular}
            }
        \end{table}
        
        \paragraph{Headline results}
        \emph{LaViLa} establishes strong or state-of-the-art performance across first- and third-person settings by leveraging dense LLM narrations and a source-aware contrastive schedule \cite{zhao2022_lavila}. On EK-100 MIR (Table~2 in \cite{zhao2022_lavila}), ZS with TimeSformer-L (TSF-L) attains 40.0 mAP (V$\rightarrow$T) and 32.2 mAP (T$\rightarrow$V), averaging 36.1 mAP; FT reaches 54.7/47.1 mAP (avg.\ 50.9). On Ego4D (Table~3), \emph{LaViLa}-L achieves 94.5\% inter-video and 63.1\% intra-video accuracy on MCQ, and R@1$=$12.05 at mIoU@0.3 on NLQ. On EGTEA (Table~4), FT with TSF-L yields 81.75\% top-1 and 76.00\% mean accuracy. On CharadesEgo (Table~5), ZS/FT mAP are 28.9/36.1. With third-person pretraining (Table~6), linear probing attains 88.1\% on UCF-101 and 61.5\% on HMDB-51.
        
        \begin{figure}[H]
            \centering
            \includegraphics[width=0.5\textwidth]{Figures/Chapter_24/LaVILA_comparison_prev_SOTA.jpg}
            \caption{Comparison to prior SOTA across egocentric and third-person video understanding; LaViLa attains new state of the art via narration-supervised alignment. Source: \cite{zhao2022_lavila}.}
            \label{fig:chapter24_lavila_sota}
        \end{figure}
        
        \paragraph{Summary of main experiments and ablations}
        Pretraining uses roughly 4M $\sim$1\,s narrated clips from Ego4D and evaluates zero-shot (ZS), finetuned (FT), and linear-probe (LP) settings across egocentric and third-person tasks~\cite{zhao2022_lavila}.
        \begin{itemize}
            \item \textbf{Narration quality and downstream effect.} \emph{How quality is measured:} On held-out Ego4D clips, NARRATOR outputs are compared to human references using standard captioning metrics—\emph{METEOR} (0–1; matches content with synonym/fragment rewards), \emph{ROUGE\mbox{-}L} (0–1; longest common subsequence overlap), and \emph{CIDEr} (higher is better; consensus with multiple references). With GPT\mbox{-}2\,XL as the NARRATOR, the paper reports METEOR $0.289$, ROUGE\mbox{-}L $0.530$, CIDEr $0.940$, indicating fluent, on\mbox{-}topic descriptions \emph{at the same timestamps as the 1\,s clips}. \emph{Why it matters:} These dense, time\mbox{-}aligned sentences provide richer supervision than sparse clip labels or noisy ASR, yielding stronger video–text alignment. \emph{Observed effect:} Using these narrations for pretraining correlates with better downstream retrieval on EK\mbox{-}100 MIR (e.g., mAP $26.2$ with GPT\mbox{-}2\,XL vs.\ $24.3$ with a smaller GPT\mbox{-}2 and $20.1$ with random\mbox{-}init GPT\mbox{-}2\,XL), demonstrating that higher caption fidelity translates into better alignment~\cite{zhao2022_lavila}.
            \item \textbf{Epic\mbox{-}Kitchens\mbox{-}100 MIR (retrieval).} EK\mbox{-}100 comprises long, egocentric cooking videos with many fine\mbox{-}grained actions; the Multi\mbox{-}Instance Retrieval task matches text queries to the correct short action segments across long videos and is scored by mAP/nDCG. With a TimeSformer\mbox{-}L backbone, ZS yields $\sim 40.0$ mAP (video$\rightarrow$text) and $\sim 32.2$ mAP (text$\rightarrow$video), and FT rises to $\sim 54.7$ and $\sim 47.1$ mAP, respectively, evidencing robust narration\mbox{-}supervised alignment in both directions~\cite{zhao2022_lavila}.
            \item \textbf{Ego4D QA and temporal localization.} Ego4D includes multiple\mbox{-}choice QA (MCQ; inter\mbox{-}video / intra\mbox{-}video) and natural language queries (NLQ) that require pinpointing when a described event occurs. LaViLa improves MCQ accuracy (e.g., $\sim 94.5\%$ inter\mbox{-}video and $\sim 63.1\%$ intra\mbox{-}video) and boosts NLQ recall at fixed temporal IoU (e.g., R@1 at mIoU$\,{=}\,0.3$ $\sim 12.05$), showing that dense narrations help the model learn to temporally ground text in long, first\mbox{-}person videos~\cite{zhao2022_lavila}.
            \item \textbf{Egocentric action recognition.} On EGTEA Gaze\mbox{+} and CharadesEgo, LaViLa attains strong ZS and FT results (e.g., EGTEA top\mbox{-}1 accuracy $>{}80\%$ with TimeSformer\mbox{-}L; CharadesEgo ZS and FT mAP surpass prior contrastive/pretext methods), indicating that narration\mbox{-}aligned features transfer beyond retrieval to closed\mbox{-}set classification~\cite{zhao2022_lavila}.
            \item \textbf{Third\mbox{-}person generalization.} Despite pretraining on egocentric footage, linear probing on trimmed third\mbox{-}person datasets confirms broader utility: with TimeSformer\mbox{-}L, UCF\mbox{-}101 and HMDB\mbox{-}51 achieve $\sim 88\%$ and $\sim 62\%$ accuracy, respectively, suggesting the learned representation is not tied to first\mbox{-}person viewpoints and generalizes to conventional action clips~\cite{zhao2022_lavila}.
        \end{itemize}
        
        \newpage
        
        \paragraph{Ablations}
        Ablations (Sec.~5.4; Tables~7--8 in \cite{zhao2022_lavila}) isolate which design choices drive alignment quality. Unless noted, numbers reference EK\mbox{-}100 multi\mbox{-}instance retrieval (MIR) mAP on 1\,s Ego4D clips as a proxy for video--text alignment.
        \begin{itemize}
            \item \textbf{Narrator quality and LLM scale.} \emph{What was tested:} The NARRATOR used to create training captions was varied among (a) GPT\mbox{-}2\,XL initialized from WebText and then visually conditioned via cross\mbox{-}attention, (b) a smaller GPT\mbox{-}2 (pretrained), and (c) GPT\mbox{-}2\,XL with random initialization trained only on video captions. Quality was measured against human narrations with METEOR/ROUGE\mbox{-}L/CIDEr, and the downstream effect was measured by EK\mbox{-}100 MIR after using each narrator’s outputs for pretraining. \emph{Results:} Pretrained GPT\mbox{-}2\,XL yields the strongest captions (METEOR $0.289$, ROUGE\mbox{-}L $0.530$, CIDEr $0.940$) and the best retrieval (mAP $26.2$) versus the smaller GPT\mbox{-}2 ($24.3$) and random\mbox{-}init GPT\mbox{-}2\,XL ($20.1$). \emph{Why it matters:} Large, pretrained language priors generate more fluent and temporally specific narrations, producing a cleaner and denser supervision signal for contrastive alignment~\cite{zhao2022_lavila}.
            
            \item \textbf{Sampling strategy for narration.} \emph{What is sampled:} At \emph{generation time}, the NARRATOR samples \emph{token sequences} (full sentences) for each video clip. The study compares decoding methods: beam search (high\mbox{-}probability single phrasing) versus \emph{nucleus sampling} (stochastic next\mbox{-}token draws from the top\mbox{-}p mass, here $p{=}0.95$), producing $K{=}10$ alternative narrations \emph{per clip}; a repeated\mbox{-}sampling variant increases this candidate pool further. \emph{Results:} Nucleus sampling outperforms beam search (mAP $29.7$ vs.\ $27.9$), and repeated sampling adds $\sim{+}1.8$ mAP. \emph{Why it matters:} Multiple diverse captions for the same clip cover alternative phrasings and event decompositions, improving robustness and generalization in contrastive training~\cite{zhao2022_lavila}.
            
            \item \textbf{Backbone capacity and input resolution.} \emph{What was tested:} TimeSformer\mbox{-}B $\rightarrow$ TimeSformer\mbox{-}L $\rightarrow$ TimeSformer\mbox{-}L@HR. \emph{Results:} Consistent gains as capacity and resolution increase (mAP $26.0 \rightarrow 29.7 \rightarrow 35.0$). \emph{Why it matters:} Stronger vision encoders exploit dense narration supervision to capture finer motion and interactions, amplifying transfer~\cite{zhao2022_lavila}.
            
            \item \textbf{Clip length and narration density.} \emph{What was tested:} Durations $\{0.5\,\mathrm{s},1\,\mathrm{s},2\,\mathrm{s}\}$ and sentences per clip (sparse $N{=}1$ vs.\ dense $N{\approx}10$). \emph{Results:} $1$\,s clips with dense narrations perform best, typically ${+}2$--${+}4$ mAP over sparser/longer settings. \emph{Why it matters:} Short, action\mbox{-}focused clips with several sentences balance coverage and precision, yielding clearer temporal grounding~\cite{zhao2022_lavila}.
            
            \item \textbf{Semi\mbox{-}supervised efficiency.} \emph{What was tested:} Training with $10\%$--$100\%$ of the narrated Ego4D data. \emph{Results:} Using only $50\%$ of the narrated data remains competitive with full\mbox{-}label baselines on EK\mbox{-}100 MIR and Ego4D tasks. \emph{Why it matters:} LLM narrations provide label\mbox{-}efficient supervision, sustaining strong performance under reduced human annotation budgets~\cite{zhao2022_lavila}.
            
            \item \textbf{Temperature setting in the contrastive loss.} \emph{What was tested:} Fixed temperature versus learned or source\mbox{-}specific temperatures. \emph{Results:} A single fixed temperature (e.g., $\tau{=}0.07$) is most stable and yields the best overall metrics in this setting. \emph{Why it matters:} Simple scaling avoids over\mbox{-}weighting noisier pseudo\mbox{-}captions or under\mbox{-}weighting clean paraphrases, leading to smoother optimization~\cite{zhao2022_lavila}.
        \end{itemize}
        
        \newpage
        
        \subsubsection{Limitations and future directions}
        \label{subsubsec:chapter24_lavila_limits}
        
        \paragraph{Observed constraints}
        \emph{LaViLa} learns strong video–text alignment from dense narrations, but several boundaries remain clear in the original paper.~\cite{zhao2022_lavila} 
        \begin{itemize}
            \item \textbf{Narration quality and bias.} Supervision ultimately inherits the style and limitations of narrated text (human or LLM-generated). Despite careful prompting and sampling, narrations can be repetitive, partially off-topic, or unevenly distributed across events, which may cap downstream temporal precision~\cite{zhao2022_lavila}.
            \item \textbf{Alignment over generation.} The dual-encoder is optimized with a contrastive objective for retrieval and recognition. It is not trained for open-ended text generation, step-by-step explanations, or multi-turn dialogue; those abilities require an autoregressive language modeling objective and an explicit interface to a generative LLM~\cite{zhao2022_lavila}.
            \item \textbf{Clip-level horizon.} Training focuses on short clips paired with sentences. This yields robust local alignment but leaves long-range reasoning (ordering, causality, procedure tracking) underconstrained unless additional mechanisms summarize or chain evidence over time~\cite{zhao2022_lavila}.
            \item \textbf{Modality scope.} The method centers on video–text. Audio is not explicitly modeled in the pretraining objective, so grounding to acoustic events or off-screen sound requires further extensions~\cite{zhao2022_lavila}.
            \item \textbf{Domain and language coverage.} Narrations are predominantly English and egocentric in the main setup, which can introduce domain or language bias when transferring to third-person or multilingual settings~\cite{zhao2022_lavila}.
        \end{itemize}
        
        \paragraph{Future work}
        These constraints suggest clear next steps for narration-supervised pretraining.
        \begin{itemize}
            \item \textbf{Audio-visual grounding.} Incorporate ASR and raw audio features with timestamped alignment so captions can reference sounds and speech, not only visuals, improving moment retrieval and event disambiguation.
            \item \textbf{Instruction-tuned conversational layers.} Add a lightweight connector from the frozen video encoder to a pretrained LLM and fine-tune with multimodal instructions to enable open-ended answers and multi-turn dialogue on video.
            \item \textbf{Long-context summarization.} Introduce hierarchical pooling or memory to aggregate many clips into compact tokens, enabling hour-scale reasoning and timeline queries without prohibitive compute.
            \item \textbf{Multilingual and broader domains.} Generate and curate narrations across languages and domains to reduce bias and improve transfer, with calibration or filtering to control LLM style drift.
            \item \textbf{Quality control of pseudo-labels.} Use confidence estimates, agreement checks, or retrieval-based filtering to keep narrated supervision precise while scaling data.
        \end{itemize}
        
        \paragraph{Bridge to instruction-tuned video–LLMs}
        Narration-supervised alignment in \emph{LaViLa} supplies dense, scalable supervision that yields a strong video encoder for downstream use.~\cite{zhao2022_lavila} This makes a natural foundation for instruction-tuned video–LLMs such as \emph{Video-LLaMA}, where a pretrained video encoder is coupled to a generative LLM via a lightweight connector and adapted on conversational data to add open-ended reasoning and dialogue over video~\cite{zhang2023_videollama,cheng2024_videollama2}.
        
    \end{enrichment}
    
    
    \newpage
    
    \begin{enrichment}[Video\mbox{-}LLaMA 1: Instruction\mbox{-}Tuned Video LLM][subsection]
        \label{enr:subsec_chapter24_videollama1}
        
        \subsubsection{Motivation}
        \label{subsubsec:chapter24_videollama1_motivation}
        
        \paragraph{Why audio--visual LLMs?}
        Most multimodal LLMs circa early 2023 target either image–text \cite{li2023_blip2,liu2023_llava,zhu2023_minigpt4} or audio–text \cite{huang2023_audiogpt}, leaving \emph{video} under-served and typically silent.\footnote{See also VideoChat \cite{li2024_videochat} and Video-ChatGPT \cite{maaz2024_video_chatgpt} for vision-only video dialogue.} Video-LLaMA1 addresses two gaps for video understanding: (i) modeling \emph{temporal change} in visual scenes, and (ii) \emph{integrating} audio with vision in a single LLM-centric framework.
        
        \paragraph{Design goal}
        Leverage strong, \emph{frozen} foundation encoders for vision and audio, plus a frozen LLM, and learn only light adapters to connect modalities, preserving priors while enabling efficient instruction tuning.
        
        \begin{figure}[H]
            \centering
            \includegraphics[width=0.75\linewidth]{Figures/Chapter_24/VideoLLaMA_architecture.jpg}
            \caption{Overall architecture of Video-LLaMA: a dual-branch design that converts video frames (left, Vision–Language branch) and audio segments (right, Audio–Language branch) into small sets of \emph{query tokens}, projects them to the LLM embedding space, and concatenates them with text tokens to condition a frozen LLM (Vicuna/LLaMA). Vision: frozen image encoder (ViT) + Video Q-Former + linear projector. Audio: frozen audio encoder (ImageBind) + Audio Q-Former + projector. This lets the LLM reason jointly over sight and sound. Adapted from \cite{zhang2023_videollama}.}
            \label{fig:chapter24_videollama_arch}
        \end{figure}
        
        \subsubsection{Method: Multi-Branch Cross-Modal Training with Q-Formers}
        \label{subsubsec:chapter24_videollama1_method}
        
        \paragraph{Problem setup and notation}
        Given a video with $N$ frames and waveform audio split into $M$ short segments, the goal is to produce a response $\hat{y}$ to a user instruction $y^{\text{instr}}$ conditioned on video and audio. Video-LLaMA1 constructs two \emph{query} sequences: $\hat{\mathbf{v}}\in\mathbb{R}^{k_V\times d}$ from frames and $\hat{\mathbf{a}}\in\mathbb{R}^{k_A\times d}$ from audio, projects them into the LLM embedding space, and concatenates them with tokenized $y^{\text{instr}}$ as a \emph{soft prompt} to drive the frozen LLM’s next-token generation (See Fig.~\ref{fig:chapter24_videollama_arch}).
        
        \paragraph{Vision–Language branch}
        Per-frame features are extracted by a \emph{frozen} image encoder (ViT/G from EVA-CLIP within BLIP-2), yielding $V=[\mathbf{v}_1,\ldots,\mathbf{v}_N]$, where $\mathbf{v}_i\in\mathbb{R}^{K_f\times d_f}$. Temporal position embeddings are added across frames. A \emph{Video Q-Former} (same architecture as BLIP-2’s Query Transformer) aggregates across time via learnable queries to produce $k_V$ video embedding vectors $\hat{\mathbf{v}}\in\mathbb{R}^{k_V\times d_v}$, followed by a linear projector mapping into the LLM token space to form video query tokens. These tokens are concatenated with text embeddings as a visual soft prompt to the frozen LLM.
        
        \paragraph{Audio–Language branch}
        Audio is uniformly segmented into $M$ chunks (typically $2$\,s each), converted into log-Mel spectrograms (128 Mel bins), and fed to a \emph{frozen} ImageBind audio encoder to obtain a sequence of segment embeddings
        $A=[\mathbf{a}_1,\ldots,\mathbf{a}_M]$, with $\mathbf{a}_m\in\mathbb{R}^{d_a}$~\cite{girdhar2023_imagebind}. \textit{What is ImageBind?} ImageBind is a multimodal foundation model trained with contrastive learning so that images, text, audio, and other modalities share a single embedding space. It uses \emph{images as a pivot}: audio is aligned to images and text is aligned to images, which \emph{binds} audio and text transitively (e.g., a “bark” sound and the word “dog” end up close), providing semantically grounded audio features without extra audio–text supervision~\cite{girdhar2023_imagebind}.
        
        An \emph{Audio Q-Former} (a lightweight, trainable query-based Transformer) with temporal position embeddings then attends over $A$ and compresses the variable-length sequence into a fixed set of $k_A$ audio queries,
        \[
        \hat{\mathbf{a}}\in\mathbb{R}^{k_A\times d_a},
        \]
        where the $k_A$ learnable query tokens aggregate salient temporal cues. A final linear projector $W_a\in\mathbb{R}^{d_a\times d_{\text{LLM}}}$ maps these queries into the LLM token space,
        \[
        \mathbf{z}^{(a)}=\hat{\mathbf{a}}\,W_a \in \mathbb{R}^{k_A \times d_{\text{LLM}}},
        \]
        yielding \emph{audio query tokens} that are concatenated with the user prompt (and, when present, visual tokens) to condition the frozen LLM for audio-grounded video dialogue.
        
        \paragraph{Training curriculum}
        Video-LLaMA1 follows a staged, dual-branch curriculum that first teaches the adapters to \emph{describe} from visual/audio inputs and then sharpens \emph{instruction following} for dialogue. Crucially, the large backbone encoders (vision: BLIP\mbox{-}2’s ViT\mbox{-}G/14 from EVA\mbox{-}CLIP; audio: ImageBind) and the language model (LLaMA/Vicuna) are kept \emph{frozen}; only the lightweight bridges (Video/Audio Q\mbox{-}Formers, temporal position embeddings, and linear projectors) are optimized in all stages \cite{zhang2023_videollama}. This design “guides the frozen LLM” using learned query tokens (soft prompts), and the paper’s “fine\mbox{-}tuning” wording refers to adapting these bridges with different datasets, not unfreezing the LLM.
        
        \newpage
        
        \begin{enumerate}[label=(\alph*), leftmargin=1.4em, itemsep=0.25em]
            \item \textbf{Vision pre\mbox{-}training (video/image$\to$text generation).} Using large vision–caption corpora (WebVid\mbox{-}2M short clips; filtered CC595k image captions), the frozen BLIP\mbox{-}2 vision encoder produces frame features. Learnable \emph{temporal position embeddings} are added to inject ordering over frames, and a \emph{Video Q\mbox{-}Former} aggregates them into a fixed number of \emph{video query tokens}. A linear projector maps these tokens to the LLM embedding space; concatenating them with text inputs prompts the \emph{frozen} LLM to generate captions. This stage prioritizes broad visual knowledge despite caption noisiness \cite[Sec.~2.1--2.2]{zhang2023_videollama}.
            \item \textbf{Vision instruction tuning (image/video dialogue).} The same visual adapters are further trained on high\mbox{-}quality instruction data (e.g., MiniGPT\mbox{-}4 image descriptions, LLaVA image instructions, Video\mbox{-}Chat video instructions) so that the \emph{frozen} LLM better follows prompts, answers questions, and maintains multi\mbox{-}turn coherence when conditioned on the learned video tokens \cite[Sec.~2.2]{zhang2023_videollama}.
            \item \textbf{Audio pre\mbox{-}training (audio$\to$text via pivot).} Due to scarce audio–text pairs, the audio branch leverages a \emph{frozen} ImageBind audio encoder to produce segment features that already live in a multimodal space aligned with images/text \cite{girdhar2023_imagebind}. Uniform 2\,s chunks are converted to 128\mbox{-}bin log\mbox{-}Mel spectrograms and encoded into a sequence $A=[\mathbf{a}_1,\ldots,\mathbf{a}_M]$. An \emph{Audio Q\mbox{-}Former} with temporal position embeddings fuses $A$ into a fixed set of audio query tokens, which are projected to the LLM space. Training uses the \emph{same vision–text} data as a \emph{pivot}: the loss is applied on text generation while conditioning the LLM on audio\mbox{-}side tokens whose features are aligned to vision via ImageBind. This yields zero\mbox{-}shot audio understanding at inference, even without direct audio–text supervision \cite[Sec.~2.1.2,~2.2.2]{zhang2023_videollama,girdhar2023_imagebind}.
        \end{enumerate}
        
        \paragraph{How images and videos share one encoder}
        Video\mbox{-}LLaMA1 uses a single frozen image encoder (the BLIP\mbox{-}2 vision tower) for both images and videos by \emph{treating an image as a 1\mbox{-}frame video}. Concretely: (i) for a static image, the encoder runs once to produce patch tokens for that single frame; (ii) for a video, $N$ frames are uniformly sampled and each frame is encoded independently by the same tower, yielding a sequence of per\mbox{-}frame tokens; (iii) learnable \emph{temporal} position embeddings are added to mark frame order; and (iv) a \emph{Video Q\mbox{-}Former} cross\mbox{-}attends over this ordered sequence and compresses it into a fixed number of visual query tokens, which are then projected into the LLM embedding space. This unified pathway avoids modality\mbox{-}specific encoders while the temporal embeddings and the Q\mbox{-}Former supply the missing “when” signal and spatio\mbox{-}temporal aggregation absent from a purely image\mbox{-}trained backbone \cite[Sec.~2.1.1--2.2]{zhang2023_videollama}.
        
        \paragraph{Positional encoding (vision \& audio)}
        Because the frozen encoders do not model time, Video\mbox{-}LLaMA1 injects \emph{learnable temporal} position embeddings \emph{after} feature extraction: per\mbox{-}frame (vision) and per\mbox{-}segment (audio) embeddings are added before the corresponding Q\mbox{-}Formers, enabling temporal reasoning without unfreezing the backbones \cite[Sec.~2.1.1--2.1.2]{zhang2023_videollama}.
        
        \paragraph{Learning objective (unified view)}
        All stages optimize the standard autoregressive language\mbox{-}modeling loss (next\mbox{-}token negative log\mbox{-}likelihood) on the frozen LLM, conditioned on the concatenation of modality query tokens and textual context:
        \[
        \mathcal{L}_{\mathrm{LM}}
        \,=\,
        -\sum_{t=1}^{T}
        \log p\!\left(
        w_t \,\middle|\,
        w_{<t},\,
        Q_v \ \text{and/or}\ Q_a,\,
        c
        \right),
        \]
        where $Q_v$/$Q_a$ are the fixed\mbox{-}length video/audio query tokens produced by the Q\mbox{-}Formers and projected to the LLM space, $c$ is the text context (caption/instruction), and $w_t$ are target tokens. No extra losses are introduced; the adapters learn to act as \emph{soft multimodal prompts} that elicit correct generations from a frozen LLM \cite[Sec.~2.2]{zhang2023_videollama}.
        
        \paragraph{Intuition and roles}
        Q-Formers act as \emph{learnable compressors} that query and distill dense frame or audio features into a small, fixed set of tokens that an LLM can reliably consume, mirroring BLIP-2 for images but extended temporally (video) and across modality (audio). ImageBind provides the audio branch with a pragmatic route to align with text even without abundant audio–text pairs. Together, they let a frozen LLM “see and hear” with minimal new parameters.
        
        \subsubsection{Architecture \& Implementation Details}
        \label{subsubsec:chapter24_videollama1_arch_impl}
        
        \paragraph{Backbones and frozen parts}
        Vision uses the BLIP-2 visual stack: EVA-CLIP ViT-G/14 + Q-Former (both \emph{frozen}). Audio uses the \emph{frozen} ImageBind audio encoder. The LLM is Vicuna/LLaMA (frozen). Trainable parts are: temporal position embeddings, Video Q-Former, Audio Q-Former, and small linear projectors to the LLM token space.
        
        \paragraph{Video tokens}
        Each frame yields $K_f$ image tokens; temporal position embeddings index frames; Video Q-Former outputs $k_V$ video query tokens. A linear projector maps these into the LLM embedding dimension; tokens are then prepended/concatenated to text embeddings as a soft prompt.
        
        \paragraph{Audio tokens}
        Audio is chunked, Mel-spectrogrammed, embedded via ImageBind, fused by the Audio Q-Former into $k_A$ tokens, then projected to the LLM space and concatenated alongside video tokens.
        
        \paragraph{GEMINI additions (intuitive recap)}
        The dual-branch design feeds a central LLM with \emph{what it sees} (Video Q-Former summary of frames) and \emph{what it hears} (Audio Q-Former summary of audio). The LLM then produces responses grounded in both modalities, e.g., recognizing a rocket launch \emph{and} describing engine roar (See Fig.~\ref{fig:chapter24_videollama_arch}).
        
        \begin{table}[H]
            \centering
            \small
            \setlength{\tabcolsep}{8pt}
            \caption{Comparison with popular multimodal LLMs: Video-LLaMA uniquely handles \emph{images}, \emph{silent videos}, and \emph{audio} jointly (Adapted from Table~1 of \cite{zhang2023_videollama}).}
            \label{tab:videollama_model_capabilities}
            \resizebox{0.85\linewidth}{!}{%
                \begin{tabular}{lccc}
                    \toprule
                    \textbf{Model Name} & \textbf{Static Image} & \textbf{Silent Video} & \textbf{Audio} \\
                    \midrule
                    BLIP-2 \cite{li2023_blip2}                     & \cmark &  &  \\
                    MiniGPT-4 \cite{zhu2023_minigpt4}             & \cmark &  &  \\
                    LLaVA \cite{liu2023_llava}                    & \cmark &  &  \\
                    mPLUG-Owl \cite{ye2024_mplug_owl}             & \cmark & \cmark &  \\
                    VideoChat \cite{li2024_videochat}             & \cmark & \cmark &  \\
                    AudioGPT \cite{huang2023_audiogpt}            &  &  & \cmark \\
                    Video-ChatGPT \cite{maaz2024_video_chatgpt}   & \cmark & \cmark &  \\
                    \midrule
                    Video-LLaMA \cite{zhang2023_videollama}       & \cmark & \cmark & \cmark \\
                    \bottomrule
                \end{tabular}%
            }
        \end{table} 
        
        \subsubsection{Experiments and Ablations}
        \label{subsubsec:chapter24_videollama1_expts}
        
        \paragraph{Qualitative capabilities}
        Video-LLaMA1 enables multi-turn \emph{audio–visual} dialogue: (a) answering questions grounded jointly in background sound and visual content; (b) describing actions over time (temporal reasoning across frames); (c) analyzing single images; and (d) recognizing well-known landmarks (Figure~\ref{fig:chapter24_videollama_examples}). These examples illustrate how the model combines the vision and audio branches to condition a frozen (or LoRA-adapted) LLM for grounded responses~\cite{zhang2023_videollama}.
        
        \begin{figure}[H]
            \centering
            \includegraphics[width=0.85\linewidth]{Figures/Chapter_24/VideoLLaMA_generated_examples.jpg}
            \caption{Examples generated by Video-LLaMA. (a) Answers based on background sound and visual content. (b) Identifies actions over time. (c) Understands static images. (d) Recognizes famous landmarks. Adapted from \cite{zhang2023_videollama}.}
            \label{fig:chapter24_videollama_examples}
        \end{figure}
        
        \paragraph{Tasks and metrics (at a glance)}
        Evaluation follows standard video–language setups~\cite{zhang2023_videollama}: (i) \emph{Video QA} on MSRVTT-QA and ActivityNet-QA (reported as answer \textbf{accuracy}); (ii) \emph{Video captioning} on MSVD/MSRVTT (reported with \textbf{CIDEr}/\textbf{BLEU}/\textbf{METEOR}); and (iii) \emph{Retrieval} on MSRVTT (reported with \textbf{Recall@K}). The paper focuses on zero-shot and instruction-tuned settings, highlighting gains attributable to the staged curriculum.
        
        \newpage
        
        \paragraph{Training stages and ablations}
        A staged curriculum underpins the stability and performance of \emph{Video\mbox{-}LLaMA1}, with pre-training for grounding followed by instruction tuning for conversational ability; empirical studies in \cite[Sec.~2.1--2.3, 4]{zhang2023_videollama} confirm the contribution of each step.
        \begin{itemize}
            \item \textbf{Vision caption pre-training $\rightarrow$ broad visual grounding.} Using large, weakly supervised corpora (WebVid-2M video clips and filtered image–caption sets), the \emph{Video Q-Former} and projector learn to produce a compact set of visual query tokens that condition the frozen LLM to generate captions, yielding stronger descriptive ability and better retrieval-style baselines than adapter-free or frozen-feature variants. \emph{Ablation:} Query-based cross-attention with temporal position embeddings outperforms naive frame pooling for video QA, indicating a learnable bottleneck is more effective for distilling spatio–temporal cues into the LLM token budget~\cite[Sec.~2.1, 4]{zhang2023_videollama}.
            .\item \textbf{Instruction tuning $\rightarrow$ QA accuracy and multi-turn dialogue.} Fine-tuning the same visual adapters together with a lightweight LLM adaptation on clean image/video instruction–response data improves answer relevance and stabilizes multi-turn consistency compared to caption-only training, showing that instruction-formatted supervision is required to elicit conversational behaviour. \emph{Ablation:} Models trained only on captions underperform on question answering and dialogue coherence despite solid visual grounding~\cite[Sec.~2.3, 4]{zhang2023_videollama}.
            .\item \textbf{Audio pivoting $\rightarrow$ zero-shot audio understanding.} With a \emph{frozen} ImageBind audio encoder, the \emph{Audio Q-Former} is trained via a vision–text pivot to produce audio query tokens aligned to the LLM space without paired audio–text datasets, enabling audio-aware responses for questions that depend on background sounds. \emph{Ablation:} The pivoted audio branch outperforms variants that omit audio, providing a practical and data-efficient route to audio grounding~\cite[Sec.~2.2, 4]{zhang2023_videollama}.
        \end{itemize}
        
        \paragraph{Positioning w.r.t.\ LaViLa and related LMMs}
        Relative to \emph{LaViLa} (narration-supervised \emph{dual-encoder} optimized for alignment and retrieval), \emph{Video\mbox{-}LLaMA1} is a \emph{generative}, instruction-tuned audio–visual language model: it supports free-form answers, multi-turn dialogue, and audio grounding while reusing frozen perception backbones through Q-Formers. Compared to image-first instruction models (e.g., BLIP-2, LLaVA, MiniGPT-4), \emph{Video\mbox{-}LLaMA1} adds temporal modeling (frame sequences with temporal embeddings and a Video Q-Former) and an explicit audio branch via ImageBind, enabling questions that depend on motion and sound. Modality coverage is summarized in Table~\ref{tab:videollama_model_capabilities}, and unified follow-ups (e.g., LLaVA-OneVision) are discussed in Sec.~\ref{enr:subsec_chapter24_llava_onevision}.
        
        \subsubsection{Limitations and Future Directions}
        \label{subsubsec:chapter24_videollama1_limits}
        
        \paragraph{Observed constraints}
        Video-LLaMA1 is an early-stage prototype. The paper highlights: \textit{(1)} performance is bounded by the scale/quality of current training data; \textit{(2)} limited ability to handle long videos due to compute and fixed token budgets; and \textit{(3)} hallucinations inherited from the frozen LLM.
        
        \paragraph{Future work}
        The authors call for higher-quality audio–video–text alignment data, longer-context modeling for movies/TV-scale inputs, and mitigation strategies for hallucination. These directions naturally motivate successors (Video-LLaMA2/3) that extend clip length, strengthen audio–visual synchronization, and scale instruction data and adapters (see next subsections in this enrichment).
        
    \end{enrichment}
    
    \newpage
    
    \begin{enrichment}[Video\mbox{-}LLaMA 2: Enhanced Understanding, Efficiency][subsection]
        \label{enr:subsec_chapter24_videollama2}
        
        \paragraph{Overview and motivation}
        \emph{Video-LLaMA2}~\cite{cheng2024_videollama2} extends \emph{Video-LLaMA1} by replacing the Q-Former connector with a compute-efficient \emph{Spatial–Temporal Convolution (STC) connector} and by introducing a stronger audio pathway and staged audio–visual training. The goals are: (i) preserve local spatial–temporal structure while reducing video tokens; (ii) scale to longer clips without exploding token budgets; and (iii) strengthen audio understanding via a modern audio encoder (BEATs) and curriculum. The design keeps modality encoders \emph{frozen} and lets a lightweight connector plus an LLM handle fusion and reasoning, improving robustness and efficiency for instruction-following video chat and QA. 
        
        % ---------- Figure: Overall pipeline ----------
        \begin{figure}[H]
            \centering
            \includegraphics[width=0.85\linewidth]{Figures/Chapter_24/VideoLLaMA2_pipeline.jpg}
            \caption{Overall pipeline of \emph{Video-LLaMA2}. Frames are encoded by a frozen image encoder and passed through the STC connector before entering the LLM; audio is converted to log-Mel features, encoded, and aligned via an MLP block. Adapted from \cite{cheng2024_videollama2}.}
            \label{fig:chapter24_videollama2_pipeline}
        \end{figure}
        
        \subsubsection{Method}
        \label{subsubsec:chapter24_videollama2_method}
        
        \paragraph{Modality branches (concise)}
        \emph{Video\mbox{-}LLaMA2} feeds a (frozen–decoder) LLM with compact tokens produced by two parallel branches~\cite[Sec.~2]{cheng2024_videollama2}:
        \begin{itemize}
            \item \textbf{Vision.} Uniformly sampled frames $\rightarrow$ frozen CLIP ViT\mbox{-}L/14 (per\mbox{-}frame features) $\rightarrow$ \textbf{STC connector} (RegStage \,$\rightarrow$\, 3D conv with downsampling $(t,s,s)$ \,$\rightarrow$\, RegStage) $\rightarrow$ small MLP projector to LLM tokens~\cite[Sec.~2.1]{cheng2024_videollama2}.
            \item \textbf{Audio.} Waveform $\rightarrow$ log\mbox{-}Mel spectrograms (128 bins) $\rightarrow$ frozen \textbf{BEATs} encoder $\rightarrow$ two\mbox{-}layer MLP to LLM tokens (no Audio Q\mbox{-}Former)~\cite[Sec.~2.2]{cheng2024_videollama2}.
        \end{itemize}
        Tokens from both branches are concatenated with the text prompt and passed to the LLM for autoregressive generation.
        
        \newpage
        
        \paragraph{STC connector: step\mbox{-}by\mbox{-}step mechanics and intuition}
        \textit{Primer on RegStage.} \emph{RegStage} is the per\mbox{-}stage building block from the RegNet family~\cite{radosavovic2020_dnds} (implemented in \texttt{timm} and invoked in the paper’s pseudo\mbox{-}code), used here purely for \emph{spatial} refinement on each frame~\cite[Alg.\,1; Sec.\,2.1]{cheng2024_videollama2}. Concretely, a RegStage stacks several lightweight 2D residual bottleneck blocks (conv $\rightarrow$ norm $\rightarrow$ activation, optional squeeze\mbox{-}and\mbox{-}excitation and/or group convolutions), with a fixed channel width across the stage and an optional stride in the first block for spatial downsampling. It \emph{does not} mix information across time—every operation is intraframe—so it functions as a per\mbox{-}frame “detail enhancer” that sharpens edges, textures, and small objects before (and after) the temporal aggregation step in STC.
        
        \emph{Why RegStage vs.\ an ad\mbox{-}hoc 2D stack?} RegNet’s blocks arise from a \emph{regular} design space discovered by large\mbox{-}scale network design exploration~\cite{radosavovic2020_dnds}: channel widths evolve by a simple quantized linear rule, depth/width are balanced per stage, and the block recipe remains constant. This regularity yields predictable compute/accuracy scaling and strong accuracy\mbox{-}per\mbox{-}FLOP at a given budget, whereas “irregular” CNN stacks (arbitrary kernel/width changes per layer) tend to be harder to scale efficiently. In \emph{Video\mbox{-}LLaMA2}, this makes RegStage an ideal choice around the single 3D aggregation layer: it is (i) \emph{frame\mbox{-}local}—preserving temporal order for the downstream 3D step; (ii) \emph{parameter\mbox{-} and FLOP\mbox{-}efficient}—suited to long clips; and (iii) \emph{detail\mbox{-}retentive}—its pre/post spatial filtering mitigates the blurring that temporal downsampling can introduce~\cite[Sec.\,2.1]{cheng2024_videollama2}.
        
        \medskip
        Let $F \in \mathbb{R}^{T \times H' \times W' \times D_v}$ denote per\mbox{-}frame features from the frozen ViT (one image per frame). The STC (RegStage $\rightarrow$ Conv3D $\rightarrow$ RegStage) transforms $F$ into an order\mbox{-}aware token sequence $Q_v$ for the LLM:
        \begin{itemize}
            \item \textbf{(1) Pre\mbox{-}aggregation spatial interaction (RegStage\,\#1).} Apply a RegStage \emph{independently on each frame} to strengthen intraframe structure:
            \[
            F_1 = \mathrm{RegStage}_1(F).
            \]
            \emph{Intuition.} This step sharpens spatial details before any temporal mixing, so that subsequent compression does not wash out fine cues needed for OCR, small objects, or delicate hand\mbox{-}object interactions~\cite[Sec.\,2.1]{cheng2024_videollama2}.
            \item \textbf{(2) Spatio\mbox{-}temporal aggregation with explicit downsampling (3D Conv).} Treat the sequence as a 3D volume and aggregate with a single 3D convolution configured to \emph{downsample} by factors $(t,s,s)$ along time/space:
            \[
            F_2 = \mathrm{Conv3D}_{(t,s,s)}(F_1).
            \]
            This reduces the lattice roughly from $(T,H',W')$ to $\big(\lceil T/t\rceil,\lceil H'/s\rceil,\lceil W'/s\rceil\big)$ while encoding short\mbox{-}range motion and local temporal context. \emph{Intuition.} A 3D kernel ``looks'' at small space–time cubes and encodes \emph{what changes, where, and when} instead of averaging away motion; explicit $(t,s,s)$ makes the token budget predictable for long clips~\cite[Sec.\,2.1; Tab.\,1]{cheng2024_videollama2}.
            \item \textbf{(3) Post\mbox{-}aggregation refinement (RegStage\,\#2).} Apply a second RegStage on the downsampled volume:
            \[
            F_3 = \mathrm{RegStage}_2(F_2).
            \]
            \emph{Intuition.} This ``cleanup'' stage restores spatial sharpness and reduces artifacts introduced by aggressive downsampling, yielding more discriminative tokens for the LLM~\cite[Sec.\,2.1]{cheng2024_videollama2}.
            \item \textbf{(4) Projection to LLM tokens (MLP).} Flatten the 3D lattice into a sequence and map each vector to the LLM embedding with a small MLP:
            \[
            Q_v = \mathrm{MLP}\!\big(\mathrm{Flatten}(F_3)\big) \in \mathbb{R}^{K \times d_{\mathrm{LLM}}},
            \]
            where $K \approx \lceil T/t\rceil \!\cdot\! \lceil H'/s\rceil \!\cdot\! \lceil W'/s\rceil$ is the resulting visual token count set by $(t,s,s)$ (e.g., the paper often uses $(2,2,2)$). The sequence order follows scanline-in-time (preserving chronology), and $Q_v$ is concatenated with text (and optional audio tokens) for generation~\cite[Sec.\,2.1]{cheng2024_videollama2}.
        \end{itemize}
        
        \emph{Why this ``sandwich'' works.} A plain stack of 3D convolutions can over\mbox{-}mix space and time too early, blurring fine spatial structure; the RegStage–Conv3D–RegStage design deliberately separates \emph{delicate and dedicated spatial refinement} (before/after) from \emph{temporal aggregation} (middle), preserving locality while encoding motion. Ablations favor this configuration—especially with downsampling $(2,2,2)$—for superior MC\mbox{-}VQA averages under tighter token budgets~\cite[Tab.\,1]{cheng2024_videollama2}.
        
        \begin{figure}[H]
            \centering
            \includegraphics[width=0.85\linewidth]{Figures/Chapter_24/VideoLLaMA2_STC.jpg}
            \caption{STC connector: RegStage $\rightarrow$ 3D convolution for spatio–temporal aggregation (e.g., downsampling $(2,2,2)$) $\rightarrow$ RegStage, followed by a small MLP to produce LLM tokens; preserves temporal order while reducing token count. Adapted from \cite{cheng2024_videollama2}.}
            \label{fig:chapter24_videollama2_stc}
        \end{figure}
        
        \paragraph{Why STC instead of a plain 3D CNN or a Q\mbox{-}Former?}
        \begin{itemize}
            \item \textbf{Versus a plain 3D CNN stack.} Repeated 3D mixing tends to smear fine details by coupling space and time at every layer; STC confines temporal aggregation to one explicit step and uses RegStage to protect and then restore spatial fidelity, improving accuracy at a comparable or smaller token budget~\cite[Sec.\,2.1; Tab.\,1]{cheng2024_videollama2}.
            \item \textbf{Versus the V1 Q\mbox{-}Former.} Attention\mbox{-}based querying is flexible but token\mbox{-}hungry on long sequences and may disturb chronological order via learned resampling; STC preserves frame order by construction, offers deterministic $(t,s,s)$ reduction, and scales linearly with clip length—yielding better MC\mbox{-}VQA averages under tighter budgets in the paper’s comparison~\cite[Sec.\,2.1; Tab.\,1]{cheng2024_videollama2}.
        \end{itemize}
        
        \newpage
        
        \paragraph{Implementation of STC in Python (from the paper)}
        
        \begin{mintedbox}{Python}
            import torch.nn as nn
            from timm.models.regnet import RegStage
            
            class STCConnector(nn.Module):
                def __init__(self, config, depth, mlp_depth):
                    # Temporal and spatial downsampling factor
                    td, sd = config.td, config.sd
                    # Input and output hidden dimension
                    in_size, out_size = config.in_size, config.out_size
                    # The first RegStage block
                    self.s1 = RegStage(depth=depth, in_chs=in_size, out_chs=out_size)
                    # Conv3D downsampler
                    self.downsampler = nn.Conv3d(in_channels=out_size,
                    out_channels=out_size,
                    kernel_size=(td, sd, sd))
                    # The second RegStage block
                    self.s2 = RegStage(depth=depth, in_chs=out_size, out_chs=out_size)
                    self.proj = build_mlp(mlp_depth, out_size, out_size)
                    
                def forward(self, x):
                    x = self.s1(x)
                    x = self.downsampler(x)
                    x = self.s2(x)
                    x = self.proj(x)
                    return x
        \end{mintedbox}
        
        \noindent \textbf{Design principles.} The authors avoid resampler-style connectors to keep token order consistent for the autoregressive LLM; introduce explicit 3D downsampling to control token count; and use RegStage blocks around the downsampler to compensate for losses from compression~\cite[Sec.~2.1; Fig.~2]{cheng2024_videollama2}.
        
        \paragraph{Training signal and integration}
        All connector parameters (RegStage blocks, 3D conv, MLP) are optimized \emph{only} through the standard next-token LM loss in captioning/QA/instruction formats, with visual/audio encoders frozen and the LLM optionally adapted with parameter-efficient tuning during instruction stages~\cite[Sec.~3]{cheng2024_videollama2}. This keeps the connector small, order-aware, and compute-efficient while enabling strong temporal modeling at inference time. 
        
        \paragraph{Key changes vs.\ V1 (what changed and why)}
        \emph{Video\mbox{-}LLaMA2} replaces the V1 \emph{Video Q\mbox{-}Former} with a \textbf{convolutional STC} to scale to long clips under tight token budgets while keeping the sensory encoders frozen~\cite[Sec.~2]{cheng2024_videollama2}. The shift is motivated by three practical needs:
        \begin{itemize}
            \item \textbf{Chronology by construction.} A single 3D convolution aggregates adjacent frames directly, preserving temporal order without learned resampling that can shuffle/sparsify frames in attention-based connectors~\cite[Sec.~2.1]{cheng2024_videollama2}.
            \item \textbf{Deterministic token control.} Explicit downsampling with stride $(t,s,s)$ (e.g., $(2,2,2)$) reduces tokens early and predictably, enabling longer contexts with stable memory/latency and better accuracy–efficiency trade-offs~\cite[Sec.~2.1; Tab.~1]{cheng2024_videollama2}.
            \item \textbf{Detail preservation around compression.} Lightweight \emph{RegStage} blocks before/after the 3D step act as spatial “sharpen/cleanup” modules, mitigating the blur introduced by temporal downsampling and improving MC\mbox{-}VQA averages at comparable or smaller token budgets~\cite[Sec.~2.1; Tab.~1]{cheng2024_videollama2}.
        \end{itemize}
        
        \textbf{Audio branch update.} The ImageBind\mbox{-}pivoted Audio Q\mbox{-}Former in V1 is replaced with a frozen \textbf{BEATs} encoder plus a small \textbf{MLP} projector, followed by a staged \emph{audio}\,$\rightarrow$\,\emph{audio\,+\,video} curriculum. This simplifies alignment, strengthens A/V synchronization, and improves audio\mbox{-}aware reasoning under limited token and compute budgets~\cite[Sec.~2.2, 3.2]{cheng2024_videollama2}.
        
        \paragraph{Architecture and implementation details}
        \textbf{Vision backbone.} Image-level CLIP ViT-L/14 processes frames independently at \(336{\times}336\), then the STC connector aggregates across time and space, producing a compact set of video tokens for the LLM~\cite[Sec.~2.1]{cheng2024_videollama2}. 
        \textbf{Audio backbone.} BEATs encodes fbank (log-Mel) spectrograms; a 2-layer MLP aligns to the LLM dimension~\cite[Sec.~2.2]{cheng2024_videollama2}. 
        \textbf{LLM.} Mistral-7B-Instruct and Mixtral-Instruct are used as decoders; modality encoders remain frozen; the connector and projector are optimized, and instruction tuning is applied for dialogue~\cite[Sec.~2]{cheng2024_videollama2}. 
        
        \paragraph{Training curriculum}
        \textbf{(i) Vision–language pre-training.} Filtered web-scale image/video–text data are used with frozen encoders and LLM; only the STC connector is optimized via token-level cross-entropy (next-token LM loss)~\cite[Sec.~3.1.1]{cheng2024_videollama2}. The curated recipe keeps \(12.2\)M pairs from \(103\)M candidates (WebVid-10M \(4.0\)M; Panda-70M \(2.8\)M; VIDAL-10M \(2.8\)M; InternVid-10M \(650\)K; CC-3M \(595\)K; DCI \(7.8\)K)~\cite[Tab.~2]{cheng2024_videollama2}. 
        \textbf{(ii) Multi-task fine-tuning.} Simultaneous captioning, classification, VQA, and instruction tuning over \(\approx 1.35\)M samples (Video–Text \(488\)K; Image–Text \(746\)K; Text-only \(120\)K)~\cite[Sec.~3.1.2; Tab.~3]{cheng2024_videollama2}. 
        \textbf{(iii) Audio \& AV curriculum.} Three stages totaling \(\sim 1.9\)M: audio-only pre-train (\(\sim 400\)K), audio instruction (\(\sim 698\)K), and \emph{joint} audio–video (\(\sim 836\)K) for synchronization and AV reasoning~\cite[Sec.~3.2; Tab.~4]{cheng2024_videollama2}. 
        \textbf{Objective.} All stages use standard autoregressive LM loss conditioned on visual/audio tokens and text, with encoders frozen and connector/projector (and LLM adapters during instruction tuning) updated~\cite[Sec.~3]{cheng2024_videollama2}.
        
        \newpage
        
        \subsubsection{Experiments and Ablations}
        \label{subsubsec:chapter24_videollama2_expts}
        
        \paragraph{STC Ablations}
        A controlled sweep (8 frames, Video-LLaVA data) tests \emph{spatial interaction} (RegStage vs.\ none) and \emph{aggregation} (2D/3D pool/conv) under explicit downsampling. The optimal configuration is \textbf{RegStage\,\checkmark\,+\,3D Conv} with \((2,2,2)\) downsampling, yielding \textbf{Avg.\ 45.1} on MV\mbox{-}Bench, EgoSchema, ActivityNet\mbox{-}QA with \textbf{576 tokens} (green row, Table~1). Weaker alternatives include 2D Pool with \((1,2,2)\) at Avg.\ 44.4 and \textbf{1152} tokens (token\mbox{-}hungry), and 3D Conv with \((2,2,2)\) but \emph{no} RegStage at Avg.\ 43.1 and \textbf{576} tokens (detail loss). \textit{Insight:} A single, early 3D fusion step captures motion efficiently, while pre/post RegStage recovers spatial sharpness, giving +2--4\% QA over 2D or plain 3D variants and enabling long\mbox{-}clip scaling without context explosion~\cite[Tab.~1]{cheng2024_videollama2}.
        
        \paragraph{Data Recipe Overview}
        Pre\mbox{-}training filters \textbf{103M} raw pairs to \textbf{12.2M} video/image\mbox{-}text pairs (e.g., WebVid\mbox{-}10M: 4.0M; Panda\mbox{-}70M: 2.8M; see Table~2). Multi\mbox{-}task fine\mbox{-}tuning uses \textbf{1.35M} samples (video\mbox{-}text 488K, image\mbox{-}text 746K, text\mbox{-}only 120K; Table~3). The audio curriculum totals \textbf{1.9M} instances (400K audio pre\mbox{-}train, 698K audio instruction, 836K audio\,+\,video joint; Table~4). \textit{Insight:} Heavy filtering (11.8\% retention) prioritizes quality over raw scale, improving transfer compared with unfiltered mixtures~\cite[Tab.~2--4]{cheng2024_videollama2}.
        
        \paragraph{Multiple\mbox{-}Choice VQA and Perception}
        With 16 frames and a 7B decoder, \emph{Video\mbox{-}LLaMA2} reports \textbf{EgoSchema 51.7\%}, \textbf{Perception\mbox{-}Test 51.4\%}, \textbf{MV\mbox{-}Bench 54.6\%}, \textbf{VideoMME 47.9/50.3\%}, and \textbf{MSVC 2.53/2.59}. Using 8 frames slightly reduces performance (e.g., MV\mbox{-}Bench 53.4\%), while scaling the decoder to Mixtral 8\,$\times$\,7B (\(\approx\)72B) lifts scores to \textbf{63.9/57.5/62.0\%} on EgoSchema/Perception\mbox{-}Test/MV\mbox{-}Bench and \textbf{61.4/63.1\%} on VideoMME (MSVC 2.61/2.61), under the same protocol~\cite[Tab.~5]{cheng2024_videollama2}.
        
        \paragraph{Open\mbox{-}Ended Video QA}
        For MSVD and ActivityNet\mbox{-}QA (accuracy/score), the 7B model attains \textbf{70.9/3.8} and \textbf{50.2/3.3}. On the Video\mbox{-}ChatGPT human rubric, it scores \textbf{3.16/3.08/3.69/2.56/3.14} for Correctness, Detail, Context, Temporal/Consistency~\cite[Tab.~6]{cheng2024_videollama2}. \textit{Insight:} Performance is competitive with image\mbox{-}first baselines on MSVD while showing stronger temporal judgments, consistent with STC’s motion preservation.
        
        \paragraph{Audio QA}
        On audio\mbox{-}only QA, \emph{Video\mbox{-}LLaMA2\mbox{-}7B} reaches \textbf{Clotho\mbox{-}AQA 70.11\%}, \textbf{TUT2017 78.40\%}, and \textbf{VocalSound 93.19\%} using \(\sim\)4k hours of audio, rivaling models trained on orders of magnitude more data (e.g., Qwen\mbox{-}Audio 7B at 57.90/64.90 with \(\sim\)137k hours)~\cite[Tab.~7]{cheng2024_videollama2}. \textit{Insight:} The BEATs\,+\,MLP path is data\mbox{-}efficient for audio grounding.
        
        \paragraph{Open\mbox{-}Ended Audio--Video QA}
        With joint audio--video instruction, the 7B model achieves \textbf{MUSIC\mbox{-}QA 79.2\%}, \textbf{AVSD 57.2\%}, and \textbf{VGGSound 70.9\%} on \(\sim\)1.8M pairs, surpassing prior open\mbox{-}source systems under comparable settings~\cite[Tab.~8]{cheng2024_videollama2}. \textit{Insight:} The staged audio\,\(\rightarrow\)\,AV curriculum tightens cross\mbox{-}modal synchronization for fine\mbox{-}grained reasoning.
        
        \begin{figure}[H]
            \centering
            \includegraphics[width=0.85\linewidth]{Figures/Chapter_24/VideoLLaMA2_qualative.jpg}
            \caption{Qualitative cases from \emph{Video\mbox{-}LLaMA2}: (a) Global scene description and affect. (b) Spatial--temporal orientation awareness. (c) Commonsense reasoning with environmental cues. (d) Fine\mbox{-}grained OCR in video. Adapted from \cite{cheng2024_videollama2}.}
            \label{fig:chapter24_videollama2_qual}
        \end{figure}
        
        \paragraph{Limitations and future directions}
        \label{par:chapter24_videollama2_limits}
        \emph{Video\mbox{-}LLaMA2} acknowledges several open challenges that shape the roadmap for video LLMs~\cite[Sec.~3--5]{cheng2024_videollama2}:
        \begin{itemize}
            \item \textbf{Long\mbox{-}context scaling.} Even with STC downsampling, reasoning beyond tens of seconds is constrained by the LLM’s context window and token budget; maintaining narrative coherence over minutes remains difficult under fixed compute and latency budgets.
            \item \textbf{Fine\mbox{-}grained temporal precision.} Aggressive \((t,s,s)\) reductions can blur boundaries of short, sequential actions (e.g., micro\mbox{-}gestures), suggesting a need for adaptive or multi\mbox{-}rate temporal modeling.
            \item \textbf{Audio--visual synchronization.} Joint training improves sync but still trails specialized AV systems on tightly coupled events (onset/offset, lip\mbox{-}speech alignment), indicating room for stronger cross\mbox{-}modal alignment objectives and curricula.
            \item \textbf{LLM choice and data bias.} The chosen decoders (Mistral/Mixtral) and filtered web corpora can limit domain robustness, multilingual coverage, and calibration under distribution shift; broader, curated instruction data and multilingual AV resources are needed.
        \end{itemize}
        \textit{Where next?} Promising directions include hierarchical long\mbox{-}video memory and tiling, adaptive multi\mbox{-}rate temporal adapters, explicit AV alignment losses/heads, and larger, more diverse instruction datasets with multilingual audio and video. These themes naturally motivate the next model in this series: \textbf{Video\mbox{-}LLaMA3}, covered next, which explores longer contexts, finer temporal localization, and tighter audio--visual coupling while preserving token efficiency.
        
        \begin{table}[H]
            \centering
            \small
            \setlength{\tabcolsep}{6pt}
            \caption{Selected MC\mbox{-}VQA/perception results from the paper at 7B (16 frames). \emph{Video\mbox{-}LLaMA2} and the 2.1 refresh improve over contemporaries under comparable settings. Metrics are accuracies unless noted.}
            \label{tab:videollama2_mcq_compact}
            \resizebox{0.95\linewidth}{!}{%
                \begin{tabular}{lccccc}
                    \toprule
                    \textbf{Model} & \textbf{EgoSchema} & \textbf{Perception\mbox{-}Test} & \textbf{MV\mbox{-}Bench} & \textbf{VideoMME} & \textbf{MSVC} \\
                    \midrule
                    Video\mbox{-}LLaMA2 (7B)~\cite{cheng2024_videollama2} & 51.7 & 51.4 & 54.6 & 47.9/50.3 & 2.53/2.59 \\
                    Video\mbox{-}LLaMA2.1 (7B)~\cite{cheng2024_videollama2} & 53.1 & 54.9 & 57.3 & 54.9/56.4 & 2.87/2.81 \\
                    VideoChat2 (7B)~\cite{li2024_mvbench} & -- & -- & 51.1 & -- & -- \\
                    LLaVA\mbox{-}NeXT\mbox{-}Video (7B)~\cite{liu2024_llava_next_video} & -- & -- & 46.5 & -- & -- \\
                    \bottomrule
            \end{tabular}}
        \end{table}
        
    \end{enrichment}
    
    \newpage
    
    \begin{enrichment}[Video\mbox{-}LLaMA 3: Frontier Multimodal Foundation Models][subsection]
        \label{enr:subsec_chapter24_videollama3}
        
        \subsubsection{Motivation}
        \label{subsubsec:chapter24_videollama3_motivation}
        
        \paragraph{A vision\mbox{-}first redesign}
        \emph{Video\mbox{-}LLaMA2} paired AnyRes tiling with a uniform spatio–temporal connector (STC) to squeeze long clips into an LLM context, but the grid remained rigid: tiling could distort aspect ratios and inflate tokens for simple scenes, while uniform downsampling tended to blur high–frequency detail (e.g., thin chart lines, small OCR text). Near\mbox{-}duplicate frames still consumed substantial budget~\cite{cheng2024_videollama2}. \emph{Video\mbox{-}LLaMA3} reframes the pipeline around \emph{visual fidelity first, efficiency second}: make the vision encoder genuinely resolution–agnostic so images and frames are ingested at native geometry, then treat a video as a sequence of correlated images and \emph{budget} tokens toward changes rather than static redundancy~\cite{zhang2025_videollama3}. In practice, that means any\mbox{-}resolution tokenization for spatial detail, a simple textualized interface (separators and timestamps) for temporality, and content\mbox{-}aware savings that extend the effective horizon without sacrificing detail.
        
        \begin{figure}[H]
            \centering
            \includegraphics[width=0.85\linewidth]{Figures/Chapter_24/VideoLLaMA3_overview.jpg}
            \caption{Pipeline of Video-LLaMA3 with two key techniques: Any-resolution Vision Tokenization (AVT) and Difference-aware Frame Pruning (DiffFP). AVT turns images/videos of any resolution into 1-D token sequences; DiffFP drops low-change regions across adjacent frames for efficient long-video processing. Adapted from \cite{zhang2025_videollama3}.}
            \label{fig:chapter24_videollama3_overview}
        \end{figure}
        
        \paragraph{Design objectives}
        \begin{itemize}
            \item \textbf{Fidelity beyond grid heuristics} Replace tiling/cropping with native\mbox{-}resolution tokenization to eliminate geometric distortion and preserve layout/text details in documents, charts, and high\mbox{-}resolution scenes~\cite[Sec.~3.1]{zhang2025_videollama3}.
            \item \textbf{Scalable token efficiency for longer videos} Plan a clear visual budget within the LLM context and steer tokens toward motion and events via order\mbox{-}preserving sampling and content\mbox{-}aware pruning, so minutes of video remain tractable and temporal reasoning deepens instead of collapsing to coarse summaries~\cite[Sec.~2--3]{zhang2025_videollama3}.
            \item \textbf{One unified, instruction\mbox{-}friendly stream} Represent images and videos in the same textualized format (newline/frame separators and simple \texttt{Time: xxs} stamps), enabling a single autoregressive interface to handle static VQA, multi\mbox{-}image comparison, long\mbox{-}video QA, and streaming dialogue~\cite[Sec.~3.3]{zhang2025_videollama3}.
            \item \textbf{Stable, staged learning} First strengthen and align the vision prior on images, then add instruction following, and finally specialize for video; ablations indicate that adhering to this curriculum improves robustness and long\mbox{-}video performance compared to collapsing stages~\cite[Sec.~4.5]{zhang2025_videollama3}.
        \end{itemize}
        
        \paragraph{Mechanisms chosen to meet these goals}
        \begin{itemize}
            \item \textbf{Any\mbox{-}resolution Vision Tokenization (AVT)} Adapts the ViT to operate at native image/frame size and aspect (resolution\mbox{-}agnostic patch tokens), removing fixed crops or rigid tiling that inflate tokens or break layout~\cite[Sec.~3.1]{zhang2025_videollama3}.
            \item \textbf{Difference\mbox{-}aware Frame Pruning (DiffFP)} Removes temporally redundant visual tokens before the LLM so long videos remain within context while preserving chronology; details follow in the next subsection~\cite[Sec.~2.2]{zhang2025_videollama3}.
            \item \textbf{Staged curriculum} A four\mbox{-}stage progression (vision adaptation $\rightarrow$ vision–language alignment $\rightarrow$ multi\mbox{-}task \emph{SFT} $\rightarrow$ video\mbox{-}centric \emph{SFT}) improves stability and transfer from strong image priors to temporal reasoning~\cite[Sec.~3]{zhang2025_videollama3}. Here, \emph{SFT} means \emph{supervised fine\mbox{-}tuning} on instruction\mbox{-}formatted input–output pairs (e.g., for the multi\mbox{-}task stage: image VQA, captioning, multi\mbox{-}image reasoning; for the video\mbox{-}centric stage: video QA, temporal grounding, streaming dialogue).
        \end{itemize}
        
        \paragraph{Scope: vision focus in V3}
        Unlike \emph{Video\mbox{-}LLaMA2}, which explored audio\mbox{-}conditioned variants, \emph{Video\mbox{-}LLaMA3} is intentionally vision\,+\,language only: the paper and models do not introduce an audio branch~\cite{zhang2025_videollama3}. This concentrates capacity and curated supervision on visual understanding, simplifies token budgeting for long clips, and aligns with benchmarks that evaluate visual comprehension (e.g., VideoMME without subtitles, MLVU, PerceptionTest, DocVQA, MathVista). The textualized interface remains compatible with future audio modules should they be interleaved later.
        
        \paragraph{Anticipated benefits over V2}
        \begin{itemize}
            \item \textbf{Sharper spatial detail} Native\mbox{-}resolution tokenization preserves small text and fine structure that uniform 3D aggregation in V2 tended to blur.
            \item \textbf{Longer effective horizons} Content\mbox{-}aware token savings target near\mbox{-}duplicate frames instead of uniformly compressing everything, enabling deeper temporal reasoning within a fixed LLM window.
        \end{itemize}
        
        \newpage
        
        \subsubsection{Method}
        \label{subsubsec:chapter24_videollama3_method}
        
        \paragraph{Pipeline at a glance}
        \emph{Video\mbox{-}LLaMA3} uses a vision–centric pipeline that keeps native spatial detail while emitting tokens the LLM can read directly~\cite[Sec.~3]{zhang2025_videollama3}:
        \begin{itemize}
            \item \textbf{Vision encoder} A pretrained ViT\mbox{-}style backbone ingests images or video frames at their \emph{native} aspect ratio/size and outputs per\mbox{-}patch features—no forced square crops or rigid tiling.
            \item \textbf{Projector} A small MLP maps encoder features to the LLM embedding dimension, yielding visual tokens that concatenate cleanly with text for unified autoregressive reasoning.
            \item \textbf{Budgeted video packing} Any\mbox{-}resolution Vision Tokenization (AVT) supplies per\mbox{-}frame, resolution\mbox{-}agnostic tokens; frames are serialized in time (optionally with simple \texttt{Time: xxs} tags) and, for long clips, a lightweight temporal compressor (DiffFP, described next) trims redundancy before the LLM.
        \end{itemize}
        This preserves fine structure (e.g., thin OCR strokes, small objects) and global layout while keeping token counts tractable for extended videos~\cite[Sec.~3]{zhang2025_videollama3}.
        
        \paragraph{Why a resolution\mbox{-}agnostic encoder}
        Fixed\mbox{-}crop pipelines and grid\mbox{-}tiling “AnyRes” heuristics can distort aspect ratios, disrupt global layout, and bloat token counts on high\mbox{-}resolution or unusual\mbox{-}aspect inputs. \emph{Video\mbox{-}LLaMA3} replaces these heuristics with a genuinely resolution\mbox{-}agnostic encoder (via AVT) so \emph{every} visual input—single images and all video frames—is processed at native size and aspect~\cite[Sec.~3.1]{zhang2025_videollama3}. The resulting token stream reflects actual visual content rather than an artificial tiling grid, which is crucial for documents, charts, and detail\mbox{-}heavy scenes, and it pairs naturally with later, change\mbox{-}aware pruning to extend the effective temporal horizon.
        
        \paragraph{Any\mbox{-}resolution Vision Tokenization (AVT)}
        AVT makes a ViT\mbox{-}based vision backbone \emph{resolution\mbox{-}agnostic}, dynamically tokenizing images or video frames at their native sizes and aspect ratios to yield variable\mbox{-}length, LLM\mbox{-}ready tokens without cropping, resizing, or rigid tiling~\cite[Sec.~3.1]{zhang2025_videollama3}. The same procedure is applied to single images $x\!\in\!\mathbb{R}^{C\times H\times W}$ and to each frame of a video $x\!\in\!\mathbb{R}^{T\times C\times H\times W}$, with temporal serialization handled \emph{after} spatial encoding.
        
        \begin{itemize}
            \item \textbf{Native\mbox{-}resolution spatial patching (before tokenization).} For each image or frame of size $H{\times}W$, extract non\mbox{-}overlapping $P{\times}P$ patches (stride $P$). This yields a grid $H'=\lceil H/P\rceil$, $W'=\lceil W/P\rceil$ and a sequence length
            \[
            K \;=\; H' W' \;=\; \Big\lceil\frac{H}{P}\Big\rceil \!\cdot\! \Big\lceil\frac{W}{P}\Big\rceil\,.
            \]
            Each patch of shape $C\!\times\!P\!\times\!P$ is flattened and linearly projected to a $d$-dimensional vector, producing a token sequence $\mathbf{z}\in\mathbb{R}^{K\times d}$ that mirrors the native grid exactly. This “patchify\,$\rightarrow$\,embed” step fixes the geometry for positional encoding and avoids any rescaling or post\mbox{-}hoc interpolation artifacts.
            
            \item \textbf{Backbone adaptation to 2D\mbox{-}RoPE (spatial geometry).} Replace the ViT’s fixed \emph{absolute} positional table with \emph{2D Rotary Position Embeddings (2D\mbox{-}RoPE)} applied to queries/keys in every self\mbox{-}attention layer. Rotary encodings inject \emph{relative} horizontal/vertical geometry via phase rotations, so the same encoder seamlessly handles arbitrary $H'\!\times\!W'$ grids and aspect ratios without resizing or tiling~\cite[Sec.~3.1]{zhang2025_videollama3,heo2024_rotarype}.
            
            \newpage
            
            \item \textbf{Packing for the LLM (budgeted temporal serialization)}%
            \begin{itemize}
                \item \textbf{Project to language space.} A lightweight two\mbox{-}layer MLP with GELU maps each per\mbox{-}frame feature matrix $\mathbf{f}_t \!\in\! \mathbb{R}^{K_t \times d}$ to $\mathbf{v}_t \!\in\! \mathbb{R}^{K_t \times d_{\text{LLM}}}$.
                \item \textbf{Images.} For a single image, the visual tokens $\mathbf{v} \!\in\! \mathbb{R}^{K \times d_{\text{LLM}}}$ are newline\mbox{-}separated and concatenated with text tokens, then fed to the LLM.
                \item \textbf{Videos (chronological stream).} For a clip with $T$ frames, serialize $\{\mathbf{v}_t\}_{t=1}^{T}$ in time to form one sequence $\mathbf{V} \!\in\! \mathbb{R}^{(\sum_t K_t) \times d_{\text{LLM}}}$. Optionally prefix each frame block with a timestamp token (e.g., \texttt{Time: xxs}) and separate frames by commas to make temporal indices explicit.
                \item \textbf{Context budgeting.} Work within the LLM context window (e.g., total $16{,}384$ tokens) by allocating a visual budget (e.g., $\leq 10{,}240$ tokens) and reserving the remainder for text.
                \item \textbf{Order\mbox{-}preserving enforcement.} If the visual stream exceeds the budget, apply simple, model\mbox{-}agnostic rules that keep chronology intact:
                \begin{itemize}
                    \item \emph{Uniform frame sampling:} increase the frame stride (e.g., decode at $1$\,fps and subsample to a target number of frames for short clips).
                    \item \emph{Fixed $2{\times}2$ spatial downsampling (post\mbox{-}encoder):} apply $2{\times}2$ pooling over the token grid to reduce each $K_t$ while preserving aspect.
                \end{itemize}
                \item \textbf{Goal.} Produce a well\mbox{-}ordered, budget\mbox{-}compliant visual sequence in which AVT preserves per\mbox{-}frame spatial fidelity; a content\mbox{-}aware pruning method for long videos is introduced next for additional savings~\cite[Sec.~3.1--3.2]{zhang2025_videollama3}.
            \end{itemize}
            
        \end{itemize}
        
        \paragraph{How 2D\mbox{-}RoPE encodes spatial relations}
        2D\mbox{-}RoPE lifts rotary embeddings from 1D to 2D image grids so self\mbox{-}attention depends on \emph{relative} offsets $(\Delta u,\Delta v)$ instead of absolute indices~\cite{heo2024_rotarype,zhang2025_videollama3}. In \emph{Video\mbox{-}LLaMA3} this rotation acts \emph{per frame} on spatial tokens; temporal order is handled later by serializing frames (optionally with \texttt{Time: xxs}) before they enter the LLM.
        
        \textbf{Attention recap.} \;
        Take two patches at grid coordinates $(u,v)$ and $(u',v')$. Let $x(u,v)$ and $x(u',v')$ be their content embeddings. In one attention head,
        \[
        q(u,v)=W_q\,x(u,v),\qquad k(u',v')=W_k\,x(u',v') .
        \]
        With \emph{2D-RoPE}, we \emph{rotate} each query/key by an angle set by its own coordinates:
        \[
        \tilde q(u,v)=R_{\phi(u,v)}\,q(u,v),\qquad
        \tilde k(u',v')=R_{\phi(u',v')}\,k(u',v') ,
        \]
        where $R_{\phi}$ is a tiny $2{\times}2$ rotation applied per channel\mbox{-}pair (head dimension is even), and the angle is
        \[
        \phi(u,v)=\theta_x\,u+\theta_y\,v
        \quad\text{(with a small bank of frequencies $\theta_x,\theta_y$ across pairs).}
        \]
        The attention score is the dot product
        \[
        \big\langle \tilde q(u,v),\,\tilde k(u',v') \big\rangle,
        \]
        and the rotations make it depend only on the \emph{offsets}
        \[
        \Delta u = u-u',\qquad \Delta v = v-v' .
        \]
        Intuition: each token is “twisted’’ by an angle tied to its $(u,v)$. When two tokens interact, only the \emph{difference} of those angles matters, so “one patch to the right’’ (e.g., $\Delta u=-1,\Delta v=0$) produces the same effect on any grid size. This is why 2D\mbox{-}RoPE is naturally resolution\mbox{-} and aspect\mbox{-}agnostic.
        
        \textbf{Why a new PE for arbitrary resolutions.} \;
        Learned tables and absolute sinusoidals are tied to a specific lattice; when $H'\!\times W'$ changes they need interpolation or reindexing and often drift off\mbox{-}distribution. RoPE encodes position as multiplicative \emph{rotations} that compose inside the dot product, so the logit becomes a function of $(\Delta u,\Delta v)$ only. The notion “one patch to the right’’ is identical on $14{\times}28$ and $28{\times}56$ grids—no tables, no interpolation.
        
        \textbf{Setup and channel\mbox{-}pair rotations (encoding a patch at $(u,v)$).} \;
        After native\mbox{-}resolution patching, a frame yields a token grid with integer indices $(u,v)$, where $u\!\in\!\{0,\dots,H'-1\}$ and $v\!\in\!\{0,\dots,W'-1\}$. For the token at $(u,v)$, split $q,k\!\in\!\mathbb{R}^{d}$ (even $d$) into $d/2$ channel pairs $(x_{2i},x_{2i+1})$, each a tiny 2D plane we can rotate by
        \[
        R_{\phi_i} \;=\;
        \begin{pmatrix}
            \cos\phi_i & -\sin\phi_i\\
            \sin\phi_i & \phantom{-}\cos\phi_i
        \end{pmatrix},
        \qquad
        \phi_i \;=\; \theta_x^{(i)}\,u \;+\; \theta_y^{(i)}\,v,
        \]
        where $\{(\theta_x^{(i)},\theta_y^{(i)})\}_{i=1}^{d/2}$ is a frequency bank (typically geometric from coarse$\!\to\!$fine scales, fixed or lightly learned per head). Apply $R_{\phi_i}$ to every pair of $q$ and $k$ to obtain the rotated vectors $\tilde q,\tilde k$.
        
        \begin{itemize}
            \item \textbf{Separable axial.} Dedicate some pairs to rows ($\phi_i=\theta_x^{(i)}u$) and others to columns ($\phi_i=\theta_y^{(i)}v$).
            \item \textbf{Mixed (diagonal\mbox{-}aware).} Use $\phi_i=\theta_x^{(i)}u+\theta_y^{(i)}v$ on all pairs to capture diagonals directly.
        \end{itemize}
        \emph{Which to use.} Both realize the same relative property; mixed is often favored in vision because many structures (strokes, edges) are not axis\mbox{-}aligned.
        
        \textbf{Why pairs and where the frequencies come from.} \;
        Treat $(x_{2i},x_{2i+1})$ as a complex coordinate $x_{2i}+\mathrm{i}x_{2i+1}$. Multiplying by $e^{\mathrm{i}\phi_i}$ (the rotation) preserves magnitude (content) while writing location into the angle. A geometric bank of $\theta$’s spreads sensitivity across spatial scales: low frequencies capture coarse layout, high frequencies capture fine detail (e.g., text strokes). Multi\mbox{-}head attention distributes these “frequency bins’’ across heads, so capacity is preserved.
        
        \textbf{Relativity in the logit (why resolution\mbox{-}agnostic).} \;
        For tokens at $(u,v)$ and $(u',v')$ (write $\Delta u{=}u{-}u'$, $\Delta v{=}v{-}v'$),
        \[
        \big\langle \tilde q(u,v),\,\tilde k(u',v') \big\rangle
        \;=\;
        \sum_{i=1}^{d/2} \mathrm{Re}\!\Big[\,\alpha_i \, e^{\,\mathrm{i}\,(\theta_x^{(i)}\Delta u + \theta_y^{(i)}\Delta v)} \Big],
        \]
        with coefficients $\alpha_i$ determined by the unrotated content. The score is therefore a multi\mbox{-}scale, Fourier\mbox{-}like function of \emph{offsets} $(\Delta u,\Delta v)$—not of absolute $(u,v)$. The same $(\Delta u,\Delta v)$ produces the same phase gap across grids, enabling clean extrapolation to new resolutions and aspect ratios.
        
        \textbf{Efficient implementation.} \;
        Because $R_{\Theta}$ is block\mbox{-}diagonal, rotation reduces to pairwise elementwise ops:
        \[
        \tilde q_{2i} = q_{2i}\cos\phi_i - q_{2i+1}\sin\phi_i,\qquad
        \tilde q_{2i+1} = q_{2i}\sin\phi_i + q_{2i+1}\cos\phi_i
        \]
        (and analogously for $\tilde k$). No dense matrix multiply is required.
        
        \textbf{Concrete intuition and example.} \;
        Think of each channel pair as a compass needle. A patch at $(u,v)$ turns each needle by $\phi_i=\theta_x^{(i)}u+\theta_y^{(i)}v$. Nearby patches turn needles by nearly the same angles (high overlap); distant patches turn them very differently (lower overlap). 
        
        \newpage
        
        For $x\!\in\!\mathbb{R}^{3\times224\times448}$ with $P{=}16$ ($H'\!=\!14$, $W'\!=\!28$), the token at $(6,10)$ rotates by $\phi_i=\theta_x^{(i)}\!\cdot\!6+\theta_y^{(i)}\!\cdot\!10$. Attending to $(7,10)$ introduces a horizontal phase gap proportional to $\Delta u{=}{-}1$. Upscale to $448{\times}896$ ($28{\times}56$) and the same neighbor relation yields the \emph{same} phase gap—this is the core of resolution\mbox{-}agnostic behavior.
        
        \textbf{Why 2D\mbox{-}RoPE over absolute/sinusoidal PE (at a glance).}
        \begin{itemize}
            \item \textbf{Relative by construction.} Offsets $(\Delta u,\Delta v)$ drive the logit, so no PE interpolation or reindexing is needed when $H'\!\times W'$ changes.
            \item \textbf{Table\mbox{-}free scaling.} Angles are computed on the fly from integer coordinates; there are no size\mbox{-}specific lookup tables to retrain or resize.
            \item \textbf{Multi\mbox{-}scale sensitivity.} A frequency bank makes attention responsive to both coarse layout and fine detail while preserving a global receptive field.
        \end{itemize}
        
        \textbf{Why time stays outside the vision PE.} \;
        \emph{Video\mbox{-}LLaMA3} applies 2D\mbox{-}RoPE \emph{within each frame} and leaves \emph{temporal} order to the LLM by serializing frame tokens in time (optionally with simple \texttt{Time: xxs} tags)~\cite[Sec.~3.1--3.2]{zhang2025_videollama3}. This reuses strong image priors, keeps the vision stack lightweight (no 3D attention/PE), adapts naturally to variable frame counts under a token budget, and exploits the LLM’s strength on long sequences for temporal reasoning.
        
        \textbf{End\mbox{-}to\mbox{-}end position handling.} \;
        \begin{itemize}
            \item \emph{Within a frame.} Tokens lie on an integer grid $(u,v)$; 2D\mbox{-}RoPE rotates $q/k$ using these coordinates, independent of flattening order.
            \item \emph{Across frames.} Tokens are concatenated chronologically; timestamps provide explicit temporal indices for the LLM.
        \end{itemize}
        
        \paragraph{Differential Frame Pruner (DiffFP)}
        Stacking per\mbox{-}frame tokens linearly with time produces long, redundant sequences dominated by static background regions. \emph{Video\mbox{-}LLaMA3} therefore introduces \textbf{DiffFP}, a simple, content\mbox{-}adaptive compressor that prunes patches with negligible temporal change while preserving key frames and motion regions \cite[Sec.~3.2]{zhang2025_videollama3}. The procedure is two\mbox{-}stage:
        
        \textit{(A) Uniform spatial downsampling (coarse bound).} Each frame is first uniformly downsampled (e.g., $2{\times}2$ bilinear) before patching/tokenization to place a coarse upper bound on per\mbox{-}frame tokens without destroying global context.
        
        \textit{(B) Difference\mbox{-}aware patch pruning (fine, adaptive).} Let a downsampled frame at time $t$ be partitioned into $H_p{\times}W_p$ patches, and let $x_t(i,j)\in\mathbb{R}^{P{\times}P{\times}C}$ denote the pixel block (or an equivalent local descriptor used by the pruner) at patch $(i,j)$. DiffFP computes per\mbox{-}patch $\ell_1$ differences to the previous frame and a frame\mbox{-}level change statistic:
        \[
        d_t(i,j)\;=\;\bigl\|x_t(i,j)-x_{t-1}(i,j)\bigr\|_1,\qquad 
        \Delta_t\;=\;\frac{1}{H_pW_p}\sum_{i=1}^{H_p}\sum_{j=1}^{W_p} d_t(i,j).
        \]
        With thresholds $\tau_{\text{patch}}$ and $\tau_{\text{frame}}$:
        \begin{itemize}
            \item \textbf{Key\mbox{-}frame keep.} If $\Delta_t\ge\tau_{\text{frame}}$ (large global change), \emph{keep all patches} of frame $t$ to robustly capture scene cuts and large motions. 
            \item \textbf{Patch\mbox{-}wise keep.} Otherwise, \emph{keep only} patches with $d_t(i,j)\ge\tau_{\text{patch}}$ and prune the rest, yielding a sparse set of motion patches for frame $t$. 
        \end{itemize}
        The resulting visual stream contains full key frames interleaved with sparse motion patches from intermediate frames, markedly shrinking the token budget while preserving the chronology and local dynamics needed for temporal reasoning. The pruned tokens are then concatenated (with text and, when present, additional modalities) and fed to the LLM for autoregressive generation.
        
        \begin{figure}[H]
            \centering
            \includegraphics[width=0.85\linewidth]{Figures/Chapter_24/VideoLLaMA3_diffFP_calc.jpg}
            \caption{Difference\mbox{-}aware Frame Pruning (DiffFP). Patches with small $\ell_1$ differences to the previous frame are pruned; high\mbox{-}difference regions and frames with large global change are kept, yielding a compact stream of key frames plus motion patches. Adapted from \cite{zhang2025_videollama3}.}
            \label{fig:chapter24_videollama3_difffp}
        \end{figure}
        
        \paragraph{Data representations for multi\mbox{-}image, video, and streaming}
        To unify static and temporal inputs in a single LLM interface, \emph{Video\mbox{-}LLaMA3} \emph{textualizes} visual tokens with lightweight, literal delimiters that make structure explicit to the decoder \cite[Sec.~3.3; Fig.~6]{zhang2025_videollama3}:
        \begin{itemize}
            \item \textbf{Multi\mbox{-}image sequences.} Visual token blocks for successive images are separated by the newline literal \texttt{\textbackslash n}, and a final newline separates the vision block from the text prompt. This preserves per\mbox{-}image boundaries while enabling cross\mbox{-}image reasoning.
            \item \textbf{Video sequences.} Each frame’s tokens are prefixed by a timestamp literal \texttt{Time: xxs} and frames are comma\mbox{-}separated, e.g., \texttt{Time: 0.0s [tokens], Time: 0.5s [tokens], \dots}. A trailing \texttt{\textbackslash n} then separates the visual stream from the text prompt. Timestamps provide explicit temporal anchors for ordering and duration.
            \item \textbf{Streaming sequences.} For long or live inputs, timestamped video token blocks and text turns are interleaved in one sequence (e.g., \texttt{Time: 2.0s [tokens]} \texttt{USER:}\,{\dots}\,\texttt{ASSISTANT:}\,{\dots}), enabling in\mbox{-}stream answers and multi\mbox{-}turn references to prior moments.
        \end{itemize}
        This delimiter\mbox{-}based serialization lets the LLM “read’’ images, videos, and streams as structured narratives, while AVT supplies faithful, resolution\mbox{-}agnostic tokens and DiffFP emphasizes informative changes over near\mbox{-}duplicate frames \cite[Fig.~2, Fig.~6]{zhang2025_videollama3}.
        
        \begin{figure}[H]
            \centering
            \includegraphics[width=0.85\linewidth]{Figures/Chapter_24/VideoLLaMA3_data_types.jpg}
            \caption{Data formats for different input types. (1) Image sequences use ``\texttt{\textbackslash n}'' to separate tokens from different images. (2) Video sequences prefix each frame with ``\texttt{Time: xxs}'', use commas to separate frames, and ``\texttt{\textbackslash n}'' to separate different videos. (3) Streaming sequences interleave timestamped video tokens with text turns. Adapted from \cite[Fig.~6]{zhang2025_videollama3}.}
            \label{fig:chapter24_videollama3_data_types}
        \end{figure}
        
        \subsubsection{Architecture \& Implementation Details}
        \label{subsubsec:chapter24_videollama3_arch}
        
        \paragraph{Backbone and projector}
        \emph{Video\mbox{-}LLaMA3} couples a SigLIP\mbox{-}initialized ViT encoder with a lightweight two\mbox{-}layer MLP projector (GELU), a difference\mbox{-}aware video compressor (DiffFP), and a Qwen\,2.5 family LLM for reasoning and generation~\cite{zhai2023_siglip,zhang2025_videollama3,qwen2025_qwen25technicalreport}. Any\mbox{-}resolution Vision Tokenization (AVT) is realized by replacing the encoder’s learned absolute positional embeddings with \textbf{2D\mbox{-}RoPE} and \emph{fine\mbox{-}tuning} the vision stack on diverse images (scenes, documents, text\mbox{-}rich content). Compared to freezing the ViT (common in earlier pipelines), this adaptation is crucial: absolute PEs are tied to a fixed grid, whereas AVT demands resolution\mbox{-}agnostic geometry; fine\mbox{-}tuning lets attention heads and patch embeddings recalibrate to rotations, stabilizes scale/aspect extrapolation, and improves small\mbox{-}detail fidelity (OCR strokes, thin chart lines). Qwen\,2.5 (e.g., 2B/7B) provides strong instruction following and long\mbox{-}context handling, while the shared vision stack keeps scaling cost moderate.
        
        \paragraph{Training paradigm}
        A \textbf{four\mbox{-}stage} curriculum builds a strong image prior first, then aligns and specializes for video~\cite{zhang2025_videollama3}:
        \begin{itemize}
            \item \textbf{Stage 1: Vision encoder adaptation.} Swap absolute PEs for 2D\mbox{-}RoPE and fine\mbox{-}tune the SigLIP ViT and the projector on diverse images while keeping the LLM frozen. \emph{Intuition.} Teach the encoder resolution\mbox{-}agnostic geometry (AVT) without language interference; freezing here would leave an absolute\mbox{-}PE mismatch that harms layout fidelity at new sizes/aspects.
            \item \textbf{Stage 2: Vision--language alignment.} Unfreeze encoder, projector, and LLM; jointly train on rich image--text data (including charts/regions) and mix in text\mbox{-}only samples. \emph{Intuition.} Co\mbox{-}adapt vision features and the LLM so the language space learns to “read’’ variable\mbox{-}length, AVT tokens; text\mbox{-}only keeps linguistic fluency intact.
            \item \textbf{Stage 3: Multi\mbox{-}task supervised fine\mbox{-}tuning (SFT).} Instruction SFT over broad image tasks plus introductory video captioning; activate DiffFP to begin controlling video token counts. \emph{Intuition.} Broaden skills and seed temporal competence while enforcing a practical token budget.
            \item \textbf{Stage 4: Video\mbox{-}centric SFT.} Focus on video QA, streaming, and temporal grounding with DiffFP active; continue mixing image\mbox{-}only and text\mbox{-}only data. \emph{Intuition.} Specialize motion/event reasoning on top of the strong image prior, while guarding against catastrophic forgetting.
        \end{itemize}
        
        \paragraph{Where AVT and DiffFP plug in}
        \textbf{AVT.} Enabled in Stage~1 by replacing absolute PEs with 2D\mbox{-}RoPE and fine\mbox{-}tuning the ViT\,+\,projector on images; thereafter, every image or frame is patchified at native aspect/size and encoded into resolution\mbox{-}agnostic tokens (no cropping/tiling). \emph{Why here.} Early adaptation lets all later stages benefit from clean geometry and faithful layout.
        
        \textbf{DiffFP.} Activated once video enters (Stages~3--4). Frames undergo fixed $2{\times}2$ spatial downsampling \emph{post\mbox{-}encoder} to bound per\mbox{-}frame token counts, then temporally redundant patches are pruned based on pixel\mbox{-}space $\ell_1$ differences w.r.t.\ the previous frame (threshold $\tau{\approx}0.1$ by default), preserving motion\mbox{-}bearing regions while cutting near\mbox{-}duplicates~\cite[Sec.~2.2]{zhang2025_videollama3}. \emph{Why here.} Token budgeting is a delivery problem for the LLM context; doing it after spatial encoding keeps per\mbox{-}frame detail sharp and removes redundancy only when needed.
        
        \newpage
        
        \paragraph{Summary of design choices}
        \begin{itemize}
            \item \textbf{SigLIP for strong visual priors.} Sigmoid\mbox{-}based contrastive pretraining transfers well to text\mbox{-}heavy and diagrammatic images; fine\mbox{-}tuning with 2D\mbox{-}RoPE teaches resolution\mbox{-}agnostic geometry, improving small\mbox{-}detail fidelity over frozen backbones~\cite{zhai2023_siglip,zhang2025_videollama3}.
            \item \textbf{Qwen\,2.5 for instruction reasoning.} A modern LLM with long\mbox{-}context and multilingual strengths; a small projector maps vision features into the LLM space for stable alignment and scalable capacity~\cite{qwen2025_qwen25technicalreport}.
            \item \textbf{Video efficiency through DiffFP.} Combine mild spatial downsampling with difference\mbox{-}aware patch pruning to fit long clips within a fixed context while emphasizing changes rather than static backgrounds~\cite{zhang2025_videollama3}.
            \item \textbf{Stagewise curriculum for stability.} Image\,$\rightarrow$\,multimodal\,$\rightarrow$\,video progressively aligns components, reduces optimization shock, preserves image/document skills, and yields better long\mbox{-}video transfer than collapsing stages~\cite{zhang2025_videollama3}.
        \end{itemize}
        
        \begin{figure}[H]
            \centering
            \includegraphics[width=0.7\linewidth]{Figures/Chapter_24/VideoLLaMA3_training_paradigm.jpg}
            \caption{Four-stage training paradigm: Vision Encoder Adaptation, Vision--Language Alignment, Multi-task Fine-tuning, and Video-centric Fine-tuning. Adapted from \cite{zhang2025_videollama3}.}
            \label{fig:chapter24_videollama3_training}
        \end{figure}
        
        \subsubsection{Experiments and Ablations}
        \label{subsubsec:chapter24_videollama3_experiments}
        
        \paragraph{Benchmarks and headline performance}
        \emph{Video\mbox{-}LLaMA3} is evaluated as a unified \emph{image+video} MLLM and reports strong results across both video and image/math/doc tasks. For the \emph{7B} variant, representative accuracies include: \textbf{MLVU (dev) 73.0\%}, \textbf{VideoMME (w/o subtitles) 66.2\%}, \textbf{PerceptionTest 72.8\%}, and \textbf{MathVista (testmini) 67.1\%}. These trends align with the design goal: AVT preserves high\mbox{-}frequency spatial detail for documents/diagrams, while DiffFP focuses the budget on temporal changes for long clips.
        
        \begin{figure}[H]
            \centering
            \includegraphics[width=0.7\linewidth]{Figures/Chapter_24/VideoLLaMA3_comparisons.jpg}
            \caption{Representative comparison across image and video benchmarks. Image\mbox{-}centric baselines (e.g., LLaVA\mbox{-}OneVision) are reported on image tasks; video\mbox{-}centric baselines (e.g., LLaVA\mbox{-}Video) on video tasks. Adapted from \cite{zhang2025_videollama3}.}
            \label{fig:chapter24_videollama3_comparisons}
        \end{figure}
        
        \begin{table}[H]
            \centering
            \small
            \setlength{\tabcolsep}{6pt}
            \renewcommand{\arraystretch}{1.1}
            \caption{Selected headline results for \emph{Video\mbox{-}LLaMA3} (7B). Accuracies (\%).}
            \label{tab:videollama3_headline}
            \begin{tabular}{lcccc}
                \toprule
                \textbf{Model} & \textbf{MLVU (dev)} & \textbf{VideoMME (w/o sub)} & \textbf{PerceptionTest} & \textbf{MathVista (testmini)} \\
                \midrule
                Video\mbox{-}LLaMA3 (7B) & 73.0 & 66.2 & 72.8 & 67.1 \\
                \bottomrule
            \end{tabular}
        \end{table}
        
        \paragraph{Effect of AVT and DiffFP}
        Ablations separate the roles of \emph{Any\mbox{-}resolution Vision Tokenization} (AVT) and \emph{Difference\mbox{-}aware Frame Pruning} (DiffFP)~\cite[Sec.~3.1; Sec.~2.2; Sec.~4]{zhang2025_videollama3}.
        \begin{itemize}
            \item \textbf{AVT (2D\mbox{-}RoPE adaptation).} Swapping absolute PEs for 2D\mbox{-}RoPE and ingesting images/frames at native aspect/size reduces geometric distortion and preserves small text/lines. The paper substantiates AVT with qualitative comparisons and aggregate benchmark gains on layout\mbox{-}sensitive tasks (documents, charts, diagrams) after the AVT stage, rather than a standalone numeric table isolating AVT alone~\cite[Fig.~2; Sec.~3.1; Sec.~4]{zhang2025_videollama3}. \emph{Intuition.} AVT’s relative, table\mbox{-}free encoding makes “one\mbox{-}patch right’’ identical across grids, enabling clean transfer to unseen resolutions/aspects.
            \item \textbf{DiffFP (video token efficiency).} After mild per\mbox{-}frame $2{\times}2$ downsampling to cap tokens, DiffFP prunes patches whose $\ell_1$ pixel differences to the previous frame fall below a fixed threshold (default $\tau{=}0.1$)~\cite[Sec.~2.2]{zhang2025_videollama3}. The paper shows accuracy–token trade\mbox{-}off curves where substantial token reductions are achieved with negligible accuracy drops on long\mbox{-}video benchmarks under fixed context budgets (see \cite[Fig.~4; Sec.~4]{zhang2025_videollama3}). \emph{Intuition.} DiffFP targets static regions while retaining motion cues, reallocating budget to informative changes.
        \end{itemize}
        
        \paragraph{Comparisons to related systems}
        Relative to \emph{Video\mbox{-}LLaMA2} (uniform 3D aggregation), \emph{Video\mbox{-}LLaMA3} keeps spatial detail sharper and scales to longer videos via content\mbox{-}aware sparsification~\cite[Sec.~2; Fig.~1]{zhang2025_videollama3,cheng2024_videollama2}. Against similarly sized \emph{Qwen2\mbox{-}VL}, it trends stronger on long\mbox{-}video and math/diagram reasoning in the authors’ composite chart, while approaching larger closed or semi\mbox{-}closed systems on several video tasks~\cite[Fig.~1]{zhang2025_videollama3}. Compared to image\mbox{-}first baselines (e.g., LLaVA\mbox{-}OneVision), \emph{Video\mbox{-}LLaMA3} maintains competitive document/multi\mbox{-}image reasoning and adds robust temporal understanding via its unified serialization interface~\cite[Sec.~3.3; Sec.~4]{zhang2025_videollama3}.
        
        \paragraph{Vision backbone ablation.}
        Encoder studies support choosing a SigLIP\mbox{-}initialized ViT: after AVT adaptation and alignment, it yields strong text\mbox{-}aware features for OCR, charts, and fine\mbox{-}grained perception, outperforming alternatives on the authors’ image benchmarks~\cite[Sec.~4.4]{zhai2023_siglip,zhang2025_videollama3}. \emph{Intuition.} SigLIP’s contrastive pretraining plus 2D\mbox{-}RoPE adaptation gives a better prior for small, high\mbox{-}frequency structures than freezing an absolute\mbox{-}PE encoder.
        
        \newpage
        
        \paragraph{Data curation and mixtures}
        A staged, quality\mbox{-}over\mbox{-}quantity recipe builds image priors first, then specializes for video~\cite[Sec.~3.2; Tables~1--4]{zhang2025_videollama3}.
        \begin{itemize}
            \item \textbf{Vision Encoder Adaptation.} \textbf{15.57M} images spanning scenes, scene text/OCR, and documents to realize AVT and resolution\mbox{-}agnostic encoding.
            \item \textbf{Vision\mbox{--}Language Alignment.} \textbf{21.97M} image\mbox{--}text pairs (incl.\ charts and fine\mbox{-}grained regions) plus text\mbox{-}only samples to retain language fluency.
            \item \textbf{Multi\mbox{-}task Fine\mbox{-}tuning.} \textbf{19.05M} instruction\mbox{-}formatted image tasks plus general video captioning to seed temporal competence; DiffFP introduced for token control.
            \item \textbf{Video\mbox{-}centric Fine\mbox{-}tuning.} \textbf{5.71M} video samples focused on video QA, streaming, and temporal grounding with DiffFP active.
        \end{itemize}
        \emph{Intuition.} Image\mbox{-}first stages establish a strong, resolution\mbox{-}agnostic visual prior; later stages add instruction following and temporal specialization while DiffFP balances accuracy and context for long clips~\cite[Sec.~3.2; Sec.~4.5]{zhang2025_videollama3}.
        
        \subsubsection{Limitations and Future Work}
        \label{subsubsec:chapter24_videollama3_limits}
        \paragraph{Long-context and token budgets.}
        Although DiffFP reduces redundancy, extremely long videos still stress context limits; further hierarchical memory or event-level summarization could help.
        
        \paragraph{Temporal precision and rare events}
        Patch-level pruning with a fixed threshold may miss subtle, short-lived cues; adaptive thresholds or learned importance could improve recall on fine actions.
        
        \paragraph{Data biases and domain transfer}
        Vision-centric emphasis leverages curated image corpora; robustness to domain shifts (e.g., niche video domains or low-light/noisy streams) may require targeted data or adapters.
        
        \paragraph{Toward \emph{Video-LLaMA4}}
        Given Chapter~\ref{enr:subsec_chapter24_videollama2} highlighted STC for efficient motion aggregation, Video-LLaMA3 generalizes the idea with AVT+DiffFP for any-resolution and long-form efficiency. The next chapter on \emph{Qwen-VL} families will revisit similar themes (high-res tokenization, streaming), and the subsequent \emph{Qwen3-VL} hints at tighter multi-granular fusion and memory scheduling—directions also natural for the Video-LLaMA line.
        
    \end{enrichment}
    
    \newpage
    
    \begin{enrichment}[Qwen-VL: Versatile Vision--Language Foundation][subsection]
        \label{enr:subsec_chapter24_qwenvl}
        
        \subsubsection{Motivation}
        \label{subsubsec:chapter24_qwenvl_motivation}
        Large multimodal systems frequently underperform on \emph{fine\mbox{-}grained} visual skills (e.g., text reading, region grounding) and often lag behind proprietary models due to limited training scale and suboptimal optimization. The Qwen\mbox{-}VL paper targets these gaps by: (i) adding a \emph{position\mbox{-}aware visual receptor} that compresses high\mbox{-}resolution visual features into a compact, LLM\mbox{-}friendly sequence; (ii) defining a concise \emph{input–output interface} to unify images, text, and bounding\mbox{-}box strings; and (iii) designing a \emph{three\mbox{-}stage} curriculum (pretraining, multi\mbox{-}task pretraining, supervised finetuning) over a multilingual, cleaned corpus~\cite{bai2023_qwenvl}.
        
        \begin{figure}[H]
            \centering
            \includegraphics[width=0.85\linewidth]{Figures/Chapter_24/QwenVL_Comparisons.jpg}
            \caption{At publication time, Qwen\mbox{-}VL achieved state\mbox{-}of\mbox{-}the\mbox{-}art results among generalist models across diverse benchmarks (schematic radar chart). Adapted from \cite{bai2023_qwenvl}.}
            \label{fig:chapter24_qwenvl_radar}
        \end{figure}
        
        \paragraph{Reading the radar chart (intuition)}
        Each spoke represents a benchmark (e.g., VQAv2, OK\mbox{-}VQA, TextVQA, OCR\mbox{-}VQA, ChartQA, RefCOCO). Larger area indicates stronger all\mbox{-}round performance. Qwen\mbox{-}VL’s polygon is notably expansive, reflecting broad generalization and especially strong text\mbox{-}rich understanding (TextVQA/OCR\mbox{-}VQA/ChartQA) relative to contemporary generalist baselines~\cite{bai2023_qwenvl}.
        
        \subsubsection{Method}
        \label{subsubsec:chapter24_qwenvl_method}
        
        \paragraph{Architecture (visual receptor + LLM)}
        Qwen\mbox{-}VL integrates a high\mbox{-}capacity vision encoder, a lightweight position\mbox{-}aware adapter, and a large language model to enable unified multimodal reasoning~\cite{bai2023_qwenvl}.
        \begin{itemize}
            \item \textbf{Vision encoder.} A pretrained OpenCLIP ViT\mbox{-}bigG (patch stride $P{=}14$) extracts a sequence of patch features at the stage\mbox{-}specific input resolution, serving as robust perceptual tokens for downstream fusion~\cite{bai2023_qwenvl}.
            \item \textbf{Position\mbox{-}aware VL adapter.} A single cross\mbox{-}attention layer with $M{=}256$ learnable query vectors compresses the variable\mbox{-}length image feature sequence into a fixed\mbox{-}length token set while injecting 2D absolute positional encodings into the attention to preserve spatial layout~\cite{bai2023_qwenvl}. \emph{Relation to BLIP\mbox{-}2 Q\mbox{-}Former:} both use learnable queries to distill visual features, but Qwen\mbox{-}VL adopts a \emph{single} cross\mbox{-}attention layer (no stacked self/cross transformer blocks), prioritizing efficiency while retaining spatial fidelity via explicit 2D position signals.
            \item \textbf{Large language model.} A Qwen\mbox{-}7B LLM consumes the $M$ adapter tokens interleaved with text to generate outputs, yielding a 9.6B\mbox{-}parameter system in total (Vision 1.9B, Adapter 0.08B, LLM 7.7B)~\cite[Table~1]{bai2023_qwenvl}.
        \end{itemize}
        
        \begin{table}[H]
            \centering
            \small
            \setlength{\tabcolsep}{6pt}
            \renewcommand{\arraystretch}{1.1}
            \caption{Qwen\mbox{-}VL model parameters (billions).}
            \label{tab:chapter24_qwenvl_params}
            \begin{tabular}{lcccc}
                \toprule
                \textbf{Component} & \textbf{Vision encoder} & \textbf{VL adapter} & \textbf{LLM} & \textbf{Total} \\
                \midrule
                \textbf{Params (B)} & 1.9 & 0.08 & 7.7 & 9.6 \\
                \bottomrule
            \end{tabular}
        \end{table}
        
        \paragraph{Input\mbox{--}output interface (tokenization and special tokens)}
        Qwen\mbox{-}VL textualizes visual content and locations so the LLM can read, reason, and \emph{also} output coordinates in plain text~\cite{bai2023_qwenvl}.
        \begin{itemize}
            \item \textbf{Image tokens.} The adapter’s $M$ visual tokens are inserted as a contiguous block wrapped by \texttt{<img>} and \texttt{</img>} sentinels to clearly demarcate visual content from natural language tokens.
            \item \textbf{Bounding boxes as text.} Boxes are normalized to $[0,1000)$ and serialized as \texttt{"(x\_tl, y\_tl), (x\_br, y\_br)"}; \texttt{<box>}...\texttt{</box>} wrap the coordinate string, and \texttt{<ref>}...\texttt{</ref>} mark the referred phrase, enabling end\mbox{-}to\mbox{-}end grounding and box generation through standard autoregression.
        \end{itemize}
        
        \paragraph{Cross\mbox{-}attention compression (derivation and intuition)}
        Let the ViT yield $F\!\in\!\mathbb{R}^{N\times d_v}$ over $N$ patches. The adapter maintains $M$ learnable queries $Z\!\in\!\mathbb{R}^{M\times d_q}$ with projections
        \[
        Q \;=\; ZW_Q,\qquad K \;=\; FW_K,\qquad V \;=\; FW_V,
        \]
        and applies 2D absolute positional encodings to $(Q,K)$ before attention. The compressed tokens are
        \[
        A \;=\; \mathrm{softmax}\!\Big(\frac{(Q{+}\mathrm{PE}_Q)(K{+}\mathrm{PE}_K)^\top}{\sqrt{d_h}}\Big),\qquad
        H \;=\; A\,V \;\in\; \mathbb{R}^{M\times d_h}.
        \]
        \emph{Intuition.} The learnable queries act like $M$ content\mbox{-}and\mbox{-}position\mbox{-}aware “slots” that selectively pool salient regions while keeping geometry via 2D PEs, yielding a compact, spatially faithful summary for the LLM~\cite{bai2023_qwenvl}.
        
        \paragraph{Training pipeline (three stages)}
        Qwen\mbox{-}VL follows a staged curriculum to first align perception with language, then enrich tasks and resolution, and finally polish instruction following~\cite[Sec.~3]{bai2023_qwenvl}.
        \begin{itemize}
            \item \textbf{Stage 1: Pretraining (224$\times$224).} Freeze the LLM and train the ViT and adapter on $\sim$1.4B cleaned image\mbox{--}text pairs (filtered from $\sim$5B) with next\mbox{-}token loss to establish basic vision\mbox{--}language alignment.
            \item \textbf{Stage 2: Multi\mbox{-}task pretraining (448$\times$448).} Unfreeze all modules and jointly train on captioning, VQA, grounding, referring grounding, grounded captioning, OCR, and pure\mbox{-}text autoregression (sequence length up to 2048), deepening high\mbox{-}resolution, fine\mbox{-}grained skills.
            \item \textbf{Stage 3: Supervised finetuning.} Freeze the ViT and finetune the adapter and LLM on curated multimodal dialogues that emphasize instruction following, multi\mbox{-}image conversation, and localization outputs.
        \end{itemize}
        
        \paragraph{Why this design}
        Compared with feeding all ViT tokens directly, query\mbox{-}based cross\mbox{-}attention keeps the LLM context small and controllable while maintaining spatial detail through 2D position signals; compared with a deeper Q\mbox{-}Former stack, a single cross\mbox{-}attention layer reduces parameters and latency yet preserves the fine\mbox{-}grained cues needed for OCR and grounding thanks to high\mbox{-}resolution multi\mbox{-}task training~\cite{bai2023_qwenvl}.
        
        \begin{figure}[H]
            \centering
            \includegraphics[width=0.70\linewidth]{Figures/Chapter_24/QwenVL_training_pipeline.jpg}
            \caption{Qwen\mbox{-}VL training pipeline: Stage~1 (low\mbox{-}res pretraining; LLM frozen), Stage~2 (high\mbox{-}res multi\mbox{-}task; all unfrozen), Stage~3 (SFT with ViT frozen). Adapted from \cite{bai2023_qwenvl}.}
            \label{fig:chapter24_qwenvl_pipeline}
        \end{figure}
        
        \paragraph{Data}
        \begin{itemize}
            \item \textbf{Pretraining (scale and cleaning).} Qwen\mbox{-}VL begins from roughly 5B image–text pairs and retains about 1.4B pairs (28\%) after aggressive quality filtering, yielding a bilingual corpus. The retained pool draws primarily from LAION\mbox{-}en/LAION\mbox{-}COCO, DataComp, Coyo, CC12M/CC3M, SBU, and COCO Captions on the English side, plus LAION\mbox{-}zh (105M) and 220M in\mbox{-}house Chinese pairs, establishing broad coverage with cleaner supervision.
            \item \textbf{Multi\mbox{-}task pretraining (what skills are taught).} Around 69M supervised samples are used to teach diverse capabilities: captioning (19.7M), general VQA (3.6M), grounding (3.5M), referring expression comprehension and grounded captioning (8.7M each), and text\mbox{-}rich OCR understanding (24.8M), alongside 7.8M pure\mbox{-}text sequences to maintain language fluency. Representative sources include VQAv2, GQA, OK\mbox{-}VQA, GRIT, Visual Genome, RefCOCO/RefCOCO+/RefCOCOg, TextVQA, OCR\mbox{-}VQA, ChartQA, AI2D, SynthDoG (en/zh), and Common Crawl PDFs/HTML~\cite[Table~3]{bai2023_qwenvl}.
            \item \textbf{Design rationale.} The data recipe is intentionally OCR\mbox{-}heavy for text reading, bilingual for cross\mbox{-}lingual robustness, and grounding\mbox{-}rich for localization; adding pure\mbox{-}text helps preserve the LLM’s linguistic priors while the vision\mbox{--}language tasks shape fine\mbox{-}grained multimodal reasoning~\cite[Sec.~3; Tables~2--3]{bai2023_qwenvl}.
        \end{itemize}
        
        \paragraph{Pseudo\mbox{-}code}
        \begin{mintedbox}{python}
            # Three-stage training of Qwen-VL (schematic)
            
            # Init
            vit = OpenCLIP_ViT_bigG()
            llm = Qwen7B()                 # frozen in Stage 1
            adapter = CrossAttnAdapter(    # 1-layer, M=256 learnable queries
            num_queries=256, use_2d_abs_pe=True
            )
            
            # Stage 1: Pretraining (224x224)
            llm.freeze()
            for batch in pretrain_loader(resolution=224):
                imgs, texts = batch
                F = vit(imgs)                      # patch features
                H = adapter.compress(F)            # M x d_h tokens
                tokens = wrap_img_tokens(H, texts) # <img> ... </img> + text
                loss = autoregressive_ce(llm, tokens)
                update(vit, adapter)
            
            # Stage 2: Multi-task pretraining (448x448)
            llm.unfreeze(); vit.unfreeze()
            for batch in multitask_loader(resolution=448, seq_len=2048):
                imgs, multimodal_tokens = batch    # interleaved image-text
                F = vit(imgs)
                H = adapter.compress(F)
                tokens = interleave(H, multimodal_tokens)
                loss = autoregressive_ce(llm, tokens)
                update(vit, adapter, llm)
            
            # Stage 3: Supervised finetuning (instruction/chat)
            vit.freeze(); llm.unfreeze()
            for batch in sft_loader():
                imgs, chat_tokens = batch
                F = vit(imgs)
                H = adapter.compress(F)
                tokens = interleave(H, chat_tokens)  # includes boxes <box>...</box>
                loss = autoregressive_ce(llm, tokens)
                update(adapter, llm)
        \end{mintedbox}
        
        \newpage
        
        \subsubsection{Architecture \& Implementation Details}
        \label{subsubsec:chapter24_qwenvl_arch}
        
        \paragraph{Backbone and adapter}
        The ViT is initialized from OpenCLIP ViT\mbox{-}bigG; the adapter is a single cross\mbox{-}attention layer with trainable queries that compresses to \(M{=}256\) tokens, augmented with 2D absolute PEs in \((Q,K)\). The language backbone is Qwen\mbox{-}7B; Table~\ref{tab:chapter24_qwenvl_params} summarizes parameter counts~\cite{bai2023_qwenvl}.
        
        \paragraph{Resolution and sequence length}
        Images are \(224{\times}224\) in Stage~1 and \(448{\times}448\) in Stage~2; interleaved image–text sequences are packed to 2048 tokens during multi\mbox{-}task pretraining~\cite{bai2023_qwenvl}.
        
        \paragraph{Special tokens and grounding format}
        To keep the interface simple and avoid overfull lines, Qwen\mbox{-}VL wraps visual tokens between short sentinels \verb|<img>| \dots \verb|</img>| and expresses grounding with \verb|<ref>| \dots \verb|</ref>| (text span) plus \verb|<box>| \dots \verb|</box>| (coordinates). Bounding boxes are normalized to $[0,1000)$ and serialized compactly as \verb|(x_1,y_1),(x_2,y_2)|, which the LLM reads and can also generate for localization~\cite{bai2023_qwenvl}.
        
        \subsubsection{Experiments and Ablations}
        \label{subsubsec:chapter24_qwenvl_experiments}
        
        \paragraph{Benchmarks and headline performance}
        Qwen\mbox{-}VL targets image understanding with three representative result slices.\; \emph{Captioning/VQA:} On Nocaps (0\mbox{-}shot) and VQAv2 it reports $121.4$ CIDEr and $79.5\%$, indicating robust vision\,$\to$\,language grounding~\cite[Table~4]{bai2023_qwenvl}. \emph{Text\mbox{-}rich VQA:} On TextVQA it reaches $63.8\%$, reflecting effective OCR\,+\,reasoning integration~\cite[Table~5]{bai2023_qwenvl}. \emph{Grounding:} On RefCOCO test\mbox{-}A it attains $92.26\%$, showcasing precise referring expression comprehension~\cite[Table~6]{bai2023_qwenvl}. The chat\mbox{-}tuned variant improves instruction following (e.g., SEED\mbox{-}Bench All $58.2$) and remains competitive on challenging zero\mbox{-}shot sets such as VizWiz ($38.9\%$)~\cite[Tables~4,7]{bai2023_qwenvl}.
        
        \paragraph{What the ablations test}
        The paper analyzes two design levers in the \emph{position\mbox{-}aware adapter + high\mbox{-}resolution} regime: the number of learnable queries that compress ViT tokens, and the attention strategy used in the ViT at $448{\times}448$ resolution.
        \begin{itemize}
            \item \textbf{How many adapter queries ($M$) to use.} The single cross\mbox{-}attention adapter pools dense ViT features into a fixed $M$\mbox{-}token summary. Appendix~E.2 shows that accuracy rises as $M$ grows and then saturates; $M{=}256$ strikes the best speed/accuracy balance at $448{\times}448$ and is adopted as default~\cite[Sec.~2.1; Appx.~E.2]{bai2023_qwenvl}. Intuition: too few queries underfit fine detail; too many increase compute with diminishing returns.
            \item \textbf{Global vs.\ window attention at high resolution.} Appendix~E.3 compares full (global) attention to windowed attention inside the ViT when moving from $224{\times}224$ to $448{\times}448$. Window attention trains more slowly (about $2.5{\times}$ longer per step at $448$ due to $\sim 4{\times}$ tokens) and is sensitive to hyperparameters; more importantly, it reduces accuracy by nearly ten points on representative recognition/grounding targets in the authors’ setting, so global attention is preferred~\cite[Sec.~3.2; Appx.~E.3]{bai2023_qwenvl}. Intuition: windowing saves FLOPs but weakens long\mbox{-}range interactions that help text reading and referring expression grounding.
        \end{itemize}
        
        \newpage
        
        \paragraph{How these results compare}
        Relative to image\mbox{-}centric assistants (e.g., BLIP\mbox{-}2, InstructBLIP, Shikra), Qwen\mbox{-}VL reports stronger text\mbox{-}heavy understanding (e.g., TextVQA $63.8\%$ vs.\ prior generalists at $42{\sim}53\%$) and competitive or better fine\mbox{-}grained grounding (e.g., RefCOCO test\mbox{-}A $92.26\%$)~\cite[Tables~4--6]{bai2023_qwenvl}. Direct score matching to video\mbox{-}focused systems (e.g., Video\mbox{-}LLaMA, LLaVA\mbox{-}Video) is not like\mbox{-}for\mbox{-}like because those benchmarks emphasize temporal reasoning; on \emph{image} tasks, Qwen\mbox{-}VL generally exceeds LLaVA\mbox{-}style baselines reported in the Qwen\mbox{-}VL tables, while video models shine on long\mbox{-}video QA outside Qwen\mbox{-}VL’s scope~\cite[Fig.~1; Tables~4--7]{bai2023_qwenvl}.
        
        \paragraph{Design choices the ablations support}
        The empirical findings consolidate three choices:
        \begin{itemize}
            \item \textbf{Keep the adapter compact yet expressive.} A single cross\mbox{-}attention layer with $M{=}256$ learnable queries is sufficient for strong captioning/VQA and grounding while keeping end\mbox{-}to\mbox{-}end latency manageable~\cite[Sec.~2.1; Appx.~E.2]{bai2023_qwenvl}.
            \item \textbf{Train at higher image resolution.} Moving from $224$ to $448$ improves text reading and fine\mbox{-}grained perception; the authors therefore raise resolution in multi\mbox{-}task pretraining and keep the ViT frozen during SFT to preserve this fidelity~\cite[Sec.~3.2; Sec.~3.3]{bai2023_qwenvl}.
            \item \textbf{Prefer global attention at high resolution.} Despite higher compute, global attention yields more stable training and clearly higher accuracy than windowed attention in the reported setting, which matters for OCR and grounding~\cite[Sec.~3.2; Appx.~E.3]{bai2023_qwenvl}.
        \end{itemize}
        
        \paragraph{Takeaways}
        A compact, position\mbox{-}aware cross\mbox{-}attention adapter with $M{=}256$ queries, coupled with higher\mbox{-}resolution multi\mbox{-}task training and global ViT attention, explains why Qwen\mbox{-}VL is strong on captioning/VQA (e.g., VQAv2 $79.5\%$), excels at text\mbox{-}centric understanding (e.g., TextVQA $63.8\%$), and remains competitive on grounding (e.g., RefCOCO test\mbox{-}A $92.26\%$) without task\mbox{-}specific heads~\cite[Tables~4--6]{bai2023_qwenvl}.
        
        \begin{figure}[H]
            \centering
            \includegraphics[width=0.75\linewidth]{Figures/Chapter_24/QwenVL_chat_examples.jpg}
            \caption{Representative Qwen\mbox{-}VL\mbox{-}Chat capabilities: multi\mbox{-}image dialogue, multilingual text reading, region grounding/localization, and code understanding. Adapted from \cite{bai2023_qwenvl}.}
            \label{fig:chapter24_qwenvl_chat}
        \end{figure}
        
        \paragraph{Qualitative capabilities}
        Demonstrations include accurate referring\mbox{-}expression grounding with returned boxes, multilingual OCR with cross\mbox{-}lingual reasoning over signs and documents, multi\mbox{-}image comparative analyses, and structured content understanding such as code reading and correction, matching the interface design and high\mbox{-}resolution training~\cite{bai2023_qwenvl}.
        
        \subsubsection{Limitations and Future Work}
        \label{subsubsec:chapter24_qwenvl_limits}
        
        While Qwen\mbox{-}VL establishes a strong generalist baseline with an efficient cross\mbox{-}attention adapter and a textualized grounding interface, several limitations in the 2023 design also outline a clear path for the next generation.
        
        \begin{itemize}
            \item \textbf{Generalist--specialist gap.} Qwen\mbox{-}VL emphasizes broad coverage across captioning, VQA, OCR\mbox{-}rich understanding, and grounding, yet single\mbox{-}task systems trained on narrowly curated data can remain ahead on their home benchmarks (e.g., chart understanding or dense scientific diagrams)~\cite[Sec.~5; Tables~4--6]{bai2023_qwenvl}. This motivates larger capacity and targeted mixtures to approach specialist quality without giving up generality.
            \item \textbf{Compression bottleneck in the adapter.} The single\mbox{-}layer, query\mbox{-}based adapter compresses variable\mbox{-}length ViT tokens to a fixed 256\mbox{-}token summary. This is compute\mbox{-}friendly, but can under-represent dense or highly cluttered scenes; the paper’s ablations select $M{=}256$ as a speed/accuracy compromise rather than an upper bound~\cite[Sec.~2.1; Appx.~E.2]{bai2023_qwenvl}. Future work can explore dynamic token budgets or multi\mbox{-}layer adapters that adapt capacity to content.
        \item \textbf{Resolution and global context trade\mbox{-}offs.} Moving from $224{\times}224$ to $448{\times}448$ improves text reading and fine detail, but also raises sequence length and training cost; windowed attention reduced accuracy in the reported setting, so the paper retained global attention with higher compute~\cite[Sec.~3.2; Appx.~E.3]{bai2023_qwenvl}. This invites designs that keep long\mbox{-}range interactions while scaling to arbitrary resolution efficiently.
        \item \textbf{Modality scope.} Qwen\mbox{-}VL is image\mbox{-}centric; it does not natively model audio or video and relies on textualized coordinates for grounding~\cite[Sec.~2--3]{bai2023_qwenvl}. Extending to temporal and auditory modalities requires position schemes and tokenization that preserve time and synchronization in addition to space.
        \item \textbf{Toward generation.} The system focuses on understanding and localization rather than producing pixels or audio; closing the loop with vision or speech generation would require integrating diffusion/flow decoders or modular generators conditioned on the LLM~\cite[Sec.~5]{bai2023_qwenvl}.
        \end{itemize}
        
        \paragraph{Bridge to Qwen2\mbox{-}VL}
        These constraints foreshadow the priorities addressed by the successor model \emph{Qwen2\mbox{-}VL}~\cite{wang2024_qwen2vl}: scaling capacity and data quality, introducing dynamic\mbox{-}resolution processing to better cover arbitrary sizes and dense layouts, and adding native video support with position schemes designed for multimodal time–space encoding. As the following summary of Qwen2-VL details, these changes directly target Qwen\mbox{-}VL’s compression and resolution trade\mbox{-}offs while broadening the modality scope.
        
    \end{enrichment}
    
    \newpage
    
    \begin{enrichment}[Qwen2-VL: Dynamic Resolution Vision--Language Modeling][subsection]
        \label{enr:subsec_chapter24_qwen2vl}
        
        \paragraph{Motivation}
        Many vision--language pipelines still resize inputs to a fixed canvas (e.g., $224{\times}224$ or scale{+}pad), which can distort aspect ratios and suppress fine details; position encodings are often 1D or absolute 2D, which are not ideal for complex page layouts or temporal reasoning~\cite{li2024_llavaonevision,cheng2024_videollama2,zhang2025_videollama3}. The Qwen\mbox{-}VL design (\S\ref{enr:subsec_chapter24_qwenvl}) alleviated these issues with a position\mbox{-}aware adapter and a textualized grounding interface, but still compressed vision to a fixed token budget at fixed training resolutions~\cite{bai2023_qwenvl}. \emph{Qwen2\mbox{-}VL} advances this line with two core ideas: \emph{naive dynamic resolution}, which ingests images/documents at or near native sizes and produces a content\mbox{-}proportional number of visual tokens, and \emph{multimodal rotary position embedding} (M\mbox{-}RoPE), which jointly encodes time, height, and width to unify text, images, and videos within one decoder~\cite{wang2024_qwen2vl}. Relative to the systems summarized in \S\ref{enr:subsec_chapter24_llava_onevision}, \S\ref{enr:subsec_chapter24_videollama2}, and \S\ref{enr:subsec_chapter24_videollama3}, Qwen2\mbox{-}VL aims for a single native\mbox{-}resolution pipeline that scales across OCR, document understanding, and long\mbox{-}video reasoning, with 2B/7B/72B variants sharing the same vision stack.
        
        \begin{figure}[H]
            \centering
            \includegraphics[width=0.8\linewidth]{Figures/Chapter_24/Qwen2_VL_capabilities.jpg}
            \caption{Illustrative capabilities of Qwen2\mbox{-}VL: multilingual OCR, document and diagram parsing, math/code reasoning, video analysis, live chat, grounding, and tool/agent interactions. Adapted from \cite{wang2024_qwen2vl}.}
            \label{fig:chapter24_qwen2vl_caps}
        \end{figure}
        
        \subsubsection{Method}
        \label{subsubsec:chapter24_qwen2vl_method}
        
        \paragraph{Design overview}
        \begin{itemize}
            \item \textbf{Naive dynamic resolution.} Images are ingested at native resolution and extreme aspect ratios without global resize; token counts scale with content via a light 2{\mbox{$\times$}}2 token merger after the ViT to control sequence length.
            \item \textbf{Multimodal RoPE (M\mbox{-}RoPE).} Rotary position encodings are decomposed into temporal ($t$), height ($h$), and width ($w$) components, enabling consistent space--time indexing across text, images, and videos for attention.
            \item \textbf{Unified image--video training.} Images are treated as two identical frames (static $t$), while videos use true $t$ with a shallow 3D stem; both pass through the same ViT and token merger before the LLM.
        \end{itemize}
        
        \newpage
        
        \paragraph{Naive dynamic resolution}
        Let an input image $\mathbf{x}\!\in\!\mathbb{R}^{H\times W\times C}$ be tokenized by a ViT with patch size $p$, producing a grid $\mathcal{G}$ of $N{=}\lceil H/p\rceil\!\times\!\lceil W/p\rceil$ patch tokens $F\!\in\!\mathbb{R}^{N\times d_v}$. Instead of resizing $\mathbf{x}$ to a single fixed canvas, \emph{Qwen2\mbox{-}VL} keeps the native grid and regulates length with a learnable $2{\times}2$ \emph{token merger}~\cite{wang2024_qwen2vl}. Concretely, for each non\mbox{-}overlapping $2{\times}2$ neighborhood of tokens $\{f_{i,j}\}_{(i,j)\in\{(2u,2v),(2u{+}1,2v),(2u,2v{+}1),(2u{+}1,2v{+}1)\}}$, the merger concatenates and projects
        \[
        m_{u,v}\;=\;\phi\!\Big(\big[f_{2u,2v};\,f_{2u{+}1,2v};\,f_{2u,2v{+}1};\,f_{2u{+}1,2v{+}1}\big]\,W_1\Big)\,W_2
        \;\in\;\mathbb{R}^{d_v},
        \]
        where $W_1,W_2$ are linear layers and $\phi$ is a pointwise nonlinearity. This reduces tokens by $\approx 4{\times}$ while preserving local structure, yielding a content\mbox{-}proportional sequence length without distorting aspect ratios. When inputs are extremely large (e.g., tall documents or 4K scans), the same merger can be \emph{applied hierarchically} (again on the merged grid) until a target budget is met, trading spatial detail for tractable context length in a controlled, locality\mbox{-}aware way. Multiple images are serialized by simple concatenation of their merged grids (each demarcated by vision sentinels) before interleaving with text in the decoder~\cite{wang2024_qwen2vl}.
        
        Videos $\mathbf{V}\!\in\!\mathbb{R}^{T\times C\times H\times W}$ are handled frame\mbox{-}wise with the same mechanism. Let $N_t$ be the per\mbox{-}frame tokens after patching and $2{\times}2$ merging; the visual sequence length is $\sum_{t=1}^{T}\!N_t$. To \emph{balance} space and time under a global budget $B_{\mathrm{vis}}$, Qwen2\mbox{-}VL uses simple policies such as: (i) per\mbox{-}frame merging depth chosen so $N_t\!\le\!N_{\max}$; (ii) uniform or content\mbox{-}aware temporal subsampling (e.g., drop low\mbox{-}motion frames) if $\sum_t N_t\!>\!B_{\mathrm{vis}}$; and (iii) capping the number of frames processed at native resolution while allowing coarser merging for the remainder~\cite{wang2024_qwen2vl}. Intuitively, this yields \emph{content\mbox{-}proportional tokens} across images and videos: dense pages or keyframes retain more tokens, while redundant regions compress, preventing token overflow in long documents or long clips without uniform, detail\mbox{-}destroying downscales.
        
        \paragraph{M\mbox{-}RoPE for space--time}
        Rotary position embedding (RoPE) encodes \emph{relative} offsets by rotating query/key channel pairs with a phase that depends on position. Standard 1D\mbox{-}RoPE uses a single index; Qwen2\mbox{-}VL generalizes this to \emph{three axes}—time, height, width—via \emph{multimodal RoPE (M\mbox{-}RoPE)}~\cite{wang2024_qwen2vl}. Each token is assigned a 3D ID $\pi\!=\!(t,h,w)$, and the model allocates disjoint channel subspaces to the three axes. Writing a query head as $\mathbf{q}\!\in\!\mathbb{R}^{d}$ with a partition $(\mathbf{q}^{(t)},\mathbf{q}^{(h)},\mathbf{q}^{(w)})$, M\mbox{-}RoPE applies axis\mbox{-}wise rotations
        \[
        \widetilde{\mathbf{q}}^{(a)} \;=\; R^{(a)}(\pi_a)\,\mathbf{q}^{(a)},\quad
        R^{(a)}(\pi_a)\;=\;\bigoplus_{i=1}^{d_a/2}
        \begin{bmatrix}
            \cos\!\big(\theta^{(a)}_i\,\pi_a\big) & -\sin\!\big(\theta^{(a)}_i\,\pi_a\big)\\[2pt]
            \sin\!\big(\theta^{(a)}_i\,\pi_a\big) & \phantom{-}\cos\!\big(\theta^{(a)}_i\,\pi_a\big)
        \end{bmatrix},
        \quad a\in\{t,h,w\},
        \]
        with analogous $\widetilde{\mathbf{k}}^{(a)}$ for keys, where $\{\theta^{(a)}_i\}$ are geometric frequencies per axis and $\oplus$ denotes block\mbox{-}diagonal composition. Concatenating the rotated subspaces gives $\widetilde{\mathbf{q}},\widetilde{\mathbf{k}}\!\in\!\mathbb{R}^{d}$ used in attention. The inner product between two tokens with IDs $\pi$ and $\pi'$ then depends on their \emph{relative} offsets $(\Delta t,\Delta h,\Delta w)$, which makes the attention scores equivariant to spatio\mbox{-}temporal translations. Qwen2\mbox{-}VL assigns IDs as follows: text tokens share a constant time index and a 1D progression along the width subspace (to preserve textual order); image tokens share a time index but use their $(h,w)$ grid locations; video tokens use frame order for $t$ and per\mbox{-}frame $(h,w)$ for space~\cite{wang2024_qwen2vl}.
        
        \newpage
        
        \paragraph{Pseudo\mbox{-}code for dynamic resolution and M\mbox{-}RoPE}
        \begin{mintedbox}{python}
            # Schematic pipeline: native-resolution vision, 2x2 token merger, M-RoPE assignment
            
            def encode_image_or_video(frames, vit, merger2x2):
                """
                frames: list of HxW images (len=1 for image; >1 for video)
                vit: ViT with 2D-RoPE on spatial axes
                merger2x2: MLP that maps 4 patch tokens -> 1 token
                """
                visual_tokens = []
                for t, img in enumerate(frames):
                    # 1) Native-resolution patching and ViT encoding (no global resize).
                    F_hw = vit.patch_encode(img)                 # [H/p, W/p, C]
                    # 2) 2x2 merger to control token count content-proportionally.
                    F_merge = block_merge_2x2(F_hw)              # [H/(2p), W/(2p), C]
                    # 3) Flatten to [N_t, C] and attach 3D position ids (t,h,w) for M-RoPE.
                    T = flatten_with_ids(F_merge, t_axis=t)      # [(N_t, C), (ids_t,h,w)]
                    visual_tokens.append(T)
                    return concat(visual_tokens)                     # Variable-length visual sequence
            
            def fuse_with_text(visual_tokens, text_tokens, llm):
                """
                Interleave markers and feed to LLM with multimodal RoPE activated on Q/K.
                """
                seq = [TOK.VISION_START] + visual_tokens + [TOK.VISION_END] + text_tokens
                return llm.generate(seq)
        \end{mintedbox}
        
        \paragraph{Why M\mbox{-}RoPE instead of 2D absolute encodings}
        M\mbox{-}RoPE replaces the adapter’s 2D absolute position signals with a factorized, rotary scheme over time, height, and width, yielding a single spatio\mbox{-}temporal reference frame for text, images, video.
        \begin{itemize}
            \item \textbf{Relative, resolution\mbox{-}agnostic geometry.} Rotary phases encode \emph{relative} $(\Delta h,\Delta w)$ offsets, improving layout transfer to unseen sizes and aspect ratios compared with absolute tables that require interpolation. 
            \item \textbf{Native temporal indexing.} A dedicated temporal axis allows attention to condition on $\Delta t$ jointly with $(\Delta h,\Delta w)$, enabling spatio\mbox{-}temporal reasoning for videos in a shared decoder space. 
            \item \textbf{Long\mbox{-}context stability.} Using rotations tied to relative offsets avoids very large absolute indices, which empirically stabilizes extrapolation to long sequences. 
        \end{itemize}
        
        \textit{Practical intuition.}
        \begin{itemize}
            \item \textbf{Video query.} For “What happens after the ball crosses the line?”, attention can prioritize patches with small positive $\Delta t$ near the line’s location in $(h,w)$, capturing immediate post\mbox{-}event dynamics. 
            \item \textbf{Document query.} For “Read the footnote below the figure”, attention can target tokens with positive $\Delta h$ under the referenced region within the same frame, preserving page geometry. 
        \end{itemize}
        \textit{Scope.} The paper focuses on text, image, and video; audio is not modeled and would require an additional axis or synchronized timestamping beyond this work~\cite{wang2024_qwen2vl}.
        
        \paragraph{Unified multimodal serialization}
        \begin{itemize}
            \item \textbf{Vision segment markers.} Visual tokens are delimited: \verb|<|{\footnotesize\texttt{|}}\verb|vision_start|\verb|>| and \verb|<|{\footnotesize\texttt{|}}\verb|vision_end|\verb|>| to keep the interface LLM-native and avoid custom decoders~\cite{wang2024_qwen2vl}.
            \item \textbf{Grounding strings.} Bounding boxes are normalized to $[0,1000)$ and serialized compactly as \verb|(x1,y1),(x2,y2)|; referred spans are output as plain text. The textual interface lets the LLM \emph{read} and \emph{generate} locations in one channel~\cite{wang2024_qwen2vl}.
        \end{itemize}
        
        \subsubsection{Architecture \& Implementation Details}
        \label{subsubsec:chapter24_qwen2vl_arch}
        
        \paragraph{Model variants}
        \begin{table}[H]
            \centering
            \small
            \setlength{\tabcolsep}{6pt}
            \renewcommand{\arraystretch}{1.1}
            \caption{Qwen2\mbox{-}VL variants and sizes.}
            \label{tab:chapter24_qwen2vl_models}
            \begin{tabular}{lcc}
                \toprule
                \textbf{Model.} & \textbf{Vision encoder (M).} & \textbf{LLM (B).} \\
                \midrule
                Qwen2\mbox{-}VL\mbox{-}2B & $\sim$675 & 1.5 \\
                Qwen2\mbox{-}VL\mbox{-}7B & $\sim$675 & 7.6 \\
                Qwen2\mbox{-}VL\mbox{-}72B & $\sim$675 & 72.0 \\
                \bottomrule
            \end{tabular}
        \end{table}
        
        \paragraph{Implementation notes}
        \begin{itemize}
            \item \textbf{Vision stack.} The ViT employs 2D\mbox{-}RoPE and a shallow 3D stem for videos, followed by a 2{\mbox{$\times$}}2 token merger MLP to reduce sequence length with minimal local detail loss~\cite{wang2024_qwen2vl}.
            \item \textbf{LLM stack.} The LLM is initialized from Qwen2 (2B/7B/72B) and trained to interleave visual and text tokens in one decoder stream with M\mbox{-}RoPE applied on attention~\cite{wang2024_qwen2vl}.
            \item \textbf{Training curriculum.} The recipe follows Qwen\mbox{-}VL’s three stages: vision{\mbox{--}}language alignment at low cost, full multi{\mbox{-}}task image{\mbox{/}}video pretraining, and instruction tuning for chat, grounding, OCR, and tool use~\cite{wang2024_qwen2vl}.
        \end{itemize}
        
        \begin{figure}[H]
            \centering
            \includegraphics[width=0.70\linewidth]{Figures/Chapter_24/Qwen2_VL_adaptiveness.jpg}
            \caption{Adaptiveness to native resolutions and extreme aspect ratios: token counts scale with visual content rather than a fixed canvas. Adapted from \cite{wang2024_qwen2vl}.}
            \label{fig:chapter24_qwen2vl_adapt}
        \end{figure}
        
        \begin{figure}[H]
            \centering
            \includegraphics[width=0.70\linewidth]{Figures/Chapter_24/Qwen2_VL_MRoPE_demonstration.jpg}
            \caption{M\mbox{-}RoPE decomposes rotary embeddings into temporal, height, and width components, unifying position encoding for text, images, and videos. Adapted from \cite{wang2024_qwen2vl}.}
            \label{fig:chapter24_qwen2vl_mrope}
        \end{figure}
        
        \subsubsection{Experiments and Ablations}
        \label{subsubsec:chapter24_qwen2vl_experiments}
        
        \paragraph{Benchmarks and headline performance}
        Qwen2\mbox{-}VL shows very strong text–rich perception and competitive general reasoning, especially at 72B parameters~\cite{wang2024_qwen2vl}. On document/OCR style tasks, Qwen2\mbox{-}VL\mbox{-}72B attains DocVQA (test) $96.5$ (GPT\mbox{-}4o $92.8$, Claude\mbox{-}3.5 Sonnet $95.2$), TextVQA (val) $85.5$, InfoVQA (test) $84.5$, OCRBench $877$, and RealWorldQA $77.8$. On broad suites it is strong but not uniformly best: MMMU (val) $64.5$ vs.\ GPT\mbox{-}4o $69.1$ and Claude\mbox{-}3.5 $68.3$; MMBench\mbox{-}EN (test) $86.5$; MME\textsubscript{sum} $2482.7$~\cite[Table~2]{wang2024_qwen2vl}. The 7B variant offers a favorable cost–quality balance, e.g., TextVQA $84.3$, OCRBench $866$, RealWorldQA $70.1$, MMMU (val) $54.1$~\cite[Table~2]{wang2024_qwen2vl}.
        
        \paragraph{Video understanding}
        Unified image–video training together with M\mbox{-}RoPE yields strong long–video results. Qwen2\mbox{-}VL\mbox{-}72B reports EgoSchema (test) $77.9$ (GPT\mbox{-}4o $72.2$), MVBench $73.6$, PerceptionTest (test) $68.0$, and Video\mbox{-}MME $71.2/77.8$ (w/o/w subtitles; GPT\mbox{-}4o $71.9/77.2$)~\cite[Table~4]{wang2024_qwen2vl}.
        
        \paragraph{Grounding}
        Referring expression comprehension scales with model size, approaching specialist detectors while retaining generality. Qwen2\mbox{-}VL\mbox{-}72B reaches RefCOCO (test\mbox{-}A) $95.3$, RefCOCO+ (test\mbox{-}A) $93.8$, and RefCOCOg (test) $90.4$, improving on Qwen\mbox{-}VL and remaining close to specialist models such as ONE\mbox{-}PEACE, UNINEXT\mbox{-}H, and G\mbox{-}DINO\mbox{-}L~\cite[Table~6]{wang2024_qwen2vl}.
        
        \paragraph{Multilingual OCR (internal)}
        On an internal multilingual OCR suite, Qwen2\mbox{-}VL\mbox{-}72B surpasses GPT\mbox{-}4o on several languages (e.g., Korean $94.5$ vs.\ $87.8$, Japanese $93.4$ vs.\ $88.3$, French $94.1$ vs.\ $89.7$), with a small shortfall on Arabic ($70.7$ vs.\ $75.9$)~\cite[Table~3]{wang2024_qwen2vl}. This reflects robust cross\mbox{-}lingual text reading while highlighting scripts that remain challenging.
        
        \paragraph{Why dynamic resolution helps}
        Ablations on Qwen2\mbox{-}VL\mbox{-}7B compare \emph{fixed} image tokens to \emph{dynamic} tokens that scale with content~\cite[Table~7]{wang2024_qwen2vl}. Using a fixed budget of $576$ image tokens yields InfoVQA (val) $65.72$, RealWorldQA $65.88$, and OCRBench $828$, whereas dynamic resolution (avg.\ $\sim 1924$ tokens for image content) attains $75.89$, $70.07$, and $866$, respectively, while still consuming \emph{fewer} tokens than the largest fixed settings. Accuracy is also robust across moderate fixed sizes (e.g., $1600{\sim}3136$), indicating that native-resolution packing plus the $2{\times}2$ merger is an efficient default for text\mbox{-}heavy and dense layouts~\cite[Table~7]{wang2024_qwen2vl}.
        
        \newpage
        
        \paragraph{Why M\mbox{-}RoPE matters}
        Replacing 1D\mbox{-}RoPE with M\mbox{-}RoPE consistently improves video tasks and maintains or slightly improves image tasks~\cite[Table~8]{wang2024_qwen2vl}. For example, PerceptionTest (test) rises from $46.6$ to $47.4$, NextQA from $43.9$ to $46.0$, and STAR from $55.5$ to $57.9$; on image benchmarks MathVista increases from $39.2$ to $43.4$ and MMBench from $58.6$ to $60.6$. These gains support the benefit of explicit $(t,h,w)$ encoding for unified space–time attention.
        
        \paragraph{Length extrapolation}
        With M\mbox{-}RoPE indexing, Qwen2\mbox{-}VL\mbox{-}72B sustains accuracy when the inference sequence length exceeds the $16{,}384$ token training limit, remaining strong toward $48$K and beyond on Video\mbox{-}MME~\cite[Fig.~5]{wang2024_qwen2vl}.
        
        \begin{figure}[H]
            \centering
            \includegraphics[width=0.70\linewidth]{Figures/Chapter_24/Qwen2_VL_accuracy_length.jpg}
            \caption{Inference length extrapolation on Video\mbox{-}MME: accuracy remains robust beyond the 16K training context, with strong performance up to long contexts. Adapted from \cite{wang2024_qwen2vl}.}
            \label{fig:chapter24_qwen2vl_len}
        \end{figure}
        
        \paragraph{Resolution sensitivity}
        Varying \verb|min_pixels| (i.e., upscaling small inputs before patching) shows that moderate increases improve perceptual and text\mbox{-}rich tasks such as InfoVQA, HallucinationBench, and OCRBench, with diminishing returns or slight drops at extreme upscales~\cite[Fig.~4]{wang2024_qwen2vl}.
        
        \begin{figure}[H]
            \centering
            \includegraphics[width=0.70\linewidth]{Figures/Chapter_24/Qwen2_VL_minpixels.jpg}
            \caption{Effect of \emph{min\_pixels}: modest upscaling tends to help text\mbox{-}rich and fine\mbox{-}structure tasks, while extreme upscaling can be counterproductive. Adapted from \cite{wang2024_qwen2vl}.}
            \label{fig:chapter24_qwen2vl_minpix}
        \end{figure}
        
        \paragraph{Scaling behavior and training curriculum}
        Performance improves with model size across OCR, general VQA, video, and math, e.g., (2B $\to$ 7B $\to$ 72B) MVBench $63.2 \to 67.0 \to 73.6$ and MathVista (testmini) $43.0 \to 58.2 \to 70.5$~\cite[Table~2; Fig.~6(a)]{wang2024_qwen2vl}. The paper also analyzes the effect of \emph{increasing training tokens during the second pretraining stage} for Qwen2\mbox{-}VL\mbox{-}7B: as the token count grows, most benchmarks improve smoothly (e.g., AI2D and InfoVQA), while some VQA scores fluctuate, consistent with task sensitivity to data mixtures~\cite[Fig.~6(b)]{wang2024_qwen2vl}. Exact per\mbox{-}stage token totals are not disclosed; the reported trend supports allocating substantial budget to the multi\mbox{-}task, native\mbox{-}resolution stage to strengthen fine\mbox{-}grained perception and long\mbox{-}context use.
        
        \subsubsection{Limitations and Future Work}
        \label{subsubsec:chapter24_qwen2vl_limits}
        
        Qwen2\mbox{-}VL introduces native dynamic resolution, content\mbox{-}proportional visual tokens, and M\mbox{-}RoPE to unify images and video, yet several design trade\mbox{-}offs remain visible in the method and ablations~\cite{wang2024_qwen2vl}. The points below clarify where the current system can struggle and which directions the literature suggests are most promising.
        
        \begin{itemize}
            \item \textbf{Resolution--efficiency trade-offs.} Dynamic resolution with a $2{\times}2$ token merger controls sequence length on most inputs, but extremely dense pages (e.g., long PDF scans, multi\mbox{-}column forms) or extreme aspect ratios can still yield very long visual sequences that push decoder context and memory~\cite[Table~2; Table~7]{wang2024_qwen2vl}. Likely remedies include locality\mbox{-}aware ViT front\mbox{-}ends (adaptive strides or applying windows only where texture is high) and \emph{hierarchical} merging that preserves fine detail selectively while keeping a coarse global map for long\mbox{-}range reasoning.
            
            \item \textbf{Temporal grounding precision.} M\mbox{-}RoPE provides a clean 3D positional scheme (time, height, width) but primarily encodes \emph{relative} order, while downstream usage often requires \emph{absolute} timestamps and robust handling of variable FPS~\cite[Table~4; Table~2; Fig.\,``Accuracy vs.\ Inference Sequence Length'']{wang2024_qwen2vl}. Incorporating wall\mbox{-}clock alignment and FPS\mbox{-}aware sampling at the tokenizer level might improve fine\mbox{-}grained event localization over long videos without sacrificing the demonstrated length extrapolation.
            
            \item \textbf{Structured extraction and precise geometry.} The textual interface excels at free\mbox{-}form answers and box grounding, but applications needing strict schemas (e.g., JSON for invoices) or exact point/segment outputs can still trail specialist parsers and detectors; note that the paper’s grounding results chiefly report boxes on RefCOCO/+/g~\cite[Table~6]{wang2024_qwen2vl}. Extending grounding beyond rectangles to points and polygons, and supervising \emph{format\mbox{-}faithful} outputs (e.g., constrained decoding for tables/graphs), are natural follow\mbox{-}ups to close this gap.
            
            \item \textbf{Long\mbox{-}context stability.} Variable visual token counts improve fidelity yet also make runtime and memory less predictable for long, interleaved image\mbox{+}text\mbox{+}video sessions~\cite[Table~7; Fig.\,``Accuracy vs.\ Inference Sequence Length'']{wang2024_qwen2vl}. A practical direction is to expose user\mbox{-}controllable token budgets and train learned token\mbox{-}pruning policies that down\mbox{-}weight redundant regions while preserving the rest. 
            
            \item \textbf{Hallucination control and attribution.} Qwen2\mbox{-}VL improves robustness on perception and instruction tests, yet open\mbox{-}ended, multi\mbox{-}hop queries can still trigger visual or factual confabulations, as reflected by mixed movement on aggregate evaluations~\cite[Table~2]{wang2024_qwen2vl}. Adding retrieval\mbox{-}augmented prompts for charts/docs and emitting visual evidence pointers (e.g., citing boxes or time spans used) can reduce ungrounded claims and aid auditing.
        
            \item \textbf{Concise outlook: Qwen2.5\mbox{-}VL and Qwen3\mbox{-}VL.} The \emph{Qwen2.5\mbox{-}VL} report refines dynamic resolution handling with more efficient vision attention, introduces absolute\mbox{-}time indexing with dynamic FPS for video, extends grounding beyond boxes, and strengthens long\mbox{-}context efficiency~\cite{qwen2025_qwen25technicalreport}. Public descriptions of \emph{Qwen3\mbox{-}VL} emphasize broader tool ecosystems and more stable long\mbox{-}horizon multimodal reasoning while retaining native\mbox{-}resolution fidelity; these directions align with the limitations outlined above.
        \end{itemize}
        
    \end{enrichment}
    
\end{enrichment}

\newpage

\begin{enrichment}[Long-Context Modeling][section]
    \label{enr:sec_chapter24_long_context}
    Videos spanning minutes to hours demand mechanisms that scale beyond quadratic attention or strict autoregression. 
    While recent VLLMs such as \emph{Video-LLaMA}, the \emph{Qwen-VL} family, and \emph{LLaVA-OneVision} have pushed broad multimodal competence (instruction-following, OCR, grounding, AnyRes/dynamic-resolutions, multi-image interleaving), they typically rely on aggressive token compression, fixed or short temporal windows, or sparsified frame sampling. 
    For \emph{truly long} contexts—hour-long streams, movie-length narratives, live feeds—these heuristics alone are not enough; models must \emph{preserve} salient details over time \emph{and} offer compute that grows sub-quadratically with sequence length.
    
    \medskip
    \noindent\textbf{A timeline of approaches focused on \emph{long} context.}
    \begin{itemize}
        \item \textbf{Memory-augmented transformers (2022).} \emph{MeMViT} \cite{wu2022_memvit} augments multiscale ViTs with segment-level external memory, pooling and reusing tokens across chunks. This turns naive “process each clip independently” into \emph{stateful} inference, extending temporal support and improving accuracy without exploding computation.
        \item \textbf{Streaming and state-space models (2023).} Selective SSMs such as \emph{Mamba} \cite{mamba2023_selective} maintain a compact recurrent state that is updated online as frames arrive, enabling \emph{linear-time} inference in sequence length. This suits long or continuous video where latency and throughput matter more than full quadratic attention.
        \item \textbf{Efficient long-video reasoning in VLLMs (2024).} \emph{LongVLM} \cite{weng2024_longvlm} integrates memory caches, streaming updates, and sparse token selection into an LLM-centric pipeline, prioritizing \emph{serving-time} constraints: it keeps the most informative spatiotemporal evidence, refreshes memory as new segments come in, and sustains coherent reasoning across long narratives.
        \item \textbf{Blockwise/sparse attention for ultra-long sequences (first release - 2024).} \emph{LWM} \cite{liu2025_lwm} demonstrates \emph{Blockwise RingAttention} to process \emph{million-token} video--language contexts. Rather than compressing away detail, it restructures attention itself (blockwise, ring connectivity) so compute and memory scale to unprecedented lengths.
    \end{itemize}
    
    \newpage
    
    \begin{enrichment}[MeMViT: Memory-Augmented Multiscale ViTs][subsection]
        \label{enr:subsec_chapter24_memvit}
        
        \paragraph{Motivation}
        Many video backbones attain strong accuracy on \emph{short} clips but become prohibitively expensive when naively scaling temporal support by feeding more frames to the model, causing quadratic growth in attention cost and ballooning GPU memory/runtime. The MeMViT paper proposes an alternative: process a long video \emph{online} as a sequence of short clips and \emph{cache transformer keys/values as memory} across iterations, so current queries can attend to a compact representation of the past with only marginal overhead~\cite{wu2022_memvit}. Concretely, MeMViT reports temporal support up to $30{\times}$ longer than baselines at just $\sim\!4.5\%$ more compute, delivering higher accuracy under the same FLOPs envelope on AVA and other tasks.\footnote{See Fig.~1 and Sec.~1 of \cite{wu2022_memvit} for the compute$\leftrightarrow$duration trade\mbox{-}off and motivation.} 
        
        \begin{figure}[H]
            \centering
            \includegraphics[width=0.85\linewidth]{Figures/Chapter_24/MeMViT_temporal_support.jpg}\\[6pt]
            \includegraphics[width=0.85\linewidth]{Figures/Chapter_24/MeMViT_idea.jpg}
            \caption{Problem setup and key idea. Traditional long\mbox{-}term scaling increases input frames and explodes compute/memory; MeMViT maintains a cached, hierarchically compressed memory and lets current queries attend to it efficiently. Adapted from \cite{wu2022_memvit}.}
            \label{fig:chapter24_memvit_intro}
        \end{figure}
        
        \paragraph{Preliminaries: ViT and MViT}
        A standard transformer layer consumes a sequence of tokens $X\!\in\!\mathbb{R}^{N\times d}$, projects to queries, keys, and values,
        \begin{equation}
            Q = XW_Q,\qquad K = XW_K,\qquad V = XW_V,
            \label{eq:memvit_linear}
        \end{equation}
        and applies scaled dot-product attention,
        \begin{equation}
            Z \;=\; \mathrm{Softmax}\!\left(\frac{QK^{\top}}{\sqrt{d}}\right)V,\qquad Z\in\mathbb{R}^{N\times d_{\text{out}}}.
            \label{eq:memvit_attn}
        \end{equation}
        \emph{ViT} flattens image patches to form $X$. For videos $x\!\in\!\mathbb{R}^{T\times C\times H\times W}$, a tubelet embedding with tube size $(t_p\!\times\!p\!\times\!p)$ yields
        \[
        N_0 \;=\; \frac{T}{t_p}\cdot\frac{H}{p}\cdot\frac{W}{p},\qquad Z_0\in\mathbb{R}^{N_0\times d_0},
        \]
        so naively attending over all tokens scales quadratically in $N_0$, which grows quickly with $T$, $H$, and $W$.
        
        \medskip
        \textbf{MViT: multiscale hierarchy + pooling attention.}
        Multiscale ViT (MViT) addresses video scale by borrowing two CNN-style principles~\cite{fan2021_mvit,li2021_improved_mvit}:
        \begin{itemize}
            \item \emph{Multi-stage pyramid.} The model is organized into stages $s=1,\dots,S$. Each stage reduces spatiotemporal resolution (token count) while increasing channel capacity, producing a sequence $Z_s\!\in\!\mathbb{R}^{N_s\times d_s}$ with $N_{s+1}\!<\!N_s$ and $d_{s+1}\!>\!d_s$. This yields large receptive fields and efficient compute at deeper layers, analogous to CNN feature pyramids.
            \item \emph{Pooling attention.} Inside a stage, attention cost is reduced by pooling along $(t,h,w)$ before forming $Q,K,V$ (``pool-then-attend''), or equivalently pooling intermediate representations that generate $Q,K,V$ (``pooling attention''). This shrinks the token dimension over which attention operates, lowering the quadratic factor in $N_s$ without discarding channel information relevant for recognition.
        \end{itemize}
        Concretely, let a stage-$s$ block view its input as a 4D tensor $X_s\!\in\!\mathbb{R}^{T_s\times H_s\times W_s\times d_s}$ (with $N_s\!=\!T_sH_sW_s$ after flattening). The \emph{improved MViT} variant adopted by MeMViT applies lightweight spatiotemporal pooling \emph{before} linear projections~\cite{li2021_improved_mvit}:
        \begin{equation}
            \bar Q_s = P_Q(X_s),\quad \bar K_s = P_K(X_s),\quad \bar V_s = P_V(X_s),\qquad
            Q_s = \bar Q_s W_Q,\; K_s = \bar K_s W_K,\; V_s = \bar V_s W_V,
            \label{eq:memvit_pool_then_linear}
        \end{equation}
        where $P_{\{\cdot\}}$ are strided average-pooling (or equivalent) operators over $(T_s,H_s,W_s)$ that reduce token count from $N_s$ to $\bar N_s\!\ll\!N_s$ while preserving channels $d_s$. Attention then runs on $(\bar N_s\times \bar N_s)$ rather than $(N_s\times N_s)$. Intuitively, early stages keep many fine-grained tokens and small $d_s$ for local motion/texture; later stages operate on few tokens with large $d_s$ for global semantics. This pyramidal design provides a compute-efficient path to long-range spatiotemporal context and forms the backbone on which MeMViT attaches its memory mechanism.
        
        \newpage
        
        \paragraph{Method: Memory Attention and Hierarchical Caching}
        \label{subsubsec:chapter24_memvit_method}
        
        \textbf{Clip-wise online attention with a rolling $K/V$ cache (flattening \& multi-head shapes).}
        Given a video $x\!\in\!\mathbb{R}^{T\times C\times H\times W}$, MeMViT streams it as clips $x^{(t)}\!\in\!\mathbb{R}^{\tau\times C\times H\times W}$ in order $t{=}1,2,\dots$~\cite[Sec.~4.1]{wu2022_memvit}. Each clip is tubelet-tokenized and passed through an MViT stage that \emph{first pools then projects} (Eq.~\eqref{eq:memvit_pool_then_linear}). After pooling along $(t,h,w)$, the stage-$s$ activations have grid shape
        \[
        \tilde Q^{(t)},\tilde K^{(t)},\tilde V^{(t)}\in\mathbb{R}^{T_p\times H_p\times W_p\times d_s},
        \]
        which are \emph{flattened} (e.g., $t$-major, then row-major over $(h,w)$) to sequences
        \[
        \bar N_s \;=\; T_pH_pW_p,\qquad
        \bar Q^{(t)},\bar K^{(t)},\bar V^{(t)}\in\mathbb{R}^{\bar N_s\times d_s}.
        \]
        Flattening is valid because self-attention is permutation-equivariant; a geometry-respecting rasterization preserves locality.
        
        \emph{Multi-head attention notation and weight shapes.} Let the model width at stage $s$ be $d_s$ and the number of heads be $h$. The per-head width is $d_h$ with
        \[
        d_s \;=\; h\,d_h.
        \]
        There are two equivalent ways to write the projection matrices:
        \begin{itemize}
            \item \textbf{Packed (all heads at once):} $W_Q,W_K,W_V\in\mathbb{R}^{d_s\times d_s}$ map $d_s\!\to\!d_s$; outputs are then reshaped to $(h,\cdot,d_h)$.
            \item \textbf{Per-head view (clearer for shapes):} for each head $r\!\in\!\{1,\dots,h\}$, $W_Q^{(r)},W_K^{(r)},W_V^{(r)}\in\mathbb{R}^{d_s\times d_h}$ map $d_s\!\to\!d_h$ and are applied in parallel, then concatenated along the last dim to recover $d_s$.
        \end{itemize}
        Both views are identical because $h\,d_h\!=\!d_s$. After attention, the $h$ head outputs (each $d_h$) are concatenated to $\mathbb{R}^{\bar N_s\times (h d_h)}{=}\mathbb{R}^{\bar N_s\times d_s}$ and mixed by the standard output projection
        \[
        W_O \in \mathbb{R}^{d_s\times d_s},
        \]
        which linearly combines head channels back into the stage width.
        
        \emph{Current-step projections (with shapes).} Using the per-head view for clarity, for head $r$:
        \[
        Q_r^{(t)}=\bar Q^{(t)}W_Q^{(r)}\in\mathbb{R}^{\bar N_s\times d_h},\;\;
        K_{r,\text{cur}}^{(t)}=\bar K^{(t)}W_K^{(r)}\in\mathbb{R}^{\bar N_s\times d_h},\;\;
        V_{r,\text{cur}}^{(t)}=\bar V^{(t)}W_V^{(r)}\in\mathbb{R}^{\bar N_s\times d_h}.
        \]
        
        \emph{Rolling $K/V$ cache.} MeMViT augments the \emph{current} keys/values with a FIFO cache from the previous $M$ clips, stopping gradients into cached steps (read-only memory)~\cite[Sec.~4.1]{wu2022_memvit}:
        \begin{align}
            \bar K^{(t)} \;:=\; \big[\,\mathrm{sg}(\bar K^{(t-M)}),\ldots,\mathrm{sg}(\bar K^{(t-1)}),\bar K^{(t)}\,\big]\in\mathbb{R}^{(\bar N_s{+}N_m)\times d_s}, \label{eq:memvit_basic_K}\\
            \bar V^{(t)} \;:=\; \big[\,\mathrm{sg}(\bar V^{(t-M)}),\ldots,\mathrm{sg}(\bar V^{(t-1)}),\bar V^{(t)}\,\big]\in\mathbb{R}^{(\bar N_s{+}N_m)\times d_s}, \label{eq:memvit_basic_V}
        \end{align}
        where $N_m$ is the number of \emph{cached} (flattened) tokens. Concatenation is along the token axis; the last dimension remains $d_s$, so the \emph{same} $W_K^{(r)},W_V^{(r)}$ apply to both current and cached rows:
        \[
        K_r^{(t)}=\bar K^{(t)}W_K^{(r)}\in\mathbb{R}^{(\bar N_s{+}N_m)\times d_h},\qquad
        V_r^{(t)}=\bar V^{(t)}W_V^{(r)}\in\mathbb{R}^{(\bar N_s{+}N_m)\times d_h}.
        \]
        Per head, attention is
        \[
        Z_r^{(t)}=\mathrm{Softmax}\!\Big(\tfrac{Q_r^{(t)}(K_r^{(t)})^\top}{\sqrt{d_h}}\Big)\,V_r^{(t)}\in\mathbb{R}^{\bar N_s\times d_h},
        \]
        then $Z^{(t)}=[Z_1^{(t)}\|\cdots\|Z_h^{(t)}]\in\mathbb{R}^{\bar N_s\times d_s}$ and $Z^{(t)}W_O$ restores stage width.
        
        \emph{Why the complexity is linear in cache size.} Only the \emph{current} $\bar N_s$ tokens emit queries; cached tokens supply keys/values but \emph{no queries}. Per head, the cost of forming the attention logits is $\mathcal{O}(\bar N_s(\bar N_s{+}N_m)d_h)$ (matrix multiply $Q_r^{(t)}[\,\bar N_s\times d_h\,]$ by $(K_r^{(t)})^\top[\,d_h\times(\bar N_s{+}N_m)\,]$), which scales \emph{linearly} with $N_m$ and avoids the $\mathcal{O}((\bar N_s{+}N_m)^2)$ blow-up of treating all tokens (past+present) as queries.
        
        \emph{Keeping $N_m$ small via learnable compression.} Cached keys/values are downsampled by spatiotemporal pooling $f_K,f_V$ that \emph{preserve} channel width $d_s$ but reduce token count by a factor such as $4{\times}2{\times}2$ over $(T,H,W)$~\cite[Sec.~4.2]{wu2022_memvit}. If the current stage has $\bar N_s$ tokens, each past clip contributes $\hat N_{m,s}=\bar N_s/16$ compressed tokens. After concatenation, augmented tensors have shape $(\bar N_s{+}M\hat N_{m,s})\times d_s$, and per head
        \[
        (\bar N_s{+}M\hat N_{m,s})\times d_s \xrightarrow{\;W_{K/V}^{(r)}\in\mathbb{R}^{d_s\times d_h}\;} (\bar N_s{+}M\hat N_{m,s})\times d_h,
        \]
        which matches the current-step projections. Because cached tokens supply \emph{only} $K/V$ and are gradient-stopped, per-head complexity is $\mathcal{O}\!\big(\bar N_s(\bar N_s{+}M\hat N_{m,s})\big)$ (linear in cache size), not $\mathcal{O}\!\big((\bar N_s{+}M\hat N_{m,s})^2\big)$.
        
        \textbf{How the rolling cache is used and updated (numerical example).}
        For $T{=}128$ processed as $\tau{=}16$-frame clips with tubelets $(t_p{=}2,p{=}16)$ at $224{\times}224$, the first stage sees $\bar N_s=\tfrac{16}{2}\cdot\tfrac{224}{16}\cdot\tfrac{224}{16}=1568$ tokens for $x^{(1)}$ and attends within the clip (empty cache). After attention, that layer stores a \emph{compressed} summary $(\hat K,\hat V)$ for future steps. At $t{=}2$, queries from $x^{(2)}$ ($\bar N_s{=}1568$) attend to the concatenation of cached $(\hat K,\hat V)$ from $x^{(1)}$ and current $(\bar K,\bar V)$ from $x^{(2)}$. The cache acts as a read-only KV store (stop-gradient), so only the present $\bar N_s$ queries are formed; past tokens do not query each other. The same mechanism applies at deeper MViT stages where $\bar N_s$ is smaller, so the relative overhead further shrinks while temporal receptive field grows hierarchically~\cite[Sec.~4.1]{wu2022_memvit}.
        
        \textbf{Pipelined memory compression (constant-time update).}
        Instead of recompressing \emph{all} cached clips every iteration, MeMViT compresses only the freshest uncompressed step and reuses older compressed entries:
        \begin{align}
            \bar K^{(t)} \;:=\; \big[\, \hat K^{(t{-}M)},\,\ldots,\,\hat K^{(t{-}2)},\, f_K\!\big(\mathrm{sg}(\bar K^{(t{-}1)})\big),\, \bar K^{(t)} \,\big], 
            \qquad \hat K^{(t')} \;=\; \mathrm{sg}\!\big(f_K(\bar K^{(t')})\big), \label{eq:memvit_pipeline}
        \end{align}
        and analogously for values. With a $4{\times}2{\times}2$ pool, each past clip contributes $1568/16=98$ tokens; for $M{=}2$, the current attention sees $1568{+}2\!\cdot\!98=1764$ keys/values, i.e., $\sim\!12\%$ overhead for a longer per-layer temporal horizon~\cite[Fig.~4; Table~1(b)]{wu2022_memvit}. Intuitively, the pipeline is a conveyor belt: each step “seals’’ the previous clip into a compact, trainable summary and shifts older summaries forward without recompression, keeping compute and memory near-constant over time.
        
        \textbf{Where memory is attached (hierarchical receptive fields).}
        Memory augmentation may be applied to all layers or a subset. Empirically, alternating memory-augmented and standard attention layers often yields the best accuracy/efficiency trade-off~\cite[Table~1(c)]{wu2022_memvit}. Shallow layers (large $\bar N$) store fine motion/texture cues; deep layers (small $\bar N$) store semantic summaries. This layered placement grows the temporal receptive field with depth while the marginal memory overhead shrinks, enabling long-term modeling (tens of seconds) at modest extra FLOPs compared with the same MViT backbone without memory~\cite[Sec.~4.1--4.2]{wu2022_memvit}.
        
        \newpage
        
        \paragraph{Algorithmic sketch (from the paper)}
        \begin{mintedbox}{python}
            # Algorithm 1 (from Wu et al., 2022): MeMViT attention (PyTorch-like)
            class MeMViTAttention():
                # pool_q, pool_k, pool_v: pooling layers
                # lin_q, lin_k, lin_v: linear layers
                # f_k, f_v: compression modules
                def __init__(self, max_mem):
                    self.m_k = []  # cached memory keys
                    self.m_v = []  # cached memory values
                    self.max_mem = max_mem
            
                def forward(self, x):
                    # compute pooled Q, K, V
                    q, k, v = pool_q(x), pool_k(x), pool_v(x)
                    
                    # compress only the immediate previous memory (pipelined)
                    cm_k = f_k(self.m_k[-1]) if len(self.m_k) > 0 else None
                    cm_v = f_v(self.m_v[-1]) if len(self.m_v) > 0 else None
                    
                    # concatenate (older compressed ..., newly compressed prev, current)
                    aug_k = cat(self.m_k[:-1] + ([cm_k] if cm_k is not None else []) + [k])
                    aug_v = cat(self.m_v[:-1] + ([cm_v] if cm_v is not None else []) + [v])
                    
                    # attention
                    z = attn(lin_q(q), lin_k(aug_k), lin_v(aug_v))
                    
                    # update caches: replace prev with its compressed copy
                    if len(self.m_k) > 0:
                        self.m_k[-1] = cm_k.detach()
                        self.m_v[-1] = cm_v.detach()
                    
                    # append current uncompressed memory for next iteration
                    self.m_k.append(k.detach())
                    self.m_v.append(v.detach())
                    
                    # enforce memory length
                    if len(self.m_k) > self.max_mem:
                        self.m_k.pop(0); self.m_v.pop(0)
                    return z
        \end{mintedbox}
        
        \begin{figure}[H]
            \centering
            \includegraphics[width=0.85\linewidth]{Figures/Chapter_24/MeMViT_cache.jpg}
            \caption{MeMViT caching and attention. Left: An online, clip\mbox{-}wise pipeline with an uncompressed cache for the immediate past and compressed caches for earlier steps. Right: At a memory\mbox{-}augmented layer, current queries attend to current keys/values plus cached, compressed memory from the past. Adapted from \cite{wu2022_memvit}.}
            \label{fig:chapter24_memvit_cache}
        \end{figure}
        
        \subsubsection{Architecture \& Implementation Details}
        \label{subsubsec:chapter24_memvit_arch}
        
        \paragraph{Backbone and stages}
        MeMViT instantiates the memory augmentation on top of MViT~\cite{fan2021_mvit,li2021_improved_mvit}, typically with an MViT\mbox{-}B backbone (16 layers) and 16\mbox{-}frame input clips at stride 4 (``$16{\times}4$''). The model proceeds through multiple stages with token downsampling (spatiotemporal pooling) between stages; memory augmentation can be placed at all or a subset of attention layers.\footnote{``Uniform half''---augmenting roughly $50\%$ of layers by alternating standard and memory attention---yields the best trade\mbox{-}off in Table~1(c) of \cite{wu2022_memvit}.}
        
        \paragraph{Data loading and training}
        During both training and inference, videos are \emph{read sequentially as clips} to mimic streaming: the implementation concatenates all videos and iterates through them in order. At a \emph{video boundary}—when the next clip belongs to a new video—any memory carried over from the previous video is \emph{masked to zero} so that unrelated context does not leak across videos. Default training follows standard MViT settings: backbone MViT-B (16 layers) with $16{\times}4$ clips, Kinetics-400 pre-training (unless stated), AVA fine-tuning for 30 epochs with SGD (batch 128), random horizontal flips and $224^2$ crops; FLOPs are reported at $224^2$ input resolution~\cite[Sec.~5]{wu2022_memvit}. 
        
        \subsubsection{Experiments and Ablations}
        \label{subsubsec:chapter24_memvit_experiments}
        
        \paragraph{Scaling strategies}
        Relative to the common baseline that \emph{increases the number of input frames $T$}, MeMViT attains much longer temporal support with near–flat increases in training/inference GPU memory, runtime, and FLOPs, while achieving \emph{higher} mAP at comparable cost~\cite[Fig.~3]{wu2022_memvit}. The below figure visualizes this trade–off: simply scaling $T$ rapidly exhausts memory and compute, whereas MeMViT’s hierarchical \emph{rolling cache} sustains long–range context at modest cost.
        
        \newpage
        
        \begin{figure}[H]
            \centering
            \includegraphics[width=0.85\linewidth]{Figures/Chapter_24/MeMViT_comparison_of_scaling_strategies.jpg}
            \caption{Comparison of scaling strategies. Increasing frames $T$ quickly explodes compute and memory; MeMViT maintains near–flat costs versus temporal support and achieves higher mAP under the same FLOPs. Adapted from \cite{wu2022_memvit}.}
            \label{fig:chapter24_memvit_scaling}
        \end{figure}
        
        \paragraph{Ablations: how memory is used}
        On AVA~\cite{gu2018_ava} with an MViT-B ($16{\times}4$) backbone~\cite{li2021_improved_mvit} pretrained on Kinetics-400, MeMViT improves from $27.0$ to $29.3$ mAP at a small FLOPs increase ($57.4{\to}58.7$G)~\cite[Table~3]{wu2022_memvit}. Layer–wise \emph{memory length} $M$ shows the best trade–off at $M{=}2$: $M{=}1$ (effective $8{\times}$ receptive field) yields $28.7$ mAP, $M{=}2$ ($16{\times}$) reaches $29.3$, and overly long $M{=}4$ ($32{\times}$) saturates to $28.8$~\cite[Table~1(a)]{wu2022_memvit}. Placing memory in \emph{about half} of the transformer layers achieves the peak $29.3$ mAP while reducing compute versus augmenting all layers~\cite[Table~1(c)]{wu2022_memvit}. For \emph{compression}, aggressive temporal downsampling is more tolerable than spatial: e.g., a $4{\times}2{\times}2$ (time{\mbox{:}}height{\mbox{:}}width) factor reaches $29.3$ mAP at $\approx 58.7$G FLOPs, whereas equally aggressive spatial compression harms accuracy~\cite[Table~1(b)]{wu2022_memvit}.
        
        \paragraph{Pipeline vs.\ naive compression}
        The \emph{pipelined} strategy---compressing only the freshest cached step while reusing earlier compressed memory---reduces training GPU memory and iteration time compared with recompressing all cached steps each iteration, without sacrificing accuracy~\cite[Fig.~4]{wu2022_memvit}. \autoref{fig:chapter24_memvit_compression} highlights the improved scaling behavior.
        
        \begin{figure}[H]
            \centering
            \includegraphics[width=0.60\linewidth]{Figures/Chapter_24/MeMViT_compression.jpg}
            \caption{Compression strategy. Pipelined memory compression lowers GPU memory and runtime compared with naively recompressing all cached steps each iteration. Adapted from \cite{wu2022_memvit}.}
            \label{fig:chapter24_memvit_compression}
        \end{figure}
        
        \paragraph{Generalization across backbones and datasets}
        Improvements persist with larger backbones and stronger pretraining. With MViT-24 ($32{\times}3$), MeMViT improves AVA mAP from $32.5$ to $34.4$ under Kinetics-700 pretraining, at similar compute ($204.4{\to}211.7$ GFLOPs)~\cite[Table~3]{wu2022_memvit}. Beyond AVA, MeMViT also improves EPIC-Kitchens-100: \emph{classification} top-1 from $44.6\%$ to $46.2\%$ and \emph{anticipation} class-mean recall@5 from $29.3\%\to 32.8\%$ (verbs) and $31.8\%\to 33.2\%$ (nouns)~\cite[Table~2(b)]{wu2022_memvit}.
        
        \paragraph{Takeaways from the ablations}
        Short memories capture local motion efficiently; moderate depth ($M{=}2$) plus selective layer placement yields the best accuracy–efficiency balance. Temporal compression can be stronger than spatial without hurting recognition, and pipelining the compression step is key to practical long-context training~\cite[Tables~1--3; Figs.~3--4]{wu2022_memvit}.
        
        \subsubsection{Limitations and Future Work}
        \label{subsubsec:chapter24_memvit_limits}
        
        MeMViT demonstrates that a rolling cache over compressed keys/values can extend temporal support at low cost, but its design choices expose several trade-offs that motivate subsequent long-context models (e.g., the next summaries on LongVLM and LWM).
        
        \begin{itemize}
            \item \textbf{Fixed window vs.\ relevance.} The memory length $M$ deterministically expands the receptive field but cannot adapt to which past clips are semantically relevant; ablations show benefits saturate beyond moderate $M$~\cite[Sec.~4.1; Table~1(a)]{wu2022_memvit}. Future directions include learned retrieval or content-aware routing to fetch only useful history instead of a rigid FIFO window.
            \item \textbf{Compression fidelity.} Pooling-based $f_K,f_V$ (e.g., $4{\times}2{\times}2$ over $T{:}H{:}W$) is efficient but lossy, potentially discarding small or fast events; the paper notes temporal compression is more tolerable than spatial~\cite[Sec.~4.2; Table~1(b)]{wu2022_memvit}. More expressive, task-aware compression or multi-granularity summaries could retain fine cues while preserving the pipelined efficiency.
            \item \textbf{Credit assignment across clips.} Cached entries are stop-gradient (\texttt{sg}), which stabilizes training without backpropagation through time~\cite[Sec.~4.1]{wu2022_memvit}. This read-only memory eases optimization but prevents learning signals from updating earlier clips; future work may explore limited or learned cross-clip credit assignment without incurring full BPTT.
            \item \textbf{Temporal grounding and position encoding.} Relative positional embeddings encode offsets $(\Delta t,\Delta h,\Delta w)$ and generalize across clip lengths~\cite[Sec.~4.3]{li2021_improved_mvit,wu2022_memvit}, yet they do not inject absolute timestamps or stream-level cues, which could aid localization in irregular or very long videos.
            \item \textbf{Backbone generality.} Although instantiated on MViT, the memory-as-augmented-(K/V) abstraction (with pipelined compression) applies to any attention layer~\cite[Sec.~4.1]{wu2022_memvit}. Subsequent models broaden this idea with dynamic retrieval, sparse/global–local attention, and hybrid positional schemes to scale context further—directions we will cover next in LongVLM and LWM at a high level.
        \end{itemize}
        
    \end{enrichment}
    
    \newpage
    
    \begin{enrichment}[LongVLM: Efficient Long-Video Reasoning][subsection]
        \label{enr:subsec_chapter24_longvlm}
        
        \paragraph{Motivation}
        Modern Video-LLMs often compress an entire video into a \emph{small, fixed} set of visual tokens via heavy pooling or a Q-Former, creating an information bottleneck that can erase fine details and blur temporal ordering across minutes of content. \emph{LongVLM}~\cite{weng2024_longvlm} addresses this by (i) constructing a \emph{long visual token sequence} that explicitly preserves short-term segment order, and (ii) fusing these \emph{local} tokens with a small set of \emph{global} semantics tokens. Only a lightweight projection is trained, keeping the visual encoder and LLM frozen, yet avoiding aggressive pre-LLM compression that harms fidelity and temporal grounding.
        
        \begin{figure}[H]
            \centering
            \includegraphics[width=0.85\linewidth]{Figures/Chapter_24/LongVLM_architecture_comparison.jpg}
            \caption{Architectural contrast and qualitative examples. Prior Video-LLMs (e.g., Video-ChatGPT, Video-LLaMA) aggressively compress to a few tokens (pooling/Q-Former), risking an information bottleneck; LongVLM preserves a longer sequence via token merging and attains more faithful, temporally grounded responses (green indicates correct text; red indicates errors). Adapted from \cite{weng2024_longvlm}.}
            \label{fig:chapter24_longvlm_compare}
        \end{figure}
        
        \paragraph{Method}
        \label{enr:subsubsec_chapter24_longvlm_method}
        
        \textbf{Setup and notation.}
        Uniformly sample $T$ frames, divide them into $S$ \emph{short-term} segments, each with $K$ frames ($T{=}S\!\cdot\!K$). A frozen visual encoder (CLIP ViT-L/14 in the paper) extracts patch tokens for each frame. Let a frame-$t$ token matrix be
        \[
        P^{t}\in\mathbb{R}^{u\times d}\!,
        \]
        with $u$ patch tokens and channel width $d$.\footnote{LongVLM follows the LLaVA family for vision$\to$language alignment, but \emph{keeps} the encoder and LLM frozen and trains only a projection, avoiding costly end-to-end tuning.} For a segment $s$, collect its $K$ frames’ tokens
        \begin{equation}
            \mathcal{V}^{s} \;=\; \big\{\,P^{t}\,\big\}_{t=1}^{K}\in\mathbb{R}^{K\times u\times d}.
            \label{eq:longvlm_vs}
        \end{equation}
        
        \medskip
        \textbf{Hierarchical token merging within each short segment.}
        To build a compact, \emph{local} representation while retaining details, LongVLM applies a hierarchical merging module $\mathcal{G}(\cdot)$ inside each segment:
        \begin{itemize}
            \item \emph{Per-step partition.} At merging step $i$ (on a current token set of size $R_i$), randomly partition tokens into two disjoint sets $\mathcal{P}_i$ and $\mathcal{Q}_i$ with $|\mathcal{P}_i|{=}r_i$, $|\mathcal{Q}_i|{=}R_i{-}r_i$.
            \item \emph{Similarity.} Split channels into $C$ heads of width $d_h$ ($d{=}C\,d_h$). For a token $p^{(p_u)}\!\in\!\mathcal{P}_i$ and $p^{(q_u)}\!\in\!\mathcal{Q}_i$, define the similarity by the \emph{head-averaged cosine}:
            \begin{equation}
                a^{\,p^{u}q^{u}}
                \;=\;
                \frac{1}{C}\sum_{c=1}^{C}
                \cos\!\Big(p^{(p_u)}_{c},\,p^{(q_u)}_{c}\Big),
                \label{eq:longvlm_cos}
            \end{equation}
            where $p_c$ denotes the $c$-th head slice.
            \item \emph{Greedy pairing and merge.} Choose the top-$r_i$ pairs with the largest $a^{\,p^{u}q^{u}}$ and \emph{average-pool} each pair to a single token:
            \begin{equation}
                \tilde{t}^{(u)}
                \;=\;
                \mathrm{AvgPool}\!\Big(\,p^{(p_u)},\,p^{(q_u)}\!\Big),
                \qquad u=1,\dots,r_i.
                \label{eq:longvlm_merge}
            \end{equation}
            Concatenate these $\{\tilde{t}^{(u)}\}_{u=1}^{r_i}$ with the unpaired tokens to obtain $R_{i+1}{=}\,R_i{-}r_i$ tokens.
            \item \emph{Iterate.} Repeat until a target budget $M$ (tokens per segment) is reached. For segment $s$ this yields a compact local feature
            \begin{equation}
                Z^{s}
                \;=\;
                \mathcal{G}(\mathcal{V}^{s})
                \in\mathbb{R}^{M\times d}.
                \label{eq:longvlm_seg_z}
            \end{equation}
        \end{itemize}
        Stack segment features in \emph{temporal order} to form the \underline{local} sequence
        \begin{equation}
            \mathcal{L}
            \;=\;
            \big\{\,Z^{s}\,\big\}_{s=1}^{S}
            \in\mathbb{R}^{(MS)\times d},
            \label{eq:longvlm_local_seq}
        \end{equation}
        which explicitly preserves the chronology of short-term segments across a long video.
        
        \medskip
        \textbf{Global semantic tokens from [CLS].}
        In parallel, LongVLM distills a \emph{global} summary by collecting the \texttt{[CLS]} tokens of each frame from $E$ (usually $E=5$) selected encoder layers $\{x^{t}_{e}\}_{t=1}^{T}$ and \emph{averaging them over time} per layer:
        \begin{equation}
            X_{e}
            \;=\;
            \mathrm{AvgPool}\!\Big(\{x^{t}_{e}\}_{t=1}^{T}\Big)\in\mathbb{R}^{d},
            \quad e=1,\dots,E,\qquad
            \mathcal{G}_{\mathrm{glob}}
            =
            \big\{X_{e}\big\}_{e=1}^{E}\in\mathbb{R}^{E\times d}.
            \label{eq:longvlm_global}
        \end{equation}
        These $E$ tokens provide high-level, video-wide context (\emph{the gist}) that complements the time-ordered, segment-level details in~\eqref{eq:longvlm_local_seq}.
        
        \medskip
        \textbf{Concatenation, projection, and prompting the LLM.}
        Concatenate \emph{global then local} tokens (empirically superior to the reverse order):
        \[
        [\;\mathcal{G}_{\mathrm{glob}} \,;\, \mathcal{L}\;]
        \in\mathbb{R}^{(E{+}MS)\times d},
        \]
        and project them into the LLM input space with a learned linear layer only (visual encoder and LLM are \emph{frozen}). The projected visual tokens are packed with the system prompt and user query and fed to the LLM to generate responses. This yields a long, information-rich visual stream for the LLM, avoiding the bottlenecks of heavy pre-LLM compression and preserving segment-level chronology.
        
        \begin{figure}[H]
            \centering
            \includegraphics[width=0.75\linewidth]{Figures/Chapter_24/LongVLM_architecture_proposal.jpg}
            \caption{LongVLM overview. Frames $\to$ visual encoder features $\to$ two streams: (i) short-term \emph{local} segment features via hierarchical token merging; (ii) \emph{global} semantics via temporally averaged \texttt{[CLS]} tokens from multiple encoder layers. Global tokens are prepended to the local, time-ordered tokens; a small projection aligns to the frozen LLM input space for instruction-following. Adapted from \cite{weng2024_longvlm}.}
            \label{fig:chapter24_longvlm_arch}
        \end{figure}
        
        \paragraph{Algorithmic sketch (token merging within a segment)}
        \begin{mintedbox}{python}
            # Pseudocode (faithful to the paper's Sec. 3.2 definitions; not source code).
            # Inputs: segment s with K frames, frame tokens {P^t in R^{u x d}}_{t=1..K}
            # Output: Z^s in R^{M x d} (M << K*u)
            
            def hierarchical_token_merging(P_list, M, C):
                # Flatten K x u tokens in the segment to a list T (length R0 = K*u)
                T = concat([P for P in P_list])          # T: [R0, d]
                R = len(T)
                while R > M:
                    # Random disjoint partition (|P_i| = r_i, |Q_i| = R - r_i)
                    P_i, Q_i = random_partition(T)
                    # Head-averaged cosine similarity between every p in P_i and q in Q_i
                    S = {}
                    for p in P_i:
                        for q in Q_i:
                            S[(p,q)] = (1/C) * sum(cos(p[c], q[c]) for c in range(C))
                    # Select top-|P_i| pairs by similarity (greedy, disjoint matching)
                    matches = top_pairs(S, k=len(P_i))
                    # Merge each matched pair by average pooling
                    merged = [avg(p, q) for (p, q) in matches]
                    # Unpaired tokens are carried over; update T and R
                    unpaired = list(set(P_i + Q_i) - set([x for pair in matches for x in pair]))
                    T = merged + unpaired
                    R = len(T)
                    # Return first M tokens in temporal order within the segment (implementation detail)
                return select_order_preserving(T, M)     # Z^s: [M, d]
        \end{mintedbox}
        
        \subsubsection{Architecture \& Implementation Details}
        \label{enr:subsubsec_chapter24_longvlm_impl}
        
        \begin{itemize}
            \item \textbf{Backbone and LLM.} LongVLM uses a frozen CLIP ViT-L/14 visual encoder and a frozen Vicuna-7B-v1.1 LLM, both initialized from LLaVA-7B-v1.1; only a single linear projection from vision features to the LLM input space is trained~\cite[Sec.~4.1]{weng2024_longvlm}. The CLIP encoder operates at $224{\times}224$ input resolution; with a $14{\times}14$ patch size this yields $u{=}256$ patch tokens per frame (plus \texttt{[CLS]}), and the similarity computation in the merging module uses $C{=}16$ heads as stated in the paper~\cite[Sec.~3.2]{weng2024_longvlm}. 
            \item \textbf{Training setup.} Finetuning is performed on the Video-ChatGPT-100K instruction dataset for $3$ epochs with learning rate $2{\times}10^{-5}$ and batch size $32$; both the visual encoder and the LLM remain frozen while the projection layer is updated. The reported wall-clock for the full $3$ epochs is approximately $3$ hours on $4\times$A100-80GB GPUs~\cite[Sec.~4.1]{weng2024_longvlm}. 
            \item \textbf{Frame sampling and segmentation.} During both training and inference, $T{=}100$ frames are uniformly sampled per video at $224^2$ resolution and divided into $S{=}10$ short-term segments with $K{=}10$ frames each ($T{=}S\!\cdot\!K$)~\cite[Sec.~4.1]{weng2024_longvlm}. Within each segment, hierarchical token merging produces $M$ compact \emph{local} tokens (paper default $M{=}30$), while \texttt{[CLS]} tokens averaged over time from $E$ selected CLIP layers provide \emph{global} semantics (paper default $E{=}5$ from the last five layers)~\cite[Sec.~3; Sec.~4.1]{weng2024_longvlm}. 
            \item \textbf{Token budget and ordering.} The total number of visual tokens fed to the LLM per video is $(M\!\times\!S){+}E \,=\, 30\times 10 + 5 \,=\, 305$. Following the ablation, global tokens are concatenated \emph{before} local tokens, i.e., [G,L], and the local sequence preserves the chronological order of segments (\(s{=}1\!\to\!S\)) when packed for the LLM~\cite[Fig.~2; Tab.~3]{weng2024_longvlm}. The paper reports that [G,L] outperforms [L,G] on the Video-ChatGPT benchmark (Mean $2.89$ vs.\ $2.82$). 
            \item \textbf{Projection and prompting.} Visual tokens are linearly projected (projection is the only trainable module) and concatenated with system instructions and user queries to form the LLM input. This preserves a long, information-rich visual stream into the LLM without aggressive pre-LLM compression, aligning with the architectural rationale illustrated in Fig.~\ref{fig:chapter24_longvlm_arch}. 
        \end{itemize}
        
        \subsubsection{Experiments and Ablations}
        \label{enr:subsubsec_chapter24_longvlm_exps}
        
        \paragraph{Benchmarks and metrics}
        LongVLM is evaluated on the Video-ChatGPT benchmark (500 ActivityNet-v1.3 videos, with 2{,}000 questions for each of five aspects: Correctness Information (CI), Detail Orientation (DO), Contextual Understanding (CU), Temporal Understanding (TU), Consistency (C)) and on zero-shot QA for ANET-QA, MSRVTT-QA, and MSVD-QA, reporting accuracy and generation quality scores.
        
        \begin{table}[H]
            \centering
            \caption{Comparison on the Video-ChatGPT benchmark (higher is better). Mean is the average over CI/DO/CU/TU/C. Data sizes follow the original papers. Numbers are from \cite[Tab.~1]{weng2024_longvlm}.}
            \label{tab:chapter24_longvlm_main_vcgpt}
            \footnotesize
            \setlength{\tabcolsep}{4pt}
            \resizebox{0.8\linewidth}{!}{%
                \begin{tabular}{l l c c c c c c}
                    \toprule
                    \textbf{Method} & \textbf{Data} & \textbf{CI} & \textbf{DO} & \textbf{CU} & \textbf{TU} & \textbf{C} & \textbf{Mean} \\
                    \midrule
                    VideoChat~\cite{li2024_videochat} & 10M & 2.25 & 2.50 & 2.54 & 1.98 & 1.84 & 2.22 \\
                    LLaMA Adapter v2~\cite{gao2023_llama_adapter_v2} & 700K & 2.03 & 2.32 & 2.30 & 1.98 & 2.15 & 2.16 \\
                    Video LLaMA~\cite{zhang2023_videollama} & 10M & 1.96 & 2.18 & 2.16 & 1.82 & 1.79 & 1.98 \\
                    Video-ChatGPT~\cite{maaz2024_video_chatgpt} & 100K & 2.50 & 2.57 & 2.69 & 2.16 & 2.20 & 2.42 \\
                    Valley~\cite{luo2025_valley} & 234K & 2.43 & 2.13 & 2.86 & 2.04 & 2.45 & 2.38 \\
                    BT-Adapter~\cite{liu2023_bt_adapter} & 10M & 2.16 & 2.46 & 2.89 & 2.13 & 2.20 & 2.37 \\
                    BT-Adapter~\cite{liu2023_bt_adapter} & 10M+100K & 2.68 & 2.69 & 3.27 & 2.34 & 2.46 & 2.69 \\
                    \textbf{LongVLM}~\cite{weng2024_longvlm} & \textbf{100K} & \textbf{2.76} & \textbf{2.86} & \textbf{3.34} & \textbf{2.39} & \textbf{3.11} & \textbf{2.89} \\
                    \bottomrule
                \end{tabular}%
            }
        \end{table}
        
        \begin{table}[H]
            \centering
            \caption{Zero-shot QA results (higher is better). Accuracy (\%) and quality \emph{Score} with data sources, reproduced from \cite[Tab.~2]{weng2024_longvlm}.}
            \label{tab:chapter24_longvlm_main_qa}
            \footnotesize
            \setlength{\tabcolsep}{3.5pt}
            \resizebox{0.85\linewidth}{!}{%
                \begin{tabular}{l l c c c c c c}
                    \toprule
                    \textbf{Method} & \textbf{Data} & \textbf{ANET-QA Acc.} & \textbf{ANET-QA Score} & \textbf{MSRVTT-QA Acc.} & \textbf{MSRVTT-QA Score} & \textbf{MSVD-QA Acc.} & \textbf{MSVD-QA Score} \\
                    \midrule
                    FrozenBiLM~\cite{yang2022_frozenbilm} & 10M & 24.7 & \textemdash{} & 16.8 & \textemdash{} & 32.2 & \textemdash{} \\
                    VideoChat~\cite{li2024_videochat} & 10M & 26.5 & 2.2 & 45.0 & 2.5 & 56.3 & 2.8 \\
                    LLaMA Adapter v2~\cite{gao2023_llama_adapter_v2} & 700K & 34.2 & 2.7 & 43.8 & 2.7 & 54.9 & 3.1 \\
                    Video LLaMA~\cite{zhang2023_videollama} & 10M & 12.4 & 1.1 & 29.6 & 1.8 & 51.6 & 2.5 \\
                    Video-ChatGPT~\cite{maaz2024_video_chatgpt} & 100K & 35.2 & 2.7 & 49.3 & 2.8 & 64.9 & 3.3 \\
                    Valley~\cite{luo2025_valley} & 234K & 45.1 & 3.2 & 51.1 & 2.9 & 60.5 & 3.3 \\
                    BT-Adapter~\cite{liu2023_bt_adapter} & 10M{+}100K & 45.7 & 3.2 & 57.0 & 3.2 & 67.5 & 3.7 \\
                    \textbf{LongVLM}~\cite{weng2024_longvlm} & \textbf{100K} & \textbf{47.6} & \textbf{3.3} & \textbf{59.8} & \textbf{3.3} & \textbf{70.0} & \textbf{3.8} \\
                    \bottomrule
                \end{tabular}%
            }
        \end{table}
        
        \begin{figure}[H]
            \centering
            \includegraphics[width=0.85\linewidth]{Figures/Chapter_24/LongVLM_quntative_results.jpg}
            \caption{Quantitative and qualitative results. Left: LongVLM is consistently on the outer envelope across aspects (CI/DO/CU/TU/C) and QA tasks. Right: A multi-turn conversation over a 3m46s video shows temporal awareness, fine detail tracking (e.g., apparel color), and plausible reasoning grounded in content. Adapted from \cite{weng2024_longvlm}.}
            \label{fig:chapter24_longvlm_results}
        \end{figure}
        
        \newpage
        
        \paragraph{Ablations}
        LongVLM conducts controlled ablations on: (i) local feature construction and global fusion, (ii) the per-segment token budget $M$, and (iii) the number of selected encoder layers $E$ used to form global \texttt{[CLS]} tokens~\cite[Sec.~4.3]{weng2024_longvlm}. The key findings are summarized below.
        
        \begin{table}[H]
            \centering
            \caption{Local vs.\ global aggregation on Video-ChatGPT (higher is better). Pooling uses 3D average pooling within each short segment; Merging uses the proposed hierarchical token merging; {[L,\,G]} concatenates Local then Global tokens, while {[G,\,L]} prepends Global before Local. Numbers from \cite[Tab.~3]{weng2024_longvlm}.}
            \label{tab:chapter24_longvlm_ablate_lg}
            \small
            \begin{tabular}{lcccccccc}
                \toprule
                \textbf{Variants} & \textbf{Local} & \textbf{Global} & \textbf{CI} & \textbf{DO} & \textbf{CU} & \textbf{TU} & \textbf{C} & \textbf{Mean} \\
                \midrule
                Pooling   & Yes & No  & 2.53 & 2.64 & 3.13 & 2.29 & 2.61 & 2.64 \\
                Merging   & Yes & No  & 2.62 & 2.74 & 3.15 & 2.23 & 2.86 & 2.72 \\
                {[L,\,G]} & Yes & Yes & 2.69 & 2.81 & 3.31 & 2.31 & 2.99 & 2.82 \\
                \textbf{[G,\,L]} & \textbf{Yes} & \textbf{Yes} & \textbf{2.76} & \textbf{2.86} & \textbf{3.34} & \textbf{2.39} & \textbf{3.11} & \textbf{2.89} \\
                \bottomrule
            \end{tabular}
        \end{table}
        
        \noindent\textit{Local \& global synergy.} Replacing naive 3D pooling with hierarchical token merging improves Mean from $2.64$ to $2.72$. Combining local and global tokens further boosts performance, and ordering matters: prepending global tokens {[G,\,L]} achieves the best Mean of $2.89$, matching the main result in Table~\ref{tab:chapter24_longvlm_main_vcgpt}.
        
        \noindent\textit{Per-segment token budget $M$.} Increasing $M$ improves the Video-ChatGPT Mean up to $\approx 30$ tokens/segment (Mean $2.89$ at $M{=}30$), while GPU memory remains nearly flat (e.g., $\approx 14.86$\,GB at $M{=}30$ on ANET-QA), with no further gains at $M{=}40$~\cite[Tab.~4]{weng2024_longvlm}.
        
        \noindent\textit{Global layers $E$.} Using global \texttt{[CLS]} tokens aggregated from the last $E{=}5$ visual encoder layers yields the strongest balance (Mean $2.89$); smaller ($E{=}1$) or larger ($E{\geq}10$) selections slightly underperform~\cite[Tab.~5]{weng2024_longvlm}.
        
        \paragraph{Qualitative analyses}
        \begin{figure}[H]
            \centering
            \includegraphics[width=0.85\linewidth]{Figures/Chapter_24/LongVLM_videogpt_examples.jpg}
            \caption{Ablation evidence: local-only vs.\ local+global. Left: Without global context, a local-only model mistakes a long jump for hurdles; adding global semantics recovers the correct event. Right: Local-only confuses an axe with a bat; global context plus local details yields the correct interpretation. Adapted from \cite{weng2024_longvlm}.}
            \label{fig:chapter24_longvlm_vcgpt_qual}
        \end{figure}
        
        \begin{figure}[H]
            \centering
            \includegraphics[width=0.85\linewidth]{Figures/Chapter_24/LongVLM_additional_examples.jpg}
            \caption{Additional generations from the Video-ChatGPT benchmark illustrating temporal grounding, fine-grained details (e.g., color, specific actions), and coherent scene understanding across diverse videos. Adapted from \cite{weng2024_longvlm}.}
            \label{fig:chapter24_longvlm_more}
        \end{figure}
        
        \paragraph{Limitations and Future Work}
        \label{enr:subsubsec_chapter24_longvlm_limits}
        \begin{itemize}
            \item \textbf{Fixed per-segment budget.} The token budget $M$ per segment is static; highly dynamic or sparse videos may benefit from \emph{adaptive} merging (retrieval- or saliency-guided) that varies the number of local tokens across segments.
            \item \textbf{Cosine-based merging.} Merging uses head-averaged cosine similarity and average pooling, which is simple and efficient but can still lose fine-grained rare cues. More expressive, learnable merging or content-aware reweighting could further preserve details.
            
            \newpage
            
            \item \textbf{Global token selection.} Global semantics rely on \texttt{[CLS]} tokens from fixed encoder layers; while effective, other summary signals (e.g., learned cross-frame prototypes or absolute timestamps) may improve localization in long, irregular streams.
            \item \textbf{Frozen backbone and LLM.} The frozen CLIP and LLM promote stability and training efficiency, but may limit domain adaptation. Lightweight adapters or partial tuning could help in specialized domains without sacrificing efficiency.
            \item \textbf{Scaling to extreme durations.} Although LongVLM already feeds a longer visual sequence than prior Video-LLMs, very long videos still stress the LLM context window. Subsequent methods (e.g., the next subsection on LWM) explore sparse/blockwise attention and retrieval to scale beyond hundreds of tokens.
        \end{itemize}
        
        \medskip
        \noindent\textbf{Bridge to LWM.} LongVLM demonstrates that preserving a longer, ordered stream of local tokens plus a few global tokens substantially reduces hallucinations and improves temporal grounding without retraining the LLM or the visual encoder. The next method, \emph{LWM}, pushes sequence length even further via scalable attention patterns and memory mechanisms designed for million-token contexts.
        
    \end{enrichment}
    
    \newpage
    
    \begin{enrichment}[LWM: Blockwise RingAttention for Million-Token Contexts][subsection]
        \label{enr:subsec_chapter24_lwm}
        
        \paragraph{Motivation}
        Long-context Video-LLMs and MLLMs have historically been constrained by quadratic-cost attention and modality-specific projections, which force aggressive pre-LLM compression and limit temporal grounding over hours of content. \emph{LWM}~\cite{liu2025_lwm} is proposed as a unified, autoregressive world model that scales the context window to $1$M tokens while operating directly on discrete vision tokens and text within a single Transformer, enabling long-video understanding and retrieval at million-length scale. 
        
        \begin{figure}[H]
            \centering
            \includegraphics[width=0.85\linewidth]{Figures/Chapter_24/LWM_context_comparisons.jpg}
            \caption{Context-size comparison across LLMs. LWM attains a one-million-token context window and is positioned at the frontier alongside large-context systems such as Gemini 1.5, substantially exceeding earlier 128K/100K or smaller context models. The context window is the effective short-term memory: larger windows allow whole books, long codebases, or hour-long videos to be processed in a single pass. Adapted from \cite{liu2025_lwm}.}
            \label{fig:chapter24_lwm_context}
        \end{figure}
        
        \paragraph{Method}
        \label{enr:subsubsec_chapter24_lwm_method}
        
        \textbf{Unified token space with discrete vision tokens.}
        \emph{LWM} maps language and vision into a \emph{single}, discrete token space processed by one causal Transformer~\cite{liu2025_lwm}. Text is tokenized with a standard BPE tokenizer; each image frame is tokenized by a pretrained VQGAN into a \(16{\times}16\) grid of codebook indices (i.e., \(256\) tokens for a \(256{\times}256\) frame). Vision spans are bracketed by special tokens \texttt{<vision>} and \texttt{</vision>}, with per-frame and end-of-vision delimiters \texttt{<eof>} and \texttt{<eov>} to mark boundaries. After concatenation, the model autoregressively predicts the next token over the \emph{joint} vocabulary (text subwords \(+\) vision codes), enabling any-to-any understanding and generation across text, image, and video \emph{without} a separate vision\(\to\)LLM projection layer.\footnote{Using discrete VQGAN indices as tokens removes the need for a continuous vision\(\to\)language projector, but requires extending the embedding and output (softmax) layers to include the vision codebook and optimizing them jointly so the decoder learns the distribution over visual codes; see \cite[Fig.~3]{liu2025_lwm}.}
        \emph{Intuition:} VQGAN turns pixels into a compact \emph{visual alphabet}. Once both modalities are “just tokens”, a single decoder can \emph{read and write} text, images, and videos in one sequence, with modality switches indicated by delimiters~\cite[Sec.~2; Fig.~3]{liu2025_lwm}.
        
        \newpage
        
        \paragraph{Training Curriculum}
        \label{enr:subsubsec_chapter24_lwm_training}
        
        The context window is not expanded to one million tokens in a single step. Training advances through a \emph{small ladder} of maximum sequence caps (for example, \(32\mathrm{K}\rightarrow 64\mathrm{K}\rightarrow 128\mathrm{K}\rightarrow 256\mathrm{K}\rightarrow 512\mathrm{K}\rightarrow 1\mathrm{M}\)). At each rung, the very same Transformer is optimized as usual; what changes is (i) how inputs are converted to tokens and then \emph{packed} up to the current cap, and (ii) how positional encodings are \emph{scaled} so they remain well behaved at the longer horizon~\cite[Sec.~3.1]{liu2025_lwm}. The model is fine-tuned at one cap until stable, then training \emph{resumes from that checkpoint} at the next cap, rather than starting from scratch.
        
        \medskip
        \noindent\textbf{From images/videos to tokens (step by step).}
        \begin{enumerate}
            \item \textit{Start with raw data (and how to handle long videos).} 
            An image arrives as \(I \in \mathbb{R}^{C\times H\times W}\) (e.g., \(3\times256\times256\)); a video as \(V \in \mathbb{R}^{T\times C\times H\times W}\) (e.g., \(T{=}120\) frames at \(4\) FPS for a \(30\)s clip). Before tokenization, apply the minimal preprocessing required by the frozen VQGAN tokenizer: resize/center-crop frames to \(256{\times}256\) and normalize pixel values as expected by the tokenizer (e.g., to \([0,1]\) or \([-1,1]\), per its training). The goal in this step is \emph{not} feature engineering but simply to ensure frames are in the canonical format the tokenizer expects so that code indices are meaningful.
            
            \medskip
            \noindent\textbf{How to fit long videos under the current context cap \(N_{\max}\).} In sub-stage training with cap \(N_{\max}\) (e.g., \(32\mathrm{K}\), \(64\mathrm{K}\), \(\ldots\), \(1\mathrm{M}\) tokens), each packed training sequence must satisfy a length budget. Let \(L_{\text{text}}\) be the text tokens in the packed sequence (prompt, question, target, etc.) and \(L_{\text{misc}}\) be delimiters and any extra small fields. The remaining \emph{vision budget} is
            \[
            B_{\text{vis}} \;=\; N_{\max} - L_{\text{text}} - L_{\text{misc}}.
            \]
            Each frame contributes approximately \(c_{\text{frame}}\approx 256 + c_{\text{delim}}\) tokens, where \(256\) comes from the \(16{\times}16\) VQGAN codes and \(c_{\text{delim}}\) accounts for \texttt{<eof>} and occasional boundary tokens (typically a small constant). This gives a maximum number of frames that can fit under the current cap:
            \[
            T_{\max} \;=\; \big\lfloor B_{\text{vis}} / c_{\text{frame}} \big\rfloor.
            \]
            If the raw video has \(S\) seconds and original FPS \(f_{\text{raw}}\) (so \(T_{\text{raw}} \!=\! S f_{\text{raw}}\) frames), reduce temporal density as follows:
            \begin{enumerate}
                \item \textbf{Temporal subsampling (preferred first).} Choose a target FPS
                \[
                f_{\text{target}} \;=\; \min\!\Big(f_{\text{raw}},\, \big\lfloor T_{\max}/S \big\rfloor\Big),
                \]
                and uniformly sample frames at stride \(\lfloor f_{\text{raw}}/f_{\text{target}}\rfloor\). This preserves chronological order while shrinking the token count linearly with FPS.
                \item \textbf{Contiguous windowing (if still too long).} If even \(f_{\text{target}}{=}1\) FPS would exceed the budget, extract a contiguous window of \(T_{\max}\) frames (e.g., pick a random start time each epoch) and \emph{discard the rest for this batch}. On subsequent steps, sample a different window so that, across training, the model still sees the entire clip.
                \item \textbf{Sliding-window splitting (optional).} Alternatively, split the video into overlapping windows of length \(\le T_{\max}\) (e.g., \(50\%\) overlap) and treat each window as a separate training example across iterations. This increases coverage without violating the cap.
            \end{enumerate}
            \noindent\textbf{Why reduce FPS or window?} Each additional frame adds \(\approx 256\) tokens. Without subsampling/windowing, long clips would blow past \(N_{\max}\) early in the curriculum (e.g., \(32\mathrm{K}\)), making batches impossible to pack and destabilizing optimization. Reducing FPS trades \emph{temporal density} for \emph{sequence feasibility} while preserving order; windowing then ensures that, over epochs, the model eventually observes all parts of the video.
            
            \noindent\textbf{Concrete example.} Suppose \(N_{\max}{=}64{,}000\), \(L_{\text{text}}{=}1{,}500\), \(L_{\text{misc}}{=}500\), so \(B_{\text{vis}}{=}62{,}000\). With \(c_{\text{frame}}\!\approx\!257\), we get \(T_{\max}{=}\lfloor 62{,}000/257 \rfloor{=}241\) frames. For a \(120\)s clip at \(f_{\text{raw}}{=}4\) FPS (\(T_{\text{raw}}{=}480\)), set \(f_{\text{target}}{=}\lfloor 241/120 \rfloor{=}2\) FPS and sample \(\approx 240\) frames uniformly. If the clip were \(1{,}200\)s long, even \(f_{\text{target}}{=}1\) FPS would exceed the budget; in that case, take a contiguous window of \(T_{\max}{=}241\) frames (about four minutes) this step, and a different window next time.
            
            \item \textit{Tokenize vision.} A \emph{frozen} VQGAN encodes each \(256{\times}256\) frame into a \(16{\times}16\) grid of codebook indices (i.e., \(256\) discrete tokens per frame). For videos, concatenate frames in order and insert \texttt{<eof>} between successive frames; wrap the whole span with \texttt{<vision>} and \texttt{</vision>}, and close with \texttt{<eov>}. Example (image): \texttt{<vision> [256 codes] </vision><eov>}. Example (video with \(T\) frames): \texttt{<vision>} \([256]\,\texttt{<eof>}\,\cdots\,\texttt{<eof>}\,[256]\) \texttt{</vision><eov>}. The resulting visual length is roughly \(256T + O(T)\) tokens (the \(O(T)\) comes from delimiters).
            
            \item \textit{Tokenize text.} Apply BPE to captions, instructions, transcripts, or questions to obtain standard text tokens. These share the same embedding/output layers as the vision codes once the vocabulary is extended.
            
            \item \textit{Interleave modalities.} Build a single sequence that mixes text and vision in the causal order required by the task, using delimiters as punctuation so the decoder can switch modalities.
            \begin{itemize}
                \item \textbf{Captioning.} \texttt{[Prompt tokens] <vision> [image codes] </vision><eov>} \\ \texttt{[Target caption tokens]}. 
                \item \textbf{Video QA.} \texttt{[Question tokens] <vision> [frame\(_1\) codes] <eof> \(\cdots\) [frame\(_T\) codes] </vision><eov> [Answer tokens]}. 
                \item \textbf{Conditional generation.} \texttt{[Instruction tokens]} \(\rightarrow\) the model emits image/video codes that a VQGAN decoder later turns into pixels.
            \end{itemize}
            
            \item \textit{Pack to the current cap.} Let \(N_{\max}\) be the current sub-stage cap (e.g., \(32\mathrm{K}\), \(64\mathrm{K}\), \(\ldots\), \(1\mathrm{M}\)). Construct training sequences by concatenating one or more interleaved examples until the total length reaches (or slightly under-fills) \(N_{\max}\), and apply a causal mask.
            \begin{itemize}
                \item \textbf{If a single example fits} (\(\leq N_{\max}\)). Pack it as is; if there is remaining space, append another short example or leave the remainder masked to the end of the packed sequence.
                \item \textbf{If a single example exceeds \(\boldsymbol{N_{\max}}\).} Use one of the above \emph{windowing} strategies so long examples still contribute signal at the current cap.
                \item \textbf{Mixture of lengths.} Because real examples vary, each batch naturally contains short and near-cap sequences. This \emph{length mixture} helps the model retain short-context competence while learning to exploit very long histories.
                \item \textbf{Concrete packing example.} Suppose \(N_{\max}=64{,}000\). A video-QA example tokenizes to \(40{,}000\) tokens; an image-captioning example tokenizes to \(8{,}000\); a text-only excerpt tokenizes to \(14{,}000\). Concatenate in order: \(40{,}000 + 8{,}000 + 14{,}000 = 62{,}000 \leq 64{,}000\). The remaining \(2{,}000\) tokens are left masked or filled with a very short snippet if available.
            \end{itemize}
            \emph{Intuition.} Think of each packed training sequence as a fixed-size ``page.'' Long stories are read in excerpts (windows), short notes are combined on the same page, and across many pages the model still sees the whole book.
            
            \newpage
            
            \item \textit{Optimize and repeat.} Feed the packed sequence \(X\in\mathbb{R}^{\leq N_{\max}\times d}\) into the causal Transformer and train with next-token cross-entropy over the \emph{joint} vocabulary (text BPE IDs + vision code IDs). The VQGAN is always frozen; the Transformer and the shared embedding/output layers are updated so the model learns to both \emph{consume} and \emph{emit} vision codes alongside text. When validation stabilizes at the current cap, increase the cap to the next rung, \emph{rescale RoPE} so positional geometry remains smooth at the longer horizon, resume from the latest checkpoint, and continue. Over epochs, windowed sampling ensures that even examples longer than \(N_{\max}\) are eventually seen in full, just not all at once.
        \end{enumerate}
        
        \noindent\textbf{Why the length ladder helps.} Jumping straight to \(1\)M tokens forces the network to master long-range structure it has never seen while also maintaining local competence; optimization often becomes unstable. By first training at \(32\mathrm{K}\), the model learns reliable local and mid-range patterns (sentences, short dialogues, tens of frames). Moving to \(64\mathrm{K}\) and \(128\mathrm{K}\) extends these habits to chapters and minutes of video. Each step “warms up” the next, so the final \(1\mathrm{M}\) stage mostly requires adapting to span-wide dependencies rather than discovering everything at once.
        
        \medskip
        \noindent\textbf{What it means to scale RoPE.}
        RoPE encodes position \(m\) by rotating each 2-D channel pair of a head vector with angle
        \[
        \phi_i(m)=m\,\omega_i,\qquad 
        \omega_i=\Theta_{\text{base}}^{-\,2i/d_h},\quad i=0,\ldots,\tfrac{d_h}{2}-1,
        \]
        using
        \[
        R(\phi)=
        \begin{bmatrix}
            \cos\phi & -\sin\phi\\
            \sin\phi & \cos\phi
        \end{bmatrix}.
        \]
        With \(q_m^{(i)},k_n^{(i)}\in\mathbb{R}^2\), RoPE has the \emph{relative} property
        \[
        \big\langle R(\phi_i(m))\,q_m^{(i)},\,R(\phi_i(n))\,k_n^{(i)}\big\rangle
        \;=\;
        \big\langle q_m^{(i)},\,R\big((n-m)\,\omega_i\big)\,k_n^{(i)}\big\rangle,
        \]
        so attention depends on the offset \(\Delta=n-m\) via rotations by \(\Delta\,\omega_i\).
        
        \medskip
        \noindent\textit{Why naïve extrapolation breaks.}
        If the context grows (e.g., $32\mathrm{K}\!\to\!1\mathrm{M}$) but $\Theta_{\text{base}}$ stays fixed, high–frequency channels wrap many times around the unit circle. Very distant tokens can become spuriously similar (phase aliasing), hurting long-range reasoning.
        
        \medskip
        \noindent\textit{LWM’s scaling rule (paper-faithful).}
        Let $s=\tfrac{N_{\text{new}}}{N_{\text{old}}}$. LWM rescales RoPE \emph{proportionally} to the new context by enlarging the base:
        \[
        \Theta_{\text{base}}^{\text{new}}=\Theta_{\text{base}}^{\text{old}}\cdot s
        \quad\Longleftrightarrow\quad
        \omega_i^{\text{new}}=\omega_i/s.
        \]
        Equivalently (index view), keep $\Theta_{\text{base}}$ and slow the index:
        \[
        \phi_i^{\text{new}}(m)=\tfrac{m}{s}\,\omega_i.
        \]
        In either view,
        \[
        \big\langle R(\phi_i^{\text{new}}(m))q^{(i)},\,R(\phi_i^{\text{new}}(n))k^{(i)}\big\rangle
        =
        \big\langle q^{(i)},\,R\!\big(\tfrac{(n{-}m)\,\omega_i}{s}\big)k^{(i)}\big\rangle,
        \]
        so the \emph{effective} distance becomes $\Delta_{\text{eff}}=\Delta/s$.
        Example: from $32\mathrm{K}$ to $1\mathrm{M}$, $s{\approx}32$; a gap of $\Delta{=}100\mathrm{K}$ “feels like” $\Delta_{\text{eff}}\!\approx\!3.1\mathrm{K}$ at the shorter cap—preventing wrap-around while preserving local geometry.
        
        \newpage
        \noindent\textit{Practical note (common variant).}
        Some implementations use a single slowdown exponent $\alpha\!\in\![0,1]$ (often $0.5$) so that
        $\Theta_{\text{base}}^{\text{new}}=\Theta_{\text{base}}^{\text{old}}\cdot s^{\alpha}$
        (equivalently $\Delta_{\text{eff}}=\Delta/s^{\alpha}$).
        Use $\alpha{=}1$ to match the paper’s “proportional to context” description; smaller $\alpha$ can be used as an engineering tweak without changing the derivation above.
        
        \medskip
        \textbf{Blockwise RingAttention for exact million-length attention.}
        Let \(X\!\in\!\mathbb{R}^{N\times d}\) be the input sequence (with \(N\) up to \(10^6\)) and \(h\) heads of size \(d_h{=}d/h\). Standard causal self-attention
        \[
        \mathrm{Attn}(Q,K,V)\;=\;\mathrm{Softmax}\!\Big(\tfrac{QK^\top}{\sqrt{d_h}}+\mathrm{mask}\Big)\,V
        \]
        is exact but materializing \(QK^\top\) and a full KV cache is prohibitive at \(N{=}10^6\).
        \emph{Blockwise RingAttention}~\cite[Sec.~3.1]{liu2025_lwm} partitions the sequence into \(G\) contiguous blocks of size \(B\) so \(N{=}GB\). Devices are arranged in a logical ring. Each device holds one \emph{query} block \(g\) and iteratively \emph{streams} key/value blocks \((K^{(g')},V^{(g')})\) from all other blocks around the ring:
        \begin{enumerate}
            \item Compute attention for local pairs \((Q^{(g)},K^{(g)},V^{(g)})\) with a fused kernel (e.g., FlashAttention), applying the causal mask to exclude future positions within the block.
            \item Receive the next \((K^{(g')},V^{(g')})\) from the neighbor, compute the masked cross-block contribution \(\mathrm{Softmax}\!\big(Q^{(g)}{K^{(g')}}^{\!\top}/\sqrt{d_h}+\mathrm{mask}\big)V^{(g')}\), and accumulate it into the output for block \(g\).
            \item Forward \((K^{(g')},V^{(g')})\) to the next device; repeat until all \(G\) key/value blocks have been visited exactly once.
        \end{enumerate}
        Because each \((\text{query block},\text{key block})\) pair is covered once under the causal mask, the result is \emph{mathematically identical} to dense attention. Only \(O(B)\) KV tokens live on a device at any moment; communication of streamed KV is overlapped with per-block compute, yielding high throughput on large device meshes~\cite[Sec.~3.1]{liu2025_lwm}.
        \emph{Intuition:} Think of a \emph{block relay}: each device keeps its queries and “meets” every other block’s keys/values as they circulate around the ring, accumulating the same full-context result without storing the entire sequence.
        
        \begin{figure}[H]
            \centering
            \includegraphics[width=0.85\linewidth]{Figures/Chapter_24/LWM_architecture.jpg}
            \caption{Architecture overview. LWM is a single autoregressive Transformer over a \emph{unified} token stream comprising BPE text and VQGAN vision codes (256 tokens per frame). Modality delimiters \texttt{<vision>}...\texttt{</vision>} and \texttt{<eof>}/\texttt{<eov>} mark boundaries; the model predicts the next token regardless of modality. Adapted from \cite{liu2025_lwm}.}
            \label{fig:chapter24_lwm_arch}
        \end{figure}
        
        \newpage
        
        \textbf{Discrete vision tokens, VQGAN, and projection-free learning.}
        LWM makes vision “native” to the decoder by adding visual \emph{tokens}—not projected features—to its vocabulary. A pretrained, \emph{frozen} VQGAN maps each $256{\times}256$ frame to a $16{\times}16$ grid of codebook indices (flattened to $256$ integers)~\cite[Fig.~3]{liu2025_lwm}; videos are tokenized frame-by-frame, concatenated with \texttt{<eof>} between frames, wrapped by \texttt{<vision>}...\texttt{</vision>}, and closed with \texttt{<eov>}. These indices are treated exactly like text BPE IDs: the Transformer’s shared embedding matrix (and tied LM head) is \emph{expanded} to include the vision codebook plus boundary tokens, and then trained end-to-end so the same decoder models
        \[
        p_\theta(x_{t+1}\mid x_{\le t}),\quad x_t \in \mathcal{V}_{\text{text}} \cup \mathcal{V}_{\text{vis}} \cup \{\texttt{<vision>},\texttt{</vision>},\texttt{<eof>},\texttt{<eov>}\}.
        \]
        No CLIP-style projector or adapter is required because VQGAN already produces \emph{discrete} IDs; all tokens live in one space, and modality switches are cued by simple delimiters rather than separate heads. During Stage~II, mixed text+vision sequences are fed under teacher forcing with a single cross-entropy over the \emph{joint} vocabulary. The decoder thereby learns to \emph{interpret} codes (e.g., answer questions conditioned on long spans of frames across \texttt{<eof>} boundaries) and to \emph{emit} coherent code sequences for conditional generation; predicted codes can be rendered back to pixels by the \emph{frozen} VQGAN decoder. Because the tokenizer never changes, the meaning of each code ID is stable throughout training, so learning concentrates where it matters—on the Transformer’s embeddings and attention—avoiding “interface drift” and collapse that can arise when a learned projector shifts. \emph{Intuition.} VQGAN supplies a fixed “visual alphabet.” Once images and videos are written as tokens, the LLM simply learns a larger language: just as it acquires word/subword syntax, it acquires visual “subword” syntax (spatial regularities within a frame; temporal patterns across \texttt{<eof>}) in the same autoregressive stream.
        
        \subsubsection{Architecture \& Implementation Details}
        \label{enr:subsubsec_chapter24_lwm_impl}
        
        \textbf{Implementation summary.}
        \begin{itemize}
            \item \textbf{Backbone.} A standard decoder-only Transformer (7B) serves as the core model for both text-only and multimodal training, optimized autoregressively over interleaved token streams~\cite{liu2025_lwm}.
            \item \textbf{What is trained vs.\ frozen.} The Transformer (initialized from a strong long-context text model) is \emph{trained} across both curriculum stages; the VQGAN vision tokenizer remains \emph{frozen}. The token embedding and output (softmax) layers are \emph{expanded} to include the vision codebook and \emph{trained} so the decoder can emit and consume visual tokens~\cite[Sec.~4]{liu2025_lwm}.
            \item \textbf{Vision tokenizer.} A pretrained VQGAN~\cite{esser2021_vqgan} (from aMUSEd) discretizes images of size \(256{\times}256\) into a \(16{\times}16\) grid of code indices (256 tokens per frame). Videos are tokenized frame-by-frame and concatenated in temporal order.
            \item \textbf{Special tokens.} Vision spans are wrapped with \texttt{<vision>}...\texttt{</vision>}; per-frame boundaries use \texttt{<eof>}; the end of an image or the last frame of a video uses \texttt{<eov>}. These delimiters teach the single decoder to switch modalities inside very long sequences~\cite[Sec.~4]{liu2025_lwm}.
            \item \textbf{Attention scaling.} Million-length context is enabled by Blockwise RingAttention (exact dense attention via blockwise ring scheduling) fused with FlashAttention-style kernels for high MFU, while rotary position embeddings (RoPE) use a scaled base parameter matched to the target context length for stability up to \(1\)M tokens~\cite[Sec.~3.1]{liu2025_lwm}.
            
            \newpage
            
            \item \textbf{Training curriculum.} Stage~I grows a text-only model from \(32\)K to \(1\)M context on long-form documents, progressively rescaling the RoPE base and initializing each length from the previous one. Stage~II introduces multimodality by interleaving text with image/video code sequences and continues progressive length increases (e.g., short sequences for stabilization, then to chat/long-form settings), preserving short-context accuracy while extending the window~\cite[Sec.~3.1; Sec.~4]{liu2025_lwm}.
            \item \textbf{Data construction.} Stage~I uses long-form book-style corpora to train long-range language modeling. Stage~II mixes large-scale image–text sources (e.g., LAION-2B-en, COYO-700M; images filtered to \(\geq 256\) px) with video–text sources (e.g., WebVid10M, InternVid10M); frames are discretized by VQGAN and packed with text using the modality delimiters, and pairs are packed to target lengths with randomized text–vision order to cover captioning and generation directions~\cite[Sec.~4]{liu2025_lwm}.
            \item \textbf{Why it works.} Discretizing vision removes modality-specific projectors and allows a single decoder to model text and vision uniformly in one token space, while RingAttention preserves \emph{exact} full-context interactions at the million-token scale so long-range dependencies in hour-long videos and long documents remain accessible during training and inference~\cite[Sec.~2; Sec.~3.1]{liu2025_lwm}.
        \end{itemize}
        
        \begin{figure}[H]
            \centering
            \includegraphics[width=0.85\linewidth]{Figures/Chapter_24/LWM_data_curation.jpg}
            \caption{Progressive data curation and training. Stage I extends language context using long books; Stage II integrates vision–language with a curriculum from images to short clips, Q\&A-style instruction data, and progressively longer videos. Pie charts show that images and short-frame videos dominate visual tokens, while mid-length documents dominate text tokens. Adapted from \cite{liu2025_lwm}.}
            \label{fig:chapter24_lwm_data}
        \end{figure}
        
        \subsubsection{Experiments and Ablations}
        \label{enr:subsubsec_chapter24_lwm_exps}
        
        \paragraph{Long-context retrieval (needle and multi-needle)}
        LWM maintains strong needle-in-a-haystack retrieval \emph{across the full 1M-token window}: accuracy is high and largely insensitive to where the needle is placed in the sequence. In multi-needle variants (several facts inserted, one question requiring synthesis), LWM remains competitive as the number of required facts grows, reflecting that exact full-context attention (via RingAttention) reliably surfaces distant evidence rather than relying on truncation or heuristics. The paper’s plot uses a mixed x-axis (0–128K log, 128K–1M linear) to show that performance does not collapse as position approaches the 1M boundary.
        
        \begin{figure}[H]
            \centering
            \includegraphics[width=0.85\linewidth]{Figures/Chapter_24/LWM_retrieval.jpg}
            \caption{Needle retrieval across context positions. LWM sustains high retrieval accuracy across positions and scales the context to $1$M tokens, while baselines are limited to shorter contexts. The x-axis is log (0–128K) then linear (128K–1M). Adapted from \cite{liu2025_lwm}.}
            \label{fig:chapter24_lwm_needle}
        \end{figure}
        
        \paragraph{Language tasks at short context}
        As the context is expanded from $32$K to $1$M, short-context language benchmarks (ARC, HellaSwag, MMLU, OpenBookQA) remain broadly stable (Table~1 in the paper), indicating that the length curriculum and mixed-length packing preserve near-field skills while extending the horizon.
        
        \paragraph{LOFT benchmarks (512K)}
        On long-document retrieval and RAG (LOFT), LWM at $512$K outperforms strong baselines on Quora and NQ and is substantially ahead on HotPotQA, highlighting the benefit of attending to the whole corpus chunk at once (no truncation artifacts). The reported scores are:
        
        \begin{table}[H]
            \centering
            \caption{LOFT at 512K context: LWM vs.\ strong baselines (selected). Higher is better.}
            \label{tab:chapter24_lwm_loft}
            \footnotesize
            \setlength{\tabcolsep}{6pt}
            \resizebox{0.7\linewidth}{!}{%
                \begin{tabular}{lccc}
                    \toprule
                    \textbf{Benchmark} & \textbf{LWM (512K)} & \textbf{GPT-4o (128K)} & \textbf{Claude 3 Opus (200K)} \\
                    \midrule
                    Quora       & \textbf{0.38} & 0.23 & 0.37 \\
                    NQ          & \textbf{0.37} & 0.22 & 0.37 \\
                    HotPotQA    & \textbf{0.72} & 0.21 & 0.32 \\
                    \bottomrule
            \end{tabular}}
        \end{table}
        
        \newpage
        
        \paragraph{Long-video understanding}
        On Long Video-MME, LWM-1M (7B) processes up to $\leq 1800$ frames and achieves strong results, including $60.8$ on the $30$–$60$\,min split (Table~4 in the paper). Intuitively, the model can downsample and still keep the entire narrative in-context, so answers can depend on events far apart in time without losing earlier evidence.
        
        \begin{figure}[H]
            \centering
            \includegraphics[width=0.85\linewidth]{Figures/Chapter_24/LWM_1hour_video_example.jpg}
            \caption{One-hour YouTube compilation QA. LWM-Chat-1M retrieves fine-grained details across hundreds of clips within one sequence, succeeding where several proprietary and open-source models either refuse, miss, or hallucinate. Adapted from \cite{liu2025_lwm}.}
            \label{fig:chapter24_lwm_1hour}
        \end{figure}
        
        \newpage
        
        \paragraph{Generation}
        Because vision is discretized, the same autoregressive decoder that models text can also \emph{emit} image and short video code sequences conditioned on text. Decoding those codes through the (frozen) VQGAN yields images and simple clips with coherent local dynamics (e.g., fireworks, waves). This is a direct consequence of training a single next-token model over a joint vocabulary of text and vision IDs.
        
        \begin{figure}[H]
            \centering
            \includegraphics[width=0.85\linewidth]{Figures/Chapter_24/LWM_generation_examples.jpg}
            \caption{Text-to-image and text-to-video generation. Top: image generation; bottom: short video sequences showing simple temporal dynamics captured by autoregressive decoding over visual codes. Adapted from \cite{liu2025_lwm}.}
            \label{fig:chapter24_lwm_gen}
        \end{figure}
        
        \subsubsection{Limitations and Future Work}
        \label{enr:subsubsec_chapter24_lwm_limits}
        
        \begin{itemize}
            \item \textbf{Compute and hardware demands.} Million-length training and inference rely on large device meshes and careful kernel fusion; although attention is exact, the system requirements are substantial and may limit accessibility.
            \item \textbf{Vocabulary expansion and modality balance.} Incorporating a vision codebook expands the vocabulary and requires curriculum tuning to preserve strong text performance while learning vision tokens at scale.
            \item \textbf{Token efficiency for very long videos.} Per-frame tokenization at fixed resolution (256 codes/frame) can become costly for multi-hour content; integrating adaptive frame rates, token pruning, or content-aware compression could further extend effective context.
            \item \textbf{Position encoding extrapolation.} Scaled RoPE is simple and empirically stable, but principled positional schemes tailored for interleaved multimodal streams may further improve generalization at extreme lengths. 
        \end{itemize}
        
    \end{enrichment}
\end{enrichment}

\newpage

\begin{enrichment}[Specialized Directions][section]
    \label{enr:sec_chapter24_specialized}
    
    Beyond short-clip classification, several specialized tasks push distinct modeling frontiers.
    
    \medskip
    \noindent\textbf{Temporal detection and localization.}
    Here the goal is not only to recognize which action occurs but also to determine \emph{when} it starts and ends in long, untrimmed videos. Methods include two--stage pipelines (proposal $\rightarrow$ classification) as well as end-to-end transformer models that directly predict temporal boundaries.
    
    \medskip
    \noindent\textbf{Video diffusion models.}
    Diffusion models extend from images to video by introducing temporal consistency modules that enforce smooth frame-to-frame evolution. A representative system, \emph{Video Diffusion Models (VDM)}, demonstrates high-fidelity synthesis and editing by scaling latent diffusion to temporal data \cite{blattmann2023_vdm}.
    
    \medskip
    \noindent\textbf{Multimodal alignment.}
    Video rarely comes alone; audio, depth, and infrared cues are often available. \emph{LanguageBind} learns a unified embedding space that aligns video with multiple sensing modalities to language, broadening supervision and enabling stronger transfer across tasks \cite{xu2023_languagebind}.
    
    \medskip
    \noindent\textbf{Layered, object-centric video effects (Omnimatte family).}
    \emph{Omnimatte} pioneered layered mattes that jointly capture objects and their visual effects (e.g., shadows, reflections) from monocular video, enabling editing and compositing \cite{lu2021_omnimatte}. The follow-up \emph{OmnimatteRF} extended this idea to neural radiance fields, allowing layered decomposition in a 3D-aware representation \cite{lu2023_omnimatterf}.
    
    \medskip
    \noindent\textbf{Related and emerging directions.}
    Efficiency-oriented backbones (e.g., UniFormerV2 \cite{li2022_uniformerv2}) and distillation-style masked pretraining (e.g., Masked Video Distillation \cite{wang2023_mvd}) are widely adopted in practice. Long-form video QA (e.g., EgoSchema) and holistic reasoning benchmarks (e.g., VideoMind) continue to push models toward higher-level cognition. Finally, recent open and commercial diffusion-based systems such as \href{https://videocrafter.github.io/}{VideoCrafter}, \href{https://runwayml.com/gen2}{Runway Gen-2}, and \href{https://pika.art}{Pika} increasingly inform pipelines for video synthesis and editing.
    
\end{enrichment}

\chapterimage{head2.png} % Chapter heading image

% Chapter-specific content starts here
\chapter{Model Compression: Quantization and Pruning}

%----------------------------------------------------------------------------------------
%	CHAPTER 26 - Model Compression: Quantization and Pruning
%----------------------------------------------------------------------------------------


\chapterimage{head2.png} % Chapter heading image

% Chapter-specific content starts here
\chapter{Foundation Models in Computer Vision}

%----------------------------------------------------------------------------------------
%	CHAPTER 26 - Foundation Models in Computer Vision
%----------------------------------------------------------------------------------------


\chapterimage{head2.png} % Chapter heading image

% Chapter-specific content starts here
\chapter{MAMBA: Multi-Agent Multi-Body Analysis}

%----------------------------------------------------------------------------------------
%	CHAPTER 27 - MAMBA: Multi-Agent Multi-Body Analysis
%----------------------------------------------------------------------------------------


\printbibliography
\end{document}